\documentclass{book}
\usepackage{amsthm,amsmath,amssymb,amsfonts,xcolor}

\usepackage{caption}
\usepackage{tikz}
%--
\begin{document}

\tikzset{ % Lista de opciones de nodo
% opci�n draw = color borde
% opacity = porcentaje transparencia
% opci�n fill = color relleno
opciones/.style={%
draw= red!50!black!50,
top color = white,
bottom color = red!50!black!20,
fill opacity = 0.6,
rectangle, font=\small\bf\sffamily}
}
\begin{center}
\begin{tikzpicture}
\draw[draw= yellow!30,fill=yellow!30] (0,0) rectangle(2,2);
\node[draw=black,rectangle, anchor=center] at (1,1) {$\bullet$};
\node[opciones, anchor=south east] at (1,1) {ancla en el sur-este};
\node[opciones, anchor=north west] at (1,1) {ancla en el norte-oeste};
\end{tikzpicture}
\captionof{figure}{Nodos rectangular con anclaje en el sureste y noroeste}
\end{center}

\end{document}