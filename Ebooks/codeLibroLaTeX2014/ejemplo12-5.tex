\documentclass{beamer}
\usepackage{amsmath,amssymb,amsfonts,latexsym,stmaryrd}
\usepackage[latin1]{inputenc}
\usepackage[T1]{fontenc}
% Conversión eps to pdf, requiere habilitar shell-escape.
% Texlive 2010 o superior No lo necesita
%\usepackage{epstopdf}
%\DeclareGraphicsExtensions{.pdf,.png,.jpg}
\usefonttheme{professionalfonts} % fuentes de LaTeX
\usetheme{Warsaw} % Tema escogido en este ejemplo
\setbeamercovered{transparent} % Velos
\newtheorem{teorema}{Teorema}
\newtheorem{ejemplo}{Ejemplo}
\newtheorem{defi}{Definici�n}
\newtheorem{coro}{Corolario}
\newtheorem{prueba}{Prueba}
\begin{document}
\begin{frame}{Campo Galois $GF(p^r)$}
	\begin{teorema} %definido en el pre�mbulo
		Sea $F$ un campo y $P(x)$ m�nico en $F[x],$ grado $P(x)\geq 1$.
	\end{teorema}
	\pause % <---- PAUSA
	\begin{ejemplo} % Entorno definido en el preámbulo
		Sea $P(x)=x^3-2 \in\,Q[x]$. $P(x)$ es irreducible. Aunque tiene una
		ra��z en $R$, a saber $2^{1/3}$, $R$ no es un campo de escisi�n para $P.$
	\end{ejemplo}
\end{frame}
\end{document}