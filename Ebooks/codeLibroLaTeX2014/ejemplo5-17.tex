\documentclass{article} %o report o book
\textheight=20cm
\textwidth=17cm
\topmargin=-2cm
\usepackage{amsmath,amssymb,amsfonts,latexsym,cancel}
\usepackage[spanish]{babel}
\usepackage[latin1]{inputenc} %Acentos desde el teclado
\usepackage[T1]{fontenc}
\usepackage{tabularx}
% Comandos especiales
\newcommand{\sen}{\mathop{\rm sen}\nolimits} %seno
\newcommand{\arcsen}{\mathop{\rm arcsen}\nolimits}
\newcommand{\arcsec}{\mathop{\rm arcsec}\nolimits}
\newcommand{\R}{\mathbb{R}}
\newcommand{\N}{\mathbb{N}}
\newcommand{\Z}{\mathbb{Z}}
\def\max{\mathop{\mbox{\rmax}}} 
\def\min{\mathop{\mbox{\rmax}}} 

\begin{document}	

\newcolumntype{D}{>{$\displaystyle}c<{$}}
%Las columnas M aceptan texto matemático ala izquierda: |l|
\newcolumntype{M}{>{$}l<{$}}
%Se usa ’tabular’ normal.
\begin{tabular}{|D|D|D|D|M|D|M|r|}\hline
	n & -1 & 0 & 1 & 2 & 3 & 5 & 5 \\ \hline
	a_n & 8 & 5 & 2 & 2 & 4 & n & n \\ \hline
	r_n & 0 & 1 & 2 & 5 & 22 & 444 & 444 \\ \hline
	s_n & 1 & 0 & 1 & 2 & 9 &7& 7\\ \hline
	F & 0 &-2 &\frac{n}{4}& \frac{n}{4} & n^2 & 2 & 2
	\rule[-0.3cm]{0cm}{1cm}\\ \hline
\end{tabular}

\end{document}