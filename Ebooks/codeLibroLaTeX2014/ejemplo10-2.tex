\documentclass{book}
\usepackage{amsthm,amsmath,amssymb,amsfonts,pstricks}
\usepackage[latin1]{inputenc}
\newcommand{\Z}{\mathbb{Z}}
\usepackage{xparse}
\newcounter{midefi}[chapter] % Contador 1.1, 1.2,... en rojo
\renewcommand{\themidefi}{{\red\thechapter.\arabic{midefi}}}
\NewDocumentEnvironment{midefinicion}{O{}O{}}{%2 argumentos opcionales
	%Inicio del entorno
	\bigskip
	\begin{minipage}{\textwidth}
		% Definición - número - descripción (opcional)
		\textbf{Definicion \;\themidefi \;{\small\sffamily #1}}
	}{% Parámetro #2: Contenido del entorno
	#2
\end{minipage}
\bigskip
}%
\begin{document}
	\begin{midefinicion}[(Principio de induccion).]
		Si $P$ es una formula en la artimetica de Peano, entonces
		$$(P(0) \land (\forall n)(P(n) \to P(n+1)) \to (\forall n)(P(n))$$
		En particular, $P$ puede ser la formula $(m < n)\to Q(m)$ y a este
		caso
		algunas personas le llaman "Induccion fuerte".
	\end{midefinicion}
\end{document}