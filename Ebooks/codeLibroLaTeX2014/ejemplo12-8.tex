\documentclass{beamer}
\usepackage{amsmath,amssymb,amsfonts,latexsym,stmaryrd}
\usepackage[latin1]{inputenc}
\usepackage[T1]{fontenc}
% Conversión eps to pdf, requiere habilitar shell-escape.
% Texlive 2010 o superior No lo necesita
%\usepackage{epstopdf}
%\DeclareGraphicsExtensions{.pdf,.png,.jpg}
\usefonttheme{professionalfonts} % fuentes de LaTeX
\usetheme{Warsaw} % Tema escogido en este ejemplo
\setbeamercovered{transparent} % Velos
\newtheorem{teorema}{Teorema}
\newtheorem{ejemplo}{Ejemplo}
\newtheorem{defi}{Definici�n}
\newtheorem{coro}{Corolario}
\newtheorem{prueba}{Prueba}
\begin{document}
\begin{frame}[fragile] % "fragile" es obligatorio
	\frametitle{Un algoritmo para buscar n�meros primos}
	\begin{semiverbatim}
		\uncover<1->{\alert<0>{int main (void)}}
		\uncover<1->{\alert<0>{\{}}
		\uncover<1->{\alert<1>{ \alert<4>{std::}vector is_prime(100,true)}}
		\uncover<1->{\alert<1>{ for (int i = 2; i < 100; i++)}}
		\uncover<2->{\alert<2>{ if (is_prime[i]))}}
		\uncover<2->{\alert<0>{ \{}}
		\uncover<3->{\alert<3>{ \alert<4>{std::}cout << i << " ";}}
		\uncover<3->{\alert<3>{ for (int j = 1; j < 100;}}
		\uncover<3->{\alert<3>{ is_prime [j] = false, j+=i);}}
		\uncover<2->{\alert<0>{ \}}}
		\uncover<1->{\alert<0>{ return 0;}}
		\uncover<1->{\alert<0>{\}}}
	\end{semiverbatim}
	\visible<4->{Notar el uso de \alert{\texttt{std::}}.}
\end{frame}
\end{document}