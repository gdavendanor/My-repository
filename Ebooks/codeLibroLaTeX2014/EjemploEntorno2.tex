\documentclass{book} 
 \usepackage{amsthm,amsmath,amssymb,amsfonts,pstricks}
 \usepackage[latin1]{inputenc}  
 \newcommand{\Z}{\mathbb{Z}}
 
 \usepackage{xparse}
 \newcounter{midefi}[chapter] % Contador 1.1, 1.2,... en rojo
 \renewcommand{\themidefi}{{\red\thechapter.\arabic{midefi}}}  
 \NewDocumentEnvironment{midefinicion}{O{}O{}}{%2 argumentos opcionales
    %Inicio del entorno
   \bigskip
   \begin{minipage}{\textwidth}
    % Definici�n - n�mero - descripci�n (opcional)
   \textbf{Definici�n \;\themidefi \;{\small\sffamily #1}} 
 
 }{% Par�metro #2: Contenido del entorno
     #2
    \end{minipage}
    \bigskip 
 }%
 
\begin{document}
  \begin{midefinicion}[(Principio de inducci�n).]
     Si $P$ es una f�rmula en la artim�tica de Peano, entonces
     $$(P(0) \land (\forall n)(P(n) \to P(n+1)) \to (\forall n)(P(n))$$
     En particular, $P$ puede ser la f�rmula $(m < n)\to Q(m)$ y a este caso
    algunas personas le llaman "Inducci�n fuerte".
 \end{midefinicion}

\end{document}
