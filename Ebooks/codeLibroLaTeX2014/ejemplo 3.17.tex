\documentclass[letterpaper, 12pt]{article} 
\usepackage[total={14cm,21cm},top=2cm, left=2cm]{geometry}
\parindent = 0mm % Sin sangr�a
\usepackage{latexsym,amsmath,amssymb,amsfonts}
\usepackage[latin1]{inputenc}
\usepackage[T1]{fontenc}
\usepackage{graphicx}
\usepackage[spanish]{babel} % Idioma espa�ol
\renewcommand{\baselinestretch}{1.1} % espaciado 1.1
\pagestyle{myheadings}
\markright{...... texto .......} % Encabezados simples
\usepackage{multicol}
\usepackage{xcolor}


\begin{document}

\begin{center}
{ \fboxsep 12pt
\fcolorbox {orange}{white}{
\begin{minipage}[t]{10cm}
$0^0$ es una expresi�n indefinida. Si $a>0$, $a^0=1$ pero $0^a=0.$
Sin embargo, convenir en que $0^0=1$ es adecuado para que
algunas f�rmulas se puedan expresar de manera sencilla,
sin recurrir a casos especiales, por ejemplo
$$e^x=\sum_{n=0}^{\infty}\frac{x^n}{n!}$$
$$(x+a)^n=\sum_{k=0}^n \binom{n}{k}x^k a^{n-k}$$
\end{minipage}
} }
\end{center}

\end{document}