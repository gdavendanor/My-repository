\documentclass{book}
\usepackage{amsthm,amsmath,amssymb,amsfonts,xcolor}
\usepackage[latin1]{inputenc}
\definecolor{azulF}{rgb}{.0,.0,.3} % Azul
\definecolor{rojoF}{RGB}{212,0,0} % Rojo
\newcommand{\Z}{\mathbb{Z}} % \Z
% Estilo B
\newtheoremstyle{estiloB}{}{}{}{}%
{\color{azulF}\bfseries} % fuente del encabezado
{.} % puntuación
{\newline} % espacio después del encabezado
{\thmname{#1}~\thmnumber{\color{rojoF} #2}\thmnote{~\color{azulF}(#3)}}
%%--
\theoremstyle{estiloB}
\newtheorem{unadefiB}{Definición}[chapter]
\newtheorem{unteoB}{Teorema}[section]
%%--
% Estilo C
\newtheoremstyle{estiloC}{}{}{}{}%
{\color{rojoF}\bfseries}%
{.}{\newline}{}%
%--
\swapnumbers % Intercambiar número-teorema
\theoremstyle{estiloC}
\newtheorem{unteoC}{Teorema}
\usepackage{tikz}
%--
\begin{document}
\begin{tikzpicture}[scale=0.8] % Escalamiento de la figura 80%
\draw (-1,0) -- (4,0) node[right] {$x$}; % Ejes
\draw (0,-1) -- (0, 2) node[left] {$y$};
% Dominio: domain = a:b
\draw[smooth, domain = 0:2, color=red] plot (\x,\x)node[right] {$y = x$
};
%\x r indica que x se mide en radianes
\draw[smooth, domain = -2:2, color=blue] plot (\x,{sin(2*\x r)+1})
node[right] {$y = \sin(2*x)+1$};
\draw[smooth, domain = -1:1, color=black] plot (\x,{exp(\x)}) node[right
] {$y = e^x$};
\end{tikzpicture}
\end{document}