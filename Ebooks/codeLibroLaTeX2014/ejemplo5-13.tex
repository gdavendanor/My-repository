\documentclass{article} %o report o book
\textheight=20cm
\textwidth=17cm
\topmargin=-2cm
\usepackage{amsmath,amssymb,amsfonts,latexsym,cancel}
\usepackage[spanish]{babel}
\usepackage[latin1]{inputenc} %Acentos desde el teclado
\usepackage[T1]{fontenc}
\usepackage{booktabs}
% Comandos especiales
\usepackage[x11names,table]{xcolor}
\newcommand{\sen}{\mathop{\rm sen}\nolimits} %seno
\newcommand{\arcsen}{\mathop{\rm arcsen}\nolimits}
\newcommand{\arcsec}{\mathop{\rm arcsec}\nolimits}
\newcommand{\R}{\mathbb{R}}
\newcommand{\N}{\mathbb{N}}
\newcommand{\Z}{\mathbb{Z}}
\def\max{\mathop{\mbox{\rmax}}}
\def\min{\mathop{\mbox{\rmax}}}


\begin{document}	
\begin{tabular}{l l l}\hline
	$\displaystyle\frac{x}{x+1}$\\
	& $\sqrt{x}$
	& $x^{2^n}$\\ \hline
\end{tabular}

\begin{tabular}{l l l}\toprule
	$\displaystyle \frac{x}{x+1}$ & $\sqrt{x}$ & $x^{2^n}$\\ \bottomrule
\end{tabular}


\begin{tabular}{l l l@{\vrule height 15pt depth 10pt width 0pt}}\hline
	$\displaystyle \frac{x}{x+1}$ \\
	& $\sqrt{x}$
	& $x^{2^n}$\\ \hline
\end{tabular}

\begin{tabular}{l l l} \hline
	Expresiones & & \\ \hline
	& & \\[-8pt]
	$\displaystyle \frac{x}{x+1}$ & $\sqrt{x}$ & $x^{2^n}$ \\[8pt]\hline
\end{tabular}
\end{document}