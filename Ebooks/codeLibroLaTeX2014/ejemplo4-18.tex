\documentclass{article} %o report o book
\textheight=20cm
\textwidth=18cm
\topmargin=-2cm
%Símbolos matemáticos de la AMS
\usepackage{amsmath,amssymb,amsfonts,latexsym,cancel}
\usepackage[spanish]{babel}
\usepackage[latin1]{inputenc} %Acentos desde el teclado
\usepackage[T1]{fontenc}
% Comandos especiales
\newcommand{\sen}{\mathop{\rm sen}\nolimits} %seno
\newcommand{\arcsen}{\mathop{\rm arcsen}\nolimits}
\newcommand{\arcsec}{\mathop{\rm arcsec}\nolimits}
\newcommand{\R}{\mathbb{R}}
\newcommand{\N}{\mathbb{N}}
\newcommand{\Z}{\mathbb{Z}}
\def\max{\mathop{\mbox{\rm máx}}} % máximo
\def\min{\mathop{\mbox{\rm mín}}} % mínimo


\begin{document}
\begin{pmatrix}
	1 & 0 & 0 & & \cdots 0 \\
	h_0 & 2(h_0+h_1) & h_1 & & \cdots 0 \\
	0 & h_1 & 2(h_1+h_2) & h_2 & \cdots 0 \\
	& \ddots & \ddots & \ddots & \\
	0 & 0 \cdots & h_{n-3} & 2(h_{n-3}+h_{n-2}) & h_{n-2} \\
	0 & 0 & & & \cdots 1 \\
\end{pmatrix} \cdot

\begin{pmatrix}
c_0\\
c_1\\
\vdots\\
c_{n-1}\\
c_n\\
\end{pmatrix}
\end{document}