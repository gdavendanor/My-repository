\documentclass{book}
\usepackage[spanish,es-nodecimaldot]{babel}
\usepackage[latin1]{inputenc}                      % Entrada de acentos
\usepackage[T1]{fontenc}
\usepackage{amsmath,amssymb,amsfonts,latexsym,cancel,stmaryrd,amsthm}

%---------------------------------------------------------------------
\usepackage{ans}   % > Debe haber al menos un ejercicio por cada secci�n
                   % > Use \exer:  si usted NO va a contestar la pregunta
                   % > Use \exera: si usted  va a contestar la pregunta
                   % > \subanswer ...(seguido de \exera). Use \par
                   %    para cambiar de p�rrafo y dejar un rengl�n al final,
                   %    se usa para delimitar la respuesta.
                   % > Subproblemas: puede usar \exer y \subexer o
                   %   \exera y \subexera
                   % > \exer\annot{hard}
\makeanswers
\artsec
\makesolutions
%---------------------------------------------------------------------

\begin{document}

\chapter{I}


\section{Lista de ejercicios con soluci�n }

\exercises

\exera Determine la ecuaci�n can�nica de la par�bola $y=2x^2-4x+1.$
\answer  $2x^2-4x+1=y \;\Rightarrow\; 2(x-1)^2= y+1 \;\Rightarrow\;(x-1)^2=\frac{1}{2}(y+1).$
 

\exera Determine la ecuaci\'{o}n can\'{o}nica de las siguientes par�bolas,\subexera $-9\,y^2 - 8\,x - 3=0$\\
  \subanswer $y^2=-\frac{8}{9}(x+\frac{3}{8})$
  %Se debe dejar un rengl�n
  
 \subexera $y^2+2 y-4 x=7$\\
 \subanswer $(y + 1)^2 = 4 (x + 2)$
 
  \subexera $x^2+2 x-2 y+5=0$\\
  \subanswer$(x + 1)^2 = 2 (y - 2)$
  
\exer Determine la ecuaci�n can�nica de la par�bola $y=4x^2-5x+1.$  
  
\eexer




\section{Lista 2 - sin soluci�n}

\exercises

\exer Determine la ecuaci�n can�nica de la par�bola $y=2x^2-4x+1.$
\solution $2x^2-4x+1=y \;\Rightarrow\; 2(x-1)^2= y+1 \;\Rightarrow\;(x-1)^2=\frac{1}{2}(y+1).$
 

\exer Determine la ecuaci\'{o}n can\'{o}nica de las siguientes par�bolas,\subexer $-9\,y^2 - 8\,x - 3=0$\\
  \solution $y^2=-\frac{8}{9}(x+\frac{3}{8})$\\
  %Se debe dejar un rengl�n
  
 \subexer $y^2+2 y-4 x=7$\\
 \solution $(y + 1)^2 = 4 (x + 2)$\\
 
  \subexer $x^2+2 x-2 y+5=0$\\
  \solution $(x + 1)^2 = 2 (y - 2)$\\
  
\eexer

Mensajes de error:\\

\begin{verbatim}
% \exera This exercise tests an error message.  
% It is marked to
% have an answer, but does not.

% \exer This exercise tests an error message too.
% \subanswer This should not be here.
\end{verbatim}




\doanswers %Para ejercicios con soluci�n

%\dosolutions % Para ejer "sin soluci�n"


\end{document}