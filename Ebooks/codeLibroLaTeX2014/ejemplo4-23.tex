\documentclass{article} %o report o book
\textheight=20cm
\textwidth=17cm
\topmargin=-2cm
%Símbolos matemáticos de la AMS
\usepackage{amsmath,amssymb,amsfonts,latexsym,cancel}
\usepackage[spanish]{babel}
\usepackage[latin1]{inputenc} %Acentos desde el teclado
\usepackage[T1]{fontenc}
% Comandos especiales
\newcommand{\sen}{\mathop{\rm sen}\nolimits} %seno
\newcommand{\arcsen}{\mathop{\rm arcsen}\nolimits}
\newcommand{\arcsec}{\mathop{\rm arcsec}\nolimits}
\newcommand{\R}{\mathbb{R}}
\newcommand{\N}{\mathbb{N}}
\newcommand{\Z}{\mathbb{Z}}
\def\max{\mathop{\mbox{\rm máx}}} % máximo
\def\min{\mathop{\mbox{\rm mín}}} % mínimo


\begin{document}
\begin{align*}
\intertext{Agrupamos,}
\frac{a+ay+ax+y}{x+y} &= \frac{ax+ay+x+y}{x+y} &\mbox{Agrupar}\\
\intertext{sacamos el factor común,}
&= \frac{a(x+y)+x+y}{x+y} &\mbox{Factor común}\\
&= \frac{(x+y)(a+1)}{x+y} &\mbox{Simplificar}\\
&= a+1
\end{align*}
\end{document}