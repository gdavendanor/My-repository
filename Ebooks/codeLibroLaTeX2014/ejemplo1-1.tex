\documentclass{article}

\usepackage[latin1]{inputenc}
\usepackage[T1]{fontenc}
\usepackage{amsmath}


\begin{document}

$0^0$ es una expresi�n indefinida.
Si $a>0$ entonces $a^0=1$ pero $0^a=0.$
Sin embargo, convenir en que $0^0=1$ es adecuado para que
algunas f�rmulas se puedan expresar de manera sencilla,
sin recurrir a casos especiales, por ejemplo

$$e^x=\sum _{n=0}^{\infty }\frac{x^n}{n!}$$
$$(x+a)^n=\sum_{k=0}^n \binom{n}{k}x^k a^{n-k}$$

\end{document}