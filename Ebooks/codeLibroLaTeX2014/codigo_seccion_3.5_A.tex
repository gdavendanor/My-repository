\documentclass{book}

% M�rgenes-------------------------------------------------------
\usepackage[total={18cm,21cm},top=2cm, left=2cm]{geometry}
\parindent=0mm

% Otros paquetes --------------------------------------------------
\usepackage{mathpazo} %fuente palatino
\usepackage{graphicx}
\usepackage{xcolor}
\usepackage{pstricks}
\usepackage[T1]{fontenc}
\usepackage[latin1]{inputenc} % 
\usepackage[spanish]{babel}   % Idioma espa�ol
\usepackage{latexsym,amsmath,amssymb,amsfonts,cancel}
% Referencias - ligas
\usepackage[hyphens]{url}
\usepackage[breaklinks,colorlinks=true, pdfstartview=FitV, linkcolor=red,
citecolor=styrmitcrnred, urlcolor=styrmitcrnblue]{hyperref}
\setcounter{chapter}{0}
\newtheorem{teo}{Teorema}[chapter]           %entorno para teoremas
\newtheorem{ejemplo}{{\it Ejemplo}}[chapter] %entorno para ejemplos
\newtheorem{defi}{Definici\'on}[chapter]     %entorno para definiciones
%Comandos -------------------------------------------------------
\newcommand{\sen}{\mathop{\rm sen}\nolimits} %seno
\newcommand{\arcsen}{\mathop{\rm arcsen}\nolimits}
\newcommand{\arcsec}{\mathop{\rm arcsec}\nolimits}


\begin{document}
  \title{\Huge Manual de \LaTeX\\
         {\small  \gray {\fontfamily{phv}\selectfont % gris y Helvetica
                         Instituto Tecnol\'ogico de Costa Rica\\
                         Escuela de Matem\'atica\\
                         Ense\~nanza de la Matem\'atica\\
                        }
          }}
  \author{Preparado por Prof. Walter Mora F. y Alexander Borb\'on A.}
  \date{2013}
  \maketitle %despliega el t\'itulo
  \tableofcontents
  
  \chapter{\LaTeX}
  \section{?`Qu� es \LaTeX?}

   ...
     \subsection{Pre�mbulo}
   ...
     \subsubsection{Acerca del T�tulo}
  
  \section{Deficiones, teoremas y ejemplos}
    \begin{defi}\label{definicion1} $f$ es de clase $C^1[a,b]$  si ....
    \end{defi}
    
    En la definici�n \ref{definicion1} se puede notar\\
    
     %Teorema
    \begin{teo} {\rm Si  $f \in C^1[a,b]$ entonces....} %fuente roman normal
    \end{teo}
    
         %Teorema
    \begin{ejemplo} Si $f(x)=\frac{1}{x-2}$ entonces  $f \in C^1[-1,1]$.
    \end{ejemplo}
  

  \addcontentsline{toc}{chapter}{Bibliograf�a}
  \begin{thebibliography}{99}
    \bibitem{Hahn} Hahn, J.``\LaTeX $\,$ for eveyone''. Prentice Hall,
              New Jersey, 1993.
    ...
  \end{thebibliography}
  
  
\end{document}



