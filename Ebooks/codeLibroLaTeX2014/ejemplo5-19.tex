\documentclass{article} %o report o book
\textheight=20cm
\textwidth=17cm
\topmargin=-2cm
\usepackage{amsmath,amssymb,amsfonts,latexsym,cancel,longtable}
\usepackage[spanish]{babel}
\usepackage[latin1]{inputenc} %Acentos desde el teclado
\usepackage[T1]{fontenc}
\usepackage[small,bf,labelsep=period]{caption}
\usepackage{booktabs}
% Comandos especiales
\newcommand{\sen}{\mathop{\rm sen}\nolimits} %seno
\newcommand{\arcsen}{\mathop{\rm arcsen}\nolimits}
\newcommand{\arcsec}{\mathop{\rm arcsec}\nolimits}
\newcommand{\R}{\mathbb{R}}
\newcommand{\N}{\mathbb{N}}
\newcommand{\Z}{\mathbb{Z}}
\def\max{\mathop{\mbox{\rm m�x}}} 
\def\min{\mathop{\mbox{\rm m�n}}} 

\begin{document}	

\begin{center}
	\begin{tabular}{cll} % usa "booktabs" y "caption"
		$i$ & $x_i$ & $y_i=f(x_i)$ \\ \midrule
		1 & $x_0=0$ & $0$ \\
		2 & $x_1=0.75$ & $-0.0409838$ \\
		3 & $x_2=1.5$ & $1.31799$ \\ \bottomrule
	\end{tabular}
	\captionof{table}{Tabla usando {\tt "captionof\{table\}\{...\}" }}
\end{center}
\end{document}