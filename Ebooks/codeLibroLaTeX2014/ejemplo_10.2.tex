\documentclass{book}
\usepackage{amsthm,amsmath,amssymb,amsfonts,pstricks}
\usepackage[latin1]{inputenc}
\newcommand{\Z}{\mathbb{Z}}
\usepackage{xparse}
\newcounter{midefi}[chapter] % Contador 1.1, 1.2,... en rojo
\renewcommand{\themidefi}{{\red\thechapter.\arabic{midefi}}}
\NewDocumentEnvironment{midefinicion}{O{}O{}}{%2 argumentos opcionales
%Inicio del entorno
\bigskip
\begin{minipage}{\textwidth}
% Definición - número - descripción (opcional)
\textbf{Definición \;\themidefi \;{\small\sffamily #1}}
}{% Parámetro #2: Contenido del entorno
#2
\end{minipage}
\bigskip
}%
\usepackage{tikz}
\usetikzlibrary{matrix,backgrounds}
\pgfdeclarelayer{wfondo}
\pgfsetlayers{wfondo,background,main}
\NewDocumentCommand{\iluminar}{O{blue!40} m m}{% rect�ngulo
\draw[#1, fill=#1] (#2.north west) rectangle (#3.south east);
% }
\begin{document}
\begin{center}
\raisebox{1cm}{$M\;=\;$}
\begin{tikzpicture}
\matrix (m)[matrix of math nodes,left delimiter=(,right delimiter=)]
{
1 & 2 & -1 & 2\\
0 & 1 & 3 & 2\\
0 & 0 & 1 & 0\\
0 & 3 & 0 & 5\\
}; % punto y coma!
% Iluminar de submatriz y elementos
\begin{pgfonlayer}{wfondo}
% Iluminar submatriz desde m_{2,2} hasta m_{4,4}
\iluminar[green!30]{m-2-2}{m-4-4}
%Iluminar elementos de una lista
\foreach \element in {m-4-1,m-3-2,m-2-3,m-1-4}{
\iluminar[violet!30]{\element}{\element} }
% Ilumnar elemento m_{1,1}
\iluminar[gray!30]{m-1-1}{m-1-1}
\end{pgfonlayer}
,\end{tikzpicture}
\end{center}
\end{document}