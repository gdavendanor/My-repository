\documentclass{book}
\usepackage[spanish]{babel}
\usepackage[latin1]{inputenc}
\usepackage[T1]{fontenc}
\usepackage{amsmath,amssymb,amsfonts,latexsym,cancel,stmaryrd,amsthm}
%---------------------------------------------------------------------
% Puede poner el archivo "ans.sty" en su carpeta de trabajo
\usepackage{ans} % > Debe haber al menos un ejercicio por cada secci�n
% > Use \exer: si usted NO va a contestar la pregunta
% > Use \exera: si usted va a contestar la pregunta
% > \subanswer ...(seguido de \exera). Use \par
% para cambiar de p�rrafo y dejar un rengl�n al final,
% �ste se usa para delimitar la respuesta.
% > Subproblemas: puede usar \exer y \subexer o
% \exera y \subexera
% > \exer\annot{hard}
\makeanswers
\artsec
\makesolutions
%---------------------------------------------------------------------
\begin{document}
\chapter{I}
\section{Lista de ejercicios con soluci�n }
%--Inicio de la lista de ejercicios
\exercises
\exera Determine la ecuaci�n can�nica de la par�bola $y=2x^2-4x+1.$
\answer $2x^2-4x+1=y \;\Rightarrow\; 2(x-1)^2= y+1 \;
\Rightarrow\;(x-1)^2=\frac{1}{2}(y+1).$
\exera Determine la ecuaci\?{o}n can\?{o}nica de las siguientes par�bolas,
\subexera $-9\,y^2 - 8\,x - 3=0$\\
\subanswer $y^2=-\frac{8}{9}(x+\frac{3}{8})$
% Se debe dejar un rengl�n para indicar el final de la respuesta
\subexera $y^2+2 y-4 x=7$\\
\subanswer $(y + 1)^2 = 4 (x + 2)$
%-
\subexera $x^2+2 x-2 y+5=0$\\
\subanswer$(x + 1)^2 = 2 (y - 2)$
%-
\exer Determine la ecuaci�n can�nica de la par�bola $y=4x^2-5x+1.$
\eexer
%--Fin de la lista de ejerciciosde esta secci�n
...
