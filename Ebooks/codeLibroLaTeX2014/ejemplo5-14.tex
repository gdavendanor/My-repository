\documentclass{article} %o report o book
\textheight=20cm
\textwidth=17cm
\topmargin=-2cm
\usepackage{amsmath,amssymb,amsfonts,latexsym,cancel}
\usepackage[spanish]{babel}
\usepackage[latin1]{inputenc} %Acentos desde el teclado
\usepackage[T1]{fontenc}
\usepackage{booktabs}
% Comandos especiales
\usepackage[x11names,table]{xcolor}
\newcommand{\sen}{\mathop{\rm sen}\nolimits} %seno
\newcommand{\arcsen}{\mathop{\rm arcsen}\nolimits}
\newcommand{\arcsec}{\mathop{\rm arcsec}\nolimits}
\newcommand{\R}{\mathbb{R}}
\newcommand{\N}{\mathbb{N}}
\newcommand{\Z}{\mathbb{Z}}
\def\max{\mathop{\mbox{\rmax}}} 
\def\min{\mathop{\mbox{\rmax}}} 

\begin{document}	

\begin{tabular}{|p{3cm}|p{10cm}|}\hline
	\textit{Representaci�n} & \textit{Notaci�n} \\ \hline
	$R_{4-4,2 }(O_{6})$ \par $R_{4}^{4-4,2 }(O_{6})$ &
	Representaci�n 3, en registro algebraico ($R^{4})$ en $\R$),
	interpretaci�n de la letra como inc�gnita (2), de la relaci�n
	entre volumen-altura-radio
	del vaso unidad ($O_{6}):\; U = \pi 2r^{2}h$ \par %Fin de párrafo
	Representaci�n 4, en registro algebraico ($R^{4})$ en el conjunto de
	los n�meros reales (4), interpretaci�n de la letra como inc�gnita (2),
	de la relaci�n entre volumen-altura-radio del vaso unidad
	($O_{6}): h = U/\pi 2r^{2}$. \\ \hline
	Resumen &
	\begin{enumerate}
		\item $R_{4-4,2 }(O_{6})$
		\item $R_{4}^{4-4,2 }(O_{6})$
		\item $R_{4-1,1 }(O_{8})$
	\end{enumerate} \\ \hline
\end{tabular}

\end{document}