\documentclass{article} %o report o book
\textheight=20cm
\textwidth=18cm
\topmargin=-2cm
%Símbolos matemáticos de la AMS
\usepackage{amsmath,amssymb,amsfonts,latexsym,cancel}
\usepackage[spanish]{babel}
\usepackage[latin1]{inputenc} %Acentos desde el teclado
\usepackage[T1]{fontenc}
% Comandos especiales
\newcommand{\sen}{\mathop{\rm sen}\nolimits} %seno
\newcommand{\arcsen}{\mathop{\rm arcsen}\nolimits}
\newcommand{\arcsec}{\mathop{\rm arcsec}\nolimits}
\newcommand{\R}{\mathbb{R}}
\newcommand{\N}{\mathbb{N}}
\newcommand{\Z}{\mathbb{Z}}
\def\max{\mathop{\mbox{\rm máx}}} % máximo
\def\min{\mathop{\mbox{\rm mín}}} % mínimo


\begin{document}
De acuerdo al lema de Euclides tenemos que
\begin{eqnarray*} % Espacio entre filas aumentado \\[0.2cm]
	\mbox{mcd}(a,b) & = & \mbox{mcd}(a-r_0q,r_0) \\[0.2cm]
	& = & \mbox{mcd}(r_1,r_0) \\[0.2cm]
	& = & \mbox{mcd}(r_1,r_0-r_1q_2)\\[0.2cm]
	& = & \mbox{mcd}(r_1,r_2) \\[0.2cm]
	& = & \mbox{mcd}(r_1-r_2q_2,r_2)\\[0.2cm]
\end{eqnarray*}
\end{document}