\documentclass[x11names,table]{article} 
\usepackage[total={14cm,21cm},top=2cm,left=2cm]{geometry}

\usepackage{latexsym,amsmath,amssymb,amsfonts}
\usepackage[latin1]{inputenc}
\usepackage[T1]{fontenc}
\usepackage{graphicx}
\usepackage[spanish]{babel} 

\usepackage{xcolor,pstricks}



\begin{document}

\begin{tabular}{|p{3cm}|p{10cm}|}\hline
\textit{Representaci�n} & \textit{Notaci�n} \\ \hline
$R_{4-4,2 }(O_{6})$ \par $R_{4}^{4-4,2 }(O_{6})$ &
Representaci�n 3, en registro algebraico ($R^{4})$ en $\bf R$),
interpretaci�n de la letra como inc�gnita (2), de la relaci�n
entre volumen-altura-radio
del vaso unidad ($O_{6}):\; U = \pi 2r^{2}h$ \par %Fin de p�rrafo
Representaci�n 4, en registro algebraico ($R^{4})$ en el conjunto de
los n�meros reales (4), interpretaci�n de la letra como inc�gnita (2),
de la relaci�n entre volumen-altura-radio del vaso unidad
($O_{6}): h = U/\pi 2r^{2}$. \\ \hline
Resumen &
\begin{enumerate}
\item $R_{4-4,2 }(O_{6})$
\item $R_{4}^{4-4,2 }(O_{6})$
\item $R_{4-1,1 }(O_{8})$
\end{enumerate} \\ \hline
\end{tabular}

\end{document}