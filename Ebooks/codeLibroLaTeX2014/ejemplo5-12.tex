\documentclass{article} %o report o book
\textheight=20cm
\textwidth=17cm
\topmargin=-2cm

\usepackage{amsmath,amssymb,amsfonts,latexsym,cancel}
\usepackage[spanish]{babel}
\usepackage[latin1]{inputenc} %Acentos desde el teclado
\usepackage[T1]{fontenc}

\usepackage{booktabs}
% Comandos especiales
\usepackage[x11names,table]{xcolor}
\newcommand{\sen}{\mathop{\rm sen}\nolimits} %seno
\newcommand{\arcsen}{\mathop{\rm arcsen}\nolimits}
\newcommand{\arcsec}{\mathop{\rm arcsec}\nolimits}
\newcommand{\R}{\mathbb{R}}
\newcommand{\N}{\mathbb{N}}
\newcommand{\Z}{\mathbb{Z}}


\begin{document}	

\begin{table}[h!]
	\centering
	\begin{tabular}{lll}
		&\multicolumn{2}{c}{Estimaci�n del error}\\
		&\multicolumn{2}{c}{absoluto y relativo}\\
		\rowcolor{LightBlue2} $x_n$ &$x_{n+1}$
		& $|x_{n+1}-x_n|/|x_{n+1}|$\\ \midrule
		-3.090721649 & 2.990721649 &1.6717\\
		-2.026511552 & 1.064210097 &0.525143859\\
		-1.205340185 & 0.821171367 &0.681277682\\ \bottomrule
	\end{tabular}
	\caption{}
\end{table}
\end{document}