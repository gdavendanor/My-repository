\documentclass{beamer}
\usepackage{amsmath,amssymb,amsfonts,latexsym,stmaryrd}
\usepackage[latin1]{inputenc}
\usepackage[T1]{fontenc}
% Conversi�n eps to pdf, requiere habilitar shell-escape.
% Texlive 2010 o superior No lo necesita
%\usepackage{epstopdf}
%\DeclareGraphicsExtensions{.pdf,.png,.jpg}
\usefonttheme{professionalfonts} % fuentes de LaTeX
\usetheme{Warsaw} % Tema escogido en este ejemplo
\setbeamercovered{transparent} % Velos
\newtheorem{teo}{Teorema}
\newtheorem{ejemplo}{Ejemplo}
\newtheorem{defi}{Definici�n}
\newtheorem{coro}{Corolario}
\newtheorem{prueba}{Prueba}
\begin{document}
\begin{frame}
\frametitle{Campo Galois $GF(p^r)$}
\framesubtitle{Resumen}
\begin{enumerate}
\item Todo dominio integral {\em finito} es un campo\\
\item Si $F$ es un campo con $q$ elementos, y $a$
es un elemento no nulo de $F$, entonces $a^{q-1}=1$\\
\item Si $F$ es un campo con $q$ elementos, entonces cualquier
$a \in \, F$ satisface la ecuaci�n $x^q-x=0$\\
\end{enumerate}
\end{frame}
\end{document}