\documentclass{beamer}
\usepackage{amsmath,amssymb,amsfonts,latexsym,stmaryrd}
\usepackage[latin1]{inputenc}
\usepackage[T1]{fontenc}
% Conversión eps to pdf, requiere habilitar shell-escape.
% Texlive 2010 o superior No lo necesita
%\usepackage{epstopdf}
%\DeclareGraphicsExtensions{.pdf,.png,.jpg}
\usefonttheme{professionalfonts} % fuentes de LaTeX
\usetheme{Warsaw} % Tema escogido en este ejemplo
\setbeamercovered{transparent} % Velos
\newtheorem{teo}{Teorema}
\newtheorem{ejemplo}{Ejemplo}
\newtheorem{defi}{Definición}
\newtheorem{coro}{Corolario}
\newtheorem{prueba}{Prueba}
\begin{document}
\begin{frame}
	\frametitle{Campo Galois $GF(p^r)$}
	\framesubtitle{Resumen}
	\begin{enumerate}[<+->] % <- Nueva opción
		\item Sea $F$ un campo con $q$ elementos y $a$ un elemento no
		nulo de $F$. Si $n$ es el orden de $a$, entonces $n|(q-1)$.
		\item Sea $p$ primo y $m(x)$ un polinomio irreducible de grado
		$r$ en $Z_p[x]$.
		Entonces la clase residual $Z_p[x]/\equiv_{m(x)}$ es un campo
		con $p^r$ elementos que contiene $Z_p$ y una raíz de $m(x)$.
		\item Sea $F$ un campo con $q$ elementos.
		Entonces $q=p^r$ con $p$ primo y $r \in \, N$
	\end{enumerate}
\end{frame}
\end{document}