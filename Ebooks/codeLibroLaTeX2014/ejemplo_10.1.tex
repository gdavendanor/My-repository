\documentclass{book}
\usepackage{amsthm,amsmath,amssymb,amsfonts,xcolor}
\usepackage[latin1]{inputenc}
\definecolor{azulF}{rgb}{.0,.0,.3} % Azul
\definecolor{rojoF}{RGB}{212,0,0} % Rojo
\newcommand{\Z}{\mathbb{Z}} % \Z
% Estilo B
\newtheoremstyle{estiloB}{}{}{}{}%
{\color{azulF}\bfseries} % fuente del encabezado
{.} % puntuación
{\newline} % espacio después del encabezado
{\thmname{#1}~\thmnumber{\color{rojoF} #2}\thmnote{~\color{azulF}(#3)}}
%%--
\theoremstyle{estiloB}
\newtheorem{unadefiB}{Definición}[chapter]
\newtheorem{unteoB}{Teorema}[section]
%%--
% Estilo C
\newtheoremstyle{estiloC}{}{}{}{}%
{\color{rojoF}\bfseries}%
{.}{\newline}{}%
%--
\swapnumbers % Intercambiar número-teorema
\theoremstyle{estiloC}
\newtheorem{unteoC}{Teorema}
%--
\begin{document}
\begin{unadefiB}[Divisibilidad]\label{divisibilidad}
Sean $a,b$ enteros con $b \not = 0.$ Decimos que $b$ divide a $a$
si existe un entero $c$ tal que $a=bc.$
\end{unadefiB}
\begin{unteoB}
Si $a,b \in \Z,\;$ y si $a|b\;$ y $\;b|a$ entonces $\;|a|=|b|$
\end{unteoB}
\begin{unteoC}
Si $a,b \in \Z,\;$ y si $a|b\;$ y $\;b|a$ entonces $\;|a|=|b|$
\end{unteoC}
\end{document}