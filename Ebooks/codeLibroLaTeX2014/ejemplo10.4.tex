\documentclass{book}
\usepackage{amsthm,amsmath,amssymb,amsfonts,xcolor}
\usepackage[latin1]{inputenc}
\definecolor{azulF}{rgb}{.0,.0,.3} % Azul
\definecolor{rojoF}{RGB}{212,0,0} % Rojo
\newcommand{\Z}{\mathbb{Z}} % \Z
% Estilo B
\newtheoremstyle{estiloB}{}{}{}{}%
{\color{azulF}\bfseries} % fuente del encabezado
{.} % puntuación
{\newline} % espacio después del encabezado
{\thmname{#1}~\thmnumber{\color{rojoF} #2}\thmnote{~\color{azulF}(#3)}}
%%--
\theoremstyle{estiloB}
\newtheorem{unadefiB}{Definici�n}[chapter]
\newtheorem{unteoB}{Teorema}[section]
%%--
% Estilo C
\newtheoremstyle{estiloC}{}{}{}{}%
{\color{rojoF}\bfseries}%
{.}{\newline}{}%
%--
\swapnumbers % Intercambiar número-teorema
\theoremstyle{estiloC}
\newtheorem{unteoC}{Teorema}
\usepackage{tikz}
\usetikzlibrary{matrix,backgrounds}
\pgfdeclarelayer{wfondo}
\pgfsetlayers{wfondo,background,main}
\NewDocumentCommand{\iluminar}{O{blue!40} m m}{% rect�ngulo
\draw[#1, fill=#1] (#2.north west) rectangle (#3.south east);
\begin{document}

\begin{center}
\raisebox{1cm}{$M\;=\;$}
\begin{tikzpicture}
\matrix (m)[matrix of math nodes,left delimiter=(,right delimiter=)]
{
1 & 2 & -1 & 2\\
0 & 1 & 3 & 2\\
0 & 0 & 1 & 0\\
0 & 3 & 0 & 5\\
}; % punto y coma!
% Iluminar de submatriz y elementos
\begin{pgfonlayer}{wfondo}
% Iluminar submatriz desde m_{2,2} hasta m_{4,4}
\iluminar[green!30]{m-2-2}{m-4-4}
%Iluminar elementos de una lista
\foreach \element in {m-4-1,m-3-2,m-2-3,m-1-4}{
\iluminar[violet!30]{\element}{\element} }
% Ilumnar elemento m_{1,1}
\iluminar[gray!30]{m-1-1}{m-1-1}
\end{pgfonlayer}
\end{tikzpicture}
\end{center}

\end{document}