\documentclass{book}
% Dimensiones-m�rgenes
\usepackage[centering,text={18cm,22cm},showframe=false]{geometry}

\usepackage{tcolorbox}        
\tcbuselibrary{skins,breakable} 

\usepackage{xparse}   % paquete para hacer entornos con par�metros
\usepackage{answers}  % paquete answers
\newtheorem{exer}{}[chapter]
\newenvironment{ejer}{\begin{exer}\normalfont}{\end{exer}}
\Newassociation{solu}{Soln}{ans}
% Entorno "ejercicios"------------------------------------------------
\NewDocumentEnvironment{ejercicios}{O{}}{%
\bigskip\begin{minipage}{\textwidth}{\bf Ejercicios}\\#1}{\end{minipage}\bigskip}
%-- Entorno con cajas --
%-- Caja
\colorlet{color1}{gray!5!white}
\definecolor{color2}{RGB}{117,184,68} 
\newtcolorbox{wwlistaejercicios}[1][]{%
arc=0mm,breakable,enhanced,colback=color1,boxrule=0pt,top=8mm, 
enlarge top by=\baselineskip/2+1mm, enlarge top at break by=0mm,pad at break=2mm,fontupper=\normalsize,
overlay ={
\node[rectangle, minimum width=4cm, top color=color2, bottom color=color2, 
inner sep=1mm,anchor=west,font=\normalsize] at ([xshift=0pt,yshift=-3mm]frame.north west)%
{\textbf{Ejercicios}};}#1}
\NewDocumentEnvironment{ejerciciosCaja}{O{}}{%
\bigskip\begin{wwlistaejercicios}%
 #1}{\end{wwlistaejercicios}\bigskip } % 
%----------------------------------------
\newcount\ansj
%\ansj=1  % sin tabla de contenidos
\ansj=\thechapter
\makeatletter % -- redefinir chapter solo con la modificaci�n pertinente
\let\stdchapter\chapter
\renewcommand*\chapter{% -- abrir y cerrar archivos ans j
\expandafter\ifx\csname Closesolutionfile\endcsname \relax\else
\Closesolutionfile{ans}\fi
\expandafter\ifx\csname Opensolutionfile\endcsname \relax\else
\Opensolutionfile{ans}[ans\number\ansj]\advance\ansj by 1\fi
                       %-- continuar
\@ifstar{\starchapter}{\@dblarg\nostarchapter}}
\newcommand*\starchapter[1]{\stdchapter*{#1}}
\def\nostarchapter[#1]#2{\stdchapter[{#1}]{#2}}
\makeatother % -- 

%-- Cerrar   el �ltimo archivo "ans"
\def\soluciones{ 
        \expandafter\ifx\csname Closesolutionfile\endcsname \relax\else
        \Closesolutionfile{ans}\fi }
%-- Imprimir soluciones del cap�tulo i
\def\solucionesCap#1{\section*{Soluciones del Cap\'{\i}tulo #1}
                     \input{ans#1}}
%---------------------------------------------------------------------

\begin{document}

\tableofcontents

\chapter{A} %--
\section{Ejercicios: Lista A1}

\begin{ejerciciosCaja}
  \begin{ejer}   Resolver $|\tan(\theta)|=1$  con $\theta \in\, R.$ 
    \begin{solu}
      {\bf Sugerencia:} Mmmmmm
    \end{solu}
  \end{ejer}
  \begin{ejer}  Resolver $|\sec(\theta)|=1$ con $\theta \in\, R^+$
    \begin{solu}
	$|\sec(\theta)|=1 \Longrightarrow....$
    \end{solu}
  \end{ejer}
\end{ejerciciosCaja}
%...
\section{Ejercicios: Lista A2}

\begin{ejercicios}
  \begin{ejer}  Resolver $a+1=2$
    \begin{solu}
          $a=1$
    \end{solu}
  \end{ejer}
  ...
\end{ejercicios}

\chapter{B} % --
            % ... No hay ejercicios aqu�

\chapter{C} % --
\section{Ejercicios: Lista C1}

\begin{ejercicios}
  \begin{ejer}  Resolver $c+1=2$
    \begin{solu}
      ...$c=1$
    \end{solu}
  \end{ejer}
  ...
\end{ejercicios}
% ----------------------------------------------------------------


% Imprimir respuestas --
\soluciones  % Obligatorio
\solucionesCap{1}
%\solucionesCap{2} % no hay en este cap�tulo
\solucionesCap{3}

\end{document}