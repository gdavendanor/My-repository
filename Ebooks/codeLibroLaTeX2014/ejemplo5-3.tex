\documentclass{article} %o report o book
\textheight=20cm
\textwidth=17cm
\topmargin=-2cm

\usepackage{amsmath,amssymb,amsfonts,latexsym,cancel}
\usepackage[spanish]{babel}
\usepackage[latin1]{inputenc} %Acentos desde el teclado
\usepackage[T1]{fontenc}
\usepackage{longtable,multirow,booktabs}
% redefinir cuadro por tabla
\AtBeginDocument{\renewcommand\tablename{Tabla}}
% Comandos especiales
\newcommand{\sen}{\mathop{\rm sen}\nolimits} %seno
\newcommand{\arcsen}{\mathop{\rm arcsen}\nolimits}
\newcommand{\arcsec}{\mathop{\rm arcsec}\nolimits}
\newcommand{\R}{\mathbb{R}}
\newcommand{\N}{\mathbb{N}}
\newcommand{\Z}{\mathbb{Z}}
\def\max{\mathop{\mbox{\rm m�x}}} % mximo
\def\min{\mathop{\mbox{\rm m�n}}} % m�nimo


\begin{document}	

%\usepackage{longtable,multirow,booktabs}
%--Redefinir cuadro por tabla
%\AtBeginDocument{\renewcommand\tablename{Tabla} }
\begin{longtable}[c]{llr} 
	\caption{Tabla grande}\label{ej1:longtable} \\ \toprule
	Nombre & Apellido & Edad \\ \midrule
	Julio & Cort\'es & 11 \\
	Marco & Villalta & 13 \\
	Mario & Rami�rez & 12 \\ \bottomrule
\end{longtable}



\end{document}