\documentclass[xcolor=pdftex, x11names,table]{article}
\usepackage[latin1]{inputenc}
\usepackage{amsmath,amsfonts,amssymb}
\usepackage{xcolor, pstricks}
%  RequierePDFLaTEX  "pdfscreen"  -------------------------------------------------%
\usepackage[spanish,screen,panelright,gray,paneltoc]{pdfscreen}
\overlayempty
\backgroundcolor{white}
%\topbuttons
\urlid{www.tec-digital.itcr.ac.cr/revistamatematica/}
\affname{Revista digital Matem�tica, Educaci�n e Internet}
\divname{Escuela de matem�tica}
\margins{.75in}{.75in}{.75in}{.75in}
% alto ancho
\screensize{7in}{10in}
\renewcommand{\baselinestretch}{1.1}
%---------------------------------------------------------------------------------


\begin{document}
\section{Qu� es la regresi�n}
El an�lisis de regresi�n es una t�cnica estad�stica que permite encontrar una ecuaci�n que aproxime una variable como funci�n de otras.  T�picamente, las variables son atributos de los individuos en una poblaci�n, y el an�lisis trabaja a partir de los valores de los atributos para alguna muestra de individuos.  La variable que se escribe como funci�n de las otras se llama \emph{resultado}, y las otras son los \emph{predictores}.
La \emph{regresi�n simple} se usa cuando hay un solo predictor.  \\

Como ejemplo de esto, al relacionar la edad $x$
en a�os con la estatura $y$ en cent�metros para ni�os menores de doce a�os,
se busca una funci�n $y = f(x)$.  Si adem�s la funci�n buscada es lineal, $y = a + bx$,
entonces se habla de \emph{regresi�n lineal simple}.\\
\clearpage
\section{Qu� es... }
\clearpage
\section{Qu� es ...}
\end{document}
