\documentclass{article}
  \usepackage[total={18cm,21cm},top=2cm, left=2cm]{geometry}
  %Paquetes adicionales
  \usepackage{latexsym,amsmath,amssymb,amsfonts}
  \usepackage[latin1]{inputenc}
  \usepackage[T1]{fontenc}
  \usepackage{graphicx}
  \usepackage[spanish]{babel} % Idioma espa�ol
  %---------------comandos especiales-------------------------------
  \newcommand{\R}{\mathbb{R}}
  \newcommand{\Z}{\mathbb{Z}}
  \newcommand{\Q}{\mathbb{Q}}
  \newcommand{\C}{\mathbb{C}}
  \newcommand{\N}{\mathbb{N}}
  \newcommand{\I}{\mathbb{I}}
  \newcommand{\F}{\mathbb{F}}
  %-----------------------------------------------------------------



\begin{document}
{\sc Instituto Tecnol�gico de Costa Rica} \hfill Tiempo 2:30 horas\\
{\sc Escuela de Matem�tica} \hfill Puntaje: 21 puntos\\
{\sc MA-0441. Primer Parcial}\\\\
{\bf Instrucciones:} Este es un examen de desarrollo, por lo tanto deben aparecer 
todos los pasos que lo llevan a su respuesta. Trabaje de manera clara y ordenada.\\

\begin{enumerate}
\item {\bf [3 Puntos]} Sea $A=\{1,b,c,d,7\}$ y $B=\{1,2,c,d\}.$
Calcule ${\cal P}(A\,\Delta\,B).$

\item {\bf [5 Puntos]} Muestre que $A-(B\,\cap\,C)=(A-B)\,\cup\,(A - C)$

\item {\bf [5 Puntos]} Mostrar que $[\;A\,\cup\,C\;\subseteq\;B\,\cup\,C
\;\;\wedge\;\; A\,\cap\,C=\emptyset\;]\;\Longrightarrow\;A\,\subseteq\,B$

\item {\bf [2 Puntos]} Sea $\Re=(\R^*,\R^*,R)$
definida por $x\,\Re\, y\;\Longleftrightarrow\; xy\;>\; 0.$

    \begin{enumerate}
    \item {\bf [3 Puntos]} Muestre  que $\Re$ es una relaci�n de equivalencia.
    \item {\bf [2 Puntos]} Determine las clases de equivalencia $\overline{1}$ y $\overline{-1}.$
    \item {\bf [1 Punto]}  Determine $\R^*/\Re$ (el conjunto cociente).
    \end{enumerate}
\end{enumerate} 
\end{document}