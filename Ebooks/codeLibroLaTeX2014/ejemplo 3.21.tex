\documentclass[letterpaper, 12pt]{article} 
\usepackage[total={14cm,21cm},top=2cm, left=2cm]{geometry}
\parindent = 0mm % Sin sangr�a
\usepackage{latexsym,amsmath,amssymb,amsfonts}
\usepackage[latin1]{inputenc}
\usepackage[T1]{fontenc}
\usepackage{graphicx}
\usepackage[spanish]{babel} % Idioma espa�ol
\renewcommand{\baselinestretch}{1.1} % espaciado 1.1
\pagestyle{myheadings}
\markright{...... texto .......} % Encabezados simples
\usepackage{multicol}
\usepackage{xcolor}
\usepackage{pstricks}
\usepackage{marginnote}
\newcommand{\R}{\mathbb{R}}
\newcommand{\Q}{\mathbb{Q}}
\newcommand{\Z}{\mathbb{Z}}
\newcommand{\I}{\mathbb{I}}
\newcommand{\C}{\mathbb{C}}
\newcommand{\E}{\mathbb{E}}
\newcommand{\N}{\mathbb{N}}

\begin{document}

\renewcommand{\labelenumi}{\Roman{enumi}.}
\renewcommand{\labelenumii}{\arabic{enumii}$)$ }
\renewcommand{\labelenumiii}{\alph{enumiii}$)$ }
\renewcommand{\labelenumiv}{$\bullet$ }
\begin{enumerate}
\item Primer nivel (en Romanos)
\begin{enumerate}
\item Segundo nivel (en numeraci�n ar�biga)
\begin{enumerate}
\item Tercer nivel (numeraci�n alfab�tica)
\begin{enumerate}
\item Cuarto nivel (usamos {\tt bullet})
\end{enumerate}
\end{enumerate}
\end{enumerate}
\end{enumerate}

\end{document}