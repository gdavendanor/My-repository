\documentclass{book}
    \usepackage{amsmath}
    \usepackage[T1]{fontenc}
    \usepackage[latin1]{inputenc}
\begin{document}
%\begin{titlepage}TITULO 1\end{titlepage} 
%\begin{titlepage} TITULO 2\end{titlepage} 
%\title{Ooops}
%\titlepage

$\pmb{3+\sin a=5\cos\mu}$\\
\mathversion{bold}$3+\sin a = 5\cos\mu$, $4+5\tan\mu$\mathversion{normal}\\\\
   $0^0$ es una expresi�n indefinida.
   Si $a>0$ entonces $a^0=1$ pero $0^a=0.$
   Sin embargo, convenir en que $0^0=1$ es adecuado para que
   algunas f�rmulas se puedan expresar de manera sencilla,
   sin recurrir a casos especiales, por ejemplo
   $$e^x=\sum _{n=0}^{\infty }\frac{x^n}{n!}$$
   $$(x+a)^n=\sum_{k=0}^n \binom{n}{k}x^k a^{n-k}$$

\centerline{\begin{minipage}{0.5\textwidth}
$\pmb{3+\sin a=5\cos\mu}$\\
\mathversion{bold}$3+\sin a = 5\cos\mu$, $4+5\tan\mu$\mathversion{normal}\\\\
   $0^0$ es una expresi�n indefinida.
   Si $a>0$ entonces $a^0=1$ pero $0^a=0.$
   Sin embargo, convenir en que $0^0=1$ es adecuado para que
   algunas f�rmulas se puedan expresar de manera sencilla,
   sin recurrir a casos especiales, por ejemplo
   $$e^x=\sum _{n=0}^{\infty }\frac{x^n}{n!}$$
   $$(x+a)^n=\sum_{k=0}^n \binom{n}{k}x^k a^{n-k}$$
\end{minipage}}

\vspace{2cm}
\hfill\begin{tabular}{@{}p{.5\linewidth}@{}}
\multicolumn{1}{@{}c@{}}{Some title} \\ 
   %----------------------------------------------------------------------------
$\pmb{3+\sin a=5\cos\mu}$\\
\mathversion{bold}$3+\sin a = 5\cos\mu$, $4+5\tan\mu$\mathversion{normal}\\\\
   $0^0$ es una expresi�n indefinida.
   Si $a>0$ entonces $a^0=1$ pero $0^a=0.$
   Sin embargo, convenir en que $0^0=1$ es adecuado para que
   algunas f�rmulas se puedan expresar de manera sencilla,
   sin recurrir a casos especiales, por ejemplo
   $$e^x=\sum _{n=0}^{\infty }\frac{x^n}{n!}$$
   $$(x+a)^n=\sum_{k=0}^n \binom{n}{k}x^k a^{n-k}$$ 
   %----------------------------------------------------------------------------
\end{tabular}


%\vspace{2cm}

\begin{minipage}[t]{0.4\textwidth} % Start the left-hand side of the page
\vspace{0pt}
  %----------------------------------------------------------------------------
$\pmb{3+\sin a=5\cos\mu}$\\
\mathversion{bold}$3+\sin a = 5\cos\mu$, $4+5\tan\mu$\mathversion{normal}\\\\
   $0^0$ es una expresi�n indefinida.
   Si $a>0$ entonces $a^0=1$ pero $0^a=0.$
   Sin embargo, convenir en que $0^0=1$ es adecuado para que
   algunas f�rmulas se puedan expresar de manera sencilla,
   sin recurrir a casos especiales, por ejemplo
   $$e^x=\sum _{n=0}^{\infty }\frac{x^n}{n!}$$
   $$(x+a)^n=\sum_{k=0}^n \binom{n}{k}x^k a^{n-k}$$ 
   %----------------------------------------------------------------------------
\end{minipage}\hfill\begin{minipage}[t]{0.4\textwidth}

  %----------------------------------------------------------------------------
$\pmb{3+\sin a=5\cos\mu}$\\
\mathversion{bold}$3+\sin a = 5\cos\mu$, $4+5\tan\mu$\mathversion{normal}\\\\
   $0^0$ es una expresi�n indefinida.
   
   $$e^x=\sum _{n=0}^{\infty }\frac{x^n}{n!}$$
   $$(x+a)^n=\sum_{k=0}^n \binom{n}{k}x^k a^{n-k}$$ 
   %----------------------------------------------------------------------------
\end{minipage}

%\vspace{2cm}

\begin{minipage}[t]{0.4\textwidth} % Start the left-hand side of the page
%Sin truco\vspace{0pt}
  %----------------------------------------------------------------------------
$\pmb{3+\sin a=5\cos\mu}$\\
\mathversion{bold}$3+\sin a = 5\cos\mu$, $4+5\tan\mu$\mathversion{normal}\\\\
   $0^0$ es una expresi�n indefinida.
   Si $a>0$ entonces $a^0=1$ pero $0^a=0.$
   Sin embargo, convenir en que $0^0=1$ es adecuado para que
   algunas f�rmulas se puedan expresar de manera sencilla,
   sin recurrir a casos especiales, por ejemplo
   $$e^x=\sum _{n=0}^{\infty }\frac{x^n}{n!}$$
   $$(x+a)^n=\sum_{k=0}^n \binom{n}{k}x^k a^{n-k}$$ 
   %----------------------------------------------------------------------------
\end{minipage}\hfill\begin{minipage}[t]{0.4\textwidth}

  %----------------------------------------------------------------------------
$\pmb{3+\sin a=5\cos\mu}$\\
\mathversion{bold}$3+\sin a = 5\cos\mu$, $4+5\tan\mu$\mathversion{normal}\\\\
   $0^0$ es una expresi�n indefinida.
   
   $$e^x=\sum _{n=0}^{\infty }\frac{x^n}{n!}$$
   $$(x+a)^n=\sum_{k=0}^n \binom{n}{k}x^k a^{n-k}$$ 
   %----------------------------------------------------------------------------
\end{minipage}

\end{document}



