\documentclass[fleqn]{article}
\usepackage[spanish,es-noshorthands]{babel}
\usepackage[utf8]{inputenc} 
\usepackage[papersize={5.5in,8.5in},total={4.5in,7.25in},centering]{geometry}
\usepackage{mathexam}
\usepackage{amsmath}
\usepackage{graphicx}
\usepackage{multicol}

\ExamClass{\includegraphics[height=16pt]{Images/logo-sed.png} Aritmética $6^{\circ}$}
\ExamName{Fracciones 2}
\ExamHead{\includegraphics[height=16pt]{Images/logo-colegio.png} IEDAB}
\newcommand{\LineaNombre}{%
\par
\vspace{\baselineskip}
Nombre:\hrulefill \; Curso: \underline{\hspace*{48pt}} \; Fecha: \underline{\hspace*{2.5cm}} \relax
\par}
\let\ds\displaystyle

\begin{document}
\ExamInstrBox{
Respuesta sin justificar mediante procedimiento no será tenida en cuenta en la calificación. Escriba sus respuestas en el espacio indicado. Tiene 45 minutos para contestar esta prueba.}
\LineaNombre
\begin{enumerate}
 \item Determine si cada una de las siguientes fracciones es "propia", "impropia" o "equivalente a la unidad"
 \begin{enumerate}
 \begin{multicols}{2}
 \item $\dfrac{2}{5}$
 \item $\dfrac{3}{3}$
 \item $\dfrac{5}{4}$
 \item $\dfrac{9}{10}$ 
 \end{multicols}
 \end{enumerate}
 \item Escriba cada número mixto como una fracción impropia
 \begin{enumerate}
 \begin{multicols}{2}
 \item $3\frac{1}{2}=$
 \item $4\frac{1}{3}=$
 \item $1\frac{2}{5}=$
 \item $2\frac{2}{3}=$ 
 \end{multicols}
 \end{enumerate}
 \item 
 \end{enumerate}
\end{document}
