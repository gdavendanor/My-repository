\documentclass[10pt,twoside]{article}
\usepackage[utf8]{inputenc}
\usepackage{amsmath}
\usepackage{amsfonts}
\usepackage{amssymb}
\usepackage[spanish,es-noshorthands]{babel}
\usepackage[T1]{fontenc}
\usepackage{lmodern}
\usepackage{graphicx,hyperref}
\usepackage{tikz,pgf}
\usepackage{multicol}
\usepackage{subfig}
\usepackage[papersize={5.5in,8.5in},width=4.75in,height=7.3in]{geometry}

\author{Comité sindical\thanks{Colegio Arborizadora Baja I.E.D.}}
%\title{
%}
\date{}
\begin{document}
\begin{minipage}{.2\textwidth}
\includegraphics[height=1.55cm]{Images/logo-colegio.png}\end{minipage}
\begin{minipage}{.75\textwidth}
\begin{center}
\textbf{\large Colegio Arborizadora Baja I.E.D. (J.M.)}\\
\textbf{\large Guía Educativa para Estudiantes Ciclos 4 y 5}\\
\end{center}
\end{minipage}\hfill
\vspace*{-10pt}
\begin{center}
\large Comité sindical
\end{center}
\vspace*{-10pt}
%\maketitle
\vspace*{20pt}
Compañero(a) docente, puede complementar el taller con los estudiantes, haciendo lectura de ésta información que se presenta a continuación
\section*{Ser pilo paga, un programa que endeuda a las familias y desfinancia la educación pública}
“Por cada estudiante que entra al programa, el gobierno colombiano realiza una inversión anual de \$3 billones de pesos (Lea: ¿Vale la pena el esfuerzo del Estado para financiar Ser Pilo Paga?). Es por este motivo que varios críticos han expresado que el programa Ser Pilo Paga no garantiza el derecho a la educación superior ya que no todos los jóvenes logran ingresar a la universidad. Lo anterior debido a que el 98\% del dinero destinado al programa ser pilo paga está destinado a universidades privadas” tomado del artículo de El espectador (abril-19-2017) “Colombia necesita nuevas universidades públicas de calidad": Carlos Caicedo
Continúa el artículo con  “Ese modelo de financiación profundiza la desigualdad en el país, pues con los mismos recursos que se paga el estudio a 40.000 beneficiarios de Ser Pilo Paga se hubiera podido pagar la educación de 200.000 estudiantes en universidades públicas de calidad”.

“Ser Pilo paga es un lucrativo negocio que legitima la exclusión universitaria de los más pobres del país. Si las universidades privadas tienen algún sentido de responsabilidad social deben ser ellas las que financien el programa Ser Pilo Paga y dejar de quitarles ese dinero a las universidades públicas”.

De los 482.000 bachilleres, que se gradúan  cada año,  solo 10.000 0 12.000  Logran ingresar al programa,  se quedan aproximadamente 470.000 por fuera de la educación superior y la fuerza productiva de un país, depende de la formación de  calidad que tengan sus jóvenes y cuando culmine el programa, que se irán a inventar? para seguir saqueando los recursos que deben ser para una gran mayoría e invertirlos en la educación superior de nuestros bachilleres no en las arcas de las universidades privadas. A futuro la situación empeorará, puesto que no hay un plan de gobierno serio que permita tener aspiraciones para el ingreso a la educación superior, que debe ser uno de los fines de la educación media, en cualquier país, que quiera salir del  subdesarrollo, es lo mìnimo.
\end{document}
