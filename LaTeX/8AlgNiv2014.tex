\documentclass[10pt,twoside]{article}
\usepackage[utf8]{inputenc}
\usepackage{amsmath}
\usepackage{amsfonts}
\usepackage{amssymb}
\usepackage[spanish,es-noshorthands]{babel}
\usepackage[T1]{fontenc}
\usepackage{lmodern}
\usepackage{graphicx,hyperref}
\usepackage{tikz,pgf}
\usepackage{multicol}
\usepackage{subfig}
\usepackage[papersize={6.5in,8.5in},width=5.5in,height=7in]{geometry}
\usepackage{fancyhdr}
\pagestyle{fancy}
\fancyhead[LE]{\includegraphics[height=12pt]{Images/logo-colegio.png} Álgebra $8^{\circ}$}
\fancyhead[RE]{}
\fancyhead[RO]{\textit{Germ\'an Avenda\~no Ram\'irez, Lic. U.D., M.Sc. U.N.}}
\fancyhead[LO]{}

\author{Germ\'an Avenda\~no Ram\'irez, Lic. U.D., M.Sc. U.N.}
\title{\begin{minipage}{.2\textwidth}
\includegraphics[height=1.75cm]{Images/logo-colegio.png}\end{minipage}
\begin{minipage}{.55\textwidth}
\begin{center}
Taller Nivelación 2014,  \\
Álgebra $8^{\circ}$
\end{center}
\end{minipage}\hfill
\begin{minipage}{.2\textwidth}
\includegraphics[height=1.75cm]{Images/logo-sed.png} 
\end{minipage}}
\date{}
\begin{document}
\maketitle
Nombre: \hrulefill Curso: \underline{\hspace*{44pt}} Fecha: \underline{\hspace*{2.5cm}}
\section*{N\'{u}meros reales}
\begin{enumerate}
\item De la lista 0, $\sqrt{2}$, $\frac{3}{4}$, $-\frac{5}{6}$, $\frac{25}{3}$, $-\sqrt{3}$, $-8$, $0.34$, $0.2\overline{3}$, 67 y $\frac{9}{7}$, identifique entre \'{e}stos:
\begin{enumerate}\begin{multicols}{2}
\item Los n\'{u}meros naturales
\item Los enteros
\item Los enteros no negativos
\item Los racionales
\item Los irracionales
\end{multicols}
\end{enumerate}
Para los problemas \ref{first}--\ref{second}, determine la propiedad de la igualdad o de los números reales que justifica cada proposición. Por ejemplo $6(-7)=-7(6)$ es cierta por la propiedad conmutativa de la multiplicación; y si $2=x+3$, entonces $x+3=2$ por la propiedad simétrica de la igualdad.
\item $7+[3+(-8)]=(7+3)+(-8)$\label{first}
\item Si $x=2$ y $x+y=9$, entonces $2+y=9$
\item $-1(x+2)=-(x+2)$
\item $3(x+4)=3(x)+3(4)$
\item $[(17)(4)](25)=(17)[(4)(25)]$
\item $x+3=3+x$
\item $3(98)+3(2)=3(98+2)$
\item $\left(\frac{3}{4}\right)\left(\frac{4}{3}\right)=1$
\item Si $4=3x-1$, entonces $3x-1=4$\label{second}
Para los ejercicios \ref{third}--\ref{fourth}, simplifique cada expresión numérica
\begin{multicols}{2}
\item $-8\frac{1}{4}+\left(-4\frac{5}{8}\right)-\left(-6\frac{3}{8}\right)$\label{third}
\item $9\frac{1}{3}-12\frac{1}{2}+\left(-4\frac{1}{6}\right)-\left(-1\frac{1}{6}\right)$
\item $-8(2)-16\div (-4)+(-2)(-2)$
\item $4(-3)-12\div (-4)+(-2)(-1)-8$
\item $[48+(-73)]+74$
\item $3-[-2(3-4)]+7$
\item $(-2)^{4}+(-1)^{3}-3^{2}$
\item $[4(-1)-2(3)]^{2}$\label{fourth}
\end{multicols}
Para los ejercicios \ref{fith}--\ref{sixth}, Simplifique cada expresión algebraica reduciendo términos semejantes
\begin{multicols}{2}
\item $3a^{2}-2b^{2}-7a^{2}-3b^{2}$\label{fith}
\item $4x-6-2x-8+x+12$
\item $-\frac{2}{3}x^{2}y-\left(-\frac{3}{4}x^{2}y\right)-\frac{5}{12}x^{2}y-2x^{2}y$
\item $-2(3a-1)+4(2a+3)-5(3a+2)$
\item $3(2x-3y)-4(3x+5y)-x$
\item $-5(x^{2}-4)-2(3x^{2}+6)+(2x^{2}-1)$\label{sixth}
\end{multicols}
\end{enumerate}
\end{document}
