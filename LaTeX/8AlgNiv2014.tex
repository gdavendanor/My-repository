\documentclass[10pt,twoside]{article}
\usepackage[utf8]{inputenc}
\usepackage{amsmath}
\usepackage{amsfonts}
\usepackage{amssymb}
\usepackage[spanish,es-noshorthands]{babel}
\usepackage[T1]{fontenc}
\usepackage{lmodern}
\usepackage{graphicx,hyperref}
\usepackage{tikz,pgf}
\usepackage{multicol}
\usepackage{subfig}
\usepackage[papersize={6.5in,8.5in},width=5.5in,height=7in]{geometry}
\usepackage{fancyhdr}
\pagestyle{fancy}
\fancyhead[LE]{\includegraphics[height=12pt]{Images/logo-colegio.png} Álgebra $8^{\circ}$}
\fancyhead[RE]{}
\fancyhead[RO]{\textit{Germ\'an Avenda\~no Ram\'irez, Lic. U.D., M.Sc. U.N.}}
\fancyhead[LO]{}

\author{Germ\'an Avenda\~no Ram\'irez, Lic. U.D., M.Sc. U.N.}
\title{\begin{minipage}{.2\textwidth}
\includegraphics[height=1.75cm]{Images/logo-colegio.png}\end{minipage}
\begin{minipage}{.55\textwidth}
\begin{center}
Taller Nivelación 2014,  \\
Álgebra $8^{\circ}$
\end{center}
\end{minipage}\hfill
\begin{minipage}{.2\textwidth}
\includegraphics[height=1.75cm]{Images/logo-sed.png} 
\end{minipage}}
\date{}
\begin{document}
\maketitle
Nombre: \hrulefill Curso: \underline{\hspace*{44pt}} Fecha: \underline{\hspace*{2.5cm}}
\section*{N\'{u}meros reales}
\begin{enumerate}
\item De la lista 0, $\sqrt{2}$, $\frac{3}{4}$, $-\frac{5}{6}$, $\frac{25}{3}$, $-\sqrt{3}$, $-8$, $0.34$, $0.2\overline{3}$, 67 y $\frac{9}{7}$, identifique entre \'{e}stos:
\begin{enumerate}\begin{multicols}{2}
\item Los n\'{u}meros naturales
\item Los enteros
\item Los enteros no negativos
\item Los racionales
\item Los irracionales
\end{multicols}
\end{enumerate}
Para los problemas \ref{first}--\ref{second}, determine la propiedad de la igualdad o de los números reales que justifica cada proposición. Por ejemplo $6(-7)=-7(6)$ es cierta por la propiedad conmutativa de la multiplicación; y si $2=x+3$, entonces $x+3=2$ por la propiedad simétrica de la igualdad.
\item $7+[3+(-8)]=(7+3)+(-8)$\label{first}
\item Si $x=2$ y $x+y=9$, entonces $2+y=9$
\item $-1(x+2)=-(x+2)$
\item $3(x+4)=3(x)+3(4)$
\item $[(17)(4)](25)=(17)[(4)(25)]$
\item $x+3=3+x$
\item $3(98)+3(2)=3(98+2)$
\item $\left(\frac{3}{4}\right)\left(\frac{4}{3}\right)=1$
\item Si $4=3x-1$, entonces $3x-1=4$\label{second}

Para los ejercicios \ref{third}--\ref{fourth}, simplifique cada expresión numérica
\begin{multicols}{2}
\item $-8\frac{1}{4}+\left(-4\frac{5}{8}\right)-\left(-6\frac{3}{8}\right)$\label{third}
\item $9\frac{1}{3}-12\frac{1}{2}+\left(-4\frac{1}{6}\right)-\left(-1\frac{1}{6}\right)$
\item $-8(2)-16\div (-4)+(-2)(-2)$
\item $4(-3)-12\div (-4)+(-2)(-1)-8$
\item $[48+(-73)]+74$
\item $3-[-2(3-4)]+7$
\item $(-2)^{4}+(-1)^{3}-3^{2}$
\item $[4(-1)-2(3)]^{2}$\label{fourth}
\end{multicols}
Para los ejercicios \ref{fith}--\ref{sixth}, Simplifique cada expresión algebraica reduciendo términos semejantes
\begin{multicols}{2}
\item $3a^{2}-2b^{2}-7a^{2}-3b^{2}$\label{fith}
\item $4x-6-2x-8+x+12$
\item $-\frac{2}{3}x^{2}y-\left(-\frac{3}{4}x^{2}y\right)-\frac{5}{12}x^{2}y-2x^{2}y$
\item $-2(3a-1)+4(2a+3)-5(3a+2)$
\item $3(2x-3y)-4(3x+5y)-x$
\item $-5(x^{2}-4)-2(3x^{2}+6)+(2x^{2}-1)$\label{sixth}
\end{multicols}
Para los ejercicios \ref{seventh}--\ref{octave}, evalúe cada expresión algebraica para los valores dados de las variables
\begin{multicols}{2}
\item $-5x+4y$, para $x=\frac{1}{2}$ y $y=-1$\label{seventh}
\item $3x^{2}-2y^{2}$, para $x=\frac{1}{4}$ y $y=-\frac{1}{2}$
\item $(3a-2b)^{2}$, para $a=-2$ y $b=3$
\item $3n^{2}-4-4n^{2}+9$, para $n=7$\end{multicols}
\item $-4(3x-1)-5(2x-1)$, para $x=-23$
\item $5(3n-1)-7(-2n+1)+4(3n-1)$, para $n=\frac{1}{2}$\label{octave}

Para los problemas \ref{first0}--\ref{last0}, transcriba cada frase a lenguaje algebraico y use $n$ para representar el número desconocido
\item 4 aumentado en 2 veces un número\label{first0}
\item Seis menos que $\frac{2}{3}$ de un número
\item 10 veces la diferencia de un número y 14
\item El cociente de un número y tres menos que este número.
\item Tres cuartos de la suma de un número y 12\label{last0}

Para los problemas \ref{first1}--\ref{last1}, responda la pregunta con una expresión algebraica
\item Yuriko puede teclear $w$ palabras en una hora. ¿Cuál es la rapidez de ella por minuto?
\item Si $n$ representa un múltiplo de 3, ¿c\'{o}mo se representa el siguiente m\'{u}ltiplo de 3?
\item El per\'{i}metro de un cuadrado es $i$ pulgadas. ¿Cu\'{a}l es la longitud de cada lado en pies? (Recuerde que un pie son 12 pulgadas)
Para los problemas \ref{first2}--\ref{last2}, solucione cada ecuación
\begin{multicols}{2}
\item $2(2x+1)-(x-4)=4(x+5)$\label{first2}
\item $2(3n-4)+3(2n-3)=-2(n+5)$
\item $\dfrac{x+6}{5}+\dfrac{x-1}{4}=2$
\item $\dfrac{2x+1}{3}+\dfrac{3x-1}{5}=\dfrac{1}{10}$
\item $\dfrac{5x+6}{2}-\dfrac{x-4}{3}=\dfrac{5}{6}$
\item $0.4(t-6)=0.3(2t+5)$
\item $0.2(x-0.5)-0.3(x+1)=0.4$\label{last2}
\end{multicols}
Solucione los problemas \ref{first3}--\ref{last3} planteando y solucionando una ecuación apropiada
\item Encuentre tres enteros consecutivos tal que la suma de la mitad del menor y un tercio del mayor es uno menos que el entero del medio.\label{first3}
\item Si el complemento de un ángulo es una d\'{e}cima parte del suplemento del \'{a}ngulo, encuentre la medida del \'{a}ngulo\label{last3}
Para los problemas \ref{first4}--\ref{last4}, encuentre el grado el polinomio
\begin{multicols}{2}
\item $-2x^{3}+4x^{2}-8x+10$\label{first4}
\item $x^{4}+11x^{2}-15$
\item $5x^{3}y+4x^{4}y^{2}-3x^{3}y^{2}$
\item $5xy^{3}+2x^{2}y^{2}-3x^{3}y^{2}$\label{last4}
\end{multicols}
Para los problemas \ref{first5}--\ref{last5}, efectúe las operaciones indicadas y simplifique
\begin{multicols}{2}
\item $(3x-2)+(4x-6)+(-2x+5)$\label{first5}
\item $(8x^{2}+9x-3)-(5x^{2}-3x-1)$
\end{multicols}
\item $(-3x^{2}-4x+8)+(5x^{2}+7x+2)-(-9x^{2}+x+6)$
\item $[8x-(5x-y+3)]-[-4y-(2x+1)]$
\begin{multicols}{2}
\item $(-2a^{2})(3ab^{2})(a^{2}b^{3})$
\item $\left(\frac{3}{4}x^{2}y^{3}\right)(12x^{3}y^{2})(3y^{3})$
\item $(-2x^{2}y^{3}z)^{3}$
\item $(3x^{n+1})(2x^{3n-1}$
\item $\dfrac{30x^{5}y^{4}}{15x^{2}y}$
\item $\dfrac{20a^{4}b^{6}}{5ab^{3}}$
\item $-2x^{3}(4x^{2}-3x-5)$
\item $(3x+2)(2x^{2}-5x+1)$
\item $(3x^{2}-x-4)(x^{2}+2x-5)$
\item $(7x-9)(x+4)$
\item $x^{2}-3)(x^{2}+8)$
\item $(2x-3)^{2}$
\item $(4x+3y)^{2}$
\item $(2x+5y)^{2}$
\item $(3x-1)(3x+1)$
\item $(2x+5)^{3}$\label{last5}
\end{multicols}
\begin{minipage}{.45\textwidth}
\item Encuentre un polinomio que represente el área de la región sombreada
\end{minipage}\hfill
\begin{minipage}{.45\textwidth}
\begin{tikzpicture}[scale=1.5]
\filldraw[fill=gray!25,even odd rule]
(-1,-1) rectangle (2,0)
(0.66,-0.33) rectangle (1.66,-0.66);
\node[below] at (1,-1){$3x+4$};
\node[right] at (2,-0.5){$x$};
\node[left] at (0.66,-0.5){$x-2$};
\node[above] at (1,0){$x-1$}; 
\end{tikzpicture}
\end{minipage}

\begin{minipage}{.45\textwidth}
\begin{tikzpicture}[scale=1.5]
\draw (0,0,0)--(2,0,0)--(2,1,0)--(0,1,0)--cycle;
\draw (0,0,1)--(2,0,1)--(2,1,1)--(0,1,1)--cycle;
\draw (0,0,0) -- (0,0,1);
\draw (2,0,0) -- (2,0,1);
\draw (2,1,0) -- (2,1,1);
\draw (0,1,0) -- (0,1,1);
\node[below] at (1,0,1){$2x$};
\node[left] at (0,.4,1){$x$};
\node[left] at (0,1,.5){$x+1$};
\end{tikzpicture}
\end{minipage}\hfill
\begin{minipage}{.45\textwidth}
\item Encuentre el polinomio que represente el volumen del sólido rectangular de la figura.
\end{minipage}
Para los problemas \ref{first6}--\ref{last6}, factorice cada polinomio
\begin{multicols}{2}
\item $10a^{2}b-5ab^{3}-15a^{3}b^{2}$\label{first6}
\item $3xy-5x^{2}y^{2}-15x^{3}y^{3}$
\item $a(x+4)+b(x+4)$
\item $y(3x-1)+7(3x-1)$
\item $6x^{3}+3x^{2}y+2xz^{2}+yz^{2}$
\item $mn+5n^{2}-4m-20n$
\item $49a^{2}-25b^{2}$
\item $36x^{2}-y^{2}$
\item $27x^{3}+64y^{3}$
\item $125a^{3}-8$\label{last6}\end{multicols}
Para los problemas \ref{first7}--\ref{last7}, factorice completamente el polinomio.
\begin{multicols}{2}
\item $x^{6}-x^{2}$\label{first7}
\item $6a^{3}b+4a^{2}b^{2}-2a^{2}bc$
\item $3w^{3}+18w^{2}-24w$
\item $16a^{2}-64a$
\item $2t^{2}-18$
\item $x^{2}-(y-1)^{2}$
\item $4n^{2}-8n$
\item $3x^{3}-15x^{2}-18x$\label{last7}
\end{multicols}
Para los problemas \ref{first8}--\ref{last8}, solucione la ecuación
\begin{multicols}{2}
\item $4x^{2}-36=0$
\item $(3x-4)^{2}-25=0$
\item $6a^{3}=54a$
\item $x^{5}=x$\label{last8}
\end{multicols}
\item El perímetro de un rectángulo es 32 m. y su área es 48 m$^{2}$. Encuentre el largo y ancho del rectángulo
\end{enumerate}
\fbox{\begin{minipage}{.99\textwidth}
\emph{Este trabajo debe ser resuelto y entregado en hoja examen para poder presentar la evaluación de nivelación. Su propósito es repasar las temáticas esenciales vistas en el año escolar.}
\end{minipage}}
\end{document}
