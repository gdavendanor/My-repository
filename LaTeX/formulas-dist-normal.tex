\documentclass[11pt,letterpaper]{article}
\usepackage[utf8]{inputenc}
\usepackage[spanish]{babel}
\usepackage{amsmath}
\usepackage{amsfonts}
\usepackage{amssymb}
\usepackage{graphicx}
\usepackage{lmodern}
\author{Laura Camila Avendaño C.}
\begin{document}
Función de densidad para la distribución normal está definida por:
\[f(x)=\dfrac{1}{\sigma \sqrt{2\pi}}e^{-\frac{1}{2}(\frac{x-\mu}{\sigma})^{2}}\], donde $\mu$ es la media que coincide con la moda y la mediana y $\sigma$ la desviación standard 

Su correspondiente función de distribución en el intervalo de $x=a$ a $x=b$ es:
\[P(a\leq x \leq b)=F(x)=\int_{a}^{b}f(x)dx\] donde $f(x)$ se define en la anterior ecuación.

Haciendo $z=\dfrac{x-\mu}{\sigma}$ se estandariza a una distribución normal, con $\mu=0$ y $\sigma=1$. Por tanto se obtiene que su función de densidad es:
\[f(z)=\dfrac{1}{\sqrt{2\pi}}e^{-\frac{z}{2}^{2}}\]
y su correspondiente función de distribución entre $0$ y $z$ es:
\[F(z)=F(0\leq z)=\dfrac{1}{\sqrt{2\pi}}\int_{0}^{z}e^{-\frac{u^{2}}{2}}du\]

\end{document}