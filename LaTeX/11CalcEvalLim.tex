\documentclass[letterpaper,fleqn]{article}
\usepackage[spanish,es-noshorthands]{babel}
\usepackage[utf8]{inputenc} 
\usepackage[papersize={6.5in,8.5in},left=1cm, right=1cm, top=1.5cm, bottom=1.7cm]{geometry}
\usepackage{mathexam}
\usepackage{amsmath}
\usepackage{graphicx}

\ExamClass{\includegraphics[height=16pt]{Images/logo-sed.png} Cálculo $11^{\circ}$}
\ExamName{Límites}
\ExamHead{\includegraphics[height=16pt]{Images/logo-colegio.png} IEDAB}
\newcommand{\LineaNombre}{%
\par
\vspace{\baselineskip}
Nombre:\hrulefill \; Curso: \underline{\hspace*{48pt}} \; Fecha: \underline{\hspace*{2.5cm}} \relax
\par}
\let\ds\displaystyle

\begin{document}
\ExamInstrBox{
Para resolver estos ejercicios debe tener en cuenta las propiedades de los límites; además debe tener presente que si al resolver directamente se obtiene indeterminación, ésta debe solucionarse mediante factorización.}
\LineaNombre
\vspace*{12pt}
\emph{Para recordar:}
\section*{Casos de factorizaci\'on}
\begin{itemize}
\item Diferencia de cuadrados: $a^{2}-b^{2}=(a-b)(a+b)$
\item Trinomios
\begin{itemize}
\item[i)] $x^{2}+3x-10$.\\

En este caso debemos buscar dos números que multiplicados den el tercer término $-10$ y sumados den el coeficiente del segundo término $3$, los cuales son 5 y $-2$. De tal forma que la factorización es: \[x^{2}+3x-10=(x-2)(x+5)\]
\item[ii)] $6x^{2}+7x-20$\\Se puede resolver este caso de forma similar al anterior, multiplicando y dividiendo por el coeficiente del primer término $6$. Así:
\begin{align*}
6x^{2}+7x-20&=\dfrac{6(6x^{2})+7x(6)-20(6)}{6}\\
&=\dfrac{36x^{2}+7(6x)-120}{6} & \mbox{Buscamos dos n\'umeros que sumados den 7 y multiplicados -120}\\
&=\dfrac{(6x+15)(6x-8)}{6} & \mbox{ los cuales son 15 y $-8$}\\
&=\dfrac{3(2x+5)2(3x-4)}{6} & \mbox{Cancelamos los factores 3 y 2 con el 6 del denominador}\\
&=(2x+5)(3x-4)
\end{align*}
\end{itemize}
\end{itemize}
\begin{enumerate}
 \item 
 \end{enumerate}

\end{document}
