\documentclass[letterpaper,fleqn]{article}
\usepackage[spanish,es-noshorthands]{babel}
\usepackage[utf8]{inputenc} 
\usepackage[papersize={6.5in,8.5in},left=1cm, right=1cm, top=1.5cm, bottom=1.7cm]{geometry}
\usepackage{mathexam}
\usepackage{amsmath}
\usepackage{graphicx}
\usepackage{tikz,pgf}

\ExamClass{\includegraphics[height=16pt]{Images/logo-sed.png} Cálculo $11^{\circ}$}
\ExamName{Límites}
\ExamHead{\includegraphics[height=16pt]{Images/logo-colegio.png} IEDAB}
\newcommand{\LineaNombre}{%
\par
\vspace{\baselineskip}
Nombre:\hrulefill \; Curso: \underline{\hspace*{48pt}} \; Fecha: \underline{\hspace*{2.5cm}} \relax
\par}
\let\ds\displaystyle

\begin{document}
\ExamInstrBox{
Para resolver estos ejercicios debe tener en cuenta las propiedades de los límites; además debe tener presente que si al resolver directamente se obtiene indeterminación, ésta debe solucionarse mediante factorización.}
\LineaNombre
\vspace*{12pt}
\emph{Para recordar:}
\section*{Casos de factorizaci\'on}
\begin{itemize}
\item Diferencia de cuadrados: $a^{2}-b^{2}=(a-b)(a+b)$
\item Trinomios
\begin{itemize}
\item[i)] $x^{2}+3x-10$.\\
En este caso debemos buscar dos números que multiplicados den el tercer término $-10$ y sumados den el coeficiente del segundo término $3$, los cuales son 5 y $-2$. De tal forma que la factorización es: \[x^{2}+3x-10=(x-2)(x+5)\]
\item[ii)] $6x^{2}+7x-20$\\Se puede resolver este caso de forma similar al anterior, multiplicando y dividiendo por el coeficiente del primer término $6$. Así:
\begin{align*}
6x^{2}+7x-20&=\dfrac{6(6x^{2})+7x(6)-20(6)}{6}\\
&=\dfrac{36x^{2}+7(6x)-120}{6} & \mbox{Buscamos dos n\'umeros que sumados}\\
&=\dfrac{(6x+15)(6x-8)}{6} & \mbox{den 7 y multiplicados }-120\\
&=\dfrac{3(2x+5)2(3x-4)}{6} & \mbox{Cancelamos los factores 3 y 2 con}\\
&=(2x+5)(3x-4) & \mbox{el 6 del denominador}
\end{align*}
\end{itemize}
\end{itemize}
\begin{enumerate}
 \item Sabiendo que
 \newpage
\begin{center}
$\ds{\lim_{x\rightarrow a}f(x)=7}$,\quad  $\ds{\lim_{x\rightarrow a}g(x)=8}$ \quad y \quad $\ds{\lim_{x\rightarrow a}h(x)=0}$ 
\end{center}
y teniendo en cuenta el álgebra de límites, resuelva si existen o no existen, justificar:
      \begin{enumerate}
	 \item $\ds{\lim_{x\rightarrow a}[f(x)+g(x)]}=$
	 \item $\ds{\lim_{x\rightarrow a}[h(x)-g(x)]}=$
	 \item $\ds{\lim_{x\rightarrow a}\dfrac{h(x))}{g(x)}}=$
	 \item $\ds{\lim_{x\rightarrow a}\dfrac{f(x)}{h(x)}}=$
	 \item $\ds{\lim_{x\rightarrow a}[f(x)\cdot g(x)]}=$
      \end{enumerate}
   \item Con base en las siguientes gráficas:
   
   \begin{minipage}{.45\textwidth}
\usetikzlibrary{arrows}
\baselineskip=10pt
\hsize=6.3truein
\vsize=8.7truein
\definecolor{cqcqcq}{rgb}{0.75,0.75,0.75}
\tikzpicture[line cap=round,line join=round,>=triangle 45,x=1.0cm,y=1.0cm]
\draw [color=cqcqcq,dash pattern=on 2pt off 2pt, xstep=1.0cm,ystep=1.0cm] (-2.71,-3.31) grid (3.22,4.35);
\draw[->,color=black] (-2.71,0) -- (3.22,0);
\foreach \x in {-2,-1,1,2,3}
\draw[shift={(\x,0)},color=black] (0pt,2pt) -- (0pt,-2pt) node[below] {$\x$};
\draw[->,color=black] (0,-3.31) -- (0,4.35);
\foreach \y in {-3,-2,-1,1,2,3,4}
\draw[shift={(0,\y)},color=black] (2pt,0pt) -- (-2pt,0pt) node[left] {$\y$};
\draw[color=black] (0pt,-10pt) node[right] {$0$};
\clip(-2.71,-3.31) rectangle (3.22,4.35);
\draw[smooth,samples=100,domain=-2.7052791361432518:3.2198952089335324] plot(\x,{(\x)*((\x)-1)*((\x)+2)});
\draw[color=black] (-2.14,-2.7) node {$f$};
\endtikzpicture
   \end{minipage}
   \begin{minipage}{.45\textwidth}
\usetikzlibrary{arrows}
\baselineskip=10pt
\hsize=6.3truein
\vsize=8.7truein
\definecolor{qqqqff}{rgb}{0.33,0.33,0.33}
\definecolor{cqcqcq}{rgb}{0.75,0.75,0.75}
\tikzpicture[line cap=round,line join=round,x=1.0cm,y=1.0cm]
\draw [color=cqcqcq,dash pattern=on 1pt off 1pt, xstep=1.0cm,ystep=1.0cm] (-2.74,-1.68) grid (2.44,4.74);
\draw[->,color=black] (-2.74,0) -- (2.44,0);
\foreach \x in {-2,-1,1,2}
\draw[shift={(\x,0)},color=black] (0pt,2pt) -- (0pt,-2pt) node[below] {$\x$};
\draw[->,color=black] (0,-1.68) -- (0,4.74);
\foreach \y in {-1,1,2,3,4}
\draw[shift={(0,\y)},color=black] (2pt,0pt) -- (-2pt,0pt) node[left] {$\y$};
\draw[color=black] (0pt,-10pt) node[right] {$0$};
\clip(-2.74,-1.68) rectangle (2.44,4.74);
\draw[smooth,samples=100,domain=-2.739828257688889:-1.0000035948936852] plot(\x,{((\x)*≤-1)*(-2*(\x)-1)});
\draw[smooth,samples=100,domain=-0.9999984125254533:2.4425399741566958] plot(\x,{((\x)*>-1)*((\x)^2-1)});
\draw (0.7,0.63) node[anchor=north west] {g};
\fill [color=qqqqff] (-1,0) circle (1.5pt);
\draw [color=qqqqff] (-1,1) circle (1.5pt);
\endtikzpicture
   \end{minipage}
   \item Evalúe los siguientes límites:
   \begin{enumerate}
   \item $\ds{\lim_{x\rightarrow 3}x^{2}-4x+6}=$\noanswer
   \item $\ds{\lim_{x\rightarrow 7}\dfrac{x^{2}-49}{x-7}}=$\noanswer
   \item $\ds{\lim_{x\rightarrow 5}\dfrac{x^{2}+3x-40}{x-5}}=$\noanswer
   \item $\ds{\lim_{x\rightarrow 3}\dfrac{2x^{2}-x-15}{x-3}}=$\noanswer
   \end{enumerate}
 \end{enumerate}

\end{document}
