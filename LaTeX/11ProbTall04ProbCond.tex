\documentclass[10pt,twoside]{article}
\usepackage[utf8]{inputenc}
\usepackage{amsmath}
\usepackage{amsfonts}
\usepackage{amssymb}
\usepackage[spanish,es-noshorthands]{babel}
\usepackage[T1]{fontenc}
\usepackage{lmodern}
\usepackage{graphicx,hyperref}
\usepackage{tikz,pgf}
\usepackage{multicol}
\usepackage{subfig}
\usepackage[papersize={6.5in,8.5in},width=5.5in,height=7in]{geometry}
\usepackage{fancyhdr}
\pagestyle{fancy}
\fancyhead[LE]{\includegraphics[height=12pt]{Images/logo-colegio.png} Probabilidad $11^{\circ}$}
\fancyhead[RE]{}
\fancyhead[RO]{\textit{Germ\'an Avenda\~no Ram\'irez, Lic. U.D., M.Sc. U.N.}}
\fancyhead[LO]{}

\author{Germ\'an Avenda\~no Ram\'irez, Lic. U.D., M.Sc. U.N.}
\title{\begin{minipage}{.2\textwidth}
\includegraphics[height=1.75cm]{Images/logo-colegio.png}\end{minipage}
\begin{minipage}{.55\textwidth}
\begin{center}
Taller, Probabilidad condicional  \\
Probabilidad $11^{\circ}$
\end{center}
\end{minipage}\hfill
\begin{minipage}{.2\textwidth}
\includegraphics[height=1.75cm]{Images/logo-sed.png} 
\end{minipage}}
\date{}
\begin{document}
\maketitle
Nombre: \hrulefill Curso: \underline{\hspace*{44pt}} Fecha: \underline{\hspace*{2.5cm}}

\section*{Ejercicios}
\begin{enumerate}
\item A trescientos televidentes se les preguntó si estaban satisfechos con la cobertura de un reciente desastre por TV.
\begin{center}
\begin{tabular}{lcc}
 & Femenino & Masculino \\ 
\hline 
Satisfecho & 80 & 55 \\ 
\hline 
No satisfecho & 120 & 45 \\ 
\hline 
\end{tabular} 
\end{center}
Un televidente se ha de seleccionar al azar de entre los encuestados
\begin{enumerate}
\item Encuentre P(satisfecho)
\item Encuentre P(satisfecho|femenino)
\item Encuentre P(satisfecho|masculino)
\end{enumerate}
\item Los sábados por la mañana son horas de gran
movimiento en el centro acuático Webster. Las lecciones de natación que van del nivel 2 de Cruz Roja, Habilidad Acuática Fundamental, al nivel 6 de Cruz Roja, Suficiencia en Natación y Aptitud, se ofrecen durante dos sesiones.
\begin{center}
\begin{tabular}{|c|c|c|}
 & Número de personas & Número de personas \\
Nivel  & en clase de 10 a.m. & en clase de 11 a.m. \\  \hline
2 & 16 & 16 \\ 
3 & 15 & 11 \\ 
4 & 9 & 7 \\ 
5 & 8 & 3 \\ 
6 & 0 & 3 \\ 
\hline 
\end{tabular} 
\end{center}
\end{enumerate}
\end{document}
