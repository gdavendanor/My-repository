\documentclass[10pt,twoside]{article}
\usepackage[utf8]{inputenc}
\usepackage{amsmath}
\usepackage{amsfonts}
\usepackage{amssymb}
\usepackage[spanish,es-noshorthands]{babel}
\usepackage[T1]{fontenc}
\usepackage{lmodern}
\usepackage{graphicx,hyperref}
\usepackage{tikz,pgf}
\usepackage{multicol}
\usepackage{subfig}
\usepackage[papersize={6.5in,8.5in},width=5.5in,height=7in]{geometry}
\usepackage{fancyhdr}
\pagestyle{fancy}
\fancyhead[LE]{\includegraphics[height=12pt]{Images/logo-colegio.png} Probabilidad $11^{\circ}$}
\fancyhead[RE]{}
\fancyhead[RO]{\textit{Germ\'an Avenda\~no Ram\'irez, Lic. U.D., M.Sc. U.N.}}
\fancyhead[LO]{}

\author{Germ\'an Avenda\~no Ram\'irez, Lic. U.D., M.Sc. U.N.}
\title{\begin{minipage}{.2\textwidth}
\includegraphics[height=1.75cm]{Images/logo-colegio.png}\end{minipage}
\begin{minipage}{.55\textwidth}
\begin{center}
Taller, Probabilidad condicional  \\
Probabilidad $11^{\circ}$
\end{center}
\end{minipage}\hfill
\begin{minipage}{.2\textwidth}
\includegraphics[height=1.75cm]{Images/logo-sed.png} 
\end{minipage}}
\date{}
\begin{document}
\maketitle
Nombre: \hrulefill Curso: \underline{\hspace*{44pt}} Fecha: \underline{\hspace*{2.5cm}}

\section*{Ejercicios}
\begin{enumerate}
\item A trescientos televidentes se les preguntó si estaban satisfechos con la cobertura de un reciente desastre por TV.
\begin{center}
\begin{tabular}{lcc}
 & Femenino & Masculino \\ 
\hline 
Satisfecho & 80 & 55 \\ 
\hline 
No satisfecho & 120 & 45 \\ 
\hline 
\end{tabular} 
\end{center}
Un televidente se ha de seleccionar al azar de entre los encuestados
\begin{enumerate}
\item Encuentre P(satisfecho)
\item Encuentre P(satisfecho|femenino)
\item Encuentre P(satisfecho|masculino)
\end{enumerate}
\item Los sábados por la mañana son horas de gran
movimiento en el centro acuático Webster. Las lecciones de natación que van del nivel 2 de Cruz Roja, Habilidad Acuática Fundamental, al nivel 6 de Cruz Roja, Suficiencia en Natación y Aptitud, se ofrecen durante dos sesiones.
\begin{center}
\begin{tabular}{ccc}
 & Número de personas & Número de personas \\
Nivel  & en clase de 10 a.m. & en clase de 11 a.m. \\  \hline
2 & 16 & 16 \\ 
3 & 15 & 11 \\ 
4 & 9 & 7 \\ 
5 & 8 & 3 \\ 
6 & 0 & 3 \\ 
\hline 
\end{tabular} 
\end{center}
Lauren, la coordinadora del programa, va a seleccionar al azar un nadador para entrevistarlo para un “spot” de la televisora local en el centro y su programa de natación. ¿Cuál es la probabilidad de que el nadador seleccionado tenga lo siguiente:
\begin{enumerate}
\item Una clase de nivel 4
\item La clase de 10 a.m.
\item Una clase de nivel 3 dada en la sesión de las 10 a.m.
\item La clase de 11 a.m. dada en la clase de nivel 5
\end{enumerate}
\item \textit{The world Factbook}, 2004, informa que los aeropuertos de Estados Unidos tienen el siguiente número de metros de pistas que son pavimentadas o no pavimentadas.
\begin{center}
\begin{tabular}{lcc}
 & \multicolumn{2}{c}{Número de aeropuertos} \\ 
Total pista (m) & Pavimentado & No pavimentado \\\hline 
Más de 3047 & 188 & 1 \\ 
2438-3047 & 221 & 7 \\ 
1524-2437 & 1375 & 160 \\ 
914-1523 & 2383 & 1718 \\ 
Menos de 914 & 961 & 7843 \\ \hline
Total & 5128 & 9729 \footnote{Fuente: The World Factbook, January 2004, \url{http://www.cia.gov/cia/
publications/factbook/geos/us.html\#People}
}\\ 
\hline 
\end{tabular} 
\end{center}
Si uno de estos aeropuertos se selecciona al azar para inspección, ¿cuál es la probabilidad de que tendrá lo siguiente:
\begin{enumerate}
\item Pistas pavimentadas
\item 914 a 2437 metros de pista
\item Menos de 1524 metros de pista y no pavimentada
\item Más de 2437 metros de pista y pavimentada
\item Pista pavimentada, dado que tiene más de 1523 metros de pista
\item No pavimentada, sabiendo que tiene menos de 1524 metros de pista
\item Menos de 1524 metros de pista, dado que no está pavimentada
\end{enumerate}
\end{enumerate}
\end{document}
