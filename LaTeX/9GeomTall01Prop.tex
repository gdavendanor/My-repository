\documentclass[10pt,twoside]{article}
\usepackage[utf8]{inputenc}
\usepackage{amsmath}
\usepackage{amsfonts}
\usepackage{amssymb}
\usepackage[spanish,es-noshorthands]{babel}
\usepackage[T1]{fontenc}
\usepackage{lmodern}
\usepackage{graphicx,hyperref}
\usepackage{tikz,pgf}
\usepackage{multicol}
\usepackage{subfig}
\usepackage[papersize={6.5in,8.5in},width=5.5in,height=7in]{geometry}
\usepackage{fancyhdr}
\pagestyle{fancy}
\fancyhead[LE]{\includegraphics[height=12pt]{Images/logo-colegio.png} Geometría $9^{\circ}$}
\fancyhead[RE]{}
\fancyhead[RO]{matematicas.german@gmail.com}
\fancyhead[LO]{}

\author{Germ\'an Avenda\~no Ram\'irez,~\thanks{Lic. Mat. U.D., M.Sc. U.N.}}
\title{\begin{minipage}{.2\textwidth}
\includegraphics[height=1.75cm]{Images/logo-colegio.png}\end{minipage}
\begin{minipage}{.55\textwidth}
\begin{center}
Taller 01\\
Proporcionalidad\\
Geometr\'{i}a $9^{\circ}$
\end{center}
\end{minipage}\hfill
\begin{minipage}{.2\textwidth}
\includegraphics[height=1.75cm]{Images/logo-sed.png} 
\end{minipage}}
\date{}
\begin{document}
\maketitle
Nombre: \hrulefill Curso: \underline{\hspace*{44pt}} Fecha: \underline{\hspace*{2.5cm}}
\paragraph*{Meta de aprendizaje}: El estudiante aplica las propiedades de las proporciones para hallar incógnitas
\subsection*{\underline{Entre uno y otro}}
\begin{enumerate}
\item Empleando una expresi\'{o}n matem\'{a}tica escribir simb\'{o}licamente las siguientes expresiones
\begin{enumerate}
\item Tengo 4 tortas para tres personas
\item Se fue ocho d\'{i}as y solo trabaj\'{o} cuatro.
\item Camilo da tres pasos en tres segundos
\item Realice la notaci\'{o}n de las estrategias empleadas a partir de la Ruleta Heur\'{i}stica.
\end{enumerate}
\end{enumerate}
\section*{Raz\'{o}n matem\'{a}tica}
La razón matemática es una expresión que se encarga de relacionar dos cantidades sin tener en cuenta el tipo. Se representa mediante un cociente indicado, $\dfrac{a}{b}$ \'{o} $a:b$. La lectura es $a$ es a $b$, también se puede emplear $a$ de $b$.
\subsection*{Partes de la raz\'{o}n}
 \[\dfrac{a}{b}=\dfrac{Antecedente}{Consecuente}\]
 La razón suele expresarse en fracción reducida (simplificada), sus partes son el antecedente $a$ y  el consecuente $b$.
 \subsubsection*{Algunos ejemplos}
 \begin{enumerate}
 \item 5 es a 7, es la razón entre días de asistencia al colegio y días de la semana.
 \item 1 es 3, razón entre cada uno de los colores de la bandera de Colombia y el total de colores de la bandera. $\dfrac{1}{3}$ o $1:3$
 \end{enumerate}
 \subsection*{\underline{Escribiendo razones}}
 \begin{enumerate}
 \item Escriba las siguientes razones, empleando las cuatro formas de expresar una razón, y en el caso que sea posible hallar la fracción reducida.
 \begin{enumerate}
 \item 15 galletas con trocitos de menta en una bolsa con 34 galletas.
 \item 16 perros pastor alemán de 24 perros.
 \item 25 conjuntos residenciales de ladrillo de cada 45.
 \item 10 tambores de 75 instrumentos.
 \item 32 vacas de 72 mamíferos.
 \item Realice la notación de las estrategias empleadas a partir de la Ruleta Heurística.
 \end{enumerate}
 \end{enumerate}
 \subsection*{\underline{De razones a tasas}}
 \begin{enumerate}
 \item Expresar en forma de razón las siguientes expresiones.
 \begin{enumerate}
 \item 8 kilómetros recorre en una  hora.
 \item 30 vueltas da un disco en un minuto.
 \item c) 45 kilogramos en una botella de 3 litros.
 \item 35 personas por metro cuadrado.
 \item Establecer las diferencias y semejanzas con las razones del numeral dos.
 \item 32 vacas de 72 mamíferos.
 \item Realice la notación de las estrategias empleadas a partir de la Ruleta Heurística.
 \end{enumerate}
  \end{enumerate}
\section*{Tasa}
Se dan casos cuando los términos de las razones, corresponden a dos medidas expresadas en diferentes unidades, denominándose \emph{tasa}.
\subsection*{Ejemplos de tasa}
\[\dfrac{125\text{ kil\'{o}metros}}{2\text{ horas}}\] Compara el n\'{u}mero de kil\'{o}metros recorridos con el n\'{u}mero de horas que dur\'{o} el viaje.
\subsection*{\underline{Expresando razones}}
\begin{enumerate}
\item Expresar como tasa cada razón. Hallar la fracción reducida en cada caso.
\begin{enumerate}
\item a) 120 palabras en 3 minutos.
\item 5 gaseosas en \$3.000
\item 395 kilómetros en 5 horas.
\item 36 millones de discos en 7 años.
\item 79.8 kilómetros con 3 galones de gasolina.
\item Realice la notación de las estrategias empleadas a partir de la Ruleta Heurística.
\end{enumerate}
\end{enumerate}
\subsection*{\underline{Aplicando}}
\begin{enumerate}
\item Dibujar en un papel cuadriculado un cuadrado de 1 por 1 unidad y otro de 2 por 2 unidades. Luego escribir la razón que compara cada  uno de los siguientes casos.
\begin{enumerate}
\item La longitud del lado del cuadrado más pequeño a la longitud del lado del cuadrado más grande.
\item El perímetro del cuadrado pequeño al perímetro del cuadrado grande.
\item El área del cuadrado pequeño al área del cuadrado grande.
\item Dibujar un cuadrado de 3 por 3. Comparar las mismas medidas entre el cuadrado de 1 por 1 unidad y el de 3 por 3 unidades.  Describir la relación que existe entre las razones de los lados, los perímetros y las áreas.
\end{enumerate}
\item Hacer un dibujo en el que la razón del número de círculos azules al número total de círculos sea $\frac{3}{5}$
\item El avestruz es el ave corredora más veloz, llegando a correr 384 kilómetros en 6 horas. ¿Cuál es la rapidez\footnote{En el idioma inglés se hace evidente la diferencia entre velocidad (velocity) y la rapidez (speed)} promedio de un avestruz?
\item Realice la notación de las estrategias empleadas a partir de la Ruleta Heurística.
\end{enumerate}
\section*{Proporciones}
La igualdad de razones forma una proporción
\[\dfrac{a}{b}=\dfrac{c}{d}, \text{\qquad a:b::c:d}\]
Se lee, $a$ es a $b$, como $c$ es a $d$. Los términos $a$ y $d$ reciben el nombre de \emph{extremos} y $b$ y $c$ se denominan \emph{medios}
\subsection*{Primera propiedad}
El Producto de extremos es igual al producto de medios.
\[\text{Si} \quad \dfrac{a}{b}=\dfrac{c}{d}\quad \text{ es una proporción, entonces }\qquad a\cdot d=b\cdot c\]
Se considera como la propiedad fundamental de las proporciones.
\subsubsection*{Ejemplo de aplicaci\'{o}n de la primera propiedad}
Verificar la propiedad fundamental para el siguiente par de razones.
\[\dfrac{5}{3}=?\dfrac{4}{7}\]
Observando la definición se deduce que 
\begin{align*}
5\cdot 7 &=? 3 \cdot 4\\
35&=?12
\end{align*}

\end{document}
