\documentclass[letterpaper,fleqn]{article}
\usepackage[spanish,es-noshorthands]{babel}
\usepackage[utf8]{inputenc} 
\usepackage[papersize={6.5in,8.5in},left=1cm, right=1cm, top=1.5cm, bottom=1.7cm]{geometry}
\usepackage{mathexam}
\usepackage{amsmath}
\usepackage{graphicx}
\usepackage{tikz,pgf}
\usepackage{multicol}

\ExamClass{\includegraphics[height=16pt]{Images/logo-sed.png} Matemáticas $11^{\circ}$}
\ExamName{Nivelación 2014}
\ExamHead{\includegraphics[height=16pt]{Images/logo-colegio.png} IEDAB}
\newcommand{\LineaNombre}{%
\par
\vspace{\baselineskip}
Nombre:\hrulefill \; Curso: \underline{\hspace*{48pt}} \; Fecha: \underline{\hspace*{2.5cm}} \relax
\par}
\let\ds\displaystyle

\begin{document}
\ExamInstrBox{
Respuesta sin justificar mediante procedimiento no será tenida en cuenta en la calificación. Escriba sus respuestas en el espacio indicado. Tiene 60 minutos para contestar esta prueba.}
\LineaNombre
\begin{enumerate}
 \item Encuentre el dominio y el rango de la función $f(x)=\sqrt{x-4}$. (Recuerde que el dominio es el conjunto de valores que toma la variable independiente $x$ y el rango es el conjunto de valores que toma la variable dependiente $y$). Haga la gráfica de la función en el espacio asignado
 
\begin{tikzpicture}
\draw[<->] (-3.5,0)--(6.5,0)node[right]{$x$};
\foreach \x in {-3,-2,-1,1,2,3,4,5,6} \draw[shift={(\x,0)}](0pt,2pt)--(0pt,-2pt)node[below] {\footnotesize $\x$};
\draw[<->](0,-1.5)--(0,3.5)node[right]{$y$};
\foreach \y in {-1,1,2,3} \draw[shift={(0,\y)}](2pt,0)--(-2pt,0)node[left]{\footnotesize $\y$};
\draw[dotted] (-3,-1)grid(6,3);
\end{tikzpicture} 
 \noanswer
 \newpage
 \item Para la función $g(x)=\left\{\begin{array}{lcl}
 3 & \mbox{ si } & x<-2\\
 x+4 & \mbox{ si } & x\geq -2
 \end{array}\right.$
 \end{enumerate}
 Encuentre:
\begin{enumerate}\begin{multicols}{2}
\item $g(0)=$
\item $g(1)=$
\item $g(-2)=$
\item $g(-1)=$
\item $\ds{\lim_{x\rightarrow 1}g(x)}=$
\item $\ds{\lim_{x\rightarrow -2^{-}}g(x)}=$
\item $\ds{\lim_{x\rightarrow 2^{+}}g(x)}=$
\item $\ds{\lim_{x\rightarrow 2}g(x)}=$
\end{enumerate}
\end{multicols}

\end{enumerate}

\end{document}
