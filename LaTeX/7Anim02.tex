\documentclass[10pt,twoside]{article}
\usepackage[utf8]{inputenc}
\usepackage{amsmath}
\usepackage{amsfonts}
\usepackage{amssymb}
\usepackage[spanish,es-noshorthands]{babel}
\usepackage[T1]{fontenc}
\usepackage{lmodern}
\usepackage{graphicx,hyperref}
\usepackage{tikz,pgf}
\usepackage{multicol}
\usepackage{subfig}
\usepackage[papersize={6.5in,8.5in},width=5.5in,height=7in]{geometry}
\usepackage{fancyhdr}
\pagestyle{fancy}
\fancyhead[LE]{\includegraphics[height=12pt]{Images/logo-colegio.png} Álgebra $8^{\circ}$}
\fancyhead[RE]{}
\fancyhead[RO]{\textit{Germ\'an Avenda\~no Ram\'irez, Lic. U.D., M.Sc. U.N.}}
\fancyhead[LO]{}

\author{Germ\'an Avenda\~no Ram\'irez, Lic. U.D., M.Sc. U.N.}
\title{\begin{minipage}{.2\textwidth}
\includegraphics[height=1.75cm]{Images/logo-colegio.png}\end{minipage}
\begin{minipage}{.55\textwidth}
\begin{center}
Taller, Animaplano 02 \\
Álgebra $8^{\circ}$
\end{center}
\end{minipage}\hfill
\begin{minipage}{.2\textwidth}
\includegraphics[height=1.75cm]{Images/logo-sed.png} 
\end{minipage}}
\date{}
\begin{document}
\maketitle
Nombre: \hrulefill Curso: \underline{\hspace*{44pt}} Fecha: \underline{\hspace*{2.5cm}}
\section*{Cuestionario}
\begin{enumerate}
 \item Reste a $10^{2}$, la suma de los 4 primeros números primos.
 \item En unidades, 1 centena, menos 2 docenas, menos 2 unidades
 \item ¿Cuánto le falta a 27, para que de 100?
 \item Al producto entre 20 y 4, reste el producto entre 9 y 2
 \item Multiplique la tercera parte de 63, por el número 2
 \item ¿Cuánto le sobra a 98, para que de 50?
 \item Multiplique por 2, el doble de 4.5
 \item Los años en 4 lustros
 \item La equivalencia en años de 5 décadas
 \item Halle $7\times \sqrt{49}$
 \item Tres veces veintitrés
 \item Halle $(3^{2})^{2}-3^{1}=$
 \item Si $23+m=100$, luego $m=$?
 \item ¿Cuánto le restamos a 110 para que de 22?
 \item En unidades cuadradas, al área del cuadrado de lado 9 unidades, sume $2^{3}$ unidades cuadradas.
 \item $\sqrt{10\,000}-10^{2}+(10\times10)=$
 \item Sume al triple del n\'umero 30, la ra\'iz cuadrada de 1
 \item Exprese como n\'umero decimal el n\'umero romano $LXXXII$
 \item Sume al triple del n\'umero 20, el triple del n\'umero 9
 \item En cent\'imetros, un metro, menos 3 dec\'imetros, m\'as 6 cent\'imetros
 \item En unidades, 3 docenas + 3 decenas
 \item Sume 7, al 50\% de 100
 \item A la cuarta parte de 200, sume dos elevado a la tres
 \item Si $m-27=20$, entonces $m=$?
 \item La suma de 15, con el producto entre 9 y 8
 \item Sume cuatro veces 22
\end{enumerate}

\end{document}
