\documentclass[letterpaper,fleqn]{article}
\usepackage[spanish,es-noshorthands]{babel}
\usepackage[utf8]{inputenc} 
\usepackage[total={7.5in,9.75in},centering]{geometry}
\usepackage{mathexam}
\usepackage{amsmath}
\usepackage{amsfonts}
\usepackage{amssymb}
\usepackage{multicol}

\usepackage{graphicx}

\ExamClass{\includegraphics[height=16pt]{Images/logo-sed.png} Matemáticas $6^{\circ}$}
\ExamName{Multiplicación y división en $\mathbb{N}$}
\ExamHead{\includegraphics[height=16pt]{Images/logo-colegio.png} IEDAB}
\newcommand{\LineaNombre}{%
\par
\vspace{\baselineskip}
Nombre:\hrulefill \; Curso: \underline{\hspace*{48pt}} \; Fecha: \underline{\hspace*{2.5cm}} \relax
\par}
\let\ds\displaystyle

\begin{document}
\ExamInstrBox{
Respuesta sin justificar mediante procedimiento no será tenida en cuenta en la calificación. Escriba sus procedimientos al respaldo o en una hoja anexa y sus respuestas en el espacio indicado. Tiene 45 minutos para contestar esta prueba.}
\LineaNombre
\begin{enumerate}
%  \item Verifique la igualdad y relacione las dos columnas poniendo la letra correspondiente dentro del par\'entesis seg\'un corresponda
%\begin{enumerate}\begin{multicols}{2}
%  \item $ 6\times5=5\times6 $
%  \item $ 5\times1=1\times5=5 $
%  \item $ (3\times5)\times4=3\times(5\times4) $
%  \item $ 3\times(5+4)=3\times5+3\times4 $
%  \item[(\;)] Propiedad distributiva
%  \item[(\;)] Propiedad asociativa
%  \item[(\;)] Propiedad conmutativa
%  \item[(\;)] Propiedad modulativa
%  \end{multicols} 
%\end{enumerate}
\item Conteste las siguientes preguntas
\begin{enumerate}
  \item ¿Qu\'e n\'umero multiplicado por 8 da 96? \underline{\hspace{2.5cm}}
  \item ¿Cu\'antas veces debo sumar 13 para que me d\'e 91?\underline{\hspace{2.5cm}}
  \item Johana compr\'o una docena de l\'apices en \$1\;800. ¿Cuánto costó cada lápiz? \underline{\hspace{2.5cm}}
  \item En una caja puedo meter 15 docenas de galletas. ¿Cuántas cajas necesito para meter 285 docenas? \underline{\hspace{2.5cm}}
\end{enumerate}
\item Completo los espacios de la tabla
\begin{center}
\begin{tabular}{c|c|c}
$ 8\times6= \underline{\qquad}$ & $ 9\times7=\underline{\qquad} $ & $ 7\times8=\underline{\qquad} $\\[5pt]\hline
$ 8\times\underline{\qquad} = 72 $ & $ 9\times\underline{\qquad}=54 $ & $ \underline{\qquad}\times7=42 $\\
\end{tabular}
\end{center}
%\item Analice si los siguientes enunciados son falsos o verdaderos. Justifique su respuesta, si es falso de un contraejemplo.
%\begin{enumerate}
%  \item La división de dos números naturales es siempre otro n\'umero natural (\qquad)\\[5pt]Just: \hrulefill
%  \item La divisi\'on de dos números naturales cumple la propiedad conmutativa (\qquad) \\[5pt]Just: \hrulefill
%  \item La división de números naturales cumple la propiedad asociativa (\qquad)\\[5pt]Just: \hrulefill
%  \item La división de números naturales cumple la propiedad modulativa (\qquad)\\[5pt]Just: \hrulefill
%  \end{enumerate}
  \item Realizo los siguientes ejercicios:
  \begin{enumerate}
    \item Selena tiene 18 docenas de manzanas para empacarlas en cajas donde sólo caben 15 manzanas, ¿cuántas cajas se necesita para empacar todas las manzanas?\noanswer
    \item Armando tenía una alcancía donde ahorraba solamente monedas de \$500. El día que abrió la alcancía contó 275 monedas. ¿Cuánto dinero tenía ahorrado?\noanswer
  \end{enumerate}
  \item A una bodega llegó el siguiente pedido:
  \begin{multicols}{2}
    17 docenas de corbatas a \$5500 cada corbata\\
  54 pares de medias a \$3550 cada par\\
  15 docenas de bufandas a \$3690 cada bufanda.\\
  35 docenas de chaquetas a \$21500 cada chaqueta.
  \end{multicols}
    Halle:\\
  \begin{enumerate}
    \item El total de corbatas \hrulefill
    \item Total de bufandas \hrulefill
    \item Total de chaquetas \hrulefill
    \item Valor total de la compra \hrulefill\\ 
    
    Si en la venta en bodega de cada artículo se gana lo siguiente:
    \begin{multicols}{2}
      Por cada corbata \$900\\
      Por cada par de medias \$250\\
      Por cada bufanda \$650\\
      Por cada chaqueta \$4500
    \end{multicols}
    \item ¿Cuál es el valor total de la ganancia?\noanswer
  \end{enumerate}
  \item Complete la siguientes tablas:\\
  \begin{minipage}{0.5\textwidth}
    \begin{tabular}{|c|c|c|}\hline
    MULTIPLICANDO & MULTIPLICADOR & PRODUCTO\\\hline
    25 & 33 &\\\hline
    13 & & 351\\\hline
    & 29 & 957\\\hline
  \end{tabular}
  \end{minipage}
  \hfill \begin{minipage}{0.4\textwidth}
     \begin{tabular}{|c|c|c|}\hline
  DIVIDENDO & DIVISOR & COCIENTE\\\hline
  221 & 13 & \\\hline
   & 7 & 29 \\\hline
   414 & & 23\\\hline
  \end{tabular}
  \end{minipage}
 
\end{enumerate}

\end{document}
