\documentclass[10pt,twoside]{article}
\usepackage[utf8]{inputenc}
\usepackage{amsmath,amsfonts,amssymb,amsthm,latexsym}
\usepackage[spanish,es-noshorthands]{babel}
\usepackage[T1]{fontenc}
\usepackage{lmodern}
\usepackage{graphicx,hyperref}
\usepackage{tikz}
\usepackage{multicol}
\usepackage{subfig}
\usepackage[papersize={8.5in,11in},width=7in,height=9.5in]{geometry}
\usepackage{fancyhdr}
\pagestyle{fancy}
\fancyhead[LE]{\url{http://germandario.byethost4.com}}
\fancyhead[RE]{}
\fancyhead[RO]{\textit{Germ\'an Dar\'io Avenda\~no Ram\'irez, Lic - M.Sc.}}
\fancyhead[LO]{}

\author{Germ\'an Avenda\~no Ram\'irez~\thanks{Lic. Mat. U.D., M.Sc. U.N.}}
\title{\begin{minipage}{.2\textwidth}
\includegraphics[height=1.75cm]{Images/logo-colegio.png}\end{minipage}
\begin{minipage}{.55\textwidth}
\begin{center}
Plan de mejoramiento III \\
Aritmética $6^{\circ}$
\end{center}
\end{minipage}\hfill
\begin{minipage}{.2\textwidth}
\includegraphics[height=1.75cm]{Images/logo-sed.png} 
\end{minipage}}
\date{}

\begin{document}
\maketitle
Nombre: \hrulefill Curso: \underline{\hspace{1cm}}  Fecha: \underline{\hspace{2cm}}\\
\section*{Representación gráfica}
Se puede representar el producto de dos números usando el concepto de área así:\\

\begin{minipage}{0.4\textwidth}
\begin{tikzpicture}
\draw (0,0) grid (3,2);
\node[below] at (1.5,0) {3};
\node[right] at (3,1) {2};
\end{tikzpicture}
\end{minipage}\hfill
\begin{minipage}{0.55\textwidth}
  $ 3\times2=6 $
\end{minipage}
Así mismo se puede representar el producto de tres números usando el concepto de volumen así:

\begin{minipage}{0.45\textwidth}
  \begin{tikzpicture}
\draw (0,0,2)--(3,0,2)--(3,2,2)--(0,2,2)--cycle;
\draw (0,2,2)--(0,2,0)--(3,2,0)--(3,2,2);
\draw (3,2,0)--(3,0,0)--(3,0,2);
\draw (1,0,2)--(1,2,2)--(1,2,0);
\draw (2,2,0)--(2,2,2)--(2,0,2);
\draw (0,1,2)--(3,1,2)--(3,1,0) [right] node{2};
\draw (0,2,1)--(3,2,1)--(3,0,1) [right] node{2};
\node [below] at (1.5,0,2) {3};
\end{tikzpicture}
\end{minipage}\hfill
\begin{minipage}{0.5\textwidth}
  $ 3\times2\times2=3\times2^2=3\times4=12 $
\end{minipage}
\section*{Actividad}
\begin{enumerate}
  \item Complete el dibujo para representar la siguiente operación. ¿Cuántos cubos forman cada caja?
  \begin{enumerate}
    \begin{minipage}{0.45\textwidth}
    \item $ 3\times(4\times2) $
    \end{minipage}\hfill
    \begin{minipage}{0.5\textwidth}
      \begin{tikzpicture}
      \draw (0,4,0)--(3,4,0)--(3,4,2)--(0,4,2)--cycle;
      \draw (0,4,2)--(0,0,2)--(3,0,2)--(3,4,2);
      \draw (3,4,0)--(3,0,0)--(3,0,2);
      \draw (0,1,2)--(1,1,2);
      \node[below] at (1.5,0,2) {3};
      \node [right] at (3,0,1) {2};
      \node [right] at (3,2,0) {4};
      \draw (1,4,2)--(1,0,2);
      \draw (2,4,2)--(2,0,2);
      \draw (0,4,1)--(3,4,1);
      \end{tikzpicture}
    \end{minipage}
    
    \begin{minipage}{0.55\textwidth}
      \begin{tikzpicture}
      \draw (0,3,0)--(5,3,0)--(5,3,2)--(0,3,2)--cycle;
      \draw (0,3,2)--(0,0,2)--(5,0,2)--(5,3,2);
      \draw (5,3,0)--(5,0,0)--(5,0,2);
      \draw (0,3,1)--(5,3,1)--(5,0,1);
      \draw (0,1,2)--(1,1,2);
      \draw (1,3,2)--(1,0,2);
      \draw (2,3,2)--(2,0,2);
      \node [below] at (2.5,0,2){5};
      \node [right] at (5,1.5,0){3};
      \node [right] at (5,0,1){2};
      \end{tikzpicture}
    \end{minipage}\hfill
    \begin{minipage}{0.4\textwidth}
    \item $ 5\times(3\times2) $      
    \end{minipage}
  \end{enumerate}
  \item Construye cajas que estén formadas por cubos como este \tikz \draw (0,.5) --(0,0) -- (.5,0) -- (.5,.5) -- (0,.5) -- (.2,.7) -- (.7,.7) --(.5,.5)--(.5,0)--(.7,.2)--(.7,.7); de manera que el número total de cubos represente las siguientes operaciones:
  \begin{enumerate}\begin{multicols}{2}
    \item $ 2\times(2\times2) $
    \item $ 3\times(2\times4) $
    \item $ 4\times(1\times3) $
    \item $ 1\times(2\times3) $
    \item $ 2\times(3\times3) $
    \item $ 4\times(4\times4) $
    \end{multicols}
  \end{enumerate}
  Observe como se pueden representar en la recta numérica el producto de factores iguales
  
  \begin{tikzpicture}[>=stealth]
\draw[|->] (0,0) node[below]{0}-- (10,0) node[below]{10};
\draw [thick] (1,-.1) node[below]{1} -- (1,0.1);
\draw [thick] (2,-.1) node[below]{2} -- (2,0.1);
\draw [thick] (3,-.1) node[below]{3} -- (3,0.1);
\draw [thick] (4,-.1) node[below]{4} -- (4,0.1);
\draw [thick] (5,-.1) node[below]{5} -- (5,0.1);
\draw [thick] (6,-.1) node[below]{6} -- (6,0.1);
\draw [thick] (7,-.1) node[below]{7} -- (7,0.1);
\draw [thick] (8,-.1) node[below]{8} -- (8,0.1);
\draw [thick] (9,-.1) node[below]{9} -- (9,0.1);
\draw [|->] (0,.3) -- (3,.3);
\draw [|->] (3,.3) -- (6,.3);
\draw [|->] (6,.3) -- (9,.3);
\node [above] at (1.5,.3) {1 vez 3};
\node [above] at (4.5,.3) {1 vez 3};
\node [above] at (7.5,.3) {1 vez 3};
\draw [|->|] (0,.9) -- (9,.9);
\node [above] at (4.5,.9) {3 veces 3};
\node [below] at (4.5,-.6) {Como producto $ 3\times3 $: Como potencia $ 3^2 $};
\end{tikzpicture}

 \begin{tikzpicture}[>=stealth]
\draw[|->] (0,0) node[below]{0}-- (10,0) node[below]{10};
\draw [thick] (1,-.1) node[below]{1} -- (1,0.1);
\draw [thick] (2,-.1) node[below]{2} -- (2,0.1);
\draw [thick] (3,-.1) node[below]{3} -- (3,0.1);
\draw [thick] (4,-.1) node[below]{4} -- (4,0.1);
\draw [thick] (5,-.1) node[below]{5} -- (5,0.1);
\draw [thick] (6,-.1) node[below]{6} -- (6,0.1);
\draw [thick] (7,-.1) node[below]{7} -- (7,0.1);
\draw [thick] (8,-.1) node[below]{8} -- (8,0.1);
\draw [|->] (0,.3) -- (4,.3);
\draw [|->] (4,.3) -- (8,.3);
\node [above] at (2,.3) {2 vez 2};
\node [above] at (6,.3) {2 vez 2};
\draw [|->|] (0,.9) -- (8,.9);
\node [above] at (4,.9) {2 veces (2 veces 2)};
\node [below] at (4,-.6) {Como producto $ 2\times2\times2 $: Como potencia $ 2^2\times2=2^3 $};
\end{tikzpicture}
\item Represente sobre la recta numérica cada potencia y escríbala como una multiplicación de factores iguales.
\begin{enumerate}\begin{multicols}{4}
  \item $ 2^2 $  \item $ 3^2 $  \item $ 4^2 $ \item $ 2^4 $
\end{multicols}
\end{enumerate}
\item \begin{enumerate}
  \item Si se forma un cuadrado con 36 cuadrados como este \tikz \draw (0,0) rectangle (.4,.4);, ¿cuántos cuadrados caben por cada lado? Dibújelo y represente su respuesta por medio de una operación.
  \item  Se forma un cubo con 8 cubitos como este \tikz \draw (0,.5) --(0,0) -- (.5,0) -- (.5,.5) -- (0,.5) -- (.2,.7) -- (.7,.7) --(.5,.5)--(.5,0)--(.7,.2)--(.7,.7);. Dibuje y explique cuáles serían las dimensiones del cubo. Represente por medio de una operación.
  \item Si se forma un rectángulo con 18 cuadrados como éste \tikz \draw (0,0) rectangle (.4,.4);, ¿cuántos cuadrados debería agregar para formar un cuadrado? Dibuje. ¿Cuántos cuadrados quedarían por cada lado? Represente por medio de una operación.
\end{enumerate}
\item Represente en la recta numérica
\begin{enumerate}\begin{multicols}{3}
  \item $2\times3$ \item $ 2^3 $ \item $3^2$
\end{multicols}
\end{enumerate}
\item Escriba el número que corresponda en cada cuadrado
\begin{enumerate}\begin{multicols}{3}
  \item $ 5^2= $ \tikz \draw (0,0) rectangle (.6,.4);
  \item $ 6^3= $ \tikz \draw (0,0) rectangle (.6,.4);
  \item $ 7^2= $ \tikz \draw (0,0) rectangle (.6,.4);
  \item $ 2^4= $ \tikz \draw (0,0) rectangle (.6,.4);
  \item $ 11^2= $ \tikz \draw (0,0) rectangle (.6,.4);
  \item $ 3^5= $ \tikz \draw (0,0) rectangle (.6,.4);
\end{multicols}
\end{enumerate}
En los anteriores ejercicios buscamos la potencia, pues conocemos la base y el exponente.
\item Escriba el número que corresponde a cada rectángulo:
\begin{enumerate}\begin{multicols}{2}
  \item $ \tikz \draw (0,0) rectangle (.6,.4);^2=25 $
  \item $ \tikz \draw (0,0) rectangle (.6,.4);^3=216 $
  \item $ \tikz \draw (0,0) rectangle (.6,.4);^2=49 $
  \item $ \tikz \draw (0,0) rectangle (.6,.4);^4=16 $
\end{multicols}
\end{enumerate}
Buscamos la base, pues conocemos la potencia y el exponente
\item Resuelvo los siguientes ejercicios y justifico la respuesta:
\begin{enumerate}\begin{multicols}{3}
  \item $ \tikz \draw (0,0) rectangle (1,.4);^2=81 $
  \item $ \tikz \draw (0,0) rectangle (1,.4);^4=81 $
  \item $ \tikz \draw (0,0) rectangle (1,.4);^2=100 $
  \item $ \tikz \draw (0,0) rectangle (1,.4);^2=144 $
  \item $ \sqrt[4]{81}= $ \tikz \draw (0,0) rectangle (1,.4);
  \item $ \sqrt[4]{625}= $ \tikz \draw (0,0) rectangle (1,.4);
  \item $ \tikz \draw (0,0) rectangle (1,.4);^4=625 $
  \item $ \sqrt[5]{32}= $ \tikz \draw (0,0) rectangle (1,.4);
  \item $ \sqrt{25}= $ \tikz \draw (0,0) rectangle (1,.4);
  \item $ \sqrt[3]{27}= $ \tikz \draw (0,0) rectangle (1,.4);
  \item $ \sqrt[2]{100}= $ \tikz \draw (0,0) rectangle (1,.4);
  \item $ \sqrt[3]{125}= $ \tikz \draw (0,0) rectangle (1,.4);
\end{multicols}
\end{enumerate}
\item ¿Cuál es el exponente?
\begin{enumerate}\begin{multicols}{3}
  \item $ 9^{\tikz \draw (0,0) rectangle (.5,.3);}=81 $
  \item $ 11^{\tikz \draw (0,0) rectangle (.5,.3);}=1331 $  
  \item $ 12^{\tikz \draw (0,0) rectangle (.5,.3);}=12 $
\end{multicols}
\end{enumerate}
\item La base es 13 y la potencia 169, ¿cuál es el exponente?
\item La potencia es 64 y la base 2, ¿cuál es el exponente?
\item La potencia es 625 y la base es 5, ¿cuál es el exponente?
\item Encuentre los siguientes logaritmos
\begin{enumerate}\begin{multicols}{3}
  \item $ \log_3{81} $  \item $ \log_5{25} $ \item $ \log_6{6} $
  \item $ \log_{10}{10} $ \item $ \log_{10}{1000} $
  \item $ \log_{10}{10000} $ \item $ \log_{10}{10\,0000} $
  \item $ \log_3{81}\Leftrightarrow 3^{\tikz \draw (0,0) rectangle (.5,.3);}=81 $
\end{multicols}
\end{enumerate}
\item Compruebe cada uno de las respuestas de los ejercicios anteriores, escribiéndolas como potencias.
\item Para hallar el cuadrado de un número, se eleva al exponente \textbf{2}. Complete:

\begin{tabular}{|c|c|c|}\hline
\textbf{Número} & \textbf{Cuadrado} & \textbf{Se lee}\\
\hline 5 & $ 5^2=5\times5=25 $ & El cuadrado de 5 es 25\\
\hline 7 & &\\
\hline & $ 8^2=8\times8=64 $ & \\
\hline 10 & &\\
\hline & & El cuadrado de 12 es 144\\
\hline 15 & & \\
\hline & $ 21^2= $ & \\ \hline
\end{tabular}
\item Para hallar el cubo de un número se eleva al exponente \textbf{3}. Complete:

\begin{tabular}{|c|c|c|}\hline
\textbf{Número} & \textbf{Cubo} & \textbf{Se lee}\\
\hline 5 & $ 5^3=5\times5\times5=125 $ & El cubo de 5 es 125 \\
\hline 7 & & \\
\hline & $ 8^3 $ & \\
\hline & & El cubo de 12 es 1728\\
\hline 15 & & \\
\hline 21 & & \\ \hline
\end{tabular}
\end{enumerate}
\end{document}
