\documentclass[10pt,twoside]{article}
\usepackage[utf8]{inputenc}
\usepackage{amsmath}
\usepackage{amsfonts}
\usepackage{amssymb}
\usepackage[spanish,es-noshorthands]{babel}
\usepackage[T1]{fontenc}
\usepackage{lmodern}
\usepackage{graphicx,hyperref}
\usepackage{tikz,pgf}
\usepackage{marvosym}
\usepackage{multicol}
\usepackage{subfig}
\usepackage[papersize={6.5in,8.5in},width=5.5in,height=7in]{geometry}
\usepackage{fancyhdr}
\pagestyle{fancy}
\fancyhead[LE]{\url{www.autistici.org/mathgerman}}
\fancyhead[RE]{}
\fancyhead[RO]{\Email~ matematicas.german@gmail.com}
\fancyhead[LO]{}

\author{Germ\'an Avenda\~no Ram\'irez~\thanks{Lic. Mat. U.D., M.Sc. U.N.}}
\title{\begin{minipage}{.2\textwidth}
\includegraphics[height=1.75cm]{Images/logo-colegio.png}\end{minipage}
\begin{minipage}{.55\textwidth}
\begin{center}
Taller Combinatoria\\
Probabilidad $11^{\circ}$
\end{center}
\end{minipage}\hfill
\begin{minipage}{.2\textwidth}
\includegraphics[height=1.75cm]{Images/logo-sed.png} 
\end{minipage}}
\date{}
\begin{document}
\maketitle
\section*{Permutaciones}
El número de permutaciones de $n$ objetos de los cuales $n_{1}$ son iguales, $n_{2}$ son iguales, . . . es
\[\dfrac{n!}{n_{1}!n_{2}!\cdot \ldots} \qquad \text{donde }\; n=n_{1}+n_{2}+\ldots\]
\subsection*{Ejemplo:}
El número de permutaciones de las letras en la palabra statistics es:
\[\dfrac{10!}{3!3!1!2!1!}=50\,400\] porque hay 3 eses, 3 tes, 1 a, 2 íes y 1 c.
\begin{enumerate}
\item ¿De cuántas maneras se pueden acomodar en línea 5 canicas de colores diferentes?
\item ¿De cuántas maneras se pueden sentar 10 personas en una banca en la que sólo hay 4 asientos disponibles?
\item Se desea sentar en hilera a 5 hombres y 4 mujeres de manera que las mujeres ocupen los lugares pares. ¿De cuántas maneras es posible hacer esto?
\item ¿Cuántos números de cuatro dígitos se pueden formar con los 10 dígitos 0, 1, 2, 3, \ldots , 9, si: a) puede haber
repeticiones, b) no puede haber repeticiones y c) no puede haber repeticiones y el último dígito debe ser cero?
\item En un librero se van a acomodar cuatro libros diferentes de matemáticas, 6 libros diferentes de física y 2 libros diferentes de química. ¿Cuántos son los acomodos posibles si: a) los libros de cada materia tienen que estar juntos y b) sólo los libros de matemáticas tienen que estar juntos?
\item Cinco canicas rojas, 2 canicas blancas y 3 azules están ordenadas en línea. Si las canicas de un mismo color no se distinguen unas de otras, ¿cuántas ordenaciones distintas se pueden tener?
\item ¿De cuántas maneras pueden sentarse 7 personas a una mesa redonda si: a) las 7 se pueden sentar en cualquier lugar y b) 2 determinadas personas no pueden sentarse juntas?
\section*{Combinaciones}
\item ¿De cuántas maneras pueden colocarse 10 objetos en dos grupos, uno de 4 y otro de 6 objetos?
\item ¿De cuántas maneras puede formarse de un grupo de 9 personas un comité de 5 personas?
\item Con 5 matemáticos y 7 físicos hay que formar un comité que conste de 2 matemáticos y 3 físicos. ¿De cuántas maneras se puede formar este comité si: a) puede incluirse a cualquiera de los matemáticos y a cualquiera de los físicos, b) hay uno de los físicos que tiene que formar parte del comité y c) hay dos de los matemáticos que no pueden formar parte del comité?
\item Una niña tiene 5 flores que son todas distintas. ¿Cuántos ramos puede formar?
\item Con 7 consonantes y 5 vocales ¿cuántas palabras con 4 consonantes distintas y 3 vocales distintas pueden formarse? No importa que las palabras no tengan significado.
\section*{Probabilidad y análisis combinatorio}
\item Una caja contiene 8 pelotas rojas, 3 blancas y 9 azules. Si se extraen 3 pelotas en forma aleatoria, determinar
la probabilidad de que: a) las 3 sean rojas, b) las 3 sean blancas, c) 2 sean rojas y 1 sea blanca, d ) por lo menos 1 sea blanca, e) se extraiga una de cada color y f ) se extraigan en el orden roja, blanca, azul.
\item De una baraja de 52 cartas bien barajadas se extraen 5 cartas. Encontrar la probabilidad de que: a) 4 sean ases; b) 4 sean ases y 1 sea rey; c) 3 sean dieces y 2 sean sotas; d ) sean 9, 10, sota, reina y rey en cualquier orden; e) 3 sean de un palo y 2 de otro palo, y f ) se obtenga por lo menos 1 as.
\item Determinar la probabilidad de tener 3 seises en cinco lanzamientos de un dado.
%\item En una fábrica se encuentra que en promedio 20\% de los tornillos producidos con una máquina están defectuosos. Si se toman aleatoriamente 10 tornillos producidos con esta máquina en un día, encontrar la probabilidad de que: a) exactamente 2 estén defectuosos, b) 2 o más estén defectuosos y c) más de 5 estén defectuosos.
\end{enumerate}

\end{document}
