\documentclass[10pt,twoside]{article}
\usepackage[utf8]{inputenc}
\usepackage{amsmath}
\usepackage{amsfonts}
\usepackage{amssymb}
\usepackage[spanish,es-noshorthands]{babel}
\usepackage[T1]{fontenc}
\usepackage{lmodern}
\usepackage{graphicx,hyperref}
\usepackage{tikz,pgf}
\usepackage{multicol}
\usepackage{subfig}
\usepackage[papersize={6.5in,8.5in},width=5.5in,height=7in]{geometry}
\usepackage{fancyhdr}
\usepackage{printsudoku}
\newcommand*{\Lpack}[1]{\textsf{#1}}
\pagestyle{fancy}
\fancyhead[LE]{\includegraphics[height=12pt]{Images/logo-colegio.png} Ética $9^{\circ}$}
\fancyhead[RE]{}
\fancyhead[RO]{\textit{Germ\'an Avenda\~no Ram\'irez, Lic. U.D., M.Sc. U.N.}}
\fancyhead[LO]{}

\author{Germ\'an Avenda\~no Ram\'irez, Lic. U.D., M.Sc. U.N.}
\title{\begin{minipage}{.2\textwidth}
\includegraphics[height=1.75cm]{Images/logo-colegio.png}\end{minipage}
\begin{minipage}{.55\textwidth}
\begin{center}
Taller 07, Juicios éticos \\
Ética $9^{\circ}$
\end{center}
\end{minipage}\hfill
\begin{minipage}{.2\textwidth}
\includegraphics[height=1.75cm]{Images/logo-sed.png} 
\end{minipage}}
\date{}
\begin{document}
\maketitle
Nombre: \hrulefill Curso: \underline{\hspace*{44pt}} Fecha: \underline{\hspace*{2.5cm}}
\section*{Continuando con los hábitos}
\subsection*{Persistencia}
Es una cualidad que denota insistencia, firmeza, empeño o tenacidad en la ejecución de una actividad o labor. La persistencia también se puede interpretar como un hábito necesario para cumplir una meta. Una persona persistente no ignora las dificultades que conlleva realizar cualquier actividad o labor, por el contrario, tiene como hábito prever posibles riesgos y adelantarse a los mismos o tener siempre una actitud positiva frente a los fracasos.

Cuando los objetivos son precisos y las metas son claras, la persistencia permitirá superar las adversidades y las limitaciones humanas. Para cosechar los frutos de la persistencia es necesario:
\begin{itemize}
 \item Realizar esfuerzos y sacrificios enfocados hacia el cumplimiento de un fin máximo, aun dejando en segundo plano la satisfacción de deseos y objetivos momentáneos. Por ejemplo, si una persona debe trabajar y estudiar para graduarse como profesional, es preciso que invierta horas de sueño para cumplir con sus responsabilidades.
\item Ser constante y no darse por vencido. Para ello se requiere paciencia y tener claro que los resultados del esfuerzo no son inmediatos sino que se producirán en meses o años. En la fábula del inicio, la liebre recapacitó cuando se vio derrotada por su propia pereza, pero no fue constante cuando se enfrentó al nuevo obstáculo de atravesar el río. Una persona persistente hace uso de su creatividad para superar las dificultades y replantear sus estrategias.
\item Demostrar coherencia armonizando las formas de pensar, sentir y actuar. Si una persona vive en constante contradicción consigo misma, será difícil que alcance metas poco definidas o que actúe en concordancia con aquellas que se ha fijado. Por ejemplo, si un deportista quiere ganar una medalla de oro, será poco coherente si permanece sentado viendo televisión, tiene una alimentación poco saludable o adquiere el vicio del tabaco.
\end{itemize}
\subsection*{Evaluación}
\begin{itemize}
 \item En equipos de 3, diseñen una meta para cumplir como curso en el lapso de un mes, en aspectos académicos, culturales o de convivencia.
\item Elaboren una cartelera que incluya el propósito, los aspectos por mejorar, las fortalezas, la estrategia a implementar y las consecuencias esperadas. Pasado este tiempo evalúen sus resultados.
\end{itemize}
\section*{Juicios éticos}
\subsection*{Lo que sé}
 Lee esta leyenda y contesta las preguntas.
\subsubsection*{Mi mejor amigo}
Dice una leyenda árabe que dos amigos viajaban por el desierto y discutieron agriamente. Uno de ellos le dio una bofetada al otro. Este, ofendido, escribió en la arena: “Hoy mi mejor amigo me dio una bofetada”.

Continuaron su camino y llegaron a un oasis, donde resolvieron bañarse. El que había sido abofeteado se estaba ahogando, y el otro acudió en su rescate. Al recuperarse, tomó un cincel y escribió en una piedra: “Hoy mi mejor amigo me salvó la vida”. Intrigado, aquel le preguntó:

¿Por qué después de que te lastimé escribiste en la arena, y ahora escribes en piedra?

Cuando un gran amigo nos ofende, debemos escribirlo en la arena, donde el viento del olvido y el perdón se encargará de borrarlo.

Cuando nos pasa algo grandioso, debemos grabarlo en la piedra del corazón, de dónde ningún viento podrá hacerlo desaparecer.\footnote{Tomado de: ``La culpa es de la vaca''\\Comp. Jaime Lopera y Martha Inés Bernal}

Responde:
\begin{enumerate}
 \item ¿El muchacho abofeteado dejó de pensar en su amigo como “el mejor” después de que este lo golpeó? Justifica tu respuesta.
 \item ¿Cuál era la idea que el muchacho abofeteado quería realmente conservar de su mejor amigo?
 \item ¿Normalmente qué hace una persona cuando se siente ofendida por otra?
 \item ¿Disgustarse con una persona es razón suficiente para hablar mal de ella o emitir un juicio en su contra?
\end{enumerate}
\subsection*{Aprendo algo nuevo}
Un \emph{juicio ético} es una opinión, idea o afirmación acerca de Si una acción humana es correcta o incorrecta. Es un ejercicio propio de la ética que a su vez se define como la reflexión sobre la dimensión moral de la persona. La palabra \emph{moral} significa \emph{costumbre}, por eso se interpreta como el conjunto de conocimientos, tradiciones, creencias, normas y valores que han sido transmitidos a las personas y que moldean su manera de actuar o de comportarse.

En este sentido podemos afirmar que hacen parte de la ética el conjunto de ideas que una persona elabora, a partir de sus propios razonamientos sobre las normas que le han sido transmitidas. Estas ideas se constituyen en juicios éticos cuando se aplican a comportamientos o acciones específicas.

En la leyenda anterior, el muchacho abofeteado y salvado nunca dejó de pensar en su amigo como “el mejor”. Era conciente de las acciones que su amigo había hecho pero sus juicios nunca buscaron el desprestigio de aquel; por el contrario demostró que tenía como principio grabar en la piedra de su corazón, los razonamientos o juicios morales que consideraba valiosos.
\subsubsection*{Regulación del juicio ético o moral}
El objeto de análisis de la ética son los actos con valor moral, realizados por los individuos de manera libre, voluntaria y consciente, es decir, aquellos sobre los cuales se ejerce de algún modo, un control racional. Pero la ética no solo observa tales actos, sino que busca emitir un juicio sobre los mismos, para intentar determinar si ellos han sido éticamente buenos o éticamente malos.

El hecho de hacer ciertas afirmaciones produce juicios de valor moral que implican a personas, grupos, situaciones, cosas o acciones. Por ejemplo, cuando se dice que “aquel político es corrupto”, se está haciendo una valoración moral.

Pero este tipo de valoración o de juicio debe ser objetivo, lo que significa que debe guardar las carácterísticas del objeto en cuestión por su condición particular, independiente de valoraciones subjetivas, las cuales por lo general se emiten como producto de una reacción emocional o impulsiva.
La objetividad permite realizar juicios morales con validez universal o principios morales.

Para poder realizar un juicio ético regulado, acerca de un acto en particular, se debe considerar en primer lugar y como ya se dijo, si éste es libre, voluntario y consciente. Un aborto espontáneo, por ejemplo, no puede ser objeto de juicio moral; porque no ha sido voluntario o no se puede tener control sobre el mismo.

En segundo lugar, para elaborar un juicio ético es imprescindible identificar el fin que busca un individuo con sus actos: ¿Es un fin en beneficio personal o colectivo? O por el contrario, ¿se realiza en detrimento de otros individuos o de la comunidad?

También exiten otros parámetros que se deben tener en cuenta a la hora de emitir un juicio moral, entre otros: la utilidad de una acción, lo perjudicial que puede llegar a ser,la responsabilidad de quien realiza la acción, su jerarquía al interior del grupo que directamente se ve afectado por sus
decisiones y el conjunto de normas que son aceptadas por la mayoría.
\subsection*{Ejercito lo aprendido}
En grupos de tres personas, discutan estas preguntas y compartan sus opiniones con el curso.
\begin{enumerate}
 \item ¿Cuáles son los elementos fundamentales que se deben tener en cuenta al realizar un juicio de valor ético?
 \item Mencionen algunos actos o hechos que no sean objeto de juicio de valor y expliquen por qué. 
 \item ¿Cuándo un juicio de valor ético estaría viciado y alterado? Argumenten su respuesta.
\end{enumerate}
\section*{SUDOKU}
Para poner en práctica ``la persistencia'', desarrollar el siguiente sudoku. Recuerde que en cada columna, fila o cuadrado $3\times3$ deben estar los números 1--9 sin repetir.

Se sugiere no poner un número hasta no estar seguro que éste es el único que debe ir allí.
\cluefont{\Large}
\cellsize{2.1\baselineskip}
\begin{minipage}{0.95\linewidth}\begin{center}
TG5 (gentle) \\
\sudoku{tg5.sud}
\end{center}\end{minipage}
\end{document}
