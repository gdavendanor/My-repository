\documentclass[10pt,twoside]{article}
\usepackage[utf8]{inputenc}
\usepackage{amsmath}
\usepackage{amsfonts}
\usepackage{amssymb}
\usepackage[spanish,es-noshorthands]{babel}
\usepackage[T1]{fontenc}
\usepackage{lmodern}
\usepackage{graphicx,hyperref}
\usepackage{tikz,pgf}
\usepackage{multicol}
\usepackage[papersize={6.5in,8.5in},width=5.5in,height=7in]{geometry}
\usepackage{fancyhdr}
\pagestyle{fancy}
\fancyhead[LE]{\includegraphics[height=12pt]{Images/logo-colegio.png} Álgebra $8^{\circ}$}
\fancyhead[RE]{}
\fancyhead[RO]{\textit{Germ\'an Avenda\~no Ram\'irez, Lic. U.D., M.Sc. U.N.}}
\fancyhead[LO]{}

\author{Germ\'an Avenda\~no Ram\'irez, Lic. U.D., M.Sc. U.N.}
\title{\begin{minipage}{.2\textwidth}
\includegraphics[height=1.75cm]{Images/logo-colegio.png}\end{minipage}
\begin{minipage}{.55\textwidth}
\begin{center}
Taller 06, Productos notables\\
Álgebra $8^{\circ}$
\end{center}
\end{minipage}\hfill
\begin{minipage}{.2\textwidth}
\includegraphics[height=1.75cm]{Images/logo-sed.png} 
\end{minipage}}
\date{}
\begin{document}
\maketitle
Nombre: \hrulefill Curso: \underline{\hspace*{44pt}} Fecha: \underline{\hspace*{2.5cm}}
\section*{Repaso}
En las pasadas clases hemos visto como hacer determinar el cuadrado de un binomio y como hacer una suma por una diferencia. Recordemos que:
\subsection*{Cuadrado de un binomio}
\[(a\pm b)^{2}=a^{1}\pm 2ab+b^{2}\]
\subsubsection*{Ejemplo:}
\begin{align*}
(3x+2y)^{2}&=(3x)^{2}+2(3x)(2y)+(2y)^{2}\\
&=9x^{2}+12xy+4y^{2}
\end{align*}
\subsection*{Suma por diferencia}
$(a+b)(a-b)=a^{2}-b^{2}$
\subsubsection*{Ejemplo}
\begin{align*}
(3x+4y)(3x-4y)&=(3x)^{2}-(4y)^{2}\\
&=9x^{2}-16y^{2}
\end{align*}
\end{document}
