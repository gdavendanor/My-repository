\documentclass[letterpaper,fleqn]{article}
\usepackage[spanish,es-noshorthands]{babel}
\usepackage[utf8]{inputenc} 
\usepackage[left=1cm, right=1cm, top=1.5cm, bottom=1.7cm]{geometry}
\usepackage{mathexam}
\usepackage{amsmath}
\usepackage{graphicx}

\ExamClass{\includegraphics[height=16pt]{Images/logo-sed.png} %Ásig $^{\circ}$}
\ExamName{%}
\ExamHead{\includegraphics[height=16pt]{Images/logo-colegio.png} IEDAB}
\newcommand{\LineaNombre}{%
\par
\vspace{\baselineskip}
Nombre:\hrulefill \; Curso: \underline{\hspace*{48pt}} \; Fecha: \underline{\hspace*{2.5cm}} \relax
\par}
\let\ds\displaystyle

\begin{document}
\ExamInstrBox{
%Respuesta sin justificar mediante procedimiento no será tenida en cuenta en la calificación. Escriba sus respuestas en el espacio indicado. Tiene 45 minutos para contestar esta prueba.}
\LineaNombre
\begin{enumerate}
 \item Una computadora genera (de manera aleatoria) pares de enteros. El primer entero es entre 1 y 5, inclusive, y el segundo es entre 1 y 4, inclusive.
\begin{enumerate}
 \item Represente el espacio muestral $S$ como un diagrama de árbol \noanswer
 \item Haga una lista de sus resultados como pares ordenados, con $x$ como el primer entero y $y$ como el segundo entero.\noanswer
\end{enumerate}
 \end{enumerate}

\end{document}
