\documentclass[letterpaper,fleqn]{article}
\usepackage[spanish,es-noshorthands]{babel}
\usepackage[utf8]{inputenc} 
\usepackage[left=1cm, right=1cm, top=1.5cm, bottom=1.7cm]{geometry}
\usepackage{mathexam}
\usepackage{amsmath,amsthm,amsfonts,amssymb}
\usepackage{graphicx}
\usepackage{tikz,pgf}
\usepackage{multicol}

\ExamClass{\includegraphics[height=16pt]{Images/logo-sed.png} Cálculo $11^{\circ}$}
\ExamName{Evaluación 01, Números $\mathbb{R}$}
\ExamHead{\includegraphics[height=16pt]{Images/logo-colegio.png} IEDAB}
\newcommand{\LineaNombre}{%
\par
\vspace{\baselineskip}
Nombre:\hrulefill \; Curso: \underline{\hspace*{48pt}} \; Fecha: \underline{\hspace*{2.5cm}} \relax
\par}
\let\ds\displaystyle

\begin{document}
\ExamInstrBox{
Respuesta sin justificar mediante procedimiento no será tenida en cuenta en la calificación. Escriba sus respuestas en el espacio indicado. Tiene 45 minutos para contestar esta prueba.}
\LineaNombre
\begin{enumerate}
   \item Sobre la línea, determine la propiedad de los números reales que se ha usado:
  \begin{enumerate}
    \item $ (a+b)(a-b)=(a-b)(a+b) $ \underline{\hspace{6cm}}
    \item $ 4(a+b)=4a+4b $ \underline{\hspace{6cm}}
    \item $ (A+1)(x+y)=(A+1)x+(A+1)y $ \underline{\hspace{6cm}}
    \item $ 3x+2y=2y+3x $ \underline{\hspace{6cm}}
  \end{enumerate}
  \item Exprese cada intervalo como una desigualdad y luego grafíquela en la recta dispuesta para ello.
  
  \begin{enumerate}
    \item $ [-3,5) $ \underline{\hspace{6cm}}
    \vspace{20pt}
    \begin{flushright}
    \begin{tikzpicture}[scale=.8,>=stealth]
\draw[<->] (0,0) -- (12,0);
\draw [thick] (1,-.1) -- (1,0.1);
\draw [thick] (2,-.1) -- (2,0.1);
\draw [thick] (3,-.1) -- (3,0.1);
\draw [thick] (4,-.1) -- (4,0.1);
\draw [thick] (5,-.1) -- (5,0.1);
\draw [thick] (6,-.1) -- (6,0.1);
\draw [thick] (7,-.1) -- (7,0.1);
\draw [thick] (8,-.1) -- (8,0.1);
\draw [thick] (9,-.1) -- (9,0.1);
\draw [thick] (10,-.1) -- (10,0.1);
\draw [thick] (11,-.1) -- (11,0.1);
\end{tikzpicture}
    \end{flushright}\vspace{20pt}
    \item $ (-\infty,5]  $ \underline{\hspace{6cm}}
    \vspace{15pt}
    \begin{flushright}
  \begin{tikzpicture}[scale=.8,>=stealth]
\draw[<->] (0,0) -- (12,0);
\draw [thick] (1,-.1) -- (1,0.1);
\draw [thick] (2,-.1) -- (2,0.1);
\draw [thick] (3,-.1) -- (3,0.1);
\draw [thick] (4,-.1) -- (4,0.1);
\draw [thick] (5,-.1) -- (5,0.1);
\draw [thick] (6,-.1) -- (6,0.1);
\draw [thick] (7,-.1) -- (7,0.1);
\draw [thick] (8,-.1) -- (8,0.1);
\draw [thick] (9,-.1) -- (9,0.1);
\draw [thick] (10,-.1) -- (10,0.1);
\draw [thick] (11,-.1) -- (11,0.1);
\end{tikzpicture}
  \end{flushright}
  \end{enumerate}
  \item Exprese en notación de intervalos y luego grafique el correspondiente intervalo:
  \begin{enumerate}
    \item $ x\geq 6 $ \noanswer
     \begin{flushright}
  \begin{tikzpicture}[scale=.8,>=stealth]
\draw[<->] (0,0) -- (12,0);
\draw [thick] (1,-.1) -- (1,0.1);
\draw [thick] (2,-.1) -- (2,0.1);
\draw [thick] (3,-.1) -- (3,0.1);
\draw [thick] (4,-.1) -- (4,0.1);
\draw [thick] (5,-.1) -- (5,0.1);
\draw [thick] (6,-.1) -- (6,0.1);
\draw [thick] (7,-.1) -- (7,0.1);
\draw [thick] (8,-.1) -- (8,0.1);
\draw [thick] (9,-.1) -- (9,0.1);
\draw [thick] (10,-.1) -- (10,0.1);
\draw [thick] (11,-.1) -- (11,0.1);
\end{tikzpicture}
  \end{flushright}
    \item $ -2< x\leq 4 $ \noanswer
     \begin{flushright}
  \begin{tikzpicture}[scale=.8,>=stealth]
\draw[<->] (0,0) -- (12,0);
\draw [thick] (1,-.1) -- (1,0.1);
\draw [thick] (2,-.1) -- (2,0.1);
\draw [thick] (3,-.1) -- (3,0.1);
\draw [thick] (4,-.1) -- (4,0.1);
\draw [thick] (5,-.1) -- (5,0.1);
\draw [thick] (6,-.1) -- (6,0.1);
\draw [thick] (7,-.1) -- (7,0.1);
\draw [thick] (8,-.1) -- (8,0.1);
\draw [thick] (9,-.1) -- (9,0.1);
\draw [thick] (10,-.1) -- (10,0.1);
\draw [thick] (11,-.1) -- (11,0.1);
\end{tikzpicture}
  \end{flushright}
  \end{enumerate}
  \item Realice las operaciones indicadas, simplificando siempre que sea posible:
  \begin{enumerate}
    \item $ \frac{5}{6}+\frac{1}{9}= $ \vspace{45pt}
    \item $ 3+\frac{3}{8}-\frac{1}{6}= $ \vspace{45pt}
    \item $ 0.75(\frac{7}{9}+\frac{2}{3})= $ \vspace{45pt}
    \item $ (\frac{2}{3}-\frac{2}{7})(\frac{1}{5}-\frac{1}{4})= $ \vspace{45pt}
    \item $ \dfrac{3-\frac{2}{3}}{\frac{1}{3}-\frac{1}{5}}= $ \vspace{45pt}
  \end{enumerate}
  \item Ubique el símbolo correcto ($<$ ,$>$, o $=$) en el espacio:
  \begin{enumerate}\begin{multicols}{2}
    \item $ 5 $ \underline{\hspace{.7cm}} $ \frac{14}{3} $
    \item $ \frac{1}{3} $ \underline{\hspace{.7cm}} $ 0.33 $
    \item $ -5 $ \underline{\hspace{.7cm}} $ -\frac{14}{3} $
    \item $ |-0.87| $ \underline{\hspace{.7cm}} $ |0.87| $
    \end{multicols}
  \end{enumerate}
  \item Exprese como una desigualdad las siguientes expresiones:
  \begin{enumerate}
    \item $ b $ es negativo \hspace*{1cm} \underline{\hspace*{4cm}}
    \item $ a $ es menor que 4 \hspace*{1cm} \underline{\hspace*{4cm}}
    \item $ q $ es menor que 8 y mayor o igual que $ -5 $ \hspace*{1cm} \underline{\hspace*{4cm}}
    \item La distancia de $ t $ a 6 es al menos 10 \hspace*{1cm} \underline{\hspace*{4cm}}
  \end{enumerate}
   \section*{Preparándonos para la Prueba Saber}
  En la recta numérica, se han señalado algunos puntos con sus respectivas coordenadas
  
  \begin{tikzpicture}
  \draw[<->] (-0.25,0) -- (10.25,0);
\draw [thick] (0,-.1)node[below]{$-1$} -- (0,0.1)node[above]{A};
\draw [thick](6,-.1)[thick]node[below]{$-\frac{1}{4}$} -- (6,0.1)node[above]{B};
\draw [thick] (7,-.1)node[below]{$-\frac{1}{8}$} -- (7,0.1)node[above]{C};
\draw [thick] (8,-.1)node[below]{$0$} -- (8,0.1)node[above]{D};
\draw [thick] (9,-.1)node[below]{$\frac{1}{8}$} -- (9,0.1)node[above]{E};
\draw [thick] (10,-.1)node[below]{$\frac{1}{4}$} -- (10,0.1)node[above]{F};
  \end{tikzpicture}
  \item Si $\overline{DE}$ se divide en $n$ segmentos congruentes, la longitud de cada uno de los $n$ segmentos es:
  \begin{enumerate}\begin{multicols}{4}
  \item $\dfrac{1}{8n}$
  \item $\dfrac{8}{n}$  
  \item $\dfrac{1}{n}$
  \item $\dfrac{4}{n}$
  \end{multicols}
  \end{enumerate}
  \answer*{Justificación}
  \item Si $M$ y $N$ son los puntos medios de $\overline{AB}$ y $\overline{CD}$ respectivamente, la longitud $MN$ es,
  \begin{enumerate}
  \begin{multicols}{4}
  \item $\dfrac{9}{15}$
  \item $\dfrac{11}{16}$
  \item $\dfrac{1}{2}$
  \item $\dfrac{5}{8}$
  \end{multicols}
  \end{enumerate}
  \answer*{Justificación}
  \item De la expresión $\left[\dfrac{1-\sqrt{3}}{2}\right]^{2}$ se puede afirmar que corresponde a un números
  \begin{enumerate}
  \item irracional y se ubica en $\overline{CD}$
  \item irracional y se ubica en $\overline{DE}$
   \item racional y se ubica en $\overline{AB}$
  \item racional y se ubica en $\overline{BD}$
  \end{enumerate}
  \answer*{Justificación}
\end{enumerate}
\end{document}
