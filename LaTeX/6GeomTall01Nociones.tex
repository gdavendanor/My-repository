\documentclass[twoside]{article}
\usepackage[utf8]{inputenc}
\usepackage{amsmath,amsfonts,amssymb,amsthm,latexsym}
\usepackage[spanish,es-noshorthands]{babel}
\usepackage[T1]{fontenc}
\usepackage{lmodern}
\usepackage{graphicx,hyperref}
\usepackage{tikz,pgf}
\usepackage{marvosym}
\usepackage{multicol}
\usepackage{fancyhdr}
\usepackage[papersize={5.5in,8.5in},left=.75cm,right=.75cm,top=1.5cm,bottom=1.25cm]{geometry}
\usepackage{fancyhdr}
\pagestyle{fancy}
\fancyhead[LE]{\Email matematicas.german@gmail.com}
\fancyhead[RE]{}
\fancyhead[RO]{\url{https://www.autistici.org/mathgerman}}
\fancyhead[LO]{}

\author{Germ\'an Avenda\~no Ram\'irez~\thanks{Lic. Mat. U.D., M.Sc. U.N.}}
\title{\begin{minipage}{.2\textwidth}
\includegraphics[height=1.75cm]{Images/logo-colegio.png}\end{minipage}
\begin{minipage}{.55\textwidth}
\begin{center}
Nociones\\
Geometría $6^{\circ}$
\end{center}
\end{minipage}\hfill
\begin{minipage}{.2\textwidth}
\includegraphics[height=1.75cm]{Images/logo-sed.png} 
\end{minipage}}
\date{}
\thispagestyle{plain}
\begin{document}
\maketitle
Nombre: \hrulefill Curso: \underline{\hspace*{44pt}} Fecha: \underline{\hspace*{2.5cm}}
\section{ACTIVIDAD: Leamos lo siguiente:}
Desde tiempos muy antiguos el hombre ha sentido necesidad por conocer y tener explicaciones racionales acerca de su medio ambiente y de los diferentes fenómenos que a diario suceden en él.\\

En la búsqueda de este conocimiento la geometría juega un papel esencial ya que permite entre otras cosas solucionar problemas que tienen que ver con ubicaciones en el espacio, mediciones de objetos, eventos y fenómenos, apreciar las bellezas de la naturaleza y las obras diseñadas o construidas por el hombre además de ser un componente muy importante en la cultura de un pueblo.\\

Así tenemos que algunos hombres a través de la historia se han dedicado a descubrir, crear, relacionar, ampliar y aplicar los conocimientos geométricos presentes en su época.\\

Dentro de estos hombres de ciencia podemos destacar a los griegos: Thales de Mileto, Pitágoras de Samos y Euclides de Alejandría, quienes vivieron entre los siglos VI y III a. c e iniciaron el estudio sistemático riguroso y deductivo de la geometría, a diferencia de los conocimientos geométricos de otras civilizaciones como la babilónica con los cuales sólo se solucionaban problemas de tipo práctico y muy particulares.\\

Otros científicos a pesar de no haber contribuido con significativos aportes al desarrollo de la geometría, sí fueron grandes propulsores de ésta. Así, tenemos el gran filósofo griego Platón (429-348 a.C.) quien en sus famosos Diálogos resaltó la importancia de la matemática y en especial de la geometría: ``Nadie que ignore la geometría puede pasar'' decía un letrero a la entrada de la Academia de Atenas, sitio de reunión de
filósofos y eruditos griegos de la época.\\

\section{ACTIVIDAD 2: En parejas}
Discutamos las siguientes preguntas:
\begin{enumerate}
\item ¿Cuáles creen ustedes que fueron los primeros conocimientos geométricos que
manejó el hombre? ¿Dónde se manifestaron?
\item ¿Qué elementos geométricos aparecen en nuestro medio ambiente y vida cotidiana?
\item ¿Qué significado encontramos a las afirmaciones ``ser un hombre recto'', ``tener posiciones verticales''?
\end{enumerate}
\section{ACTIVIDAD: Estudia los puntos y las rectas}
\subsection*{Analiza lo siguiente:}
Si observamos la punta del lápiz, un semáforo, los ojos de una res, las pecas o lunares de una persona, la esquina de un cubo, tenemos buenos ejemplos de lo que es punto.\\

Sin embargo la idea de punto es abstracta y se llega incluso al caso de considerarlo como un término indefinido aun sabiendo que todos tenemos una idea más o menos clara de lo que es el punto.\\

Eso que estamos pensando en este momento acerca de lo que es punto, es lo que es el punto.\\

\begin{itemize}
\item Observa los siguientes gráficos:

 \tikz[scale=2] \draw (0,.5) --(0,0) -- (.5,0) -- (.5,.5) -- (0,.5) -- (.2,.7) -- (.7,.7) --(.5,.5)--(.5,0)--(.7,.2)--(.7,.7); \tikz \filldraw (3,0) circle (.2cm); \tikz \draw (5,0)--(7,0)--(6,1.5)--(5,0) ; \tikz \filldraw (0,0) circle (2pt);
\end{itemize}
\end{document}
