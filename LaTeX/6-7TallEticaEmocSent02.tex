\documentclass[10pt,twoside]{article}
\usepackage[utf8]{inputenc}
\usepackage{amsmath,amsfonts,amssymb,amsthm,latexsym}
\usepackage[spanish,es-noshorthands]{babel}
\usepackage[T1]{fontenc}
\usepackage{lmodern}
\usepackage{graphicx,hyperref}
\usepackage{tikz}
\usepackage{multicol}
\usepackage{subfig}
\usepackage{marvosym}
\usepackage[papersize={5.5in,8.5in},left=.75cm,right=.75cm,top=1.5cm,bottom=1.25cm]{geometry}
\usepackage{fancyhdr}
\pagestyle{fancy}
\fancyhead[LE]{\url{http://www.autistici.org/mathgerman}}
\fancyhead[RE]{}
\fancyhead[RO]{\Email iedabgerman@autistici.org}
\fancyhead[LO]{}

\author{Germ\'an Dar\'io Avenda\~no Ram\'irez, \thanks{Lic. Mat. U.D., M.Sc. U.N.}}
\title{\begin{minipage}{0.15\textwidth}\includegraphics[height=1.7cm]{Images/logo-colegio.png}
\end{minipage}\hfill \begin{minipage}{0.85\textwidth}\begin{center}
Emociones y sentimientos \\Guía - ética $6^{\circ}$\end{center}
\end{minipage}}
\date{}

\begin{document}
\maketitle
Nombre: \hrulefill Curso: \underline{\hspace{1cm}}  Fecha: \underline{\hspace{2cm}}\\
\section*{Reflexiono - Actividad 1}
\begin{itemize}
  \item ¿Cómo expreso mi alegría?
\begin{multicols}{2}
  Me alegra:\\Me entristece: 
\end{multicols}
  \item ¿Qué hago o puedo hacer para sembrar alegría a mi alrededor?
  \item ¿Qué me pone de mal humor?
  \item ¿Qué puedes hacer para sonreír y evitar el mal humor?
\end{itemize}
LA TERNURA. Enternecernos es señal de grandeza, de bondad y de humanidad, no de debilidad.\\
Es humano conmoverse, emocionarse hasta las lágrimas.\\
De ves en cuando es conveniente que las lágrimas rueden por las mejillas aún de los varones a quienes la sociedad les ha prohibido llorar.
\begin{multicols}{2}
  \begin{itemize}
    \item[-] Ante un niño siento: \ldots
    \item[-] Ante un enfermo siento: \ldots
    \item[-] Ante una mujer embarazada siento: \ldots
    \item[-] Ante un animalito siento: \ldots
    \item[-] Ante una flor siento: \ldots
    \item[-] Ante un moribundo siento: \ldots
    \item[-] Ante un pordiosero siento: \ldots
  \end{itemize}
\end{multicols}
LA ADMIRACIÓN. Vivimos rodeados de demasiadas cosas bellas y sublimes: paisajes, flores, frutos, animales, seres humanos, estrellas, armonías\ldots\\
HAZ UN INVENTARIO DE TODO LO HERMOSO QUE TE RODEA EN ESTE
LUGAR Y EN ESTE MOMENTO.\\
Lamentablemente nos hemos acostumbrado a la belleza. Ya somos incapaces de contemplar... y por lo tanto de disfrutar. Hemos perdido la capacidad de contemplación, de admiración, de emoción ante las bellezas que nos depara la naturaleza o que hemos creado los seres humanos.\\
Es preciso pues que ejercitemos nuestros SENTIMIENTOS ESTÉTICOS.
\begin{itemize}\begin{multicols}{2}
  \item Al salir el sol siento: \ldots
  \item Al atardecer siento: \ldots
  \item Al anochecer siento: \ldots
  \item Contemplando la luna y las estrellas siento: \ldots 
  \item Al escuchar el trinar de los pájaros, la armonía de las cascadas y torrentes\ldots siento.\end{multicols}
\end{itemize}
LA GRATITUD. Todos tenemos tantas cosas qué agradecer. Hacia muchas personas deberíamos estar agradecidos: por nuestra vida, nuestra salud, nuestro trabajo, etcétera.
\begin{center}
LA GRATITUD ES EL MÁS NOBLE Y ESCASO DE LOS SENTIMIENTOS.
\end{center}
\begin{itemize}\begin{multicols}{2}
  \item Debo estar agradecido por:
  \item Debo estar agradecido con:\end{multicols}
  \item El agradecimiento se manifiesta:
\end{itemize}
EMOCIONES Y SENTIMIENTOS PELIGROSOS QUE DEBEMOS EVITAR O
CONTROLAR.\\
LOS CELOS. Consisten en no querer que los demás tengan lo que yo tengo, hagan lo que yo hago, sepan lo que yo sé, etcétera.
\begin{itemize}
  \item Tener celos demuestra:
\end{itemize}
LA ENVIDIA. Consiste en rabiar, sufrir o entristecerse porque otros triunfan, están mejor que nosotros, tienen más, son más, saben más\ldots\\
El envidioso prefiere que los otros no progresen. Todo el adelanto de los demás lo llena de pesar y de inquietud.\\
En lugar de exaltar a los triunfadores e imitarlos, los envidiosos se entristecen de ello. Lo correcto sería ver en los éxitos y los triunfos de los demás una llamada y una enseñanza
para el propio progreso. La envidia paraliza la vida y acaba con la alegría de vivir.\\
\begin{center}
LA ENVIDIA ES LA PASIÓN MÁS COMÚN Y ESTERILIZANTE DE LA VIDA Y DEL PROGRESO.
\end{center}
\textbf{FÁBULA.} Posada sobre un árbol estaba una bandada de pájaros. Un hombre que por allí pasaba arrojó sobre el césped un pan. Uno de los pájaros voló presuroso y agarró con su pico una miga. Los demás pájaros se abalanzaron sobre él para quitársela. Uno de ellos se la comió. Cuando los pájaros volvieron a buscar el pan, no lo encontraron; un gato se lo había comido.
\section*{Reflexiona - Actividad 2}
\begin{itemize}
  \item ¿Qué enseñanzas te trae esta fábula para tu vida?
  \item ¿Sucede esto con frecuencia a tu alrededor? Da ejemplos.
\end{itemize}
  \begin{center}
  LA VIDA ES UN INMENSO Y GENEROSO BANQUETE;
ESTA LLENA DE OPORTUNIDADES PARA TODOS.
  \end{center}
EL RESENTIMIENTO. Es un anhelo constante de desquite contra los triunfadores. Muchos viven mordiéndose la lengua y fraguando venganzas. Lo único que se logra es amargarse la vida y crear malestar alrededor.
\begin{center}
LOS CELOS, LA ENVIDIA Y EL RESENTIMIENTO SON
EL GRAN LASTRE EMOCIONAL QUE IMPIDE EL
PROGRESO Y LA FELICIDAD PERSONAL Y DE LA
HUMANIDAD, A PESAR DE ESTAR LLENA
DE OPORTUNIDADES PARA TODOS.
\end{center}
EL MIEDO. Tener miedo es natural; todos lo tenemos o hemos tenido. Lo grave es que el miedo paralice nuestra vida.\\\\
Hay miedos reales pero hay otros infundados como los siguientes:
\begin{itemize}\begin{multicols}{2}
  \item El miedo al qué dirán los otros.
  \item El miedo a los padres y a la autoridad.
  \item El miedo al riesgo.
  \item El miedo al cambio.
  \item El miedo al fracaso
  \item El miedo a\ldots (En tu cuaderno añade otros).  
\end{multicols}
\end{itemize}
EL LAMENTO Y LA QUEJUMBRE. La queja y el lamento pueden servir de desahogo transitorio pero cuando se convierten en una costumbre bloquean nuestra capacidad de vivir y acaban con la autoestima y la seguridad en nosotros mismos.\\
\emph{No puedo convertirme en un quejumbroso ni podemos permitir que nuestra sociedad se convierta en un coro de plañideras que sirve sólo para conducir cadáveres a los cementerios.}
\section*{Actividad 3 - En el cuaderno}
Dibuja en forma de caricaturas o recortes las caras de aquellos
que tienen sentimientos negativos o positivos: alegre, envidioso.
Junto a cada cara coloca una frase que ilustre el sentimiento.
\section*{Actividad 4 - lea}
\subsection*{La jaula}
\textit{Allá en una lejana pradera, en las riberas de un arroyo cristalino, encontré una jaula cuyas barras habían sido armadas por una mano maestra. En una de sus esquinas yacía, muerto, un pájaro; en otra, había dos tácitas, una sin agua y otra sin grano.\\
Me puse entonces a observar y a meditar en todo lo que tenía frente a mí, y me pareció que en el espectáculo de ese pájaro muerto y en la voz del arroyo, y en esas dilatadas praderas,
había una lección que hablaba a la conciencia e interrogaba nuestras profundas intimidades.\\
Medité, y descubrí que ese pájaro humilde había muerto al lado del arroyo luchando desesperadamente contra la sed; y que en medio de esas vastas praderas, cuna de la vida, había perecido de hambre.\\
Momentos después vi la jaula transformarse en un cuerpo humano transparente, y el pájaro en un corazón con una profunda herida, del centro de la cual manaba una sangre de un color rojo vivo; y vi que los bordes de la herida se habían transformado en los labios de una mujer triste.\\
Y oí salir de esa herida una voz que decía: “Yo soy el corazón humano, esclavo de la materia y víctima de las leyes terrenales. En medio de las grandes bellezas de la creación, y en las riberas de los manantiales de la vida, fui apresado en la jaula de unas leyes que el hombre ha dictado a los sentimientos. Y en las manos del amor, y ante los altares erigidos a la belleza, fui sacrificado sin piedad y morí en el abandono. Porque todo lo
que la belleza y el amor generosamente brindan, me fue vedado. Todo lo que me atraía es, según las leyes de los hombres, una vergüenza; y todo lo que deseaba, una vil degradación.\\
“Sí soy el corazón humano que fui confinado en una prisión hecha por unas leyes sociales que me privaron de mis fuerzas, y estrangularon mis sueños. Y con unas cadenas imaginarias, me redujeron a la impotencia, y así perecí. Y abandonado en los oscuros callejones de una civilización sin sentimientos de justicia, rendí mi último aliento ante una humanidad que tiene la lengua paralizada y los ojos secos, pero que siempre sonríe”.}
\begin{flushright}
(JALIL GIBRAN. “De las tempestades”.)
\end{flushright}
\subsection*{Actividad 5 - Reflexiona}
\begin{itemize}
  \item ¿Cuáles son los pensamientos que más te impactaron del anterior escrito?
  \item ¿Eres libre en tus sentimientos, o estás enjaulado o encadenado? ¿Cuáles son tus barrotes y tus cadenas?
  \item ¿Cuál es el mensaje de esta fábula?. Escríbelo en una corta frase.
  \item ¿Sucede entre nosotros? Da ejemplos.
\end{itemize}
\end{document}
