\documentclass[10pt,twoside]{article}
\usepackage[utf8]{inputenc}
\usepackage{amsmath,amsfonts,amssymb,amsthm,latexsym}
\usepackage[spanish,es-noshorthands]{babel}
\usepackage[T1]{fontenc}
\usepackage{lmodern}
\usepackage{graphicx,hyperref}
\usepackage{pgf,tikz}
\usepackage{multicol}
\usepackage{subfig}
\usepackage[papersize={6.5in,8.5in},width=5.5in,height=7in]{geometry}
\usepackage{fancyhdr}
\pagestyle{fancy}
\fancyhead[LE]{\url{http://germandario.byethost4.com}}
\fancyhead[RE]{}
\fancyhead[RO]{\textit{Germ\'an Dar\'io Avenda\~no Ram\'irez, Lic - M.Sc.}}
\fancyhead[LO]{}

\author{Germ\'an Dar\'io Avenda\~no Ram\'irez, Lic. - M.Sc.}
\title{\begin{minipage}{0.15\textwidth}\includegraphics[height=1.7cm]{Images/logo-colegio.png}
\end{minipage}\hfill \begin{minipage}{0.85\textwidth}\begin{center}
ANIMAPLANO 08\\MATEMÁTICAS $8^{\circ}$\end{center}
\end{minipage}}
\date{}

\begin{document}
\maketitle
Nombre: \hrulefill Curso: \underline{\hspace{1cm}}  Fecha: \underline{\hspace{2cm}}\\
\section*{ANIMAPLANO}
\begin{tikzpicture}
\draw[help lines] (0,0) grid (9,9);
\node[left] at (0,8) {11};
\node[left] at (0,7) {21};
\node[left] at (0,6){31};
\node[left] at (0,5){41};
\node[left] at (0,4){51};
\node[left] at (0,3){61};
\node[left] at (0,2){71};
\node[left] at (0,1){81};
\node[left] at (0,0){91};
\node[above] at (0,9){1};
\node[above] at (1,9){2};
\node[above] at (2,9){3};
\node[above] at (3,9){4};
\node[above] at (4,9){5};
\node[above] at (5,9){6};
\node[above] at (6,9){7};
\node[above] at (7,9){8};
\node[above] at (8,9){9};
\node[above] at (9,9){10};
\end{tikzpicture}
\begin{enumerate}
  \item Los minutos que hay en 3/4 de hora=
\begin{multicols}{2}
  \item Si $ m-37=46 $, entonces $ m= $
  \item Halle $ (\sqrt{100})^2-2^3= $
  \item $ (\sqrt{100}\times\sqrt{100})-3^2= $
  \item Resuelva $ 3^2\times3^2= $
  \item En años, 1 siglo, menos 28 años
  \item 1/2 centena + 1 docena =
  \item Calcule $ 100-83= $
\end{multicols}  
  \item Las veces en que está 9 en 81=
  \item Sume los lados con los vértices del pentágono
\begin{multicols}{2}
   \item $ (-5)+(-12)+37= $
  \item Si $ x=6 $, entonces $ x^2+3= $
  \item Si $ x=6 $, entonces $ x^2+x+6= $
  \item Si $ n=8 $, luego $ 8n+(n-3)= $
\end{multicols} 
  \item Sume 5 docenas, más 2 decenas, más 7 unidades
  \item 1 siglo, menos 1 lustro
  \item Represente $ LXXXVI= $
  \item Represente $ XLVIII= $
  \item Si $ A=\{2,5,7,9\} $, $ B=\{1,5,8,9\} $, $ A\cap B=\{\hspace{1cm}\} $, el producto de los elementos de $ A\cap B $ es:
  \item En unidades 1/4 de centena=
  \item Sea $ A=\{10,5,2\} $, $ B=\{5,10,6\} $ $ A \cup  B=\{\hspace{1cm}\}  $, la suma de los elementos de $ A\cup B $ es:
  \item $ (2\times4!)-\sqrt{49}= $
  \item Halle $ 8^2-\sqrt{9}= $
  \item 1 docena + 3 decenas=
  \item El triple de 10, más $ \sqrt{16}= $
  \item $ (159\div3)= $
  \item Lo que le falta al número 22 para que sea 105 es:
  \item Resuelva $ 9^2+3^1= $
  \item Si $ m-31=45 $, entonces $ m= $
  \item Si $ 27+m=94 $, entonces $ m= $
  \item Halle $ 100-54 $
\end{enumerate}
\end{document}
