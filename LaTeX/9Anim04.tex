\documentclass[letterpaper,11pt,twoside]{article}
\usepackage[utf8]{inputenc}
\usepackage{amsmath,amsfonts,amssymb,amsthm,latexsym}
\usepackage[spanish,es-noshorthands]{babel}
\usepackage[T1]{fontenc}
\usepackage{lmodern}
\usepackage{graphicx,hyperref}
\usepackage{tikz,pgf}
\usepackage{multicol}
\usepackage{fancyhdr}
\usepackage[height=9.5in,width=7in]{geometry}
\usepackage{fancyhdr}
\pagestyle{fancy}
\fancyhead[LE]{matematicas.german@gmail.com}
\fancyhead[RE]{}
\fancyhead[RO]{\url{https://www.autistici.org/mathgerman}}
\fancyhead[LO]{}

\author{Germ\'an Avenda\~no Ram\'irez~\thanks{Lic. Mat. U.D., M.Sc. U.N.}}
\title{\begin{minipage}{.2\textwidth}
\includegraphics[height=1.75cm]{Images/logo-colegio.png}\end{minipage}
\begin{minipage}{.55\textwidth}
\begin{center}
Animaplano 4\\
Matemáticas $9^{\circ}$
\end{center}
\end{minipage}\hfill
\begin{minipage}{.2\textwidth}
\includegraphics[height=1.75cm]{Images/logo-sed.png} 
\end{minipage}}
\date{}
\thispagestyle{plain}
\begin{document}
\maketitle
Nombre: \hrulefill Curso: \underline{\hspace*{44pt}} Fecha: \underline{\hspace*{2.5cm}}
\begin{multicols}{2}
\section*{Cuestionario}
\begin{enumerate}
\item La cuarta parte del ángulo llano
\item El área de un rectángulo cuyo largo es 6 unidades más que el ancho y cuyo perímetro es 32.
\item El cuádruple de 14
\item El séxtuple del quinto número primo.
\item El entero más cercano a 74,8952
\item El largo de un rectángulo cuya área es 2664 y su ancho es 36
\item El largo de un rectángulo cuyo perímetro es 228 y su ancho mide 50 unidades.
\item La quinta parte de 315.
\item El perímetro de un cuadrado cuyo lado mide $3^{2}$ cm.
\item Al racionalizar $\dfrac{\sqrt{1250}}{\sqrt{2}}$ queda:
\item Si $x+y=57$ y, $x-y=29$, luego $x=$?
\item En el ejercicio anterior $y=$?
\item La solución positiva al solucionar la ecuación cuadrática $x^{2}-2x-15=0$
\item La solución positiva de la ecuación cuadrática $x^{2}-2x-24=0$
\item Extremidades totales en 3 centauros\footnote{En la mitología griega, el centauro, ``matador de toros'', ``cien fuertes'', en latín (Centaurus/Centauri) es una criatura con la cabeza, los brazos y el torso de un humano y el cuerpo y las patas de un caballo. Las versiones femeninas reciben el nombre de centáurides (Tomado de wikipedia)}
\item En metros, 200 dm más 900 cm
\item Si $n^{5}-16=16$, halle $n^{0}\times 39=$
\item Si $a^{n}\times a^{m}=a^{14}$, $m=2^{3}$, entonces $m\times n=$?
\item La moda en el grupo de datos \{39, 69, 75, 69, 70, 25, 69\} es:
\item Reste 1/3 de 63, a 1/5 de 500:
\item Número de minutos en 5\,820 segundos
\item Si $a+b=94$, $a^{2}=64$, luego $b=$?
\item El doble de 1/5 de 185
\item Si $a=126$, y, $b=a/2$, entonces $b=$?
\item En metros, medio Hm más 10 dm
\item $25\sqrt{2}-7\sqrt{2}+22\sqrt{2}=$ \underline{\hspace{.7cm}} veces $\sqrt{2}$
\item En años, doce lustros y un año.
\item En unidades, una centena menos el séptimo número primo.
\item La pendiente de la recta dada por la ecuación $-y=45-81x$
\item Reste 6 años a 1 siglo
\item La suma de dos números es 168 y el mayor excede al menor en 24. El número mayor es:
\item Con base en el ejercicio anterior, el número menor es:
\item Sume 4 m$^{2}$, al área de un cuadrado cuyo perímetro es 36 m.
\item El triple del doble de la tercera parte del número 48
\item En años, 20 lustros -- 3 años:
\end{enumerate}
\end{multicols}
\section{Animaplano}
\begin{center}
\begin{tikzpicture}
%Animaplano (0,0)--(9,9)
 \fill (0,0) node[above]{0} circle (0.2ex);
 \fill (1,0) node[above]{1} circle (0.2ex);
 \fill (2,0) node[above]{2} circle (0.2ex);
 \fill (3,0) node[above]{3} circle (0.2ex);
 \fill (4,0) node[above]{4} circle (0.2ex);
 \fill (5,0) node[above]{5} circle (0.2ex);
 \fill (6,0) node[above]{6} circle (0.2ex);
 \fill (7,0) node[above]{7} circle (0.2ex);
 \fill (8,0) node[above]{8} circle (0.2ex);
 \fill (9,0) node[above]{9} circle (0.2ex);
 \fill (0,-1) node[left]{10} circle (0.2ex);
 \fill (1,-1) circle (0.2ex);
 \fill (2,-1) circle (0.2ex);
 \fill (3,-1) circle (0.2ex);
 \fill (4,-1) circle (0.2ex);
 \fill (5,-1) circle (0.2ex);
 \fill (6,-1) circle (0.2ex);
 \fill (7,-1) circle (0.2ex);
 \fill (8,-1) circle (0.2ex);
 \fill (9,-1) circle (0.2ex);
 \fill (0,-2) node[left]{20} circle (0.2ex);
 \fill (1,-2) circle (0.2ex);
 \fill (2,-2) circle (0.2ex);
 \fill (3,-2) circle (0.2ex);
 \fill (4,-2) circle (0.2ex);
 \fill (5,-2) circle (0.2ex);
 \fill (6,-2) circle (0.2ex);
 \fill (7,-2) circle (0.2ex);
 \fill (8,-2) circle (0.2ex);
 \fill (9,-2) circle (0.2ex);
 \fill (0,-3) node[left]{30} circle (0.2ex);
 \fill (1,-3) circle (0.2ex);
 \fill (2,-3) circle (0.2ex);
 \fill (3,-3) circle (0.2ex);
 \fill (4,-3) circle (0.2ex);
 \fill (5,-3) circle (0.2ex);
 \fill (6,-3) circle (0.2ex);
 \fill (7,-3) circle (0.2ex);
 \fill (8,-3) circle (0.2ex);
 \fill (9,-3) circle (0.2ex);
 \fill (0,-4) node[left]{40} circle (0.2ex);
 \fill (1,-4) circle (0.2ex);
 \fill (2,-4) circle (0.2ex);
 \fill (3,-4) circle (0.2ex);
 \fill (4,-4) circle (0.2ex);
 \fill (5,-4) circle (0.2ex);
 \fill (6,-4) circle (0.2ex);
 \fill (7,-4) circle (0.2ex);
 \fill (8,-4) circle (0.2ex);
 \fill (9,-4) node[right]{49} circle (0.2ex);
 \fill (0,-5) node[left]{50} circle (0.2ex);
 \fill (1,-5) circle (0.2ex);
 \fill (2,-5) circle (0.2ex);
 \fill (3,-5) circle (0.2ex);
 \fill (4,-5) circle (0.2ex);
 \fill (5,-5) circle (0.2ex);
 \fill (6,-5) circle (0.2ex);
 \fill (7,-5) circle (0.2ex);
 \fill (8,-5) circle (0.2ex);
 \fill (9,-5) circle (0.2ex);
 \fill (0,-6) node[left]{60} circle (0.2ex);
 \fill (1,-6) circle (0.2ex);
 \fill (2,-6) circle (0.2ex);
 \fill (3,-6) circle (0.2ex);
 \fill (4,-6) circle (0.2ex);
 \fill (5,-6) circle (0.2ex);
 \fill (6,-6) circle (0.2ex);
 \fill (7,-6) circle (0.2ex);
 \fill (8,-6) circle (0.2ex);
 \fill (9,-6) circle (0.2ex);
 \fill (0,-7) node[left]{70} circle (0.2ex);
 \fill (1,-7) circle (0.2ex);
 \fill (2,-7) circle (0.2ex);
 \fill (3,-7) circle (0.2ex);
 \fill (4,-7) circle (0.2ex);
 \fill (5,-7) circle (0.2ex);
 \fill (6,-7) circle (0.2ex);
 \fill (7,-7) circle (0.2ex);
 \fill (8,-7) circle (0.2ex);
 \fill (9,-7) circle (0.2ex);
 \fill (0,-8) node[left]{80} circle (0.2ex);
 \fill (1,-8) circle (0.2ex);
 \fill (2,-8) circle (0.2ex);
 \fill (3,-8) circle (0.2ex);
 \fill (4,-8) circle (0.2ex);
 \fill (5,-8) circle (0.2ex);
 \fill (6,-8) circle (0.2ex);
 \fill (7,-8) circle (0.2ex);
 \fill (8,-8) circle (0.2ex);
 \fill (9,-8) circle (0.2ex);
 \fill (0,-9) node[left]{90} circle (0.2ex);
 \fill (1,-9) circle (0.2ex);
 \fill (2,-9) circle (0.2ex);
 \fill (3,-9) circle (0.2ex);
 \fill (4,-9) circle (0.2ex);
 \fill (5,-9) circle (0.2ex);
 \fill (6,-9) circle (0.2ex);
 \fill (7,-9) circle (0.2ex);
 \fill (8,-9) circle (0.2ex);
 \fill (9,-9) node[right]{99} circle (0.2ex);
 % \draw (5,-4)--(5,-5)--(6,-5)--(6,-6)--(5,-7)--(4,-7)--(4,-6)--(3,-6)--(6,-3)--(5,-2)--(3,-4)--(4,-1)--(5,0)--(6,0)--(8,-1)--(9,-2)--(9,-3)--(8,-4)--(9,-6)--(9,-7)--(7,-9)--(6,-8)--(4,-7)--(3,-6)--(1,-5)--(0,-4)--(1,-6)--(3,-8)--(1,-8)--(4,-9)--(6,-9)--(2,-7)--(5,-8)--(6,-9)--(7,-9);
\end{tikzpicture}
\end{center}
\end{document}
