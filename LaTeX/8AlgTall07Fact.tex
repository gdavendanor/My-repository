\documentclass[10pt,twoside]{article}
\usepackage[utf8]{inputenc}
\usepackage{amsmath}
\usepackage{amsfonts}
\usepackage{amssymb}
\usepackage[spanish,es-noshorthands]{babel}
\usepackage[T1]{fontenc}
\usepackage{lmodern}
\usepackage{graphicx,hyperref}
\usepackage{tikz,pgf}
\usepackage{multicol}
\usepackage{subfig}
\usepackage[papersize={6.5in,8.5in},width=5.5in,height=7in]{geometry}
\usepackage{fancyhdr}
\pagestyle{fancy}
\fancyhead[LE]{\includegraphics[height=12pt]{Images/logo-colegio.png} Álgebra $8^{\circ}$}
\fancyhead[RE]{}
\fancyhead[RO]{\textit{Germ\'an Avenda\~no Ram\'irez, Lic. U.D., M.Sc. U.N.}}
\fancyhead[LO]{}

\author{Germ\'an Avenda\~no Ram\'irez, Lic. U.D., M.Sc. U.N.}
\title{\begin{minipage}{.2\textwidth}
\includegraphics[height=1.75cm]{Images/logo-colegio.png}\end{minipage}
\begin{minipage}{.55\textwidth}
\begin{center}
Taller 07, Factorización  \\
Álgebra $8^{\circ}$
\end{center}
\end{minipage}\hfill
\begin{minipage}{.2\textwidth}
\includegraphics[height=1.75cm]{Images/logo-sed.png} 
\end{minipage}}
\date{}
\begin{document}
\maketitle
Nombre: \hrulefill Curso: \underline{\hspace*{44pt}} Fecha: \underline{\hspace*{2.5cm}}
\section*{Taller}
\subsection*{Quiz de conceptos}
Para los problemas 1--10, conteste V o F
\begin{enumerate}
\item La factorizaci\'{o}n es el proceso inverso a la multiplicaci\'{o}n.
\item La propiedad distributiva de la forma $ab+ac=a(b+c)$ es aplicada para factorizar polinomios
\item Un polinomio puede ser factorizado de m\'{u}ltiples formas, pero solo una es la completa.
\item El factor común mayor de $6x^{2}y^{3}-12x^{3}y^{2}+18x^{4}y$ es $2x^{2}y$
\item Si el producto de $x$ y $y$ es cero, entonces $x$ es cero y/o $y$ es cero.
\item El factor común siempre es un monomio
\item Si la factorización de un polinomio puede ser factorizada nuevamente, entonces el polinomio no está completamente factorizado
\item El polinomio factorizado, $3a(2a^{2}+4)$, está completamente factorizado.
\item Las soluciones de la ecuación $x(x+2)=7$ son 7 y 5
\item El conjunto solución para $x^{2}=7x$ es 7
\end{enumerate}
\subsection*{Ejercicios}
Para los ejercicios 1--10, clasifique cada número como primo o compuesto
\begin{enumerate}
\begin{multicols}{3}
\item 63
\item 81
\item 59
\item 63
\item 51
\item 69
\item 91
\item 119
\item 71
\item 101
\end{multicols}
Para los problemas 11--20, factorice cada número compuesto como producto de números primos. Por ejemplo, $30=2\cdot 3 \cdot 5$
\begin{multicols}{3}
\item 28
\item 39
\item 44
\item 49
\item 56
\item 64
\item 72
\item 84
\item 87
\item 91
\end{multicols}
\end{enumerate}
\end{document}
