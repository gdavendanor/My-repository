\documentclass[10pt,twoside]{article}
\usepackage[utf8]{inputenc}
\usepackage{amsmath}
\usepackage{amsfonts}
\usepackage{amssymb}
\usepackage[spanish,es-noshorthands]{babel}
\usepackage[T1]{fontenc}
\usepackage{lmodern}
\usepackage{graphicx,hyperref}
\usepackage{tikz,pgf}
\usepackage{multicol}
\usepackage{subfig}
\usepackage[papersize={6.5in,8.5in},width=5.5in,height=7in]{geometry}
\usepackage{fancyhdr}
\pagestyle{fancy}
\fancyhead[LE]{\includegraphics[height=12pt]{Images/logo-colegio.png} Álgebra $8^{\circ}$}
\fancyhead[RE]{}
\fancyhead[RO]{\textit{Germ\'an Avenda\~no Ram\'irez, Lic. U.D., M.Sc. U.N.}}
\fancyhead[LO]{}

\author{Germ\'an Avenda\~no Ram\'irez, Lic. U.D., M.Sc. U.N.}
\title{\begin{minipage}{.2\textwidth}
\includegraphics[height=1.75cm]{Images/logo-colegio.png}\end{minipage}
\begin{minipage}{.55\textwidth}
\begin{center}
Taller 07, Factorización  \\
Álgebra $8^{\circ}$
\end{center}
\end{minipage}\hfill
\begin{minipage}{.2\textwidth}
\includegraphics[height=1.75cm]{Images/logo-sed.png} 
\end{minipage}}
\date{}
\begin{document}
\maketitle
Nombre: \hrulefill Curso: \underline{\hspace*{44pt}} Fecha: \underline{\hspace*{2.5cm}}
\section*{Taller}
\subsection*{Quiz de conceptos}
Para los problemas 1--10, conteste V o F
\begin{enumerate}
\item La factorizaci\'{o}n es el proceso inverso a la multiplicaci\'{o}n.
\item La propiedad distributiva de la forma $ab+ac=a(b+c)$ es aplicada para factorizar polinomios
\item Un polinomio puede ser factorizado de m\'{u}ltiples formas, pero solo una es la completa.
\item El factor común mayor de $6x^{2}y^{3}-12x^{3}y^{2}+18x^{4}y$ es $2x^{2}y$
\item Si el producto de $x$ y $y$ es cero, entonces $x$ es cero y/o $y$ es cero.
\item El factor común siempre es un monomio
\item Si la factorización de un polinomio puede ser factorizada nuevamente, entonces el polinomio no está completamente factorizado
\item El polinomio factorizado, $3a(2a^{2}+4)$, está completamente factorizado.
\item Las soluciones de la ecuación $x(x+2)=7$ son 7 y 5
\item El conjunto solución para $x^{2}=7x$ es 7
\end{enumerate}
\subsection*{Ejercicios}
Para los ejercicios 1--10, clasifique cada número como primo o compuesto
\begin{enumerate}
\begin{multicols}{5}
\item 63
\item 81
\item 59
\item 63
\item 51
\item 69
\item 91
\item 119
\item 71
\item 101
\end{multicols}
Para los problemas 11--20, factorice cada número compuesto como producto de números primos. Por ejemplo, $30=2\cdot 3 \cdot 5$
\begin{multicols}{5}
\item 28
\item 39
\item 44
\item 49
\item 56
\item 64
\item 72
\item 84
\item 87
\item 91
\end{multicols}
Para los problemas 21--24, determine si el polinomio está completamente factorizado
\item $6x^{2}+12xy^{2}=2xy(3x+6y)$
\item $2a^{3}b^{2}+4a^{2}b^{2}=4a^{2}b^{2}\left(\frac{1}{2}a+1\right)$
\item $10m^{2}n^{3}+15m^{4}n^{2}=5m^{2}n(2n^{2}+3m^{2}n)$
\item $24ab+12bc-18bd=6b(4a+2c-3d)$

Para los ejercicios 25--37, factorice completamente

\begin{multicols}{2}
\item $12x+8y$
\item $15x^{2}+6x$
\item $42y^{2}-6y$
\item $27xy-36y$
\item $12x^{3}-10x^{2}$
\item $24a^{3}b^{2}+36a^{2}b$
\item $15x^{4}y^{2}-45x^{5}y^{4}$
\item $6x^{5}-18x^{3}+24x$
\item $9x^{2}-17x^{4}+21x^{5}$
\item $8x^{5}y^{3}-6x^{4}y^{5}+12x^{2}y^{3}$
\item $x(y-1)+5(y-1)$
\item $5x(a-b)+y(a-b)$
\item $x(x-1)-3(x-1)$
\end{multicols}
Para los ejercicios 38--46 , factorice por agrupación de términos
\begin{multicols}{2}
\item $ax-2x+ay-2y$
\item $2ax-bx+2ay-by$
\item $5ax-5bx-2ay+2by$
\item $3bx+3x+by+y$
\item $ax^{2}-2x^{2}+3a-6$
\item $2bx+cy+cx+2by$
\item $2a^{2}-3bc-2ab+3ac$
\item $x^{2}-2x+5x-10$
\item $3x^{2}+18x-2x-12$
\end{multicols}
Para los ejercicios \ref{ej-pri6}--\ref{ej-ult6}, resuelva cada ecuación
\begin{multicols}{2}
\item $x^{2}+9x=0$\label{ej-pri6}
\item $x^{2}-14x=0$
\item $b^{2}=-7b$
\item $-6x=2x^{2}$
\item $-4x^{2}+9x=0$
\item $3x=11x^{2}$
\item $x-6x^{2}=0$
\item $-5a=-a^{2}$\label{ej-ult6}
\end{multicols}
Para los ejercicios \ref{ej-pri7}--\ref{ej-ult7}, solucione cada ecuación para la variable indicada
\begin{multicols}{2}
\item $ax^{2}+bx=0$ para $x$ \label{ej-pri7}
\item $3ay^{2}=by$ para $y$
\item $y^{2}-ay+2by-2ab=0$ para $y$
\item $x^{2}+ax+bx+ab=0$ para $x$ \label{ej-ult7}
\end{multicols}
Para los problemas \ref{ej-pri8}--\ref{ej-ult8}, plantee la ecuaci\'{o}n y solucione el problema
\item Suponga que el \'{a}rea de un cuadrado es seis veces su per\'{i}metro. Encuentre la longitud del lado del cuadrado
\item Encuentre la longitud del radio de un círculo cuya circunferencia es numéricamente igual a su área.
\item Encuentre la longitud del radio de un esfera cuya superficie es numéricamente igual a su volumen. (Recuerde que la superficie de la esfera es $S_{s}=4\pi r^{2}$ y su volumen es $V_{s}=\frac{4}{3}\pi r^{3}$)
\end{enumerate}
\end{document}
