\documentclass[letterpaper,twoside]{article}
\usepackage[utf8]{inputenc}
\usepackage{amsmath,amsfonts,amssymb,amsthm,latexsym}
\usepackage[spanish,es-noshorthands]{babel}
\usepackage[T1]{fontenc}
\usepackage{lmodern}
\usepackage{graphicx,hyperref}
\usepackage{tikz,pgf}
\usepackage{marvosym}
\usepackage{multicol}
\usepackage{fancyhdr}
\usepackage[papersize={5.5in,8.5in},height=7.75in,width=4.75in]{geometry}
\usepackage{fancyhdr}
\pagestyle{fancy}
\fancyhead[LE]{\Email matematicas.german@gmail.com}
\fancyhead[RE]{\url{https://www.autistici.org/mathgerman}}
\fancyhead[RO]{\url{https://www.autistici.org/mathgerman}}
\fancyhead[LO]{\Email matematicas.german@gmail.com}

\author{Germ\'an Avenda\~no Ram\'irez~\thanks{Lic. Mat. U.D., M.Sc. U.N.}}
\title{\begin{minipage}{.2\textwidth}
\includegraphics[height=1.75cm]{Images/logo-colegio.png}\end{minipage}
\begin{minipage}{.55\textwidth}
\begin{center}
Votaciones\\
Estadística $9^{\circ}$
\end{center}
\end{minipage}\hfill
\begin{minipage}{.2\textwidth}
\includegraphics[height=1.75cm]{Images/logo-sed.png} 
\end{minipage}}
\date{}
\thispagestyle{plain}
\begin{document}
\maketitle
Nombre: \hrulefill Curso: \underline{\hspace*{44pt}} Fecha: \underline{\hspace*{2.5cm}}
\section*{Elecciones}
Conteste en su cuaderno
\begin{enumerate}
\item Si en las pasadas elecciones en Bogotá realizadas el 25 de octubre del año 2015, 2'811\,209 personas votaron para elegir alcalde, de un total de 5'453\,086 personas habilitadas para votar. Los votos válidos fueron 2'729\,902. Calcule el porcentaje de votación y a su vez el porcentaje de abstención (los que no votaron).
\item Si el total de votos en blanco fue de 99\,359, 24\,037 votos no marcados y 57\,270 votos nulos para alcalde,  de 2'729\,902 votos válidos, calcule el porcentaje de los votos en blanco, no marcados y votos nulos.
\item Complete la siguiente tabla
\begin{center}
\begin{tabular}{|l|r|p{1.9cm}|p{1.9cm}|}
\hline 
Candidato & Votos & \% del total de votos válidos & \% del total de votantes habilitados\\ 
\hline 
Enrique Peñalosa & 903\,764 & &\\ 
\hline 
Rafael Pardo & 778\,05 & &\\ 
\hline 
Clara López & 498\,718 & &\\ 
\hline 
Francisco Santos & 327\,852 & &\\ 
\hline 
Ricardo Arias & 91\,082 & &\\ 
\hline 
Daniel Raisbeck & 20\,537 & &\\ 
\hline 
Alexandre Vernot & 7\,279 & &\\ 
\hline 
Retirado & 2\,595 & &\\ 
\hline 
Retirado & 666 & &\\ 
\hline 
\end{tabular} 
\end{center}
\item Cree que en una democracia verdadera, un alcalde pueda ser elegido con ese porcentaje de votación respecto al total de votantes habilitados? Podría ser legítima su administración? ¿Realmente se está representando a las mayorías?
\end{enumerate}
\end{document}
