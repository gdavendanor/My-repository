\documentclass[10pt,letterpaper]{article}
\usepackage[utf8]{inputenc}
\usepackage[spanish]{babel}
\usepackage{graphicx}
\usepackage{lmodern}
\author{Germán Avendaño Ramírez~\thanks{Delegado Sindical, Colegio Arborizadora Baja j.m.}}
\title{Informe Día E en el marco de la Jornada de Desobediencia Civil convocada por Fecode}
\begin{document}
\maketitle
Como bien es sabido, el 25 de marzo, sin consultar a los maestros y como una medida reaccionaria frente a la posibilidad de un Paro Nacional Indefinido, el MEN decretó el denominado día E (Día de la excelencia).\\

Este día en el colegio se desarrolló en la primera parte de la jornada una ligera mirada al material que enviaron para el desarrollo de la actividad.\\

En la segunda parte de la jornada, sobre las 10:00 a.m. se desarrolla parte de la actividad propuesta por FECODE en el marco de la \emph{Jornada de Desobediencia Civil}.\\

Debido al tiempo, no se puede cumplir con la primera actividad que consistía en ver el documental \emph{Granito de Arena}, que versa sobre la educación en México. Se empieza por hacer la lectura de los documentos recomendados para tal fin.\\

Se trabaja sobre el documento del profesor Alejandro Álvarez Gallego\footnote{Rector del Instituto Pedagógico Nacional I.P.N. y profesor de la Universidad Pedagógica Nacional}, \emph{A propósito del día E convocado por el MEN}, donde se hace un análisis de esta medida adoptada por el MEN. El análisis se centra en los siguientes aspectos:
\begin{itemize}
\item \textbf{Medida inconsulta:} El MEN no tiene en cuenta a la comunidad educativa, ni al magisterio para tomar esta medida. El propósito del día E es establecer un Plan de Mejoramiento Mínimo Anual (P.M.M.A) con el fin de mejorar cada uno de los indicadores del \emph{Índice Sintético de Calidad Educativa ISCE}, que no se sabe de dónde sale este índice y quién o quiénes dicen que este índice mide realmente la calidad de la educación. Al respecto no hay ningún estudio serio que lo avale. Este \emph{índice} se compone de los siguientes indicadores:
\begin{itemize}
\item \emph{Progreso:} Mide cuánto mejora año tras año el resultado de las Pruebas Saber de lenguaje y matemáticas en $3^{\circ}$, $5^{\circ}$ y $9^{\circ}$ grados.
\item \emph{Desempeño:} Mide el promedio institucional en las mismas pruebas, comparado con el promedio nacional y la entidad territorial.
\item \emph{Eficiencia:} Mide la tasa de promoción de los estudiantes de un grado a otro.
\item \emph{Ambiente escolar:} Mide el seguimiento que los maestros hacen a las tareas de los estudiantes y el ambiente del aula, esto según encuestas aplicadas a los estudiantes en el momento de contestar las pruebas saber.
\end{itemize}
Se considera que el MEN se extralimita en sus funciones pues sus funciones deberían estar centradas en garantizar el acceso real y efectivo de los niños y niñas a la educación y su permanencia además de incidir sobre los factores que favorecen la calidad como la promoción docente, los recursos educativos, la innovación e investigación educativa, la orientación educativa y profesional, la inspección y evaluación del proceso educativo. Todo esto es lo que realmente falta y todo esto es lo que debería hacer el Estado  y lo que ha dejado de hacer durante décadas.
\item \textbf{Una medida Ingenua}: Creer que con base en los resultados de las pruebas saber se está midiendo la calidad de la educación (ya que dos de los cuatro componentes del ISCE se basan en estos resultados) es no asumir el problema de la calidad seriamente. Además con un P.M.M.A. no se solucionarían los problemas más sentidos de la educación como son la procedencia socio-económica de los estudiantes y el número de estudiantes por aula, aspectos sobre los cuales no pueden incidir directamente los maestros y maestras y sí es responsabilidad directa del Estado.
\item \textbf{Una medida insuficiente:} Se debiera tener en cuenta lo que hacemos los maestros y maestras a diario. Para nada se habla aquí del PEI, carta de navegación de las instituciones educativas. Se debiera centrar la estrategia si realmente se quiere incidir en el mejoramiento de la calidad educativa en el mejoramiento de la infraestructura, las posibilidades de acceso, de asequibilidad, de aceptabilidad, y de adaptabilidad (las 4 A) para garantizar plenamente el derecho a la educación.
\item \textbf{Una medida infantilizante:} Creer que el problema de la calidad de la educación se puede superar con buen ánimo, actitud positiva, sacrificio y esfuerzo como en un campeonato de fútbol, es desconocer la complejidad de la educación. Todas estas características se deben tener en cuenta pero no son las fundamentales.
\end{itemize}
Para finalizar se dejan abiertas algunas preguntas planteadas por el profesor Álvarez:
\begin{itemize}
\item ¿Están seguros que la calidad de la educación es equiparable a los resultados de aprendizaje medidos en las pruebas saber?
\item ¿Están seguros que el ISCE es suficiente para determinar el grado en que se encuentra la calidad educativa en cada colegio?
\item ¿Creen de verdad que con el P:M.M.A. se va a mejorar ese índice?
\item ¿Qué hacen los directivos, los maestros y la comunidad educativa con sus planes de desarrollo institucional?, ¿los reducen a este  P:M.M.A? ¿sobran? Porque en realidad este plan que surge de este día E, es mucho más sencillo de elaborar y de hacerle seguimiento; lo que no es seguro es que mejoren los indicadores, por la sencilla razón de que el ISCE no tiene en cuenta los factores más determinantes en los procesos de enseñanza-aprendizaje.
\end{itemize}
También se hace lectura rápida del documento \emph{DEL DÍA E AL DÍA I DE INDIGNACIÓN - DESOBEDIENCIA CIVIL A UNA NEFASTA POLÍTICA}\footnote{CEID Fecode}
\end{document}