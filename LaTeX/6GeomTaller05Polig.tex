\documentclass[10pt,twoside]{article}
\usepackage[utf8]{inputenc}
\usepackage{amsmath}
\usepackage{amsfonts}
\usepackage{amssymb}
\usepackage[spanish,es-noshorthands]{babel}
\usepackage[T1]{fontenc}
\usepackage{lmodern}
\usepackage{graphicx,hyperref}
\usepackage{tikz,pgf}
\usepackage{multicol}
\usepackage[papersize={6.5in,8.5in},width=5.5in,height=7in]{geometry}
\usepackage{fancyhdr}
\pagestyle{fancy}
\fancyhead[LE]{\includegraphics[height=12pt]{Images/logo-colegio.png} Geometría $6^{\circ}$}
\fancyhead[RE]{}
\fancyhead[RO]{\textit{Germ\'an Avenda\~no Ram\'irez, Lic. U.D., M.Sc. U.N.}}
\fancyhead[LO]{}

\author{Germ\'an Avenda\~no Ram\'irez, Lic. U.D., M.Sc. U.N.}
\title{\begin{minipage}{.2\textwidth}
\includegraphics[height=1.75cm]{Images/logo-colegio.png}\end{minipage}
\begin{minipage}{.55\textwidth}
\begin{center}
Taller 05, Polígonos \\
Geometría $6^{\circ}$
\end{center}
\end{minipage}\hfill
\begin{minipage}{.2\textwidth}
\includegraphics[height=1.75cm]{Images/logo-sed.png} 
\end{minipage}}
\date{}
\begin{document}
\maketitle
Nombre: \hrulefill Curso: \underline{\hspace*{44pt}} Fecha: \underline{\hspace*{2.5cm}}
\section*{Evaluaci\'on}
\begin{itemize}
\item Observa el grupo de figuras y resuelve.
\end{itemize}
%Uncomment next line if XeTeX is used
%\def\pgfsysdriver{pgfsys-xetex.def}
\usetikzlibrary{arrows}
\baselineskip=10pt
\hsize=6.3truein
\vsize=8.7truein
\definecolor{zzttqq}{rgb}{0.27,0.27,0.27}
\tikzpicture[line cap=round,line join=round,>=triangle 45,x=1.0cm,y=1.0cm,scale=.8]
\clip(-3.45,-1.38) rectangle (12.24,3.39);
\fill[color=zzttqq,fill=zzttqq,fill opacity=0.1] (-2,-1) -- (-1,-1) -- (-0.16,-0.46) -- (0.26,0.45) -- (0.11,1.44) -- (-0.54,2.2) -- (-1.5,2.48) -- (-2.46,2.2) -- (-3.11,1.44) -- (-3.26,0.45) -- (-2.84,-0.46) -- cycle;
\fill[color=zzttqq,fill=zzttqq,fill opacity=0.1] (2,-1) -- (4,-1) -- (5,0.73) -- (4,2.46) -- (2,2.46) -- (1,0.73) -- cycle;
\draw [color=zzttqq] (-2,-1)-- (-1,-1);
\draw [color=zzttqq] (-1,-1)-- (-0.16,-0.46);
\draw [color=zzttqq] (-0.16,-0.46)-- (0.26,0.45);
\draw [color=zzttqq] (0.26,0.45)-- (0.11,1.44);
\draw [color=zzttqq] (0.11,1.44)-- (-0.54,2.2);
\draw [color=zzttqq] (-0.54,2.2)-- (-1.5,2.48);
\draw [color=zzttqq] (-1.5,2.48)-- (-2.46,2.2);
\draw [color=zzttqq] (-2.46,2.2)-- (-3.11,1.44);
\draw [color=zzttqq] (-3.11,1.44)-- (-3.26,0.45);
\draw [color=zzttqq] (-3.26,0.45)-- (-2.84,-0.46);
\draw [color=zzttqq] (-2.84,-0.46)-- (-2,-1);
\draw [color=zzttqq] (2,-1)-- (4,-1);
\draw [color=zzttqq] (4,-1)-- (5,0.73);
\draw [color=zzttqq] (5,0.73)-- (4,2.46);
\draw [color=zzttqq] (4,2.46)-- (2,2.46);
\draw [color=zzttqq] (2,2.46)-- (1,0.73);
\draw [color=zzttqq] (1,0.73)-- (2,-1);
\draw (6,3)-- (8,2);
\draw (5.5,1.98)-- (7,-1);
\draw (7,-1)-- (8,2);
\draw (9.87,2.74)-- (8.76,1.87);
\draw (10.35,2.71)-- (11.7,2.35);
\draw (11.7,2.35)-- (11.97,1.36);
\draw (11.97,1.36)-- (11.01,-0.77);
\draw (11.01,-0.77)-- (9.06,-0.8);
\draw (9.06,-0.8)-- (8.76,1.87);
\endtikzpicture
\begin{itemize}
\item[a.] Identifica la figura que no sea polígono.
\item[b.] Dibuja dos de los polígonos en tu cuaderno y señala con un
color los lados y con otro los ángulos.
\item[c.] Dibujar dos polígonos y dos líneas poligonales abiertas
\end{itemize}
\subsection*{Ejercito lo aprendido}
\subsubsection*{Marcapáginas en origammi}
Consigue un cuadrado de papel de diez centímetros de lado.\\
Sigue los pasos y responde en cada caso.

\begin{minipage}{.35\textwidth}
\begin{tikzpicture}
\filldraw[gray!35] (0,0)node[black,below]{B}--(2,2)--(0,4)node[left,black]{A}--(-2,2)--cycle;
\draw[gray!50] (0,0)--(0,4);
\draw[gray!50](-2,2)--(2,2);
\node[left] at (0,2){O};
\end{tikzpicture}
\end{minipage}
\begin{minipage}{.6\textwidth}
Dobla el cuadrado por cada una de sus diagonales.\\
Ábrelo nuevamente,
\begin{itemize}
\item ¿Qué tipo de líneas se formaron?
\item >> ¿Qué tipo de relación guardan entre ellas?
\end{itemize}
Marca el vértice superior con la letra A y el inferior con la
letra B.\\
Marca el punto de corte con la letra O.
\end{minipage}

\begin{minipage}{.6\textwidth}
Lleva los vértices A y B hasta el punto central O.
\begin{itemize}
\item ¿Qué sucede con $\overleftrightarrow{AB}$ ¿En cuántas partes queda dividido?
\item ¿Qué clase de figura se forma al dejar doblado el  cuadrado?
\end{itemize}
\end{minipage}
\begin{minipage}{.35\textwidth}
\begin{tikzpicture}
\filldraw[gray!35] (-2,2)node[left,black]{D}--(-1,3)--(1,3)--(2,2)node[right,black]{E}--(1,1)--(-1,1)--cycle;
\draw[gray!50](-2,2)--(0,2)--(-1,3)--(0,2)node[black,left]{O}--(1,3)--(0,2)--(2,2)--(0,2)-- (1,1)--(0,2)--(-1,1);
\end{tikzpicture}
\end{minipage}

\begin{minipage}{.35\textwidth}
\begin{tikzpicture}
\filldraw[gray!35](-2,2)node[above,black]{D}--(0,2)node[above,black]{O}--(2,2)node[above,black]{E}--(1,1)node[below,black]{F}--(-1,1)node[below,black]{G}--cycle;
\draw[gray!50](0,1)--(0,2);
\end{tikzpicture}
\end{minipage}
\begin{minipage}{.6\textwidth}
\begin{itemize}
\item Dobla el hexágono por el segmento $\overline{DC}$, hacia abajo.
	 ¿Qué forma tiene el papel doblado en este paso?
\item Marca los vértices laterales superiores
\end{itemize}
\end{minipage}

\begin{minipage}{.35\textwidth}
\begin{tikzpicture}
\filldraw[gray!50](-1,1)node[left,black]{G}--(1,1)node[right,black]{F}--(0,2)node[above,black]{O}--cycle;
\filldraw[gray!40](-1,1)--(-.1,0)node[below left,black]{D}--(0,2)--cycle;
\filldraw[gray!40](1,1)--(.1,0)node[below right,black]{E}--(0,2)--cycle;
\end{tikzpicture}
\end{minipage}
\begin{minipage}{.6\textwidth}
\begin{itemize}
\item Dobla hacia adelante para formar los segmentos $\overline{OF}$
y $\overline{OG}$.
\end{itemize}
\end{minipage}

\begin{minipage}{.6\textwidth}
\begin{itemize}
\item Dobla las puntas dentro de un bolsillito que se formó.\\
	 ¿Qué clase de figura se formó?\\
¿Cuánto miden sus ángulos?
\end{itemize}
\end{minipage}
\begin{minipage}{.6\textwidth}
\begin{tikzpicture}
\filldraw[gray!50](-1,1)--(0,2)--(1,1)--cycle;
\end{tikzpicture}
\end{minipage}

\begin{minipage}{.35\textwidth}
\begin{tikzpicture}
\filldraw[gray!20](0,1) rectangle(2,4);
\filldraw[gray!50](1.2,4)--(2,3.2)--(2,4)-- cycle;
\end{tikzpicture}
\end{minipage}
\begin{minipage}{.6\textwidth}
Utiliza la figura que acabaste de hacer para marcar la
página que te interesa de un libro.
\begin{itemize}
\item ¿Qué clase de ángulo se forma en la parte superior del triángulo? 
\item ¿Qué clase de figura se formó?\\
¿Cuánto miden sus ángulos?
\end{itemize}
\end{minipage}

\begin{minipage}{.6\textwidth}
¿Cómo te sentiste al desarrollar la actividad propuesta?\\
¿Te sirvieron los temas que aprendiste en esta guía?
\begin{itemize}
\item Observa la figura. Escribe falso o verdadero según sea el caso.
es paralela a
\end{itemize}
\begin{enumerate}
\item[a.] $\overleftrightarrow{AB}$ es paralela a $\overleftrightarrow{DE}$ (\quad)
\item[b.] La recta $ \overleftrightarrow{FC}$ pasa también por el punto O.(\quad)	
\item[c.] El ángulo $\angle BOD$ es recto.(\quad )
\item[d.] El ángulo $\angle FOA$ es agudo.(\quad )
\item[e.] El polígono tiene forma de pentágono.(\quad)	
\end{enumerate}
\end{minipage}
\begin{minipage}{.35\textwidth}
\begin{tikzpicture}
\filldraw[gray!50](-1,1)node[left,black]{A}--(0,2)node[above,black]{B}--(1,1)node[right,black]{C}--(1,-1)node[black,right]{D}--(0,-2)node[below,black]{E}--(-1,-1)node[below left,black]{F}--cycle;
\draw[gray!25](0,-2)--(0,2);
\draw[gray!25](-1,0)--(1,0);
\draw[gray!75](-1,1)--(1,-1);
\draw[gray!75](-1,-1)--(0,0)node[right,black]{O};
\draw[gray!75](-.1,.1)--(1,1);
\end{tikzpicture}
\end{minipage}
\section*{Evaluaci\'on}
\subsection*{Qu\'{e} aprend\'{i}}
\begin{enumerate}
\item Comienza a trazar imaginariamente líneas (horizontales,
verticales o diagonales) en el dibujo de modo que esto te
permita realizar las siguientes actividades.
\begin{enumerate}
\item Identifica un punto.
\item Identifica un segmento.
\item Señala un ángulo de cada tipo (recto, agudo, obtuso).
\end{enumerate}
\item Coloca los números de los cuadriláteros en la casilla que
correspondan de acuerdo a la característica señalada. Tenga en cuenta las figuras de abajo.
\begin{center}
\begin{tabular}{|l|c|}
\hline 
\hspace*{40pt} Características & Cuadriláteros \\ 
\hline 
Dos pares de lados paralelos &  \\ 
\hline 
Dos lados paralelos y dos no  &  \\ 
\hline 
Ningún lado paralelo &  \\ 
\hline 
Tiene al menos un ángulo obtuso &  \\ 
\hline 
Tiene todos sus ángulos rectos &  \\ 
\hline 
\end{tabular} 
\end{center}
Observa el grupo de figuras y completa la tabla anterior
\begin{center}
\includegraphics[scale=.65]{Images/Figuras.png} 
\end{center}
\subsection{Cómo me ven los demás}
Formen grupos de dos a tres personas
\begin{enumerate}
\item Investiguen cómo elaborar alguna figura en papel.Y
practiquénla.
\item Enséñenles a sus compañeros a realizarla aprove-
chando los temas que se trabajaron en esta guía.
\item Evalúen entre todos el trabajo que realizan cada uno
de los grupos.
\end{enumerate}
\subsection{Me autoeval\'{u}o}
\item Responde según la manera en la que te desenvolviste en
el desarrollo del módulo.
\end{enumerate}
\begin{center}
\begin{tabular}{|p{9cm}|c|c|c|}
\hline 
 & Sí & A veces & No \\ 
\hline 
Identifica los conceptos básicos de la
Geometría. &  &  &  \\ 
\hline 
Reconoce las características, las clases, las
relaciones y las propiedades de los ángulos. &  &  &  \\ 
\hline 
Reconoce las posiciones relativas de las rectas
en el plano. &  &  &  \\ 
\hline 
Clasifica polígonos a partir de sus
características. &  &  &  \\ 
\hline 
Se interesa por conocer las opiniones de sus
compañeros y presentar con claridad las suyas. &  &  &  \\ 
\hline 
Se preocupa por preparar sus trabajos y
exposiciones. &  &  &  \\ 
\hline 
Acepta sus errores o dificultades y trata de
superarlos. &  &  &  \\ 
\hline 
\end{tabular} 

\end{center}\end{document}
