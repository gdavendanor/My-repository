\documentclass[10pt,twoside]{article}
\usepackage[utf8]{inputenc}
\usepackage{amsmath}
\usepackage{amsfonts}
\usepackage{amssymb}
\usepackage[spanish,es-noshorthands]{babel}
\usepackage[T1]{fontenc}
\usepackage{lmodern}
\usepackage{graphicx,hyperref}
\usepackage{tikz,pgf}
\usepackage{multicol}
\usepackage[papersize={6.5in,8.5in},width=5.5in,height=7in]{geometry}
\usepackage{fancyhdr}
\pagestyle{fancy}
\fancyhead[LE]{\includegraphics[height=12pt]{Images/logo-colegio.png} Geometría $6^{\circ}$}
\fancyhead[RE]{}
\fancyhead[RO]{\textit{Germ\'an Avenda\~no Ram\'irez, Lic. U.D., M.Sc. U.N.}}
\fancyhead[LO]{}

\author{Germ\'an Avenda\~no Ram\'irez, Lic. U.D., M.Sc. U.N.}
\title{\begin{minipage}{.2\textwidth}
\includegraphics[height=1.75cm]{Images/logo-colegio.png}\end{minipage}
\begin{minipage}{.55\textwidth}
\begin{center}
Taller 05, Polígonos \\
Geometría $6^{\circ}$
\end{center}
\end{minipage}\hfill
\begin{minipage}{.2\textwidth}
\includegraphics[height=1.75cm]{Images/logo-sed.png} 
\end{minipage}}
\date{}
\begin{document}
\maketitle
Nombre: \hrulefill Curso: \underline{\hspace*{44pt}} Fecha: \underline{\hspace*{2.5cm}}
\section*{Evaluaci\'on}
\begin{itemize}
\item Observa el grupo de figuras y resuelve.
\end{itemize}
%Uncomment next line if XeTeX is used
%\def\pgfsysdriver{pgfsys-xetex.def}
\usetikzlibrary{arrows}
\baselineskip=10pt
\hsize=6.3truein
\vsize=8.7truein
\definecolor{zzttqq}{rgb}{0.27,0.27,0.27}
\tikzpicture[line cap=round,line join=round,>=triangle 45,x=1.0cm,y=1.0cm,scale=.8]
\clip(-3.45,-1.38) rectangle (12.24,3.39);
\fill[color=zzttqq,fill=zzttqq,fill opacity=0.1] (-2,-1) -- (-1,-1) -- (-0.16,-0.46) -- (0.26,0.45) -- (0.11,1.44) -- (-0.54,2.2) -- (-1.5,2.48) -- (-2.46,2.2) -- (-3.11,1.44) -- (-3.26,0.45) -- (-2.84,-0.46) -- cycle;
\fill[color=zzttqq,fill=zzttqq,fill opacity=0.1] (2,-1) -- (4,-1) -- (5,0.73) -- (4,2.46) -- (2,2.46) -- (1,0.73) -- cycle;
\draw [color=zzttqq] (-2,-1)-- (-1,-1);
\draw [color=zzttqq] (-1,-1)-- (-0.16,-0.46);
\draw [color=zzttqq] (-0.16,-0.46)-- (0.26,0.45);
\draw [color=zzttqq] (0.26,0.45)-- (0.11,1.44);
\draw [color=zzttqq] (0.11,1.44)-- (-0.54,2.2);
\draw [color=zzttqq] (-0.54,2.2)-- (-1.5,2.48);
\draw [color=zzttqq] (-1.5,2.48)-- (-2.46,2.2);
\draw [color=zzttqq] (-2.46,2.2)-- (-3.11,1.44);
\draw [color=zzttqq] (-3.11,1.44)-- (-3.26,0.45);
\draw [color=zzttqq] (-3.26,0.45)-- (-2.84,-0.46);
\draw [color=zzttqq] (-2.84,-0.46)-- (-2,-1);
\draw [color=zzttqq] (2,-1)-- (4,-1);
\draw [color=zzttqq] (4,-1)-- (5,0.73);
\draw [color=zzttqq] (5,0.73)-- (4,2.46);
\draw [color=zzttqq] (4,2.46)-- (2,2.46);
\draw [color=zzttqq] (2,2.46)-- (1,0.73);
\draw [color=zzttqq] (1,0.73)-- (2,-1);
\draw (6,3)-- (8,2);
\draw (5.5,1.98)-- (7,-1);
\draw (7,-1)-- (8,2);
\draw (9.87,2.74)-- (8.76,1.87);
\draw (10.35,2.71)-- (11.7,2.35);
\draw (11.7,2.35)-- (11.97,1.36);
\draw (11.97,1.36)-- (11.01,-0.77);
\draw (11.01,-0.77)-- (9.06,-0.8);
\draw (9.06,-0.8)-- (8.76,1.87);
\endtikzpicture
\begin{itemize}
\item[a.] Identifica la figura que no sea polígono.
\item[b.] Dibuja dos de los polígonos en tu cuaderno y señala con un
color los lados y con otro los ángulos.
\item[c.] Dibujar dos polígonos y dos líneas poligonales abiertas
\end{itemize}
\subsection*{Ejercito lo aprendido}
\subsubsection*{Marcapáginas en origammi}
Consigue un cuadrado de papel de diez centímetros de lado.
Sigue los pasos y responde en cada caso.
\begin{minipage}{.45\textwidth}
\begin{tikzpicture}
\filldraw[gray!15] (0,0)--(2,2)--(0,4)--(-2,2)--cycle;
\draw[gray!30] (0,0)--(0,4);
\end{tikzpicture}
\end{minipage}
\begin{minipage}{.5\textwidth}
Dobla el cuadrado por cada una de sus diagonales.\\
Ábrelo nuevamente,
\begin{itemize}
\item ¿Qué tipo de líneas se formaron?
\item >> ¿Qué tipo de relación guardan entre ellas?
\end{itemize}
Marca el vértice superior con la letra A y el inferior con la
letra B.\\
Marca el punto de corte con la letra O.
\end{minipage}
\end{document}
