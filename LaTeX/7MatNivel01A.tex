\documentclass[letterpaper,fleqn]{article}
\usepackage[spanish,es-noshorthands]{babel}
\usepackage[utf8]{inputenc} 
\usepackage[papersize={6.5in,8.5in},left=1cm, right=1cm, top=1.5cm, bottom=1.7cm]{geometry}
\usepackage{mathexam}
\usepackage{amsmath}
\usepackage{graphicx}
\usepackage{multicol}
\usepackage{textcomp}

\ExamClass{\includegraphics[height=16pt]{Images/logo-sed.png} Matematicas $7^{\circ}$}
\ExamName{Nivelación 1}
\ExamHead{\includegraphics[height=16pt]{Images/logo-colegio.png} IEDAB}
\newcommand{\LineaNombre}{%
\par
\vspace{\baselineskip}
Nombre:\hrulefill \; Curso: \underline{\hspace*{24pt}} \; Fecha: \underline{\hspace*{2cm}} \relax
\par}
\let\ds\displaystyle

\begin{document}
\ExamInstrBox{
Respuesta sin justificar mediante procedimiento no será tenida en cuenta en la calificación. Escriba sus respuestas en el espacio indicado. Tiene 45 minutos para contestar esta prueba.}
\LineaNombre
\begin{enumerate}
 \item Escribe el entero que representa  las  siguientes situaciones:
  \begin{enumerate}
  \begin{multicols}{2}
  \item 50 m de altura \rule{1.5cm}{0.5pt}
  \item Debo \$3.275 \rule{1.5cm}{0.5pt}
  \item 13 grados bajo cero \rule{1.5cm}{0.5pt}
  \item 450 años A.C. \rule{1.5cm}{0.5pt}
  \item 2500 años D.C.\rule{1.5cm}{0.5pt}
  \item 28 m de profundidad \rule{1.5cm}{0.5pt}
    \end{multicols}
     \end{enumerate}
\item Escribe el signo $<$ (menor que), $>$ (mayor que) ó $=$ (igual)  según corresponda:
\begin{enumerate}
\begin{multicols}{3}
  \item 3 \rule{0.5cm}{0.5pt} 5
  \item 4 \rule{0.5cm}{0.5pt} 0
  \item 0 \rule{0.5cm}{0.5pt} 12
  \item -2 \rule{0.5cm}{0.5pt} 0
  \item 0 \rule{0.5cm}{0.5pt} -3
  \item -6 \rule{0.5cm}{0.5pt} -2
  \item -8 \rule{0.5cm}{0.5pt} -15
  \item $|-3|$ \rule{0.5cm}{0.5pt} $|3|$
  \item $|4|$ \rule{0.5cm}{0.5pt} $|-4|$
\end{multicols}
\end{enumerate}
\item Ordena de menor a mayor los siguientes conjuntos
\begin{enumerate}
  \item $A = \{10, -10, 15, -2, 0 , 30, -25\}$\hrulefill
  \item $B = \{-3, 5, 0, -12, 15,-18, 20\}$ \hrulefill
\end{enumerate}
  \item Anota el opuesto simétrico de los siguientes enteros:
\begin{enumerate}
\begin{multicols}{3}
\item $10$ opuesto \rule{1.25cm}{0.5pt}
\item $-12$ opuesto \rule{1.25cm}{0.5pt}
\item $0$ opuesto \rule{1.25cm}{0.5pt}
\item $|-3|$ opuesto \rule{1.25cm}{0.5pt}
\item $a$ opuesto \rule{1.25cm}{0.5pt}
\item $-x$ opuesto \rule{1.25cm}{0.5pt} 
\end{multicols}
\end{enumerate}
\item Dadas las siguientes temperaturas de cinco días de la semana registradas en cierta ciudad del Sur de Argentina. Responde:\\
\begin{tabular}{|l|c|c|c|c|c|}
\hline Temperatura & Lunes & Martes & Miércoles & Jueves & Viernes \\
\hline Máxima \textcelsius & 12 & 10 & 18 & 9 & -1\\
\hline Mínima \textcelsius & 0 & -2 & -5 & -8 & -10\\
\hline
\end{tabular}
\begin{enumerate}
  \item ¿Qué día se produjo la menor de las temperaturas mínimas? \hrulefill
  \item ¿Cuál fue la mayor de las temperaturas máximas? \hrulefill
  \item Ordena las temperaturas mínimas de menor a mayor.\noanswer
    \item Ordena las temperaturas máximas de mayor a menor.\noanswer[.2in]
  \item Calcula la diferencia de temperaturas (entre máxima y mínima) los días miércoles y viernes\\
  Miércoles: \hrulefill\\
  Viernes: \hrulefill
\end{enumerate}
\item Resuelve  estos  problemas,  anotando  la  operación  y  la  respuesta:
\begin{enumerate}
  \item Si pierdes 13 láminas en un juego  y 17 láminas en otro. ¿Cuántas láminas has perdido en total?\noanswer[.25in]
  \item Un equipo de fútbol  tiene 8 goles a favor y en otro partido recibió 15 y anotó 5 goles. ¿Cuántas goles tiene  a favor en total?\noanswer[.25in]
  \item Un submarino descendió 36 metros y luego subió 15 metros. ¿A qué profundidad se encuentra?\noanswer[.25in]
\end{enumerate}
\item Una sustancia química que está a 12\textcelsius \, bajo cero se calienta en un mechero hasta que alcanza una temperatura de 25\textcelsius \, sobre cero. ¿Cuántos grados subió? \noanswer
\item ¿Cuántos años transcurrieron desde la muerte de Alejandro Magno (año 323 A.C.) hasta el nacimiento de Galileo Galilei (año 1564 D.C.)? \noanswer
\item Escribe un conjunto de números enteros negativos que sean menores que $-12$ y mayores o iguales que $-20$.\noanswer
\item ¿Cuál es la diferencia de nivel entre un punto que está a 2.500 metros sobre el nivel del mar y otro que está a 145 metros bajo el nivel del mar? \noanswer
 \end{enumerate}
\end{document}
