\documentclass[fleqn]{article}
\usepackage[spanish,es-noshorthands]{babel}
\usepackage[utf8]{inputenc} 
\usepackage[papersize={6.5in,8.5in},left=1cm, right=1cm, top=1.5cm, bottom=1.7cm]{geometry}
\usepackage{mathexam}
\usepackage{amsmath}
\usepackage{graphicx}
\usepackage{tikz,pgf}
\usepackage{multicol}

\ExamClass{\includegraphics[height=16pt]{Images/logo-sed.png} Aritmética $6{\circ}$}
\ExamName{``Sustentación P. Mejoramiento 3''}
\ExamHead{\includegraphics[height=16pt]{Images/logo-colegio.png} IEDAB}
\newcommand{\LineaNombre}{%
\par
\vspace{\baselineskip}
Nombre:\hrulefill \; Curso: \underline{\hspace*{48pt}} \; Fecha: \underline{\hspace*{2.5cm}} \relax
\par}
\let\ds\displaystyle

\begin{document}
\ExamInstrBox{
Respuesta sin justificar mediante procedimiento no será tenida en cuenta en la calificación. Escriba sus respuestas en el espacio indicado. Tiene 45 minutos para contestar esta prueba.}
\LineaNombre
\begin{enumerate}
  \item Complete el dibujo para representar la siguiente operación. ¿Cuántos cubos forman ésta caja?
  
      \begin{minipage}{0.55\textwidth}
      \begin{tikzpicture}
      \draw (0,3,0)--(5,3,0)--(5,3,2)--(0,3,2)--cycle;
      \draw (0,3,2)--(0,0,2)--(5,0,2)--(5,3,2);
      \draw (5,3,0)--(5,0,0)--(5,0,2);
      \draw (0,3,1)--(5,3,1)--(5,0,1);
      \draw (0,1,2)--(1,1,2);
      \draw (1,3,2)--(1,0,2);
      \draw (2,3,2)--(2,0,2);
      \node [below] at (2.5,0,2){5};
      \node [right] at (5,1.5,0){3};
      \node [right] at (5,0,1){2};
      \end{tikzpicture}
    \end{minipage}\hfill
    \begin{minipage}{0.4\textwidth}
     $ 5\times(3\times2) $      
    \end{minipage}
    \item Represente en la recta numérica
    \begin{enumerate}
    \item $4\times 3=$
\begin{center}
    \begin{tikzpicture}[scale=.9]
    \draw[|->] (0,0)--(13,0);
    \foreach \x in {0,1,2,3,4,5,6,7,8,9,10,11,12} \draw[shift={(\x,0)},color=black] (0pt,2pt) -- (0pt,-2pt) node[below] {\footnotesize $\x$};
    \end{tikzpicture}
\end{center}
    \item $4^{2}=$
    \begin{center}
    \begin{tikzpicture}[scale=.7]
    \draw[|->] (0,0)--(17,0);
    \foreach \x in {0,1,2,3,4,5,6,7,8,9,10,11,12,13,14,15,16} \draw[shift={(\x,0)},color=black] (0pt,2pt) -- (0pt,-2pt) node[below] {\footnotesize $\x$};
    \end{tikzpicture}
    \end{center}
    \end{enumerate}
    \item Resuelva cada una de las siguientes operaciones haciendo el procedimiento al frente
    \begin{enumerate}
    \item $8^{3}=$ \noanswer
    \item $\sqrt{324}=$ \noanswer
    \item $\log_{(3)}243=$ \noanswer
    \item $12^{2}=$ \noanswer
    \end{enumerate}
    \newpage
    \item Resuelvo los siguientes ejercicios y justifico la respuesta:
\begin{enumerate}
  \item $ \tikz \draw (0,0) rectangle (1,.4);^2=144 $ \qquad porque  \noanswer
  \item $ \sqrt[4]{81}= $ \tikz \draw (0,0) rectangle (1,.4); \qquad porque  \noanswer
  \item $ \sqrt[4]{625}= $ \tikz \draw (0,0) rectangle (1,.4); \qquad porque  \noanswer
  \item $ \tikz \draw (0,0) rectangle (1,.4);^4=625 $ \qquad porque  \noanswer
  \item $ \sqrt[5]{32}= $ \tikz \draw (0,0) rectangle (1,.4); \qquad porque  \noanswer
  \item $ \sqrt{25}= $ \tikz \draw (0,0) rectangle (1,.4); \qquad porque  \noanswer
   \item $ \tikz \draw (0,0) rectangle (1,.4);^2=169 $ \qquad porque  \noanswer
  \item $ \tikz \draw (0,0) rectangle (1,.4);^4=256 $ \qquad porque  \noanswer
  \item $ \tikz \draw (0,0) rectangle (1,.4);^2=400 $ \qquad porque  \noanswer
\end{enumerate}
\item ¿Cuál es el exponente?
\begin{enumerate}
  \item $ 12^{\tikz \draw (0,0) rectangle (.5,.3);}=12 $ \qquad porque  \noanswer
  \item $8^{\tikz \draw (0,0) rectangle (.5,.3);}=512$ \qquad porque  \noanswer
    \item $ 9^{\tikz \draw (0,0) rectangle (.5,.3);}=81 $ \qquad porque  \noanswer
  \item $ 11^{\tikz \draw (0,0) rectangle (.5,.3);}=1331 $  \qquad porque  \noanswer
\end{enumerate}
 \end{enumerate}

\end{document}
