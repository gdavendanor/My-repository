\documentclass[10pt,twoside]{article}
\usepackage[utf8]{inputenc}
\usepackage{amsmath,amsfonts,amssymb,amsthm,latexsym}
\usepackage[spanish,es-noshorthands]{babel}
\usepackage[T1]{fontenc}
\usepackage{lmodern}
\usepackage{graphicx,hyperref}
\usepackage{tikz}
\usepackage{multicol}
\usepackage{subfig}
\usepackage[papersize={6.5in,8.5in},width=5.5in,height=7in]{geometry}
\usepackage{fancyhdr}
\pagestyle{fancy}
\fancyhead[LE]{\url{http://germandario.byethost4.com}}
\fancyhead[RE]{}
\fancyhead[RO]{\textit{Germ\'an Dar\'io Avenda\~no Ram\'irez, Lic - M.Sc.}}
\fancyhead[LO]{}

\author{Silvia Malaver~\thanks{Germán Avendaño Ramírez}}
\title{\begin{minipage}{0.15\textwidth}\includegraphics[height=1.7cm]{Images/logo-colegio.png}
\end{minipage}\hfill \begin{minipage}{0.85\textwidth}\begin{center}
Animaplano - Expresiones algebraicas\\Álgebra $8^{\circ}$\end{center}
\end{minipage}}
\date{}

\begin{document}
\maketitle
Haga en su cuaderno un plano con 100 puntos, donde cada punto está numerado del 1--100. Luego ubique la respuesta en el plano y una los puntos en orden el orden dado.
\section{Animaplano}
\begin{enumerate}
  \item Exprese como número decimal el número romano LXXXV
  \item Halle $ 10^2-4^2 $
  \item Represente en años 1 siglo menos 1 lustro
  \item La mitad de 192
  \item Al cuádruple de 25, reste el triple del número 5
  \begin{multicols}{2}
    \item El 50\% de 130
  \item 3 docenas + 2 decenas
  \end{multicols}
  \item Sume al triple de 9, el triple del número 10
  \begin{multicols}{2}
    \item $ (\sqrt{64}\times\sqrt{81})+(14\div2)= $
  \item Halle $ 11\times(3\times3) $
  \end{multicols}
  \item El producto entre 25 y 4 disminuído en 4
\begin{multicols}{2}
  \item Si $ n-26=50 $, entonces $ n= $
  \item $ (9\times3)+(80\div2)= $
  \item $ (\sqrt{64}\times2^3)+\sqrt{16}= $
  \item La tercera parte de 114
\end{multicols}  
  \item El máximo común divisor de los números 5 y 10
\begin{multicols}{2}
   \item Hall $ (3!\times3!)-2^1= $
  \item El cuádruple de 21
  \item Resuelva $ 4!\times3= $
  \item $ 100-m=39 $, entonces $ m= $
  \item $ 20,7+14,9+5,4= $
  \item Halle $ 4!-2!= $
   \item El sexto número primo
  \item el número de lados de un pentágono
\end{multicols} 
  \item Los vértices del hexágono
  \item El mínimo común múltiplo de los números 9, 6, 3
 \begin{multicols}{2}
   \item El décimo número primo
  \item En años, medio siglo
 \end{multicols} 
  \item 1/2 siglo, más 1 década, más 2 lustros
  \item Encuentre el resultado si $ n=8 $, entonces $ (9\times n)+7= $
\end{enumerate}
\section{Uso de signos de agrupación}
Recordemos que el inverso aditivo de $3$ es $(-3)$ porque
\[ 3+(-3)=0 \qquad \text{ pero } \qquad 3+(-3)=3-3=0 \]
Generalizando si $a$ es un número real, su inverso aditivo es $-a$ porque
\[ a+(-a)=0 \qquad \text{ pero } \qquad a+(-a)=a-a=0 \]
Por otro lado si $a$ y $b$ son números reales enteros, $a + b$ es un número real, y su inverso aditivo es $- (a + b)$ por lo tanto:
\[ (a+b)+[-(a+b)]=0 \]
Como $a + b - a - b = 0$, al comparar esta expresión con la expresión anterior, se puede concluir que $- (a + b) = -a - b$.
\subsection{Actividad}
\begin{enumerate}
  \item En el párrafo anterior se concluyó que $- (a + b) = - a - b$. Describa este hecho oralmente resaltando la acción que tiene el signo - (menos) sobre el paréntesis.
  \item Encuentro el inverso aditivo de cada una de las siguientes expresiones
\begin{enumerate}\begin{multicols}{5}
  \item $-8$ \item $-a$ \item $ 5b $ \item $ (5+4) $ \item $ (x+y) $
  \item $ (x-y) $  \item $ -(m+n) $ \item $ -(m-n) $ \item $ -(8-u) $ \item $ -(a-b) $
\end{multicols}
 \end{enumerate}
  \item En mi cuaderno completo las siguientes igualdades:
  \begin{enumerate}\begin{multicols}{2}
    \item $ -a-5=-(\qquad \qquad) $  \item $ -m+3=-(\qquad\qquad) $
    \item $ h-1=-(\qquad\qquad) $ \item $ 5+x^2-x=-(\qquad \qquad) $
    \item $ -x+8-y^2=-x-(\qquad \qquad) $ \item $ -x^2-3x-2=-(\qquad \qquad\quad) $
  \end{multicols}
  \end{enumerate}
  \item Simplifico cada una de las siguientes expresiones:
\begin{enumerate}\begin{multicols}{2}
  \item $ (4x+3x-8x)+(5y-2y) $ \item $ (5x^2-2x^2y)-(8xy^2+3xy^2) $
  \item $ 5mn-(8mn-3m)-2m $  \item $ -(-x^2-2x+1)+(-3+2x+x^2) $
\end{multicols}
\end{enumerate}
\end{enumerate}
\section{Concluyamos}
Si se hace necesario eliminar un signo de agrupación precedido de el signo $-$ (menos) entonces todos los términos dentro de él cambian de signo y si está precedido de signo $+$ (más) los términos mantienen su signo. Recíprocamente, para agrupar varios términos en un signo de agrupación precedido de signo $-$ (menos); es necesario cambiar el signo a cada uno de los términos agrupados.
\subsection*{Ejemplos}
\begin{itemize}
  \item $ x-2y+3=-(-x+2y-3) $ \item $ -(1-2y)+(3x-y)=-1+2y+3x-y=3x+y-1 $ 
\end{itemize}
\subsection{Operaciones con polinomios}
Para sumar dos o más expresiones algebraicas se suman los términos semejantes.
\subsection*{Ejemplo}
Sumar $ x^2 - x + 8\text{ con }2x^2 - 5x - 3 $
\begin{align*}
(x^2 - x + 8) + (2x^2 - 5x - 3) &= x^2 - x + 8 + 2x^2 - 5x - 3\\
&= x^2 + 2x^2 - x - 5x + 8 - 3\\
&= 3x^2 - 6x + 5
\end{align*}
Como toda expresión algebraica es una representación simbólica de los
números reales, la suma de expresiones cumple las mismas propiedades de ellos.\\\\
Para sumar dos o más expresiones algebraicas se puede escribir una a
continuación de otra y luego se reducen los términos semejantes.\\\\
Otra forma de sumar dos o más polinomios es: primero ordenarlos de acuerdo a algún criterio definido y luego colocarlos uno debajo de otro en tal forma que términos semejantes queden en una misma columna, para por último efectuar la operación.
\section{Actividad}
\begin{enumerate}
  \item Supongamos que tenemos dos montones de naranjas. Un montón tiene $a$ naranjas.Expresamos simbólicamente el número de naranjas que hay en el segundo montón si en él hay:
\begin{enumerate}
  \item Doce naranjas menos que en el primero.
  \item 7 veces lo que tiene el primero.
  \item La sexta parte de las naranjas que hay en el primer montón.
\end{enumerate}
  \item Escribimos la igualdad de dos expresiones que representen el número de cabezas de ganado que hay en tres manadas. La primera tiene el doble que la segunda, la tercera tiene el doble de cabezas que la primera. En total hay 63 reses ¿Cuántas cabezas hay en cada manada?
\item Hallemos el perímetro y el área de la siguiente figura. Además encuentre el valor del perímetro y el área si $ x=2 $ y $ y=4 $

\begin{tikzpicture}
\node [above] at (1,0) {x};
\node [right] at (2,-1) {x};
\node [left] at (0,-3) {3x};
\node [below] at (4,-6) {2y};
\node [above] at (7,0) {x};
\draw (0,0) -- (2,0) -- (2,-2) -- (6,-2) -- (6,0) -- (8,0) -- (8,-6) -- (0,-6) -- (0,0);
\end{tikzpicture}
\end{enumerate}
\end{document}
