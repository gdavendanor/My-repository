\documentclass[twoside]{article}
\usepackage[utf8]{inputenc}
\usepackage{amsmath,amsfonts,amssymb,amsthm,latexsym}
\usepackage[spanish,es-noshorthands]{babel}
\usepackage[T1]{fontenc}
\usepackage{lmodern}
\usepackage{graphicx,hyperref}
\usepackage{tikz,pgf}
\usepackage{marvosym}
\usepackage{multicol}
\usepackage{fancyhdr}
\usepackage[papersize={6.5in,8.5in},left=.75cm,right=.75cm,top=1.5cm,bottom=1.25cm]{geometry}
\usepackage{fancyhdr}
\pagestyle{fancy}
\fancyhead[LE]{Colegio Arborizadora Baja}
\fancyhead[RE]{PEI:``Hacia una cultura para el desarrollo sostenible''}
\fancyfoot[RO]{}
\fancyhead[LO]{\url{www.autistici.org/mathgerman}}
\fancyfoot[RE]{Nit 830024976-8, Código DANE 11100103084-8}
\fancyfoot[LE]{\Email gavendanor@colarborizadorabaja.edu.co}
\fancyhead[RO]{\Email cedarborizadoraba19@redp.edu.co}

\author{Germ\'an Avenda\~no Ram\'irez~\thanks{Lic. Mat. U.D., M.Sc. U.N.}}
\title{\begin{minipage}{.2\textwidth}
\includegraphics[height=1.75cm]{Images/logo-colegio.png}\end{minipage}
\begin{minipage}{.55\textwidth}
\begin{center}
Animaplano 06\\
Matemáticas $7^{\circ}$
\end{center}
\end{minipage}\hfill
\begin{minipage}{.2\textwidth}
\includegraphics[height=1.75cm]{Images/logo-sed.png} 
\end{minipage}}
\date{}
\thispagestyle{plain}
\begin{document}
\maketitle
Nombre: \hrulefill Curso: \underline{703} Fecha: \underline{\hspace*{2.5cm}}\\

\fbox{\fbox{\parbox{5.5in}{\centering
\textit{Respuesta que requiera procedimiento y éste no se muestre, no será válida. Puede usar una hoja anexa a éste cuestionario para hacer los procedimientos respectivos}}}}
\section{Cuestionario}
\begin{enumerate}
  \item Exprese como número decimal el número romano LXXXV
  \item Halle $ 10^2-4^2 $
  \item Represente en años 1 siglo menos 1 lustro
  \item La mitad de 192
  \item Al cuádruple de 25, reste el triple del número 5
  \begin{multicols}{2}
    \item El 50\% de 130
  \item 3 docenas + 2 decenas
  \end{multicols}
  \item Sume al triple de 9, el triple del número 10
  \begin{multicols}{2}
    \item $ (\sqrt{64}\times\sqrt{81})+(14\div2)= $
  \item Halle $ 11\times(3\times3) $
  \end{multicols}
  \item El producto entre 25 y 4 disminuído en 4
\begin{multicols}{2}
  \item Si $ n-26=50 $, entonces $ n= $
  \item $ (9\times3)+(80\div2)= $
  \item $ (\sqrt{64}\times2^3)+\sqrt{16}= $
  \item La tercera parte de 114
\end{multicols}  
  \item El máximo común divisor de los números 5 y 10
\begin{multicols}{2}
   \item Halle $ (3!\times3!)-2^1= $
  \item El cuádruple de 21
  \item Resuelva $ 4!\times3= $
  \item $ 100-m=39 $, entonces $ m= $
  \item $ 20,7+14,9+5,4= $
  \item Halle $ 4!-2!= $
   \item El sexto número primo
  \item el número de lados de un pentágono
\end{multicols} 
  \item Los vértices del hexágono
  \item El mínimo común múltiplo de los números 9, 6, 3
 \begin{multicols}{2}
   \item El décimo número primo
  \item En años, medio siglo
 \end{multicols} 
  \item 1/2 siglo, más 1 década, más 2 lustros
  \item Encuentre el resultado si $ n=8 $, entonces $ (9\times n)+7= $
\end{enumerate}
\section*{Animaplano}
\begin{center}
\begin{tikzpicture}
 \fill (1,0) node[above]{1} circle (0.2ex);
 \fill (2,0) node[above]{2} circle (0.2ex);
 \fill (3,0) node[above]{3} circle (0.2ex);
 \fill (4,0) node[above]{4} circle (0.2ex);
 \fill (5,0) node[above]{5} circle (0.2ex);
 \fill (6,0) node[above]{6} circle (0.2ex);
 \fill (7,0) node[above]{7} circle (0.2ex);
 \fill (8,0) node[above]{8} circle (0.2ex);
 \fill (9,0) node[above]{9} circle (0.2ex);
 \fill (10,0) node[above]{10} circle (0.2ex);
 \fill (1,-1) node[left]{11} circle (0.2ex);
 \fill (2,-1) circle (0.2ex);
 \fill (3,-1) circle (0.2ex);
 \fill (4,-1) circle (0.2ex);
 \fill (5,-1) circle (0.2ex);
 \fill (6,-1) circle (0.2ex);
 \fill (7,-1) circle (0.2ex);
 \fill (8,-1) circle (0.2ex);
 \fill (9,-1) circle (0.2ex);
 \fill (10,-1) circle (0.2ex);
 \fill (1,-2) node[left]{21} circle (0.2ex);
 \fill (2,-2) circle (0.2ex);
 \fill (3,-2) circle (0.2ex);
 \fill (4,-2) circle (0.2ex);
 \fill (5,-2) circle (0.2ex);
 \fill (6,-2) circle (0.2ex);
 \fill (7,-2) circle (0.2ex);
 \fill (8,-2) circle (0.2ex);
 \fill (9,-2) circle (0.2ex);
 \fill (10,-2) circle (0.2ex);
 \fill (1,-3) node[left]{31} circle (0.2ex);
 \fill (2,-3) circle (0.2ex);
 \fill (3,-3) circle (0.2ex);
 \fill (4,-3) circle (0.2ex);
 \fill (5,-3) circle (0.2ex);
 \fill (6,-3) circle (0.2ex);
 \fill (7,-3) circle (0.2ex);
 \fill (8,-3) circle (0.2ex);
 \fill (9,-3) circle (0.2ex);
 \fill (10,-3) circle (0.2ex);
 \fill (1,-4) node[left]{41} circle (0.2ex);
 \fill (2,-4) circle (0.2ex);
 \fill (3,-4) circle (0.2ex);
 \fill (4,-4) circle (0.2ex);
 \fill (5,-4) circle (0.2ex);
 \fill (6,-4) circle (0.2ex);
 \fill (7,-4) circle (0.2ex);
 \fill (8,-4) circle (0.2ex);
 \fill (9,-4) circle (0.2ex);
 \fill (10,-4) node[right]{50} circle (0.2ex);
 \fill (1,-5) node[left]{51} circle (0.2ex);
 \fill (2,-5) circle (0.2ex);
 \fill (3,-5) circle (0.2ex);
 \fill (4,-5) circle (0.2ex);
 \fill (5,-5) circle (0.2ex);
 \fill (6,-5) circle (0.2ex);
 \fill (7,-5) circle (0.2ex);
 \fill (8,-5) circle (0.2ex);
 \fill (9,-5) circle (0.2ex);
 \fill (10,-5) circle (0.2ex);
 \fill (1,-6) node[left]{61} circle (0.2ex);
 \fill (2,-6) circle (0.2ex);
 \fill (3,-6) circle (0.2ex);
 \fill (4,-6) circle (0.2ex);
 \fill (5,-6) circle (0.2ex);
 \fill (6,-6) circle (0.2ex);
 \fill (7,-6) circle (0.2ex);
 \fill (8,-6) circle (0.2ex);
 \fill (9,-6) circle (0.2ex);
 \fill (10,-6) circle (0.2ex);
 \fill (1,-7) node[left]{71} circle (0.2ex);
 \fill (2,-7) circle (0.2ex);
 \fill (3,-7) circle (0.2ex);
 \fill (4,-7) circle (0.2ex);
 \fill (5,-7) circle (0.2ex);
 \fill (6,-7) circle (0.2ex);
 \fill (7,-7) circle (0.2ex);
 \fill (8,-7) circle (0.2ex);
 \fill (9,-7) circle (0.2ex);
 \fill (10,-7) circle (0.2ex);
 \fill (1,-8) node[left]{81} circle (0.2ex);
 \fill (2,-8) circle (0.2ex);
 \fill (3,-8) circle (0.2ex);
 \fill (4,-8) circle (0.2ex);
 \fill (5,-8) circle (0.2ex);
 \fill (6,-8) circle (0.2ex);
 \fill (7,-8) circle (0.2ex);
 \fill (8,-8) circle (0.2ex);
 \fill (9,-8) circle (0.2ex);
 \fill (10,-8) circle (0.2ex);
 \fill (1,-9) node[left]{91} circle (0.2ex);
 \fill (2,-9) circle (0.2ex);
 \fill (3,-9) circle (0.2ex);
 \fill (4,-9) circle (0.2ex);
 \fill (5,-9) circle (0.2ex);
 \fill (6,-9) circle (0.2ex);
 \fill (7,-9) circle (0.2ex);
 \fill (8,-9) circle (0.2ex);
 \fill (9,-9) circle (0.2ex);
 \fill (10,-9) node[right]{100} circle (0.2ex);
 \draw (5,-8)--(4,-8)--(5,-9)--(6,-9)--(5,-8)--(5,-6)--(6,-5)--(7,-5)--(9,-7)--(9,-9)--(6,-9)--(6,-7)--(7,-6)--(8,-6)--(8,-3)--(5,0)--(4,-3)--(4,-8)--(2,-7)--(1,-6)--(1,-4)--(2,-2)--(3,-1)--(5,0)--(6,0)--(8,-1)--(9,-2)--(10,-4)--(10,-6)--(9,-7);
\end{tikzpicture}
 \end{center}
\end{document}
