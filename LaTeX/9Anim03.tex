\documentclass[10pt,twoside]{article}
\usepackage[utf8]{inputenc}
\usepackage{amsmath}
\usepackage{amsfonts}
\usepackage{amssymb}
\usepackage[spanish,es-noshorthands]{babel}
\usepackage[T1]{fontenc}
\usepackage{lmodern}
\usepackage{graphicx,hyperref}
\usepackage{tikz,pgf}
\usepackage{multicol}
\usepackage{subfig}
\usepackage[papersize={6.5in,8.5in},width=5.5in,height=7in]{geometry}
\usepackage{fancyhdr}
\pagestyle{fancy}
\fancyhead[LE]{\includegraphics[height=12pt]{Images/logo-colegio.png} Matemáticas $11^{\circ}$}
\fancyhead[RE]{}
\fancyhead[RO]{\textit{Germ\'an Avenda\~no Ram\'irez, Lic. U.D., M.Sc. U.N.}}
\fancyhead[LO]{}

\author{Germ\'an Avenda\~no Ram\'irez, Lic. U.D., M.Sc. U.N.}
\title{\begin{minipage}{.2\textwidth}
\includegraphics[height=1.75cm]{Images/logo-colegio.png}\end{minipage}
\begin{minipage}{.55\textwidth}
\begin{center}
Animaplano 03, $9^{\circ}$  \\
Matemáticas $11^{\circ}$
\end{center}
\end{minipage}\hfill
\begin{minipage}{.2\textwidth}
\includegraphics[height=1.75cm]{Images/logo-sed.png} 
\end{minipage}}
\date{}
\begin{document}
\maketitle
Nombre: \hrulefill Curso: \underline{\hspace*{44pt}} Fecha: \underline{\hspace*{2.5cm}}
\section*{Cuestionario}
\begin{enumerate}
 \item Reste $4^{2}$, del triple de 30
 \item $\frac{5}{8}$, de un ángulo de $120^{\circ}$
 \item En años, 1/2 década + 12 lustros:
 \item El triple de la mitad de 44
 \item En $y=75x+3$, la pendiente es:
 \item Si $y-x=25$ para $x=51$, luego $y=$?
 \item El doble de $7^{2}$
 \item El área de un rectángulo de perímetro 40 y cuyo largo es 6 unidades más que el ancho.
 \item $(2^{2})^{2}\cdot (2^{4}\div 2^{2})+2^{3}+2=$
 \item Perímetro de un pentágono regular de lado 17 cm
 \item Décimo noveno número primo
 \item Convierta 5700 dm (decímetros) a Dm (Decámetros)
 \item $\sqrt{400}+\sqrt{121}+\sqrt{25}=$
 \item Si $a=b$, $a\cdot b=169$, luego $2b=$?
 \item Los meses en $\frac{2}{5}$ de un lustro:
 \item $\frac{1}{3}$ de $\frac{1}{3}$ de 297
 \item $\sqrt{64}\cdot \sqrt{8^{2}}=$
 \item En $y=74x-13$, la pendiente es:
 \item Dos rectas son paralelas si $m_{1}=m_{2}$ V:73 \quad F:85
 \item 1/3 del triple de 61
 \item El complemento de $49^{\circ}$. (Dos ángulos son complementarios si suman $90^{\circ}$)
 \item La media aritmética entre 24, 20, 22, 23, 21
 \item El sexto número primo
 \item $\log_{2}256+\log_{3}81+\log100=$
 \item $\sqrt[3]{216}=$
 \item En horas, 1/3 de dos días:
 \item 
\end{enumerate}
\begin{tikzpicture}
 \draw (4,-7)--(5,-7)--(5,-6)--(6,-6)--(5,-7)--(6,-7)--(8,-9)--(1,-9)--(4,-7)--(5,-8)--(7,-6)--(7,-5)--(6,-3)--(4,-2)--(3,-3)--(4,-6)--(4,-7)--(3,-7)--(1,-6)--(1,-4)--(2,-2)--(3,-1)--(4,-1)--(6,0)--(6,-1)--(
\end{tikzpicture}

\end{document}
