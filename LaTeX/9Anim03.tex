\documentclass[11pt,twoside,letter]{article}
\usepackage[utf8]{inputenc}
\usepackage{amsmath}
\usepackage{amsfonts}
\usepackage{amssymb}
\usepackage[spanish,es-noshorthands]{babel}
\usepackage[T1]{fontenc}
\usepackage{lmodern}
\usepackage{graphicx,hyperref}
\usepackage{tikz,pgf}
\usepackage{multicol}
\usepackage{subfig}
\usepackage[width=7.5in,height=9.5in]{geometry}
\usepackage{fancyhdr}
\pagestyle{fancy}
\fancyhead[LE]{\url{www.autistici.org/mathgerman}}
\fancyhead[RE]{}
\fancyhead[RO]{\textit{Germ\'an Avenda\~no Ram\'irez, Lic. U.D., M.Sc. U.N.}}
\fancyhead[LO]{}

\author{Germ\'an Avenda\~no Ram\'irez~\thanks{Lic. Mat. U.D., M.Sc. U.N.}}
\title{\begin{minipage}{.2\textwidth}
\includegraphics[height=1.75cm]{Images/logo-colegio.png}\end{minipage}
\begin{minipage}{.55\textwidth}
\begin{center}
Animaplano 03, $9^{\circ}$  \\
Matemáticas $11^{\circ}$
\end{center}
\end{minipage}\hfill
\begin{minipage}{.2\textwidth}
\includegraphics[height=1.75cm]{Images/logo-sed.png} 
\end{minipage}}
\date{}
\begin{document}
\maketitle
Nombre: \hrulefill Curso: \underline{\hspace*{44pt}} Fecha: \underline{\hspace*{2.5cm}}
\section*{Cuestionario}
\begin{enumerate}
 \item Reste $4^{2}$, del triple de 30
 \item $\frac{5}{8}$, de un ángulo de $120^{\circ}$
 \item En años, $\frac{1}{2}$ década + 12 lustros:
 \item El triple de la mitad de 44
 \item En $y=75x+3$, la pendiente es: ("recuerde que $m$ es la pendiente de una recta cuya ecuaci\'{o}n es $y=mx+b$")
 \item Si $y-x=25$ para $x=51$, luego $y=$?
 \item El doble de $7^{2}$
 \item El área de un rectángulo de perímetro 40 y cuyo largo es 6 unidades más que el ancho.
 \item $(2^{2})^{2}\cdot (2^{4}\div 2^{2})+2^{3}+2=$
 \item Perímetro de un pentágono regular de lado 17 cm
 \item Décimo noveno número primo
 \item Convierta 5700 dm (decímetros) a Dm (Decámetros). Un decámetro (Dm) equivale a 10m y cada metro equivale a 10 decímetros (dm).
 \item $\sqrt{400}+\sqrt{121}+\sqrt{25}=$
 \item Si $a=b$, $a\cdot b=169$, luego $2b=$?
 \item Los meses en $\frac{2}{5}$ de un lustro:
 \item $\frac{1}{3}$ de $\frac{1}{3}$ de 297
 \item $\sqrt{64}\cdot \sqrt{8^{2}}=$
 \item En $y=74x-13$, la pendiente es:
 \item Dos rectas son paralelas si $m_{1}=m_{2}$ V:73 \quad F:85
 \item 1/3 del triple de 61
 \item El complemento de $49^{\circ}$. (Dos ángulos son complementarios si suman $90^{\circ}$)
 \item La media aritmética (promedio) entre 24, 20, 22, 23, 21
 \item El sexto número primo
 \item $\log_{2}256+\log_{3}81+\log100=$ (Recuerde que $\log_{a}x=n$ s\'{i} y s\'{o}lo s\'{i} $a^{n}=x$)
 \item $\sqrt[3]{216}=$
 \item En horas, 1/3 de dos días:
 \item Pendiente de la recta que pasa por los puntos $\left(\frac{3}{2},2\right)$ y $(3,8)$. (La pendiente de la recta que pasa por los puntos $(x_{1},y_{1})$ y $(x_{2},y_{2})$ está dada por $m=\dfrac{y_{2}-y_{1}}{x_{2}-x_{1}}$)
 \item $\{(-28)\div(-4)\}\cdot(-2)+28$
 \item La distancia entre los puntos A(2,3) y B(10,18) es? (haga el dibujo y recuerde que el teorema de Pitágoras se aplica en los triángulos rectángulos)
 \item El valor de la abscica en el punto (28,18). (Todo punto en el plano es una pareja ordenda $(x,y)$, donde $x$ es la abscisa y $y$ es la ordenada)
 \item Los 3/5 de 1/3 de 135
 \item Si $y=\frac{x}{3}+2$ \; y \; $x=105$, entonces $y=$?
 \item 1/2 de un número, sumado con 1/3 de 99, da 62. ¿El número es?
 \item El décimo término de la progresión aritmética --4, 4, 12, \ldots (la diferencia entre dos términos consecutivos de una progresión aritmética es constante, siempre es el mismo número)
 \item $4^{3}+\log_{5}125=$
\end{enumerate}
\section*{Animaplano}
\begin{center}
\begin{tikzpicture}
 \fill (0,0) node[above]{0} circle (0.2ex);
 \fill (1,0) node[above]{1} circle (0.2ex);
 \fill (2,0) node[above]{2} circle (0.2ex);
 \fill (3,0) node[above]{3} circle (0.2ex);
 \fill (4,0) node[above]{4} circle (0.2ex);
 \fill (5,0) node[above]{5} circle (0.2ex);
 \fill (6,0) node[above]{6} circle (0.2ex);
 \fill (7,0) node[above]{7} circle (0.2ex);
 \fill (8,0) node[above]{8} circle (0.2ex);
 \fill (9,0) node[above]{9} circle (0.2ex);
 \fill (0,-1) node[left]{10} circle (0.2ex);
 \fill (1,-1) circle (0.2ex);
 \fill (2,-1) circle (0.2ex);
 \fill (3,-1) circle (0.2ex);
 \fill (4,-1) circle (0.2ex);
 \fill (5,-1) circle (0.2ex);
 \fill (6,-1) circle (0.2ex);
 \fill (7,-1) circle (0.2ex);
 \fill (8,-1) circle (0.2ex);
 \fill (9,-1) circle (0.2ex);
 \fill (0,-2) node[left]{20} circle (0.2ex);
 \fill (1,-2) circle (0.2ex);
 \fill (2,-2) circle (0.2ex);
 \fill (3,-2) circle (0.2ex);
 \fill (4,-2) circle (0.2ex);
 \fill (5,-2) circle (0.2ex);
 \fill (6,-2) circle (0.2ex);
 \fill (7,-2) circle (0.2ex);
 \fill (8,-2) circle (0.2ex);
 \fill (9,-2) circle (0.2ex);
 \fill (0,-3) node[left]{30} circle (0.2ex);
 \fill (1,-3) circle (0.2ex);
 \fill (2,-3) circle (0.2ex);
 \fill (3,-3) circle (0.2ex);
 \fill (4,-3) circle (0.2ex);
 \fill (5,-3) circle (0.2ex);
 \fill (6,-3) circle (0.2ex);
 \fill (7,-3) circle (0.2ex);
 \fill (8,-3) circle (0.2ex);
 \fill (9,-3) circle (0.2ex);
 \fill (0,-4) node[left]{40} circle (0.2ex);
 \fill (1,-4) circle (0.2ex);
 \fill (2,-4) circle (0.2ex);
 \fill (3,-4) circle (0.2ex);
 \fill (4,-4) circle (0.2ex);
 \fill (5,-4) circle (0.2ex);
 \fill (6,-4) circle (0.2ex);
 \fill (7,-4) circle (0.2ex);
 \fill (8,-4) circle (0.2ex);
 \fill (9,-4) node[right]{49} circle (0.2ex);
 \fill (0,-5) node[left]{50} circle (0.2ex);
 \fill (1,-5) circle (0.2ex);
 \fill (2,-5) circle (0.2ex);
 \fill (3,-5) circle (0.2ex);
 \fill (4,-5) circle (0.2ex);
 \fill (5,-5) circle (0.2ex);
 \fill (6,-5) circle (0.2ex);
 \fill (7,-5) circle (0.2ex);
 \fill (8,-5) circle (0.2ex);
 \fill (9,-5) circle (0.2ex);
 \fill (0,-6) node[left]{60} circle (0.2ex);
 \fill (1,-6) circle (0.2ex);
 \fill (2,-6) circle (0.2ex);
 \fill (3,-6) circle (0.2ex);
 \fill (4,-6) circle (0.2ex);
 \fill (5,-6) circle (0.2ex);
 \fill (6,-6) circle (0.2ex);
 \fill (7,-6) circle (0.2ex);
 \fill (8,-6) circle (0.2ex);
 \fill (9,-6) circle (0.2ex);
 \fill (0,-7) node[left]{70} circle (0.2ex);
 \fill (1,-7) circle (0.2ex);
 \fill (2,-7) circle (0.2ex);
 \fill (3,-7) circle (0.2ex);
 \fill (4,-7) circle (0.2ex);
 \fill (5,-7) circle (0.2ex);
 \fill (6,-7) circle (0.2ex);
 \fill (7,-7) circle (0.2ex);
 \fill (8,-7) circle (0.2ex);
 \fill (9,-7) circle (0.2ex);
 \fill (0,-8) node[left]{80} circle (0.2ex);
 \fill (1,-8) circle (0.2ex);
 \fill (2,-8) circle (0.2ex);
 \fill (3,-8) circle (0.2ex);
 \fill (4,-8) circle (0.2ex);
 \fill (5,-8) circle (0.2ex);
 \fill (6,-8) circle (0.2ex);
 \fill (7,-8) circle (0.2ex);
 \fill (8,-8) circle (0.2ex);
 \fill (9,-8) circle (0.2ex);
 \fill (0,-9) node[left]{90} circle (0.2ex);
 \fill (1,-9) circle (0.2ex);
 \fill (2,-9) circle (0.2ex);
 \fill (3,-9) circle (0.2ex);
 \fill (4,-9) circle (0.2ex);
 \fill (5,-9) circle (0.2ex);
 \fill (6,-9) circle (0.2ex);
 \fill (7,-9) circle (0.2ex);
 \fill (8,-9) circle (0.2ex);
 \fill (9,-9) node[right]{99} circle (0.2ex);
% \draw (4,-7)--(5,-7)--(5,-6)--(6,-6)--(5,-7)--(6,-7)--(8,-9)--(1,-9)--(4,-7)--(5,-8)--(7,-6)--(7,-5)--(6,-3)--(4,-2)--(3,-3)--(4,-6)--(4,-7)--(3,-7)--(1,-6)--(1,-4)--(2,-2)--(3,-1)--(4,-1)--(6,0)--(6,-1)--(4,0)--(4,-1)--(7,-1)--(8,-2)--(7,-2)--(7,-3)--(8,-5)--(8,-6)--(7,-6);
\end{tikzpicture}\end{center}

\end{document}
