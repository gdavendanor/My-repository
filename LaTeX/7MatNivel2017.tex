\documentclass[twoside,letterpaper]{article}
\usepackage[utf8]{inputenc}
\usepackage{amsmath,amsfonts,amssymb,amsthm,latexsym}
\usepackage[spanish,es-noshorthands]{babel}
\usepackage[T1]{fontenc}
\usepackage{lmodern}
\usepackage{graphicx,hyperref}
\usepackage{tikz,pgf}
\usepackage{marvosym}
\usepackage{multicol}
\usepackage{fancyhdr}
\usepackage[left=.75cm,right=.75cm,top=1.5cm,bottom=1.25cm]{geometry}
\usepackage{fancyhdr}
\pagestyle{fancy}
\fancyhead[LE]{Colegio Arborizadora Baja}
\fancyhead[RE]{PEI:``Hacia una cultura para el desarrollo sostenible''}
\fancyfoot[RO]{\Email iedabgerman@autistici.org}
\fancyhead[LO]{\url{www.autistici.org/mathgerman}}
\fancyfoot[RE]{\Email cedarborizadoraba19@redp.edu.co}
\fancyfoot[LE]{Calle 59I \#44A - 02 \Telefon 7313994 - 7313995}
\fancyhead[RO]{Nit 830024976-8, Código DANE 11100103084-8}

\author{Germ\'an Avenda\~no Ram\'irez~\thanks{Lic. Mat. U.D., M.Sc. U.N.}}
\title{\begin{minipage}{.2\textwidth}
\includegraphics[height=1.75cm]{Images/logo-colegio.png}\end{minipage}
\begin{minipage}{.55\textwidth}
\begin{center}
Nivelación 2017\\
Matemáticas $7^{\circ}$
\end{center}
\end{minipage}\hfill
\begin{minipage}{.2\textwidth}
\includegraphics[height=1.75cm]{Images/logo-sed.png} 
\end{minipage}}
\date{}
\thispagestyle{plain}
\begin{document}
\maketitle
Nombre: \hrulefill Curso: \underline{\hspace*{44pt}} Fecha: \underline{\hspace*{2.5cm}}
%\begin{minipage}{.95\textwidth}
%\fbox{\textit{No raye ni dañe esta hoja para que pueda usarla otro compañero}}
%\end{minipage}
\begin{multicols}{2}
\section*{Número $\mathbb{Z}$}
Determine qué números enteros corresponden a cada situación
\begin{enumerate}
\item El miércoles, la temperatura fue de 24 grados sobre cero. El miércoles fue de 2 grados bajo cero.
\item \emph{Temperaturas extremas:} La temperatura más alta que se ha dado en la tierra fue de 950'000,000 $^{\circ}F$. La temperatura más baja fue de aproximadamente 460 $^{\circ}F$ bajo cero.
\item \emph{Edificio Empire State:} El edificio Empire State tiene una altura total, incluyendo la antena iluminada en su parte más alta, de 1454 pies. Los cimientos tienen una profundidad de 55 pies bajo el nivel del suelo.

Ubique cada número sobre una recta numérica
\begin{multicols}{2}
\item $\dfrac{10}{3}$
\item $-5.2$
\end{multicols}
Convierta a notación decimal
\begin{multicols}{3}
\item $-\dfrac{7}{8}$
\item $\dfrac{5}{6}$
\item $-\dfrac{7}{6}$
\item $\dfrac{2}{5}$
\item $-\dfrac{1}{2}$
\item $-8\dfrac{7}{25}$
\end{multicols}
Use <, > en \tikz \draw (0,0) rectangle (.4,.4); según el caso
\begin{multicols}{2}
\item 8 \tikz \draw (0,0) rectangle (.4,.4); 0
\item $-8$ \tikz \draw (0,0) rectangle (.4,.4); 3
\item $-8$ \tikz \draw (0,0) rectangle (.4,.4); $-5$
\item $-5$ \tikz \draw (0,0) rectangle (.4,.4); $-11$
\item $-6$ \tikz \draw (0,0) rectangle (.4,.4); $-5$
\item 2.14 \tikz \draw (0,0) rectangle (.4,.4); 1.24
\item $-14.5$ \tikz \draw (0,0) rectangle (.4,.4); 0.011
\item $-12\frac{5}{8}$ \tikz \draw (0,0) rectangle (.4,.4); $-6\dfrac{3}{8}$
\item $\dfrac{5}{12}$ \tikz \draw (0,0) rectangle (.4,.4); $\dfrac{11}{25}$ 
\end{multicols}
Encuentre la descomposición en factores primos de:
\begin{multicols}{2}
\item 102
\item 864
\end{multicols}
\item Encuentre el m.c.m. de 48, 56 y 64
\item Ordene de menor a mayor \[-8\frac{7}{8}, 7, -5, |-6|, 4, |3|, -8\frac{5}{8}, -100, 0, 1^{7}, \dfrac{14}{4}, -\dfrac{67}{8}\]

Encuentre el opuesto (inverso aditivo) de:
\begin{multicols}{2}
\item $-5.6$
\item 0
\end{multicols}
\item Evalúe $-x$ cuando $x$ es --19

Calcule y simplifique:
\begin{multicols}{2}
\item  $7+(-9)$
\item $3.6+(-3.6)$
\item $\dfrac{2}{3}+\left( -\dfrac{9}{8}\right)$
\item $-14+5$
\item $-4.1-6.3$
\item $-\dfrac{1}{4}-\left(-\dfrac{3}{5}\right)$
\item $-8-(-4)$
\item $12.3-14.1$
\item $16-(-9)-20-(-4)$
\item $17-(-25)+15-(-18)$
\end{multicols}
\item ¿Cuál es la razón, si un auto recorre 150 m en 12 segundos?
\item Multiplique $(-2)(5)$
\item Divida $\dfrac{-48}{-16}$

Calcule y simplifique
\begin{multicols}{2}
\item $12\,854\cdot 750\,000$
\item $4\frac{2}{9}-2\frac{7}{18}$
\item $-\dfrac{3}{14}\div \dfrac{6}{7}$
\item $\dfrac{2}{27}\cdot \left(-\dfrac{9}{16}\right)$
\end{multicols}
\item $32\div [(-2)(-8)-(15-(-1))]$
\item ¿De qué número es 4 el $12\frac{1}{2}$\%?
\item El trece por ciento de 600 estudiantes de un colegio, están en grado sexto. ¿Cuánto estudiantes están en 6$^{\circ}$
\item Un recipiente para hacer un postre se llena con $1\frac{1}{4}$ tazas de harina y una recipiente para hacer una torta, se llena con $1\frac{2}{3}$ de tazas de harina. ¿Cuántas tazas de harina se necesitan para llenar los dos recipientes?
\item Un atleta da 6.5 vueltas a una pista circular. Si la pista mide 0.7 km, ¿qué distancia recorre el atleta?
\item Encuentre la medida del ángulo faltante
\begin{center}
\begin{tikzpicture}
\draw (0,0) --(3,0)node[above]{$x$}--(4.4,2) --cycle;
\node[right] at (0.4,.15){$28^{\circ}$};
\node[left] at (4.2,1.45) {{$32^{\circ}$}};
\end{tikzpicture}
\end{center}
\section*{Números $\mathbb{Q}$}
\item \emph{Pasaportes:} En 2010, 14'794\,604 pasaportes americanos fueron usados. Este número decreció a 12'613\,153 en 2011 y luego se incrementó a 13'125\,829 en 2012. ¿En qué porcentaje disminuyó de 2010 a 2011? ¿En qué porcentaje se incrementó de 2011 a 2012? Aproxime las respuestas a la décima más cercana.
\item Encuentre el porcentaje correspondiente a $\frac{9}{8}$
\item Escriba en notación de fracción la razón 5 a 0.5
\item Escriba <, > o = en el cuadrado, para hacer verdadero el enunciado $\frac{5}{7}$ \tikz \draw (0,0) rectangle (.4,.4); $\frac{6}{8}$. Justifique su respuesta.
\item Estime aproximando al centésima más cercana$263,961+32,090+127.89$
\item Calcule $46-[4(6+4\div 2)+2\times 3-5]$

Calcule y simplifique
\item $487,094+6,936+21,120$
\begin{multicols}{2}
\item $\dfrac{6}{5}+1\frac{5}{6}$
\item $3\frac{1}{3}-2\frac{2}{3}$
\item $\dfrac{7}{9}\cdot \dfrac{3}{14}$
\item $46.012\times 0.03$
\end{multicols}
\item $431.2\div 35.2$

Solucione
\begin{multicols}{2}
\item $36\cdot x=3420$
\item $\dfrac{2}{15}\cdot t=\dfrac{6}{5}$
\end{multicols}
\item $\dfrac{y}{25}=\dfrac{24}{15}$
\item \emph{Atención en museos:} Entre los museos de arte más importantes, el Museo de Arte de Indianápolis se encuentra en el sexto lugar de acuerdo al porcentaje de visitantes de la población metropolitana. En 2007, el número de visitantes fue de 30.31\% de la población de la ciudad, la cual era de 1'525,104. Cuál fue el total de visitantes atendidos en el museo?
\item En cierto momento durante la temporada 2008-2009 de la NBA (National Basketball Asociation) los "Cleveland Cavaliers" ganaron 39 de 49 juegos. A esta razón, cuánto juegos podrían haber ganado en toda la temporada que fue de 82 juegos?
\item \emph{Precion por unidad:} Una botella de 200 onzas de detergente líquido cuesta 14.99 dólares. ¿Cuál es el precio de una onza?
\item En un mapa, 1 cm representa 80 kilómetros. ¿Cuántos kilómetros representarán $\frac{3}{4}$ cm?
\item \emph{Reglitas:} ¿Cuántas reglitas de $1\frac{4}{5}$ de decímetro pueden ser cortadas de una regla grande de 9 dm?
\item Reste y simplifique $\dfrac{14}{25}-\dfrac{3}{20}$
\item Si $a$ es el 50\% de $b$, entonces $b$ qué porcentaje es de $a?$
\section*{Estadística}
Solución guiada. Rellen el cuadrado con el número apropiado
\item La media de 60, 45, 115, 15 y 35 es:
\[\dfrac{60+45+\square+15+35}{\square}=\dfrac{\square}{5}=\square\]
Para cada conjunto de datos, encuentre la media, la mediana y la(s) moda(s) si existe(n)
\item 2.12, 18.42, 9.37, 43.89
\item 160, 102, 102, 116, 160, 116
\item $\dfrac{1}{2}, \dfrac{3}{4}, \dfrac{7}{8}, \dfrac{5}{4}$
\item 38.2, 38.2, 38.2, 38.2
\paragraph*{Reducción del tamaño:} Las empresas acostumbran reducir el tamaño de sus productos para ponerles el mismo precio que manejaban antes y dar la sensación que no están subiendo los precios. La siguiente tabla muestra la lista de productos reducidos en tamaño.

\begin{tabular}{|c|c|c|c|}
\hline 
 & \multicolumn{2}{c}{Tamaño (onzas)}  &  \\ 
\hline 
Producto & viejo & nuevo & Reducción\\ 
\hline 
Crema helado Breyer & 56 & 48 & 14\% \\ 
\hline 
Mayonesa Hellman & 32 & 30 & 6\% \\ 
\hline 
Chocolate negro Hershey & 9 & 6.8 & 15\% \\ 
\hline 
Comida de gatos Iams & 6 & 5.5 & 8\% \\ 
\hline 
Papas Nabisco & 16 & 15.25 & 5\% \\ 
\hline 
Crema de maní Skippy & 18 & 16.3 & 9\% \\ 
\hline 
Jugo de naranja Tropicana & 96 & 89 & 7\% \\ 
\hline 
\end{tabular} 
\item ¿Cuántas onzas de crema de helado Breyer hay menos en el nuevo empaque, respecto al viejo?
\item ¿En qué porcentaje se redujo el tamaño del empaque de la mayonesa Hellman?
\item ¿Qué producto en la tabla muestra un mayor porcentaje de reducción en su tamaño?
\item Cuánto menos jugo hay en el nuevo envase de jugo de naranja Tropicana con respecto al empaque viejo?
\item ¿Cuál producto en la tabla muestra un porcentaje de reducción en su tamaño?
\end{enumerate}
\end{multicols}

\end{document}
