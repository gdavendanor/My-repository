\documentclass[twoside]{article}
\usepackage[utf8]{inputenc}
\usepackage{amsmath,amsfonts,amssymb,amsthm,latexsym}
\usepackage[spanish,es-noshorthands]{babel}
\usepackage[T1]{fontenc}
\usepackage{lmodern}
\usepackage{graphicx,hyperref}
\usepackage{tikz,pgf}
\usepackage{marvosym}
\usepackage{multicol}
\usepackage{fancyhdr}
\usepackage[papersize={5.5in,8.5in},left=.75cm,right=.75cm,top=1.35cm,bottom=1.1cm]{geometry}
\usepackage{fancyhdr}
\pagestyle{fancy}
\fancyhead[LE]{Colegio Arborizadora Baja}
\fancyhead[RE]{PEI:``Hacia una cultura para el desarrollo sostenible''}
\fancyfoot[RO]{\Email iedabgerman@autistici.org}
\fancyhead[LO]{\url{www.autistici.org/mathgerman}}
\fancyfoot[RE]{\Email cedarborizadoraba19@redp.edu.co}
\fancyfoot[LE]{Calle 59I \#44A - 02 \Telefon 7313994 - 7313995}
\fancyhead[RO]{Nit 830024976-8, Código DANE 11100103084-8}

\author{Germ\'an Avenda\~no Ram\'irez~\thanks{Lic. Mat. U.D., M.Sc. U.N.}}
\title{\begin{minipage}{.2\textwidth}
\includegraphics[height=1.75cm]{Images/logo-colegio.png}\end{minipage}
\begin{minipage}{.55\textwidth}
\begin{center}
Método de eliminación\\
Álgebra $9^{\circ}$
\end{center}
\end{minipage}\hfill
\begin{minipage}{.2\textwidth}
\includegraphics[height=1.75cm]{Images/logo-sed.png} 
\end{minipage}}
\date{}
\thispagestyle{plain}
\begin{document}
\maketitle
%Nombre: \hrulefill Curso: \underline{\hspace*{44pt}} Fecha: \underline{\hspace*{2.5cm}}
\begin{minipage}{.95\textwidth}
\fbox{\textit{No raye ni dañe esta hoja para que pueda usarla otro compañero}}
\end{minipage}
\paragraph*{a.} Solucione usando el método de eliminación por reducción directamente
\begin{enumerate}
\begin{multicols}{3}
\item $\displaystyle{ \left\{
{ x-y=7
\atop
x+y=5
}
\right.} $
\item $\displaystyle{\left\{
x+y=8
\atop
-x+2y=7
\right.}$
\item $\displaystyle{\left\{
5x-y=5
\atop
3x+y=11
\right.
}$
\item $\displaystyle{\left\{
4a+3b=7
\atop
-4a+b=5\right.}$
\item $\displaystyle{\left\{
8x-5y=-9
\atop
3x+5y=-2\right.}$
\item $\displaystyle{\left\{
4x-5y=7
\atop
-4x+5y=7\right.}$
\end{multicols}
\paragraph*{b.} Solucione usando el principio multiplicativo primero, luego sume las ecuaciones para eliminar una incógnita.
\begin{multicols}{3}
\item $\displaystyle{\left\{
x+y=-7
\atop
3x+y=-9\right.}$
\item $\displaystyle{\left\{
3x-y=8
\atop
x+2y=5\right.}$
\item $\displaystyle{\left\{
x-y=5
\atop
4x-5y=17\right.}$
\item $\displaystyle{\left\{
2u-3v=-1
\atop
3u+4v=24\right.}$
\item $\displaystyle{\left\{
2a+3b=-1
\atop
3a+5b=-2\right.}$
\item $\displaystyle{\left\{
x=3y
\atop
5x+14=y\right.}$
\item $\displaystyle{\left\{
2x+5y=16
\atop
3x-2y=5\right.}$
\item $\displaystyle{\left\{
p=32+q
\atop
3p=8q+6\right.}$
\item $\displaystyle{\left\{
3x-2y=10
\atop
-6x+4y=-20\right.}$
\item $\displaystyle{\left\{
0.06x+0.05y=0.07
\atop
0.4x-0.3y=1.1\right.}$
\item $\displaystyle{\left\{
\frac{1}{3}x+\frac{3}{2}y=\frac{5}{4}
\atop
\frac{3}{4}x-\frac{5}{6}y=\frac{3}{8}\right.}$
\item $\displaystyle{\left\{
-4.5x+7.5y=6
\atop
-x+1.5y=5\right.}$
\end{multicols}
\paragraph*{c.} En los ejercicios siguientes, complete la oración en su cuaderno con una palabra de la lista siguiente en lugar del número:

solución, pendiente, y-intercepto pendiente-intercepto, gráfica.
\item Las rectas paralelas tienen la misma \underline{\hspace*{5pt}1\hspace*{5pt}} y diferente \underline{\hspace*{5pt}2\hspace*{5pt}}
\item Una \underline{\hspace*{5pt}3\hspace*{5pt}} de un sistema de dos ecuaciones es una pareja ordenada que hace a las dos ecuaciones verdaderas
\end{enumerate}


\end{document}
