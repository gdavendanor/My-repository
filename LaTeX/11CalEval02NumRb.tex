\documentclass[fleqn,twocolumn,landscape]{article}
\usepackage[spanish,es-noshorthands]{babel}
\usepackage[utf8]{inputenc} 
\usepackage[left=.75cm, right=.75cm, top=1.5cm, bottom=1.25cm]{geometry}
\usepackage{mathexam}
\usepackage{amsmath}
\usepackage{amsfonts}
\usepackage{amssymb}
\usepackage{graphicx}
\usepackage{multicol}

\ExamClass{\includegraphics[height=16pt]{Images/logo-sed.png} Cálculo $11^{\circ}$}
\ExamName{``2a Evaluación $\mathbb{R}$''}
\ExamHead{\includegraphics[height=16pt]{Images/logo-colegio.png} IEDAB}
\newcommand{\LineaNombre}{%
\par
\vspace{\baselineskip}
Nombre:\hrulefill \; Curso: \underline{\hspace*{48pt}} \; Fecha: \underline{\hspace*{2.5cm}} \relax
\par}
\let\ds\displaystyle

\begin{document}
\ExamInstrBox{
Respuesta sin justificar mediante procedimiento no será tenida en cuenta en la calificación. Escriba sus respuestas en el espacio indicado. Tiene 45 minutos para contestar esta prueba.}
\LineaNombre
\begin{enumerate}
 \item Sean $x$, $y$ y $z$ números reales con $x>0$, $y<0$ y $z>0$. Halle el signo de cada expresión
 \begin{enumerate}
 \item $-(x)$
 \item $-xy$
 \item $xyz$
 \item $xy^{2}z$
 \item $x^{2}y^{2}z$
 \end{enumerate}
 \item Evalúe las siguientes expresiones
 \begin{enumerate}
 \item $(-5)^{2}=$\noanswer
 \item $-(5)^{2}=$\noanswer
 \item $5^{-2}=$\noanswer
 \item $\dfrac{5^{2}}{5^{3}}=$\noanswer
 \item $\dfrac{\sqrt{27}}{\sqrt{3}}=$\noanswer
 \end{enumerate}
 \item Simplifique las siguientes expresiones:
 \begin{enumerate}
 \item $(\sqrt{5}+\sqrt{3})(\sqrt{5}-\sqrt{3})=$\vspace*{20pt}
 \item $(\sqrt{2}-\sqrt{3})^{2}=$\vspace*{20pt}
 \item $\sqrt{\frac{2}{3}}\sqrt{75}=$ \vspace*{20pt}
 \item $\dfrac{\sqrt[3]{81}}{\sqrt[3]{64}}=$\vspace*{20pt}
  \item $\sqrt{45}-\sqrt{20}+\sqrt{125}=$\vspace*{20pt}
 \end{enumerate}
 \item Las dimensiones de un aula son 9 m de largo, 4 m de ancho y 3 m de alto. ¿Cuál es la mayor distancia a la que pueden encontrarse dos zancudos dentro del aula?\noanswer
 \section*{Prueba saber}
 \item Se puede encontrar números racionales mayores que $k$, de manera que sean cada vez más cercanos a él, calculando $k +\frac{1}{j}$ (con $j$ entero positivo). Cuanto más grande sea $j$, más cercano a $k$ será el racional construido. ¿Cuántos números racionales se pueden construir cercanos a $k$ y menores que $k + \frac{1}{11}$?
\begin{enumerate}
\item 10, que es la cantidad de racionales menores que 1
\item Una cantidad infinita, pues existen infinitos números enteros mayores que 11
\item 11, que es el número que equivale en este caso a $j$
\item Uno, pues el racional más cercano a $k$ se halla con $j=10$, es decir, con $k+0.1$
\end{enumerate}
\answer*[0pt]{Just:}
 \end{enumerate}
\end{document}
