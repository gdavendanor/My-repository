\documentclass[10pt,twoside]{article}
\usepackage[utf8]{inputenc}
\usepackage{amsmath}
\usepackage{amsfonts}
\usepackage{amssymb}
\usepackage[spanish,es-noshorthands]{babel}
\usepackage[T1]{fontenc}
\usepackage{lmodern}
\usepackage{graphicx,hyperref}
\usepackage{tikz,pgf}
\usepackage{marvosym}
\usepackage{multicol}
\usepackage{subfig}
\usepackage[papersize={6.5in,8.5in},width=5.5in,height=7in]{geometry}
\usepackage{fancyhdr}
\pagestyle{fancy}
\fancyhead[LE]{\url{www.autistici.org/mathgerman}}
\fancyhead[RE]{}
\fancyhead[RO]{\Email~ matematicas.german@gmail.com}
\fancyhead[LO]{}

\author{Germ\'an Avenda\~no Ram\'irez~\thanks{Lic. Mat. U.D., M.Sc. U.N.}}
\title{\begin{minipage}{.2\textwidth}
\includegraphics[height=1.75cm]{Images/logo-colegio.png}\end{minipage}
\begin{minipage}{.55\textwidth}
\begin{center}
Taller 06, prob. condicional \\
Probabilidad $11^{\circ}$
\end{center}
\end{minipage}\hfill
\begin{minipage}{.2\textwidth}
\includegraphics[height=1.75cm]{Images/logo-sed.png} 
\end{minipage}}
\date{}
\begin{document}
\maketitle
Nombre: \hrulefill Curso: 110\underline{\hspace{12pt}}  Fecha: \underline{\hspace{2cm}}\\

Recordemos que:\\
$P(A|B)$: Se lee como: "La probabilidad de que suceda A dado que ha sucedido B". A esto se le conoce como \emph{probabilidad condicional} y se puede calcular directamente según el problema o usando la siguiente ecuación:
\[P(A|B)=\dfrac{P(A\cap B)}{P(B)} \]
\section*{Problemas}
\begin{enumerate}
\item Un artículo de USA Today titulado “Yum Brands hace dinastía en China” (7 de febrero, 2005) informa sobre cómo la Yum Brands, la empresa de restaurantes más grande del mundo, está llevando la industria de
comida rápida a China, India y otros países grandes. La Yum Brands, filial de PepsiCo, ha estado entregando un crecimiento de utilidades de dos dígitos en el año pasado.
\begin{center}
\begin{tabular}{|c|r@{\,}l|r@{\,}l|}
\hline 
Tienda & USA & & En otros países& \\ 
\hline 
KFC & 5&450 & 7&676 \\ 
Pizza Hut & 6&306 & 4&680 \\ 
Taco Bell & 5&030 & &123 \\ 
Long John Silver's & &485 & & \hspace*{5pt}33 \\ 
A\&W All-American & &485 & &209 \\ \hline
Total & 18&471 & 12&791 \\ 
\hline 
\end{tabular} 
\end{center}
Supongamos que cuando el director general de Yum Brands fue entrevistado para este artículo, se le hicieron las siguientes preguntas. ¿Cómo podría haber contestado con base en la tabla siguiente?
\begin{enumerate}
\item ¿Qué porcentaje de sus locales está en Estados
Unidos?
\item ¿Qué porcentaje de sus locales está en otros países?
\item ¿Qué porcentaje de sus tiendas son Pizza Huts?
\item ¿Qué porcentaje de sus tiendas son Taco Bell dado
que la ubicación es en Estados Unidos?
\item ¿Qué porcentaje de sus locales está en otros países dado que la tienda es una A\&W All-American?
\item ¿Qué porcentaje de sus tiendas es KFC dado que la ubicación es en otros países?
\end{enumerate}

\end{enumerate}
\end{document}
