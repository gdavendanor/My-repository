\documentclass[fleqn]{article}
\usepackage[spanish,es-noshorthands]{babel}
\usepackage[utf8]{inputenc} 
\usepackage[papersize={5.5in,8.5in},total={4.5in,7.25in},centering]{geometry}
\usepackage{mathexam}
\usepackage{amsmath}
\usepackage{graphicx}
\usepackage{multicol}
\usepackage{tikz}
\ExamClass{\includegraphics[height=16pt]{Images/logo-sed.png} Aritmética $6^{\circ}$}
\ExamName{Niv. Fracciones 2}
\ExamHead{\includegraphics[height=16pt]{Images/logo-colegio.png} IEDAB}
\newcommand{\LineaNombre}{%
\par
\vspace{\baselineskip}
Nombre:\hrulefill \; Curso: \underline{\hspace*{48pt}} \; Fecha: \underline{\hspace*{2.5cm}} \relax
\par}
\let\ds\displaystyle

\begin{document}
\ExamInstrBox{
Respuesta sin justificar mediante procedimiento no será tenida en cuenta en la calificación. Escriba sus respuestas en el espacio indicado. Tiene 45 minutos para contestar esta prueba.}
\LineaNombre
\begin{enumerate}
 \item Determine si cada una de las siguientes fracciones es ``propia'', ``impropia'' o ``equivalente a la unidad''
  \begin{enumerate}
 \begin{multicols}{2}
 \item $\dfrac{3}{4}$
 \item $\dfrac{2}{2}$
 \item $\dfrac{7}{5}$
 \item $\dfrac{9}{11}$ 
 \end{multicols}
 \end{enumerate}
 \item Escriba cada número mixto como una fracción impropia
 \begin{enumerate}
 \begin{multicols}{2}
 \item $3\frac{1}{2}=$\noanswer
 \item $4\frac{1}{3}=$ \noanswer
 \item $1\frac{2}{5}=$\noanswer
 \item $2\frac{2}{3}=$ \noanswer
 \end{multicols}
 \end{enumerate}
 \item Determine la fracción o el número mixto correspondiente a cada una de las letras $A$, $B$ y $C$ en la siguiente recta numérica:
 \begin{center}
 \begin{tikzpicture}[scale=2]
 \draw[|->] (0,0) -- (3,0);
\foreach \x in {1,2}
\draw (\x,-1pt)--(\x,2pt);
\foreach \x in {.2,.4,.6,...,2.8}\draw(\x,-1pt)--(\x,1pt);
\node[below] (1) at (1,0) {1};
\node[below](2) at (2,0){2};
\node[below] at (.6,0){A};
\node[below] at (1.4,0){B};
\node[below] at (2.2,0){C};
 \end{tikzpicture}
 \end{center}\noanswer
 \item Descubra la fracción más simple correspondiente en cada caso
\begin{enumerate}
\begin{multicols}{3}
\item \begin{tikzpicture}[scale=.5]
\fill[gray] rectangle (2,2);
\draw grid (4,3);
\end{tikzpicture}
\item  \begin{tikzpicture}[scale=.5]
\fill[gray] rectangle (1,2);
\fill[gray] (1,0) rectangle (2,1);
\fill[gray] (3,2) rectangle (4,4);
\fill[gray](2,3) rectangle (3,4);
\draw grid (4,4);
\end{tikzpicture}
\item \begin{tikzpicture}[scale=.5]
\draw grid (4,4);
\fill[gray] rectangle (1,1);
\fill[gray] (1,1) rectangle (2,2);
\fill[gray](2,2) rectangle (3,3);
\fill[gray](3,3) rectangle (4,4);
\end{tikzpicture}
\end{multicols}
\end{enumerate}
\newpage
\item Efectúe las siguientes operaciones y simplique la respuesta si se puede
\begin{enumerate}
\item $\dfrac{2}{7}+\dfrac{3}{7}=$\noanswer[15pt]
\item $\dfrac{7}{9}-\dfrac{4}{9}=$\noanswer[15pt]
\item $\dfrac{2}{3}+\dfrac{1}{4}=$\noanswer[15pt]
\item $\dfrac{3}{4}-\dfrac{1}{3}=$\noanswer[15pt]
\item $\dfrac{2}{9}\cdot \dfrac{3}{5}=$\noanswer[15pt]
\item $\dfrac{4}{5}\div \dfrac{2}{3}=$\noanswer[15pt]
\end{enumerate}
\item Coloree la fracción correspondiente en cada caso. Cada rectángulo de $ 3\times 4 $ representa la unidad.
\begin{enumerate}
\begin{multicols}{3}
\item $\dfrac{1}{3}$ \tikz \draw[scale=.5,help lines](0,0) grid (4,3);
\item $\dfrac{3}{4}$ \tikz \draw[scale=.5,help lines](0,0) grid (4,3);
\item $\dfrac{1}{4}$ \tikz \draw[scale=.5,help lines](0,0) grid (4,3);
\end{multicols}
 \end{enumerate}
 \item Un granjero vende $\dfrac{1}{3}$ del terreno de su granja, alquila  $\dfrac{1}{5}$ y cultiva lo que le queda. ¿Qué parte de la granja cultiva? Para resolverlo, puede hacer uso del siguiente dibujo si así lo desea.
\begin{center}
\begin{tikzpicture}
 \draw grid (5,3);
 \end{tikzpicture}
\end{center}
 \end{enumerate}
\end{document}
