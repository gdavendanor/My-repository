\documentclass[fleqn]{article}
\usepackage[spanish,es-noshorthands]{babel}
\usepackage[utf8]{inputenc} 
\usepackage[papersize={5.5in,8.5in},total={4.5in,7.25in},centering]{geometry}
\usepackage{mathexam}
\usepackage{amsmath}
\usepackage{graphicx}
\usepackage{multicol}

\ExamClass{\includegraphics[height=16pt]{Images/logo-sed.png} Probabilidad $11^{\circ}$}
\ExamName{``Prob. Elemental''}
\ExamHead{\includegraphics[height=16pt]{Images/logo-colegio.png} IEDAB}
\newcommand{\LineaNombre}{%
\par
\vspace{\baselineskip}
Nombre:\hrulefill \; Curso: \underline{\hspace*{48pt}} \; Fecha: \underline{\hspace*{2.5cm}} \relax
\par}
\let\ds\displaystyle

\begin{document}
\ExamInstrBox{
Respuesta sin justificar mediante procedimiento no será tenida en cuenta en la calificación. Escriba sus respuestas en el espacio indicado. Tiene 45 minutos para contestar esta prueba.}
\LineaNombre
\begin{enumerate}
 \item Se lanzan 4 monedas. Construye el espacio muestral y cada uno de los siguientes sucesos y halla la probabilidad correspondiente
 \begin{enumerate}
 \item $A=$ ``Sacar 2 caras y dos sellos'' \noanswer
 \item $B=$ ``Sacar 3 caras y un sello'' \noanswer
 \item $\overline{A}=$ \noanswer
 \item $A\cup B=$ \noanswer
 \item $A\cap B=$ \noanswer
 \end{enumerate}
 \item Sea el experimento aleatorio ''lanzar un dado``. Halla la probabilidad de los sucesos:
\begin{enumerate}
\item $A_{1}=$"Sacar un número"; $A_{1}=\{$ \noanswer
\item $A_{3}=$"sacar un número menor que 3" \noanswer
\item $A_{2}=$"sacar un número primo" \noanswer
\item $A_{4=}$"sacar un número par mayor que 4" \noanswer
\item $A_{5}=$"sacar un número par o mayor que 4"\noanswer
\end{enumerate}
\item Halla la probabilidad de que al lanzar dos dados aparezca:
\begin{enumerate}
 \item en el primero par y en el segundo mayor que 2 \noanswer
  \newpage
\item en el primero un número impar y en el segundo un múltiplo de 3 \noanswer
\end{enumerate}
\item Calcula la probabilidad de que al lanzar dos dados la suma de sus puntos sea:
\begin{enumerate}
\item 5 \noanswer
\item 7 \noanswer
\item mayor o igual que 10 \noanswer
\item múltiplo de 3 \noanswer
\end{enumerate}
\item En un instituto hay 1.000 alumnos repartidos por cursos de esta forma:

\begin{tabular}{|c|c|c|c|c|}
\hline 
 & Primero & Segundo & Tercero & Cuarto \\ 
\hline 
Chicos & 120 & 100 & 95 & 85 \\ 
\hline 
Chicas & 200 & 150 & 130 & 120 \\ 
\hline 
\end{tabular}

Elegido un alumno al azar, calcula las siguientes probabilidades:
\begin{enumerate}
\item Ser chica \noanswer
\item Ser chico \noanswer
\item Ser alumno de primero \noanswer
\item Ser alumno de segundo \noanswer
\item Ser alumno de tercero \noanswer
\item Ser alumno de cuarto \noanswer
\item Ser chica y alumno de cuarto \noanswer
\item Ser chico y alumno de segundo \noanswer
\end{enumerate}
 \end{enumerate}

\end{document}
