\documentclass[letterpaper,fleqn]{article}
\usepackage[spanish,es-noshorthands]{babel}
\usepackage[utf8]{inputenc} 
\usepackage[papersize={5.5in,8.5in},total={4.5in,7.25in},centering]{geometry}
\usepackage{mathexam}
\usepackage{amsmath}
\usepackage{graphicx}
\usepackage{multicol}

\ExamClass{\includegraphics[height=16pt]{Images/logo-sed.png} Aritmética $6^{\circ}$}
\ExamName{``Nivel. matemáticas''}
\ExamHead{\includegraphics[height=16pt]{Images/logo-colegio.png} IEDAB}
\newcommand{\LineaNombre}{%
\par
\vspace{\baselineskip}
Nombre:\hrulefill \; Curso: \underline{\hspace*{48pt}} \; Fecha: \underline{\hspace*{2.5cm}} \relax
\par}
\let\ds\displaystyle

\begin{document}
\ExamInstrBox{
\emph{No marque ni dañe esta hoja.} Escriba sus respuestas en una hoja anexa. Respuesta sin justificar mediante procedimiento no será tenida en cuenta en la calificación. Tiene 45 minutos para contestar esta prueba.}
%\LineaNombre
\begin{enumerate}
\item Efectúe las siguientes operaciones en el orden indicado. Recuerde que primero se deben efectuar las multiplicaciones y divisiones antes que las adiciones y sustracciones, cuando no se indique lo contrario con paréntesis.
\begin{enumerate}
%\begin{multicols}{2}
\item $25+4\cdot (2+3)=$
\item $4096\div 32=$ 
\item $15+3\cdot 13+24\div 3-3\cdot 4=$ 
\item $5000-2909=$
\item $6+2(4+5\times 3)=$
\item $[15\div (3+2)]\times 5-2=$
%\end{multicols}
\end{enumerate}
 \item ¿Cuál es el valor del dígito en cada caso?
 \begin{enumerate}
% \begin{multicols}{2}
 \item 3 en el número $2'730,480$ 
 \item 5 en el número $3'568,749$  
 %\end{multicols}
 \end{enumerate}
 \item Un número tiene un 1 en las centenas y decenas, un 3 en las unidades de mil, un 2 en las unidades de millón, un 4 en las unidades, un 6 en las centenas de mil y un 8 en las decenas de mil. ¿Cuál es el número?
 \item ¿Cuál es la diferencia entre \$5 millones y \$250,000? 
 \item Un dólar americano valía ayer aproximadamente \$2,945 pesos colombianos. ¿Cuántos dólares americanos se necesitan para cambiar \$80,000 pesos colombianos?
  \item En un colegio hay 1000 estudiantes, en primaria hay 315 y en secundaria 187 más que en primaria. ¿Cuántos estudiantes hay en preescolar?
 \item Número que al sustraerle (restarle) 549 da como resultado 8361:
 \item Halle el cociente y residuo al hacer la siguiente división \;
 $1273\div 25=$
 \item Escriba el número que tiene: 1 unidad, 3 decenas, 2 centenas, 8 unidades de mil y 5 decenas de mil.
 \item Escriba el número que tiene: 3 decenas, 8 unidades de mil, 4 centenas de mil y 9 unidades de millón.
 \end{enumerate}
\end{document}
