\documentclass[10pt,addpoints]{exam}
\usepackage[utf8]{inputenc}
\usepackage[spanish]{babel}
\usepackage{amsmath}
\usepackage{amsfonts}
\usepackage{amssymb}
\usepackage{graphicx}
\usepackage[papersize={6.5in,8.5in},width=5.5in,height=7in]{geometry}
\usepackage{printsudoku}
\extrawidth{.5in}
\pagestyle{headandfoot}
\runningheader{Matemáticas}{}{Autoevaluación III período}
\runningfooter{Final}{Page \thepage\ of \numpages}{Germán Avendaño Ramírez}
\begin{document}
\title{\includegraphics[height=18pt]{Images/logo-colegio.png} Autoevaluaci\'on}
\author{Germ\'an Avendaño Ram\'irez\\Matem\'aticas\\Colegio Arborizadora Baja I.E.D.}
\date{}
\maketitle
\begin{center}
\fbox{\fbox{\parbox{5.5in}{\centering
Queridos estudiantes, deben evaluarse cada aspecto en una escala de 1-5 en el espacio asignado y a la derecha de cada respuesta un compañero debe evaluarles los mismos items de acuerdo a lo siguiente:\\
1 Nunca \hspace*{.75cm}2 Raras ocasiones \hspace*{.75cm}3 Algunas veces\hspace*{.75cm}4 Casi siempre\hspace*{.75cm}5 Siempre}}}
\end{center}
\vspace{0.1in}
\makebox[\textwidth]{Nombres: \underline{\hspace*{8cm}} curso:\underline{\hspace*{1.5cm}} Fecha: \enspace\hrulefill}
\begin{questions}
\question[5]
Presento tareas, talleres y consultas bien realizadas y en el tiempo estipulado para ello

\begin{oneparchoices}
\choice[1] Nunca
\choice[2] Raras ocasiones
\choice[3] Algunas veces
\choice[4] Casi siempre
\choice[5] Siempre 
\end{oneparchoices}
\answerline
\question[5]
Atiendo a las clases con respeto e interés.
\answerline
\question[5]
En las evaluaciones escritas y orales soy lo suficientemente claro y cuando hay lugar a reclamaciones las hago en forma adecuada y siguiendo el conducto regular.
\answerline
\question[5]
No requiero supervisión del docente durante las clases, siempre me responsabilizo de las actividades asignadas
\answerline
\question[5] Mis aportes en las clases son coherentes con los temas tratados.
\answerline
\question[5] Realizo intervenciones en las clases en forma lógica y organizada, respetando las de mis compañeros.
\answerline
\question[5] Cuando sé con anterioridad el tema que se va a tratar llego documentado, demostrando interés por la asignatura.
\answerline
\question[5] Permito que la clase se desarrolle normalmente
\answerline
\question[5] Entiendo con claridad los conceptos básicos de la asignatura tratados durante el período

\begin{oneparchoices}
\choice[1] Nunca
\choice[2] Raras ocasiones
\choice[3] Algunas veces
\choice[4] Casi siempre
\choice[5] Siempre 
\end{oneparchoices}
\answerline
\question[5] Asimilo con facilidad las ideas principales de los temas y los utilizo en la cotidianidad
\answerline
\question[5] Mi comportamiento favorece el buen proceso del conocimiento.
\answerline
\question[5] Por mi propia cuenta profundizo e investigo los temas de la asignatura.
\answerline
\question[5] Comparto mis saberes con mis compañeros, valorando las ideas de los demás.
\answerline
\question[5] Asisto puntualmente a todas las clases y actividades programadas por la institución.
\answerline
\question[5] Tengo buena disposición para escuchar lo que me favorece en la apropiación del conocimiento.
\answerline
\question[5] Atiendo y muestro interés por las explicaciones y conceptos de los profesores y compañeros.
\answerline
\question[5] No me copio de los talleres y trabajos.
\answerline
\question[5] Pregunto en clase cuando tengo dudas
\answerline
\question[5] Cuando me entregan una evaluación inmediatamente corrijo mis errores y la anexo al cuaderno según la indicación del maestro
\answerline
\question[5] Constantemente llevo el registro de mis calificaciones de tal manera que al finalizar el período sé mis notas sin necesidad que el profesor me las diga
\answerline
\end{questions}
Ahora en la siguiente tabla, sume los puntajes que haya obtenido en cada página de este autoevaluación y luego sume estos dos puntajes para obtener su puntaje total. Ahora pida a un compañero que haga una heteroevaluación y la consigne al lado de sus respuestas para que consigne en la otra tabla:

\begin{minipage}{.5\textwidth}
\begin{center}
\gradetable[h][pages]
\end{center} 
\end{minipage}
\begin{minipage}{.5\textwidth}
 \begin{center}
\gradetable[h][pages]
\end{center}
\end{minipage}

\section*{Sudokus}
Solucionar los sudokus de la siguiente página
\renewcommand*{\puzzlefile}{se5.sud}
\writepuzzle%
{.2...5863}{56.2.3.9.}{.3...7251}%
{..975....}{..6..47.9}{.7..286..}%
{6.58...7.}{8....1..6}{3.7.6..4.}%
[SSBL easy 5]
\vfill
\begin{minipage}{0.95\linewidth}\begin{center}
SE5 (easy) \\
\sudoku{se5.sud}
\end{center}\end{minipage}

\renewcommand*{\puzzlefile}{se10.sud}
\writepuzzle%
{..6.497..}{.82.1.6..}{79..8.145}%
{649.5.27.}{..7.6..5.}{.3..72.96}%
{.2....81.}{.7..28...}{......5.7}%
[SSBL easy 10]
\vfill
\begin{minipage}{0.95\linewidth}\begin{center}
SE10 (easy) \\
\sudoku{se10.sud}
\end{center}\end{minipage}

\end{document}