\documentclass[10pt,letterpaper]{article}
\usepackage[utf8]{inputenc}
\usepackage[spanish]{babel}
\usepackage{lmodern}
\usepackage{tikz}
\author{Germán Avendaño Ramírez}
\title{Problemas Olimpiadas}
\begin{document}
\maketitle
\begin{enumerate}
\item Juan dice alternadamente los múltiplos de 7 comenzando en 7 y saltándose cada número que contenga al dígito 3 o que la suma de sus dígitos sea múltiplo de 3. Diana va escribiendo la suma de los números que Juan dice. ¿Cuál es la mayor suma menor a 300 que obtiene Diana?
\item Teresa escribe los siguientes 7 números
\[(2^1+1)2^{2},(2^{2}+1)2^{3},(2^{3}+1)2^{4},(2^{4}+1)2^{5},(2^{5}+1)2^{6},(2^{6}+1)2^{7},(2^{7}+1)2^{8}\]
Hallar el mayor número primo que divide a alguno de estos 7 números.
\item Simón quiere cercar su jardín triangular, el cual tiene delimitado su perímetro por unos arbustos. Si pone la cerca, en forma triangular, hasta antes de los arbustos entonces necesita comprar 57 metros de cerca, pero si pone la cerca, en forma triangular justo después de los arbustos entonces para el lado de la cerca que medía 12 metros ahora necesita 4 metros más. ¿Cuánto metros más de cerca necesita si quiere cercar después de los arbustos en forma triangular?
\paragraph*{Nota:} Puede asumir que todos los arbustos son iguales.
\item Daniel derramó por accidente su bebida sobre la hoja donde había escrito el precio total que le ofrecían dos empresas para un concierto. Tu Boleta y Tiquet Shop. Como no sabía si iban sus primos paternos además de los maternos, había calculado el precio de 1, 13 y 19 boletas que le ofrecía las dos empresas. Con la bebida se borraron el precio de 13 boletas de Tiquet Shop y de 19 boletas de Tu Boleta. Los valores de Tiquet Shop fueron de \$3\,700 para una boleta y \$36\,100 para diecinueve boletas. Los valores de Tu Boleta fueron de \$3\,250 para una boleta y \$26\,650 pra trece boletas. Como los valores escritos incluían un valor fijo de envío de cada empresa, sin importar la cantidad de boletas que compre, él pudo averiguar dónde le quedaba mejor comprar las boletas si por lo menos un grupo de sus primos fuera. Al final en total fueron nueve personas. ¿Cuántos pesos se ahorraron en total cuando compraron las boletas?
\item El trapecio de la figura está formado por tres triángulos rectángulos como se muestra. Hallar el perímetro del trapecio.
\begin{tikzpicture}

\end{tikzpicture}
\draw (0,0)--(6,0);
\draw (0,0)--(6,3.2);
\draw (6,0)--(6,7.7);
\draw (6,3.2)--(3.6,7.7);
\draw (0.0,0.0)--(3.6,7.7);
\end{tikzpicture}
\end{enumerate}
\end{document}