\documentclass[fleqn]{article}
\usepackage[spanish,es-noshorthands]{babel}
\usepackage[utf8]{inputenc} 
\usepackage[papersize={6.5in,8.5in},total={5.5in,7.25in},centering]{geometry}
\usepackage{mathexam}
\usepackage{amsmath}
\usepackage{graphicx}
\usepackage{multicol}

\ExamClass{\includegraphics[height=16pt]{Images/logo-sed.png} Cálculo $11^{\circ}$}
\ExamName{"Progresiones y sucesiones"}
\ExamHead{\includegraphics[height=16pt]{Images/logo-colegio.png} IEDAB}
\newcommand{\LineaNombre}{%
\par
\vspace{\baselineskip}
Nombre:\hrulefill \; Curso: \underline{\hspace*{48pt}} \; Fecha: \underline{\hspace*{2.5cm}} \relax
\par}
\let\ds\displaystyle

\begin{document}
\ExamInstrBox{
Respuesta sin justificar mediante procedimiento no será tenida en cuenta en la calificación. Escriba sus respuestas en el espacio indicado. Tiene 55 minutos para contestar esta prueba.}
\LineaNombre
\section*{Para recordar}
Una progresi\'on aritmética tiene como término general \fbox{$a_{n}=a_{1}+(n-1)d$}, donde $d$ es la distancia o diferencia que hay entre dos términos consecutivos.\\

Una progresión geométrica tiene como término genral \fbox{$a_{n}=a_{1}r^{n-1}$},
donde $r$ es la razón geométrica.
\begin{enumerate}
\item Halle el término siguiente en las sucesiones indicadas;
  \begin{enumerate}
    \begin{multicols}{2}
    \item --8, --14, --20, --26, \ldots \noanswer
    \item --6, --3, 0, 3, 6, \ldots \noanswer
    \item $-3$, 0, 3, 6, 9 \ldots \noanswer
    \item --5, --1, 3, 7, 11 \ldots \noanswer
    \end{multicols}
\end{enumerate}
\item Encuentre:
\begin{enumerate}
\item El tercer término de la sucesión cuyo primer término $a_1=3$ y su término n-ésimo definido por recurrencia es $a_n=a_{n-1}-14$  \noanswer[.75in]
\item El octavo término de la sucesión cuyo término general es $a_n=-6+5(n-1)$ \noanswer[.75in]
\end{enumerate}
\newpage
\item Halle el término general $a_{n}$ de una progresión aritmética
\begin{enumerate}
\item cuyo primer término es 5 y su diferencia $d$ es $-2$. \noanswer
\item cuyo primer término es 3 y su segundo término es 7.\noanswer
\end{enumerate}
\item En una granja hay 65 pollos y cada día nacen 25. ¿cuántos habrá al cabo de 30 días si no muere ninguno?\noanswer
\item Cada día me duplican el dinero que tengo y me dan 2 dólares más. Si el primer día tengo 25 dólares, construya la sucesión que indica el dinero que tengo cada día. Hágalo para una semana.\noanswer
\end{enumerate}
\end{document}