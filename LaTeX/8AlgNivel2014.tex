\documentclass[letterpaper,fleqn]{article}
\usepackage[spanish,es-noshorthands]{babel}
\usepackage[utf8]{inputenc} 
\usepackage[left=1cm, right=1cm, top=1.5cm, bottom=1.7cm]{geometry}
\usepackage{mathexam}
\usepackage{amsmath}
\usepackage{graphicx}
\usepackage{tikz,pgf}

\ExamClass{\includegraphics[height=16pt]{Images/logo-sed.png} Álgebra $8^{\circ}$}
\ExamName{Nivelación 2014}
\ExamHead{\includegraphics[height=16pt]{Images/logo-colegio.png} IEDAB}
\newcommand{\LineaNombre}{%
\par
\vspace{\baselineskip}
Nombre:\hrulefill \; Curso: \underline{\hspace*{48pt}} \; Fecha: \underline{\hspace*{2.5cm}} \relax
\par}
\let\ds\displaystyle

\begin{document}
\ExamInstrBox{
Respuesta sin justificar mediante procedimiento no será tenida en cuenta en la calificación. Escriba sus respuestas en el espacio indicado. Tiene 60 minutos para contestar esta prueba.}
\LineaNombre
\begin{enumerate}
 \item Para los ejercicios \ref{third}--\ref{fourth}, simplifique cada expresión numérica
 \begin{enumerate}
 \item $-8\frac{1}{4}+\left(-4\frac{5}{8}\right)-\left(-6\frac{3}{8}\right)$\label{third}\noanswer
\item $-8(2)-16\div (-4)+(-2)(-2)$\noanswer
\item  $[4(-1)-2(3)]^{2}$\label{fourth}\noanswer
 \end{enumerate}
 \item Simplifique cada expresi\'{o}n algebraica reduciendo términos semejantes
 \begin{enumerate}
 \item $3a^{2}-2b^{2}-7a^{2}-3b^{2}$\noanswer
\item $-\frac{2}{3}x^{2}y-\left(-\frac{3}{4}x^{2}y\right)-\frac{5}{12}x^{2}y-2x^{2}y$\noanswer
\item $-5(x^{2}-4)-2(3x^{2}+6)+(2x^{2}-1)$\noanswer
 \end{enumerate}
 \item Resuelva los siguientes problemas
 \begin{enumerate}
 \item Encuentre tres enteros consecutivos tal que la suma de la mitad del menor y un tercio del mayor es uno menos que el entero del medio.\noanswer
 \item Si el complemento de un ángulo es una d\'{e}cima parte del suplemento del \'{a}ngulo, encuentre la medida del \'{a}ngulo\noanswer
 \end{enumerate}
 \begin{minipage}{.45\textwidth}
\item Encuentre un polinomio que represente el área de la región sombreada
\end{minipage}\hfill
\begin{minipage}{.45\textwidth}
\begin{tikzpicture}[scale=1.5]
\filldraw[fill=gray!25,even odd rule]
(-1,-1) rectangle (2,0)
(0.66,-0.33) rectangle (1.66,-0.66);
\node[below] at (1,-1){$3x+4$};
\node[right] at (2,-0.5){$x$};
\node[left] at (0.66,-0.5){$x-2$};
\node[above] at (1,0){$x-1$}; 
\end{tikzpicture}
\end{minipage}\noanswer
\begin{minipage}{.45\textwidth}
\item Encuentre el polinomio que represente el volumen del sólido rectangular de la figura.
\end{minipage}\hfill
\begin{minipage}{.45\textwidth}
\begin{tikzpicture}[scale=1.5]
\draw (0,0,0)--(2,0,0)--(2,1,0)--(0,1,0)--cycle;
\draw (0,0,1)--(2,0,1)--(2,1,1)--(0,1,1)--cycle;
\draw (0,0,0) -- (0,0,1);
\draw (2,0,0) -- (2,0,1);
\draw (2,1,0) -- (2,1,1);
\draw (0,1,0) -- (0,1,1);
\node[below] at (1,0,1){$2x$};
\node[left] at (0,.4,1){$x$};
\node[left] at (0,1,.5){$x+1$};
\end{tikzpicture}
\end{minipage}
\noanswer
\newpage
\item Efectúe las operaciones indicadas
\begin{enumerate}
\item $(3x-2)+(4x-6)+(-2x+5)$\noanswer
\item $[8x-(5x-y+3)]-[-4y-(2x+1)]$\noanswer
\item $(-2a^{2})(3ab^{2})(a^{2}b^{3})$\noanswer
\item $-2x^{3}(4x^{2}-3x-5)$\noanswer
\item $\dfrac{20a^{4}b^{6}}{5ab^{3}}$\noanswer
\item $(7x-9)(x+4)$\noanswer
\end{enumerate}
\item Factorice los siguientes polinomios
\begin{enumerate}
\item $10a^{2}b-5ab^{3}-15a^{3}b^{2}$\noanswer
\item $a(x+4)+b(x+4)$\noanswer
\item $mn+5n^{2}-4m-20n$\noanswer
\item $36x^{2}-y^{2}$\noanswer
\end{enumerate}
 \end{enumerate}

\end{document}
