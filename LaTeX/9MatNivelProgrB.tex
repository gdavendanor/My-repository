\documentclass[fleqn]{article}
\usepackage[spanish,es-noshorthands]{babel}
\usepackage[utf8]{inputenc} 
\usepackage[papersize={6.5in,8.5in},left=1cm, right=1cm, top=1.5cm, bottom=1.7cm]{geometry}
\usepackage{mathexam}
\usepackage{amsmath}
\usepackage{graphicx}
\usepackage{multicol}
\usepackage{printsudoku}

\ExamClass{\includegraphics[height=16pt]{Images/logo-sed.png} Matemáticas $9^{\circ}$}
\ExamName{`Nivelación Progresiones'}
\ExamHead{\includegraphics[height=16pt]{Images/logo-colegio.png} IEDAB}
\newcommand{\LineaNombre}{%
\par
\vspace{\baselineskip}
Nombre:\hrulefill \; Curso: \underline{\hspace*{48pt}} \; Fecha: \underline{\hspace*{2.5cm}} \relax
\par}
\let\ds\displaystyle

\begin{document}
\ExamInstrBox{
Respuesta sin justificar mediante procedimiento no será tenida en cuenta en la calificación. Escriba sus respuestas en el espacio indicado. Tiene 45 minutos para contestar esta prueba.}
\LineaNombre
\begin{enumerate}
 \item Halle los dos términos siguientes en las siguientes sucesiones:
 \begin{enumerate}
 \begin{multicols}{2}
 \item 5, 8, 11, 14, 17, \underline{\hspace*{12pt}}, \underline{\hspace*{12pt}}
 \item 11, 8, 5, 2, \underline{\hspace*{12pt}}, \underline{\hspace*{12pt}}
 \item $1$, $\frac{3}{8}$, $\frac{5}{27}$, $\frac{7}{64}$, \underline{\hspace*{12pt}}, \underline{\hspace*{12pt}} 
  \item 1, 4, 9, 16, 25, \underline{\hspace*{12pt}}, \underline{\hspace*{12pt}}
 \end{multicols}
 \end{enumerate}
 \item Determine si las siguientes sucesiones son progresiones y las que lo sean, indique si son aritméticas o geométricas. Halle el término n-ésimo para las que sean progresiones.
 \begin{enumerate}
 \item --2, 6, 14, 22, 30, \ldots \noanswer
 \item 1, 2, 4, 8, \ldots \noanswer
 \item 4, 1, $\frac{1}{4}$, $\frac{1}{16}$, \ldots \noanswer
 \item --1, 2, 7, 14, 23, \ldots \noanswer
 \end{enumerate}
 \item Halle los siete primeros términos de las progresiones cuyo término n-ésimo es:
 \begin{enumerate}
 \item $a_{n}=3n-7$\noanswer[0.25in]
 \item $a_{n}=6(\frac{1}{2})^{n-1}$ \noanswer[0.25in]
 \end{enumerate}
 \newpage
 \item Sabiendo que la suma de $n$ términos en una progresión aritmética es: $S_{n}=\dfrac{(a_{1}+a_{n})n}{2}$, halle la suma de los 90 primeros números pares.\noanswer
 \item Se deja caer una pelota desde una altura de 18 metros. Cada vez que rebota en el suelo alcanza una altura igual a los 2/3 de la altura anterior. Construye la sucesión que nos da la altura alcanzada tras los sucesivos rebotes. ¿Qué tipo de sucesión es? Halla el término general.\noanswer
 \end{enumerate}
 \paragraph*{Punto extra-bonificación:} Solucione el siguiente sudoku
 \cluefont{\Large}
\cellsize{1.5\baselineskip}
 \renewcommand*{\puzzlefile}{se5.sud}
\begin{center}
\begin{minipage}{0.45\linewidth}\begin{center}
SE20 (easy) \\
\sudoku{se20.sud}
\end{center}\end{minipage}
\end{center}
\end{document}
