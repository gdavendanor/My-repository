\documentclass[fleqn]{article}
\usepackage[spanish,es-noshorthands]{babel}
\usepackage[utf8]{inputenc} 
\usepackage[papersize={6.5in,8.5in},left=1cm, right=1cm, top=1.5cm, bottom=1.7cm]{geometry}
\usepackage{mathexam}
\usepackage{amsmath}
\usepackage{graphicx}

\ExamClass{\includegraphics[height=16pt]{Images/logo-sed.png} Matemáticas $6^{\circ}$}
\ExamName{Recomendaciones I, Sustentación}
\ExamHead{\includegraphics[height=16pt]{Images/logo-colegio.png} IEDAB}
\newcommand{\LineaNombre}{%
\par
\vspace{\baselineskip}
Nombre:\hrulefill \; Curso: \underline{603} \; Fecha: \underline{\hspace*{2.5cm}} \relax
\par}
\let\ds\displaystyle

\begin{document}
\ExamInstrBox{
Respuesta sin justificar mediante procedimiento no será tenida en cuenta en la calificación. Escriba sus respuestas en el espacio indicado. Tiene 45 minutos para contestar esta prueba.}
\LineaNombre
\begin{enumerate}
   \item Ordene los números usando < o > según el caso: 
      \begin{enumerate}
	 \item De menor a mayor los siguientes números: 4050, 4500, 4005, 4555, 40005\noanswer
	 \item De mayor a menor los siguientes números: 6040, 6400, 64000, 6004, 60400, 60404\noanswer
      \end{enumerate}
   \item Busca el término desconocido e indica su nombre en las siguientes operaciones:
   \begin{enumerate}
   \item $329+\underline{\hspace*{48pt}}=1206$\noanswer
   \item $\underline{\hspace*{60pt}}-4208=524$\noanswer
   \item $324\times \underline{\hspace*{60pt}}=15552 $\noanswer
   \item $21216\div \underline{\hspace*{60pt}}=624$\noanswer
   \end{enumerate}
   \item Juan tiene 15 años más que su primo Pedro. Pedro tiene 18 años más que su hermano Simón. Si Simón tiene 14 años, ¿cuántos años tienen entre los 3?\noanswer[2in]
   \newpage
   \item El domingo salí de casa con una cierta cantidad de dinero. Pagué \$7500 en la
entrada del cine y me compré dos paquetes de papas fritas a cinco \$850 cada uno y un
jugo de \$1250. Cuando llegué a casa tenía \$2400. ¿Con cuánto dinero salí
de casa?\noanswer
\item Un agricultor recogió 24\,364 kilos de peras. El primer día vendió la mitad. De la otra
mitad, se le estropearon 450 kilos. ¿Cuántos kilos le quedaron para vender el segundo
día?\noanswer
\item Un niño está de vacaciones y envía cartas a sus 6 amigos, en cada carta pone 6 postales y en cada postal un sello que vale 6 dólares. ¿Cuántos dólares se ha gastado en sellos?\noanswer
\end{enumerate}
\end{document}
