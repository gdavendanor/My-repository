\documentclass[letterpaper,11pt,twoside]{article}
\usepackage[utf8]{inputenc}
\usepackage{amsmath,amsfonts,amssymb,amsthm,latexsym}
\usepackage[spanish,es-noshorthands]{babel}
\usepackage[T1]{fontenc}
\usepackage{lmodern}
\usepackage{graphicx,hyperref}
\usepackage{tikz,pgf}
\usepackage{multicol}
\usepackage{fancyhdr}
\usepackage[height=9.5in,width=7in]{geometry}
\usepackage{fancyhdr}
\pagestyle{fancy}
\fancyhead[LE]{matematicas.german@gmail.com}
\fancyhead[RE]{}
\fancyhead[RO]{\url{https://www.autistici.org/mathgerman}}
\fancyhead[LO]{}

\author{Germ\'an Avenda\~no Ram\'irez~\thanks{Lic. Mat. U.D., M.Sc. U.N.}}
\title{\begin{minipage}{.2\textwidth}
\includegraphics[height=1.75cm]{Images/logo-colegio.png}\end{minipage}
\begin{minipage}{.55\textwidth}
\begin{center}
Taller 11, luchar si paga\\
Ética $9^{\circ}$
\end{center}
\end{minipage}\hfill
\begin{minipage}{.2\textwidth}
\includegraphics[height=1.75cm]{Images/logo-sed.png} 
\end{minipage}}
\date{}
\thispagestyle{plain}
\begin{document}
\maketitle
Nombre: \hrulefill Curso: \underline{\hspace*{44pt}} Fecha: \underline{\hspace*{2.5cm}}
\begin{multicols}{2}
 \section*{A propósito del paro de la comunidad educativa}
 Los maestros y maestras de Colombia, representados en la Fecode~\footnote{Federación Colombiana de Trabajadores de la Educación}, la cual es la federación que agrupa a los sindicatos regionales de maestros y maestras presentamos un pliego de peticiones que contenía los siguientes puntos:
 \begin{enumerate}
 \item \textbf{Nivelación salarial}: Hacer efectivo el proceso de \emph{nivelación salarial} acordado con el presidente de la República en mayo del año 2014
 \item \textbf{Carrera docente:} 
 \begin{enumerate}
 \item  Consensuar el nuevo sistema de ASCENSO y REUBICACIÓN de nivel salarial docentes del Decreto 1278 de 2002
 \item Estímulo económico para el grado 14 (D. 2277) que tengan especialización, maestrías y doctorados, acorde con el tiempo acreditado. Reajuste salarial para los docentes etno-educadores.
\end{enumerate}  
 \item \textbf{Política educativa}: Como el gobierno propone implementar la jornada única fijando como plazo el año 2030, con lo cual estamos de acuerdo si se implementa conforme a la ley 115 de 1994, para lo cual se requiere:
 \begin{enumerate}
 \item Infraestructura
 \item Relación técnica de estudiante/docente según las normas NTC 4595 de 1999 del INCONTEC.
\item Dotación, infraestructura, alimentación, transporte escolar, salario profesional, PREESCOLAR DE 3 GRADOS
\item Desmonte inmediato de la modalidad tutores Sena.
 \end{enumerate}
 \item \textbf{Salud digna}: Salud digna. Un servicio de salud que se corresponda con el pliego de condiciones Con-
tratado.\\
Inclusión de los contenidos del pliego de condiciones de los
contratos de salud en un acto administrativo que garantice su
vigencia, permanencia y ajustes de acuerdo a las exigencias sanitarias y satisfacción de las nuevas necesidades. Que el nuevo pliego contemple los ajustes normativos conducentes a garantizar un servicio de calidad\ldots
 \end{enumerate}
\end{multicols}


\end{document}
