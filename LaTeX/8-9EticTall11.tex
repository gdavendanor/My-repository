\documentclass[letterpaper,11pt,twoside]{article}
\usepackage[utf8]{inputenc}
\usepackage{amsmath,amsfonts,amssymb,amsthm,latexsym}
\usepackage[spanish,es-noshorthands]{babel}
\usepackage[T1]{fontenc}
\usepackage{lmodern}
\usepackage{graphicx,hyperref}
\usepackage{tikz,pgf}
\usepackage{multicol}
\usepackage{fancyhdr}
\usepackage[height=9.5in,width=7in]{geometry}
\usepackage{fancyhdr}
\pagestyle{fancy}
\fancyhead[LE]{matematicas.german@gmail.com}
\fancyhead[RE]{}
\fancyhead[RO]{\url{https://www.autistici.org/mathgerman}}
\fancyhead[LO]{}

\author{Germ\'an Avenda\~no Ram\'irez~\thanks{Lic. Mat. U.D., M.Sc. U.N.}}
\title{\begin{minipage}{.2\textwidth}
\includegraphics[height=1.75cm]{Images/logo-colegio.png}\end{minipage}
\begin{minipage}{.55\textwidth}
\begin{center}
Taller 11, luchar si paga\\
Ética $9^{\circ}$
\end{center}
\end{minipage}\hfill
\begin{minipage}{.2\textwidth}
\includegraphics[height=1.75cm]{Images/logo-sed.png} 
\end{minipage}}
\date{}
\thispagestyle{plain}
\begin{document}
\maketitle
Nombre: \hrulefill Curso: \underline{\hspace*{44pt}} Fecha: \underline{\hspace*{2.5cm}}
\begin{multicols}{2}
 \section*{A propósito del paro de la comunidad educativa}
 Los maestros y maestras de Colombia, representados en la Fecode~\footnote{Federación Colombiana de Trabajadores de la Educación}, federación que agrupa a los sindicatos regionales de maestros y maestras presentamos un pliego de peticiones el 26 de febrero del presente que contenía los siguientes puntos:
 \begin{enumerate}
 \item \textbf{Nivelación salarial}: Hacer efectivo el proceso de \emph{nivelación salarial} acordado con el presidente de la República en mayo del año 2014
 \item \textbf{Carrera docente:} 
 \begin{enumerate}
 \item  Consensuar el nuevo sistema de ASCENSO y REUBICACIÓN de nivel salarial docentes del Decreto 1278 de 2002
 \item Estímulo económico para el grado 14 (D. 2277) que tengan especialización, maestrías y doctorados, acorde con el tiempo acreditado. Reajuste salarial para los docentes etno-educadores.
\end{enumerate}  
 \item \textbf{Política educativa}: Como el gobierno propone implementar la jornada única fijando como plazo el año 2030, con lo cual estamos de acuerdo si se implementa conforme a la ley 115 de 1994, para lo cual se requiere:
 \begin{enumerate}
 \item Infraestructura
 \item Relación técnica de estudiante/docente según las normas NTC 4595 de 1999 del INCONTEC.
\item Dotación, infraestructura, alimentación, transporte escolar, salario profesional, PREESCOLAR DE 3 GRADOS
\item Desmonte inmediato de la modalidad tutores Sena.
 \end{enumerate}
 \item \textbf{Salud digna}: Salud digna. Un servicio de salud que se corresponda con el pliego de condiciones Contratado.\\
Inclusión de los contenidos del pliego de condiciones de los
contratos de salud en un acto administrativo que garantice su
vigencia, permanencia y ajustes de acuerdo a las exigencias sanitarias y satisfacción de las nuevas necesidades. Que el nuevo pliego contemple los ajustes normativos conducentes a garantizar un servicio de calidad\ldots
\item \textbf{Bienestar}:
\begin{itemize}
\item Recursos para construcción de la sede de la Federación
\item Continuidad a los juegos del magisterio, cultura y folclor
\item Garantías sindicales
\end{itemize}
Como punto adicional
\item Cumplimiento de acuerdos firmados entre los cuales se destacan
\begin{itemize}
\item Suspensión o congelamiento de las politicas de privatización de la educación. Desmonte de los procesos
generados por la aplicación de la Ley 1294 de 2009 y su Decreto reglamentario 2355 de 2009 (colegios en concesión y banco de oferentes)
\item Organizar institucionalmente la campaña “La escuela territorio de paz”. El tratamiento y las estrategias a
proponer y trabajar en el período del post-conflicto (las escuelas y la educación en el post-conflicto).
\end{itemize}
 \end{enumerate}
 Luego de radicado este pliego de peticiones, se mantuvieron negociaciones con el gobierno hasta el 20 de abril, tiempo durante el cual la ministra hizo presencia solamente un día durante 40 minutos y se limitó entonces a enviar negociadores que no tenían suficiente poder de negociación.
\section*{Actividad}
Lea con atención lo anterior para poder desarrollar las actividades siguientes:
\begin{enumerate}
\item ¿Qué quiere decir FECODE?
\item ¿Cuánto días duró el proceso de negociación anterior al Paro, luego de radicado el pliego de peticiones del magisterio?
\item ¿Qué escuchó, vió o leyó en noticias acerca del paro?
\item ¿Cree que son justas las reclamaciones de los docentes?
\item ¿Por qué cree que este paro nos debía convocar a todos, los estudiantes, los padres de familia y los maestros y maestras?
\item La ministra para desprestigiar a los maestros y maestras aseguró que en promedio los maestros nos ganamos \$2'500\,000 mensuales, lo cual resultó siendo evidentemente falso. Para ello, a continuación se presenta la tabla salarial con el número de maestros en cada grado y nivel salarial, complete la tabla, diligenciando la 4 columna y determine el promedio salarial de los maestros del Decreto 1278.
\begin{enumerate}
\item Halle el número total de maestros del decreto 1278
\item Halle el salario total pagado a todos los maestros en cada categoría y nivel salarial
\item Sume el total de salarios que se pagan 
\item Divida el total de salarios entre el número total de maestros del decreto 1278.
\item En cuánto difiere el promedio real del promedio que dió la ministra.
\end{enumerate}
\begin{tabular}{|c|r|c|p{2cm}|}
\hline 
Grado & \# maestros & Salario  & \# maestros $\times$ Salario \\ 
\hline 
1A & 18\,397 & \$1'121\,819 &  \\ 
\hline 
1B & 1\,321 & \$1'430\,005 &  \\ 
\hline 
1C & 200 & \$1'843\,384 &  \\ 
\hline 
1D & 34 & \$2'285\,199 &  \\ 
\hline 
2A & 75\,407 & \$1'411\,890 &  \\ 
\hline 
2A Esp. & 19\,878 & \$1'534\,628 &  \\ 
\hline 
2A Mag. & 2\,250 & \$1'623\,873 &  \\ 
\hline 
2A Doc. & 2 & \$1'835\,000 &  \\ 
\hline 
2B & 9\,757 & \$1'960\,000 &  \\ 
\hline 
2B Esp. & 7\,270 & \$1'960\,718 &  \\ 
\hline 
2B Mag. & 882 & \$2'212\,532 &  \\ 
\hline 
2B Doc. & 1 & \$2'398\,254 &  \\ 
\hline 
2C & 2\,257 & \$2'154\,714 &  \\ 
\hline 
2C Esp. & 2\,070 & \$2'429\,075 &  \\ 
\hline 
2C Mag. & 245 & \$2'477\,921 &  \\ 
\hline 
2C Doc. & 1 & \$2'801\,128 &  \\ 
\hline 
2D & 385 & \$2'574\,881 &  \\ 
\hline 
2D Esp. & 400 & \$2'874\,648 &  \\ 
\hline 
2D Mag. & 38 & \$2'961\,113 &  \\ 
\hline 
2D Doc & 0 & \$3'347\,345 &  \\ 
\hline 
3A & 1702 & \$2'363\,041 &  \\ 
\hline 
3A Doc. & 16 & \$3'134\,755 &  \\ 
\hline 
3B & 998 & \$2'797\,931 &  \\ 
\hline 
3B Doc. & 10 & \$3'679\,813 &  \\ 
\hline 
3C & 571 & \$3'460\,354 &  \\ 
\hline 
3C Doc. & 3 & \$4\,646\,663 &  \\ 
\hline 
3D & 156 & \$4'009\,527 &  \\ 
\hline 
3D Doc. & 4 & \$5'334\,216 &  \\ 
\hline 
Total &  &  &  \\ 
\hline 
\end{tabular} \footnote{Estudio hecho por John Alexander Granados Rico, asesor CEID-FECODE}
\item Para consultar en su casa
\begin{enumerate}
\item ¿Quién es la ministra de educación?
\item ¿Quién es el ministro de hacienda?
\item ¿Quién es el presidente de Fecode?
\item ¿Cuántos maestros que trabajan en el sector oficial hay en colombia?
\item Con base en el resultado de la tabla, calcule el número de maestros que hay del decreto 2277
\end{enumerate}
\end{enumerate}
\end{multicols}
\end{document}
