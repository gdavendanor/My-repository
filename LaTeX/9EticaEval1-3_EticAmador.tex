\documentclass[10pt,addpoints]{exam}
\usepackage[utf8]{inputenc}
\usepackage[spanish,es-noshorthands]{babel}
\usepackage{hyperref}
\usepackage{amsmath}
\usepackage{amsfonts}
\usepackage{amssymb}
\usepackage{graphicx}
\usepackage{tikz}
\usepackage{multicol}
\usepackage[papersize={5.5in,8.5in},top=.75cm,bottom=.75cm,left=.75cm,right=.75cm]{geometry}
%\printanswers
\begin{document}
\title{\begin{minipage}{.2\textwidth}
        \includegraphics[height=1.75cm]{Images/logo-colegio.png}
       \end{minipage}
\begin{minipage}{.55\textwidth}
 \begin{center}
Ética para Amador \\Ética $9^{\circ}$
\end{center}
\end{minipage}
\begin{minipage}{.2\textwidth}
\includegraphics[height=1.75cm]{Images/logo-sed.png} 
\end{minipage}
}
\author{Germ\'{a}n Avendaño Ram\'{i}rez\\Lic. Matemáticas U.D., M.Sc. U.N.}
\date{}
\maketitle
\begin{center}
\fbox{\fbox{\parbox{4.75in}{\centering
Conteste en el espacio asignado para tal fin}}}
\end{center}
\vspace{0.1in}
\makebox[\textwidth]{Nombres: \hrulefill, curso:\underline{\hspace{48pt}}, fecha:\underline{\hspace{3cm}}}
\begin{questions}
\question
Haga una breve síntesis de cada uno de los tres primeros capítulos del libro Ética para Amador.
\begin{parts}
\part Capítulo 1 \fillwithdottedlines{2in}
\part Capítulo 2 \fillwithdottedlines{2in}
\part Capítulo 3 \fillwithdottedlines{2in}
\end{parts}
\question Conteste las siguientes preguntas:
\begin{parts}
\part La palabra moral viene del vocablo latino \emph{mores} que significa \fillwithdottedlines{.25in}
\part ¿Que diferencia se puede establecer entre una orden, una construmbre y un capricho? \fillwithdottedlines{1in}
\part ¿Cuál es la diferencia entre el comportamiento de Héctor, el personaje de la Iliada de Homero y las termitas del ejemplo del autor? \fillwithdottedlines{1in}
\end{parts}
\question ¿Para qué le puede servir lo leído en estos tres primeros capítulos del libro Ética para Amador?
\fillwithdottedlines{1in}
\end{questions}
\end{document}