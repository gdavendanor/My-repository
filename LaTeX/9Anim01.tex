\documentclass[letterpaper,11pt,twoside]{article}
\usepackage[utf8]{inputenc}
\usepackage{amsmath,amsfonts,amssymb,amsthm,latexsym}
\usepackage[spanish,es-noshorthands]{babel}
\usepackage[T1]{fontenc}
\usepackage{lmodern}
\usepackage{graphicx,hyperref}
\usepackage{tikz,pgf}
\usepackage{multicol}
\usepackage{fancyhdr}
\usepackage[height=9.5in,width=7in]{geometry}
\usepackage{fancyhdr}
\pagestyle{fancy}
\fancyhead[LE]{matematicas.german@gmail.com}
\fancyhead[RE]{}
\fancyhead[RO]{\url{https://www.autistici.org/mathgerman}}
\fancyhead[LO]{}

\author{Germ\'an Avenda\~no Ram\'irez~\thanks{Lic. Mat. U.D., M.Sc. U.N.}}
\title{\begin{minipage}{.2\textwidth}
\includegraphics[height=1.75cm]{Images/logo-colegio.png}\end{minipage}
\begin{minipage}{.55\textwidth}
\begin{center}
Animaplano 01
%Tema\\
Matemáticas $9^{\circ}$
\end{center}
\end{minipage}\hfill
\begin{minipage}{.2\textwidth}
\includegraphics[height=1.75cm]{Images/logo-sed.png} 
\end{minipage}}
\date{}
\thispagestyle{plain}
\begin{document}
\maketitle
Nombre: \hrulefill Curso: \underline{\hspace*{44pt}} Fecha: \underline{\hspace*{2.5cm}}
\section*{Cuestionario}
\begin{enumerate}
\item El doble del quinto n\'{u}mero primo
\item El noveno n\'{u}mero primo
\item La longitud de un rect\'{a}ngulo ($x$) cuyo per\'{i}metro es 102 y cuya longitud es el doble del ancho.
\item En años, 8 décadas menos 6 años
\item El triple del triple de 7
\item El valor de $x$ para que se de una proporci\'{o}n en $\frac{x}{3}=\frac{159}{9}$
\item El cuádruple del quinto número primo
\item $2^{6}-\frac{27}{9}$
\item El mínimo común múltiplo entre 4 y 13
\item El número de minutos en una hora
\item $2\cdot 5^{2}$
\item El undécimo número primo
\item El m.c.m entre 11 y 2
\item El M.C.D entre 48 y 18
\item El número de lados de un pentágono
\item La altura de un poste cuya sombra mide 21 m. mientras a la misma hora una persona que mide 1.70 m. proyecta una sombre de 2.55 m.
\item El M.C.D entre 18 y 15
\item El 6 número primo
\item El mínimo común múltiplo entre 5 y 3
\item No de la avenida en Bogotá que se denomina también avenida El Dorado
\item Los kilómetros que recorre un ciclista en 3 minutos rodando a una rapidez media de 15km/h.
\item El área de un rectángulo de largo 13 y ancho 5
\item $(3^{2})^{2}+2^{3}=$
\item $\sqrt{100}^{2}-1=$
\item Número de dos cifras que suman 14 y su diferencia es 0.
\item En minutos, $\frac{3}{4}$ de hora
\item 
\end{enumerate}

\end{document}
