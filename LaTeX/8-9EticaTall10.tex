\documentclass[10pt,twoside]{article}
\usepackage[utf8]{inputenc}
\usepackage{amsmath}
\usepackage{amsfonts}
\usepackage{amssymb}
\usepackage[spanish,es-noshorthands]{babel}
\usepackage[T1]{fontenc}
\usepackage{lmodern}
\usepackage{graphicx,hyperref}
\usepackage{tikz,pgf}
\usepackage{multicol}
\usepackage{subfig}
\usepackage[papersize={6.5in,8.5in},width=5.5in,height=7in]{geometry}
\usepackage{sudoku}
\usepackage{fancyhdr}
\pagestyle{fancy}
\fancyhead[LE]{\url{www.autistici.org/mathgerman}}
\fancyhead[RE]{}
\fancyhead[RO]{matematicas.german@gmail.com}
\fancyhead[LO]{}

\author{Germ\'an Avenda\~no Ram\'irez~\thanks{Lic. Mat. U.D., M.Sc. U.N.}}
\title{\begin{minipage}{.2\textwidth}
\includegraphics[height=1.75cm]{Images/logo-colegio.png}\end{minipage}
\begin{minipage}{.55\textwidth}
\begin{center}
Taller 10\\
Respeto por la diferencia\\
Ética $9^{\circ}$
\end{center}
\end{minipage}\hfill
\begin{minipage}{.2\textwidth}
\includegraphics[height=1.75cm]{Images/logo-sed.png} 
\end{minipage}}
\date{}
\begin{document}
\maketitle
Nombre: \hrulefill Curso: \underline{\hspace*{44pt}} Fecha: \underline{\hspace*{2.5cm}}
\section*{Aplico lo aprendido}
\begin{enumerate}
\item Lee la siguiente noticia con detenimiento. Luego, resuelve las
actividades.
\end{enumerate}
Capacitar a profesores y alumnos de colegio para tratar a niño especial ordenó Corte Constitucional.

El alto tribunal falló una acción de tutela a favor del menor de 11 años, que sufre de un leve problema mental y fue retirado de un colegio de Barrancabermeja (Santander).

La Corte dio un plazo de 48 horas a la Alcaldía de esa ciudad
para que diseñe un programa para capacitar en atención de niños especiales a la totalidad de alumnos y profesores de primaria y secundaria. El menor, según la valoración hecha por los médicos, presenta un síndrome de hiperactividad y de atención. Su madre lo matriculó en la concentración escolar Luis Carlos Galán, que ofrecía un programa especial para niños con discapacidad o problemas de aprendizaje. El programa atendía un grupo de 12 infantes bajo la
orientación de una maestra especializada, que durante tres años los dirigió con buenos resultados y logró promoverlos de kínder al grado primero de primaria. Pero el proyecto se acabó y el niño fue asignado entonces a un curso normal. Días después, la madre del niño fue llamada por la directora de grupo, quien le comunicó que no podía tener más a su hijo. Según la maestra, golpeaba a sus compañeros de clase y les quitaba sus cosas. La Secretaría de Educación reubicó al menor en otro colegio cercano a su residencia, pero allí se presentó la misma situación, lo que llevó a la familia
a dejarlo en su casa. La madre puso una tutela contra la Alcaldía exigiendo que reactivaran el programa para niños especiales que funcionaba en el Luis Carlos Galán. Según ella, su situación económica le impedía inscribir a su hijo en una institución privada y la decisión de la administración local violaba el derecho del menor a la educación, a la vida digna y a la igualdad. El caso fue estudiado por el Juzgado Primero Civil de Barrancabermeja, que rechazó la solicitud. Consideró que la Alcaldía actuó de acuerdo con lo establecido por el Ministerio de Educación, que aplica desde 1994 un proceso de integración educativa. Este nuevo sistema está
contemplado en la Ley General de Educación y es de obligatorio cumplimiento, aseguró la Alcaldía al responder la tutela. La Corte consideró, sin embargo, que en Barrancabermeja no se cumplió el proceso de tal manera que no se afectaran los derechos del niño.

"Resulta claro que el proceso de incorporación de niño a la nueva institución educativa no se efectuó de manera pacífica, debido a que no existió un adecuado acompañamiento por parte de la Secretaría de Educación", señalaron los magistrados. Y agregaron que "se evidencia una falta de capacitación a nivel de toda la comunidad educativa, aspecto que generó el conflicto inicial que terminó con el retiro momentáneo del menor del sistema educativo", agregó la Corte. En la sentencia, el alto tribunal ordena al instituto de Bienestar Familiar y a la Defensoría del Pueblo vigilar la orden dada a la Alcaldía de capacitar a todas las personas relacionadas con el sistema educativo, incluyendo a los padres de familia y los alumnos.~\footnote{El Tiempo, 5 de septiembre de 2009}
\begin{enumerate}
\item[2.] Completa la ficha en tu cuaderno escribiendo las acciones descritas en la noticia y la función constitucional que corresponde a cada entidad:
\begin{center}
 \begin{tabular}{|c|c|c|}
\hline 
\textbf{Entidad} & \textbf{Acción} & \textbf{Función} \\ 
\hline 
Corte Constitucional &  &  \\ 
\hline 
Alcaldía &  &  \\ 
\hline 
Ministerio de Educación &  &  \\ 
\hline 
Defensoría del pueblo &  &  \\ 
\hline 
\end{tabular}
 \end{center} 
\item[3.] Elabora una lista de elementos que debió tener en cuenta la Corte para tomar su decisión teniendo en cuenta a:
\begin{itemize}
\item El niño con el problema mental
\item La madre de familia
\item La maestra
\item La comunidad educativa
\item La Alcaldía y la Secretaría de Educación
\end{itemize}
\item[4.] Discute con tu compañero de al lado:
\begin{itemize}
\item ¿Qué atentados contra la diferencia se evidenciaron en este caso? ¿Por qué?
\item ¿Qué estrategias habría podido implementar el grupo de compañeros para favorecer el aprendizaje del protagonista de la historia?
\item ¿Cuáles eran los derechos que se estaban vulnerando en este caso?
\item ¿Qué otros organismos estatales habrían podido intervenir de acuerdo con sus funciones constitucionales?
\end{itemize}
\item[5.] ¿Qué situaciones similares se han presentado en tu región? ¿Cuál ha sido la reacción de los organismos y entidades estatales?
\end{enumerate}
\section*{Evaluaci\'{o}n}
\subsection*{¿Qu\'{e} aprend\'{i}?}
\begin{enumerate}
\item Copia en tu cuaderno las letras que corresponden a cada
oración. Escribe frente a cada una si se refiere a un juicio o a una norma:
\begin{enumerate}
\item Quién bota basura al río carece por completo de conciencia ambiental.
\item Ningún menor de edad puede estar fuera de casa después
de la medianoche.
\item Las aguas del río no son propiedad privada y pueden ser
utilizadas por cualquier persona.
\item Los habitantes de los pueblos vecinos que vienen el día de mercado, no son de fiar.
\item En la plaza sólo se pueden instalar puestos para la venta los días viernes.
\end{enumerate}
\item Completa el párrafo en tu cuaderno.
La Constitución es la \underline{\hspace*{1cm}} fundamental del \underline{\hspace*{1.5cm}}. En ella se establecen las bases de la \underline{\hspace*{1.5cm}}, pues así nos permite conocer nuestros \underline{\hspace*{1.5cm}} y nos garantiza que estos se \underline{\hspace*{1.5cm}}. En ella se definen quienes \underline{\hspace*{1.5cm}} las leyes, quienes se encargan de su \underline{\hspace*{2cm}} y quienes de \underline{\hspace*{2cm}}. Esto significa que estipula cuál es la Estructura del Estado, conformada por tres \underline{\hspace*{1.5cm}} independientes pero que deben trabajar en armonía: \underline{\hspace*{2cm}}, \underline{\hspace*{2cm}} y judicial. Las entidades y los funcionarios que pertenecen a ellas son vigilados por los \underline{\hspace*{2cm}} de \underline{\hspace*{2cm}}.
\item Imagina un caso para explicar cada uno de los atentados
contra la diferencia:
\begin{multicols}{2}
\begin{itemize}
\item Estereotipos
\item Prejuicios
\item Exclusión
\item Segregación
\end{itemize}
\end{multicols}
\subsection*{¿C\'{o}mo me ven los dem\'{a}s?}
\item Diseña en una hoja un formato como el siguiente. Pídele a un compañero, un familiar adulto y un profesor que registren su opinión sobre tu desempeño en cada aspecto:
\begin{center}
\begin{tabular}{|c|c|c|c|}
\hline 
\textbf{Aspecto} & \textbf{Familiar} & \textbf{Profesor} & \textbf{Compañero} \\ 
\hline 
Disciplina &  &  &  \\ 
\hline 
Disposición &  &  &  \\ 
\hline 
Laboriosidad &  &  &  \\ 
\hline 
Cumplimiento de normas &  &  &  \\ 
\hline 
Respeto por la diferencia &  &  &  \\ 
\hline 
\end{tabular} 
\end{center}
\subsection*{Reflexiona:}
\begin{itemize}
\item ¿Cuáles son tus fortalezas?
\item ¿En qué aspectos puedes mejorar? ¿Cómo lo puedes lograr?
\item Me autoexamino
\end{itemize}
\item Teniendo en cuenta las anteriores respuestas, escribe los objetivos específicos que puedes alcanzar en cada aspecto y diseña un plan de acción para cada uno:
\begin{itemize}
\item Cumplimiento de normas en casa
\item Cumplimiento del manual de convivencia de mi institución
\item Relación con los compañeros que menos me agradan
\item Rendimiento académico
\end{itemize}
\end{enumerate}
\section*{Reto}
Pongo a prueba mi persistencia para resolver el siguiente sudoku
\setlength\sudokusize{5cm}
\begin{sudoku}
| | | |4|5|3| | | |.
| | | |6|8|9| | | |.
| | | |1|7|2| | | |.
|5|7|9| | | |6|8|4|.
|2|8|6| | | |3|1|5|.
|1|4|3| | | |2|7|9|.
| | | |3|2|5| | | |.
| | | |8|9|4| | | |.
| | | |7|1|6| | | |.
\end{sudoku}
%\begin{sudoku}
%|9|6|1|4|5|3|7|2|8|.
%|7|2|4|6|8|9|5|3|1|.
%|8|3|5|1|7|2|4|9|6|.
%|5|7|9|2|3|1|6|8|4|.
%|2|8|6|9|4|7|3|1|5|.
%|1|4|3|5|6|8|2|7|9|.
%|6|1|8|3|2|5|9|4|7|.
%|3|5|7|8|9|4|1|6|2|.
%|4|9|2|7|1|6|8|5|3|.
%\end{sudoku}

\end{document}
