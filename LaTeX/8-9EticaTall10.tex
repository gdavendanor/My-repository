\documentclass[10pt,twoside]{article}
\usepackage[utf8]{inputenc}
\usepackage{amsmath}
\usepackage{amsfonts}
\usepackage{amssymb}
\usepackage[spanish,es-noshorthands]{babel}
\usepackage[T1]{fontenc}
\usepackage{lmodern}
\usepackage{graphicx,hyperref}
\usepackage{tikz,pgf}
\usepackage{multicol}
\usepackage{subfig}
\usepackage[papersize={6.5in,8.5in},width=5.5in,height=7in]{geometry}
\usepackage{fancyhdr}
\pagestyle{fancy}
\fancyhead[LE]{\url{www.autistici.org/mathgerman}}
\fancyhead[RE]{}
\fancyhead[RO]{matematicas.german@gmail.com}
\fancyhead[LO]{}

\author{Germ\'an Avenda\~no Ram\'irez~\thanks{Lic. Mat. U.D., M.Sc. U.N.}}
\title{\begin{minipage}{.2\textwidth}
\includegraphics[height=1.75cm]{Images/logo-colegio.png}\end{minipage}
\begin{minipage}{.55\textwidth}
\begin{center}
Taller 10\\
Ética $9^{\circ}$
\end{center}
\end{minipage}\hfill
\begin{minipage}{.2\textwidth}
\includegraphics[height=1.75cm]{Images/logo-sed.png} 
\end{minipage}}
\date{}
\begin{document}
\maketitle
Nombre: \hrulefill Curso: \underline{\hspace*{44pt}} Fecha: \underline{\hspace*{2.5cm}}
\section*{Aplico lo aprendido}
\begin{enumerate}
\item Lee la siguiente noticia con detenimiento. Luego, resuelve las
actividades.
\end{enumerate}
Capacitar a profesores y alumnos de colegio para tratar a niño
especial ordenó Corte Constitucional

El alto tribunal falló una acción de tutela a favor del menor de
11 años, que sufre de un leve problema mental y fue retirado de
un colegio de Barrancabermeja (Santander).

La Corte dio un plazo de 48 horas a la Alcaldía de esa ciudad
para que diseñe un programa para capacitar en atención de niños
especiales a la totalidad de alumnos y profesores de primaria y
secundaria. El menor, según la valoración hecha por los médicos,
presenta un síndrome de hiperactividad y de atención. Su madre lo
matriculó en la concentración escolar Luis Carlos Galán, que ofrecía
un programa especial para niños con discapacidad o problemas de
aprendizaje. El programa atendía un grupo de 12 infantes bajo la
orientación de una maestra especializada, que durante tres años
los dirigió con buenos resultados y logró promoverlos de kínder
al grado primero de primaria. Pero el proyecto se acabó y el niño
fue asignado entonces a un curso normal. Días después, la madre
del niño fue llamada por la directora de grupo, quien le comunicó
que no podía tener más a su hijo. Según la maestra, golpeaba a
sus compañeros de clase y les quitaba sus cosas. La Secretaría de
Educación reubicó al menor en otro colegio cercano a su residencia,
pero allí se presentó la misma situación, lo que llevó a la familia
a dejarlo en su casa. La madre puso una tutela contra la Alcaldía
exigiendo que reactivaran el programa para niños especiales
que funcionaba en el Luis Carlos Galán. Según ella, su situación
económica le impedía inscribir a su hijo en una institución privada
y la decisión de la administración local violaba el derecho del
menor a la educación, a la vida digna y a la igualdad. El caso fue
estudiado por el Juzgado Primero Civil de Barrancabermeja, que
rechazó la solicitud. Consideró que la Alcaldía actuó de acuerdo
con lo establecido por el Ministerio de Educación, que aplica desde
1994 un proceso de integración educativa. Este nuevo sistema está
contemplado en la Ley General de Educación y es de obligatorio
cumplimiento, aseguró la Alcaldía al responder la tutela. La Corte
consideró, sin embargo, que en Barrancabermeja no se cumplió el
proceso de tal manera que no se afectaran los derechos del niño.

"Resulta claro que el proceso de incorporación de niño a la nueva institución
educativa no se efectuó de manera pacífica, debido a que no existió un adecuado acompañamiento por parte de la Secretaría de Educación", señalaron los magistrados. Y agregaron que "se evidencia una falta de capacitación a nivel de toda la comunidad educativa, aspecto que generó el conflicto inicial que terminó con el retiro momentáneo del menor del sistema educativo", agregó la Corte. En la sentencia, el alto tribunal ordena al instituto de Bienestar Familiar y a la Defensoría del Pueblo vigilar la orden dada a la Alcaldía de capacitar a todas las personas relacionadas con el sistema educativo, incluyendo a los padres de familia y los alumnos.

\end{document}
