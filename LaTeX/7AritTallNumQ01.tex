\documentclass[10pt,twoside]{article}
\usepackage[utf8]{inputenc}
\usepackage{amsmath}
\usepackage{amsfonts}
\usepackage{amssymb}
\usepackage[spanish,es-noshorthands]{babel}
\usepackage[T1]{fontenc}
\usepackage{lmodern}
\usepackage{graphicx,hyperref}
\usepackage{tikz,pgf}
\usepackage{marvosym}
\usepackage{multicol}
\usepackage{subfig}
\usepackage[papersize={6.5in,8.5in},width=5.55in,height=7.1in]{geometry}
\usepackage{fancyhdr}
\pagestyle{fancy}
\fancyhead[LE]{Colegio Arborizadora Baja}
\fancyhead[RE]{PEI:``Hacia una cultura para el desarrollo sostenible''}
\fancyfoot[RO]{\Email iedabgerman@autistici.org}
\fancyhead[LO]{\url{www.autistici.org/mathgerman}}
\fancyfoot[RE]{\Email cedarborizadoraba19@redp.edu.co}
\fancyfoot[LE]{Calle 59I \#44A - 02 \Telefon 7313994 - 7313995}
\fancyhead[RO]{Nit 830024976-8, Código DANE 11100103084-8}

\author{Germ\'an Avenda\~no Ram\'irez~\thanks{Lic. Mat. U.D., M.Sc. U.N.}}
\title{\begin{minipage}{.2\textwidth}
\includegraphics[height=1.75cm]{Images/logo-colegio.png}\end{minipage}
\begin{minipage}{.55\textwidth}
\begin{center}
Taller Números racionales $\mathbb{Q}$\\$7^{\circ}$
\end{center}
\end{minipage}\hfill
\begin{minipage}{.2\textwidth}
\includegraphics[height=1.75cm]{Images/logo-sed.png} 
\end{minipage}}
\date{}
\begin{document}
\maketitle
\section*{Números fraccionarios}
\subsection*{Nivel 1}
\begin{enumerate}
\item Enumera los términos de una fracción y di qué indica cada
uno de ellos. Pon varios ejemplos.
\item ¿Qué fracción de hora son 20 minutos? Y ¿35 minutos? Y
¿55 minutos?
\item Para elaborar un tarro de frutas se han necesitado 400 gramos de plátanos, 350 gramos de fresas, 250 gramos de azúcar y 50 gramos de manzanas. ¿Qué fracción del total representa cada uno de estos productos?
\item Calcula:
\begin{enumerate}
\begin{multicols}{4}
\item $\frac{5}{10}$ de 90
\item $-\frac{7}{9}$ de 72
\item $\frac{3}{4}$ de 42
\item $\frac{5}{9}$ de 540
\end{multicols}
\end{enumerate}
\item En una clase de 24 alumnos 5/8 son chicos. ¿Cuántos chicos
y chicas hay en clase?
\item El depósito de un coche tiene una capacidad de 63 litros de
gasolina, si gasta los 5/9 en una excursión, ¿cuántos litros le
quedan al volver de viaje?
\item En la puerta de un cine hay 12 mujeres por cada 8 hombres
y 16 niños. ¿Cuál es la relación entre hombres y mujeres?
¿Entre hombres y niños? Y ¿Entre mujeres y niños?
\item Indica de las siguientes fracciones cuáles dan como resultado un número natural y cuáles un número decimal:
\begin{enumerate}
\begin{multicols}{5}
\item $\frac{3}{2}$
\item $\frac{12}{3}$
\item $\frac{17}{4}$
\item $\frac{27}{9}$
\item $\frac{14}{6}$
\item $\frac{19}{8}$
\item $\frac{21}{7}$
\item $\frac{6}{9}$
\item $\frac{12}{12}$
\item $\frac{3}{4}$
\end{multicols}
\end{enumerate}
\item Indica en las fracciones siguientes cuáles son mayores, iguales o menores que la unidad:
\begin{enumerate}
\begin{multicols}{5}
\item $\frac{1}{7}$
\item $\frac{3}{5}$
\item $\frac{9}{9}$
\item $\frac{5}{3}$
\item $\frac{17}{2}$
\item $\frac{2}{9}$
\item $\frac{16}{4}$
\item $\frac{18}{17}$
\item $\frac{5}{5}$
\item $\frac{12}{12}$
\end{multicols}
\end{enumerate}
\item Las fracciones siguientes son menores que la unidad. ¿Qué
fracción falta en cada una de ellas para completar la unidad?
\begin{enumerate}
\begin{multicols}{6}
\item $\frac{3}{7}$
\item $\frac{3}{8}$
\item $\frac{9}{2}$
\item $\frac{11}{16}$
\item $\frac{7}{13}$
\item $\frac{5}{9}$
\end{multicols}
\end{enumerate}
\item ¿Qué fracción sobra en cada una de las siguientes para obtener la unidad?
\begin{enumerate}
\begin{multicols}{6}
\item $\frac{6}{5}$
\item $\frac{5}{4}$
\item $\frac{16}{9}$
\item $\frac{8}{6}$
\item $\frac{17}{13}$
\item $\frac{25}{19}$
\end{multicols}
\end{enumerate}
\item ¿Entre que números naturales consecutivos están comprendidas las fracciones siguientes?
\begin{enumerate}
\begin{multicols}{5}
\item $\frac{7}{5}$
\item $\frac{12}{5}$
\item $\frac{12}{3}$
\item $\frac{18}{17}$
\item $\frac{21}{5}$
\end{multicols}
\end{enumerate}
\item Representa en la recta numérica las siguientes fracciones:
\begin{enumerate}
\begin{multicols}{7}
\item $\frac{3}{9}$
\item $\frac{4}{5}$
\item $\frac{6}{8}$
\item $\frac{4}{4}$
\item $\frac{5}{3}$
\item $\frac{6}{4}$
\item $\frac{8}{2}$
\end{multicols}
\end{enumerate}
\item Escribe dos fracciones equivalentes a cada una de estas:
\begin{enumerate}
\begin{multicols}{5}
\item $\frac{13}{5}$
\item $\frac{7}{14}$
\item $\frac{5}{2}$
\item $\frac{45}{18}$
\item $\frac{3}{21}$
\end{multicols}
\end{enumerate}
Explique cómo la ha hecho
\item ¿Son equivalentes las parejas de fracciones siguientes?:
\begin{enumerate}
\begin{multicols}{3}
\item $\frac{15}{4}$ y $\frac{75}{35}$
\item $\frac{33}{42}$ y $\frac{132}{168}$
\item $\frac{17}{62}$ y $\frac{51}{185}$
\end{multicols}
\end{enumerate}
\item Halla la fracción irreducible de cada una de las fracciones siguientes:
\begin{enumerate}
\begin{multicols}{5}
\item $\frac{150}{105}$
\item $\frac{72}{450}$
\item $\frac{264}{200}$
\item $\frac{716}{99}$
\item $\frac{225}{75}$
\end{multicols}
\end{enumerate}
\item En un campeonato de atletismo uno de los saltadores de altura consigue saltar más de dos metros 13 veces de 52 intentos, su contrincante salta más de 2 metros 11 veces de 44 intentos. ¿Cuál de los dos ha ganado?
\end{enumerate}
\end{document}
