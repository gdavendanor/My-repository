\documentclass[10pt,twoside]{article}
\usepackage[utf8]{inputenc}
\usepackage{amsmath}
\usepackage{amsfonts}
\usepackage{amssymb}
\usepackage[spanish,es-noshorthands]{babel}
\usepackage[T1]{fontenc}
\usepackage{lmodern}
\usepackage{graphicx,hyperref}
\usepackage{tikz,pgf}
\usepackage{multicol}
\usepackage{subfig}
\usepackage[papersize={6.5in,8.5in},width=5.5in,height=7in]{geometry}
\usepackage{fancyhdr}
\pagestyle{fancy}
\fancyhead[LE]{\url{www.autistici.org/mathgerman}}
\fancyhead[RE]{}
\fancyhead[RO]{matematicas.german@gmail.com}
\fancyhead[LO]{}

\author{Germ\'an Avenda\~no Ram\'irez~\thanks{Lic. Mat. U.D., M.Sc. U.N.}}
\title{\begin{minipage}{.2\textwidth}
\includegraphics[height=1.75cm]{Images/logo-colegio.png}\end{minipage}
\begin{minipage}{.55\textwidth}
\begin{center}
Taller 09\\
El derecho a la diferencia\\
Ética $9^{\circ}$
\end{center}
\end{minipage}\hfill
\begin{minipage}{.2\textwidth}
\includegraphics[height=1.75cm]{Images/logo-sed.png} 
\end{minipage}}
\date{}
\begin{document}
\maketitle
Nombre: \hrulefill Curso: \underline{\hspace*{44pt}} Fecha: \underline{\hspace*{2.5cm}}
\section*{Lo que s\'{e}}
Leer lo siguiente
\subsection*{Creencias religiosas desplazan a Arhuacos}
Cuarenta indígenas arhuacos denunciaron torturas, persecución y discriminación por parte de las máximas autoridades de sus resguardos, por el sólo hecho de vincularse a la Iglesia Pentecostal y abandonar sus reverencias de carácter religiosa a los Mamos y al dios Kuka Zerankua, al que adora esta etnia. Por esta razón abandonaron Sabanas de Jordán,
la región donde viven en la zona nororiental de la Sierra Nevada de Santa Marta y llegaron ayer a Valledupar. Después de caminar 8 horas llegaron cansados, pero decididos a que se les respete su pensamiento religioso, dijo el pastor Jairo Salcedo, quien los albergó en un templo Pentecostal de la capital del Cesar\ldots Salcedo aseguró que estos nuevos desplazados, pero ahora por la religión, temen volver debido a
las represalias de que pueden ser objeto. Queremos un pronunciamiento del alcalde de Valledupar\ldots en defensa de los derechos de los indígenas, no solamente en su cultura sino de la fe en Jesucristo\ldots

El pastor Salcedo explicó que el evangelio llegó a esas regiones hace unos 40 años con la aborigen María Eugenia Solis, quien vivió en la civilización y llevó el mensaje. Los indígenas creyeron y desde ese entonces están con esta religión. Los indígenas que se vincularon a la iglesia Pentecostal culturalmente conservan sus costumbres, tales como su vestimenta de mantas, sombreros, abarcas, mochilas, collares, etc. La única diferencia es que no llevan el poporo y en su reemplazo cargan la Biblia.\footnote{Periódico El Tiempo, 23 de Mayo de 1997 (Fragmento)}

\end{document}
