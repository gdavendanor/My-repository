\documentclass[10pt,twoside]{article}
\usepackage[utf8]{inputenc}
\usepackage{amsmath}
\usepackage{amsfonts}
\usepackage{amssymb}
\usepackage[spanish,es-noshorthands]{babel}
\usepackage[T1]{fontenc}
\usepackage{lmodern}
\usepackage{graphicx,hyperref}
\usepackage{tikz,pgf}
\usepackage{multicol}
\usepackage{subfig}
\usepackage[papersize={6.5in,8.5in},width=5.5in,height=7in]{geometry}
\usepackage{fancyhdr}
\pagestyle{fancy}
\fancyhead[LE]{\url{www.autistici.org/mathgerman}}
\fancyhead[RE]{}
\fancyhead[RO]{matematicas.german@gmail.com}
\fancyhead[LO]{}

\author{Germ\'an Avenda\~no Ram\'irez~\thanks{Lic. Mat. U.D., M.Sc. U.N.}}
\title{\begin{minipage}{.2\textwidth}
\includegraphics[height=1.75cm]{Images/logo-colegio.png}\end{minipage}
\begin{minipage}{.55\textwidth}
\begin{center}
Taller 09\\
El derecho a la diferencia\\
Ética $9^{\circ}$
\end{center}
\end{minipage}\hfill
\begin{minipage}{.2\textwidth}
\includegraphics[height=1.75cm]{Images/logo-sed.png} 
\end{minipage}}
\date{}
\begin{document}
\maketitle
Nombre: \hrulefill Curso: \underline{\hspace*{44pt}} Fecha: \underline{\hspace*{2.5cm}}
\paragraph*{Lo que s\'{e}:} Lea lo siguiente
\section*{Creencias religiosas desplazan a Arhuacos}
Cuarenta indígenas arhuacos denunciaron torturas, persecución y discriminación por parte de las máximas autoridades de sus resguardos, por el sólo hecho de vincularse a la Iglesia Pentecostal y abandonar sus reverencias de carácter religiosa a los Mamos y al dios Kuka Zerankua, al que adora esta etnia. Por esta razón abandonaron Sabanas de Jordán,
la región donde viven en la zona nororiental de la Sierra Nevada de Santa Marta y llegaron ayer a Valledupar. Después de caminar 8 horas llegaron cansados, pero decididos a que se les respete su pensamiento religioso, dijo el pastor Jairo Salcedo, quien los albergó en un templo Pentecostal de la capital del Cesar\ldots Salcedo aseguró que estos nuevos desplazados, pero ahora por la religión, temen volver debido a
las represalias de que pueden ser objeto. Queremos un pronunciamiento del alcalde de Valledupar\ldots en defensa de los derechos de los indígenas, no solamente en su cultura sino de la fe en Jesucristo\ldots

El pastor Salcedo explicó que el evangelio llegó a esas regiones hace unos 40 años con la aborigen María Eugenia Solis, quien vivió en la civilización y llevó el mensaje. Los indígenas creyeron y desde ese entonces están con esta religión. Los indígenas que se vincularon a la iglesia Pentecostal culturalmente conservan sus costumbres, tales como su vestimenta de mantas, sombreros, abarcas, mochilas, collares, etc. La única diferencia es que no llevan el poporo y en su reemplazo cargan la Biblia.\footnote{Periódico El Tiempo, 23 de Mayo de 1997 (Fragmento)}
\begin{enumerate}
\item ¿Cuál era el problema central que aquejaba a los arhuacos
que llegaron a Valledupar?
\item ¿Qué derechos veían vulnerados los indígenas pentecostales?
¿Por qué?
\item ¿Cómo actúas frente a las personas que profesan una religión
diferente a la tuya? ¿Cómo reacciona la población de tu municipio?
\item ¿Por qué crees que algunas personas eligen apartarse de las
tradiciones y creencias de la mayoría de miembros de su
comunidad?
\end{enumerate}
\paragraph*{Aprendo algo nuevo}
\section*{Ser diferente}
Así como sus huellas dactilares, todos los seres humanos son únicos e irrepetibles. Como seres racionales, los individuos están dotados de la capacidad de crear y de expresarse a través de manifestaciones culturales (costumbres, comportamientos, creencias, valores, lenguajes, símbolos, festividades, vestimenta \ldots). La combinación de características físicas y culturales, hace que sea imposible encontrar dos seres humanos iguales. Las características que identifican a cada ser humano son resultado de su herencia genética, su lugar de origen, el
tipo de familia a la que pertenece, los compañeros y amigos que ha tenido, la educación que ha recibido, los problemas que ha enfrentado, además de muchas otras condiciones. Incluso los miembros de una misma familia, que han sido criados en condiciones semejantes, difieren en sus virtudes, defectos, gustos, creencias, formas de reaccionar, entre
otras características. Con mayor razón, existirán diferencias entre los miembros de una comunidad o los ciudadanos que integran una nación. Sin embargo, en vez de ser una dificultad, la diferencia es la que permite a los individuos compartir y crecer como personas pues es un motivo para
dialogar, discutir, convencer, hacer amigos y compartir.

Pese a esta realidad, vemos con frecuencia que el \textbf{derecho a ser diferentes} se ve amenazado por acciones que pretenden homogeneizar a los miembros de una comunidad bajo un criterio dominante. En el ejemplo de los indígenas arhuacos que adoptaron una religión diferente a la de sus ancestros y abandonaron las prácticas que iban en contradicción con sus creencias, fueron víctimas de atropellos. A diario, muchas personas deben enfrentarse a situaciones similares debido a sus diferencias en
aspectos relacionados con la ideología, la religión, la posición
política, la opción sexual, entre otros. En estos casos, se está
violentando la \textbf{dignidad}, aquel principio máximo que indica que todos los seres humanos nacen con los mismos derechos y que estos deben ser respetados y defendidos, sin importar nuestro aspecto físico, edad, sexo, posición económica, opinión política, religión o cualquier otra condición.
\paragraph*{Ejercito lo aprendido:} 
Responda las preguntas en el cuaderno y luego compara con el compañero de al lado.
\begin{enumerate}
\item ¿Cómo podemos actuar con respeto frente a la dignidad de
las personas diferentes a nosotros?
\item ¿Qué ejemplos de intolerancia frente a las diferencias vemos a diario en los medios de comunicación? Cita tres casos que recuerdes.
\end{enumerate}
\paragraph*{Aprendo algo nuevo}
\section*{Autenticidad y originalidad}
Las diferencias manifiestas en las formas de pensar, sentir
y actuar de los individuos hacen que los seres humanos
sean únicos, dotados de \textbf{originalidad}. Esto significa tener la capacidad de crear un \textbf{estilo propio}, de innovar, de medirse de acuerdo con ciertas capacidades y objetivos. En ese sentido, la autenticidad es la expresión clara y real del ser, que se ve favorecida si en la sociedad existe un clima de tolerancia que le permita actuar al individuo. Es contraria a la falsedad, actitud que lleva a las personas a mentir, fingir, actuar con hipocresía o reducir sus actos a la imitación de los demás.

Las personas auténticas son \textbf{honestas}, no fingen de acuerdo a la ocasión o la conveniencia, son \textbf{responsables} porque cumplen con sus obligaciones y asumen las consecuencias de sus actos.

Sin embargo, ser auténtico no significa ser impulsivo o
actuar de acuerdo a estímulos momentáneos como la ira o
decir lo que cada quien piensa con el ánimo de "ser sincero" sin importar las implicaciones que sus afirmaciones tengan sobre los demás. Significa firmeza en las convicciones y respeto por las diferencias que se tienen frente a los semejantes.

La autenticidad se fundamenta en el libre desarrollo de la
personalidad, sin ir contra las leyes o los derechos de los
demás, y en la definición clara de intereses y metas. Implica
que el individuo sea \textbf{coherente}, es decir, que sus actos, pensamientos y sentimientos se encuentran en armonía y
que le permitan ser fiel a sus principios. Este atributo, permite además sentir satisfacción, plenitud y gozo cuando se toman decisiones y se actúa en concordancia con éstas, en vez de dejarse llevar constantemente por las opiniones de otros.
La persona auténtica se acepta tal cual es, reflexiona sobre
sus falencias y busca superarlas. Aunque admire a otros por
sus cualidades o éxito, no busca copiar su personalidad sino
relacionarse con ellos o seguir su ejemplo para crecer como
ser humano.

\paragraph*{Ejercito lo aprendido:} Identifico y evidencio
\begin{enumerate}
\item Identifique las características que hacen auténticos a
algunos miembros del curso. Para ello, cada uno escogerá a tres compañeros y escribirá las virtudes y defectos que lo hacen únicos.
\item Responda en su cuaderno
\begin{enumerate}
\item ¿Por qué puedes considerarte una persona original y
auténtica?
\item ¿Cuál de tus compañeros podría servirte de ejemplo a
seguir? ¿Por qué?
\item ¿Qué significa para ti tener una personalidad
original?
\end{enumerate}
\end{enumerate}
\paragraph*{Aprendo algo nuevo}
\section*{Atentados contra la diferencia}
Aunque todos los seres humanos somos únicos, las dife-
rencias no siempre son aceptadas y respetadas, situación
que genera conflictos producto de la intolerancia, tal como
ocurría en la historia inicial. Los atentados más frecuentes
contra la diferencia son:
\begin{itemize}
\item \textbf{Estereotipos}: son imágenes \textbf{simplificadas} y con frecuencia \textbf{negativas} de un grupo de personas. Los más frecuentes se asocian con el origen étnico o regional, la posición económica y la religión. Por ejemplo, en Colombia es frecuente escuchar comentarios, refranes y chistes relacionados con la pereza de los costeños
y los opitas, el mal humor de los santandereanos o el
bajo nivel intelectual de los pastusos. Estos estereotipos
resultan ofensivos, ignoran las particularidades de
los miembros de cada grupo cultural, encasillan a las
personas según una generalización y atentan contra una
pluralidad enriquecedora.
\item \textbf{Prejuicios}: son los conceptos que se forman sobre una
persona o grupo \textbf{sin tener las evidencias} y los argumentos
necesarios para ello pues no existe un conocimiento o
experiencia previa. Cuando se piensa que una persona de
bajos recursos será un ladrón potencial o que una mujer no
estará capacitada para determinado empleo, se elaboran
prejuicios que van en contra de la igualdad y la dignidad de
los seres humanos.
\item \textbf{Exclusión}: los estereotipos y prejuicios conducen al \textbf{rechazo} de una persona o grupo, impidiéndoles una relación
plena con los demás individuos de la sociedad y el disfrute
de derechos colectivos. Si una persona no puede participar
de la vida económica y cultural de una sociedad, se verá
afectada por problemas como la pobreza, el desempleo o la
falta de formación académica.
\item \textbf{Segregación}: es el \emph{aislamiento} o la separación a la que se ve sometida una persona o un grupo hasta verse privada del acceso a recursos o servicios básicos o marginada espacialmente debido a su posición económica, religión, etnia, género, ideología o cualquier otra condición. Un ejemplo de ello ocurre cuando a los niños no católicos se
les impide ingresar a ciertas instituciones educativas sin su
partida de bautizo. Por respeto a la diferencia, la Constitución de 1991 consagró la libertad de cultos.
\end{itemize}
\paragraph*{Ejercito lo aprendido}: En equipos de tres integrantes elaboren una lista de atentados contra la diferencia que sean comunes en su región. Escojan tres
de ellos y elaboren un cuadro, mapa conceptual o esquema para explicar las consecuencias de estas acciones. Cada grupo expondrá su trabajo para los compañeros de otro grado.\\

\textit{La ética no consiste en formular preceptos caídos o dictados desde el cielo, sino que es consecuencia de tomar consciencia de lo que somos.}\footnote{Albert Jacquard}
\end{document}
