\documentclass[10pt,twoside]{article}
\usepackage[utf8]{inputenc}
\usepackage{amsmath}
\usepackage{amsfonts}
\usepackage{amssymb}
\usepackage[spanish,es-noshorthands]{babel}
\usepackage[T1]{fontenc}
\usepackage{lmodern}
\usepackage{graphicx,hyperref}
\usepackage{tikz,pgf}
\usepackage{multicol}
\usepackage{subfig}
\usepackage[papersize={6.5in,8.5in},width=5.5in,height=7in]{geometry}
\usepackage{fancyhdr}
\pagestyle{fancy}
\fancyhead[LE]{\url{www.autistici.org/mathgerman}}
\fancyhead[RE]{}
\fancyhead[RO]{matematicas.german@gmail.com}
\fancyhead[LO]{}

\author{Germ\'an Avenda\~no Ram\'irez~\thanks{Lic. Mat. U.D., M.Sc. U.N.}}
\title{\begin{minipage}{.2\textwidth}
\includegraphics[height=1.75cm]{Images/logo-colegio.png}\end{minipage}
\begin{minipage}{.55\textwidth}
\begin{center}
Taller 09\\
El derecho a la diferencia\\
Ética $9^{\circ}$
\end{center}
\end{minipage}\hfill
\begin{minipage}{.2\textwidth}
\includegraphics[height=1.75cm]{Images/logo-sed.png} 
\end{minipage}}
\date{}
\begin{document}
\maketitle
Nombre: \hrulefill Curso: \underline{\hspace*{44pt}} Fecha: \underline{\hspace*{2.5cm}}
\section*{Lo que s\'{e}}
Leer lo siguiente
\subsection*{Creencias religiosas desplazan a Arhuacos}
Cuarenta indígenas arhuacos denunciaron torturas, persecución y discriminación por parte de las máximas autoridades de sus resguardos, por el sólo hecho de vincularse a la Iglesia Pentecostal y abandonar sus reverencias de carácter religiosa a los Mamos y al dios Kuka Zerankua, al que adora esta etnia. Por esta razón abandonaron Sabanas de Jordán,
la región donde viven en la zona nororiental de la Sierra Nevada de Santa Marta y llegaron ayer a Valledupar. Después de caminar 8 horas llegaron cansados, pero decididos a que se les respete su pensamiento religioso, dijo el pastor Jairo Salcedo, quien los albergó en un templo Pentecostal de la capital del Cesar\ldots Salcedo aseguró que estos nuevos desplazados, pero ahora por la religión, temen volver debido a
las represalias de que pueden ser objeto. Queremos un pronunciamiento del alcalde de Valledupar\ldots en defensa de los derechos de los indígenas, no solamente en su cultura sino de la fe en Jesucristo\ldots

El pastor Salcedo explicó que el evangelio llegó a esas regiones hace unos 40 años con la aborigen María Eugenia Solis, quien vivió en la civilización y llevó el mensaje. Los indígenas creyeron y desde ese entonces están con esta religión. Los indígenas que se vincularon a la iglesia Pentecostal culturalmente conservan sus costumbres, tales como su vestimenta de mantas, sombreros, abarcas, mochilas, collares, etc. La única diferencia es que no llevan el poporo y en su reemplazo cargan la Biblia.\footnote{Periódico El Tiempo, 23 de Mayo de 1997 (Fragmento)}
\begin{enumerate}
\item ¿Cuál era el problema central que aquejaba a los arhuacos
que llegaron a Valledupar?
\item ¿Qué derechos veían vulnerados los indígenas pentecostales?
¿Por qué?
\item ¿Cómo actúas frente a las personas que profesan una religión
diferente a la tuya? ¿Cómo reacciona la población de tu municipio?
\item ¿Por qué crees que algunas personas eligen apartarse de las
tradiciones y creencias de la mayoría de miembros de su
comunidad?
\end{enumerate}
\section*{Aprendo algo nuevo}
\subsection*{Ser diferente}
Así como sus huellas dactilares, todos los seres humanos son únicos e irrepetibles. Como seres racionales, los individuos están dotados de la capacidad de crear y de expresarse a través de manifestaciones culturales (costumbres, comportamientos, creencias, valores, lenguajes, símbolos, festividades, vestimenta \ldots). La combinación de características físicas y culturales, hace que sea imposible encontrar dos seres humanos iguales. Las características que identifican a cada ser humano son resultado de su herencia genética, su lugar de origen, el
tipo de familia a la que pertenece, los compañeros y amigos que ha tenido, la educación que ha recibido, los problemas que ha enfrentado, además de muchas otras condiciones. Incluso los miembros de una misma familia, que han sido criados en condiciones semejantes, difieren en sus virtudes, defectos, gustos, creencias, formas de reaccionar, entre
otras características. Con mayor razón, existirán diferencias entre los miembros de una comunidad o los ciudadanos que integran una nación. Sin embargo, en vez de ser una dificultad, la diferencia es la que permite a los individuos compartir y crecer como personas pues es un motivo para
dialogar, discutir, convencer, hacer amigos y compartir.

Pese a esta realidad, vemos con frecuencia que el \textbf{derecho a ser diferentes} se ve amenazado por acciones que pretenden homogeneizar a los miembros de una comunidad bajo un criterio dominante. En el ejemplo de los indígenas arhuacos que adoptaron una religión diferente a la de sus ancestros y abandonaron las prácticas que iban en contradicción con sus creencias, fueron víctimas de atropellos. A diario, muchas personas deben enfrentarse a situaciones similares debido a sus diferencias en
aspectos relacionados con la ideología, la religión, la posición
política, la opción sexual, entre otros. En estos casos, se está
violentando la \textbf{dignidad}, aquel principio máximo que indica que todos los seres humanos nacen con los mismos derechos y que estos deben ser respetados y defendidos, sin importar nuestro aspecto físico, edad, sexo, posición económica, opinión política, religión o cualquier otra condición.
\section*{Ejercito lo aprendido}
Responda las preguntas en el cuaderno y luego compara con el compañero de al lado.
\begin{enumerate}
\item ¿Cómo podemos actuar con respeto frente a la dignidad de
las personas diferentes a nosotros?
\item ¿Qué ejemplos de intolerancia frente a las diferencias vemos a diario en los medios de comunicación? Cita tres casos que recuerdes.

\end{enumerate}
\end{document}
