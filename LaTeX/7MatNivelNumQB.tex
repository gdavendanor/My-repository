\documentclass[fleqn]{article}
\usepackage[spanish,es-noshorthands]{babel}
\usepackage[utf8]{inputenc} 
\usepackage[papersize={6.5in,8.5in},left=1cm, right=1cm, top=1.5cm, bottom=1.7cm]{geometry}
\usepackage{mathexam}
\usepackage{amsmath}
\usepackage{graphicx}
\usepackage{multicol}
\usepackage{tikz,pgf}
\usepackage{printsudoku}
\newcount\segmentsleft
\tikzset{pics/.cd,
  circle fraction/.style args={#1/#2}{code={%
\segmentsleft=#1\relax
\pgfmathloop
\ifnum\segmentsleft<1\else
\ifnum\segmentsleft<#2 \edef\n{\the\segmentsleft}\else\def\n{#2}\fi
\begin{scope}[shift={(\pgfmathcounter,0)}]
\foreach \i [evaluate={\a=360/#2*(\i-1)+90;}] in {1,...,\n}
  \fill[fill=gray] (0,0) -- (\a:3/8) arc (\a:\a+360/#2:3/8) -- cycle;
\draw circle [radius=3/8];
\ifnum#2>1
  \foreach \i [evaluate={\a=360/#2*(\i-1);}] in {1,...,#2}
    \draw (0,0) -- (90+\a:3/8);
\fi
\end{scope}
\advance\segmentsleft by-#2
\repeatpgfmathloop
  }}
}

\ExamClass{\includegraphics[height=16pt]{Images/logo-sed.png} Matemáticas $7^{\circ}$}
\ExamName{`Nivelación números fraccionarios'}
\ExamHead{\includegraphics[height=16pt]{Images/logo-colegio.png} IEDAB}
\newcommand{\LineaNombre}{%
\par
\vspace{\baselineskip}
Nombre:\hrulefill \; Curso: \underline{\hspace*{48pt}} \; Fecha: \underline{\hspace*{2.5cm}} \relax
\par}
\let\ds\displaystyle

\begin{document}
\ExamInstrBox{
Respuesta sin justificar mediante procedimiento no será tenida en cuenta en la calificación. Puede anexar una hoja con las operaciones requeridas. Escriba sus respuestas en el espacio indicado. Tiene 45 minutos para contestar esta prueba.}
\LineaNombre
\begin{enumerate}
 \item Ubique las siguientes fracciones sobre la recta numérica
 \begin{enumerate}
 \begin{multicols}{2}
 \item $\dfrac{3}{5}$
 \item $\dfrac{7}{5}$ 
 \end{multicols}
 \end{enumerate}
 \begin{tikzpicture}[scale=3]
\draw[<->] (-1,0)--(2.5,0);
\foreach \x in {0,1,2} \draw[shift={(\x,0)},color=black] (0pt,2pt) -- (0pt,-2pt) node[below] {\footnotesize $\x$};
\foreach \x in {-.8,-.6,-.4,-.2,0.2,.4,.6,.8,1.2,1.4,1.6,1.8,2.2,2.4} \draw[shift={(\x,0)},color=black] (0pt,1pt) -- (0pt,-1pt);
\end{tikzpicture}
 \item Determine la fracción correspondiente al área sombreada en cada dibujo.
 
\begin{tikzpicture}
\foreach \numerator/\denominator [count=\y] 
  in {1/1, 1/3, 2/4, 3/5, 8/8}{
  %\node at (-1/2,-\y) {$\frac{\numerator}{\denominator}$};
  \pic  at (0, -\y) {circle fraction={\numerator/\denominator}};
}
\foreach \numerator/\denominator [count=\y] 
  in {4/1, 10/3, 20/6, 30/7, 40/15}{
  %\node at (-1/2,-\y) {$\frac{\numerator}{\denominator}$};
  \pic  at (5, -\y) {circle fraction={\numerator/\denominator}};
}
\end{tikzpicture} 
\item Ordene de menor a mayor los siguientes grupos de fracciones:
\begin{enumerate}
\begin{multicols}{2}
\item $\frac{7}{8}$, $\frac{7}{6}$, $\frac{7}{3}$, $\frac{7}{2}$, $\frac{7}{10}$, $\frac{7}{12}$
\item $\frac{3}{4}$, $\frac{7}{12}$, $\frac{5}{3}$, $\frac{1}{6}$, $\frac{3}{12}$
\end{multicols}
\end{enumerate}\noanswer[.25in]
\item Realice las siguientes operaciones y simplifique si es posible:
\begin{enumerate}
\begin{multicols}{2}
\item $\frac{4}{5}+\frac{7}{10}=$
\item $\frac{7}{9}-\frac{5}{12}=$
\end{multicols}\noanswer[.1in]
\begin{multicols}{2}
\item $\frac{5}{7}+\frac{2}{3}-\frac{5}{21}=$
\item $\frac{2}{3}\cdot \frac{3}{5}\cdot\frac{3}{4}=$
\end{multicols}\noanswer[.1in]
\begin{multicols}{2}
\item $\frac{5}{12}\div \frac{3}{8}=$\noanswer
\item $7\div\frac{3}{5}\cdot\frac{2}{3}\cdot\frac{3}{4}=$\noanswer
\end{multicols}
\end{enumerate}
\item Marta estudia 3 asignaturas en una carrera de ingeniería. Dedica 1/4 del tiempo de estudio para preparar la primera asignatura y 2/3 para estudiar la segunda. ¿Qué fracción del tiempo de estudio dedica para preparar la tercera asignatura?\noanswer
\item Convierta las siguientes fracciones impropias a número mixto
\begin{enumerate}
\begin{multicols}{2}
\item $\frac{7}{5}$ 
\item $\frac{5}{3}$
\end{multicols}
\noanswer[.25in]
\end{enumerate}
\item Convierta los siguientes números mixtos a fracciones
\begin{enumerate}
\begin{multicols}{2}
\item $4\frac{2}{5}$
\item $5\frac{3}{3}$
\end{multicols}
\end{enumerate}
\noanswer[.25in]
 \end{enumerate}
\paragraph*{Sudoku:} Punto extra
\cluefont{\Large}
\cellsize{1.5\baselineskip}
\begin{minipage}{0.45\linewidth}\begin{center}
SE15 (easy) \\
\sudoku{se15.sud}
\end{center}\end{minipage}
\end{document}
