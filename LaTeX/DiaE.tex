\documentclass[10pt]{article}
\usepackage[utf8]{inputenc}
\usepackage[spanish]{babel}
\usepackage{graphicx}
\usepackage{amsmath}
\usepackage{amsfonts}
\usepackage{amssymb}
\usepackage{lmodern}
\usepackage[papersize={5.5in,8.5in},left=1.25cm,right=1.25cm,top=1.25cm,bottom=1.25cm]{geometry}
\usepackage{url}
\author{Comité sindical Arborizadora Baja j.m.}
\title{DESOBEDIENCIA CIVIL DIA E}
\date{25 de marzo de 2015}
\begin{document}
\maketitle
Cordial saludo compañeros y compañeras\\

A continuación presentamos a ustedes las actividades sugeridas por nuestra organización sindical para la jornada de desobediencia civil.
\begin{itemize}
\item[a.] No descargar el protocolo o reporte del ISCE
\item[b.] No realizar el taller, no seguir ninguno de sus pasos, no ver los videos y no colocar los afiches
\item[c.] No diligenciar los formatos solicitados de los acuerdos que se quieren imponer.
\item[d.] Hacer un taller alternativo que analice la responsabilidad de las políticas educativas y de las autoridades de gobierno en los resultados de la educación, junto con lo que se busca imponer en el plan de desarrollo
\item[e.] Concluir el taller alternativo con un escrito que recoja peticiones sobre lo que las instituciones educativas, el magisterio y el alumnado requieren para ofrecer unas condiciones distintas en el sistema educativo
\end{itemize}
\section*{Taller alternativo}
\begin{enumerate}
\item Ver el video-documental “Granito de arena- la educación en México y sus problemas”. Disponible en: \url{/https://www.youtube.com/watch?v=Mu2E-iTARJo&list=PLS09vmehZoIdDLV3vXrtzHsCrcAQNSCIW/}
\item Con base en lo visto en el video hacer una comparación entre la situación de México y la de la educación en Colombia. Sacar unas conclusiones enfatizando en nuestro país y en la problemática de la institución educativa.
\item Hacer lectura de los documentos de análisis y critica sobre el día E y el Índice Sintético de la Calidad de la Educación ISCE.
\item Recabar análisis críticos sobre las pruebas PISA y SABER y el informe COMPARTIR.
\item Colectivamente analizar la incidencia de las políticas educativas en la educación y de lo contemplado en el plan de desarrollo
\item Identificar problemas cruciales de la educación: presupuesto y financiación, políticas nacionales y municipales, hábitat y condiciones espaciales de los estudiantes en el aula y demás espacios educativos o posibles situaciones de hacinamiento, condiciones de plantas físicas, dotación de plantas físicas y espacios educativos, disponibilidad de materiales educativos y didácticos, condiciones sociales de los estudiantes, espacios para la reflexión académica y pedagógica, condiciones laborales de la profesión docente, y otras que cada institución quiera agregar.
\item Elaborar un escrito de protesta y unas peticiones para modificar las condiciones físicas y académicas de las instituciones educativas
\end{enumerate}
\section*{Otros insumos}
\subsection*{A prop\'{o}sito del d\'{i}a E, por Alejandro \'{A}lvarez Gallego, rector IPN}
Para el 25 de marzo del año 2015 el Ministerio de Educación convocó una jornada de trabajo en todos y cada uno de los colegios oficiales y privados del país, que han llamado día E, o día de la excelencia educativa. Este día queda institucionalizado para realizar todos los años, según Decreto Ministerial 0325 de 2015 (aunque en la página del MEN se dice que es un decreto presidencial, en realidad está firmado por la Ministra). Con un despliegue publicitario propio de las campañas de Estado, se ha logrado crear una expectativa importante alrededor del tema. Por ser la educación uno de los tres ejes prioritarios de las políticas del actual gobierno, el mismo presidente de la República está encabezando la campaña y prometió asistir a un colegio a dirigir la jornada.

En mi condición de pedagogo\footnote{Mi	postura	no es institucional, es personal}, no puedo dejar de hacer algunas observaciones que pueden contribuir a la discusión que esta propuesta genera, asumiendo que en ella se involucra a las comunidades educativas que son, por antonomasia,deliberantes y tienen por Constitución responsabilidades en la definición de las políticas educativas. Me permito calificar esta jornada no como el día de la excelencia, sino como el de las cuatro íes: por inconsulta, ingenua, insuficiente e infantilizante.

Calidad Educativa ISCE, con el cuál se anuncia que se podrá medir mejor este esquivo propósito. La Jornada tiene como finalidad elaborar un plan anual institucional para lograr una meta de Mejoramiento Mínimo Anual (M.M.A); dicho plan debe proponer cómo mejorar cada uno de los indicadores que componen el índice sintético, con lo cuál se avanzaría en la calidad educativa. Los cuatro indicadores o componentes del índice son: Progreso, que mide cuánto mejora año a año el resultado de las pruebas Saber de lenguaje y matemáticas en 3$^{\circ}$, 5$^{\circ}$ y 9$^{\circ}$ grados -- Desempeño, mide el promedio institucional en las mismas pruebas, comparado con el promedio nacional y la entidad territorial -- Eficiencia, mide la tasa de promoción de los estudiantes de un grado a otro -- Ambiente escolar, mide el seguimiento que los maestros hacen a las tareas de los estudiantes y el ambiente del aula, esto según encuestas aplicadas a los estudiantes en el momento de contestar las pruebas saber.

¿Quién, cómo y cuando definió que este índice mediría la calidad? ¿Quién, cómo y cuándo definió que esta estrategia de elaborar en un día un Plan de M.M.A sería la clave para mejorar la calidad?. Por lo expresado en el decreto, el Ministerio se
arroga esta función haciendo mención a la responsabilidad constitucional de ejercer la suprema inspección y vigilancia (artículo 67 de la Constitución) y a la función establecida en la Ley 115, de  "(\ldots) atender en forma permanente los factores que favorecen la calidad y el mejoramiento de la educación".

Considero que con esta medida el MEN se extralimita en sus funciones, pues cuando la Constitución y la Ley les otorga estas funciones les está diciendo expresamente que deben “(\ldots) garantizar el adecuado cubrimiento del servicio y asegurar a los menores las condiciones para su acceso y permanencia en el sistema educativo”; y la Ley 115 también les dice en el artículo citado que los factores que favorecen la calidad son, especialmente “(\ldots) la cualificación y
formación de os educadores, la promoción docente, los recursos y métodos educativos, la innovación e investigación educativa, la orientación educativa y profesional, la inspección y evaluación del proceso educativo”. Todo esto es lo que en realidad falta, todo esto es lo que debería el Estado estar procurando y es lo que tiene un rezago de décadas. No hay que ser un exégeta agudo para darse cuenta que el Estado debe inspeccionar y evaluar, pero sobre la base de que garantice las condiciones adecuadas para prestar un servicio de calidad y con
suficiente cubrimiento; y esto es lo que no hay.

El Ministerio se extralimita pues la tarea de orientar las instituciones educativas de acuerdo a un Proyecto Educativo Institucional, a un modelo pedagógico y a un currículo flexible, le corresponde al gobierno escolar constituido por un Consejo Académico, un Consejo Directivo colegiado, un estudiante Personero, un Consejo de Padres y un Consejo Estudiantil, sin contar con la libertad de cátedra y la
autonomía profesional que tienen los maestros en la toma de decisiones sobre la enseñanza. Pero además olvida que la Ley ordenó que las políticas educativas fueran discutidas en Foros anuales municipales, departamentales y nacional, con la participación de toda la comunidad educativa, y que deberían funcionar Juntas colegiadas municipales, departamentales y nacional, para asesorar a las instancias de gobierno en la toma de decisiones. Nada de esto se tuvo en cuenta para crear el ISCE y el día E.
\subsubsection*{Una medida ingenua}
La calidad es una noción polisémica y ha servido para que Tiirios y Troyanos justifiquen cualquier acción. Una administración anterior había definido que la calidad era un círculo virtuoso que funcionaba con la evaluación estandarizada en pruebas censales de competencias básicas (pruebas que dicha administración llamó Saber), luego con la identificación de las deficiencias en los aprendizajes de estudiantes detectadas a través de esas pruebas y finalmente con estrategias de capacitación de los docentes responsables de dichas deficiencias. Ya ese supuesto era ingenuo, pues reducía la calidad a aprendizajes de competencias básicas y responsabilizaba a los maestros de las deficiencias identificadas. Aunque el modelo funcionó por varios años, los resultados están a la vista: sus propios sistemas de medición no han movido la aguja con la que se cree medir un asunto tan complejo.

Con esta sorpresiva medida (el día E) se quieren enriquecer los indicadores de medición. Se advierte que las solas pruebas saber no son suficientes para determinar cómo va la calidad, sin embargo dos de los cuatro componentes del índice siguen centrados en ellas. Lo nuevo sería el indicador de promoción anual y el ambiente escolar, dividido en dos preguntas que contestaron los alumnos de tres grados; es ingenuo creer que estas dos variables tan escuetas puedan aportar algo significativo al complejo asunto de la calidad. Todo este despliegue publicitario y de recursos en torno a la jornada no se compadece con la urgencia que tiene el país de tomar en serio el problema.

La Secretaría de Educación de Bogotá ha querido agregar dos variables más, que son de suma importancia para tener en cuenta: el número de estudiantes y de sedes por institución y la procedencia socioeconómica de los estudiantes. Pero agregar estos dos items significa entonces que para mejorar el ISCE no basta un Plan de M.M.A. con el que los maestros y las directivas se comprometan, pues de ellos no dependen tales variables tan determinantes en los resultados de las pruebas, en la promoción y en el ambiente escolar.

Creer que con un Plan de M.M.A de estas características, hecho en un día, se va a lograr mejorar la calidad y alcanzar que Colombia sea la más educada en el año 2025, es a todas luces ingenuo. Es demasiada aspiración para tan precaria medida.
\subsubsection*{Una medida insuficiente}
En aras de la discusión podríamos decir que esta medida no es suficiente pero ayuda, pero el problema es otro. Los planes de mejoramiento están instituidos en las dinámicas de los colegios desde la Ley 115. El Ministerio se adelantó a decir que esta jornada no era para ajustar el plan de estudios ni para modificar el PEI, es decir que las medidas que se adopten no van a cambiar sustancialmente las dinámicas propias de la institución. Si en verdad se tratara de hacer un plan de
mejoramiento que intervenga para elevar el ISCE, hay que mirar el plan de estudios y el PEI, sin duda. Pero es que eso es lo que hace permanentemente la Institución, para eso existen los cuerpos colegiados y todo el gobierno escolar. Pareciera que el Ministerio desconfiara de sus rectores, de sus maestros y de la
comunidad educativa toda. ¿Qué creen que se hace en las jornadas de desarrollo institucional? ¿qué creen que hacen los maestros bimestre a bimestre, o semestre a semestre cuando evalúan a sus estudiantes?, ¿cuando entregan boletines a los
padres de familia?, ¿cuando conversan con sus estudiantes sobre los problemas que tienen en sus materias, o en la familia, o en el barrio?. No vamos a extendernos acá, pero la vida de los colegios es intensa y los directivos y los maestros están durante toda una jornada, y más, atendiendo, no solamente el día a día de la tarea educativa, sino planes estratégicos, planes de mejoramiento, programas y proyectos que buscan permanentemente mejorar su trabajo y el rendimiento de sus estudiantes.
 
La jornada E, en ese sentido es absolutamente insuficiente, lo que requieren los colegios, sus directivos y maestros y la comunidad en general es mejorar las condiciones de infraestructura, las posibilidades de acceso, de ascequibilidad, de aceptabilidad y de adaptabilidad (las 4 A) de las que habló Katerin Tomasevsky), es decir necesitan que el Derecho a la educación sea plenamente garantizado por el Estado. Por supuesto que allí las comunidades y los maestros tiene un papel, pero el día E, reduce el problema de la calidad a un Plan que la institución elabore en un día en torno a cuatro indicadores que como ya dijimos son muy pobres para
dar cuenta de la compleja tarea de la formación de niños y jóvenes.
\subsubsection*{Una medida infantilizante}
Lo que más preocupa de esta propuesta es que viene diseñada con un taller que determina minuto a minuto lo que se debe hacer. Es un taller que desdice de la profesionalidad con la que maestros y directivos trabajan. En primer lugar se usa una metáfora a todas luces infantilizante. Comparar la práctica pedagógica, el quehacer del maestro y la administración de los colegios con un partido de futbol y con el esfuerzo que la selección Colombia hizo en el pasado mundial para superar los resultados de los años anteriores, es creer que se trata de una competencia y que todo es asunto de actitud, de dedicación,
sacrificio y esfuerzo. Nada más ajeno a lo que se hace en la escuela. La calidad no depende de la actitud que se asuma frente a ella, es una factor a tener en cuenta, sin duda, pero el asunto no se puede reducir a eso. Esto ya está dicho. El
punto acá es el intento de llevar a los maestros de la mano de Mario Alberto Yepes (a quien respetamos como futbolista, pero no es pedagogo) con un video motivacional estilo coaching, que se envió a todos los colegios, para que tomen conciencia de la importancia de su labor y para que se esfuercen más, tal como lo hicieron los futbolistas en el mundial.

Luego, diseñaron unas actividades para realizar en grupos, o por parejas, para que se haga evidente que es necesario trabajar en grupo y que todos unidos lograremos llegar más alto en el cumplimiento de las metas que el ISCE establece.
Están previstos los minutos de descanso, el paso a paso de cada actividad y lo que textualmente debe decir el moderador. En realidad no será un moderador quien dirija la jornada sino un ventrílocuo que debe repetir sin matices los textos
incorporados en el material que se entrega, esto cuando no haya ayudas tecnológicas; si las hay, se enviaron videos con un personaje hablando para cumplir el rol. Junto a ello hay hojas para elaborar avioncitos de papel, entre otros materiales para recortar, pintar y plegar.

Son seis horas de actividades dirigidas que ni siquiera se le dejó a la imaginación del gobierno escolar para que las diseñara; seis horas para que al final se haga el Plan de M.M.A. que deberá entregarse al MEN, desde donde se hará un
seguimiento riguroso para que se cumpla. Si se logran las metas, habrá un estímulo económico equivalente a un salario para los maestros y los directivos. Suponemos que esto es resultado de lo que la OCDE y el Estudio Compartir (2014) recomendaron con respecto al salario de los maestros que urge equilibrar como salarios profesionales. Imaginamos que ante la decisión gubernamental de incrementar los salarios prefirieron hacerlo condicionado al mejoramiento del rendimiento en los aprendizajes; es la lógica empresarial: “te estimulo si me das
resultados”.
\subsubsection*{Preguntas finales}
A los creativos del Ministerio que diseñaron esta jornada y para los políticos que decidieron mejorar los salarios por esta vía del estímulo a los mejores, quisiera dejarles planteadas unas preguntas:

¿Están seguros que la calidad de la educación es equiparable a los resultados de aprendizaje medidos en las pruebas saber?

¿Están seguros que el ISCE es suficiente para determinar el grado en que se encuentra la calidad educativa en cada colegio?

¿Creen de verdad que con el plan de M.M.A. se va a mejorar ese índice?

¿Qué hacen los directivos, los maestros y la comunidad educativa con sus planes de desarrollo institucional?, ¿los reducen a este plan de M.M.A? ¿sobran? Porque en realidad este plan que surge de este día E, es mucho más sencillo de elaborar
y de hacerle seguimiento; lo que no es seguro es que mejoren los indicadores, por la sencilla razón de que el ISCE no tiene en cuenta los factores más determinantes en los procesos de enseñanza-aprendizaje.

Para las comunidades educativas, especialmente los maestros y directivos:

¿Cómo se sienten tratados con este taller y con esta convocatoria?

Yo me siento ofendido
\end{document}