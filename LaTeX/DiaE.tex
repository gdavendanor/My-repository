\documentclass[10pt]{article}
\usepackage[utf8]{inputenc}
\usepackage[spanish]{babel}
\usepackage{graphicx}
\usepackage{lmodern}
\usepackage[papersize={5.5in,8.5in},left=1.25cm,right=1.25cm,top=1.25cm,bottom=1.25cm]{geometry}
\usepackage{url}
\author{Comité sindical Arborizadora Baja j.m.}
\title{DESOBEDIENCIA CIVIL DIA E}
\date{25 de marzo de 2015}
\begin{document}
\maketitle
Cordial saludo compañeros y compañeras

A continuación presentamos a ustedes las actividades sugeridas por nuestra organización sindical para la jornada de desobediencia civil.
\begin{itemize}
\item[a.] No descargar el protocolo o reporte del ISCE
\item[b.] No realizar el taller, no seguir ninguno de sus pasos, no ver los videos y no colocar los afiches
\item[c.] No diligenciar los formatos solicitados de los acuerdos que se quieren imponer.
\item[d.] Hacer un taller alternativo que analice la responsabilidad de las políticas educativas y de las autoridades de gobierno en los resultados de la educación, junto con lo que se busca imponer en el plan de desarrollo
\item[e.] Concluir el taller alternativo con un escrito que recoja peticiones sobre lo que las instituciones educativas, el magisterio y el alumnado requieren para ofrecer unas condiciones distintas en el sistema educativo
\end{itemize}
\section*{Taller alternativo}
\begin{enumerate}
\item Ver el video-documental “Granito de arena- la educación en México y sus problemas”. Disponible en: \url{/https://www.youtube.com/watch?v=Mu2E-iTARJo&list=PLS09vmehZoIdDLV3vXrtzHsCrcAQNSCIW/}
\item Con base en lo visto en el video hacer una comparación entre la situación de México y la de la educación en Colombia. Sacar unas conclusiones enfatizando en nuestro país y en la problemática de la institución educativa.
\item Hacer lectura de los documentos de análisis y critica sobre el día E y el Índice Sintético de la Calidad de la Educación ISCE.
\item Recabar análisis críticos sobre las pruebas PISA y SABER y el informe COMPARTIR.
\item Colectivamente analizar la incidencia de las políticas educativas en la educación y de lo contemplado en el plan de desarrollo
\item Identificar problemas cruciales de la educación: presupuesto y financiación, políticas nacionales y municipales, hábitat y condiciones espaciales de los estudiantes en el aula y demás espacios educativos o posibles situaciones de hacinamiento, condiciones de plantas físicas, dotación de plantas físicas y espacios educativos, disponibilidad de materiales educativos y didácticos, condiciones sociales de los estudiantes, espacios para la reflexión académica y pedagógica, condiciones laborales de la profesión docente, y otras que cada institución quiera agregar.
\item Elaborar un escrito de protesta y unas peticiones para modificar las condiciones físicas y académicas de las instituciones educativas
\end{enumerate}
\section*{Otros insumos}

\end{document}