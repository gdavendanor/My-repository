\documentclass[10pt]{article}
\usepackage[utf8]{inputenc}
\usepackage[spanish]{babel}
\usepackage{graphicx}
\usepackage{amsmath}
\usepackage{amsfonts}
\usepackage{amssymb}
\usepackage{lmodern}
\usepackage[papersize={5.5in,8.5in},left=1.1cm,right=1.1cm,top=1.25cm,bottom=1.25cm]{geometry}
\usepackage{url}
\author{Comité sindical Arborizadora Baja j.m.}
\title{DESOBEDIENCIA CIVIL DIA E}
\date{25 de marzo de 2015}
\begin{document}
\maketitle
Cordial saludo compañeros y compañeras\\

A continuación presentamos a ustedes las actividades sugeridas por nuestra organización sindical para la jornada de desobediencia civil.
\begin{itemize}
\item[a.] No descargar el protocolo o reporte del ISCE
\item[b.] No realizar el taller, no seguir ninguno de sus pasos, no ver los videos y no colocar los afiches
\item[c.] No diligenciar los formatos solicitados de los acuerdos que se quieren imponer.
\item[d.] Hacer un taller alternativo que analice la responsabilidad de las políticas educativas y de las autoridades de gobierno en los resultados de la educación, junto con lo que se busca imponer en el plan de desarrollo
\item[e.] Concluir el taller alternativo con un escrito que recoja peticiones sobre lo que las instituciones educativas, el magisterio y el alumnado requieren para ofrecer unas condiciones distintas en el sistema educativo
\end{itemize}
\section*{Taller alternativo}
\begin{enumerate}
\item Ver el video-documental “Granito de arena- la educación en México y sus problemas”. Disponible en: \url{/https://www.youtube.com/watch?v=Mu2E-iTARJo&list=PLS09vmehZoIdDLV3vXrtzHsCrcAQNSCIW/}
\item Con base en lo visto en el video hacer una comparación entre la situación de México y la de la educación en Colombia. Sacar unas conclusiones enfatizando en nuestro país y en la problemática de la institución educativa.
\item Hacer lectura de los documentos de análisis y critica sobre el día E y el Índice Sintético de la Calidad de la Educación ISCE.
\item Recabar análisis críticos sobre las pruebas PISA y SABER y el informe COMPARTIR.
\item Colectivamente analizar la incidencia de las políticas educativas en la educación y de lo contemplado en el plan de desarrollo
\item Identificar problemas cruciales de la educación: presupuesto y financiación, políticas nacionales y municipales, hábitat y condiciones espaciales de los estudiantes en el aula y demás espacios educativos o posibles situaciones de hacinamiento, condiciones de plantas físicas, dotación de plantas físicas y espacios educativos, disponibilidad de materiales educativos y didácticos, condiciones sociales de los estudiantes, espacios para la reflexión académica y pedagógica, condiciones laborales de la profesión docente, y otras que cada institución quiera agregar.
\item Elaborar un escrito de protesta y unas peticiones para modificar las condiciones físicas y académicas de las instituciones educativas
\end{enumerate}
\section*{Otros insumos}
\subsection*{A prop\'{o}sito del d\'{i}a E\footnote{Alejandro \'{A}lvarez Gallego, Rector IPN}}
Para el 25 de marzo del año 2015 el Ministerio de Educación convocó una jornada de trabajo en todos y cada uno de los colegios oficiales y privados del país, que han llamado día E, o día de la excelencia educativa. Este día queda institucionalizado para realizar todos los años, según Decreto Ministerial 0325 de 2015 (aunque en la página del MEN se dice que es un decreto presidencial, en realidad está firmado por la Ministra). Con un despliegue publicitario propio de las campañas de Estado, se ha logrado crear una expectativa importante alrededor del tema. Por ser la educación uno de los tres ejes prioritarios de las políticas del actual gobierno, el mismo presidente de la República está encabezando la campaña y prometió asistir a un colegio a dirigir la jornada.

En mi condición de pedagogo\footnote{Mi	postura	no es institucional, es personal}, no puedo dejar de hacer algunas observaciones que pueden contribuir a la discusión que esta propuesta genera, asumiendo que en ella se involucra a las comunidades educativas que son, por antonomasia,deliberantes y tienen por Constitución responsabilidades en la definición de las políticas educativas. Me permito calificar esta jornada no como el día de la excelencia, sino como el de las cuatro íes: por inconsulta, ingenua, insuficiente e infantilizante.

Calidad Educativa ISCE, con el cuál se anuncia que se podrá medir mejor este esquivo propósito. La Jornada tiene como finalidad elaborar un plan anual institucional para lograr una meta de Mejoramiento Mínimo Anual (M.M.A); dicho plan debe proponer cómo mejorar cada uno de los indicadores que componen el índice sintético, con lo cuál se avanzaría en la calidad educativa. Los cuatro indicadores o componentes del índice son: Progreso, que mide cuánto mejora año a año el resultado de las pruebas Saber de lenguaje y matemáticas en 3$^{\circ}$, 5$^{\circ}$ y 9$^{\circ}$ grados -- Desempeño, mide el promedio institucional en las mismas pruebas, comparado con el promedio nacional y la entidad territorial -- Eficiencia, mide la tasa de promoción de los estudiantes de un grado a otro -- Ambiente escolar, mide el seguimiento que los maestros hacen a las tareas de los estudiantes y el ambiente del aula, esto según encuestas aplicadas a los estudiantes en el momento de contestar las pruebas saber.

¿Quién, cómo y cuando definió que este índice mediría la calidad? ¿Quién, cómo y cuándo definió que esta estrategia de elaborar en un día un Plan de M.M.A sería la clave para mejorar la calidad?. Por lo expresado en el decreto, el Ministerio se
arroga esta función haciendo mención a la responsabilidad constitucional de ejercer la suprema inspección y vigilancia (artículo 67 de la Constitución) y a la función establecida en la Ley 115, de  "(\ldots) atender en forma permanente los factores que favorecen la calidad y el mejoramiento de la educación".

Considero que con esta medida el MEN se extralimita en sus funciones, pues cuando la Constitución y la Ley les otorga estas funciones les está diciendo expresamente que deben “(\ldots) garantizar el adecuado cubrimiento del servicio y asegurar a los menores las condiciones para su acceso y permanencia en el sistema educativo”; y la Ley 115 también les dice en el artículo citado que los factores que favorecen la calidad son, especialmente “(\ldots) la cualificación y
formación de os educadores, la promoción docente, los recursos y métodos educativos, la innovación e investigación educativa, la orientación educativa y profesional, la inspección y evaluación del proceso educativo”. Todo esto es lo que en realidad falta, todo esto es lo que debería el Estado estar procurando y es lo que tiene un rezago de décadas. No hay que ser un exégeta agudo para darse cuenta que el Estado debe inspeccionar y evaluar, pero sobre la base de que garantice las condiciones adecuadas para prestar un servicio de calidad y con
suficiente cubrimiento; y esto es lo que no hay.

El Ministerio se extralimita pues la tarea de orientar las instituciones educativas de acuerdo a un Proyecto Educativo Institucional, a un modelo pedagógico y a un currículo flexible, le corresponde al gobierno escolar constituido por un Consejo Académico, un Consejo Directivo colegiado, un estudiante Personero, un Consejo de Padres y un Consejo Estudiantil, sin contar con la libertad de cátedra y la
autonomía profesional que tienen los maestros en la toma de decisiones sobre la enseñanza. Pero además olvida que la Ley ordenó que las políticas educativas fueran discutidas en Foros anuales municipales, departamentales y nacional, con la participación de toda la comunidad educativa, y que deberían funcionar Juntas colegiadas municipales, departamentales y nacional, para asesorar a las instancias de gobierno en la toma de decisiones. Nada de esto se tuvo en cuenta para crear el ISCE y el día E.
\subsubsection*{Una medida ingenua}
La calidad es una noción polisémica y ha servido para que Tiirios y Troyanos justifiquen cualquier acción. Una administración anterior había definido que la calidad era un círculo virtuoso que funcionaba con la evaluación estandarizada en pruebas censales de competencias básicas (pruebas que dicha administración llamó Saber), luego con la identificación de las deficiencias en los aprendizajes de estudiantes detectadas a través de esas pruebas y finalmente con estrategias de capacitación de los docentes responsables de dichas deficiencias. Ya ese supuesto era ingenuo, pues reducía la calidad a aprendizajes de competencias básicas y responsabilizaba a los maestros de las deficiencias identificadas. Aunque el modelo funcionó por varios años, los resultados están a la vista: sus propios sistemas de medición no han movido la aguja con la que se cree medir un asunto tan complejo.

Con esta sorpresiva medida (el día E) se quieren enriquecer los indicadores de medición. Se advierte que las solas pruebas saber no son suficientes para determinar cómo va la calidad, sin embargo dos de los cuatro componentes del índice siguen centrados en ellas. Lo nuevo sería el indicador de promoción anual y el ambiente escolar, dividido en dos preguntas que contestaron los alumnos de tres grados; es ingenuo creer que estas dos variables tan escuetas puedan aportar algo significativo al complejo asunto de la calidad. Todo este despliegue publicitario y de recursos en torno a la jornada no se compadece con la urgencia que tiene el país de tomar en serio el problema.

La Secretaría de Educación de Bogotá ha querido agregar dos variables más, que son de suma importancia para tener en cuenta: el número de estudiantes y de sedes por institución y la procedencia socioeconómica de los estudiantes. Pero agregar estos dos items significa entonces que para mejorar el ISCE no basta un Plan de M.M.A. con el que los maestros y las directivas se comprometan, pues de ellos no dependen tales variables tan determinantes en los resultados de las pruebas, en la promoción y en el ambiente escolar.

Creer que con un Plan de M.M.A de estas características, hecho en un día, se va a lograr mejorar la calidad y alcanzar que Colombia sea la más educada en el año 2025, es a todas luces ingenuo. Es demasiada aspiración para tan precaria medida.
\subsubsection*{Una medida insuficiente}
En aras de la discusión podríamos decir que esta medida no es suficiente pero ayuda, pero el problema es otro. Los planes de mejoramiento están instituidos en las dinámicas de los colegios desde la Ley 115. El Ministerio se adelantó a decir que esta jornada no era para ajustar el plan de estudios ni para modificar el PEI, es decir que las medidas que se adopten no van a cambiar sustancialmente las dinámicas propias de la institución. Si en verdad se tratara de hacer un plan de
mejoramiento que intervenga para elevar el ISCE, hay que mirar el plan de estudios y el PEI, sin duda. Pero es que eso es lo que hace permanentemente la Institución, para eso existen los cuerpos colegiados y todo el gobierno escolar. Pareciera que el Ministerio desconfiara de sus rectores, de sus maestros y de la
comunidad educativa toda. ¿Qué creen que se hace en las jornadas de desarrollo institucional? ¿qué creen que hacen los maestros bimestre a bimestre, o semestre a semestre cuando evalúan a sus estudiantes?, ¿cuando entregan boletines a los
padres de familia?, ¿cuando conversan con sus estudiantes sobre los problemas que tienen en sus materias, o en la familia, o en el barrio?. No vamos a extendernos acá, pero la vida de los colegios es intensa y los directivos y los maestros están durante toda una jornada, y más, atendiendo, no solamente el día a día de la tarea educativa, sino planes estratégicos, planes de mejoramiento, programas y proyectos que buscan permanentemente mejorar su trabajo y el rendimiento de sus estudiantes.
 
La jornada E, en ese sentido es absolutamente insuficiente, lo que requieren los colegios, sus directivos y maestros y la comunidad en general es mejorar las condiciones de infraestructura, las posibilidades de acceso, de ascequibilidad, de aceptabilidad y de adaptabilidad (las 4 A) de las que habló Katerin Tomasevsky), es decir necesitan que el Derecho a la educación sea plenamente garantizado por el Estado. Por supuesto que allí las comunidades y los maestros tiene un papel, pero el día E, reduce el problema de la calidad a un Plan que la institución elabore en un día en torno a cuatro indicadores que como ya dijimos son muy pobres para
dar cuenta de la compleja tarea de la formación de niños y jóvenes.
\subsubsection*{Una medida infantilizante}
Lo que más preocupa de esta propuesta es que viene diseñada con un taller que determina minuto a minuto lo que se debe hacer. Es un taller que desdice de la profesionalidad con la que maestros y directivos trabajan. En primer lugar se usa una metáfora a todas luces infantilizante. Comparar la práctica pedagógica, el quehacer del maestro y la administración de los colegios con un partido de futbol y con el esfuerzo que la selección Colombia hizo en el pasado mundial para superar los resultados de los años anteriores, es creer que se trata de una competencia y que todo es asunto de actitud, de dedicación,
sacrificio y esfuerzo. Nada más ajeno a lo que se hace en la escuela. La calidad no depende de la actitud que se asuma frente a ella, es una factor a tener en cuenta, sin duda, pero el asunto no se puede reducir a eso. Esto ya está dicho. El
punto acá es el intento de llevar a los maestros de la mano de Mario Alberto Yepes (a quien respetamos como futbolista, pero no es pedagogo) con un video motivacional estilo coaching, que se envió a todos los colegios, para que tomen conciencia de la importancia de su labor y para que se esfuercen más, tal como lo hicieron los futbolistas en el mundial.

Luego, diseñaron unas actividades para realizar en grupos, o por parejas, para que se haga evidente que es necesario trabajar en grupo y que todos unidos lograremos llegar más alto en el cumplimiento de las metas que el ISCE establece.
Están previstos los minutos de descanso, el paso a paso de cada actividad y lo que textualmente debe decir el moderador. En realidad no será un moderador quien dirija la jornada sino un ventrílocuo que debe repetir sin matices los textos
incorporados en el material que se entrega, esto cuando no haya ayudas tecnológicas; si las hay, se enviaron videos con un personaje hablando para cumplir el rol. Junto a ello hay hojas para elaborar avioncitos de papel, entre otros materiales para recortar, pintar y plegar.

Son seis horas de actividades dirigidas que ni siquiera se le dejó a la imaginación del gobierno escolar para que las diseñara; seis horas para que al final se haga el Plan de M.M.A. que deberá entregarse al MEN, desde donde se hará un
seguimiento riguroso para que se cumpla. Si se logran las metas, habrá un estímulo económico equivalente a un salario para los maestros y los directivos. Suponemos que esto es resultado de lo que la OCDE y el Estudio Compartir (2014) recomendaron con respecto al salario de los maestros que urge equilibrar como salarios profesionales. Imaginamos que ante la decisión gubernamental de incrementar los salarios prefirieron hacerlo condicionado al mejoramiento del rendimiento en los aprendizajes; es la lógica empresarial: “te estimulo si me das
resultados”.
\subsubsection*{Preguntas finales}
A los creativos del Ministerio que diseñaron esta jornada y para los políticos que decidieron mejorar los salarios por esta vía del estímulo a los mejores, quisiera dejarles planteadas unas preguntas:

¿Están seguros que la calidad de la educación es equiparable a los resultados de aprendizaje medidos en las pruebas saber?

¿Están seguros que el ISCE es suficiente para determinar el grado en que se encuentra la calidad educativa en cada colegio?

¿Creen de verdad que con el plan de M.M.A. se va a mejorar ese índice?

¿Qué hacen los directivos, los maestros y la comunidad educativa con sus planes de desarrollo institucional?, ¿los reducen a este plan de M.M.A? ¿sobran? Porque en realidad este plan que surge de este día E, es mucho más sencillo de elaborar
y de hacerle seguimiento; lo que no es seguro es que mejoren los indicadores, por la sencilla razón de que el ISCE no tiene en cuenta los factores más determinantes en los procesos de enseñanza-aprendizaje.

Para las comunidades educativas, especialmente los maestros y directivos:

¿Cómo se sienten tratados con este taller y con esta convocatoria?

Yo me siento ofendido
\subsection*{DEL DÍA E AL DÍA I DE INDIGNACIÓN.
DESOBEDIENCIA CIVIL A UNA NEFASTA POLÍTICA\footnote{CEID FECODE}}
Indignación es lo que produce la expedición del Decreto 0325 de 2015 que obliga a la celebración del Día E en las instituciones educativas del país, es eufemísticamente llamado así porque se le asignó la función de estar dedicado a la excelencia educativa, es una carga más para la rendición de cuentas de las instituciones educativas, agobiadas, desde hace mucho tiempo, por el cúmulo de demandas y de activismo indiscriminado que satura el quehacer de las instituciones educativas con un sinfín de procedimientos técnico instrumentales que no tienen ningún sentido educativo. \textbf{Indignación} por varios motivos: porque el Día E no es ninguna propuesta novedosa, sino que es una mentirosa estrategia publicitaria para promover una falsa idea de la imagen de este gobierno; porque es una acción puntual que aborda los resultados pero desconoce las realidades educativas del país; porque es una jornada que no logrará nada, ni siquiera su institucionalización tendrá efecto en mejorar aspectos importantes del sistema educativo colombiano.

Con políticas continuistas y estrategias como estas que no apuntan a fortalecer la financiación, ni a transformar las condiciones materiales, pedagógicas y culturales de la
escuela y su entorno, no se logrará un avance importante en materia educativa. Ni esta estrategia ni las anteriores que la política educativa ha implementado de tiempo atrás, atienden a las condiciones y procesos de lo que debería ser la educación; por el contrario, la despedagogizan, la desfinancian, la someten al control del gasto público, la condenan al
hacinamiento y la marginación, a su confinamiento en plantas físicas deterioradas que únicamente cuentan, en muchos casos, con salones de clase que funcionan con un mínimo de profesores y sin personal administrativo y de servicios generales, sin recursos económicos para mejoras en las plantas, sin dotación ni recursos didácticos; son estrategias que buscan privatizarla, reducir el rol del Estado, destruir el PEI y la autonomía escolar, empobrecer el currículo en la perspectiva de competencias y estándares; y que proyectan precarizar las condiciones salariales y académicas del magisterio pisoteando su dignidad y el derecho a una educación digna a los estudiantes. Así, estaremos muy lejos de ser “la más educada”, como lo anuncian con tono rimbombante los gobernantes actuales.

\textbf{Indignación} es lo que genera este tipo de actividades instrumentales que no apuntan al mejoramiento estructural de la educación. \textbf{Indignación}, porque en lugar de buscar las condiciones, oportunidades y posibilidades para mejorar cualitativamente la educación en nuestro país, se proponen este tipo de estrategias cínicas con el propósito de reforzar las
políticas educativas neoliberales y esconder, bajo maquillaje mediático, sus rotundos fracasos. Anuncios como este, buscan mayor impacto publicitario, se hacen para generar una campaña propagandística destinada a remozar la imagen del gobernante de turno; sin efecto en la mejora de lo que anuncian. \textbf{Indignación}, porque la educación se pone al servicio de la propaganda de gobierno con el propósito de elevar ficticiamente la imagen
del presidente actual, sacrificando las condiciones de posibilidad para un mejor presente y futuro de la educación. El Día E se presenta como noticia, pero en realidad es propaganda,
es publicidad bien manipulada y llena de sensiblería que relaciona el buen papel desempeñado por la selección de fútbol de Colombia en el pasado mundial, con el anhelo de paz y una mejor educación; campaña que además es muy costosa en su producción y divulgación, dinero que bien podría utilizarse de manera directa en beneficio de los estudiantes.

Curiosamente esa publicidad manifiesta: “nosotros también demostraremos que comenzará a cambiar la historia de la educación”. Por eso, \textbf{indigna} el cinismo de tal afirmación, propio de las políticas educativas que desde hace más de dos décadas buscan perpetuarse. Y el Día E no va a cambiar ninguna historia, es solamente una estrategia más para imponer la política en curso. Descaradamente va a reforzar la misma historia que ya tenemos, pues todo el montaje del Día E está diseñado para justificar y profundizar las pruebas SABER,
que son el vértice donde converge la política de privatización y control ideológico sobre la educación, para someterla a los destinos de las lógicas del mercado que producen la
enajenación del conocimiento y del ser humano mediante su subsunción como mercancías.

También \textbf{indigna} el tratamiento que se le da al calendario y a las jornadas escolares. De un tiempo atrás se suprimieron las jornadas pedagógicas y se confinaron a las semanas de desarrollo institucional. Ahora sí resulta que una sola jornada de instrucción/entrenamiento y rendición de cuentas es importante, mucho más que el receso escolar y las semanas de planeamiento institucional, El instructivo del gobierno es ofensivo e irrespetuoso al señalar que no es una jornada para el PEI, para la autonomía escolar o para revisar el plan de estudios; es como si esos procesos no fueran importantes y se tuvieran que reemplazar por la jornada del Día E, anunciando también que el plan de estudios, o el estudio del currículo y del PEI se deben cambiar por la revisión instrumental de los resultados institucionales en las pruebas SABER. Esta estrategia es de un solo día, cuando la reflexión pedagógica debe ser constante y permanente todo el año, organizada y distribuida durante los períodos lectivos, realizada mediante jornadas pedagógicas que estén relacionadas con las de planeamiento y desarrollo institucional

\textbf{Indigna} la forma como los procesos pedagógicos se reducen a acciones puntuales que se revisan y deciden en un solo día. Hay un doble reduccionismo en esta acción: por un lado, todos los procesos del año escolar se confinan a las revisiones y prescripciones en un solo día, y por otro, las dinámicas escolares se reducen únicamente a resultados parciales, desconociendo la complejidad de los procesos pedagógicos, didácticos y de vida que se dan al interior de las instituciones educativas y las condiciones de posibilidad que los constituyen. Un día del año para revisar los resultados institucionales y definir acciones es la aplicación de una visión simplista, reduccionista, inmediatista y cortoplacista sobre la educación. El acontecimiento educativo es continuo, cotidiano y transcurre sucesivamente en los días de la vida escolar; por eso, la revisión de lo que ocurre en esos tiempos y espacios no es de un solo día, es constante y permanente como el acontecer de lo cotidiano. 

\textbf{Indigna} que toda la actividad se hace para individualizar únicamente en la institución educativa y en los docentes recae toda la responsabilidad de los resultados académicos de los estudiantes. Así como hay responsabilidades colectivas, existen procesos que se escapan del ámbito de influencia que tienen las instituciones y los docentes. Preocupa cómo se genera una obsesión individualizante que delega toda la responsabilidad en institución y docente, pero al mismo tiempo, sirve de burladero de las responsabilidades que han tenido las equivocadas políticas neoliberales llevadas a cabo por tantos años. Mientras se delega en la institución y los docentes, se ocultan y evaden las responsabilidades del Estado frente a la educación, su financiación y la creación de las condiciones materiales, pedagógicas, didácticas y culturales que son bases fundamentales para generar los espacios y eventos en que se da la experiencia de la formación.

\textbf{Indigna} la forma como se reduce todo el proceso y la experiencia de formación, únicamente en la perspectiva de los resultados institucionales en las pruebas SABER. Toda la estrategia del Día E está centrada en revisar los resultados de éstas y en generar iniciativas para responder únicamente a ello. Hace más de cincuenta años que se aprendió que la preparación no era solamente para los exámenes, sino para la vida. Pero acciones que entronizan los exámenes o pruebas SABER cómo el único instrumento posible para la educación, sacrifican la idea de la formación para la vida y la reducen a preparación para los exámenes. Esto se busca con el Día E, es un dispositivo para adaptar funcionalmente a las instituciones educativas con la única meta de responder satisfactoriamente en las pruebas externas nacionales e internacionales como indicador que mide la eficiencia en la capacitación de capital humano.

\textbf{Indigna} que se utilice esta campaña publicitaria y todo su contenido, para buscar más razones que justifican la instrumentalización como vía a la privatización. De hecho, el
lanzamiento de la campaña se hizo desde un colegio en concesión, demostrando que el compromiso del gobierno es con la empresa privada, con los mercaderes de la educación, que como buenos seguidores del capitalismo cleptocrático, muestran su afán por apropiarse de los recursos públicos.

\textbf{Indigna} que se empodere el Índice Sintético de Calidad de la Educación --ISCE-, diseñado exclusivamente para cumplir los mandatos de la OCDE, porque está lejos de recoger una
idea de la realidad escolar, solo lee lo que le interesa a este organismo y al neoliberalismo, pues la única fuente de consulta para fijar este índice son las pruebas SABER, PISA y su
instrumento de factores asociados al desempeño educativo, instrumento que hace una lectura pobre y sesgada de algunos componentes del entorno escolar, sin hacer análisis
profundos que interroguen las razones y las causas de los aspectos sociales y personales que constituyen dificultades para los estudiantes. Es un índice de medición que refuerza la
individualización de la responsabilidad en las instituciones y los docentes, además, se presenta como un ranking para posicionar las clasificaciones y las odiosas comparaciones
que justifican mayor desigualdad. Es un capítulo más de las ideas que satanizan y al mismo tiempo idealizan la labor del docente, pues sigue siendo el villano y el héroe de la
excelencia educativa, sin considerar que el entorno social, las condiciones, las políticas educativas, los organismos financieros internacionales y los gobernantes tienen un culposo
papel y son responsables de las crisis en la educación.

Plantea un giro total en las prioridades educativas ubicando en el centro el entrenamiento para las pruebas SABER y PISA, sacrificando componentes propios de la educación. El ISCE únicamente se refiere a cuatro aspectos, desconociendo los fines de la educación y la formación integral, los contextos escolares y familiares de los estudiantes y el impacto
negativo de las políticas educativas. Estos son:
\begin{enumerate}
\item Desempeño actual: se refiere a los resultados de las pruebas SABER en relación con la región y el país.
\item Progreso en los últimos años: es la comparación de los resultados de las pruebas SABER de años anteriores para cuantificar el mejoramiento en los resultados.
\item Eficiencia: se refiere al cálculo de la tasa de repitencia. Instituciones que tiendan a una tasa de repitencia en cero o cercana a ella, supuestamente tienen más excelencia
educativa; este indicador promueve el retorno subrepticio de la promoción automática, que también ha sido responsable de los aspectos negativos que están ocurriendo en la educación.
\item Ambiente escolar: esta noción es compleja y de difícil caracterización e imposible de reducir a términos cuantitativos. Sin embargo, lo hacen. Para ello se valen de
indicadores centrados en el aprendizaje, utilizados en las encuestas de los factores asociados al desempeño escolar y que formaban parte del cuestionario cuando se aplicó la prueba SABER, que ocultan las dificultades generadas al interior de la
escuela por las condiciones de desigualdad social.
\end{enumerate}
En consecuencia, el ISCE posiciona las áreas de la prueba SABER como las únicas importantes del currículo, desconoce la formación integral, relega a un lugar secundario la
formación en humanidades, artes y deportes, enfatiza la formación por competencias, pretende imponer el regreso de la promoción automática y asume el ambiente escolar como
un problema interno del aula y no como un efecto de las condiciones sociales e históricas por las que atraviesa el país.

Otra consecuencia que pretende imponer esta estrategia es la justificación de las políticas de coacción sobre el docente y la institución educativa; las políticas de evaluación y control
que junto a ideas promovidas por COMPARTIR, Empresarios por la Educación, monitoreados por la OCDE, buscan posicionar la meritocracia como una manera de obtener cambios y reestructuración en la profesión y la carrera docente, modificando el régimen salarial mediante la flexibilización salarial y la creación de la remuneración por méritos o
incentivos. En todo este proceso se presenta a los docentes y a las instituciones educativas como incapaces e irresponsables, a la vez que justifican la implementación de incentivos y no de un salario que dignifique y reconozca a los docentes como profesionales. Además, condiciona los recursos destinados a las instituciones en la medida que éstos serán asignados de acuerdo al cumplimiento de las metas y acciones según el ISCE.

Por su parte, el instructivo llamado “taller” de “corte conductista” quiere concluir con unos acuerdos que comprometen a docentes e instituciones pero no a los gobiernos. Bajo la
tutela del neoliberalismo, los docentes hacemos planes de mejoramiento con resultados limitados, no por ausencia de compromiso y dedicación, sino por el impacto de las políticas
educativas en las instituciones y las condiciones del contexto, que no se compadecen con los problemas y necesidades educativas del pueblo colombiano; por el contrario, son un obstáculo para potenciar la capacidad formativa de las instituciones educativas.

Todo esto confirma que el gobierno, más que propiciar el diálogo con los diferentes sectores del país para dar solución real a las múltiples problemáticas políticas, sociales,
educativas y económicas, impone políticas que tienen como propósito generar las condiciones para cumplir con los requerimientos de la OCDE y acceder al podio de los
países integrantes de ésta. Por eso, no extrañan afirmaciones como: \emph{“Llegó la hora de que colegios pasen al tablero cada año. El Ministerio de Educación se la juega por un nuevo modelo de educación. Así se hará”}, claro autoritariamente, bajo el discurso de un país en paz y de un Plan de Desarrollo Nacional 2014-2018 cuyo lema “paz, equidad y educación” se convertirá en la práctica en simple retórica, tal como se manifiesta en el articulado propuesto para ser aprobado en El Congreso.
\subsection{EL DIA E EN RESUMIDAS CUENTAS, MOTIVOS PARA LA DESOBEDIENCIA}
\begin{enumerate}
\item Esa estrategia busca implementar la política que hay, política que lleva más de veinte años y que ha fracasado.
\item Es una estrategia de un día cuando la reflexión pedagógica debe ser constante y permanente todo el año
3. Desconoce las jornadas de desarrollo institucional y su trabajo: Desconoce el PEI y el trabajo del consejo académico, los comités de evaluación y promoción, y las jornadas pedagógicas.
\item El ISCE (Índice Sintético de la Calidad de la Educación) no recoge una idea de la realidad escolar
\item El ISCE y sus conclusiones individualizan las responsabiliza de lo educativo en la institución educativa y en los docentes desconociendo el impacto negativo de la política en ese proceso de deterioro. Evade la responsabilidad de la política del gobierno nacional y regional en ese proceso.
\item Busca favorecer la privatización de la educación y los colegios en concesiones por eso el lanzamiento de la campaña se lanzó desde una colegio en concesión. 
\item Plantea un giro total en las prioridades educativas colocando en el centro la preparación para las pruebas SABER. Plantea un giro total en las prioridades educativas colocando en el centro la preparación para las pruebas saber y
sacrificando todos los otros componentes de la educación. El ISCE, solo se refiere a 4 aspectos, dejando por fuera los fines de la educación y la formación integral. Los 4 aspectos y su impacto son:
\begin{enumerate}
\item Desempeño actual: se refiere a los resultados en las pruebas SABER
\item Progreso en los últimos años: es la comparación histórica en resultados en las pruebas saber de los años anteriores según la cual se puede ponderar el mejoramiento o empeoramiento en los resultados
\item Eficiencia, se refiere al cálculo de la tasa de repitencia. Quienes tiendan a tener una tasa de repitiencia en cero o cercana a cero, supuestamente tienen más excelencia educativa; entonces, este indicador promueve el retorno subrepticio de la promoción automática, que también ha sido responsable de los aspectos negativos que están ocurriendo en la educación.
\item Ambiente escolar. Si bien, esta noción es amplia y de difícil caracterización e imposible de reducir a términos cuantitativos. Sin embargo, lo hacen. Para ello se valen de dos indicadores utilizados en las encuestas a los factores asociados al desempeño escolar y que formaban parte del cuestionario cuando se aplicó la prueba SABER.
\item En consecuencia, el ISCE posiciona las áreas de la prueba saber cómo las únicas importantes del currículo, desconoce la formación integral, relega a un lugar secundario la formación en humanidades, artes y deportes, enfatiza la formación
por competencias, pretende imponer el regreso de la promoción automática y ve el ambiente escolar como un problema interno del aula y no como un efecto de las condiciones sociales e históricas por las que atraviesa el país.
\end{enumerate}
\item Justifica las políticas de coacción sobre el docente, políticas de evaluación y control, junto con la flexibilización salarial y la creación de la remuneración por méritos o
incentivos
\item El taller es una estrategia conductista que quiere concluir con unos acuerdos que refuerzan lo dicho en 6, (que compromete a docentes e instituciones pero no a los
gobiernos) Bajo la tutela del neoliberalismo llevamos años haciendo planes de apoyo y mejoramiento, dirigidos y controlados
\end{enumerate}
\end{document}