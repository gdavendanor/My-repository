% $Header: /Users/joseph/Documents/LaTeX/beamer/solutions/generic-talks/generic-ornate-15min-45min.en.tex,v 90e850259b8b 2007/01/28 20:48:30 tantau $

\documentclass{beamer}

% This file is a solution template for:

% - Giving a talk on some subject.
% - The talk is between 15min and 45min long.
% - Style is ornate.



% Copyright 2004 by Till Tantau <tantau@users.sourceforge.net>.
%
% In principle, this file can be redistributed and/or modified under
% the terms of the GNU Public License, version 2.
%
% However, this file is supposed to be a template to be modified
% for your own needs. For this reason, if you use this file as a
% template and not specifically distribute it as part of a another
% package/program, I grant the extra permission to freely copy and
% modify this file as you see fit and even to delete this copyright
% notice. 


\mode<presentation>
{
  \usetheme{Warsaw}
  % or ...

  \setbeamercovered{transparent}
  % or whatever (possibly just delete it)
}


\usepackage[spanish]{babel}
% or whatever

\usepackage[latin1]{inputenc}
% or whatever

\usepackage{times}
\usepackage[T1]{fontenc}
% Or whatever. Note that the encoding and the font should match. If T1
% does not look nice, try deleting the line with the fontenc.


\title[] % (optional, use only with long paper titles)
{Normas de urbanidad en reuniones de campo abierto}

\subtitle
{¿Qué son las normas de urbanidad?} % (optional)

\author[Arellano, Avendaño, Cubillos, Gallo, Pineda, Ruíz, Tovar] % (optional, use only with lots of authors)
{Camila Avendaño, William Ruiz, Sebastián Gallo, Sergio Arellano, Felipe Cubillos, Camilo Tovar, Sebastián Pineda}
% - Use the \inst{?} command only if the authors have different
%   affiliation.

\institute[Gabriel Bentacourt] % (optional, but mostly needed)
{
 \inst{Colegio Gabriel Betancourt Mejía}%
 Ética\\
 }
% - Use the \inst command only if there are several affiliations.
% - Keep it simple, no one is interested in your street address.

\date[] % (optional)
{4 de mayo de 2016}

\subject{Talks}
% This is only inserted into the PDF information catalog. Can be left
% out. 



% If you have a file called "university-logo-filename.xxx", where xxx
% is a graphic format that can be processed by latex or pdflatex,
% resp., then you can add a logo as follows:

% \pgfdeclareimage[height=0.5cm]{university-logo}{university-logo-filename}
% \logo{\pgfuseimage{university-logo}}



% Delete this, if you do not want the table of contents to pop up at
% the beginning of each subsection:
%\AtBeginSubsection[]
%{
 % \begin{frame}<beamer>{Outline}
 %   \tableofcontents[currentsection,currentsubsection]
 % \end{frame}
%}


% If you wish to uncover everything in a step-wise fashion, uncomment
% the following command: 

%\beamerdefaultoverlayspecification{<+->}


\begin{document}

\begin{frame}
  \titlepage
\end{frame}

\begin{frame}{Outline}
  \tableofcontents
  % You might wish to add the option [pausesections]
\end{frame}


% Since this a solution template for a generic talk, very little can
% be said about how it should be structured. However, the talk length
% of between 15min and 45min and the theme suggest that you stick to
% the following rules:  

% - Exactly two or three sections (other than the summary).
% - At *most* three subsections per section.
% - Talk about 30s to 2min per frame. So there should be between about
%   15 and 30 frames, all told.

\section{Urbanidad}

\begin{frame}{¿Qué es la urbanidad?}%{Subtitles are optional.}
La urbanidad es cortesanía, comedimiento, atención y buen modo. Todo esto contribuye a tener una mejor convivencia con los demás. Cualquier sociedad cuenta con unas normas de comportamiento no escritas en la mayor parte de los casos pero que sin su tutela nos haría ser un grupo de seres incivilizados.
\end{frame}

\begin{frame}{}
Decir perdón, gracias y por favor son fundamentales en el trato con la sociedad
\end{frame}
\section{Normas}
\begin{frame}{Primera norma}
La primera norma básica en espacios abiertos es el silencio
\end{frame}
\begin{frame}{Segunda norma}
Si se tienen niños estar pendientes que no molesten a otras personas
\end{frame}
\begin{frame}{Tercera norma}
Cuidado de no invadir el terreno de otras personas
\end{frame}
\begin{frame}{Cuarta norma}
No deben tirarse desperdicios en la calle
\end{frame}
\begin{frame}{Quinta norma}
Los animales de compañía deben tener una vigilancia constante y cumplir las reglamentaciones establecidas para ellos.
\end{frame}
\begin{frame}
No se debe utilizar el río o playa como si fuera bañera de casa.
\end{frame}

\begin{frame}{Séptima norma}
El fuego está totalmente prohibido en la playa y en el campo
\end{frame}

\end{document}


