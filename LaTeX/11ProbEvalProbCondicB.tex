\documentclass[fleqn]{article}
\usepackage[spanish,es-noshorthands]{babel}
\usepackage[utf8]{inputenc} 
\usepackage[left=1cm, right=1cm, top=1.5cm, bottom=1.7cm]{geometry}
\usepackage{mathexam}
\usepackage{amsmath}
\usepackage{graphicx}

\ExamClass{\includegraphics[height=16pt]{Images/logo-sed.png} Probabilidad $11^{\circ}$}
\ExamName{``Probabilidad condicional''}
\ExamHead{\includegraphics[height=16pt]{Images/logo-colegio.png} IEDAB}
\newcommand{\LineaNombre}{%
\par
\vspace{\baselineskip}
Nombre:\hrulefill \; Curso: \underline{\hspace*{36pt}} \; Fecha: \underline{\hspace*{2.5cm}} \relax
\par}
\let\ds\displaystyle

\begin{document}
\ExamInstrBox{
Observe como se desarrolla el ejemplo y con base en lo trabajado en el último taller, resuelva la presente evaluación.}
\LineaNombre
\section*{Ejercicio - Ejemplo}
\begin{enumerate}
\item[A] Una caja de galletas contiene siete galletas de chocolate y cuatro de coco, otra caja tiene cinco de chocolate y tres de coco. Si elegimos una caja al azar y de ella se extraen dos galletas sin reemplazamiento, calcula:
\begin{enumerate}
\item[i] ¿Cuál es la probabilidad de que las dos sean de chocolate?\\

Sea $A_{1}$: "Sacar galleta de chocolate por primera vez" y $A_{2}$: Sacar galleta por segunda vez" Mientras que I y II designan los sucesos de escoger la caja I o la caja II respectivamente. Por tanto la probabilidad pedida será:
\begin{align*}
P(A_{1}\cap A_{2})&=P(I)P(A_{1}|I)P(A_{2}|(A_{1}|I))+P(II)P(A_{1}|II)P(A_{2}|(A_{1}|II))\\
&=\left(\frac{1}{2}\right)\left(\frac{7}{7+4}\right)\left(\frac{6}{6+4}\right)+\left(\frac{1}{2}\right)\left(\frac{5}{5+3}\right)\left(\frac{4}{4+3}\right)\\
&=\left(\frac{1}{2}\right)\left(\frac{7}{11}\right)\left(\frac{6}{10}\right)+\left(\frac{1}{2}\right)\left(\frac{5}{8}\right)\left(\frac{4}{7}\right)\\
&=\frac{21}{110}+\frac{5}{28}\\
&=\frac{294}{1540}+\frac{275}{1540}\\
&=\frac{569}{1540}
\end{align*}
\item[ii] ¿Y de que las dos sean de coco?. Con un proceso similar al anterior se puede resolver.
\item[iii] ¿Cuál es la probabilidad de que sea una de chocolate y una de coco?
\end{enumerate}
\end{enumerate}
\subsection*{Soluci\'on}

 
\begin{enumerate}
	\item \label{prob:1} Una bola se extrae aleatoriamente de una caja que contiene 6 bolas rojas, 4 bolas blancas y 5 bolas azules. Determinar la probabilidad de que sea:
	\begin{enumerate}
	\item Roja \noanswer
	\item Blanca \noanswer
	\item Azul\noanswer
	\item No roja \noanswer
	\item Roja o blanca\noanswer
	\end{enumerate}
	\newpage
   \item Un talego contiene 7 bolas blancas y 3 bolas negras; otro contiene 4 bolas blancas y 5 bolas negras. Si se extrae una bola de cada talego, hallar la probabilidad de que:
      \begin{enumerate}
	 \item Ambas sean blancas\noanswer
	 \item ambas sean negras \noanswer
	 \item Una sea blanca y una sea negra\noanswer
      \end{enumerate}
   \item Se extraen tres bolas consecutivamente de la caja del problema~\ref{prob:1}. Hallar la probabilidad de que sea extraigan en el orden roja, blanca y azul si las bolas:
   \begin{enumerate}
   \item Se reemplazan\noanswer
   \item No se reemplazan\noanswer
   \end{enumerate}
   \end{enumerate}
\end{document}
