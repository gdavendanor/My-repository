\documentclass[letterpaper,11pt,twoside]{article}
\usepackage[utf8]{inputenc}
\usepackage{amsmath,amsfonts,amssymb,amsthm,latexsym}
\usepackage[spanish,es-noshorthands]{babel}
\usepackage[T1]{fontenc}
\usepackage{lmodern}
\usepackage{graphicx,hyperref}
\usepackage{tikz,pgf}
\usepackage{multicol}
\usepackage{fancyhdr}
\usepackage[height=9.5in,width=7in]{geometry}
\usepackage{fancyhdr}
\pagestyle{fancy}
\fancyhead[LE]{\includegraphics[height=12pt]{Images/logo-colegio.png} Probabilidad $11^{\circ}$}
\fancyhead[RE]{}
\fancyhead[RO]{\textit{Germ\'an Avenda\~no Ram\'irez, Lic. U.D., M.Sc. U.N.}}
\fancyhead[LO]{}

\author{Germ\'an Avenda\~no Ram\'irez, Lic. U.D., M.Sc. U.N.}
\title{\begin{minipage}{.2\textwidth}
\includegraphics[height=1.75cm]{Images/logo-colegio.png}\end{minipage}
\begin{minipage}{.55\textwidth}
\begin{center}
Taller, Probabilidad eventos empíricos\\
Probabilidad $11^{\circ}$
\end{center}
\end{minipage}\hfill
\begin{minipage}{.2\textwidth}
\includegraphics[height=1.75cm]{Images/logo-sed.png} 
\end{minipage}}
\date{}
\thispagestyle{plain}
\begin{document}
\maketitle
Nombre: \hrulefill Curso: \underline{\hspace*{44pt}} Fecha: \underline{\hspace*{2.5cm}}
\begin{multicols}{2}
 \begin{enumerate}
  \item Si usted lanza un dado 40 veces y 9 de los tiros resultan en un ``5'', ¿qué probabilidad empírica se observó para el evento ``5''?
  \item Explique por qué una probabilidad empírica, una proporción observada, y una frecuencia relativa son en realidad tres nombres diferentes para lo mismo.
  \item Millones de personas viajan en ferrocarril todos los años. La Asociación Nacional de Pasajeros de Ferrocarril proporciona las siguientes cantidades de viajes en 2004.
  \begin{table}
\begin{center}
\begin{tabular}{ll}
Sistema ferroviario & Viajeros (millones)\\ \hline
Sistema Amtrak & 25.0\\
Corredor Noroeste & 14.2\\
Suburbano + 0este & 10.8 \hline
  \end{tabular}
  \end{center}\caption{Fuente: Nationa Association of Railroad Passengers \url{http://www.infoplease.com/ipa/A0855824.html}}
              \end{table} 
\begin{enumerate}
 \item ¿Qué porcentaje de pasajeros de ferrocarril usaron
el sistema Amtrak en 2004?
\item Si uno de estos pasajeros ha de ser entrevistado, ¿cuál es la probabilidad de que él haya viajado en el sistema Amtrak en 2004 si es seleccionado al azar?
\item Explique la diferencia y la relación entre preguntas y respuestaas  a las partes \textit{a} y \textit{b}.
\end{enumerate}
\item El Webster Aquatic Center ofrece varios niveles de lecciones de natación todo el año. Las lecciones vespertinas de lunes y miércoles de marzo de 2005 incluyeron clases desde bebés a adultos. El números en cada clasificación aparece en la tabla siguiente:
\begin{table}
\begin{center}
\begin{tabular}{ll}
Tipo de lección de natación & Núm. de participantes\\ \hline
Bebés & 15\\
Bebé muy pequeño & 12\\
Renacuajos & 12\\
Nivel 2 & 15\\
Nivel 3 & 10\\
Nivel 4 & 6\\
Nivel 5 & 2\\
Nivel 6 & 1\\
Adultos & 4\\ \hline
Total & 77
\end{tabular}
\end{center}\caption{Fuente: American Payroll Association, \url{http://www.AmericanPayroll.org}}
            \end{table} 
Si se selecciona al azar un participante, encuentre la probabilidad de lo siguiente:
\begin{enumerate}
 \item El participante está en bebés muy pequeños
 \item El participante están en la lección para adultos
 \item El participante está en una lección de nivel 2 a nivel 6
\end{enumerate}
\item En septiembre de 2004, la American Payroll Association publicó los resultados de su encuesta nacional de semana de nómina 2004. Una de las preguntas
inquiría acerca del ingreso familiar anual.

 \end{enumerate}

\end{multicols}


\end{document}
