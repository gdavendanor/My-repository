\documentclass[11pt,twoside]{article}
\usepackage[utf8]{inputenc}
\usepackage{amsmath,amsfonts,amssymb,amsthm,latexsym}
\usepackage[spanish,es-noshorthands]{babel}
\usepackage[T1]{fontenc}
\usepackage{lmodern}
\usepackage{graphicx,hyperref}
\usepackage{tikz,pgf}
\usepackage{multicol}
%\usepackage{fancyhdr}
\usepackage[papersize={6.5in,8.5in},height=7in,width=5.5in]{geometry}
%\usepackage{fancyhdr}
\pagestyle{plain}
%\fancyhead[LE]{\includegraphics[height=12pt]{Images/logo-colegio.png} Probabilidad $11^{\circ}$}
%\fancyhead[RE]{}
%\fancyhead[RO]{\textit{Germ\'an Avenda\~no Ram\'irez, Lic. U.D., M.Sc. U.N.}}
%\fancyhead[LO]{}

\author{Germ\'an Avenda\~no Ram\'irez, \thanks{Lic. U.D., M.Sc. U.N., Email: matematicas.german@gmail.com, Web: %https://www.autistici.org/mathgerman
}}
\title{\begin{minipage}{.2\textwidth}
\includegraphics[height=1.75cm]{Images/logo-colegio.png}\end{minipage}
\begin{minipage}{.55\textwidth}
\begin{center}
Taller 02\\
Probabilidad eventos empíricos\\
Probabilidad $11^{\circ}$
\end{center}
\end{minipage}\hfill
\begin{minipage}{.2\textwidth}
\includegraphics[height=1.75cm]{Images/logo-sed.png} 
\end{minipage}}
\date{}
\thispagestyle{plain}
\begin{document}
\maketitle
Nombre: \hrulefill Curso: \underline{\hspace*{44pt}} Fecha: \underline{\hspace*{2.5cm}}
 \begin{enumerate}
  \item Si usted lanza un dado 40 veces y 9 de los tiros resultan en un ``5'', ¿qué probabilidad empírica se observó para el evento ``5''?
  \item Explique por qué una probabilidad empírica, una proporción observada, y una frecuencia relativa son en realidad tres nombres diferentes para lo mismo.
  \item Millones de personas viajan en ferrocarril todos los años. La Asociación Nacional de Pasajeros de Ferrocarril proporciona las siguientes cantidades de viajes en 2004.
  \begin{table}[h!]
\begin{center}
\begin{tabular}{ll}
Sistema ferroviario & Viajeros (millones)\\ \hline
Sistema Amtrak & 25.0\\
Corredor Noroeste & 14.2\\
Suburbano + 0este & 10.8 \\ \hline
  \end{tabular}\caption{Fuente: National Association of Railroad Passengers \url{http://www.infoplease.com/ipa/A0855824.html}}
  \end{center}
              \end{table} 
\begin{enumerate}
 \item ¿Qué porcentaje de pasajeros de ferrocarril usaron
el sistema Amtrak en 2004?
\item Si uno de estos pasajeros ha de ser entrevistado, ¿cuál es la probabilidad de que él haya viajado en el sistema Amtrak en 2004 si es seleccionado al azar?
\item Explique la diferencia y la relación entre preguntas y respuestaas  a las partes \textit{a} y \textit{b}.
\end{enumerate}
\item El Webster Aquatic Center ofrece varios niveles de lecciones de natación todo el año. Las lecciones vespertinas de lunes y miércoles de marzo de 2005 incluyeron clases desde bebés a adultos. El números en cada clasificación aparece en la tabla siguiente:
\begin{center}
\begin{tabular}{ll}
Tipo de lección de natación & Núm. de participantes\\ \hline
Bebés & 15\\
Bebé muy pequeño & 12\\
Renacuajos & 12\\
Nivel 2 & 15\\
Nivel 3 & 10\\
Nivel 4 & 6\\
Nivel 5 & 2\\
Nivel 6 & 1\\
Adultos & 4\\ \hline
Total & 77
\end{tabular}
\end{center}
Si se selecciona al azar un participante, encuentre la probabilidad de lo siguiente:
\begin{enumerate}
 \item El participante está en bebés muy pequeños
 \item El participante están en la lección para adultos
 \item El participante está en una lección de nivel 2 a nivel 6
\end{enumerate}
\item En septiembre de 2004, la American Payroll Association publicó los resultados de su encuesta nacional de semana de nómina 2004. Una de las preguntas inquiría acerca del ingreso familiar anual.\\
Suponga que una de las personas que respondieron las encuesta ha de ser seleccionado al azar para una entrevista de seguimiento. Encuentre la probabilidad de los siguientes eventos.
\begin{table}[h!]
\begin{center}
\begin{tabular}{llr@{.}l}
Ingreso familiar anual & Número &\multicolumn{2}{l}{Porcentaje} \\ \hline
Menos de \$15\,000 & 423 & 1&9\% \\
\$15\,001--\$30\,000 & 2225 & 9&8\%\\
\$30\,001--\$50\,000 & 5394 & 223&9\%\\
\$50\,001--\$75\,000 & 5772 & 25&5\%\\
\$75\,001--\$100\,000 & 4730 & 20&9\%\\
\$100\,001--\$150\,000 & 3065 & 13&6\%\\
Más de \$150\,000 & 984 & 4&4\% \\\hline
\end{tabular}\caption{Fuente: American Payroll Association, \url{http://www.AmericanPayroll.org}}
\end{center}
            \end{table} 

\begin{enumerate}
 \item El ingreso familiar del encuestado es \$50\,000 o menos
 \item El ingreso familiar del encuestado es \$75\,001 o más
 \item El ingreso familiar del encuestado es entre \$30\,000 y \$100\,000.
 \item El ingreso familiar del encuestado es al menos \$100\,001
\end{enumerate}
\item El U.S. Department of Transportation publica
anualmente el número de quejas de consumidores
contra las principales aerolíneas por categoría. A continuación aparecen las cifras para 2002
\begin{table}[!h]
 \begin{center}
\begin{tabular}{p{3.5cm}p{2cm}|lp{2cm}}
Categoría de queja & Número de quejas & Categoría de queja & Número de quejas\\ \hline
Problemas en vuelo & 2031 & Sobreventa & 454\\
Servicio a clientes & 1715 & Tarifas & 523\\
Equipaje & 1421 & Incapacidad & 477\\
Reservaciones/venta de boletos/abordar & 1159 & Publicidad & 68\\
Devoluciones & 1106 & Otras & 322 \\ \hline
 \end{tabular}
 \end{center}
\caption{Fuente: Office of Aviation Enforcement \& Proceedings, U.S. Department of
Transportation, Air Travel Consumer Report, \url{http://www.infoplease.com/ipa/
A0198353.html}}
\end{table} 
Si una de estas quejas se selecciona al azar para evaluación de seguimiento, ¿cuál es la probabilidad de que la queja sea:
\begin{enumerate}
 \item relacionada con problemas en vuelo?
 \item acerca del servicio a clientes o equipaje?
 \item relativa a las reservaciones/boletos/abordar o devoluciones o sobreventa?
 \item que no sea de equipaje?
\end{enumerate}
\item Un número de un solo dígito ha de seleccionarse al azar. Haga una lista del espacio muestral
\item Se lanza un solo dado. ¿Cuál es la probabilidad de que el número en su cara superior sea lo siguiente?
\begin{enumerate}
 \item Un 3
 \item Un número impar
 \item Un número menor a 5
 \item Un número no mayor de 3
\end{enumerate}
\item Se lanza un par de dados. Si $P(5)$ indica que la suma de los números superiores del dado sea 5, encontrar las probabilidades $P(6)$, $P(7)$, $P(8)$, $P(9)$, $P(10)$, $P(11)$ y $P(12)$
\item  Una caja contiene un billete de cada uno de lo
siguiente: \$1, \$5, \$10 y \$20.
\begin{enumerate}
 \item Se selecciona uno al azar; haga una lista del espacio muestral.
 \item Se sacan dos billetes al azar (sin reposición); haga una lista del espacio muestral como un diagrama de árbol.
\end{enumerate}
 \end{enumerate}

\end{document}
