\documentclass[letterpaper,10pt,twoside]{article}
\usepackage[utf8]{inputenc}
\usepackage{amsmath,amsfonts,amssymb,amsthm,latexsym}
\usepackage[spanish,es-noshorthands]{babel}
\usepackage[T1]{fontenc}
\usepackage{lmodern}
\usepackage{graphicx,hyperref}
\usepackage{tikz,pgf}
\usepackage{multicol}
\usepackage{fancyhdr}
\usepackage{marvosym}
\usepackage[height=9.5in,width=7in]{geometry}
\usepackage{fancyhdr}
\pagestyle{fancy}
\fancyhead[LE]{\Email matematicas.german@gmail.com}
\fancyhead[RE]{\url{https://www.autistici.org/mathgerman}}
\fancyhead[RO]{\Email matematicas.german@gmail.com}
\fancyhead[LO]{\url{https://www.autistici.org/mathgerman}}

\author{Germ\'an Avenda\~no Ram\'irez~\thanks{Lic. Mat. U.D., M.Sc. U.N.}}
\title{\begin{minipage}{.2\textwidth}
\includegraphics[height=1.75cm]{Images/logo-colegio.png}\end{minipage}
\begin{minipage}{.55\textwidth}
\begin{center}
Animaplano 02\\
Matemáticas $11^{\circ}$
\end{center}
\end{minipage}\hfill
\begin{minipage}{.2\textwidth}
\includegraphics[height=1.75cm]{Images/logo-sed.png} 
\end{minipage}}
\date{}
\thispagestyle{plain}
\begin{document}
\maketitle
Nombre: \hrulefill Curso: \underline{\hspace*{44pt}} Fecha: \underline{\hspace*{2.5cm}}
\begin{multicols}{2}
\section*{Cuestionario}
Resuelva haciendo los procedimientos al respaldo. Respuesta que no se pueda verificar con procedimiento, no será tenida en cuenta. Haga el animaplano en media hoja cuadriculada, para anexarla a ésta.
 \begin{enumerate}
 \item $\sqrt[3]{64}\cdot 4^{2}=$
 \item $4^{3}+1=$
 \item El triple del décimo número primo.
 \item El doceavo número primo.
 \item El décimo número primo.
 \item El octavo número primo.
 \item El triple del máximo común divisor de 18, 24 y 30.
 \item $\sqrt{225}+\sqrt{169}=$
 \item $|4!|+\sqrt{81}+\sqrt{16}=$\footnote{El valor absoluto de un número $x$ se escribe $|x|$ y corresponde al valor absoluto sin su signo, por ejemplo $|3|=3$, y, $|-3|=3$}
 \item Si $\sqrt{x+1}=6$, entonces $x=$?
 \item Reste $2^{2}$ al número $7^{2}$
 \item El coeficiente de la suma: $22xy-3xy+35xy=$ es:
 \item El complemento\footnote{Dos ángulos son complementarios si suma es 90$^{\circ}$} de $(4^{2})^{\circ}$ grados es:
 \item El quíntuplo de $5^{2}-10$ es:
 \item El coeficiente de $(9xy)^{2}+x^{2}y^{2}+(4xy)^{2}=$ al simplificar es:
 \item Reste $\sqrt{100}$ de $\sqrt{10\,000}$
 \item Si $\frac{n}{2}=6^{2}$, entonces $n$ es?
 \item Resuelva $10^{2}-(3^{0}+2^{0}+1^{0})=$
 \item El quíntuplo de raíz cuadrada de 225
 \item Halle: $9^{2}+9^{0}+9=$
 \item El coeficiente del segundo término en el desarrollo de $(9a+4b)^{2}=$\footnote{Recuerde que $(a+b)^{2}=a^{2}+2ab+b^{2}$}
 \item $25+32+40-25-32+33=$
 \item La edad de Juan es 7 veces la segunda potencia de 3
 \item En años, 10 lustros más un año
 \item Decimotercero número primo
 \item Valor absoluto de la raíz cuadrada de $32^{2}$
 \item Si $\frac{x}{3}=13$, entonces 2/3 de $x=$?
 \item 1/6 del ángulo recto
 \item El total de ojos de 5 cíclopes\footnote{El cíclope es un personaje de la mitología griega que tenía un solo ojo}
 \item El valor de $n$ en la siguiente secuencia $625\;(11)\;96\; ; \;225\;(n)\;144$ es:
 \item $3\cdot\sqrt{49}=$
 \item $4\sqrt{64}=$
 \item El doble del quinto número primo
 \item Reste $11x^{2}$ de $25x^{2}$ su coeficiente es?
 \item Resuelva $3^{2}+3^{3}$
 \item Las edades de Juan y Pedro suman 91 años. Si Pedro es un año mayor que Juan, la edad de Pedro es:
 \item Con base en el anterior problema, la edad de Juan es:
 \end{enumerate}
\end{multicols}


\end{document}
