\documentclass[10pt,twoside]{article}
\usepackage[utf8]{inputenc}
\usepackage{amsmath,amsfonts,amssymb,amsthm,latexsym}
\usepackage[spanish,es-noshorthands]{babel}
\usepackage[T1]{fontenc}
\usepackage{lmodern}
\usepackage{graphicx,hyperref}
\usepackage{tikz,pgf}
\usepackage{multicol}
\usepackage{fancyhdr}
\usepackage{marvosym}
\usepackage[papersize={6.5in,8.5in},width=5.5in,height=7in]{geometry}
\usepackage{fancyhdr}
\pagestyle{fancy}
\fancyhead[LE]{\Email gavendanor@colarborizadorabaja.edu.co}
\fancyhead[RE]{\url{https://www.autistici.org/mathgerman}}
\fancyhead[RO]{\Email matematicas.german@gmail.com}
\fancyhead[LO]{\url{https://www.autistici.org/mathgerman}}

\author{Germ\'an Avenda\~no Ram\'irez~\thanks{Lic. Mat. U.D., M.Sc. U.N.}}
\title{\begin{minipage}{.2\textwidth}
\includegraphics[height=1.75cm]{Images/logo-colegio.png}\end{minipage}
\begin{minipage}{.55\textwidth}
\begin{center}
Animaplano 02\\
Matemáticas $9^{\circ}$
\end{center}
\end{minipage}\hfill
\begin{minipage}{.2\textwidth}
\includegraphics[height=1.75cm]{Images/logo-sed.png} 
\end{minipage}}
\date{}
\thispagestyle{plain}
\begin{document}
\maketitle
Nombre: \hrulefill Curso: \underline{\hspace*{44pt}} Fecha: \underline{\hspace*{2.5cm}}
\section*{Cuestionario}
Resuelva haciendo los procedimientos en una hora anexa donde debe hacer el animaplano. Recuerde hacer 100 puntos numerados del 1 al 100.

\paragraph*{Pirámide numérica}: Resuelva y grafique los ejercicios del 1 al 7, de acuerdo al índice que aparece en algunos rectángulos

\begin{center}
\begin{tikzpicture}[scale=.75]
\draw (-1,0) rectangle node {287} (1,1);
\draw (-2,-1) rectangle node {158} (0,0);
\draw ((0,-1) rectangle node {129} (2,0);
\draw (-3,-2) rectangle node {94}(-1,-1);
\draw (-1,-2) rectangle node {\hspace{38pt}$^1$} (1,-1);
\draw (1,-2) rectangle node {\hspace{38pt}$^2$} (3,-1);
\draw (-4,-3) rectangle node {\hspace{38pt}$^3$} (-2,-2);
\draw (-2,-3) rectangle node {\hspace{38pt}$^4$} (0,-2);
\draw (0,-3) rectangle node {27} (2,-2);
\draw (2,-3) rectangle node {38} (4,-2);
\draw (-5,-4) rectangle node {38} (-3,-3);
\draw (-3,-4) rectangle node {\hspace{38pt}$^6$} (-1,-3);
\draw (-1,-4) rectangle node {\hspace{38pt}$^7$} (1,-3);
\draw (1,-4) rectangle node {9} (3,-3);
\draw (3,-4) rectangle node {\hspace{38pt}$^5$} (5,-3);
\end{tikzpicture}
\end{center}
 \begin{enumerate}
 \begin{multicols}{7}
 \item \underline{\hspace*{24pt}}
 \item \underline{\hspace*{24pt}}
 \item \underline{\hspace*{24pt}}
 \item \underline{\hspace*{24pt}}
 \item \underline{\hspace*{24pt}}
 \item \underline{\hspace*{24pt}}
 \item  \underline{\hspace*{24pt}}
 \end{multicols}
 
\begin{multicols}{2}
 \item $\sqrt{225}+\sqrt{169}=$
 \item $4!+\sqrt{81}+\sqrt{16}=$
 \item Si $\sqrt{x+1}=6$, entonces $x=$?
 \item Reste $2^{2}$ al número $7^{2}$
\end{multicols}
 \item El coeficiente de la suma: $22xy-3xy+35xy=$ es:
 \item El complemento\footnote{Dos ángulos son complementarios si suma es 90$^{\circ}$} de $(4^{2})^{\circ}$ grados es:
 \item El quíntuplo de $5^{2}-10$ es:
 \item El coeficiente de $(9xy)^{2}+x^{2}y^{2}+(4xy)^{2}=$ al simplificar es:
\begin{multicols}{2}
 \item Reste $\sqrt{100}$ de $\sqrt{10\,000}$
 \item Si $\frac{n}{2}=6^{2}$, entonces $n$ es?
\end{multicols}
\begin{multicols}{2}
 \item Resuelva $10^{2}-(3^{0}+2^{0}+1^{0})=$
 \item El quíntuplo de raíz cuadrada de 225
\end{multicols}
 \item Halle: $9^{2}+9^{0}+9=$
 \item El coeficiente del segundo término en el desarrollo de $(9a+4b)^{2}=$\footnote{Recuerde que $(a+b)^{2}=a^{2}+2ab+b^{2}$}
 \item $25+32+40-25-32+33=$
 \item La edad de Juan es 7 veces la segunda potencia de 3
 \item En años, 10 lustros más un año
 \item El quinto término de la secuencia 5, 11, 19, 29, es?
 \item Valor absoluto\footnote{El valor absoluto de un número $x$ se escribe $|x|$ y corresponde al valor absoluto sin su signo, por ejemplo $|3|=3$, y, $|-3|=3$} de la raíz cuadrada de $23^{2}$
\begin{multicols}{2}
 \item Si $\frac{x}{3}=13$, entonces 2/3 de $x=$?
 \item 1/6 del ángulo recto
\end{multicols}
 \item El total de ojos de 5 cíclopes\footnote{El cíclope es un personaje de la mitología griega que tenía un solo ojo}
 \item El valor de $n$ en la siguiente secuencia $625\;(11)\;196\; ; \;225\;(n)\;144$ es:
\begin{multicols}{2}
 \item $3\cdot\sqrt{49}=$
 \item $4\sqrt{64}=$
\end{multicols}
\begin{multicols}{2}
 \item El doble del quinto número primo
 \item Reste $11x^{2}$ de $25x^{2}$ su coeficiente es?
\end{multicols}
 \item Resuelva $3^{2}+3^{3}$
 \item Las edades de Juan y Pedro suman 91 años. Si Pedro es un año mayor que Juan, la edad de Pedro es:
 \item Con base en el anterior problema, la edad de Juan es:
 \end{enumerate}
\begin{center}
\begin{tikzpicture}[scale=.75]
 \fill (1,0) node[above]{1} circle (0.2ex);
 \fill (2,0) node[above]{2} circle (0.2ex);
 \fill (3,0) node[above]{3} circle (0.2ex);
 \fill (4,0) node[above]{4} circle (0.2ex);
 \fill (5,0) node[above]{5} circle (0.2ex);
 \fill (6,0) node[above]{6} circle (0.2ex);
 \fill (7,0) node[above]{7} circle (0.2ex);
 \fill (8,0) node[above]{8} circle (0.2ex);
 \fill (9,0) node[above]{9} circle (0.2ex);
 \fill (10,0) node[above]{10} circle (0.2ex);
 \fill (1,-1) node[left]{11} circle (0.2ex);
 \fill (2,-1) circle (0.2ex);
 \fill (3,-1) circle (0.2ex);
 \fill (4,-1) circle (0.2ex);
 \fill (5,-1) circle (0.2ex);
 \fill (6,-1) circle (0.2ex);
 \fill (7,-1) circle (0.2ex);
 \fill (8,-1) circle (0.2ex);
 \fill (9,-1) circle (0.2ex);
 \fill (10,-1) circle (0.2ex);
 \fill (1,-2) node[left]{21} circle (0.2ex);
 \fill (2,-2) circle (0.2ex);
 \fill (3,-2) circle (0.2ex);
 \fill (4,-2) circle (0.2ex);
 \fill (5,-2) circle (0.2ex);
 \fill (6,-2) circle (0.2ex);
 \fill (7,-2) circle (0.2ex);
 \fill (8,-2) circle (0.2ex);
 \fill (9,-2) circle (0.2ex);
 \fill (10,-2) circle (0.2ex);
 \fill (1,-3) node[left]{31} circle (0.2ex);
 \fill (2,-3) circle (0.2ex);
 \fill (3,-3) circle (0.2ex);
 \fill (4,-3) circle (0.2ex);
 \fill (5,-3) circle (0.2ex);
 \fill (6,-3) circle (0.2ex);
 \fill (7,-3) circle (0.2ex);
 \fill (8,-3) circle (0.2ex);
 \fill (9,-3) circle (0.2ex);
 \fill (10,-3) circle (0.2ex);
 \fill (1,-4) node[left]{41} circle (0.2ex);
 \fill (2,-4) circle (0.2ex);
 \fill (3,-4) circle (0.2ex);
 \fill (4,-4) circle (0.2ex);
 \fill (5,-4) circle (0.2ex);
 \fill (6,-4) circle (0.2ex);
 \fill (7,-4) circle (0.2ex);
 \fill (8,-4) circle (0.2ex);
 \fill (9,-4) circle (0.2ex);
 \fill (10,-4) node[right]{50} circle (0.2ex);
 \fill (1,-5) node[left]{51} circle (0.2ex);
 \fill (2,-5) circle (0.2ex);
 \fill (3,-5) circle (0.2ex);
 \fill (4,-5) circle (0.2ex);
 \fill (5,-5) circle (0.2ex);
 \fill (6,-5) circle (0.2ex);
 \fill (7,-5) circle (0.2ex);
 \fill (8,-5) circle (0.2ex);
 \fill (9,-5) circle (0.2ex);
 \fill (10,-5) circle (0.2ex);
 \fill (1,-6) node[left]{61} circle (0.2ex);
 \fill (2,-6) circle (0.2ex);
 \fill (3,-6) circle (0.2ex);
 \fill (4,-6) circle (0.2ex);
 \fill (5,-6) circle (0.2ex);
 \fill (6,-6) circle (0.2ex);
 \fill (7,-6) circle (0.2ex);
 \fill (8,-6) circle (0.2ex);
 \fill (9,-6) circle (0.2ex);
 \fill (10,-6) circle (0.2ex);
 \fill (1,-7) node[left]{71} circle (0.2ex);
 \fill (2,-7) circle (0.2ex);
 \fill (3,-7) circle (0.2ex);
 \fill (4,-7) circle (0.2ex);
 \fill (5,-7) circle (0.2ex);
 \fill (6,-7) circle (0.2ex);
 \fill (7,-7) circle (0.2ex);
 \fill (8,-7) circle (0.2ex);
 \fill (9,-7) circle (0.2ex);
 \fill (10,-7) circle (0.2ex);
 \fill (1,-8) node[left]{81} circle (0.2ex);
 \fill (2,-8) circle (0.2ex);
 \fill (3,-8) circle (0.2ex);
 \fill (4,-8) circle (0.2ex);
 \fill (5,-8) circle (0.2ex);
 \fill (6,-8) circle (0.2ex);
 \fill (7,-8) circle (0.2ex);
 \fill (8,-8) circle (0.2ex);
 \fill (9,-8) circle (0.2ex);
 \fill (10,-8) circle (0.2ex);
 \fill (1,-9) node[left]{91} circle (0.2ex);
 \fill (2,-9) circle (0.2ex);
 \fill (3,-9) circle (0.2ex);
 \fill (4,-9) circle (0.2ex);
 \fill (5,-9) circle (0.2ex);
 \fill (6,-9) circle (0.2ex);
 \fill (7,-9) circle (0.2ex);
 \fill (8,-9) circle (0.2ex);
 \fill (9,-9) circle (0.2ex);
 \fill (10,-9) node[right]{100} circle (0.2ex);
 \draw (4,-6)--(5,-6)--(7,-5)--(7,-3)--(9,-2)--(9,-1)--(8,-1)--(8,-2)--(7,-3)--(5,-3)--(5,-4)--(4,-5)--(4,-7)--(5,-7)--(8,-9)--(10,-8)--(2,-7)--(7,-9)--(5,-7)--(1,-9)--(2,-7)--(3,-7)--(3,-6)--(1,-5)--(1,-4)--(3,-2)--(6,-2)--(5,-1)--(5,0)--(3,-0)--(1,-2)--(2,-3)--(2,-2)--(4,-1)--(6,-3)--(6,-4)--(5,-4);
\end{tikzpicture}
\end{center}

\end{document}
