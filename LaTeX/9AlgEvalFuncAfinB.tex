\documentclass[fleqn]{article}
\usepackage[spanish,es-noshorthands]{babel}
\usepackage[utf8]{inputenc} 
\usepackage[papersize={6.5in,8.5in},left=1cm, right=1cm, top=1.5cm, bottom=1.7cm]{geometry}
\usepackage{mathexam}
\usepackage{amsmath}
\usepackage{graphicx}
\usepackage{tikz,pgf}

\ExamClass{\includegraphics[height=16pt]{Images/logo-sed.png} Álgebra $9^{\circ}$}
\ExamName{"Función afín y lineal"}
\ExamHead{\includegraphics[height=16pt]{Images/logo-colegio.png} IEDAB}
\newcommand{\LineaNombre}{%
\par
\vspace{\baselineskip}
Nombre:\hrulefill \; Curso: \underline{\hspace*{48pt}} \; Fecha: \underline{\hspace*{2.5cm}} \relax
\par}
\let\ds\displaystyle

\begin{document}
\ExamInstrBox{
Respuesta sin justificar mediante procedimiento no será tenida en cuenta en la calificación. Escriba sus respuestas en el espacio indicado. Tiene 55 minutos para contestar esta prueba.}
\LineaNombre
\begin{enumerate}
\item Un operador de celular cobra a 150 pesos el minuto. Si se designa $t$, el tiempo en minutos y $f(t)$ el precio que se paga por $t$ minutos, 
\begin{enumerate}
\item complete la siguiente tabla: \hspace*{20pt}
\begin{tabular}{|c|c|c|c|c|c|c|c|c|c|c|}
\hline 
$t$ & 0 & 1 & 2 & 3 & 4 & 5 & 6 & 7 & 8 & 9 \\ 
\hline 
$f(t)$ &  &  &  &  &  &  &  &  &  &  \\ 
\hline 
\end{tabular}

\begin{minipage}{.3\textwidth}
\item Escoja una escala conveniente para hacer la gráfica del tiempo $t$ en minutos, contra el precio ($f(t)$) en \$
\end{minipage}
\begin{minipage}{.65\textwidth}
\definecolor{cqcqcq}{rgb}{0.75,0.75,0.75}
\begin{center}
\begin{tikzpicture}[scale=.7]
\draw[color=cqcqcq,dash pattern=on 2pt off 2pt, xstep=1.0cm,ystep=1.0cm](-0.3,-.3)grid(9.3,9.3);
\draw[<->] (-.3,0) -- (9.3,0)node[right]{$t$ (m)};
\draw[<->] (0,-.3) -- (0,9.3)node[right]{$f(t)$ (\$)};
\foreach \x in {1} \draw[shift={(\x,0)},color=black] (0pt,2pt) -- (0pt,-2pt) node[below] {$\x$};
\end{tikzpicture}
\end{center}
\end{minipage}
\item ¿Cuánto deberá pagar un cliente por una llamada de 30 minutos?\noanswer
\item Si se pagara por segundos en vez de pagar por minutos, ¿cuánto deberá pagar un cliente que hable 3 minutos y medio?\noanswer
\end{enumerate}
\newpage
\begin{minipage}{.35\textwidth}
\item De acuerdo al siguiente gráfico, que muestra la distancia recorrida en kilómetros por un ciclista que viaja con una rapidez constante, encuentre la distancia recorrida ($f(t)$) por el ciclista, al cabo de:
\end{minipage}
\begin{minipage}{.6\textwidth}
\definecolor{cqcqcq}{rgb}{0.75,0.75,0.75}
\begin{center}
\begin{tikzpicture}[scale=.7]
\draw[color=cqcqcq,dash pattern=on 2pt off 2pt, xstep=1.0cm,ystep=1.0cm](-0.3,-.3)grid(9.3,9.3);
\draw[<->] (-.3,0) -- (9.3,0)node[right]{$t$ (horas)};
\draw[<->] (0,-.3) -- (0,9.3)node[right]{$f(t)$ (km)};
\foreach \x in {1} \draw[shift={(\x,0)},color=black] (0pt,2pt) -- (0pt,-2pt) node[below] {$\x$};
\draw[thick] (-0.1,2)node[left]{60}--(.1,2);
\draw plot[domain=0:9]  (\x,\x);
\end{tikzpicture}
\end{center}
\end{minipage}
\begin{enumerate}
\item 4 horas\noanswer
\item 6 horas\noanswer
\item 3 horas y media\noanswer
\end{enumerate}
Ahora encuentra la rapidez media del ciclista (pendiente de la recta) en kilómetros por hora (km/h)\noanswer
\end{enumerate}
\end{document}
