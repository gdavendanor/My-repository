\documentclass[fleqn]{article}
\usepackage[spanish,es-noshorthands]{babel}
\usepackage[utf8]{inputenc} 
\usepackage[papersize={6.5in,8.5in},left=1cm, right=1cm, top=1.5cm, bottom=1.7cm]{geometry}
\usepackage{mathexam}
\usepackage{amsmath}
\usepackage{graphicx}
\usepackage{multicol}

\ExamClass{\includegraphics[height=16pt]{Images/logo-sed.png} Matemáticas $11^{\circ}$}
\ExamName{``Sustentación P. Mejoramiento 3''}
\ExamHead{\includegraphics[height=16pt]{Images/logo-colegio.png} IEDAB}
\newcommand{\LineaNombre}{%
\par
\vspace{\baselineskip}
Nombre:\hrulefill \; Curso: \underline{\hspace*{48pt}} \; Fecha: \underline{\hspace*{2.5cm}} \relax
\par}
\let\ds\displaystyle

\begin{document}
\ExamInstrBox{
Respuesta sin justificar mediante procedimiento no será tenida en cuenta en la calificación. Escriba sus respuestas en el espacio indicado. Tiene 45 minutos para contestar esta prueba.}
\LineaNombre
\section*{Cálculo}
\begin{enumerate}
 \item Determine los 5 primeros términos de la sucesión cuyo término n-ésimo es $a_{n}=n^{2}+2$
 \begin{enumerate}
 \begin{multicols}{2}
 \item $a_{1}=$
 \item $a_{2}=$
 \item $a_{3}=$
 \item $a_{4}=$
 \item $a_{5}=$ 
 \end{multicols}
 \end{enumerate}
 \item Una sucesión aritmética inicia con --2, 1, 4, 7, \ldots
 \begin{enumerate}
 \item Encuentre la diferencia común $d$ para esta sucesión.\noanswer
\item Determine una fórmula para el n-ésimo término $a_{n}$ de la sucesión.\noanswer
\item Halle el décimoquinto (15$^{o}$) término de la sucesión.\noanswer
 \end{enumerate}
 \item Una sucesión geométrica inicia con 12, 4, 4/3, 4/9, \ldots
 \begin{enumerate}
\item Determine la razón común $r$ de esta sucesión \noanswer
 \newpage
\item Encuentre una fórmula para el n-ésimo término $a_{n}$ de la sucesión.\noanswer
\item Calcule el décimo término de la sucesión\noanswer
\end{enumerate}
\item Un cachorro pesa 0.85 lb al nacer, y cada semana gana 24\% de peso. Sea $a_{n}$ su peso en libras al final de la n-ésima semana de vida.
\begin{enumerate}
\item Encuentre una fórmula para $a_{n}$\noanswer
\item ¿Cuánto pesa el cachorro cuando tiene seis semanas de vida?\noanswer
\item ¿Es la sucesión $a_{1},a_{2},a_{3},\ldots$ aritmética, o geométrica o de ninguno de los dos tipos? \noanswer
\end{enumerate}
\section*{Probabilidad}
\item ¿Cuántos números de tres dígitos pueden formarse con tres 4, cuatro 2 y dos 3?\noanswer
 \end{enumerate}

\end{document}
