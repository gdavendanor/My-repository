\documentclass[letterpaper,twoside]{article}
\usepackage[utf8]{inputenc}
\usepackage{amsmath,amsfonts,amssymb,amsthm,latexsym}
\usepackage[spanish,es-noshorthands]{babel}
\usepackage[T1]{fontenc}
\usepackage{lmodern}
\usepackage{graphicx,hyperref}
\usepackage{tikz,pgf}
\usepackage{multicol}
\usepackage{subfig}
\usepackage{marvosym}
\usepackage{fancyhdr}
\usepackage[includeheadfoot,left=0.4in,right=0.3in,top=0.3in,bottom=0.3in]{geometry}
\pagestyle{fancy}
\fancyhead[LE]{\url{www.autistici.org/mathgerman}}
\fancyhead[RE]{}
\fancyhead[RO]{\Email~iedabgerman@autistici.org}
\fancyhead[LO]{}
\date{}
\author{Germán Avendaño Ramírez~\thanks{Lic. Mat. U.D.; M.Sc. U.N.}}
\title{\begin{minipage}{.2\textwidth}
\includegraphics[height=1.75cm]{Images/logo-colegio.png}
\end{minipage}
\begin{minipage}{.55\textwidth}
\begin{center}
Recomendaciones III período \\
Matemáticas $11^{\circ}$
\end{center}
\end{minipage}\hfill
\begin{minipage}{.2\textwidth}
\includegraphics[height=1.75cm]{Images/logo-sed.png} 
\end{minipage}}
\begin{document}
\maketitle
Desarrolle los siguientes ejercicios como preparación para la sustentación de las temáticas vistas en el 3er período académico
\section{Cálculo}
\begin{enumerate}
\item Determine los primeros cuatros términos y el décimo término de la sucesión cuyo n-ésimo término es $a_{n}=n^{2}-1$
\item Una sucesión está definida recursivamente por $a_{n+2}=a_{n}^{2}-a_{n+1}$, donde $a_{1}=1$ y $a_{2}=1$. Calcule $a_{5}$.
\item Una sucesión aritmética inicia con 2, 5, 8, 11, 14, \ldots
\begin{enumerate}
\item Encuentre la diferencia común $d$ para esta sucesión.
\item Determine una fórmula para el n-ésimo término $a_{n}$ de la sucesión.
\item Halle el trigésimoquinto término de la sucesión.
\end{enumerate}
\item Una sucesión geométrica inicia con 12, 3, 3/4, 3/16, 3/64, \ldots
\begin{enumerate}
\item Determine la razón común $r$ de esta sucesión
\item Encuentre una fórmula para el n-ésimo término $a_{n}$ de la sucesión.
\item Calcule el décimo término de la sucesión
\end{enumerate}
\item El primer término de una sucesión geométrica es 25, y el cuarto término es $\frac{1}{5}$
\begin{enumerate}
\item Determine la razón común $r$ y el quinto término.
\item Calcule la suma parcial de los primeros ocho términos.
\end{enumerate}
\item El primer término de una sucesión aritmética es 10 y el décimo término es 2.
\begin{enumerate}
\item Encuentre la diferencia común y el centésimo término de la sucesión
\item Calcule la suma parcial de los primeros diez términos.
\end{enumerate}
\item Sea $a_{1}, a_{2}, a_{3}, \ldots$ una sucesión geométrica con término inicial $a$ y razón común $r$. Demuestre que $a_{1}^{2},a_{2}^{2},a_{3}^{2}, \ldots$ es también una sucesión geométrica encontrando su razón común.
\item Escriba la expresión sin usar la notación sigma y, luego, calcule la suma.
\begin{enumerate}
\begin{multicols}{2}
\item $\displaystyle\sum_{n=1}^{5}\;(1-n^{2})$
\item $\displaystyle \sum_{n=1}^{6}\;(-1)^{n}2^{n-2}$
\end{multicols}
\end{enumerate}
\item Calcule la suma
\begin{enumerate}
\begin{multicols}{2}
\item $\dfrac{1}{3}+\dfrac{2}{3^{2}}+\dfrac{2^{2}}{3^{3}}+\dfrac{2^{3}}{3^{4}}+\ldots+\dfrac{2^{9}}{3^{10}}$
\item $1+\dfrac{1}{2^{1/2}}+\dfrac{1}{2}+\dfrac{1}{2^{3/2}}+\ldots$
\end{multicols}
\end{enumerate}
\item Mediante inducción matemáticas demuestre que, para todos los números naturales $n$,
\[1^{2}+2^{2}+3^{2}+\ldots+n^{2}=\dfrac{n(n+1)(2n+1)}{6}\]
\item Desarrolle $(2x+y^{2})^{5}$
\item Encuentre el término que contiene $x^{3}$ en el desarrollo del binomio $(3x-2)^{10}$.
\item Un cachorro pesa 0.85 lb al nacer, y cada semana gana 24\% de peso. Sea $a_{n}$ su peso en libras al final de la n-ésima semana de vida.
\begin{enumerate}
\item Encuentre una fórmula para $a_{n}$
\item ¿Cuánto pesa el cachorro cuando tiene seis semanas de vida?
\item ¿Es la sucesión $a_{1},a_{2},a_{3},\ldots$ aritmética, o geométrica o de ninguno de los dos tipos?
\end{enumerate}
\section{Probabilidad}
\item ¿Cuántos números de tres dígitos pueden formarse con tres 4, cuatro 2 y dos 3?
\item ¿Cuántos números de cinco dígitos pueden formarse con los dígitos 1, 2, 3, \ldots, 9
\item ¿De cuántas maneras pueden sentarse a una mesa redonda 3 hombres y 3 mujeres si: 
\begin{enumerate}
\item sin ninguna restricción
\item hay dos mujeres que no pueden sentarse juntas
\item cada mujer debe estar entre dos hombres?
\end{enumerate} 
\item ¿De cuántas maneras pueden seleccionarse 2 hombres, 4 mujeres, 3 niños y 3 niñas de un grupo de 6 hombres, 8 mujeres, 4 niños y 5 niñas si: a) no hay ninguna restricción y b) hay un hombre y una mujer que tienen que seleccionarse?
\item ¿De cuántas maneras puede dividirse un grupo de 10 personas en: a) dos grupos de 7 y 3 personas y b) tres grupos de 4, 3, y 2 personas?
\item Encontrar la cantidad de: a) combinaciones y b) permutaciones de cuatro letras que pueden formarse con las letras de la palabra Tennessee.
\end{enumerate}
\end{document}
