\documentclass[10pt,twoside]{article}
\usepackage[utf8]{inputenc}
\usepackage{amsmath}
\usepackage{amsfonts}
\usepackage{amssymb}
\usepackage[spanish,es-noshorthands]{babel}
\usepackage[T1]{fontenc}
\usepackage{lmodern}
\usepackage{marvosym}
\usepackage{graphicx,hyperref}
\usepackage{tikz,pgf}
\usepackage{multicol}
\usepackage{subfig}
\usepackage[papersize={6.5in,8.5in},total={5.5in,7.25in},centering]{geometry}
\usepackage{fancyhdr}
\pagestyle{fancy}
\fancyhead[LE]{\Email iedabgerman@autistici.org}
\fancyhead[RE]{\url{www.autistici.org/mathgerman}}
\fancyhead[RO]{\Email iedabgerman@autistici.org}
\fancyhead[LO]{\url{www.autistici.org/mathgerman}}

\author{Germ\'an Avenda\~no Ram\'irez~\thanks{Lic. Mat. U.D., M.Sc. U.N.}}
\title{\begin{minipage}{.2\textwidth}
\includegraphics[height=1.75cm]{Images/logo-colegio.png}\end{minipage}
\begin{minipage}{.55\textwidth}
\begin{center}
Animaplano 02\\
Matemáticas $11^{\circ}$
\end{center}
\end{minipage}\hfill
\begin{minipage}{.2\textwidth}
\includegraphics[height=1.75cm]{Images/logo-sed.png} 
\end{minipage}}
\date{}
\thispagestyle{plain}
\begin{document}
\maketitle
Nombre: \hrulefill Curso: \underline{\hspace*{44pt}} Fecha: \underline{\hspace*{2.5cm}}

\vspace*{10pt}
Haga el animaplano en media hojita cuadriculada con 100 puntos numerándolos del 1-100 y anéxela a ésta hoja. Al final debe entregar el animaplano resuelto a los/as  estudiantes encargados/as, para que estos/as a su vez se lo entregue a la coordinadora Gloria Daza.
\begin{itemize}
 \item[1102] Encargados: Omar Duque y Santiago Cabezas
 \item[1101] Encargados: Angie Velandia y Brayan Infante 
\end{itemize}
 Responda las preguntas 1 a 3 teniendo en cuenta la siguiente información.

 Un escalador quiere subir una montaña de 141 m, en el primer intento sube 74 m y resbala 16 m. En el segundo intento alcanza la cima.
 \begin{enumerate}
 \item ¿A qué altura quedó en el primer intento luego de haber resbalado?
 \item ¿Qué distancia subió en el segundo intento midiendo desde el punto donde se resbaló?
 \item Señale la operación correcta para determinar la distancia $d$ que subió en el segundo intento
 \begin{center}
\begin{tabular}{|c|}
\hline 
$d=141+73+15$ \\ 
\hline 
(43) \\ 
\hline 
\end{tabular} 
\begin{tabular}{|c|}
\hline 
$d=141-73+15$ \\ 
\hline 
(37) \\ 
\hline 
\end{tabular} 
\begin{tabular}{|c|}
\hline 
$d=141-73-15$ \\ 
\hline 
(29) \\ 
\hline 
\end{tabular} 
 \end{center}
 En las siguientes operaciones cada letra representa un dígito entre 1 y 9. Halle sus valores numéricos y empleálos para desarrollar los ejercicios planteados.

$ \begin{array}{ccccc}
  & I & S & 4 & A \\ 
+ & A & S & I & 4 \\ \hline 
1 & 1 & 1 & 1 & I
 \end{array} $ \hspace*{.5cm}
  $\begin{array}{ccccc}
 & E & E & 0 & 0 \\ 
- & 1 & U & V & U \\ \hline
 &  & 3 & 5 & E
 \end{array} $ \hspace*{.4cm}
 $\begin{array}{cccc}
  &  & M & O \\ 
 & \times & 1 & O \\ \hline
1 & 3 & 1 & 1
 \end{array} $ \hspace*{.5cm}
 $\begin{array}{ccc|cc}
 T & 2 & T & T & T \\ \cline{4-5}
  &  & 0 & T & T
 \end{array} $
\begin{center}
 \begin{tabular}{|c|c|c|c|c|c|c|c|c|}
 \hline 
 I & S & A & M & E & T & U & V & O \\ 
 \hline 
 &  &  &  &  &  &  &  &  \\ 
 \hline 
 \end{tabular} 
\end{center}
 \begin{multicols}{2}
 \item $V^{E}+S\cdot M=$
 \item $E^{I}-U\cdot O=$
 \item $A^{A+T}-A=$
 \item $A[(S\cdot M)-T]=$
 \item $(S^{E}\cdot E)+M^{E}=$
 \item $M(V+I)=$
 \item $(M+T)\cdot (O+E)=$
 \item $(E^{A}\cdot A^{E})+I=$
 \item $O+(V\cdot S)=$
 \item $T+E+V+S+M=$
 \end{multicols}
Responda los numerales 14 al 25 teniendo en cuenta los siguientes conjuntos.

$\begin{array}{ccc}
A=\{13,17,23,62\} & B=\{6,17,73,96\} & C=\{13,24,83,96\}\\
D=\{6,16,23,42\} & E=\{15,24,42,62\} & F=\{15,16,73,83\}
\end{array} $
\begin{multicols}{3}
\item $A \cap B=\{\quad\}$
\item $B\cap D=\{\quad\}$
\item $D\cap F=\{\quad\}$
\item $E\cap F=\{\quad\}$
\item $(E\cup D)\cup C=\{\quad\}$
\item $A\cap (C\cup F)=\{\quad\}$
\item $(A-C)\cap D=\{\quad\}$
\item $E\cap (D-B)=\{\quad\}$
\item $A\cap (E\cup F)=\{\quad\}$
\item $(B\cup A)\cap F=\{\quad\}$
\item $C\cap F=\{\quad\}$
\item $B\cap F=\{\quad\}$
\end{multicols}
Relacione los valores de las siguientes básculas y determine el valor de las figuras indicadas.
\begin{center}
% Graphic for TeX using PGF
% Title: /home/german/Diagrama1.dia
% Creator: Dia v0.97.3
% CreationDate: Wed Jul 15 17:14:04 2015
% For: german
% \usepackage{tikz}
% The following commands are not supported in PSTricks at present
% We define them conditionally, so when they are implemented,
% this pgf file will use them.
\ifx\du\undefined
  \newlength{\du}
\fi
\setlength{\du}{15\unitlength}
\begin{tikzpicture}[scale=.85]
\pgftransformxscale{1.000000}
\pgftransformyscale{-1.000000}
\definecolor{dialinecolor}{rgb}{0.000000, 0.000000, 0.000000}
\pgfsetstrokecolor{dialinecolor}
\definecolor{dialinecolor}{rgb}{1.000000, 1.000000, 1.000000}
\pgfsetfillcolor{dialinecolor}
\definecolor{dialinecolor}{rgb}{1.000000, 1.000000, 1.000000}
\pgfsetfillcolor{dialinecolor}
\fill (1.948750\du,3.050000\du)--(1.948750\du,4.950000\du)--(5.301250\du,4.950000\du)--(5.301250\du,3.050000\du)--cycle;
\pgfsetlinewidth{0.100000\du}
\pgfsetdash{}{0pt}
\pgfsetdash{}{0pt}
\pgfsetmiterjoin
\definecolor{dialinecolor}{rgb}{0.000000, 0.000000, 0.000000}
\pgfsetstrokecolor{dialinecolor}
\draw (1.948750\du,3.050000\du)--(1.948750\du,4.950000\du)--(5.301250\du,4.950000\du)--(5.301250\du,3.050000\du)--cycle;
% setfont left to latex
\definecolor{dialinecolor}{rgb}{0.000000, 0.000000, 0.000000}
\pgfsetstrokecolor{dialinecolor}
\node at (3.625000\du,4.195000\du){178 Kg};
\definecolor{dialinecolor}{rgb}{1.000000, 1.000000, 1.000000}
\pgfsetfillcolor{dialinecolor}
\fill (6.898750\du,2.990000\du)--(6.898750\du,4.890000\du)--(10.251250\du,4.890000\du)--(10.251250\du,2.990000\du)--cycle;
\pgfsetlinewidth{0.100000\du}
\pgfsetdash{}{0pt}
\pgfsetdash{}{0pt}
\pgfsetmiterjoin
\definecolor{dialinecolor}{rgb}{0.000000, 0.000000, 0.000000}
\pgfsetstrokecolor{dialinecolor}
\draw (6.898750\du,2.990000\du)--(6.898750\du,4.890000\du)--(10.251250\du,4.890000\du)--(10.251250\du,2.990000\du)--cycle;
% setfont left to latex
\definecolor{dialinecolor}{rgb}{0.000000, 0.000000, 0.000000}
\pgfsetstrokecolor{dialinecolor}
\node at (8.575000\du,4.135000\du){248 Kg};
\definecolor{dialinecolor}{rgb}{1.000000, 1.000000, 1.000000}
\pgfsetfillcolor{dialinecolor}
\fill (11.938750\du,3.030000\du)--(11.938750\du,4.930000\du)--(15.291250\du,4.930000\du)--(15.291250\du,3.030000\du)--cycle;
\pgfsetlinewidth{0.100000\du}
\pgfsetdash{}{0pt}
\pgfsetdash{}{0pt}
\pgfsetmiterjoin
\definecolor{dialinecolor}{rgb}{0.000000, 0.000000, 0.000000}
\pgfsetstrokecolor{dialinecolor}
\draw (11.938750\du,3.030000\du)--(11.938750\du,4.930000\du)--(15.291250\du,4.930000\du)--(15.291250\du,3.030000\du)--cycle;
% setfont left to latex
\definecolor{dialinecolor}{rgb}{0.000000, 0.000000, 0.000000}
\pgfsetstrokecolor{dialinecolor}
\node at (13.615000\du,4.175000\du){129 Kg};
\definecolor{dialinecolor}{rgb}{1.000000, 1.000000, 1.000000}
\pgfsetfillcolor{dialinecolor}
\fill (16.640000\du,3.040000\du)--(16.640000\du,4.940000\du)--(19.690000\du,4.940000\du)--(19.690000\du,3.040000\du)--cycle;
\pgfsetlinewidth{0.100000\du}
\pgfsetdash{}{0pt}
\pgfsetdash{}{0pt}
\pgfsetmiterjoin
\definecolor{dialinecolor}{rgb}{0.000000, 0.000000, 0.000000}
\pgfsetstrokecolor{dialinecolor}
\draw (16.640000\du,3.040000\du)--(16.640000\du,4.940000\du)--(19.690000\du,4.940000\du)--(19.690000\du,3.040000\du)--cycle;
% setfont left to latex
\definecolor{dialinecolor}{rgb}{0.000000, 0.000000, 0.000000}
\pgfsetstrokecolor{dialinecolor}
\node at (18.165000\du,4.185000\du){97 Kg};
\definecolor{dialinecolor}{rgb}{1.000000, 1.000000, 1.000000}
\pgfsetfillcolor{dialinecolor}
\fill (1.888750\du,8.040000\du)--(1.888750\du,9.940000\du)--(5.241250\du,9.940000\du)--(5.241250\du,8.040000\du)--cycle;
\pgfsetlinewidth{0.100000\du}
\pgfsetdash{}{0pt}
\pgfsetdash{}{0pt}
\pgfsetmiterjoin
\definecolor{dialinecolor}{rgb}{0.000000, 0.000000, 0.000000}
\pgfsetstrokecolor{dialinecolor}
\draw (1.888750\du,8.040000\du)--(1.888750\du,9.940000\du)--(5.241250\du,9.940000\du)--(5.241250\du,8.040000\du)--cycle;
% setfont left to latex
\definecolor{dialinecolor}{rgb}{0.000000, 0.000000, 0.000000}
\pgfsetstrokecolor{dialinecolor}
\node at (3.565000\du,9.185000\du){155 Kg};
\definecolor{dialinecolor}{rgb}{1.000000, 1.000000, 1.000000}
\pgfsetfillcolor{dialinecolor}
\fill (6.878750\du,8.030000\du)--(6.878750\du,9.930000\du)--(10.231250\du,9.930000\du)--(10.231250\du,8.030000\du)--cycle;
\pgfsetlinewidth{0.100000\du}
\pgfsetdash{}{0pt}
\pgfsetdash{}{0pt}
\pgfsetmiterjoin
\definecolor{dialinecolor}{rgb}{0.000000, 0.000000, 0.000000}
\pgfsetstrokecolor{dialinecolor}
\draw (6.878750\du,8.030000\du)--(6.878750\du,9.930000\du)--(10.231250\du,9.930000\du)--(10.231250\du,8.030000\du)--cycle;
% setfont left to latex
\definecolor{dialinecolor}{rgb}{0.000000, 0.000000, 0.000000}
\pgfsetstrokecolor{dialinecolor}
\node at (8.555000\du,9.175000\du){127 Kg};
\definecolor{dialinecolor}{rgb}{1.000000, 1.000000, 1.000000}
\pgfsetfillcolor{dialinecolor}
\fill (11.918750\du,8.070000\du)--(11.918750\du,9.970000\du)--(15.271250\du,9.970000\du)--(15.271250\du,8.070000\du)--cycle;
\pgfsetlinewidth{0.100000\du}
\pgfsetdash{}{0pt}
\pgfsetdash{}{0pt}
\pgfsetmiterjoin
\definecolor{dialinecolor}{rgb}{0.000000, 0.000000, 0.000000}
\pgfsetstrokecolor{dialinecolor}
\draw (11.918750\du,8.070000\du)--(11.918750\du,9.970000\du)--(15.271250\du,9.970000\du)--(15.271250\du,8.070000\du)--cycle;
% setfont left to latex
\definecolor{dialinecolor}{rgb}{0.000000, 0.000000, 0.000000}
\pgfsetstrokecolor{dialinecolor}
\node at (13.595000\du,9.215000\du){166 Kg};
\definecolor{dialinecolor}{rgb}{1.000000, 1.000000, 1.000000}
\pgfsetfillcolor{dialinecolor}
\fill (16.708750\du,8.110000\du)--(16.708750\du,10.010000\du)--(20.061250\du,10.010000\du)--(20.061250\du,8.110000\du)--cycle;
\pgfsetlinewidth{0.100000\du}
\pgfsetdash{}{0pt}
\pgfsetdash{}{0pt}
\pgfsetmiterjoin
\definecolor{dialinecolor}{rgb}{0.000000, 0.000000, 0.000000}
\pgfsetstrokecolor{dialinecolor}
\draw (16.708750\du,8.110000\du)--(16.708750\du,10.010000\du)--(20.061250\du,10.010000\du)--(20.061250\du,8.110000\du)--cycle;
% setfont left to latex
\definecolor{dialinecolor}{rgb}{0.000000, 0.000000, 0.000000}
\pgfsetstrokecolor{dialinecolor}
\node at (18.385000\du,9.255000\du){146 Kg};
\definecolor{dialinecolor}{rgb}{1.000000, 1.000000, 1.000000}
\pgfsetfillcolor{dialinecolor}
\fill (6.898750\du,13.100000\du)--(6.898750\du,15.000000\du)--(10.251250\du,15.000000\du)--(10.251250\du,13.100000\du)--cycle;
\pgfsetlinewidth{0.100000\du}
\pgfsetdash{}{0pt}
\pgfsetdash{}{0pt}
\pgfsetmiterjoin
\definecolor{dialinecolor}{rgb}{0.000000, 0.000000, 0.000000}
\pgfsetstrokecolor{dialinecolor}
\draw (6.898750\du,13.100000\du)--(6.898750\du,15.000000\du)--(10.251250\du,15.000000\du)--(10.251250\du,13.100000\du)--cycle;
% setfont left to latex
\definecolor{dialinecolor}{rgb}{0.000000, 0.000000, 0.000000}
\pgfsetstrokecolor{dialinecolor}
\node at (8.575000\du,14.245000\du){177 Kg};
\definecolor{dialinecolor}{rgb}{1.000000, 1.000000, 1.000000}
\pgfsetfillcolor{dialinecolor}
\fill (11.988750\du,12.990000\du)--(11.988750\du,14.890000\du)--(15.341250\du,14.890000\du)--(15.341250\du,12.990000\du)--cycle;
\pgfsetlinewidth{0.100000\du}
\pgfsetdash{}{0pt}
\pgfsetdash{}{0pt}
\pgfsetmiterjoin
\definecolor{dialinecolor}{rgb}{0.000000, 0.000000, 0.000000}
\pgfsetstrokecolor{dialinecolor}
\draw (11.988750\du,12.990000\du)--(11.988750\du,14.890000\du)--(15.341250\du,14.890000\du)--(15.341250\du,12.990000\du)--cycle;
% setfont left to latex
\definecolor{dialinecolor}{rgb}{0.000000, 0.000000, 0.000000}
\pgfsetstrokecolor{dialinecolor}
\node at (13.665000\du,14.135000\du){168 Kg};
\pgfsetlinewidth{0.100000\du}
\pgfsetdash{}{0pt}
\pgfsetdash{}{0pt}
\pgfsetbuttcap
\pgfsetmiterjoin
\pgfsetlinewidth{0.100000\du}
\pgfsetbuttcap
\pgfsetmiterjoin
\pgfsetdash{}{0pt}
\definecolor{dialinecolor}{rgb}{1.000000, 1.000000, 1.000000}
\pgfsetfillcolor{dialinecolor}
\pgfpathellipse{\pgfpoint{3.075000\du}{2.475000\du}}{\pgfpoint{0.475000\du}{0\du}}{\pgfpoint{0\du}{0.475000\du}}
\pgfusepath{fill}
\definecolor{dialinecolor}{rgb}{0.000000, 0.000000, 0.000000}
\pgfsetstrokecolor{dialinecolor}
\pgfpathellipse{\pgfpoint{3.075000\du}{2.475000\du}}{\pgfpoint{0.475000\du}{0\du}}{\pgfpoint{0\du}{0.475000\du}}
\pgfusepath{stroke}
\pgfsetbuttcap
\pgfsetmiterjoin
\pgfsetdash{}{0pt}
\definecolor{dialinecolor}{rgb}{0.000000, 0.000000, 0.000000}
\pgfsetstrokecolor{dialinecolor}
\draw (3.075000\du,2.000000\du)--(3.075000\du,2.950000\du);
\pgfsetbuttcap
\pgfsetmiterjoin
\pgfsetdash{}{0pt}
\definecolor{dialinecolor}{rgb}{0.000000, 0.000000, 0.000000}
\pgfsetstrokecolor{dialinecolor}
\draw (2.600000\du,2.475000\du)--(3.550000\du,2.475000\du);
\pgfsetlinewidth{0.100000\du}
\pgfsetdash{}{0pt}
\pgfsetdash{}{0pt}
\pgfsetbuttcap
\pgfsetmiterjoin
\pgfsetlinewidth{0.100000\du}
\pgfsetbuttcap
\pgfsetmiterjoin
\pgfsetdash{}{0pt}
\definecolor{dialinecolor}{rgb}{1.000000, 1.000000, 1.000000}
\pgfsetfillcolor{dialinecolor}
\pgfpathellipse{\pgfpoint{4.425000\du}{2.475000\du}}{\pgfpoint{0.525000\du}{0\du}}{\pgfpoint{0\du}{0.475000\du}}
\pgfusepath{fill}
\definecolor{dialinecolor}{rgb}{0.000000, 0.000000, 0.000000}
\pgfsetstrokecolor{dialinecolor}
\pgfpathellipse{\pgfpoint{4.425000\du}{2.475000\du}}{\pgfpoint{0.525000\du}{0\du}}{\pgfpoint{0\du}{0.475000\du}}
\pgfusepath{stroke}
\pgfsetbuttcap
\pgfsetmiterjoin
\pgfsetdash{}{0pt}
\definecolor{dialinecolor}{rgb}{0.000000, 0.000000, 0.000000}
\pgfsetstrokecolor{dialinecolor}
\draw (4.053772\du,2.139127\du)--(4.796228\du,2.810872\du);
\pgfsetbuttcap
\pgfsetmiterjoin
\pgfsetdash{}{0pt}
\definecolor{dialinecolor}{rgb}{0.000000, 0.000000, 0.000000}
\pgfsetstrokecolor{dialinecolor}
\draw (4.053772\du,2.810872\du)--(4.796228\du,2.139127\du);
\pgfsetlinewidth{0.100000\du}
\pgfsetdash{}{0pt}
\pgfsetdash{}{0pt}
\pgfsetbuttcap
\pgfsetmiterjoin
\pgfsetlinewidth{0.100000\du}
\pgfsetbuttcap
\pgfsetmiterjoin
\pgfsetdash{}{0pt}
\definecolor{dialinecolor}{rgb}{1.000000, 1.000000, 1.000000}
\pgfsetfillcolor{dialinecolor}
\pgfpathellipse{\pgfpoint{7.515000\du}{2.565000\du}}{\pgfpoint{0.475000\du}{0\du}}{\pgfpoint{0\du}{0.475000\du}}
\pgfusepath{fill}
\definecolor{dialinecolor}{rgb}{0.000000, 0.000000, 0.000000}
\pgfsetstrokecolor{dialinecolor}
\pgfpathellipse{\pgfpoint{7.515000\du}{2.565000\du}}{\pgfpoint{0.475000\du}{0\du}}{\pgfpoint{0\du}{0.475000\du}}
\pgfusepath{stroke}
\pgfsetbuttcap
\pgfsetmiterjoin
\pgfsetdash{}{0pt}
\definecolor{dialinecolor}{rgb}{0.000000, 0.000000, 0.000000}
\pgfsetstrokecolor{dialinecolor}
\draw (7.515000\du,2.090000\du)--(7.515000\du,3.040000\du);
\pgfsetbuttcap
\pgfsetmiterjoin
\pgfsetdash{}{0pt}
\definecolor{dialinecolor}{rgb}{0.000000, 0.000000, 0.000000}
\pgfsetstrokecolor{dialinecolor}
\draw (7.040000\du,2.565000\du)--(7.990000\du,2.565000\du);
\pgfsetlinewidth{0.100000\du}
\pgfsetdash{}{0pt}
\pgfsetdash{}{0pt}
\pgfsetbuttcap
\pgfsetmiterjoin
\pgfsetlinewidth{0.100000\du}
\pgfsetbuttcap
\pgfsetmiterjoin
\pgfsetdash{}{0pt}
\definecolor{dialinecolor}{rgb}{1.000000, 1.000000, 1.000000}
\pgfsetfillcolor{dialinecolor}
\pgfpathellipse{\pgfpoint{8.615000\du}{2.515000\du}}{\pgfpoint{0.525000\du}{0\du}}{\pgfpoint{0\du}{0.475000\du}}
\pgfusepath{fill}
\definecolor{dialinecolor}{rgb}{0.000000, 0.000000, 0.000000}
\pgfsetstrokecolor{dialinecolor}
\pgfpathellipse{\pgfpoint{8.615000\du}{2.515000\du}}{\pgfpoint{0.525000\du}{0\du}}{\pgfpoint{0\du}{0.475000\du}}
\pgfusepath{stroke}
\pgfsetbuttcap
\pgfsetmiterjoin
\pgfsetdash{}{0pt}
\definecolor{dialinecolor}{rgb}{0.000000, 0.000000, 0.000000}
\pgfsetstrokecolor{dialinecolor}
\draw (8.243773\du,2.179128\du)--(8.986228\du,2.850873\du);
\pgfsetbuttcap
\pgfsetmiterjoin
\pgfsetdash{}{0pt}
\definecolor{dialinecolor}{rgb}{0.000000, 0.000000, 0.000000}
\pgfsetstrokecolor{dialinecolor}
\draw (8.243773\du,2.850873\du)--(8.986228\du,2.179128\du);
\pgfsetlinewidth{0.100000\du}
\pgfsetdash{}{0pt}
\pgfsetdash{}{0pt}
\pgfsetbuttcap
\pgfsetmiterjoin
\pgfsetlinewidth{0.100000\du}
\pgfsetbuttcap
\pgfsetmiterjoin
\pgfsetdash{}{0pt}
\definecolor{dialinecolor}{rgb}{1.000000, 1.000000, 1.000000}
\pgfsetfillcolor{dialinecolor}
\pgfpathmoveto{\pgfpoint{9.375000\du}{2.100000\du}}
\pgfpathlineto{\pgfpoint{9.875000\du}{2.100000\du}}
\pgfpathcurveto{\pgfpoint{9.944036\du}{2.100000\du}}{\pgfpoint{10.000000\du}{2.290279\du}}{\pgfpoint{10.000000\du}{2.525000\du}}
\pgfpathcurveto{\pgfpoint{10.000000\du}{2.759721\du}}{\pgfpoint{9.944036\du}{2.950000\du}}{\pgfpoint{9.875000\du}{2.950000\du}}
\pgfpathlineto{\pgfpoint{9.375000\du}{2.950000\du}}
\pgfpathcurveto{\pgfpoint{9.305964\du}{2.950000\du}}{\pgfpoint{9.250000\du}{2.759721\du}}{\pgfpoint{9.250000\du}{2.525000\du}}
\pgfpathcurveto{\pgfpoint{9.250000\du}{2.290279\du}}{\pgfpoint{9.305964\du}{2.100000\du}}{\pgfpoint{9.375000\du}{2.100000\du}}
\pgfusepath{fill}
\definecolor{dialinecolor}{rgb}{0.000000, 0.000000, 0.000000}
\pgfsetstrokecolor{dialinecolor}
\pgfpathmoveto{\pgfpoint{9.375000\du}{2.100000\du}}
\pgfpathlineto{\pgfpoint{9.875000\du}{2.100000\du}}
\pgfpathcurveto{\pgfpoint{9.944036\du}{2.100000\du}}{\pgfpoint{10.000000\du}{2.290279\du}}{\pgfpoint{10.000000\du}{2.525000\du}}
\pgfpathcurveto{\pgfpoint{10.000000\du}{2.759721\du}}{\pgfpoint{9.944036\du}{2.950000\du}}{\pgfpoint{9.875000\du}{2.950000\du}}
\pgfpathlineto{\pgfpoint{9.375000\du}{2.950000\du}}
\pgfpathcurveto{\pgfpoint{9.305964\du}{2.950000\du}}{\pgfpoint{9.250000\du}{2.759721\du}}{\pgfpoint{9.250000\du}{2.525000\du}}
\pgfpathcurveto{\pgfpoint{9.250000\du}{2.290279\du}}{\pgfpoint{9.305964\du}{2.100000\du}}{\pgfpoint{9.375000\du}{2.100000\du}}
\pgfusepath{stroke}
% setfont left to latex
\definecolor{dialinecolor}{rgb}{0.000000, 0.000000, 0.000000}
\pgfsetstrokecolor{dialinecolor}
\node at (9.625000\du,2.613194\du){};
\pgfsetlinewidth{0.100000\du}
\pgfsetdash{}{0pt}
\pgfsetdash{}{0pt}
\pgfsetbuttcap
\pgfsetmiterjoin
\pgfsetlinewidth{0.100000\du}
\pgfsetbuttcap
\pgfsetmiterjoin
\pgfsetdash{}{0pt}
\definecolor{dialinecolor}{rgb}{1.000000, 1.000000, 1.000000}
\pgfsetfillcolor{dialinecolor}
\fill (13.600000\du,2.950000\du)--(14.450000\du,2.950000\du)--(14.025000\du,1.785556\du)--cycle;
\definecolor{dialinecolor}{rgb}{0.000000, 0.000000, 0.000000}
\pgfsetstrokecolor{dialinecolor}
\draw (13.600000\du,2.950000\du)--(14.450000\du,2.950000\du)--(14.025000\du,1.785556\du)--cycle;
% setfont left to latex
\definecolor{dialinecolor}{rgb}{0.000000, 0.000000, 0.000000}
\pgfsetstrokecolor{dialinecolor}
\node at (14.025000\du,2.729444\du){};
\pgfsetlinewidth{0.100000\du}
\pgfsetdash{}{0pt}
\pgfsetdash{}{0pt}
\pgfsetbuttcap
\pgfsetmiterjoin
\pgfsetlinewidth{0.100000\du}
\pgfsetbuttcap
\pgfsetmiterjoin
\pgfsetdash{}{0pt}
\definecolor{dialinecolor}{rgb}{1.000000, 1.000000, 1.000000}
\pgfsetfillcolor{dialinecolor}
\fill (18.500000\du,2.000000\du)--(19.450000\du,2.000000\du)--(18.975000\du,2.450000\du)--cycle;
\definecolor{dialinecolor}{rgb}{0.000000, 0.000000, 0.000000}
\pgfsetstrokecolor{dialinecolor}
\draw (18.500000\du,2.000000\du)--(19.450000\du,2.000000\du)--(18.975000\du,2.450000\du)--cycle;
\pgfsetbuttcap
\pgfsetmiterjoin
\pgfsetdash{}{0pt}
\definecolor{dialinecolor}{rgb}{1.000000, 1.000000, 1.000000}
\pgfsetfillcolor{dialinecolor}
\fill (18.500000\du,2.900000\du)--(19.450000\du,2.900000\du)--(18.975000\du,2.450000\du)--cycle;
\definecolor{dialinecolor}{rgb}{0.000000, 0.000000, 0.000000}
\pgfsetstrokecolor{dialinecolor}
\draw (18.500000\du,2.900000\du)--(19.450000\du,2.900000\du)--(18.975000\du,2.450000\du)--cycle;
\pgfsetlinewidth{0.100000\du}
\pgfsetdash{}{0pt}
\pgfsetdash{}{0pt}
\pgfsetbuttcap
\pgfsetmiterjoin
\pgfsetlinewidth{0.100000\du}
\pgfsetbuttcap
\pgfsetmiterjoin
\pgfsetdash{}{0pt}
\definecolor{dialinecolor}{rgb}{1.000000, 1.000000, 1.000000}
\pgfsetfillcolor{dialinecolor}
\fill (3.265479\du,7.150000\du)--(4.215479\du,7.150000\du)--(3.740479\du,7.470000\du)--cycle;
\definecolor{dialinecolor}{rgb}{0.000000, 0.000000, 0.000000}
\pgfsetstrokecolor{dialinecolor}
\draw (3.265479\du,7.150000\du)--(4.215479\du,7.150000\du)--(3.740479\du,7.470000\du)--cycle;
\pgfsetbuttcap
\pgfsetmiterjoin
\pgfsetdash{}{0pt}
\definecolor{dialinecolor}{rgb}{1.000000, 1.000000, 1.000000}
\pgfsetfillcolor{dialinecolor}
\fill (3.265479\du,7.790000\du)--(4.215479\du,7.790000\du)--(3.740479\du,7.470000\du)--cycle;
\definecolor{dialinecolor}{rgb}{0.000000, 0.000000, 0.000000}
\pgfsetstrokecolor{dialinecolor}
\draw (3.265479\du,7.790000\du)--(4.215479\du,7.790000\du)--(3.740479\du,7.470000\du)--cycle;
\pgfsetlinewidth{0.100000\du}
\pgfsetdash{}{0pt}
\pgfsetdash{}{0pt}
\pgfsetbuttcap
\pgfsetmiterjoin
\pgfsetlinewidth{0.100000\du}
\pgfsetbuttcap
\pgfsetmiterjoin
\pgfsetdash{}{0pt}
\definecolor{dialinecolor}{rgb}{1.000000, 1.000000, 1.000000}
\pgfsetfillcolor{dialinecolor}
\fill (4.925000\du,6.950000\du)--(5.500000\du,7.425000\du)--(4.925000\du,7.900000\du)--(4.350000\du,7.425000\du)--cycle;
\definecolor{dialinecolor}{rgb}{0.000000, 0.000000, 0.000000}
\pgfsetstrokecolor{dialinecolor}
\draw (4.925000\du,6.950000\du)--(5.500000\du,7.425000\du)--(4.925000\du,7.900000\du)--(4.350000\du,7.425000\du)--cycle;
\pgfsetbuttcap
\pgfsetmiterjoin
\pgfsetdash{}{0pt}
\definecolor{dialinecolor}{rgb}{0.000000, 0.000000, 0.000000}
\pgfsetstrokecolor{dialinecolor}
\draw (4.350000\du,7.425000\du)--(5.500000\du,7.425000\du);
\pgfsetlinewidth{0.100000\du}
\pgfsetdash{}{0pt}
\pgfsetdash{}{0pt}
\pgfsetbuttcap
\pgfsetmiterjoin
\pgfsetlinewidth{0.100000\du}
\pgfsetbuttcap
\pgfsetmiterjoin
\pgfsetdash{}{0pt}
\definecolor{dialinecolor}{rgb}{1.000000, 1.000000, 1.000000}
\pgfsetfillcolor{dialinecolor}
\fill (7.693508\du,6.954854\du)--(8.268508\du,7.429854\du)--(7.693508\du,7.904854\du)--(7.118508\du,7.429854\du)--cycle;
\definecolor{dialinecolor}{rgb}{0.000000, 0.000000, 0.000000}
\pgfsetstrokecolor{dialinecolor}
\draw (7.693508\du,6.954854\du)--(8.268508\du,7.429854\du)--(7.693508\du,7.904854\du)--(7.118508\du,7.429854\du)--cycle;
\pgfsetbuttcap
\pgfsetmiterjoin
\pgfsetdash{}{0pt}
\definecolor{dialinecolor}{rgb}{0.000000, 0.000000, 0.000000}
\pgfsetstrokecolor{dialinecolor}
\draw (7.118508\du,7.429854\du)--(8.268508\du,7.429854\du);
\pgfsetlinewidth{0.100000\du}
\pgfsetdash{}{0pt}
\pgfsetdash{}{0pt}
\pgfsetbuttcap
\pgfsetmiterjoin
\pgfsetlinewidth{0.100000\du}
\pgfsetbuttcap
\pgfsetmiterjoin
\pgfsetdash{}{0pt}
\definecolor{dialinecolor}{rgb}{1.000000, 1.000000, 1.000000}
\pgfsetfillcolor{dialinecolor}
\pgfpathmoveto{\pgfpoint{12.100000\du}{6.962037\du}}
\pgfpathcurveto{\pgfpoint{12.370000\du}{6.798843\du}}{\pgfpoint{12.505000\du}{6.744444\du}}{\pgfpoint{12.775000\du}{6.744444\du}}
\pgfpathcurveto{\pgfpoint{13.045000\du}{6.744444\du}}{\pgfpoint{13.180000\du}{6.798843\du}}{\pgfpoint{13.450000\du}{6.962037\du}}
\pgfpathlineto{\pgfpoint{13.450000\du}{7.832407\du}}
\pgfpathcurveto{\pgfpoint{13.180000\du}{7.995602\du}}{\pgfpoint{13.045000\du}{8.050000\du}}{\pgfpoint{12.775000\du}{8.050000\du}}
\pgfpathcurveto{\pgfpoint{12.505000\du}{8.050000\du}}{\pgfpoint{12.370000\du}{7.995602\du}}{\pgfpoint{12.100000\du}{7.832407\du}}
\pgfpathlineto{\pgfpoint{12.100000\du}{6.962037\du}}
\pgfusepath{fill}
\definecolor{dialinecolor}{rgb}{0.000000, 0.000000, 0.000000}
\pgfsetstrokecolor{dialinecolor}
\pgfpathmoveto{\pgfpoint{12.100000\du}{6.962037\du}}
\pgfpathcurveto{\pgfpoint{12.370000\du}{6.798843\du}}{\pgfpoint{12.505000\du}{6.744444\du}}{\pgfpoint{12.775000\du}{6.744444\du}}
\pgfpathcurveto{\pgfpoint{13.045000\du}{6.744444\du}}{\pgfpoint{13.180000\du}{6.798843\du}}{\pgfpoint{13.450000\du}{6.962037\du}}
\pgfpathlineto{\pgfpoint{13.450000\du}{7.832407\du}}
\pgfpathcurveto{\pgfpoint{13.180000\du}{7.995602\du}}{\pgfpoint{13.045000\du}{8.050000\du}}{\pgfpoint{12.775000\du}{8.050000\du}}
\pgfpathcurveto{\pgfpoint{12.505000\du}{8.050000\du}}{\pgfpoint{12.370000\du}{7.995602\du}}{\pgfpoint{12.100000\du}{7.832407\du}}
\pgfpathlineto{\pgfpoint{12.100000\du}{6.962037\du}}
\pgfusepath{stroke}
\pgfsetbuttcap
\pgfsetmiterjoin
\pgfsetdash{}{0pt}
\definecolor{dialinecolor}{rgb}{0.000000, 0.000000, 0.000000}
\pgfsetstrokecolor{dialinecolor}
\pgfpathmoveto{\pgfpoint{12.100000\du}{6.962037\du}}
\pgfpathcurveto{\pgfpoint{12.370000\du}{7.125231\du}}{\pgfpoint{12.505000\du}{7.179630\du}}{\pgfpoint{12.775000\du}{7.179630\du}}
\pgfpathcurveto{\pgfpoint{13.045000\du}{7.179630\du}}{\pgfpoint{13.180000\du}{7.125231\du}}{\pgfpoint{13.450000\du}{6.962037\du}}
\pgfusepath{stroke}
% setfont left to latex
\definecolor{dialinecolor}{rgb}{0.000000, 0.000000, 0.000000}
\pgfsetstrokecolor{dialinecolor}
\node at (12.775000\du,7.594213\du){};
\pgfsetlinewidth{0.100000\du}
\pgfsetdash{}{0pt}
\pgfsetdash{}{0pt}
\pgfsetbuttcap
\pgfsetmiterjoin
\pgfsetlinewidth{0.100000\du}
\pgfsetbuttcap
\pgfsetmiterjoin
\pgfsetdash{}{0pt}
\definecolor{dialinecolor}{rgb}{1.000000, 1.000000, 1.000000}
\pgfsetfillcolor{dialinecolor}
\pgfpathellipse{\pgfpoint{14.315000\du}{7.415000\du}}{\pgfpoint{0.475000\du}{0\du}}{\pgfpoint{0\du}{0.475000\du}}
\pgfusepath{fill}
\definecolor{dialinecolor}{rgb}{0.000000, 0.000000, 0.000000}
\pgfsetstrokecolor{dialinecolor}
\pgfpathellipse{\pgfpoint{14.315000\du}{7.415000\du}}{\pgfpoint{0.475000\du}{0\du}}{\pgfpoint{0\du}{0.475000\du}}
\pgfusepath{stroke}
\pgfsetbuttcap
\pgfsetmiterjoin
\pgfsetdash{}{0pt}
\definecolor{dialinecolor}{rgb}{0.000000, 0.000000, 0.000000}
\pgfsetstrokecolor{dialinecolor}
\draw (14.315000\du,6.940000\du)--(14.315000\du,7.890000\du);
\pgfsetbuttcap
\pgfsetmiterjoin
\pgfsetdash{}{0pt}
\definecolor{dialinecolor}{rgb}{0.000000, 0.000000, 0.000000}
\pgfsetstrokecolor{dialinecolor}
\draw (13.840000\du,7.415000\du)--(14.790000\du,7.415000\du);
\pgfsetlinewidth{0.100000\du}
\pgfsetdash{}{0pt}
\pgfsetdash{}{0pt}
\pgfsetbuttcap
\pgfsetmiterjoin
\pgfsetlinewidth{0.100000\du}
\pgfsetbuttcap
\pgfsetmiterjoin
\pgfsetdash{}{0pt}
\definecolor{dialinecolor}{rgb}{1.000000, 1.000000, 1.000000}
\pgfsetfillcolor{dialinecolor}
\pgfpathmoveto{\pgfpoint{16.840000\du}{6.946681\du}}
\pgfpathcurveto{\pgfpoint{17.110000\du}{6.783487\du}}{\pgfpoint{17.245000\du}{6.729089\du}}{\pgfpoint{17.515000\du}{6.729089\du}}
\pgfpathcurveto{\pgfpoint{17.785000\du}{6.729089\du}}{\pgfpoint{17.920000\du}{6.783487\du}}{\pgfpoint{18.190000\du}{6.946681\du}}
\pgfpathlineto{\pgfpoint{18.190000\du}{7.817052\du}}
\pgfpathcurveto{\pgfpoint{17.920000\du}{7.980246\du}}{\pgfpoint{17.785000\du}{8.034644\du}}{\pgfpoint{17.515000\du}{8.034644\du}}
\pgfpathcurveto{\pgfpoint{17.245000\du}{8.034644\du}}{\pgfpoint{17.110000\du}{7.980246\du}}{\pgfpoint{16.840000\du}{7.817052\du}}
\pgfpathlineto{\pgfpoint{16.840000\du}{6.946681\du}}
\pgfusepath{fill}
\definecolor{dialinecolor}{rgb}{0.000000, 0.000000, 0.000000}
\pgfsetstrokecolor{dialinecolor}
\pgfpathmoveto{\pgfpoint{16.840000\du}{6.946681\du}}
\pgfpathcurveto{\pgfpoint{17.110000\du}{6.783487\du}}{\pgfpoint{17.245000\du}{6.729089\du}}{\pgfpoint{17.515000\du}{6.729089\du}}
\pgfpathcurveto{\pgfpoint{17.785000\du}{6.729089\du}}{\pgfpoint{17.920000\du}{6.783487\du}}{\pgfpoint{18.190000\du}{6.946681\du}}
\pgfpathlineto{\pgfpoint{18.190000\du}{7.817052\du}}
\pgfpathcurveto{\pgfpoint{17.920000\du}{7.980246\du}}{\pgfpoint{17.785000\du}{8.034644\du}}{\pgfpoint{17.515000\du}{8.034644\du}}
\pgfpathcurveto{\pgfpoint{17.245000\du}{8.034644\du}}{\pgfpoint{17.110000\du}{7.980246\du}}{\pgfpoint{16.840000\du}{7.817052\du}}
\pgfpathlineto{\pgfpoint{16.840000\du}{6.946681\du}}
\pgfusepath{stroke}
\pgfsetbuttcap
\pgfsetmiterjoin
\pgfsetdash{}{0pt}
\definecolor{dialinecolor}{rgb}{0.000000, 0.000000, 0.000000}
\pgfsetstrokecolor{dialinecolor}
\pgfpathmoveto{\pgfpoint{16.840000\du}{6.946681\du}}
\pgfpathcurveto{\pgfpoint{17.110000\du}{7.109876\du}}{\pgfpoint{17.245000\du}{7.164274\du}}{\pgfpoint{17.515000\du}{7.164274\du}}
\pgfpathcurveto{\pgfpoint{17.785000\du}{7.164274\du}}{\pgfpoint{17.920000\du}{7.109876\du}}{\pgfpoint{18.190000\du}{6.946681\du}}
\pgfusepath{stroke}
% setfont left to latex
\definecolor{dialinecolor}{rgb}{0.000000, 0.000000, 0.000000}
\pgfsetstrokecolor{dialinecolor}
\node at (17.515000\du,7.578857\du){};
\pgfsetlinewidth{0.100000\du}
\pgfsetdash{}{0pt}
\pgfsetdash{}{0pt}
\pgfsetbuttcap
\pgfsetmiterjoin
\pgfsetlinewidth{0.100000\du}
\pgfsetbuttcap
\pgfsetmiterjoin
\pgfsetdash{}{0pt}
\definecolor{dialinecolor}{rgb}{1.000000, 1.000000, 1.000000}
\pgfsetfillcolor{dialinecolor}
\pgfpathmoveto{\pgfpoint{18.850000\du}{6.813194\du}}
\pgfpathcurveto{\pgfpoint{19.030000\du}{6.976389\du}}{\pgfpoint{19.120000\du}{6.976389\du}}{\pgfpoint{19.300000\du}{6.813194\du}}
\pgfpathcurveto{\pgfpoint{19.480000\du}{6.650000\du}}{\pgfpoint{19.570000\du}{6.650000\du}}{\pgfpoint{19.750000\du}{6.813194\du}}
\pgfpathlineto{\pgfpoint{19.750000\du}{7.792361\du}}
\pgfpathcurveto{\pgfpoint{19.570000\du}{7.629167\du}}{\pgfpoint{19.480000\du}{7.629167\du}}{\pgfpoint{19.300000\du}{7.792361\du}}
\pgfpathcurveto{\pgfpoint{19.120000\du}{7.955556\du}}{\pgfpoint{19.030000\du}{7.955556\du}}{\pgfpoint{18.850000\du}{7.792361\du}}
\pgfpathlineto{\pgfpoint{18.850000\du}{6.813194\du}}
\pgfusepath{fill}
\definecolor{dialinecolor}{rgb}{0.000000, 0.000000, 0.000000}
\pgfsetstrokecolor{dialinecolor}
\pgfpathmoveto{\pgfpoint{18.850000\du}{6.813194\du}}
\pgfpathcurveto{\pgfpoint{19.030000\du}{6.976389\du}}{\pgfpoint{19.120000\du}{6.976389\du}}{\pgfpoint{19.300000\du}{6.813194\du}}
\pgfpathcurveto{\pgfpoint{19.480000\du}{6.650000\du}}{\pgfpoint{19.570000\du}{6.650000\du}}{\pgfpoint{19.750000\du}{6.813194\du}}
\pgfpathlineto{\pgfpoint{19.750000\du}{7.792361\du}}
\pgfpathcurveto{\pgfpoint{19.570000\du}{7.629167\du}}{\pgfpoint{19.480000\du}{7.629167\du}}{\pgfpoint{19.300000\du}{7.792361\du}}
\pgfpathcurveto{\pgfpoint{19.120000\du}{7.955556\du}}{\pgfpoint{19.030000\du}{7.955556\du}}{\pgfpoint{18.850000\du}{7.792361\du}}
\pgfpathlineto{\pgfpoint{18.850000\du}{6.813194\du}}
\pgfusepath{stroke}
% setfont left to latex
\definecolor{dialinecolor}{rgb}{0.000000, 0.000000, 0.000000}
\pgfsetstrokecolor{dialinecolor}
\node at (19.300000\du,7.390972\du){};
\pgfsetlinewidth{0.100000\du}
\pgfsetdash{}{0pt}
\pgfsetdash{}{0pt}
\pgfsetbuttcap
\pgfsetmiterjoin
\pgfsetlinewidth{0.100000\du}
\pgfsetbuttcap
\pgfsetmiterjoin
\pgfsetdash{}{0pt}
\definecolor{dialinecolor}{rgb}{1.000000, 1.000000, 1.000000}
\pgfsetfillcolor{dialinecolor}
\pgfpathmoveto{\pgfpoint{12.415000\du}{1.990000\du}}
\pgfpathlineto{\pgfpoint{12.915000\du}{1.990000\du}}
\pgfpathcurveto{\pgfpoint{12.984036\du}{1.990000\du}}{\pgfpoint{13.040000\du}{2.180279\du}}{\pgfpoint{13.040000\du}{2.415000\du}}
\pgfpathcurveto{\pgfpoint{13.040000\du}{2.649721\du}}{\pgfpoint{12.984036\du}{2.840000\du}}{\pgfpoint{12.915000\du}{2.840000\du}}
\pgfpathlineto{\pgfpoint{12.415000\du}{2.840000\du}}
\pgfpathcurveto{\pgfpoint{12.345964\du}{2.840000\du}}{\pgfpoint{12.290000\du}{2.649721\du}}{\pgfpoint{12.290000\du}{2.415000\du}}
\pgfpathcurveto{\pgfpoint{12.290000\du}{2.180279\du}}{\pgfpoint{12.345964\du}{1.990000\du}}{\pgfpoint{12.415000\du}{1.990000\du}}
\pgfusepath{fill}
\definecolor{dialinecolor}{rgb}{0.000000, 0.000000, 0.000000}
\pgfsetstrokecolor{dialinecolor}
\pgfpathmoveto{\pgfpoint{12.415000\du}{1.990000\du}}
\pgfpathlineto{\pgfpoint{12.915000\du}{1.990000\du}}
\pgfpathcurveto{\pgfpoint{12.984036\du}{1.990000\du}}{\pgfpoint{13.040000\du}{2.180279\du}}{\pgfpoint{13.040000\du}{2.415000\du}}
\pgfpathcurveto{\pgfpoint{13.040000\du}{2.649721\du}}{\pgfpoint{12.984036\du}{2.840000\du}}{\pgfpoint{12.915000\du}{2.840000\du}}
\pgfpathlineto{\pgfpoint{12.415000\du}{2.840000\du}}
\pgfpathcurveto{\pgfpoint{12.345964\du}{2.840000\du}}{\pgfpoint{12.290000\du}{2.649721\du}}{\pgfpoint{12.290000\du}{2.415000\du}}
\pgfpathcurveto{\pgfpoint{12.290000\du}{2.180279\du}}{\pgfpoint{12.345964\du}{1.990000\du}}{\pgfpoint{12.415000\du}{1.990000\du}}
\pgfusepath{stroke}
% setfont left to latex
\definecolor{dialinecolor}{rgb}{0.000000, 0.000000, 0.000000}
\pgfsetstrokecolor{dialinecolor}
\node at (12.665000\du,2.503194\du){};
\pgfsetlinewidth{0.100000\du}
\pgfsetdash{}{0pt}
\pgfsetdash{}{0pt}
\pgfsetbuttcap
\pgfsetmiterjoin
\pgfsetlinewidth{0.100000\du}
\pgfsetbuttcap
\pgfsetmiterjoin
\pgfsetdash{}{0pt}
\definecolor{dialinecolor}{rgb}{1.000000, 1.000000, 1.000000}
\pgfsetfillcolor{dialinecolor}
\fill (16.761475\du,3.000277\du)--(17.611475\du,3.000277\du)--(17.186475\du,1.835833\du)--cycle;
\definecolor{dialinecolor}{rgb}{0.000000, 0.000000, 0.000000}
\pgfsetstrokecolor{dialinecolor}
\draw (16.761475\du,3.000277\du)--(17.611475\du,3.000277\du)--(17.186475\du,1.835833\du)--cycle;
% setfont left to latex
\definecolor{dialinecolor}{rgb}{0.000000, 0.000000, 0.000000}
\pgfsetstrokecolor{dialinecolor}
\node at (17.186475\du,2.779722\du){};
\pgfsetlinewidth{0.100000\du}
\pgfsetdash{}{0pt}
\pgfsetdash{}{0pt}
\pgfsetbuttcap
\pgfsetmiterjoin
\pgfsetlinewidth{0.100000\du}
\pgfsetbuttcap
\pgfsetmiterjoin
\pgfsetdash{}{0pt}
\definecolor{dialinecolor}{rgb}{1.000000, 1.000000, 1.000000}
\pgfsetfillcolor{dialinecolor}
\fill (2.051475\du,7.990277\du)--(2.901475\du,7.990277\du)--(2.476475\du,6.825833\du)--cycle;
\definecolor{dialinecolor}{rgb}{0.000000, 0.000000, 0.000000}
\pgfsetstrokecolor{dialinecolor}
\draw (2.051475\du,7.990277\du)--(2.901475\du,7.990277\du)--(2.476475\du,6.825833\du)--cycle;
% setfont left to latex
\definecolor{dialinecolor}{rgb}{0.000000, 0.000000, 0.000000}
\pgfsetstrokecolor{dialinecolor}
\node at (2.476475\du,7.769722\du){};
\pgfsetlinewidth{0.100000\du}
\pgfsetdash{}{0pt}
\pgfsetdash{}{0pt}
\pgfsetbuttcap
\pgfsetmiterjoin
\pgfsetlinewidth{0.100000\du}
\pgfsetbuttcap
\pgfsetmiterjoin
\pgfsetdash{}{0pt}
\definecolor{dialinecolor}{rgb}{1.000000, 1.000000, 1.000000}
\pgfsetfillcolor{dialinecolor}
\fill (8.115479\du,12.090000\du)--(9.065479\du,12.090000\du)--(8.590479\du,12.540000\du)--cycle;
\definecolor{dialinecolor}{rgb}{0.000000, 0.000000, 0.000000}
\pgfsetstrokecolor{dialinecolor}
\draw (8.115479\du,12.090000\du)--(9.065479\du,12.090000\du)--(8.590479\du,12.540000\du)--cycle;
\pgfsetbuttcap
\pgfsetmiterjoin
\pgfsetdash{}{0pt}
\definecolor{dialinecolor}{rgb}{1.000000, 1.000000, 1.000000}
\pgfsetfillcolor{dialinecolor}
\fill (8.115479\du,12.990000\du)--(9.065479\du,12.990000\du)--(8.590479\du,12.540000\du)--cycle;
\definecolor{dialinecolor}{rgb}{0.000000, 0.000000, 0.000000}
\pgfsetstrokecolor{dialinecolor}
\draw (8.115479\du,12.990000\du)--(9.065479\du,12.990000\du)--(8.590479\du,12.540000\du)--cycle;
\pgfsetlinewidth{0.100000\du}
\pgfsetdash{}{0pt}
\pgfsetdash{}{0pt}
\pgfsetbuttcap
\pgfsetmiterjoin
\pgfsetlinewidth{0.100000\du}
\pgfsetbuttcap
\pgfsetmiterjoin
\pgfsetdash{}{0pt}
\definecolor{dialinecolor}{rgb}{1.000000, 1.000000, 1.000000}
\pgfsetfillcolor{dialinecolor}
\pgfpathmoveto{\pgfpoint{9.565000\du}{12.090000\du}}
\pgfpathlineto{\pgfpoint{10.065000\du}{12.090000\du}}
\pgfpathcurveto{\pgfpoint{10.134036\du}{12.090000\du}}{\pgfpoint{10.190000\du}{12.280279\du}}{\pgfpoint{10.190000\du}{12.515000\du}}
\pgfpathcurveto{\pgfpoint{10.190000\du}{12.749721\du}}{\pgfpoint{10.134036\du}{12.940000\du}}{\pgfpoint{10.065000\du}{12.940000\du}}
\pgfpathlineto{\pgfpoint{9.565000\du}{12.940000\du}}
\pgfpathcurveto{\pgfpoint{9.495964\du}{12.940000\du}}{\pgfpoint{9.440000\du}{12.749721\du}}{\pgfpoint{9.440000\du}{12.515000\du}}
\pgfpathcurveto{\pgfpoint{9.440000\du}{12.280279\du}}{\pgfpoint{9.495964\du}{12.090000\du}}{\pgfpoint{9.565000\du}{12.090000\du}}
\pgfusepath{fill}
\definecolor{dialinecolor}{rgb}{0.000000, 0.000000, 0.000000}
\pgfsetstrokecolor{dialinecolor}
\pgfpathmoveto{\pgfpoint{9.565000\du}{12.090000\du}}
\pgfpathlineto{\pgfpoint{10.065000\du}{12.090000\du}}
\pgfpathcurveto{\pgfpoint{10.134036\du}{12.090000\du}}{\pgfpoint{10.190000\du}{12.280279\du}}{\pgfpoint{10.190000\du}{12.515000\du}}
\pgfpathcurveto{\pgfpoint{10.190000\du}{12.749721\du}}{\pgfpoint{10.134036\du}{12.940000\du}}{\pgfpoint{10.065000\du}{12.940000\du}}
\pgfpathlineto{\pgfpoint{9.565000\du}{12.940000\du}}
\pgfpathcurveto{\pgfpoint{9.495964\du}{12.940000\du}}{\pgfpoint{9.440000\du}{12.749721\du}}{\pgfpoint{9.440000\du}{12.515000\du}}
\pgfpathcurveto{\pgfpoint{9.440000\du}{12.280279\du}}{\pgfpoint{9.495964\du}{12.090000\du}}{\pgfpoint{9.565000\du}{12.090000\du}}
\pgfusepath{stroke}
% setfont left to latex
\definecolor{dialinecolor}{rgb}{0.000000, 0.000000, 0.000000}
\pgfsetstrokecolor{dialinecolor}
\node at (9.815000\du,12.603194\du){};
\pgfsetlinewidth{0.100000\du}
\pgfsetdash{}{0pt}
\pgfsetdash{}{0pt}
\pgfsetbuttcap
\pgfsetmiterjoin
\pgfsetlinewidth{0.100000\du}
\pgfsetbuttcap
\pgfsetmiterjoin
\pgfsetdash{}{0pt}
\definecolor{dialinecolor}{rgb}{1.000000, 1.000000, 1.000000}
\pgfsetfillcolor{dialinecolor}
\pgfpathmoveto{\pgfpoint{12.415000\du}{12.040000\du}}
\pgfpathlineto{\pgfpoint{12.915000\du}{12.040000\du}}
\pgfpathcurveto{\pgfpoint{12.984036\du}{12.040000\du}}{\pgfpoint{13.040000\du}{12.230279\du}}{\pgfpoint{13.040000\du}{12.465000\du}}
\pgfpathcurveto{\pgfpoint{13.040000\du}{12.699721\du}}{\pgfpoint{12.984036\du}{12.890000\du}}{\pgfpoint{12.915000\du}{12.890000\du}}
\pgfpathlineto{\pgfpoint{12.415000\du}{12.890000\du}}
\pgfpathcurveto{\pgfpoint{12.345964\du}{12.890000\du}}{\pgfpoint{12.290000\du}{12.699721\du}}{\pgfpoint{12.290000\du}{12.465000\du}}
\pgfpathcurveto{\pgfpoint{12.290000\du}{12.230279\du}}{\pgfpoint{12.345964\du}{12.040000\du}}{\pgfpoint{12.415000\du}{12.040000\du}}
\pgfusepath{fill}
\definecolor{dialinecolor}{rgb}{0.000000, 0.000000, 0.000000}
\pgfsetstrokecolor{dialinecolor}
\pgfpathmoveto{\pgfpoint{12.415000\du}{12.040000\du}}
\pgfpathlineto{\pgfpoint{12.915000\du}{12.040000\du}}
\pgfpathcurveto{\pgfpoint{12.984036\du}{12.040000\du}}{\pgfpoint{13.040000\du}{12.230279\du}}{\pgfpoint{13.040000\du}{12.465000\du}}
\pgfpathcurveto{\pgfpoint{13.040000\du}{12.699721\du}}{\pgfpoint{12.984036\du}{12.890000\du}}{\pgfpoint{12.915000\du}{12.890000\du}}
\pgfpathlineto{\pgfpoint{12.415000\du}{12.890000\du}}
\pgfpathcurveto{\pgfpoint{12.345964\du}{12.890000\du}}{\pgfpoint{12.290000\du}{12.699721\du}}{\pgfpoint{12.290000\du}{12.465000\du}}
\pgfpathcurveto{\pgfpoint{12.290000\du}{12.230279\du}}{\pgfpoint{12.345964\du}{12.040000\du}}{\pgfpoint{12.415000\du}{12.040000\du}}
\pgfusepath{stroke}
% setfont left to latex
\definecolor{dialinecolor}{rgb}{0.000000, 0.000000, 0.000000}
\pgfsetstrokecolor{dialinecolor}
\node at (12.665000\du,12.553194\du){};
\pgfsetlinewidth{0.100000\du}
\pgfsetdash{}{0pt}
\pgfsetdash{}{0pt}
\pgfsetbuttcap
\pgfsetmiterjoin
\pgfsetlinewidth{0.100000\du}
\pgfsetbuttcap
\pgfsetmiterjoin
\pgfsetdash{}{0pt}
\definecolor{dialinecolor}{rgb}{1.000000, 1.000000, 1.000000}
\pgfsetfillcolor{dialinecolor}
\pgfpathellipse{\pgfpoint{14.115000\du}{12.515000\du}}{\pgfpoint{0.475000\du}{0\du}}{\pgfpoint{0\du}{0.475000\du}}
\pgfusepath{fill}
\definecolor{dialinecolor}{rgb}{0.000000, 0.000000, 0.000000}
\pgfsetstrokecolor{dialinecolor}
\pgfpathellipse{\pgfpoint{14.115000\du}{12.515000\du}}{\pgfpoint{0.475000\du}{0\du}}{\pgfpoint{0\du}{0.475000\du}}
\pgfusepath{stroke}
\pgfsetbuttcap
\pgfsetmiterjoin
\pgfsetdash{}{0pt}
\definecolor{dialinecolor}{rgb}{0.000000, 0.000000, 0.000000}
\pgfsetstrokecolor{dialinecolor}
\draw (14.115000\du,12.040000\du)--(14.115000\du,12.990000\du);
\pgfsetbuttcap
\pgfsetmiterjoin
\pgfsetdash{}{0pt}
\definecolor{dialinecolor}{rgb}{0.000000, 0.000000, 0.000000}
\pgfsetstrokecolor{dialinecolor}
\draw (13.640000\du,12.515000\du)--(14.590000\du,12.515000\du);
\pgfsetlinewidth{0.100000\du}
\pgfsetdash{}{0pt}
\pgfsetdash{}{0pt}
\pgfsetbuttcap
\pgfsetmiterjoin
\pgfsetlinewidth{0.100000\du}
\pgfsetbuttcap
\pgfsetmiterjoin
\pgfsetdash{}{0pt}
\definecolor{dialinecolor}{rgb}{1.000000, 1.000000, 1.000000}
\pgfsetfillcolor{dialinecolor}
\pgfpathmoveto{\pgfpoint{8.721429\du}{6.950000\du}}
\pgfpathlineto{\pgfpoint{9.578571\du}{6.950000\du}}
\pgfpathcurveto{\pgfpoint{9.707143\du}{7.170000\du}}{\pgfpoint{9.750000\du}{7.280000\du}}{\pgfpoint{9.750000\du}{7.500000\du}}
\pgfpathcurveto{\pgfpoint{9.750000\du}{7.720000\du}}{\pgfpoint{9.707143\du}{7.830000\du}}{\pgfpoint{9.578571\du}{8.050000\du}}
\pgfpathlineto{\pgfpoint{8.721429\du}{8.050000\du}}
\pgfpathcurveto{\pgfpoint{8.592857\du}{7.830000\du}}{\pgfpoint{8.550000\du}{7.720000\du}}{\pgfpoint{8.550000\du}{7.500000\du}}
\pgfpathcurveto{\pgfpoint{8.550000\du}{7.280000\du}}{\pgfpoint{8.592857\du}{7.170000\du}}{\pgfpoint{8.721429\du}{6.950000\du}}
\pgfusepath{fill}
\definecolor{dialinecolor}{rgb}{0.000000, 0.000000, 0.000000}
\pgfsetstrokecolor{dialinecolor}
\pgfpathmoveto{\pgfpoint{8.721429\du}{6.950000\du}}
\pgfpathlineto{\pgfpoint{9.578571\du}{6.950000\du}}
\pgfpathcurveto{\pgfpoint{9.707143\du}{7.170000\du}}{\pgfpoint{9.750000\du}{7.280000\du}}{\pgfpoint{9.750000\du}{7.500000\du}}
\pgfpathcurveto{\pgfpoint{9.750000\du}{7.720000\du}}{\pgfpoint{9.707143\du}{7.830000\du}}{\pgfpoint{9.578571\du}{8.050000\du}}
\pgfpathlineto{\pgfpoint{8.721429\du}{8.050000\du}}
\pgfpathcurveto{\pgfpoint{8.592857\du}{7.830000\du}}{\pgfpoint{8.550000\du}{7.720000\du}}{\pgfpoint{8.550000\du}{7.500000\du}}
\pgfpathcurveto{\pgfpoint{8.550000\du}{7.280000\du}}{\pgfpoint{8.592857\du}{7.170000\du}}{\pgfpoint{8.721429\du}{6.950000\du}}
\pgfusepath{stroke}
\pgfsetbuttcap
\pgfsetmiterjoin
\pgfsetdash{}{0pt}
\definecolor{dialinecolor}{rgb}{0.000000, 0.000000, 0.000000}
\pgfsetstrokecolor{dialinecolor}
\pgfpathmoveto{\pgfpoint{9.578571\du}{6.950000\du}}
\pgfpathcurveto{\pgfpoint{9.450000\du}{7.170000\du}}{\pgfpoint{9.407143\du}{7.280000\du}}{\pgfpoint{9.407143\du}{7.500000\du}}
\pgfpathcurveto{\pgfpoint{9.407143\du}{7.720000\du}}{\pgfpoint{9.450000\du}{7.830000\du}}{\pgfpoint{9.578571\du}{8.050000\du}}
\pgfusepath{stroke}
% setfont left to latex
\definecolor{dialinecolor}{rgb}{0.000000, 0.000000, 0.000000}
\pgfsetstrokecolor{dialinecolor}
\node at (9.064286\du,7.576553\du){};
\pgfsetlinewidth{0.100000\du}
\pgfsetdash{}{0pt}
\pgfsetdash{}{0pt}
\pgfsetbuttcap
\pgfsetmiterjoin
\pgfsetlinewidth{0.100000\du}
\pgfsetbuttcap
\pgfsetmiterjoin
\pgfsetdash{}{0pt}
\definecolor{dialinecolor}{rgb}{1.000000, 1.000000, 1.000000}
\pgfsetfillcolor{dialinecolor}
\pgfpathmoveto{\pgfpoint{6.861429\du}{12.040000\du}}
\pgfpathlineto{\pgfpoint{7.718571\du}{12.040000\du}}
\pgfpathcurveto{\pgfpoint{7.847143\du}{12.260000\du}}{\pgfpoint{7.890000\du}{12.370000\du}}{\pgfpoint{7.890000\du}{12.590000\du}}
\pgfpathcurveto{\pgfpoint{7.890000\du}{12.810000\du}}{\pgfpoint{7.847143\du}{12.920000\du}}{\pgfpoint{7.718571\du}{13.140000\du}}
\pgfpathlineto{\pgfpoint{6.861429\du}{13.140000\du}}
\pgfpathcurveto{\pgfpoint{6.732857\du}{12.920000\du}}{\pgfpoint{6.690000\du}{12.810000\du}}{\pgfpoint{6.690000\du}{12.590000\du}}
\pgfpathcurveto{\pgfpoint{6.690000\du}{12.370000\du}}{\pgfpoint{6.732857\du}{12.260000\du}}{\pgfpoint{6.861429\du}{12.040000\du}}
\pgfusepath{fill}
\definecolor{dialinecolor}{rgb}{0.000000, 0.000000, 0.000000}
\pgfsetstrokecolor{dialinecolor}
\pgfpathmoveto{\pgfpoint{6.861429\du}{12.040000\du}}
\pgfpathlineto{\pgfpoint{7.718571\du}{12.040000\du}}
\pgfpathcurveto{\pgfpoint{7.847143\du}{12.260000\du}}{\pgfpoint{7.890000\du}{12.370000\du}}{\pgfpoint{7.890000\du}{12.590000\du}}
\pgfpathcurveto{\pgfpoint{7.890000\du}{12.810000\du}}{\pgfpoint{7.847143\du}{12.920000\du}}{\pgfpoint{7.718571\du}{13.140000\du}}
\pgfpathlineto{\pgfpoint{6.861429\du}{13.140000\du}}
\pgfpathcurveto{\pgfpoint{6.732857\du}{12.920000\du}}{\pgfpoint{6.690000\du}{12.810000\du}}{\pgfpoint{6.690000\du}{12.590000\du}}
\pgfpathcurveto{\pgfpoint{6.690000\du}{12.370000\du}}{\pgfpoint{6.732857\du}{12.260000\du}}{\pgfpoint{6.861429\du}{12.040000\du}}
\pgfusepath{stroke}
\pgfsetbuttcap
\pgfsetmiterjoin
\pgfsetdash{}{0pt}
\definecolor{dialinecolor}{rgb}{0.000000, 0.000000, 0.000000}
\pgfsetstrokecolor{dialinecolor}
\pgfpathmoveto{\pgfpoint{7.718571\du}{12.040000\du}}
\pgfpathcurveto{\pgfpoint{7.590000\du}{12.260000\du}}{\pgfpoint{7.547143\du}{12.370000\du}}{\pgfpoint{7.547143\du}{12.590000\du}}
\pgfpathcurveto{\pgfpoint{7.547143\du}{12.810000\du}}{\pgfpoint{7.590000\du}{12.920000\du}}{\pgfpoint{7.718571\du}{13.140000\du}}
\pgfusepath{stroke}
% setfont left to latex
\definecolor{dialinecolor}{rgb}{0.000000, 0.000000, 0.000000}
\pgfsetstrokecolor{dialinecolor}
\node at (7.204286\du,12.666553\du){};
\end{tikzpicture}

\end{center}
\begin{multicols}{3}
\item % Graphic for TeX using PGF
% Title: /home/german/Diagrama2.dia
% Creator: Dia v0.97.3
% CreationDate: Wed Jul 15 17:18:27 2015
% For: german
% \usepackage{tikz}
% The following commands are not supported in PSTricks at present
% We define them conditionally, so when they are implemented,
% this pgf file will use them.
\ifx\du\undefined
  \newlength{\du}
\fi
\setlength{\du}{15\unitlength}
\begin{tikzpicture}
\pgftransformxscale{1.000000}
\pgftransformyscale{-1.000000}
\definecolor{dialinecolor}{rgb}{0.000000, 0.000000, 0.000000}
\pgfsetstrokecolor{dialinecolor}
\definecolor{dialinecolor}{rgb}{1.000000, 1.000000, 1.000000}
\pgfsetfillcolor{dialinecolor}
\pgfsetlinewidth{0.100000\du}
\pgfsetdash{}{0pt}
\pgfsetdash{}{0pt}
\pgfsetbuttcap
\pgfsetmiterjoin
\pgfsetlinewidth{0.100000\du}
\pgfsetbuttcap
\pgfsetmiterjoin
\pgfsetdash{}{0pt}
\definecolor{dialinecolor}{rgb}{1.000000, 1.000000, 1.000000}
\pgfsetfillcolor{dialinecolor}
\pgfpathellipse{\pgfpoint{3.765000\du}{2.715000\du}}{\pgfpoint{0.475000\du}{0\du}}{\pgfpoint{0\du}{0.475000\du}}
\pgfusepath{fill}
\definecolor{dialinecolor}{rgb}{0.000000, 0.000000, 0.000000}
\pgfsetstrokecolor{dialinecolor}
\pgfpathellipse{\pgfpoint{3.765000\du}{2.715000\du}}{\pgfpoint{0.475000\du}{0\du}}{\pgfpoint{0\du}{0.475000\du}}
\pgfusepath{stroke}
\pgfsetbuttcap
\pgfsetmiterjoin
\pgfsetdash{}{0pt}
\definecolor{dialinecolor}{rgb}{0.000000, 0.000000, 0.000000}
\pgfsetstrokecolor{dialinecolor}
\draw (3.765000\du,2.240000\du)--(3.765000\du,3.190000\du);
\pgfsetbuttcap
\pgfsetmiterjoin
\pgfsetdash{}{0pt}
\definecolor{dialinecolor}{rgb}{0.000000, 0.000000, 0.000000}
\pgfsetstrokecolor{dialinecolor}
\draw (3.290000\du,2.715000\du)--(4.240000\du,2.715000\du);
\end{tikzpicture}
 =
\item \input{Tikz/Diagrama4} =
\item % Graphic for TeX using PGF
% Title: /home/german/Diagrama2.dia
% Creator: Dia v0.97.3
% CreationDate: Wed Jul 15 17:23:45 2015
% For: german
% \usepackage{tikz}
% The following commands are not supported in PSTricks at present
% We define them conditionally, so when they are implemented,
% this pgf file will use them.
\ifx\du\undefined
  \newlength{\du}
\fi
\setlength{\du}{15\unitlength}
\begin{tikzpicture}
\pgftransformxscale{1.000000}
\pgftransformyscale{-1.000000}
\definecolor{dialinecolor}{rgb}{0.000000, 0.000000, 0.000000}
\pgfsetstrokecolor{dialinecolor}
\definecolor{dialinecolor}{rgb}{1.000000, 1.000000, 1.000000}
\pgfsetfillcolor{dialinecolor}
\pgfsetlinewidth{0.100000\du}
\pgfsetdash{}{0pt}
\pgfsetdash{}{0pt}
\pgfsetbuttcap
\pgfsetmiterjoin
\pgfsetlinewidth{0.100000\du}
\pgfsetbuttcap
\pgfsetmiterjoin
\pgfsetdash{}{0pt}
\definecolor{dialinecolor}{rgb}{1.000000, 1.000000, 1.000000}
\pgfsetfillcolor{dialinecolor}
\pgfpathmoveto{\pgfpoint{3.415000\du}{2.240000\du}}
\pgfpathlineto{\pgfpoint{3.915000\du}{2.240000\du}}
\pgfpathcurveto{\pgfpoint{3.984036\du}{2.240000\du}}{\pgfpoint{4.040000\du}{2.430279\du}}{\pgfpoint{4.040000\du}{2.665000\du}}
\pgfpathcurveto{\pgfpoint{4.040000\du}{2.899721\du}}{\pgfpoint{3.984036\du}{3.090000\du}}{\pgfpoint{3.915000\du}{3.090000\du}}
\pgfpathlineto{\pgfpoint{3.415000\du}{3.090000\du}}
\pgfpathcurveto{\pgfpoint{3.345964\du}{3.090000\du}}{\pgfpoint{3.290000\du}{2.899721\du}}{\pgfpoint{3.290000\du}{2.665000\du}}
\pgfpathcurveto{\pgfpoint{3.290000\du}{2.430279\du}}{\pgfpoint{3.345964\du}{2.240000\du}}{\pgfpoint{3.415000\du}{2.240000\du}}
\pgfusepath{fill}
\definecolor{dialinecolor}{rgb}{0.000000, 0.000000, 0.000000}
\pgfsetstrokecolor{dialinecolor}
\pgfpathmoveto{\pgfpoint{3.415000\du}{2.240000\du}}
\pgfpathlineto{\pgfpoint{3.915000\du}{2.240000\du}}
\pgfpathcurveto{\pgfpoint{3.984036\du}{2.240000\du}}{\pgfpoint{4.040000\du}{2.430279\du}}{\pgfpoint{4.040000\du}{2.665000\du}}
\pgfpathcurveto{\pgfpoint{4.040000\du}{2.899721\du}}{\pgfpoint{3.984036\du}{3.090000\du}}{\pgfpoint{3.915000\du}{3.090000\du}}
\pgfpathlineto{\pgfpoint{3.415000\du}{3.090000\du}}
\pgfpathcurveto{\pgfpoint{3.345964\du}{3.090000\du}}{\pgfpoint{3.290000\du}{2.899721\du}}{\pgfpoint{3.290000\du}{2.665000\du}}
\pgfpathcurveto{\pgfpoint{3.290000\du}{2.430279\du}}{\pgfpoint{3.345964\du}{2.240000\du}}{\pgfpoint{3.415000\du}{2.240000\du}}
\pgfusepath{stroke}
% setfont left to latex
\definecolor{dialinecolor}{rgb}{0.000000, 0.000000, 0.000000}
\pgfsetstrokecolor{dialinecolor}
\node at (3.665000\du,2.753194\du){};
\end{tikzpicture}
 =
\item \input{Tikz/Diagrama6} =
\item \input{Tikz/Diagrama7} =
\item % Graphic for TeX using PGF
% Title: /home/german/Diagrama2.dia
% Creator: Dia v0.97.3
% CreationDate: Wed Jul 15 17:28:12 2015
% For: german
% \usepackage{tikz}
% The following commands are not supported in PSTricks at present
% We define them conditionally, so when they are implemented,
% this pgf file will use them.
\ifx\du\undefined
  \newlength{\du}
\fi
\setlength{\du}{15\unitlength}
\begin{tikzpicture}
\pgftransformxscale{1.000000}
\pgftransformyscale{-1.000000}
\definecolor{dialinecolor}{rgb}{0.000000, 0.000000, 0.000000}
\pgfsetstrokecolor{dialinecolor}
\definecolor{dialinecolor}{rgb}{1.000000, 1.000000, 1.000000}
\pgfsetfillcolor{dialinecolor}
\pgfsetlinewidth{0.100000\du}
\pgfsetdash{}{0pt}
\pgfsetdash{}{0pt}
\pgfsetbuttcap
\pgfsetmiterjoin
\pgfsetlinewidth{0.100000\du}
\pgfsetbuttcap
\pgfsetmiterjoin
\pgfsetdash{}{0pt}
\definecolor{dialinecolor}{rgb}{1.000000, 1.000000, 1.000000}
\pgfsetfillcolor{dialinecolor}
\fill (3.893508\du,2.254854\du)--(4.468508\du,2.729854\du)--(3.893508\du,3.204854\du)--(3.318508\du,2.729854\du)--cycle;
\definecolor{dialinecolor}{rgb}{0.000000, 0.000000, 0.000000}
\pgfsetstrokecolor{dialinecolor}
\draw (3.893508\du,2.254854\du)--(4.468508\du,2.729854\du)--(3.893508\du,3.204854\du)--(3.318508\du,2.729854\du)--cycle;
\pgfsetbuttcap
\pgfsetmiterjoin
\pgfsetdash{}{0pt}
\definecolor{dialinecolor}{rgb}{0.000000, 0.000000, 0.000000}
\pgfsetstrokecolor{dialinecolor}
\draw (3.318508\du,2.729854\du)--(4.468508\du,2.729854\du);
\end{tikzpicture}
 =
\item % Graphic for TeX using PGF
% Title: /home/german/Diagrama2.dia
% Creator: Dia v0.97.3
% CreationDate: Wed Jul 15 17:29:08 2015
% For: german
% \usepackage{tikz}
% The following commands are not supported in PSTricks at present
% We define them conditionally, so when they are implemented,
% this pgf file will use them.
\ifx\du\undefined
  \newlength{\du}
\fi
\setlength{\du}{15\unitlength}
\begin{tikzpicture}
\pgftransformxscale{1.000000}
\pgftransformyscale{-1.000000}
\definecolor{dialinecolor}{rgb}{0.000000, 0.000000, 0.000000}
\pgfsetstrokecolor{dialinecolor}
\definecolor{dialinecolor}{rgb}{1.000000, 1.000000, 1.000000}
\pgfsetfillcolor{dialinecolor}
\pgfsetlinewidth{0.100000\du}
\pgfsetdash{}{0pt}
\pgfsetdash{}{0pt}
\pgfsetbuttcap
\pgfsetmiterjoin
\pgfsetlinewidth{0.100000\du}
\pgfsetbuttcap
\pgfsetmiterjoin
\pgfsetdash{}{0pt}
\definecolor{dialinecolor}{rgb}{1.000000, 1.000000, 1.000000}
\pgfsetfillcolor{dialinecolor}
\pgfpathmoveto{\pgfpoint{3.461429\du}{2.240000\du}}
\pgfpathlineto{\pgfpoint{4.318571\du}{2.240000\du}}
\pgfpathcurveto{\pgfpoint{4.447143\du}{2.460000\du}}{\pgfpoint{4.490000\du}{2.570000\du}}{\pgfpoint{4.490000\du}{2.790000\du}}
\pgfpathcurveto{\pgfpoint{4.490000\du}{3.010000\du}}{\pgfpoint{4.447143\du}{3.120000\du}}{\pgfpoint{4.318571\du}{3.340000\du}}
\pgfpathlineto{\pgfpoint{3.461429\du}{3.340000\du}}
\pgfpathcurveto{\pgfpoint{3.332857\du}{3.120000\du}}{\pgfpoint{3.290000\du}{3.010000\du}}{\pgfpoint{3.290000\du}{2.790000\du}}
\pgfpathcurveto{\pgfpoint{3.290000\du}{2.570000\du}}{\pgfpoint{3.332857\du}{2.460000\du}}{\pgfpoint{3.461429\du}{2.240000\du}}
\pgfusepath{fill}
\definecolor{dialinecolor}{rgb}{0.000000, 0.000000, 0.000000}
\pgfsetstrokecolor{dialinecolor}
\pgfpathmoveto{\pgfpoint{3.461429\du}{2.240000\du}}
\pgfpathlineto{\pgfpoint{4.318571\du}{2.240000\du}}
\pgfpathcurveto{\pgfpoint{4.447143\du}{2.460000\du}}{\pgfpoint{4.490000\du}{2.570000\du}}{\pgfpoint{4.490000\du}{2.790000\du}}
\pgfpathcurveto{\pgfpoint{4.490000\du}{3.010000\du}}{\pgfpoint{4.447143\du}{3.120000\du}}{\pgfpoint{4.318571\du}{3.340000\du}}
\pgfpathlineto{\pgfpoint{3.461429\du}{3.340000\du}}
\pgfpathcurveto{\pgfpoint{3.332857\du}{3.120000\du}}{\pgfpoint{3.290000\du}{3.010000\du}}{\pgfpoint{3.290000\du}{2.790000\du}}
\pgfpathcurveto{\pgfpoint{3.290000\du}{2.570000\du}}{\pgfpoint{3.332857\du}{2.460000\du}}{\pgfpoint{3.461429\du}{2.240000\du}}
\pgfusepath{stroke}
\pgfsetbuttcap
\pgfsetmiterjoin
\pgfsetdash{}{0pt}
\definecolor{dialinecolor}{rgb}{0.000000, 0.000000, 0.000000}
\pgfsetstrokecolor{dialinecolor}
\pgfpathmoveto{\pgfpoint{4.318571\du}{2.240000\du}}
\pgfpathcurveto{\pgfpoint{4.190000\du}{2.460000\du}}{\pgfpoint{4.147143\du}{2.570000\du}}{\pgfpoint{4.147143\du}{2.790000\du}}
\pgfpathcurveto{\pgfpoint{4.147143\du}{3.010000\du}}{\pgfpoint{4.190000\du}{3.120000\du}}{\pgfpoint{4.318571\du}{3.340000\du}}
\pgfusepath{stroke}
% setfont left to latex
\definecolor{dialinecolor}{rgb}{0.000000, 0.000000, 0.000000}
\pgfsetstrokecolor{dialinecolor}
\node at (3.804286\du,2.866553\du){};
\end{tikzpicture}
 =
\item % Graphic for TeX using PGF
% Title: /home/german/Diagrama2.dia
% Creator: Dia v0.97.3
% CreationDate: Wed Jul 15 17:30:08 2015
% For: german
% \usepackage{tikz}
% The following commands are not supported in PSTricks at present
% We define them conditionally, so when they are implemented,
% this pgf file will use them.
\ifx\du\undefined
  \newlength{\du}
\fi
\setlength{\du}{15\unitlength}
\begin{tikzpicture}
\pgftransformxscale{1.000000}
\pgftransformyscale{-1.000000}
\definecolor{dialinecolor}{rgb}{0.000000, 0.000000, 0.000000}
\pgfsetstrokecolor{dialinecolor}
\definecolor{dialinecolor}{rgb}{1.000000, 1.000000, 1.000000}
\pgfsetfillcolor{dialinecolor}
\pgfsetlinewidth{0.100000\du}
\pgfsetdash{}{0pt}
\pgfsetdash{}{0pt}
\pgfsetbuttcap
\pgfsetmiterjoin
\pgfsetlinewidth{0.100000\du}
\pgfsetbuttcap
\pgfsetmiterjoin
\pgfsetdash{}{0pt}
\definecolor{dialinecolor}{rgb}{1.000000, 1.000000, 1.000000}
\pgfsetfillcolor{dialinecolor}
\pgfpathmoveto{\pgfpoint{3.290000\du}{2.457593\du}}
\pgfpathcurveto{\pgfpoint{3.560000\du}{2.294398\du}}{\pgfpoint{3.695000\du}{2.240000\du}}{\pgfpoint{3.965000\du}{2.240000\du}}
\pgfpathcurveto{\pgfpoint{4.235000\du}{2.240000\du}}{\pgfpoint{4.370000\du}{2.294398\du}}{\pgfpoint{4.640000\du}{2.457593\du}}
\pgfpathlineto{\pgfpoint{4.640000\du}{3.327963\du}}
\pgfpathcurveto{\pgfpoint{4.370000\du}{3.491157\du}}{\pgfpoint{4.235000\du}{3.545556\du}}{\pgfpoint{3.965000\du}{3.545556\du}}
\pgfpathcurveto{\pgfpoint{3.695000\du}{3.545556\du}}{\pgfpoint{3.560000\du}{3.491157\du}}{\pgfpoint{3.290000\du}{3.327963\du}}
\pgfpathlineto{\pgfpoint{3.290000\du}{2.457593\du}}
\pgfusepath{fill}
\definecolor{dialinecolor}{rgb}{0.000000, 0.000000, 0.000000}
\pgfsetstrokecolor{dialinecolor}
\pgfpathmoveto{\pgfpoint{3.290000\du}{2.457593\du}}
\pgfpathcurveto{\pgfpoint{3.560000\du}{2.294398\du}}{\pgfpoint{3.695000\du}{2.240000\du}}{\pgfpoint{3.965000\du}{2.240000\du}}
\pgfpathcurveto{\pgfpoint{4.235000\du}{2.240000\du}}{\pgfpoint{4.370000\du}{2.294398\du}}{\pgfpoint{4.640000\du}{2.457593\du}}
\pgfpathlineto{\pgfpoint{4.640000\du}{3.327963\du}}
\pgfpathcurveto{\pgfpoint{4.370000\du}{3.491157\du}}{\pgfpoint{4.235000\du}{3.545556\du}}{\pgfpoint{3.965000\du}{3.545556\du}}
\pgfpathcurveto{\pgfpoint{3.695000\du}{3.545556\du}}{\pgfpoint{3.560000\du}{3.491157\du}}{\pgfpoint{3.290000\du}{3.327963\du}}
\pgfpathlineto{\pgfpoint{3.290000\du}{2.457593\du}}
\pgfusepath{stroke}
\pgfsetbuttcap
\pgfsetmiterjoin
\pgfsetdash{}{0pt}
\definecolor{dialinecolor}{rgb}{0.000000, 0.000000, 0.000000}
\pgfsetstrokecolor{dialinecolor}
\pgfpathmoveto{\pgfpoint{3.290000\du}{2.457593\du}}
\pgfpathcurveto{\pgfpoint{3.560000\du}{2.620787\du}}{\pgfpoint{3.695000\du}{2.675185\du}}{\pgfpoint{3.965000\du}{2.675185\du}}
\pgfpathcurveto{\pgfpoint{4.235000\du}{2.675185\du}}{\pgfpoint{4.370000\du}{2.620787\du}}{\pgfpoint{4.640000\du}{2.457593\du}}
\pgfusepath{stroke}
% setfont left to latex
\definecolor{dialinecolor}{rgb}{0.000000, 0.000000, 0.000000}
\pgfsetstrokecolor{dialinecolor}
\node at (3.965000\du,3.089769\du){};
\end{tikzpicture}
 =
\item % Graphic for TeX using PGF
% Title: /home/german/Diagrama2.dia
% Creator: Dia v0.97.3
% CreationDate: Wed Jul 15 17:31:29 2015
% For: german
% \usepackage{tikz}
% The following commands are not supported in PSTricks at present
% We define them conditionally, so when they are implemented,
% this pgf file will use them.
\ifx\du\undefined
  \newlength{\du}
\fi
\setlength{\du}{15\unitlength}
\begin{tikzpicture}
\pgftransformxscale{1.000000}
\pgftransformyscale{-1.000000}
\definecolor{dialinecolor}{rgb}{0.000000, 0.000000, 0.000000}
\pgfsetstrokecolor{dialinecolor}
\definecolor{dialinecolor}{rgb}{1.000000, 1.000000, 1.000000}
\pgfsetfillcolor{dialinecolor}
\pgfsetlinewidth{0.100000\du}
\pgfsetdash{}{0pt}
\pgfsetdash{}{0pt}
\pgfsetbuttcap
\pgfsetmiterjoin
\pgfsetlinewidth{0.100000\du}
\pgfsetbuttcap
\pgfsetmiterjoin
\pgfsetdash{}{0pt}
\definecolor{dialinecolor}{rgb}{1.000000, 1.000000, 1.000000}
\pgfsetfillcolor{dialinecolor}
\pgfpathmoveto{\pgfpoint{3.290000\du}{2.362396\du}}
\pgfpathcurveto{\pgfpoint{3.470000\du}{2.525590\du}}{\pgfpoint{3.560000\du}{2.525590\du}}{\pgfpoint{3.740000\du}{2.362396\du}}
\pgfpathcurveto{\pgfpoint{3.920000\du}{2.199201\du}}{\pgfpoint{4.010000\du}{2.199201\du}}{\pgfpoint{4.190000\du}{2.362396\du}}
\pgfpathlineto{\pgfpoint{4.190000\du}{3.341562\du}}
\pgfpathcurveto{\pgfpoint{4.010000\du}{3.178368\du}}{\pgfpoint{3.920000\du}{3.178368\du}}{\pgfpoint{3.740000\du}{3.341562\du}}
\pgfpathcurveto{\pgfpoint{3.560000\du}{3.504757\du}}{\pgfpoint{3.470000\du}{3.504757\du}}{\pgfpoint{3.290000\du}{3.341562\du}}
\pgfpathlineto{\pgfpoint{3.290000\du}{2.362396\du}}
\pgfusepath{fill}
\definecolor{dialinecolor}{rgb}{0.000000, 0.000000, 0.000000}
\pgfsetstrokecolor{dialinecolor}
\pgfpathmoveto{\pgfpoint{3.290000\du}{2.362396\du}}
\pgfpathcurveto{\pgfpoint{3.470000\du}{2.525590\du}}{\pgfpoint{3.560000\du}{2.525590\du}}{\pgfpoint{3.740000\du}{2.362396\du}}
\pgfpathcurveto{\pgfpoint{3.920000\du}{2.199201\du}}{\pgfpoint{4.010000\du}{2.199201\du}}{\pgfpoint{4.190000\du}{2.362396\du}}
\pgfpathlineto{\pgfpoint{4.190000\du}{3.341562\du}}
\pgfpathcurveto{\pgfpoint{4.010000\du}{3.178368\du}}{\pgfpoint{3.920000\du}{3.178368\du}}{\pgfpoint{3.740000\du}{3.341562\du}}
\pgfpathcurveto{\pgfpoint{3.560000\du}{3.504757\du}}{\pgfpoint{3.470000\du}{3.504757\du}}{\pgfpoint{3.290000\du}{3.341562\du}}
\pgfpathlineto{\pgfpoint{3.290000\du}{2.362396\du}}
\pgfusepath{stroke}
% setfont left to latex
\definecolor{dialinecolor}{rgb}{0.000000, 0.000000, 0.000000}
\pgfsetstrokecolor{dialinecolor}
\node at (3.740000\du,2.940174\du){};
\end{tikzpicture}
 =
\end{multicols}
 \end{enumerate}
 \begin{center}
\begin{tikzpicture}
 \fill (1,0) node[above]{1} circle (0.2ex);
 \fill (2,0) node[above]{2} circle (0.2ex);
 \fill (3,0) node[above]{3} circle (0.2ex);
 \fill (4,0) node[above]{4} circle (0.2ex);
 \fill (5,0) node[above]{5} circle (0.2ex);
 \fill (6,0) node[above]{6} circle (0.2ex);
 \fill (7,0) node[above]{7} circle (0.2ex);
 \fill (8,0) node[above]{8} circle (0.2ex);
 \fill (9,0) node[above]{9} circle (0.2ex);
 \fill (10,0) node[above]{10} circle (0.2ex);
 \fill (1,-1) node[left]{11} circle (0.2ex);
 \fill (2,-1) circle (0.2ex);
 \fill (3,-1) circle (0.2ex);
 \fill (4,-1) circle (0.2ex);
 \fill (5,-1) circle (0.2ex);
 \fill (6,-1) circle (0.2ex);
 \fill (7,-1) circle (0.2ex);
 \fill (8,-1) circle (0.2ex);
 \fill (9,-1) circle (0.2ex);
 \fill (10,-1) circle (0.2ex);
 \fill (1,-2) node[left]{21} circle (0.2ex);
 \fill (2,-2) circle (0.2ex);
 \fill (3,-2) circle (0.2ex);
 \fill (4,-2) circle (0.2ex);
 \fill (5,-2) circle (0.2ex);
 \fill (6,-2) circle (0.2ex);
 \fill (7,-2) circle (0.2ex);
 \fill (8,-2) circle (0.2ex);
 \fill (9,-2) circle (0.2ex);
 \fill (10,-2) circle (0.2ex);
 \fill (1,-3) node[left]{31} circle (0.2ex);
 \fill (2,-3) circle (0.2ex);
 \fill (3,-3) circle (0.2ex);
 \fill (4,-3) circle (0.2ex);
 \fill (5,-3) circle (0.2ex);
 \fill (6,-3) circle (0.2ex);
 \fill (7,-3) circle (0.2ex);
 \fill (8,-3) circle (0.2ex);
 \fill (9,-3) circle (0.2ex);
 \fill (10,-3) circle (0.2ex);
 \fill (1,-4) node[left]{41} circle (0.2ex);
 \fill (2,-4) circle (0.2ex);
 \fill (3,-4) circle (0.2ex);
 \fill (4,-4) circle (0.2ex);
 \fill (5,-4) circle (0.2ex);
 \fill (6,-4) circle (0.2ex);
 \fill (7,-4) circle (0.2ex);
 \fill (8,-4) circle (0.2ex);
 \fill (9,-4) circle (0.2ex);
 \fill (10,-4) node[right]{50} circle (0.2ex);
 \fill (1,-5) node[left]{51} circle (0.2ex);
 \fill (2,-5) circle (0.2ex);
 \fill (3,-5) circle (0.2ex);
 \fill (4,-5) circle (0.2ex);
 \fill (5,-5) circle (0.2ex);
 \fill (6,-5) circle (0.2ex);
 \fill (7,-5) circle (0.2ex);
 \fill (8,-5) circle (0.2ex);
 \fill (9,-5) circle (0.2ex);
 \fill (10,-5) circle (0.2ex);
 \fill (1,-6) node[left]{61} circle (0.2ex);
 \fill (2,-6) circle (0.2ex);
 \fill (3,-6) circle (0.2ex);
 \fill (4,-6) circle (0.2ex);
 \fill (5,-6) circle (0.2ex);
 \fill (6,-6) circle (0.2ex);
 \fill (7,-6) circle (0.2ex);
 \fill (8,-6) circle (0.2ex);
 \fill (9,-6) circle (0.2ex);
 \fill (10,-6) circle (0.2ex);
 \fill (1,-7) node[left]{71} circle (0.2ex);
 \fill (2,-7) circle (0.2ex);
 \fill (3,-7) circle (0.2ex);
 \fill (4,-7) circle (0.2ex);
 \fill (5,-7) circle (0.2ex);
 \fill (6,-7) circle (0.2ex);
 \fill (7,-7) circle (0.2ex);
 \fill (8,-7) circle (0.2ex);
 \fill (9,-7) circle (0.2ex);
 \fill (10,-7) circle (0.2ex);
 \fill (1,-8) node[left]{81} circle (0.2ex);
 \fill (2,-8) circle (0.2ex);
 \fill (3,-8) circle (0.2ex);
 \fill (4,-8) circle (0.2ex);
 \fill (5,-8) circle (0.2ex);
 \fill (6,-8) circle (0.2ex);
 \fill (7,-8) circle (0.2ex);
 \fill (8,-8) circle (0.2ex);
 \fill (9,-8) circle (0.2ex);
 \fill (10,-8) circle (0.2ex);
 \fill (1,-9) node[left]{91} circle (0.2ex);
 \fill (2,-9) circle (0.2ex);
 \fill (3,-9) circle (0.2ex);
 \fill (4,-9) circle (0.2ex);
 \fill (5,-9) circle (0.2ex);
 \fill (6,-9) circle (0.2ex);
 \fill (7,-9) circle (0.2ex);
 \fill (8,-9) circle (0.2ex);
 \fill (9,-9) circle (0.2ex);
 \fill (10,-9) node[right]{100} circle (0.2ex);
 %Solucion
 \draw (8,-5)--(7,-6)--(7,-3)--(6,-4)--(6,-5)--(8,-7)--(7,-8)--(6,-8)--(6,-6)--(7,-7)--(9,-7)--(9,-2)--(8,-1)--(7,-1)--(6,0)--(6,-1)--(5,-1)--(4,-2)--(3,-1)--(3,-2)--(2,-4)--(2,-6)--(3,-7)--(3,-8)--(6,-9)--(8,-9)--(10,-7)--(10,-6)--(9,-5)--(8,-3)--(8,-5)--(9,-6)--(8,-6)--(8,-7);
\end{tikzpicture}
\end{center}
\end{document}
