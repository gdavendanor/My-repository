\documentclass[fleqn]{article}
\usepackage[spanish,es-noshorthands]{babel}
\usepackage[utf8]{inputenc} 
\usepackage[papersize={6.5in,8.5in},left=1cm, right=1cm, top=1.5cm, bottom=1.7cm]{geometry}
\usepackage{mathexam}
\usepackage{amsmath}
\usepackage{graphicx}
\usepackage{multicol}
\usepackage{tikz,pgf}
\newcount\segmentsleft
\tikzset{pics/.cd,
  circle fraction/.style args={#1/#2}{code={%
\segmentsleft=#1\relax
\pgfmathloop
\ifnum\segmentsleft<1\else
\ifnum\segmentsleft<#2 \edef\n{\the\segmentsleft}\else\def\n{#2}\fi
\begin{scope}[shift={(\pgfmathcounter,0)}]
\foreach \i [evaluate={\a=360/#2*(\i-1)+90;}] in {1,...,\n}
  \fill[fill=gray] (0,0) -- (\a:3/8) arc (\a:\a+360/#2:3/8) -- cycle;
\draw circle [radius=3/8];
\ifnum#2>1
  \foreach \i [evaluate={\a=360/#2*(\i-1);}] in {1,...,#2}
    \draw (0,0) -- (90+\a:3/8);
\fi
\end{scope}
\advance\segmentsleft by-#2
\repeatpgfmathloop
  }}
}

\ExamClass{\includegraphics[height=16pt]{Images/logo-sed.png} Matemáticas $7^{\circ}$}
\ExamName{`Números fraccionarios'}
\ExamHead{\includegraphics[height=16pt]{Images/logo-colegio.png} IEDAB}
\newcommand{\LineaNombre}{%
\par
\vspace{\baselineskip}
Nombre:\hrulefill \; Curso: \underline{\hspace*{48pt}} \; Fecha: \underline{\hspace*{2.5cm}} \relax
\par}
\let\ds\displaystyle

\begin{document}
\ExamInstrBox{
Respuesta sin justificar mediante procedimiento no será tenida en cuenta en la calificación. Escriba sus respuestas en el espacio indicado. Tiene 45 minutos para contestar esta prueba.}
\LineaNombre
\begin{enumerate}
 \item En las siguientes fracciones, distinga cuál es el numerador y cuál es el denominador y ubíquelas sobre la recta numérica.
 \begin{enumerate}
 \begin{multicols}{2}
 \item $\dfrac{2}{5}$
 \item $\dfrac{9}{5}$ 
 \end{multicols}
 \end{enumerate}
 \begin{tikzpicture}[scale=3]
\draw[<->] (-1,0)--(2.5,0);
\foreach \x in {0,1,2} \draw[shift={(\x,0)},color=black] (0pt,2pt) -- (0pt,-2pt) node[below] {\footnotesize $\x$};
\foreach \x in {-.8,-.6,-.4,-.2,0.2,.4,.6,.8,1.2,1.4,1.6,1.8,2.2,2.4} \draw[shift={(\x,0)},color=black] (0pt,1pt) -- (0pt,-1pt);
\end{tikzpicture}
 \item Determine la fracción correspondiente al área sombreada en cada dibujo.
 
\begin{tikzpicture}
\foreach \numerator/\denominator [count=\y] 
  in {1/1, 1/3, 2/4, 3/5, 8/8}{
  %\node at (-1/2,-\y) {$\frac{\numerator}{\denominator}$};
  \pic  at (0, -\y) {circle fraction={\numerator/\denominator}};
}
\foreach \numerator/\denominator [count=\y] 
  in {4/1, 10/3, 20/6, 30/7, 40/15}{
  %\node at (-1/2,-\y) {$\frac{\numerator}{\denominator}$};
  \pic  at (5, -\y) {circle fraction={\numerator/\denominator}};
}
\end{tikzpicture} 
 \item Calcule:
 \begin{enumerate}
 \item $\frac{3}{5}$ de 90 \noanswer
 \item $\frac{5}{8}$ de 72 \noanswer
 \item $\frac{7}{9}$ de 63 \noanswer
 \end{enumerate}
 \newpage
 \item Determine si cada una de las siguientes parejas de fracciones son equivalentes entre sí
 \begin{enumerate}
 \item $\dfrac{2}{7}$ y $\dfrac{26}{91}$ \noanswer[.5in]
 \item $\dfrac{4}{5}$ y $\dfrac{48}{60}$ \noanswer[.5in]
 \item $\dfrac{7}{13}$ y $\dfrac{70}{135}$ \noanswer[.5in]
 \end{enumerate}
 \item Para elaborar una torta se han utilizado 600 gr de harina, 300 gramos de zanahoria, 250 gr de mantequilla, 150 gr de azúcar y 100 gr de leche. ¿Qué fracción del total representan cada uno de estos ingredientes?\noanswer
 \end{enumerate}

\end{document}
