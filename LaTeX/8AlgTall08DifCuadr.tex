\documentclass[10pt,twoside]{article}
\usepackage[utf8]{inputenc}
\usepackage{amsmath}
\usepackage{amsfonts}
\usepackage{amssymb}
\usepackage[spanish,es-noshorthands]{babel}
\usepackage[T1]{fontenc}
\usepackage{lmodern}
\usepackage{graphicx,hyperref}
\usepackage{tikz,pgf}
\usepackage{multicol}
\usepackage{subfig}
\usepackage[papersize={6.5in,8.5in},width=5.5in,height=7in]{geometry}
\usepackage{fancyhdr}
\pagestyle{fancy}
\fancyhead[LE]{\includegraphics[height=12pt]{Images/logo-colegio.png} Álgebra $8^{\circ}$}
\fancyhead[RE]{}
\fancyhead[RO]{\textit{Germ\'an Avenda\~no Ram\'irez, Lic. U.D., M.Sc. U.N.}}
\fancyhead[LO]{}

\author{Germ\'an Avenda\~no Ram\'irez, Lic. U.D., M.Sc. U.N.}
\title{\begin{minipage}{.2\textwidth}
\includegraphics[height=1.75cm]{Images/logo-colegio.png}\end{minipage}
\begin{minipage}{.55\textwidth}
\begin{center}
Taller 08, Diferencia de cuadrados y cubos\\
Álgebra $8^{\circ}$
\end{center}
\end{minipage}\hfill
\begin{minipage}{.2\textwidth}
\includegraphics[height=1.75cm]{Images/logo-sed.png} 
\end{minipage}}
\date{}
\begin{document}
\maketitle
Nombre: \hrulefill Curso: \underline{\hspace*{44pt}} Fecha: \underline{\hspace*{2.5cm}}\\

A continuaci\'{o}n se explican dos casos de factorizaci\'{o}n a abordar en este taller.
\subsection*{Diferencia de cuadrados}
Se presenta como su nombre lo indica cuando existe una diferencia entre dos cantidades o expresiones que son cuadrados perfectos y se factoriza seg\'{u}n el siguiente patr\'{o}n:
\[a^{2}-b^{2}=(a-b)(a+b)\]
Siempre que se tenga una diferencia de cuadrados perfectos, se factoriza como una suma por una diferencia de sus ra\'{i}ces.
\subsubsection*{Ejemplo 1}
Factorizar $x^{2}-16$

Se observa que tanto $x^{2}$ como 16 son cuadrados perfectos, ya que $x^{2}$ es el cuadrado de $x$ y 16 es el cuadrado de 4. Luego factorizamos as\'{i}:
\begin{align*}
x^{2}-16&=x^{2}-4^{2} & \mbox{Diferencia de cuadrados}\\
&=(x-4)(x+4) & \mbox{suma por diferencia}
\end{align*}
\subsubsection*{Ejemplo 2:}
Factorizar $4x^{2}-9y^{2}$

Nuevamente observamos que tanto 4 como $x^{2}$ son cuadrados perfectos, as\'{i} como 9 y $y^{2}$. M\'{a}s espec\'{i}ficamente podemos asumir que $4x^{2}$ es el cuadrado de $2x$ y que $9y^{2}$ es el cuadrado de $3y$. As\'{i} que factorizamos as\'{i}:
\begin{align*}
4x^{2}-9y^{2}&=2^{2}x^{2}-3^{2}y^{2} & \mbox{Cada término es cuadrado perfecto}\\
&=(2x)^{2}-(3y)^{2}& \mbox{Se expresa como Diferencia de cuadrados}\\
&=(2x-3y)(2x+3y) & \mbox{Se factoriza}
\end{align*}
A veces se debe factorizar completamente porque uno de los factores es a su vez una diferencia de cuadrados, como en los siguientes ejemplos
\subsubsection*{Ejemplo 3:} 
\[16x^{4}-81y^{4}\]
Se procede a factorizar como ya sabemos:
\begin{align*}
16x^{4}-81y^{4}&=4^{2}(x^{2})^{2}-9^{2}(y^{2})^{2} & \mbox{Los términos son C. P.}\\
&=(4x^{2})^{2}-(9y^{2}) & \mbox{Se expresa como diferencia de C.P.}\\
&=(4x^{2}-9y^{2})(4x^{2}+9y^{2}) & \mbox{El primer factor es una Dif. de C.P.}\\
&=\left((2x)^{2}-(3y)^{2}\right)(4x^{2}+9y^{2}) & \mbox{Se expresa el primer factor como una D. de C.P.}\\
&=(2x-3y)(2x+3y)(4x^{2}+9y^{2}) & \mbox{Se factoriza a su vez el 1er factor}
\end{align*}
\subsubsection*{Ejemplo 4:}
\[(x-1)^{2}-(x+4)^{2}\]
Claramente se observa una Dif. de C.P. Luego se procede así:
\begin{align*}
(x-1)^{2}-(x+4)^{2}&=\left((x-1)+(x+4)\right)\left((x-1)-(x+4)\right)\\
&=(x-1+x+4)(x-1-x-4) & \mbox{Destruyendo los paréntesis internos}\\
&=(2x+3)(-5) & \mbox{Reduciendo términos semejantes}\\
&=-5(2x+3)
\end{align*}
\subsubsection*{Ejemplo 5:}
\[48y^{3}-27y\]
Aquí no se observan claramente los C.P. Entonces debemos ver si primero podemos aplicar factor común. Evidentemente sí
\begin{align*}
48y^{3}-27y&=3y(16y^{2}-9) & \mbox{Aplicando Factor común}\\
&=3y(4y+3)(4y-3) & \mbox{Aplicando nuevamente Dif de C.}
\end{align*}
\section*{Quiz conceptual}
Para los siguientes enunciados escriba V o F según corresponda.
\begin{enumerate}
\item[a.] Un binomio que tiene dos cuadrados perfectos que se restan es una diferencia de cuadrados.
\item[b.] La suma de dos cuadrados es factorizable usando enteros.
\item[c.] La suma de dos cubos se puede factorizar usando enteros.
\item[d.] La diferencia de dos cuadrados es factorizable.
\item[e.] La diferencia de dos cubos es factorizable
\item[f.] Para factorizar es aconsejable inspeccionar que se pueda aplicar factor común en primera instancia.
\item[g.] El polinomio $4x^{2}+y^{2}$ se factoriza como $(2x+y)(2x+y)$
\item[h.] La factorización completa de $y^{4}-81$ es $(y^{2}+9)(y^{2}-9)$
\item[i.] La ecuación $x^{2}=-9$ no tiene soluciones reales.
\item[j.] La ecuación $abc=0$ si y sólo sí $a=0$ 
\end{enumerate}
\section*{Ejercicios}
Factorice usando el caso diferencia de cuadrados.
\begin{enumerate}
\begin{multicols}{2}
\item $x^{2}-9$
\item $4x^{2}-49$
\item $x^{2}-64y^{2}$
\item $x^{2}y^{2}-a^{2}b^{2}$
\item $x^{6}-9y^{2}$
\item $25-49n^{2}$
\item $(3x+5y)^{2}-y^{2}$
\item $x^{2}-(y-5)^{2}$
\item $16s^{2}-(3t+1)^{2}$
\item $(x-1)^{2}-(x-8)^{2}$
\end{multicols}
Factorice cada uno de los siguientes polinomios completamente. Indique cuáles no son factorizables usando coeficientes enteros. No olvide los casos vistos antes, como "factor común"
\begin{multicols}{2}
\item $8x^{2}-72$
\item $7x^{2}+28$
\item $5y^{2}-80$
\item $x^{3}y^{2}-xy^{2}$
\item $x^{4}-16$
\item $4x^{2}+9$
\item $20x^{3}+45x$
\item $12x^{3}-27xy^{2}$
\item $1-16x^{4}$
\item $20x-5x^{3}$
\item $9x^{2}-81y^{2}$
\item $2x^{5}-162x$
\end{multicols}
Para los siguientes ejercicios, use la suma o diferencia de cubos para factorizar.
\begin{multicols}{2}
\item $a^{3}-27$
\item $x^{3}+8$
\item $8x^{3}+27y^{3}$
\item $1-8x^{3}$
\item $125x^{3}+27y^{3}$
\item $x^{6}+y^{6}$
\end{multicols}
Para los problemas siguientes, encuentre todos los números reales que son solución de cada ecuación.
\begin{multicols}{2}
\item $x^{2}-1=0$
\item $4y^{2}=25$
\item $3x^{2}-108=0$
\item $4x^{3}=64x$
\item $54-6x^{2}=0$
\item $x^{5}-x=0$
\item $4x^{3}+12x=0$
\end{multicols}
Para los problemas siguientes, plantee una ecuación y soluciónela para resolver el problema.
\item El cubo de un número es igual a su cuadrado. Encuentre el número
\item La suma de las áreas de dos cuadrados es 26 $m^{2}$. El lado del cuadrado grande es cinco veces el lado del cuadrado pequeño. Encuentre las dimensiones de cada cuadrado.
\item Suponga que el largo de un rectángulo es $1\frac{1}{3}$ veces su ancho. El área del rectángulo es 48 $cm^{2}$. Encuentre el largo y ancho del rectángulo.
\item La superficie total de un cono circular recto es 108$\pi$ cm$^{2}$. Si la altura del cono es dos veces la longitud del radio de la base, encuentre la longitud del radio.
\item La altura de un triángulo es $\frac{1}{3}$ la longitud del lado sobre el que se dibuja la altura. Si el área del triángulo es 6 cm$^{2}$, encuentre su altura.
\end{enumerate}
\end{document}
