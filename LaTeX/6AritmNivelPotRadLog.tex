\documentclass[twosides]{article}
\usepackage[utf8]{inputenc}
\usepackage{amsmath,amsfonts,amssymb,amsthm,latexsym}
\usepackage[spanish,es-noshorthands]{babel}
\usepackage[T1]{fontenc}
\usepackage{lmodern}
\usepackage{graphicx,hyperref}
\usepackage{tikz}
\usepackage{multicol}
\usepackage{subfig}
\usepackage[papersize={5.5in,8.5in},width=7in,height=9in]{geometry}

\author{Germ\'an Avenda\~no Ram\'irez~\thanks{Lic. Mat. U.D.; M.Sc. U.N.}}
\title{\begin{minipage}{0.15\textwidth}\includegraphics[height=1.7cm]{Images/logo-colegio.png}
\end{minipage}\hfill \begin{minipage}{0.85\textwidth}\begin{center}
Potenciación, radicación y logaritmación\\Evaluación aritmética $6^{\circ}$\end{center}
\end{minipage}}
\date{}

\begin{document}
\maketitle
Nombre: \hrulefill Curso: \underline{\hspace{1cm}}  Fecha: \underline{\hspace{2cm}}\\
\begin{enumerate}
  \item Encuentre las 5 primeras potencias de los números
  \begin{enumerate}
    \item $ 2 $: \hrulefill
    \item $ 3 $: \hrulefill
    \item $ 4 $: \hrulefill
    \item $ 5 $: \hrulefill
  \end{enumerate}
  \item Resuelva cada una de las siguientes operaciones e identifique la base, el exponente y la potencia
  \begin{enumerate}\begin{multicols}{2}
    \item $ 5^3= $: \underline{\hspace{6cm}}
    \item $ 8^3= $: \hrulefill\end{multicols}
  \end{enumerate}
  \item Encuentre el número que falta en el paréntesis en cada una de las siguientes operaciones indicadas:
  \begin{enumerate}\begin{multicols}{2}
    \item $ 4^6= $[\underline{\hspace{1cm}}]
    \item $ [\hspace{0.5cm}]^3=125 $
    \item $ 6^{[\hspace{0.2cm}]}=216 $
    \item $ \sqrt[3]{512}=[\hspace{0.5cm}] $
  \end{multicols}
  \end{enumerate}
  \item Determine las siguientes raíces, descomponiendo cada número en sus factores primos y usando la siguiente propiedad:
  \[ \sqrt[n]{a\cdot b}=\sqrt[n]{a}\cdot\sqrt[n]{b} \]
  \textbf{Ejemplo:} $ \sqrt{144}=\sqrt{4^2\cdot3^2 }= \sqrt{4^2}\cdot\sqrt{3^2}=4\cdot3=12$\\
  Ya que al descomponer 144 en sus factores primos obtenemos\\
  \[ 144=2^4\cdot3^2=2^2\cdot2^2\cdot3^2=4^2\cdot3^2 \]
  \begin{enumerate}
    \item $ \sqrt{625}= $: \hrulefill
    \item $ \sqrt[3]{5832}= $: \hrulefill
  \end{enumerate}
    \item Escriba en forma de potencia cada una de las siguientes raíces:
    \begin{enumerate}\begin{multicols}{2}
      \item $ \sqrt[3]{343}=7 $ \underline{\hspace{3cm}}
      \item $ \sqrt{1000000}=1000 $ \underline{\hspace{3cm}}
\end{multicols}
   \end{enumerate}
  \item Escriba en forma de potencia cada uno de los siguientes logaritmos:
  \begin{enumerate}\begin{multicols}{2}
    \item $ \log_4{64}=3 $: \underline{\hspace{3cm}}
    \item $ \log_6{216}=3 $: \underline{\hspace{3cm}}
  \end{multicols}
  \end{enumerate}
\end{enumerate}
\end{document}
