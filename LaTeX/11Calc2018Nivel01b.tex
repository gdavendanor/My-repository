\documentclass[fleqn]{article}
\usepackage[spanish,es-noshorthands]{babel}
\usepackage[utf8]{inputenc} 
\usepackage[papersize={6.5in,8.5in},total={5.5in,7.25in},centering]{geometry}
\usepackage{mathexam}
\usepackage{amsmath,amsthm,amsfonts,amssymb}
\usepackage{graphicx}
\usepackage{tikz,pgf}
\usepackage{multicol}

\ExamClass{\includegraphics[height=16pt]{Images/logo-sed.png} Matemáticas $11^{\circ}$}
\ExamName{``Nivelación Per. 1, Form \textbf{B}''}
\ExamHead{\includegraphics[height=16pt]{Images/logo-colegio.png} IEDAB}
\newcommand{\LineaNombre}{%
\par
\vspace{\baselineskip}
Nombre:\hrulefill \; Curso: \underline{\hspace*{48pt}} \; Fecha: \underline{\hspace*{2.5cm}} \relax
\par}
\let\ds\displaystyle

\begin{document}
\ExamInstrBox{
\emph{No marque ni dañe esta hoja}. Escriba sus respuestas con procedimiento cuando éste se requiera, en una hoja. Respuesta sin justificar mediante procedimiento no será tenida en cuenta en la calificación. Tiene 45 minutos para contestar esta prueba.}
%\LineaNombre
\begin{enumerate}
   \item Determine la propiedad de los números reales que se ha usado:
  \begin{enumerate}
    \item $ (x+y)(p-q)=(p-q)(x+y) $
    \item $ (C+D)(x+y)=(C+D)x+(C+D)y $ 
  \end{enumerate}
  \item Exprese el intervalo \;$ [-3,5) $ \; como una desigualdad y luego grafíquela en una recta.
  
%    \begin{flushright}
%    \begin{tikzpicture}[scale=.8,>=stealth]
%\draw[<->] (0,0) -- (12,0);
%\draw [thick] (1,-.1) -- (1,0.1);
%\draw [thick] (2,-.1) -- (2,0.1);
%\draw [thick] (3,-.1) -- (3,0.1);
%\draw [thick] (4,-.1) -- (4,0.1);
%\draw [thick] (5,-.1) -- (5,0.1);
%\draw [thick] (6,-.1) -- (6,0.1);
%\draw [thick] (7,-.1) -- (7,0.1);
%\draw [thick] (8,-.1) -- (8,0.1);
%\draw [thick] (9,-.1) -- (9,0.1);
%\draw [thick] (10,-.1) -- (10,0.1);
%\draw [thick] (11,-.1) -- (11,0.1);
%\end{tikzpicture}
%    \end{flushright}\vspace{20pt}

  \item Exprese en notación de intervalos la desigualdad \;$ x\leq 3 $\; y luego grafique el correspondiente intervalo:
%      \begin{flushright}
%  \begin{tikzpicture}[scale=.8,>=stealth]
%\draw[<->] (0,0) -- (12,0);
%\draw [thick] (1,-.1) -- (1,0.1);
%\draw [thick] (2,-.1) -- (2,0.1);
%\draw [thick] (3,-.1) -- (3,0.1);
%\draw [thick] (4,-.1) -- (4,0.1);
%\draw [thick] (5,-.1) -- (5,0.1);
%\draw [thick] (6,-.1) -- (6,0.1);
%\draw [thick] (7,-.1) -- (7,0.1);
%\draw [thick] (8,-.1) -- (8,0.1);
%\draw [thick] (9,-.1) -- (9,0.1);
%\draw [thick] (10,-.1) -- (10,0.1);
%\draw [thick] (11,-.1) -- (11,0.1);
%\end{tikzpicture}
%  \end{flushright}
  \item Realice las operaciones indicadas, simplificando siempre que sea posible:
  \begin{enumerate}
  \begin{multicols}{2}
    \item $ 5+\dfrac{3}{5}-\dfrac{1}{6}= $ 
    \item $ 0.25\left(\dfrac{5}{7}+\dfrac{2}{3}\right)= $ 
    \item $ \left(\dfrac{3}{4}-\dfrac{2}{5}\right)\left(\dfrac{1}{5}-\dfrac{1}{4}\right)= $ 
    \item $ \dfrac{4-\frac{2}{5}}{\frac{1}{4}-\frac{1}{3}}= $   
  \end{multicols}
  \end{enumerate}
  \item Determine el orden en los siguientes pares de números, usando los símbolos $<$, $>$, o $=$ según corresponda:
  \begin{enumerate}\begin{multicols}{2}
    \item $ 6 $ , \qquad $ \dfrac{17}{3} $
    \item $ \dfrac{2}{3} $ , \qquad $ 0.66 $
    \item $ -4 $ , \qquad $ -\dfrac{15}{4} $
    \item $ |-0.75| $ , \qquad $ |0.75| $
    \end{multicols}
  \end{enumerate}
  \item Exprese como una desigualdad las siguientes expresiones:
  \begin{enumerate}
    \item $ q $ es menor que 5 y mayor o igual que $ -2 $ 
    \item $3x$ es negativo 
  \end{enumerate}
    \item En los ejercicios siguientes cuente el número de formas en que puede hacerse cada procedimiento.
  \begin{enumerate}
  \item Alinear a cuatro personas para una fotografía.
  \item Sentar a 9 personas en una banca en la que sólo hay 3 asientos disponibles
  \item ¿De cuántas maneras puede formarse de un grupo de 9 personas un comité de 4 personas?
  \end{enumerate}
   \section*{Preparándonos para la Prueba Saber}
\item Se desea adquirir un terreno de forma cuadrada con un perímetro entre 4 y 20 metros. Si $x$ representa el lado del terreno, los valores que puede tomar $x$ para que el
perímetro del terreno cumpla la condición dada son
\begin{enumerate}
\begin{multicols}{4}
\item $0<x<16$
\item $2<x<10$
\item $1<x<5$
\item $4<x<20$
\end{multicols}
\end{enumerate}
\begin{minipage}{.4\textwidth}
\item Andrea construyó una cometa con cuatro triángulos de papel que cortó de dos rectángulos con las medidas que se señalan en los dibujos
\end{minipage}
\begin{minipage}{.55\textwidth}
\begin{tikzpicture}[scale=.85]
\draw (0,0) rectangle (4,3);
\node[node font=\small]  at (1,2.5) {Triangulo 1};
\draw[dashed] (0,0) --node {50cm} (4,3);
\node[node font=\small]  at (2.75,.5) {Triangulo 2};
\draw[|-|](0,-.2)--node[below]{40cm}(4,-.2);
\draw[|-|] (4.2,0)--node[right]{30 cm}(4.2,3); 
\draw (6,0) rectangle (8,1.5);
\draw[dashed] (6,0)--node[node font=\small]{25cm}(8,1.5);
\node[node font=\tiny] at (6.7,1.2){Triang 3};
\node[node font=\tiny] at (7.2,.2){Triang 4};
\draw[|-|] (8.2,0)--node[right]{15 cm}(8.2,1.5);
\end{tikzpicture} 
\end{minipage}

\begin{minipage}{.5\textwidth}
\begin{tikzpicture}[scale=.85]
\draw (0,0) -- (6,0) --(3,4)node[above]{$K$}--cycle;
\draw (3,4)--(3,-2);
\draw (1.5,0)--(3,-2)node[below]{$S$}--(4.5,0);
\draw[dashed,|-|](3.1,4.1)--node[right]{50 cm}(6.1,0.1);
\node[node font=\small] (1) at (1.6,.8) {Triangulo 1};
\node[node font=\small] (2) at (4.2,.8) {Triangulo 2};
\node[node font=\tiny] (3) at (2.2,-.2) {Triang 3};
\node[node font=\tiny] (4) at (3.7,-.2) {Triang 4};
\draw[dashed,|-|](3,.15)--node[above,node font=\small]{15cm}(4.5,.15);
\end{tikzpicture}
\end{minipage}
\begin{minipage}{.45\textwidth}
La cometa armada tiene la forma anterior:
\end{minipage}
La distancia entre los puntos $K$ y $S$ es
\begin{enumerate}
\begin{multicols}{4}
\item 75 cm
\item 60 cm
\item 55 cm
\item 40 cm
\end{multicols}
\end{enumerate}
\end{enumerate}
\end{document}
