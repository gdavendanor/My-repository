\documentclass[fleqn]{article}
\usepackage[spanish,es-noshorthands]{babel}
\usepackage[utf8]{inputenc} 
\usepackage[left=1cm, right=1cm, top=1.5cm, bottom=1.7cm]{geometry}
\usepackage{mathexam}
\usepackage{amsmath}
\usepackage{graphicx}

\ExamClass{\includegraphics[height=16pt]{Images/logo-sed.png} Matemáticas $6^{\circ}$}
\ExamName{Evaluación, Adición, sustracción y multiplicación en $\mathbb{N}$}
\ExamHead{\includegraphics[heiconfirmed your subscription. You can always request a mail-back of your password when you edit your personal options. ght=16pt]{Images/logo-colegio.png} IEDAB}
\newcommand{\LineaNombre}{%
\par
\vspace{\baselineskip}
Nombre:\hrulefill \; Curso: \underline{603} \; Fecha: \underline{\hspace*{2.5cm}} \relax
\par}
\let\ds\displaystyle

\begin{document}
\ExamInstrBox{
Respuesta sin justificar mediante procedimiento no será tenida en cuenta en la calificación. Escriba sus respuestas en el espacio indicado. Tiene 45 minutos para contestar esta prueba.}
\LineaNombre
\begin{enumerate}
   \item Ordene los números usando $<$ o $>$ según el caso: 
      \begin{enumerate}
	 \item De menor a mayor los siguientes números: 4050, 4500, 4005, 4555, 40005\answer
	 \item De mayor a menor los siguientes números: 6040, 6400, 64000, 6004, 60400, 60404\answer
      \end{enumerate}
  \item Complete el siguiente cuadro, con los nombres de cada término involucrado según la operación:
  \begin{tabular}{|c|c|c|c|c|}
\hline 
Operación &  &  &  &  \\ 
\hline 
 & $13+15=28$ & 13 & 15 & 28 \\ 
\hline 
 & 30--12=18 & • & • & • \\ 
\hline 
• & • & • & • & • \\ 
\hline 
\end{tabular} 
   \item Busca el término desconocido e indica su nombre en las siguientes operaciones:
   \begin{enumerate}
   \item $329+\underline{\hspace*{48pt}}=1206$\noanswer
   \item $\underline{\hspace*{60pt}}-4208=524$\noanswer
   \item $324\times \underline{\hspace*{60pt}}=15552 $\noanswer
   \end{enumerate}
      \newpage
   \item If $h(x) = \sqrt{x^2 + 2} - 1$, find a \textbf{non-trivial} decomposition of $h$ into $f$ and $g$ such that $h = f\circ g$.
      \answer*{$f(x)=$}\addanswer*{$g(x)=$}
   \item Find the first two derivatives of the function $f(x) = x^2\cos(x)$.  Simplify
      your answers as much as possible.  Show all your work.
      \answer*{$f'(x)=$}\answer*{$f''(x)=$}
      \newpage
   \item Find the derivative of the function $\ds{f(x) = \int_{x^2}^2
      \frac{\cos(t)}{t} \,dt}$.\answer[1in plus 1fill]
   \item Set up, but do not evaluate, the integral for the volume of the solid obtained by rotating the area between the curves $y = x$ and $y = \sqrt{x}$ about the $x$-axis.\noanswer
\end{enumerate}
\end{document}
