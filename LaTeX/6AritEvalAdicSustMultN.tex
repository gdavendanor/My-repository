\documentclass[fleqn]{article}
\usepackage[spanish,es-noshorthands]{babel}
\usepackage[utf8]{inputenc} 
\usepackage[papersize={6.5in,8.5in},left=1cm, right=1cm, top=1.5cm, bottom=1.7cm]{geometry}
\usepackage{mathexam}
\usepackage{amsmath}
\usepackage{amsfonts}
\usepackage{amssymb}

\usepackage{graphicx}

\ExamClass{\includegraphics[height=16pt]{Images/logo-sed.png} Matemáticas $6^{\circ}$}
\ExamName{Evaluación, Operaciones en $\mathbb{N}$}
\ExamHead{\includegraphics[height=16pt]{Images/logo-colegio.png} IEDAB}
\newcommand{\LineaNombre}{%
\par
\vspace{\baselineskip}
Nombre:\hrulefill \; Curso: \underline{603} \; Fecha: \underline{\hspace*{2.5cm}} \relax
\par}
\let\ds\displaystyle

\begin{document}
\ExamInstrBox{
Respuesta sin justificar mediante procedimiento no será tenida en cuenta en la calificación. Escriba sus respuestas en el espacio indicado. Tiene 45 minutos para contestar esta prueba.}
\LineaNombre
\begin{enumerate}
   \item Ordene los números usando $<$ o $>$ según el caso: 
      \begin{enumerate}
	 \item De menor a mayor los siguientes números: 4050, 4500, 4005, 4555, 40005\answer
	 \item De mayor a menor los siguientes números: 6040, 6400, 64000, 6004, 60400, 60404\answer
      \end{enumerate}
   El cardinal de un conjunto es el número de elementos del conjunto.
  \item En la finca de Jerónimo, hay tres corrales. En el primero hay 234 reses, en el segundo hay 37 reses y en el tercero 208 reses.¿Cuántas reses tiene Jerónimo en su finca?
  \begin{enumerate}
  \item Operación \underline{\hspace*{80pt}} \tikz 
  \end{enumerate}
   \item Busca el término desconocido e indica su nombre en las siguientes operaciones:
   \begin{enumerate}
   \item $329+\underline{\hspace*{48pt}}=1206$\noanswer
   \item $\underline{\hspace*{60pt}}-4208=524$\noanswer
   \item $324\times \underline{\hspace*{60pt}}=15552 $\noanswer
   \end{enumerate}
\end{enumerate}
\end{document}
