\documentclass[10pt,twoside]{article}
\usepackage[utf8]{inputenc}
\usepackage{amsmath}
\usepackage{amsfonts}
\usepackage{amssymb}
\usepackage[spanish,es-noshorthands]{babel}
\usepackage[T1]{fontenc}
\usepackage{lmodern}
\usepackage{graphicx,hyperref}
\usepackage{tikz,pgf}
\usepackage{multicol}
\usepackage{subfig}
\usepackage[papersize={6.5in,8.5in},width=5.5in,height=7in]{geometry}
\usepackage{fancyhdr}
\pagestyle{fancy}
\fancyhead[LE]{\includegraphics[height=12pt]{Images/logo-colegio.png} Álgebra $8^{\circ}$}
\fancyhead[RE]{}
\fancyhead[RO]{\textit{Germ\'an Avenda\~no Ram\'irez, Lic. U.D., M.Sc. U.N.}}
\fancyhead[LO]{}

\author{Germ\'an Avenda\~no Ram\'irez, Lic. U.D., M.Sc. U.N.}
\title{\begin{minipage}{.2\textwidth}
\includegraphics[height=1.75cm]{Images/logo-colegio.png}\end{minipage}
\begin{minipage}{.55\textwidth}
\begin{center}
Taller 04, Expresiones algebraicas \\
Álgebra $8^{\circ}$
\end{center}
\end{minipage}\hfill
\begin{minipage}{.2\textwidth}
\includegraphics[height=1.75cm]{Images/logo-sed.png} 
\end{minipage}}
\date{}
\begin{document}
\maketitle
Nombre: \hrulefill Curso: \underline{\hspace*{44pt}} Fecha: \underline{\hspace*{2.5cm}}
\section*{Continuaci\'{o}n Nivel I}
Resuelve los siguientes problemas
\begin{enumerate}
\item ¿Qué número aumentado en 17 da 47?
\item La diferencia entre un número y 5 es 8. Calcula ese número.
\item Repartir 300 euros entre tres amigos de modo que cada uno reciba 5 euros más que el anterior.
\item Entre Luis y Antonio reúnen 840 euros. Sabiendo que Antonio tiene 125 euros más que Luis, calcular los euros que tiene cada uno.
\item Repartir 300 euros entre tres personas de modo que la segunda reciba 16 euros más que la primera y la tercera 28 euros más que la segunda.
\item Los 7/13 del valor de un balón más 45 pts suman 675 pts. ¿Cuánto vale el balón?
\item Un balón de reglamento y una bicicleta me han costado 40.000 pts. Si la bicicleta vale el cuádruplo que el balón, ¿cuánto vale cada uno?
\item Una persona gasta 1/2 de su sueldo en comida; 1/5 de su sueldo en vivienda y 1/6 de su sueldo en vestido. Si todavía le sobran 20.000 pts, ¿cuánto gana de sueldo?
\end{enumerate}
\section*{Nivel II}
\begin{enumerate}
\item Expresa por medio de lenguaje algebraico:
\begin{enumerate}
\item La suma de los cuadrados de dos números cualesquiera
\item El cuadrado de la suma de dos números cualesquiera
\item La diferencia de los cuadrados de dos números
\item El cuadrado de la diferencia de dos números
\item La suma de dos números multiplicada por su diferencia
\item La raíz cuadrada del producto de dos números
\item El triple de un número menos la mitad de ese número
\item El doble del cubo de un número más la quinta parte del cuadrado de ese número
\end{enumerate}
\item Escribe con lenguaje algebraico:
\begin{enumerate}
\item El perímetro de un cuadrado, cuyo lado mide $x$ cm es \ldots
\item El área del cuadrado anterior es \ldots
\item El valor de $h$ kg de tomates a 80 pts el kg es \ldots
\item La edad de una persona que ahora tiene $m$ años, será dentro de 5 años \ldots
\item La edad de una persona que ahora tiene $m$ años, era hace 5 años \ldots
\item El valor del lado de un cuadrado cuyo perímetro es $p$ será \ldots
\item El perímetro de un rectángulo cuya base mide $b$ y cuya altura mide $a$ es \ldots
\item El dinero que me falta para tener 100 pts si tengo $h$ pts es \ldots
\end{enumerate}
\item Si Juan tiene $x$ pts y Luis tiene $y$ pts, escribe estas igualdades algebraicas:
\begin{enumerate}
\item Juan tiene doble dinero que Luis
\item Entre los dos tienen 500 pts
\item Luis tiene la mitad de dinero que Juan
\item Juan tiene 20 pts más que Luis
\item Luis tiene 20 pts menos que Juan
\end{enumerate}
\item Escribe el significado de estas expresiones algebraicas:
\begin{tabbing}
\hspace{2.5cm}\=\hspace{2.5cm}\=\hspace{2.5cm}\=\kill
$(a+b)^{2}$; \> $a^{2}+b^{2}$; \> $a^{2}-b^{2}$; \> $(a-b)^{2}$; \\ 
$a^{2}-b\div 2$; \> $3x^{2}+2b^{3}$; \> $\sqrt{3a-2b}$; \> $a/5+2b^{3}$; 
\end{tabbing} 
\begin{tabular}{|c|c|c|c|c|}
\hline 
$a$ & $b$ & $c$ & Expresión algebraica & Valor numérico \rule[-0.3cm]{0cm}{0.8cm}
 \\ 
\hline 
$+2$ & $+3$ & $+4$ & $(a+b+c)^{2}$ & \rule[-0.3cm]{0cm}{0.8cm}
 \\ 
\hline 
$-5$ & $+6$ & $-2$ & $2a-\frac{b}{3}+c^{2}$ & \rule[-0.3cm]{0cm}{0.8cm}
 \\ 
\hline 
$+6$ & 0 & $-4$ & $2(a-b)+\dfrac{b+c}{2}$ & \rule[-0.3cm]{0cm}{0.8cm}
 \\ 
\hline 
$-1$ & $-2$ & $-3$ & $-2(a-c)+3(b-a)$ &  \\ 
\hline 
$+1$ & $-8$ & $+2$ & $a^{2}-(b+c)^{2}$ &  \\ 
\hline 
\end{tabular} 
\end{enumerate}
\end{document}
