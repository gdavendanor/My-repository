\documentclass[10pt,twoside]{article}
\usepackage[utf8]{inputenc}
\usepackage{amsmath,amsfonts,amssymb}
\usepackage[spanish,es-noshorthands]{babel}
\usepackage[T1]{fontenc}
\usepackage{lmodern}
\usepackage{graphicx,hyperref}
\usepackage{tikz,pgf}
\usepackage{marvosym}
\usepackage{color}
\usepackage{multicol}
\usepackage{subfig}
\usepackage[papersize={6.5in,8.5in},total={5.5in,7.15in},centering]{geometry}
\usepackage{fancyhdr}
\pagestyle{fancy}
\fancyhead[LE]{Colegio Arborizadora Baja}
\fancyhead[RE]{PEI:``Hacia una cultura para el desarrollo sostenible''}
\fancyfoot[RO]{\Email gavendanor@colarborizadorabaja.edu.co}
\fancyhead[LO]{\url{www.autistici.org/mathgerman}}
\fancyfoot[RE]{\Email cedarborizadoraba19@redp.edu.co}
\fancyfoot[LE]{Calle 59I \#44A - 02 \Telefon 7313994 - 7313995}
\fancyhead[RO]{Nit 830024976-8, Código DANE 11100103084-8}

\author{Germ\'an Avenda\~no Ram\'irez~\thanks{Lic. Mat. U.D., M.Sc. U.N.}}
\title{\begin{minipage}{.2\textwidth}
\includegraphics[height=1.75cm]{Images/logo-colegio.png}\end{minipage}
\begin{minipage}{.55\textwidth}
\begin{center}
Taller 02 \\
Aritmética $6^{\circ}$
\end{center}
\end{minipage}\hfill
\begin{minipage}{.2\textwidth}
\includegraphics[height=1.75cm]{Images/logo-sed.png} 
\end{minipage}}
\date{}
\begin{document}
\maketitle
%Nombre: \hrulefill Curso: \underline{\hspace*{44pt}} Fecha: \underline{\hspace*{2.5cm}}
\begin{enumerate}
\item Representa gráficamente la descomposición pitagórica de cada número dado cuadrado perfecto
\begin{enumerate}
\begin{multicols}{2}
\item Número 9 \; \includegraphics[scale=.25]{Images/CuadradoPerfecto.png} 
\item Número 16
\item Número 25
\item Número 81
\end{multicols}
\end{enumerate}
\item Escribe las decenas de millón que tiene cada número:
\begin{enumerate}\begin{multicols}{3}
\item 345'061\,410 \item 1\,518'433\,001 \item 946\,642.
\end{multicols}
\end{enumerate}
\item Dado el número 845\,042 identifica la cifra de:
\begin{enumerate}
\begin{multicols}{2}
\item Las decenas. \item Las centenas.
\item Las unidades de mil.
\item Las centenas de mil.
\end{multicols}
\end{enumerate}
\item Escribe el número que tiene
\begin{enumerate}
\item 4 decenas de mil, 2 unidades, 0 centenas, 1
unidad de mil y 7 decenas.
\item 7 Unidades de millón, 4 centenas y 3 unidades.
\item 9 centenas de millón, 2 unidades de millón,
7 decenas y 1 unidad.
\item 2 decenas de millón, 4 unidades de millón,
8 unidades de mil y 1 decena.
\end{enumerate}
\item Escribe con palabras cada número:
\begin{enumerate}
\begin{multicols}{4}
\item 7\,416.
\item 135\,008.
\item 24'402\,683.
\item 800'724\,001.
\end{multicols}
\end{enumerate}
\item Escribe el valor posicional de la cifra señalada:
\begin{enumerate}
\begin{multicols}{4}
\item 53\textcolor{red}{6}\,245
\item \textcolor{red}{1}8\,416
\item 45\,\textcolor{red}{6}58'360\,288
\item 5\textcolor{red}{6}'230\,341
\end{multicols}
\end{enumerate}
\item Escribe el número que corresponde a cada expresión:
\begin{enumerate}
\item $(4\times 10^{4})+(6\times 10^{3})+(1\times 10^{2})+(7\times 10^{1})+2=$
\item $(5\times 10^{7})+(3\times 10^{6})+(0\times 10^{5})+(2\times 10^{4})+(7\times 10^{3})+(2\times 10^{2})+(9\times 10^{1})+5=$
\item $(2\times 10^{2})+(9\times 10^{1})+3=$
\item $(1\times 10^{6})+(5\times 10^{3})+(2\times 10^{2})=$
\end{enumerate}
\item Problema para discutir con el/la compañero/a. Si un número tiene:\\
El dígito de las unidades de mil es 5\\
El dígito de las centenas es 2\\
El dígito de las decenas de mil es el doble del dígito de las centenas\\
Las unidades y las decenas de mil tienen el mismo valor y la suma de todos sus dígitos es 22\\

El número es: \tikz \draw (0,0) rectangle (.5,.5);\tikz \draw (0,0) rectangle (.5,.5);\tikz \draw (0,0) rectangle (.5,.5);\tikz \draw (0,0) rectangle (.5,.5);\tikz \draw (0,0) rectangle (.5,.5);
\end{enumerate}
\fbox{\parbox{5.25in}{\textbf{Diversión matemática}\\ \textbf{A la cacería de un 53}\\ Con cinco veces el número 5, tres veces el número 3 y los
signos Matemáticos $+$, $-4$, $\times$, $\div$ y () forma expresiones
matemáticas que sean igual a 53.}}
\section*{Actividad 2}
\begin{enumerate}
\item Clasifica los números de la siguiente lista en: Naturales $\mathbb{N}$ y NO naturales
\begin{enumerate}
\begin{multicols}{6}
\item 4
\item $\dfrac{1}{4}$
\item 12
\item 0.2
\item 100
\item $\dfrac{3}{5}$
\item 12.45
\item 160\,001
\item 5
\item 0.3
\item 5\,200
\item 2.5
\end{multicols}
\end{enumerate}
\item Ordena de mayor a menor los siguientes números naturales: 3, 7, 2, 5, 0, 10, 15
\item Escribe el sucesor y el antecesor de cada uno de los siguientes números naturales:
\begin{enumerate}
\begin{multicols}{6}
\item 89
\item 101
\item 499
\item 1\,000
\item 32
\item 487
\end{multicols}
\end{enumerate}
%\item Escribe con palabras el nombre, según la posición que ocupen los siguientes números ordinales, así por ejemplo:
%\begin{enumerate}
%\begin{multicols}{3}
%\item 8$^{\circ}$ \dots Octavo
%\item 11$^{\circ}$ \dots
%\item 13$^{\circ}$ \ldots
%\item 25$^{\circ}$
%\item 30$^{\circ}$
%\item 59$^{\circ}$
%\end{multicols}
%\end{enumerate}
\end{enumerate}
\end{document}
