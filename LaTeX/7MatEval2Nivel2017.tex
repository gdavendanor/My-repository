\documentclass[fleqn]{article}
\usepackage[spanish,es-noshorthands]{babel}
\usepackage[utf8]{inputenc} 
\usepackage[papersize={5.5in,8.5in},left=1cm, right=1cm, top=1.5cm, bottom=1.7cm]{geometry}
\usepackage{mathexam}
\usepackage{amsmath}
\usepackage{graphicx}
\usepackage{tikz,pgf}

\ExamClass{\includegraphics[height=16pt]{Images/logo-sed.png} Matemáticas $7^{\circ}$}
\ExamName{Nivelación 2, 2017}
\ExamHead{\includegraphics[height=16pt]{Images/logo-colegio.png} IEDAB}
\newcommand{\LineaNombre}{%
\par
\vspace{\baselineskip}
Nombre:\hrulefill \; Curso: \underline{\hspace*{48pt}} \; Fecha: \underline{\hspace*{2.5cm}} \relax
\par}
\let\ds\displaystyle

\begin{document}
\ExamInstrBox{
Respuesta sin justificar mediante procedimiento no será tenida en cuenta en la calificación. Escriba sus respuestas en el espacio indicado. Tiene 45 minutos para contestar esta prueba.}
\LineaNombre
\begin{enumerate}
 \item Calcule y simplifique
 \begin{enumerate}
 \item $12\,854\cdot 750\,000=$\noanswer[12pt]
 \item $32\div [(-2)(-8)-(15-(-1))]=$ \noanswer[12pt]
 \item $-\dfrac{3}{14}\div \dfrac{6}{7}=$\noanswer[12pt]
 \item $\dfrac{2}{27}\cdot \left(-\dfrac{9}{16}\right)=$\noanswer[12pt]
 \end{enumerate}
\begin{minipage}{.45\textwidth}
\item Encuentre la medida del ángulo faltante
\end{minipage}\hfill
\begin{minipage}{.5\textwidth}
\begin{center}
\begin{tikzpicture}
\draw (0,0) --(3,0)node[above]{$x$}--(4.4,2) --cycle;
\node[right] at (0.4,.15){$28^{\circ}$};
\node[left] at (4.2,1.45) {{$32^{\circ}$}};
\end{tikzpicture}
\end{center}
\end{minipage}
\noanswer
\item \emph{Pasaportes:} En 2010, 14'794\,604 pasaportes americanos fueron usados. Este número decreció a 12'613\,153 en 2011 y luego se incrementó a 13'125\,829 en 2012. ¿En qué porcentaje disminuyó de 2010 a 2011? ¿En qué porcentaje se incrementó de 2011 a 2012? Aproxime las respuestas a la décima más cercana.\noanswer
\newpage
\item El trece por ciento de 600 estudiantes de un colegio, están en grado sexto. ¿Cuántos estudiantes están en 6$^{\circ}$\noanswer
\item Calcule $46-[4(6+4\div 2)+2\times 3-5]=$
\item \emph{Reducción del tamaño:} Las empresas acostumbran reducir el tamaño de sus productos para ponerles el mismo precio que manejaban antes y dar la sensación que no están subiendo los precios. La siguiente tabla muestra la lista de productos reducidos en tamaño.

\begin{tabular}{|c|c|c|c|}
\hline 
 & \multicolumn{2}{c}{Tamaño (onzas)}  &  \\ 
\hline 
Producto & viejo & nuevo & Reducción\\ 
\hline 
Crema helado Breyer & 56 & 48 & 14\% \\ 
\hline 
Mayonesa Hellman & 32 & 30 & 6\% \\ 
\hline 
Chocolate negro Hershey & 9 & 6.8 & 15\% \\ 
\hline 
Comida de gatos Iams & 6 & 5.5 & 8\% \\ 
\hline 
Papas Nabisco & 16 & 15.25 & 5\% \\ 
\hline 
Crema de maní Skippy & 18 & 16.3 & 9\% \\ 
\hline 
Jugo de naranja Tropicana & 96 & 89 & 7\% \\ 
\hline 
\end{tabular} 
\begin{enumerate}
\item ¿Cuántas onzas de crema de helado Breyer hay menos en el nuevo empaque, respecto al viejo?\noanswer
\item ¿En qué porcentaje se redujo el tamaño del empaque de la mayonesa Hellman?\noanswer
\item ¿Qué producto en la tabla muestra un mayor porcentaje de reducción en su tamaño?\noanswer
\item Cuánto menos jugo hay en el nuevo envase de jugo de naranja Tropicana con respecto al empaque viejo?\noanswer
\item ¿Cuál producto en la tabla muestra un porcentaje de reducción en su tamaño?\noanswer
\end{enumerate}
 \end{enumerate}

\end{document}
