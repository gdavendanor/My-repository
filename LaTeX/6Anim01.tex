\documentclass[10pt,twoside]{article}
\usepackage[utf8]{inputenc}
\usepackage{amsmath}
\usepackage{amsfonts}
\usepackage{amssymb}
\usepackage[spanish,es-noshorthands]{babel}
\usepackage[T1]{fontenc}
\usepackage{lmodern}
\usepackage{graphicx,hyperref}
\usepackage{tikz,pgf}
\usepackage{multicol}
\usepackage{subfig}
\usepackage[papersize={6.5in,8.5in},width=5.5in,height=7in]{geometry}
\usepackage{fancyhdr}
\pagestyle{fancy}
\fancyhead[LE]{\includegraphics[height=12pt]{Images/logo-colegio.png} Geometría $6^{\circ}$}
\fancyhead[RE]{}
\fancyhead[RO]{\textit{Germ\'an Avenda\~no Ram\'irez, Lic. U.D., M.Sc. U.N.}}
\fancyhead[LO]{}

\author{Germ\'an Avenda\~no Ram\'irez, Lic. U.D., M.Sc. U.N.}
\title{\begin{minipage}{.2\textwidth}
\includegraphics[height=1.75cm]{Images/logo-colegio.png}\end{minipage}
\begin{minipage}{.55\textwidth}
\begin{center}
Animaplano 01 \\
Matemáticas $6^{\circ}$
\end{center}
\end{minipage}\hfill
\begin{minipage}{.2\textwidth}
\includegraphics[height=1.75cm]{Images/logo-sed.png} 
\end{minipage}}
\date{}
\begin{document}
\maketitle
Nombre: \hrulefill Curso: \underline{\hspace*{44pt}} Fecha: \underline{\hspace*{2.5cm}}
\section*{Cuestionario}
Realice en forma ordenada cada operación y luego con los resultados, una los puntos para formar una figura. Cuando se combinan operaciones como la adición y la multiplicación, o, la multiplicación y la sustracción, primero se debe resolver la multiplicación a menos que existan paréntesis que indiquen lo contrario; miremos este ejemplo:
\[6\times 9-24=(6\times 9)-24=(54)-24=30\]
\begin{enumerate}
\item $3\times 7-6=$
\item $8\times 3-8=$
\item $21\div 3=$
\item $\frac{1}{5}\times 20=$
\item $4^{2}-1=$
\item $3\times 7+2=$
\item $4\times 8=$
\item $6\times 7+10=$
\item $9\times 6+1=$
\item $9\times 5 +40=$
\item $9\times 7+30=$
\item $8\times 6+50=$
\item $8\times 7+30=$
\item $7\times 8=$
\item $7\times 7+10$
\item $7^{2}-10=$
\item $4\times 7=$
\item $6\times 6-20=$
\end{enumerate}
\section*{Animaplano}
\end{document}
