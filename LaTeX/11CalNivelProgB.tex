\documentclass[fleqn]{article}
\usepackage[spanish,es-noshorthands]{babel}
\usepackage[utf8]{inputenc} 
\usepackage[papersize={6.5in,8.5in},left=1cm, right=1cm, top=1.5cm, bottom=1.7cm]{geometry}
\usepackage{mathexam}
\usepackage{amsmath}
\usepackage{graphicx}

\ExamClass{\includegraphics[height=16pt]{Images/logo-sed.png} Cálculo $11^{\circ}$}
\ExamName{"Nivelación, Progresiones y sucesiones"}
\ExamHead{\includegraphics[height=16pt]{Images/logo-colegio.png} IEDAB}
\newcommand{\LineaNombre}{%
\par
\vspace{\baselineskip}
Nombre:\hrulefill \; Curso: \underline{\hspace*{48pt}} \; Fecha: \underline{\hspace*{2.5cm}} \relax
\par}
\let\ds\displaystyle

\begin{document}
\ExamInstrBox{
Respuesta sin justificar mediante procedimiento no será tenida en cuenta en la calificación. Escriba sus respuestas en el espacio indicado. Tiene 55 minutos para contestar esta prueba.}
\LineaNombre
\section*{Para recordar}
Una progresi\'on aritmética tiene como término general \fbox{$a_{n}=a_{1}+(n-1)d$}, donde $d$ es la distancia o diferencia que hay entre dos términos consecutivos.

Una progresión geométrica tiene como término genral \fbox{$a_{n}=a_{1}r^{n-1}$},
donde $r$ es la razón geométrica.
\begin{enumerate}
\item Halle los tres términos siguientes en las sucesiones indicadas y determine si son progresiones, en el caso que sean progresiones, determinar si son aritméticas o geométricas
\begin{enumerate}
\item 2, 6, 10, 14, \ldots \noanswer
\item 1, 8, 27, \ldots \noanswer
\item $2$, $\frac{3}{8}$, $\frac{4}{27}$, \ldots \noanswer
\item 4, 8, 16, \ldots \noanswer
\end{enumerate}
\item Halle los siete primeros términos de una progresión aritmética:
\begin{enumerate}
\item cuyo primer término es $-2$ y su diferencia $d$ es 4 \noanswer[.75in]
\item cuyo segundo término es 5 y su diferencia $d$ es 3 \noanswer[.75in]
\end{enumerate}
\newpage
\item Halle el término general $a_{n}$ de una progresión aritmética
\begin{enumerate}
\item cuyo primer término es 3 y su diferencia $d$ es $-2$. \noanswer
\item cuyo primer término es 4 y su segundo término es 7.\noanswer
\end{enumerate}
\item En una granja hay 60 pollos y cada día nacen 15. ¿cuántos habrá al cabo de 30 días si no muere ninguno?\noanswer
\item Cada día me duplican el dinero que tengo y me dan 3 dólares más. Si el primer día tengo 25 dólares, construya la sucesión que indica el dinero que tengo cada día. Hágalo para una semana.\noanswer
\end{enumerate}
\end{document}