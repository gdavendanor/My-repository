\documentclass[letterpaper,fleqn]{article}
\usepackage[spanish,es-noshorthands]{babel}
\usepackage[utf8]{inputenc} 
\usepackage[left=1cm, right=1cm, top=1.5cm, bottom=1.7cm]{geometry}
\usepackage{mathexam}
\usepackage{amsmath}
\usepackage{graphicx}

\ExamClass{\includegraphics[height=16pt]{Images/logo-sed.png} Aritmética $6^{\circ}$}
\ExamName{Sustentación Plan de Mejoramiento 2 período}
\ExamHead{\includegraphics[height=16pt]{Images/logo-colegio.png} IEDAB}
\newcommand{\LineaNombre}{%
\par
\vspace{\baselineskip}
Nombre:\hrulefill \; Curso: \underline{\hspace*{48pt}} \; Fecha: \underline{\hspace*{2.5cm}} \relax
\par}
\let\ds\displaystyle

\begin{document}
\ExamInstrBox{
Respuesta sin justificar mediante procedimiento no será tenida en cuenta en la calificación. Escriba sus respuestas en el espacio indicado. Tiene 45 minutos para contestar esta prueba.}
\LineaNombre
\begin{enumerate}
 \item Cada adición tiene dos sustracciones asociadas y cada sustracción, tiene una sustracción y una adición asociadas. Por ejemplo:
 $15-8=7$, tiene asociadas la adición $7+8=15$ y la sustracción $15-7=8$. Con base en lo anterior, determine la adición y sustracción asociadas a las siguientes sustracciones.
 \begin{enumerate}
 \item $538-215=$\noanswer
 \item $879-598=$\noanswer
 \end{enumerate}
 \item Resuelva las siguientes operaciones combinadas. (\textit{Recuerde que primero se resuelven las multiplicaciones y divisiones antes que las adiciones y sustracciones})
 \begin{enumerate}
 \item $348-240\div 15+34\cdot 19=$\noanswer
 \item $1492-729\div 27+465\cdot 32=$\noanswer
 \end{enumerate}
 \item El producto de dos números es 782 y uno de los números es igual al cociente de 391 entres 23. Halle los números.\noanswer
 \item ¿Cuál es el número que al dividirlo entre 53 su cociente es 48 y su residuo es 17? (Recuerde que en una división siempre se cumple que:
 \[Dividendo=Divisor\cdot Cociente+Residuo\]\noanswer
 \item En un almacén se han vendido ayer 14 camisas más que hoy. Si entre ayer y hoy se han vendido 100 camisas, 
\begin{enumerate}
\item ¿cuántos camisas se vendieron ayer?\noanswer
\item ¿Cuántos camisas se vendieron hoy?\noanswer
\end{enumerate} 
\newpage
 \item En una granja se han vendido 1235 huevos. Si una docena y media de huevos cuesta 6300.
 \begin{enumerate}
 \item ¿Cuánto vale cada huevo?\noanswer
 \item ¿Cuánto valen los huevos vendidos en la granja?\noanswer
 \end{enumerate}  
 \item Un camionero carga en su camión 4 televisores y tres microondas. Si cada televisor pesa como tres microondas y en total ha cargado 75 kilos ¿Cuánto pesa cada aparato?\noanswer
 \item Cada gallina de una granja pone dos huevos en tres días. ¿Cuántos días tardarán cuatro gallinas en poner tres docenas de huevos?\noanswer
 \item Si dos pantalonetas de la misma talla y marca cuestan \$$18\,000$, ¿cuánto cuestan 15 pantalonetas de la misma marca y talla?\noanswer
 \item Si 18 camisas de la misma marca y talla le costaron a un comerciante 431820. ¿Cuánto le costó cada camisa?\noanswer
 \end{enumerate}

\end{document}
