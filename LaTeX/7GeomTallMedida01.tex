\documentclass[twoside]{article}
\usepackage[utf8]{inputenc}
\usepackage{amsmath,amsfonts,amssymb,amsthm,latexsym}
\usepackage[spanish,es-noshorthands]{babel}
\usepackage[T1]{fontenc}
\usepackage{lmodern}
\usepackage{graphicx,hyperref}
\usepackage{tikz,pgf}
\usepackage{marvosym}
\usepackage{multicol}
\usepackage{fancyhdr}
\usepackage[papersize={5.5in,8.5in},left=.75cm,right=.75cm,top=1.5cm,bottom=1.25cm]{geometry}
\usepackage{fancyhdr}
\pagestyle{fancy}
\fancyhead[LE]{Colegio Arborizadora Baja}
\fancyhead[RE]{PEI:``Hacia una cultura para el desarrollo sostenible''}
\fancyfoot[RO]{\Email iedabgerman@autistici.org}
\fancyhead[LO]{\url{www.autistici.org/mathgerman}}
\fancyfoot[RE]{\Email cedarborizadoraba19@redp.edu.co}
\fancyfoot[LE]{Calle 59I \#44A - 02 \Telefon 7313994 - 7313995}
\fancyhead[RO]{Nit 830024976-8, Código DANE 11100103084-8}

\author{Germ\'an Avenda\~no Ram\'irez~\thanks{Lic. Mat. U.D., M.Sc. U.N.}}
\title{\begin{minipage}{.2\textwidth}
\includegraphics[height=1.75cm]{Images/logo-colegio.png}\end{minipage}
\begin{minipage}{.55\textwidth}
\begin{center}
La medida\\
Geometría $7^{\circ}$
\end{center}
\end{minipage}\hfill
\begin{minipage}{.2\textwidth}
\includegraphics[height=1.75cm]{Images/logo-sed.png} 
\end{minipage}}
\date{}
\thispagestyle{plain}
\begin{document}
\maketitle
%Nombre: \hrulefill Curso: \underline{\hspace*{44pt}} Fecha: \underline{\hspace*{2.5cm}}
\begin{minipage}{.95\textwidth}
\fbox{\textit{No raye ni dañe esta hoja para que pueda usarla otro compañero}}
\end{minipage}
\section*{Nivel 1}
\begin{enumerate}
\item Analiza si las siguientes propiedades de los objetos representan o no una magnitud y en caso afirmativo di a cual:
\begin{enumerate}
\begin{multicols}{2}
\item El largo de una habitación.
\item La fiebre de una persona.
\item La altura de tu clase.
\item La superficie de tu clase.
\item El ruido del motor.
\item El color de tus ojos.
\item La simpatía de una persona.
\item La duración de una clase.
\end{multicols}
\end{enumerate}
\item Di que unidad de medida utilizarías para medir en los siguientes casos:
\begin{enumerate}
\item La distancia entre dos casas en un pueble pequeño.
\item El tiempo que tardas en contar desde 1 hasta 50.
\item Lo que pesa un cordón de oro.
\item La superficie de una habitación.
\item La duración de un partido de baloncesto.
\end{enumerate}
\item ¿Cómo se expresa una medida? Pon tres ejemplos.
\item Expresa con la medida más adecuada:
\begin{enumerate}
\item La distancia de Bogotá a Medellín.
\item La masa de un niño de 10 años.
\item La longitud de un lapicero.
\item La capacidad de un garrafón.
\item La masa de un camión.
\item La distancia de una pared a otra de la clase.
\item La capacidad de una piscina.
\item La masa de un sacapuntas.
\end{enumerate}
\item Realiza un esquema sencillo de las unidades de: longitud, capacidad y masa
\item Completa:
\begin{enumerate}
\begin{multicols}{2}
\item 1 Dl = \ldots l = \ldots dl
\item 1 Kl = \ldots dl = \ldots Dl
\item 3 Hl = \ldots l = \ldots dl
\item 240 Dl = \ldots ml = \ldots dl
\end{multicols}
\end{enumerate}
\item Completa:
\begin{enumerate}
\begin{multicols}{2}
\item 1 Kg = ... Hg = ... Dg
\item 1 g = ... dg = ... mg
\end{multicols}
\end{enumerate}
\item Ordena de mayor a menor las masas siguientes:
\begin{enumerate}
\item 3.025 gramos, 2,02 Kilogramos, 20 Hectogramos, 202,3 Decagramos.
\item 299,9 Kilogramos, 0,3 Toneladas.
\end{enumerate}
\item Pasar a Hectogramos (Hg) sumando al final los resultados.
\begin{enumerate}
\item 6Mg + 12 Hg + 15 Dg + 6 mg =
\item 8Kg  + 126g =
\end{enumerate}
\item Las medidas reglamentarias de una cancha de tenis son: largo 23,77 metros; ancho 8,23 metros. ¿Cuál será el perímetro de dicha cancha? (el perímetro es la suma de todos sus lados)
\item España tiene 3.904 Kilómetros de costas. Expresa esta longitud en cinco unidades diferentes.
\item Un avión anfibio utilizó para apagar un incendio 55 Hectolitros de agua. ¿Cuántos litros de agua gastó? ¿Cuántos Kilolitros?
\item Una ballena azul pesa unos 100.000 Kilogramos. Solo su lengua pesa 4 Toneladas. ¿Cuántos Kilogramos pesa su lengua? ¿Cuántas Toneladas pesa la ballena?
\item Pilar avanza 0,60 metros en un paso y Adela 0,65 metros. Si salen del mismo punto y en el mismo sentido, ¿a qué distancia se encontrarán la una de la otra cuando hayan recorrido 400 pasos cada una? ¿Y si fueran en distinto sentido?
\item ¿Cuántas botellas de 3/4 de litro se pueden llenar con el contenido de una cubeta de 2.400 litros?
%\item Un camión lleva 24.300 kilos de trigo. Si la Tonelada vale a 25.000 pesetas. ¿Cuánto vale la carga que lleva el camión?
\end{enumerate}
\end{document}
