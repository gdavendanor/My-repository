\documentclass[10pt,twoside]{article}
\usepackage[utf8]{inputenc}
\usepackage{amsmath}
\usepackage{amsfonts}
\usepackage{amssymb}
\usepackage[spanish,es-noshorthands]{babel}
\usepackage[T1]{fontenc}
\usepackage{lmodern}
\usepackage{graphicx,hyperref}
\usepackage{tikz,pgf}
\usepackage{multicol}
\usepackage{subfig}
\usepackage[papersize={6.5in,8.5in},width=5.5in,height=7in]{geometry}
\usepackage{fancyhdr}
\pagestyle{fancy}
\fancyhead[LE]{\includegraphics[height=12pt]{Images/logo-colegio.png} Cálculo $11^{\circ}$}
\fancyhead[RE]{}
\fancyhead[RO]{\textit{Germ\'an Avenda\~no Ram\'irez, Lic. U.D., M.Sc. U.N.}}
\fancyhead[LO]{}

\author{Germ\'an Avenda\~no Ram\'irez, Lic. U.D., M.Sc. U.N.}
\title{\begin{minipage}{.2\textwidth}
\includegraphics[height=1.75cm]{Images/logo-colegio.png}\end{minipage}
\begin{minipage}{.55\textwidth}
\begin{center}
Taller, Calculando límites algebraicamente\\
Cálculo $11^{\circ}$
\end{center}
\end{minipage}\hfill
\begin{minipage}{.2\textwidth}
\includegraphics[height=1.75cm]{Images/logo-sed.png} 
\end{minipage}}
\date{}
\begin{document}
\maketitle
Nombre: \hrulefill Curso: \underline{\hspace*{44pt}} Fecha: \underline{\hspace*{2.5cm}}
\section*{Propiedades de los l\'{i}mites}
Para resolver l\'{i}mites algebraicamente, es necesario y \'{u}til, aplicar sus propiedades
\begin{enumerate}
\item $\displaystyle{\lim_{x \rightarrow a}[f(x)+g(x)]=\lim_{x \rightarrow a}f(x)+\lim_{x\rightarrow a}g(x)}$ \hfill Límite de una suma
\item $\displaystyle{\lim_{x\rightarrow a}[f(x)-g(x)]=\lim_{x\rightarrow a}f(x)-\lim_{x\rightarrow a}g(x)}$ \hfill Límite de una diferencia
\item $\displaystyle{\lim_{x\rightarrow a}[cf(x)]=c\lim_{x\rightarrow a}f(x)}$ \hfill Límite de una constante por una función
\item 
\end{enumerate}
\end{document}
