\documentclass[fleqn]{article}
\usepackage[spanish,es-noshorthands]{babel}
\usepackage[utf8]{inputenc} 
\usepackage[papersize={5.5in,8.5in},left=1cm, right=1cm, top=1.5cm, bottom=1.7cm]{geometry}
\usepackage{mathexam}
\usepackage{amsmath}
\usepackage{graphicx}

\ExamClass{\includegraphics[height=16pt]{Images/logo-sed.png} Álgebra $9^{\circ}$}
\ExamName{Números complejos}
\ExamHead{\includegraphics[height=16pt]{Images/logo-colegio.png} IEDAB}
\newcommand{\LineaNombre}{%
\par
\vspace{\baselineskip}
Nombre:\hrulefill \; Curso: \underline{\hspace*{30pt}} \; Fecha: \underline{\hspace*{2.5cm}} \relax
\par}
\let\ds\displaystyle

\begin{document}
\ExamInstrBox{
Respuesta sin justificar mediante procedimiento no será tenida en cuenta en la calificación. Escriba sus respuestas en el espacio indicado. Tiene 45 minutos para contestar esta prueba.}
\LineaNombre
\begin{enumerate}
 \item Simplifique cada una de las siguientes expresiones:
 \begin{enumerate}
 \item $4\sqrt{-100}=$\noanswer
 \item $6\sqrt{-75}+4\sqrt{-48}=$ \noanswer
 \end{enumerate}
 \item Sabiendo que $i^{0}=1$, $i^{1}=i$, $i^{2}=-1$, $i^{3}=-i$ y $i^{4}=1$, hallar
 \begin{enumerate}
 \item $i^{23}=$\noanswer
 \item $i^{30}=$\noanswer
 \end{enumerate}
 \newpage
 \item Resuelva las siguientes operaciones
 \begin{enumerate}
 \item $(5-4i)+(2+7i)=$ \noanswer
 \item $(5-4i)-(2+7i)=$ \noanswer
 \item $(5-4i)(2+7i)=$ \noanswer
 \item $\dfrac{5-4i}{2+7i}=$ \noanswer
 \end{enumerate}
 \end{enumerate}

\end{document}
