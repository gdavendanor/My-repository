\documentclass[11pt]{article}
\usepackage[utf8]{inputenc}
\usepackage{amsmath}
\usepackage{amsfonts}
\usepackage{amssymb}
\usepackage[spanish,es-noshorthands]{babel}
\usepackage[T1]{fontenc}
\usepackage{lmodern}
\usepackage{graphicx,hyperref}
\usepackage{tikz,pgf}
\usepackage{multicol}
\usepackage{subfig}
\usepackage[papersize={6.5in,8.5in},includeheadfoot,left=0.4in,right=0.4in,top=0.3in,bottom=0.2in]{geometry}
\usepackage{fancyhdr}
\pagestyle{fancy}
\fancyhead[LE]{\url{https://www.autistici.org/mathgerman}}
\fancyhead[RE]{}
\fancyhead[RO]{\textit{matematicas.german@gmail.com}}
\fancyhead[LO]{}

\author{Germ\'an Dar\'io Avenda\~no Ram\'irez~\thanks{Lic. Mat. U.D., M.Sc. U.N.}}
\title{\begin{minipage}{0.15\textwidth}\includegraphics[height=2cm]{Images/logo-colegio.png}
\end{minipage}\hfill \begin{minipage}{0.7\textwidth}\begin{center}
Quiz - Progresiones\\Cálculo $11^{\circ}$\end{center}
\end{minipage}\hfill
\begin{minipage}{0.15\textwidth}
\includegraphics[height=2cm]{SparkleShare/My-repository/LaTeX/Images/logo-sed.png} 
\end{minipage}}
\date{}

\begin{document}
\maketitle
Nombre: \hrulefill Curso: 110\underline{\hspace{12pt}}  Fecha: \underline{\hspace{2cm}}
\section*{Para recordar}
Una progresi\'on aritmética tiene como término general \fbox{$a_{n}=a_{1}+(n-1)d$}, donde $d$ es la distancia o diferencia que hay entre dos términos consecutivos.\\

Una progresión geométrica tiene como término genral \fbox{$a_{n}=a_{1}r^{n-1}$},
donde $r$ es la razón geométrica.
\begin{enumerate}
\item Halle los dos términos siguientes en las sucesiones indicadas y determine si son progresiones, en el caso que sean progresiones, determinar si son aritméticas o geométricas y hallar su término general ($a_{n}$)
\begin{enumerate}
\item 2, 5, 8, 11, 14, 17, \ldots \vspace*{40pt}
\item 0, 3, 8, 15, 24, \ldots\vspace*{40pt}
\item $\frac{1}{2}$, $\frac{2}{5}$, $\frac{3}{10}$, $\frac{4}{17}$, \ldots \vspace*{40pt}
\item 4, 8, 16, 32, \ldots \vspace*{40pt}
\end{enumerate}
\item Halle los siete primeros términos de una progresión aritmética:
\begin{enumerate}
\item cuyo primer término es -2 y su diferencia $d$ es 3\vspace*{30pt}
\item cuyo segundo término es 3 y su diferencia $d$ es 4\vspace*{30pt}
\end{enumerate}
\item Halle el término general $a_{n}$ de una progresión aritmética
\begin{enumerate}
\item cuyo primer término es 5 y su diferencia $d$ es $-3$.\vspace*{40pt}
\item cuyo primer término es 2 y su segundo término es 7.\vspace*{40pt}
\end{enumerate}
\item En una granja hay 75 pollos y cada día nacen 25. ¿cuántos habrá al cabo de 30 días si no muere ninguno?\vspace*{50pt}
\item Cada día me duplican el dinero que tengo y me dan 2 dólares más. Si el primer día tengo 15 dólares, construya la sucesión que indica el dinero que tengo cada día. Hágalo para una semana.
\end{enumerate}
\end{document}
