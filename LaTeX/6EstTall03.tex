\documentclass[10pt,twoside]{article}
\usepackage[utf8]{inputenc}
\usepackage{amsmath}
\usepackage{amsfonts}
\usepackage{amssymb}
\usepackage[spanish,es-noshorthands]{babel}
\usepackage[T1]{fontenc}
\usepackage{lmodern}
\usepackage{graphicx,hyperref}
\usepackage{tikz,pgf}
\usepackage{multicol}
\usepackage{subfig}
\usepackage[papersize={6.5in,8.5in},width=5.5in,height=7in]{geometry}
\usepackage{fancyhdr}
\pagestyle{fancy}
\fancyhead[LE]{\includegraphics[height=12pt]{Images/logo-colegio.png} Estadística $6^{\circ}$}
\fancyhead[RE]{}
\fancyhead[RO]{\textit{Germ\'an Avenda\~no Ram\'irez, Lic. U.D., M.Sc. U.N.}}
\fancyhead[LO]{}

\author{Germ\'an Avenda\~no Ram\'irez, Lic. U.D., M.Sc. U.N.}
\title{\begin{minipage}{.2\textwidth}
\includegraphics[height=1.75cm]{Images/logo-colegio.png}\end{minipage}
\begin{minipage}{.55\textwidth}
\begin{center}
Taller 03, Tablas de Frecuencia  \\
Estadística $6^{\circ}$
\end{center}
\end{minipage}\hfill
\begin{minipage}{.2\textwidth}
\includegraphics[height=1.75cm]{Images/logo-sed.png} 
\end{minipage}}
\date{}
\begin{document}
\maketitle
Nombre: \hrulefill Curso: \underline{\hspace*{44pt}} Fecha: \underline{\hspace*{2.5cm}}
\section*{Aprendo algo nuevo}
Observe cómo registraron la información sobre el peso de un grupo de estudiantes Milena y Simón.
\begin{center}
\begin{tabular}{|c|p{3.5cm}|p{6cm}|}
\hline 
Peso en kg & Número de alumnos (Frecuencia) & Parte del total de alumnos que tienen ese peso \\ 
\hline 
41 & 1 & 1/10 \\ 
\hline 
43 & 1 & 1/10 \\ 
\hline 
47 & 2 & 2/10 \\ 
\hline 
49 & 4 & 4/10 \\ 
\hline 
52 & 2 & 2/10 \\ 
\hline \hline
Total & 10 & 10/10=1 \\ \hline
\end{tabular} 
\end{center}
\subsection*{Actividad 1}
\begin{itemize}
\item ¿Cuántos datos se registraron en esa actividad?
\item ¿Cuántas estudiantes del total pesan 41 kilogramos?
\item ¿Cuántos pesan 49?
\end{itemize}
La tercera fila de esa tabla, da información sobre qué parte de la población en estudio o de una de muestra (parte de la población), corresponde a la característica analizada. Esos valores se conocen como frecuencias relativas.

Por ejemplo, la frecuencia relativa que corresponde a 47 kilogramos es 2/10, es decir de diez estudiantes dos pesan 47 kilogramos.
\begin{itemize}
\item ¿Cuál es la frecuencia absoluta de 49 kilogramos?
\item ¿Cuál es la frecuencia relativa del mismo peso? ¿Qué significa ese valor?
\end{itemize}
Milena y Simón siguen completando datos en la tabla.
Analiza y en tu cuaderno, completa la tabla.

\end{document}
