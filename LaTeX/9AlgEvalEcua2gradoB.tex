\documentclass[letterpaper,fleqn]{article}
\usepackage[spanish,es-noshorthands]{babel}
\usepackage[utf8]{inputenc} 
\usepackage[left=1cm, right=1cm, top=1.5cm, bottom=1.7cm]{geometry}
\usepackage{mathexam}
\usepackage{amsmath}
\usepackage{graphicx}

\ExamClass{\includegraphics[height=16pt]{Images/logo-sed.png} Álgebra $9^{\circ}$}
\ExamName{``Ecuación de 2o grado''}
\ExamHead{\includegraphics[height=16pt]{Images/logo-colegio.png} IEDAB}
\newcommand{\LineaNombre}{%
\par
\vspace{\baselineskip}
Nombre:\hrulefill \; Curso: \underline{\hspace*{48pt}} \; Fecha: \underline{\hspace*{2.5cm}} \relax
\par}
\let\ds\displaystyle

\begin{document}
\ExamInstrBox{
Respuesta sin justificar mediante procedimiento no será tenida en cuenta en la calificación. Escriba sus respuestas en el espacio indicado. Tiene 50 minutos para contestar esta prueba.}
\LineaNombre
\vspace*{12pt}

Conteste la primera pregunta a conciencia
\begin{enumerate}
 \item[0.] El tiempo que le he dedicado a preparar esta evaluación fue de:
 \end{enumerate}
 \vspace*{12pt}
 Existen dos métodos para solucionar una ecuación de segundo grado de la forma $ax^{2}+bx+c=0$, uno emplea la factorización y el otro la solución general cuya expresión es: 
$x=\dfrac{-b\pm \sqrt{b^{2}-4ac}}{2a}$
 Encuentre la solución de las siguientes ecuaciones por el método que Ud prefiera.
 \begin{enumerate}
 \item $x^{2}-49=0$ \noanswer
 \item $x^{2}-8x=0$ \noanswer
 \item $x^{2}+5x-36=0$ \noanswer
 \item $6x^{2}+x-12=0$\noanswer
 Solucione el siguiente problema planteando una ecuación de segundo grado que luego debe solucionar para responder la pregunta
 \item El área de un rectángulo es 119 cm$^{2}$ y su largo es 10 cm más que su ancho. Encuentre las dimensiones (largo y ancho) del rectángulo. \noanswer
 \end{enumerate}

\end{document}
