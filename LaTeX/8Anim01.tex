\documentclass[10pt,twoside]{article}
\usepackage[utf8]{inputenc}
\usepackage{amsmath}
\usepackage{amsfonts}
\usepackage{amssymb}
\usepackage[spanish,es-noshorthands]{babel}
\usepackage[T1]{fontenc}
\usepackage{lmodern}
\usepackage{graphicx,hyperref}
\usepackage{tikz,pgf}
\usepackage{multicol}
\usepackage{subfig}
\usepackage[papersize={6.5in,8.5in},width=5.5in,height=7in]{geometry}
\usepackage{fancyhdr}
\pagestyle{fancy}
\fancyhead[LE]{\includegraphics[height=12pt]{Images/logo-colegio.png} Cálculo $11^{\circ}$}
\fancyhead[RE]{}
\fancyhead[RO]{\textit{Germ\'an Avenda\~no Ram\'irez, Lic. U.D., M.Sc. U.N.}}
\fancyhead[LO]{}

\author{Germ\'an Avenda\~no Ram\'irez, Lic. U.D., M.Sc. U.N.}
\title{\begin{minipage}{.2\textwidth}
\includegraphics[height=1.75cm]{Images/logo-colegio.png}\end{minipage}
\begin{minipage}{.55\textwidth}
\begin{center}
Animaplano 01, 8$^{\circ}$,  \\
Cálculo $11^{\circ}$
\end{center}
\end{minipage}\hfill
\begin{minipage}{.2\textwidth}
\includegraphics[height=1.75cm]{Images/logo-sed.png} 
\end{minipage}}
\date{}
\begin{document}
\maketitle
Nombre: \hrulefill Curso: \underline{\hspace*{44pt}} Fecha: \underline{\hspace*{2.5cm}}
\section*{Cuestionario}
Conteste al frente de cada pregunta, haciendo los procedimientos respectivos
\begin{enumerate}
 \item Halle 1/4 del número 100
 \item Siendo $ab=57$, si $a=3$, $b=$?
 \item Si el dividendo es 56, el cociente es 2, el divisor es:
 \item Sea $A=\{7,5,4\}$, $B=\{5,7,6\}$, la suma de los elementos de $A\cup B$ es:
 \item En años, 3 décadas + 1 año.
 \item Halle y sume los números primos entre 23, 15, 27, 12, 21 y 2
 \item Multiplique por 5, las veces en que está el 5 en 35
 \item En unidades, 1/2 centena + 1 docena
 \item El doble de la mitad del número 66
 \item Sume $2^{2}+3^{3}+4^{2}-1^{2}=$
 \item $(-11)(2)(-2)=$
 \item Si $a=11$, luego $a+a+a=$
 \item Si $m+n=31$, ¿cuál será la nueva suma si $m$ aumenta en 7 y $n$ aumenta en 13?
 \item $(3\times4!)-$ raíz cuadrada de 100
\end{enumerate}

\end{document}
