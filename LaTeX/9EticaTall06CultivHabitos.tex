\documentclass[10pt,twoside]{article}
\usepackage[utf8]{inputenc}
\usepackage{amsmath}
\usepackage{amsfonts}
\usepackage{amssymb}
\usepackage[spanish,es-noshorthands]{babel}
\usepackage[T1]{fontenc}
\usepackage{lmodern}
\usepackage{graphicx,hyperref}
\usepackage{tikz,pgf}
\usepackage{multicol}
\usepackage{subfig}
\usepackage[papersize={6.5in,8.5in},width=5.5in,height=7in]{geometry}
\usepackage{fancyhdr}
\pagestyle{fancy}
\fancyhead[LE]{\includegraphics[height=12pt]{Images/logo-colegio.png} Ética $9^{\circ}$}
\fancyhead[RE]{}
\fancyhead[RO]{\textit{Germ\'an Avenda\~no Ram\'irez, Lic. U.D., M.Sc. U.N.}}
\fancyhead[LO]{}

\author{Germ\'an Avenda\~no Ram\'irez, Lic. U.D., M.Sc. U.N.}
\title{\begin{minipage}{.2\textwidth}
\includegraphics[height=1.75cm]{Images/logo-colegio.png}\end{minipage}
\begin{minipage}{.55\textwidth}
\begin{center}
Taller 06, Cultivo de hábitos  \\
Étitica $9^{\circ}$
\end{center}
\end{minipage}\hfill
\begin{minipage}{.2\textwidth}
\includegraphics[height=1.75cm]{Images/logo-sed.png} 
\end{minipage}}
\date{}
\begin{document}
\maketitle
Nombre: \hrulefill Curso: \underline{\hspace*{44pt}} Fecha: \underline{\hspace*{2.5cm}}
\section*{Lo que s\'{e}}
\subsection*{Antes de comenzar}
\begin{enumerate}
\item ¿Recuerdas la fábula de Esopo sobre la liebre y la tortuga?\\
Lee la siguiente versión, compárala con la original y saca tus propias conclusiones. Luego, desarrolla las actividades.
\end{enumerate}
\subsubsection*{La liebre y la tortuga}
“Una tortuga y una liebre siempre discutían sobre quién
era más rápida. Para dirimir el asunto, decidieron correr
una carrera. Acordaron la ruta, las condiciones y empezó
la competencia. La liebre, sabiéndose veloz, arrancó a toda
velocidad y corrió rauda durante algún trecho. Luego pensó
que llevaba buena ventaja y decidió sentarse bajo un árbol
para descansar, recuperar fuerzas y continuar luego.

Pronto se durmió. Entre tanto la tortuga con su paso lento pero
constante y persistente, la alcanzó, la superó y terminó llegando en
primer lugar, declarándose vencedora indiscutible. Pero la versión
no termina allí. La liebre, decepcionada consigo misma por haber
perdido debido a su pereza y descuido, reflexionó sobre el asunto
y reconoció sus errores, prometiéndose que no la vencerían nunca
más. Desafió nuevamente a la tortuga y corrió con tenacidad de
principio a fin, sin dar por sentada su ventaja y esta vez su triunfo fue
evidente. Pero la tortuga no se conformaba; sabía que como estaba
planteada la carrera, no ganaría. Entonces propuso cambiar la ruta a
lo que accedió la liebre. Se inició una nueva competencia con ventaja
evidente de la liebre que corrió diligente hasta encontrar un río y se
paró frente a él porque no sabía nadar. La tortuga llegó a su lado
al rato, entró al agua y nadando atravesó el río, continuó la marcha
dejando atrás a la liebre y finalmente ganó la carrera”.\footnote{\url{www.epconsultores.com}. Adaptación.
}
\begin{enumerate}
\item Responde en tu cuaderno: En ambos casos,
¿por qué triunfó la tortuga en la carrera?
\item ¿Qué similitudes y diferencias encuentras
entre tu actitud y la de la tortuga?
\item En equipos de cuatro integrantes, propongan
una o más moralejas asociadas a la fábula
anterior, discútanlas y planteen estrategias
para aplicarlas en el curso.
\end{enumerate}
En el módulo anterior, se definieron los hábitos como las virtudes
que facilitan a las personas actuar correctamente, hacer el bien
y superar las adversidades. Toda acción con consecuencias
favorables amerita repetirse y es allí donde tienen fundamento
los hábitos. Éstos se cultivan cuando las personas reflexionan
continuamente en torno a ellos, no son conformistas y se
muestran dispuestas, disciplinadas, laboriosas y persistentes,
cualidades que a su vez se asocian a numerosas habilidades como
lo demostró la tortuga frente a su adversaria.
\end{document}
