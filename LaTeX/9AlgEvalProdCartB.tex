\documentclass[letterpaper,fleqn]{article}
\usepackage[spanish,es-noshorthands]{babel}
\usepackage[utf8]{inputenc} 
\usepackage[papersize={6.5in,8.5in},left=1cm, right=1cm, top=1.5cm, bottom=1.7cm]{geometry}
\usepackage{mathexam}
\usepackage{amsmath}
\usepackage{graphicx}
\usepackage{tikz,pgf}

\ExamClass{\includegraphics[height=16pt]{Images/logo-sed.png} Álgebra $9^{\circ}$}
\ExamName{Producto cartesiano}
\ExamHead{\includegraphics[height=16pt]{Images/logo-colegio.png} IEDAB}
\newcommand{\LineaNombre}{%
\par
\vspace{\baselineskip}
Nombre:\hrulefill \; Curso: \underline{\hspace*{48pt}} \; Fecha: \underline{\hspace*{2.5cm}} \relax
\par}
\let\ds\displaystyle

\begin{document}
\ExamInstrBox{
Respuesta sin justificar mediante procedimiento no será tenida en cuenta en la calificación. Escriba sus respuestas en el espacio indicado. Tiene 30 minutos para contestar esta prueba.}
\LineaNombre
\begin{enumerate}
 \item Juan Esteban posee 3 pantalones, uno negro, otro café y uno azul y dispone de 4 camisas, una blanca, otra negra, una café y una color azul. Responda
 \begin{enumerate}
 \item ¿De cuántas maneras diferentes puede vestirse Juan Esteban con sus 3 pantalones y sus 4 camisas? \underline{\hspace{1in}}
 \item Represente mediante un conjunto o en un diagrama sagital todas las posibilidades que encuentre.¿Cómo se denomina este conjunto? \noanswer
\end{enumerate}
\item Dados los conjuntos $A=\{2,3,5\}$ y $B=\{1,2,3,4,5\}$ encuentre:
\begin{enumerate}
\item El producto cartesiano $A\times B$\\
$A \times B=$ \noanswer[.5in]
\item Encuentre todas las parejas de la relación $R_{1}=\{(x,y), \: x \in A, y \in B \; \wedge \; y=x-1\}$ y grafíquelas en el plano cartesiano\\

$R_{1}=\{$ \noanswer[.5in]
\newpage
\begin{center}
\documentclass{standalone}
\usepackage[utf8]{inputenc}
\usepackage[spanish,es-noshorthands]{babel}
\usepackage{tikz}
\usetikzlibrary{babel,arrows,calc}

\begin{document}
\begin{tikzpicture}[scale=.9]
\def\t{3}
\def\u{3}
\draw[style=help lines] (-\t,-\u) grid (\t,\u);
\draw[<->] (-\t,0)--(\t,0) node[right] {A};
\foreach \x in {1} \draw[shift={(\x,0)},color=black] (0pt,2pt) -- (0pt,-2pt) node[below] {\footnotesize $\x$};
\draw[<->](0,-\u)--(0,\u)node[above]{B};
\foreach \y in {1} \draw[shift={(0,\y)},color=black] (2pt,0)--(-2pt,0) node[left]{\footnotesize $\y$};
\end{tikzpicture}
\end{document}
\end{center}
\item Encuentre el dominio, codominio y rango de la relación $R_{1}$
\begin{enumerate}
\item Dominio de $R_{1}=$
\item Codominio de $R_{1}=$
\item Rango de $R_{1}=$
\end{enumerate}
\end{enumerate}
\item Dado el conjunto $F=\{-4\leq x<4\}$
\begin{enumerate}
\item ¿Cuántos elementos (parejas) tiene el producto cartesiano $F\times F$? \underline{\hspace*{1in}}
\item Encuentre las parejas pertenecientes a la relación $R_{2}=\{(x,y): x,y \in F \wedge y=\pm x\}$ y ubíquelas en el plano $F\times F$\\
$R_{2}=\{$
\begin{center}
\begin{tikzpicture}[scale=.9]
\draw[dotted,style=help lines] (-4,-4) grid (3,3);
\draw[<->] (-4.2,0)--(3.2,0) node[right] {A};
\foreach \x in {1} \draw[shift={(\x,0)},color=black] (0pt,2pt) -- (0pt,-2pt) node[below] {\footnotesize $\x$};
\draw[<->](0,-3.2)--(0,4.2)node[left]{B};
\foreach \y in {1} \draw[shift={(0,\y)},color=black] (2pt,0)--(-2pt,0) node[left]{\footnotesize $\y$};
\end{tikzpicture}
\end{center}
\end{enumerate}
 \end{enumerate}

\end{document}
