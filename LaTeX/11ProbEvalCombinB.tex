\documentclass[letterpaper,fleqn]{article}
\usepackage[spanish,es-noshorthands]{babel}
\usepackage[utf8]{inputenc} 
\usepackage[papersize={6.5in,8.5in}, left=1cm, right=1cm, top=1.5cm, bottom=1.7cm]{geometry}
\usepackage{mathexam}
\usepackage{amsmath}
\usepackage{graphicx}

\ExamClass{\includegraphics[height=16pt]{Images/logo-sed.png} Probabilidad $11^{\circ}$}
\ExamName{``Combinatoria''}
\ExamHead{\includegraphics[height=16pt]{Images/logo-colegio.png} IEDAB}
\newcommand{\LineaNombre}{%
\par
\vspace{\baselineskip}
Nombre:\hrulefill \; Curso: \underline{\hspace*{48pt}} \; Fecha: \underline{\hspace*{2.5cm}} \relax
\par}
\let\ds\displaystyle

\begin{document}
\ExamInstrBox{
Respuesta sin justificar mediante procedimiento no será tenida en cuenta en la calificación. Escriba sus respuestas en el espacio indicado. Tiene 45 minutos para contestar esta prueba.}
\LineaNombre
\section*{Permutaciones}
\begin{enumerate}
 \item ¿De cuántas maneras pueden ordenarse 7 libros en un librero si:
 \begin{enumerate}
 \item pueden ordenarse como se desee\noanswer
 \item hay 3 libros que deben estar juntos \noanswer
 \item hay 2 libros que deben estar al final \noanswer
\end{enumerate}   
\item ¿Cuántos números de cinco dígitos diferentes pueden formarse con los dígitos 1, 2, 3, \ldots , 9 si: 
\begin{enumerate}
\item el número debe ser non\noanswer
\newpage
\item si los dos primeros dígitos de cada número tienen que ser pares\noanswer
\end{enumerate} 
\item ¿De cuántas maneras se pueden sentar 5 personas en un sofá si el sofá sólo tiene 3 asientos?\noanswer
\section*{Combinaciones}
\item A partir de 5 profesionales de la estadística y 6 economistas, se va a formar un grupo que conste de 3 profesionales de la
estadística y 2 economistas. ¿Cuántos comités diferentes pueden formarse si: 
\begin{enumerate}
\item no hay restricción alguna\noanswer
\item hay 2 profesionales de la estadística que deben estar en el comité\noanswer
\item hay un economista que no puede formar parte del comité\noanswer
\end{enumerate}
\item ¿De cuántas maneras se pueden seleccionar 6 de 10 preguntas?\noanswer
 \end{enumerate}

\end{document}
