\documentclass[fleqn]{article}
\usepackage[spanish,es-noshorthands]{babel}
\usepackage[utf8]{inputenc} 
\usepackage[left=1cm, right=1cm, top=1.5cm, bottom=1.7cm]{geometry}
\usepackage{mathexam}
\usepackage{amsmath}
\usepackage{graphicx}
\usepackage{tikz,pgf}

\ExamClass{\includegraphics[height=16pt]{Images/logo-sed.png} Aritmética $6^{\circ}$}
\ExamName{Evaluación Operaciones en N}
\ExamHead{\includegraphics[height=16pt]{Images/logo-colegio.png} IEDAB}
\newcommand{\LineaNombre}{%
\par
\vspace{\baselineskip}
Nombre:\hrulefill \; Curso: \underline{\hspace*{48pt}} \; Fecha: \underline{\hspace*{2.5cm}} \relax
\par}
\let\ds\displaystyle

\begin{document}
\ExamInstrBox{
Respuesta sin justificar mediante procedimiento no será tenida en cuenta en la calificación. Escriba sus respuestas en el espacio indicado. Tiene 45 minutos para contestar esta prueba.}
\LineaNombre
\begin{enumerate}
\begin{minipage}{.5\textwidth}
   \item ¿Cuál es el área del terreno en metros cuadrados? \noanswer 
\end{minipage}\hfill
\begin{minipage}{.5\textwidth}
   \begin{tikzpicture}
 \draw (0,0) rectangle (4,3);
 \node[below] at (2,0){40m};
 \node[right] at (4,1.5){30m} ;
\end{tikzpicture}
\end{minipage}
\item Si el terreno cuesta 
 

\end{enumerate}
\end{document}
