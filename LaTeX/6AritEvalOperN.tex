\documentclass[fleqn]{article}
\usepackage[spanish,es-noshorthands]{babel}
\usepackage[utf8]{inputenc} 
\usepackage[left=1cm, right=1cm, top=1.5cm, bottom=1.7cm]{geometry}
\usepackage{mathexam}
\usepackage{amsmath}
\usepackage{graphicx}
\usepackage{tikz,pgf}

\ExamClass{\includegraphics[height=16pt]{Images/logo-sed.png} Aritmética $6^{\circ}$}
\ExamName{Evaluación Operaciones en N}
\ExamHead{\includegraphics[height=16pt]{Images/logo-colegio.png} IEDAB}
\newcommand{\LineaNombre}{%
\par
\vspace{\baselineskip}
Nombre:\hrulefill \; Curso: \underline{\hspace*{48pt}} \; Fecha: \underline{\hspace*{2.5cm}} \relax
\par}
\let\ds\displaystyle

\begin{document}
\ExamInstrBox{
Respuesta sin justificar mediante procedimiento no será tenida en cuenta en la calificación. Escriba sus respuestas en el espacio indicado. Tiene 45 minutos para contestar esta prueba.}
\LineaNombre
\begin{enumerate}
\begin{minipage}{.5\textwidth}
   \item ¿Cuál es el área del terreno en metros cuadrados?  
\end{minipage}\hfill
\begin{minipage}{.5\textwidth}
   \begin{tikzpicture}
 \draw (0,0) rectangle (4,3);
 \node[below] at (2,0){40m};
 \node[right] at (4,1.5){30m} ;
\end{tikzpicture}
\end{minipage}
\noanswer
\item Si el terreno cuesta \$23'760\,000, ¿cuál es el valor de cada metro cuadrado?\noanswer
\item Si un padre quiere repartir el terreno entre sus 7 hijos, ¿es posible dividir el terreno en 7 partes iguales? ¿Por qué? \noanswer
\item Realicen las siguientes divisiones y diga si son exactas o no. 
\begin{enumerate}
 \item $168\div12=$
 \item $1278\div18=$
\end{enumerate}
\item De la división inexacta del punto anterior, ubique los términos en el siguiente gráfico:
\begin{tabbing}
DIVIDENDO \= = \= (DIVISOR $\times$ COCIENTE) \= + \= RESIDUO\\
\tikz \draw (0,0) rectangle (2.2,.5); \> = \> \tikz \draw (0,0) rectangle (4.5,.5);  \> + \> \tikz \draw (0,0)rectangle(1.8,.5); 
\end{tabbing} \noanswer
\item Resuelve la adición y escribe las dos restas asociadas a cada suma:
\begin{enumerate}
 \item $24+89=$\noanswer
 \item $43+98=$\noanswer
\end{enumerate}
\item Resuelve y escribe la suma y la otra resta asociadas a las siguientes restas
\begin{enumerate}
 \item $98-45=$\noanswer
 \item $205-128=$\noanswer
\end{enumerate}
\item 
\end{enumerate}
\end{document}
