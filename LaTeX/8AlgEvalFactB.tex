\documentclass[letterpaper,fleqn]{article}
\usepackage[spanish,es-noshorthands]{babel}
\usepackage[utf8]{inputenc} 
\usepackage[papersize={6.5in,8.5in},left=1cm, right=1cm, top=1.5cm, bottom=1.7cm]{geometry}
\usepackage{mathexam}
\usepackage{amsmath}
\usepackage{graphicx}
\usepackage{multicol}
\ExamClass{\includegraphics[height=16pt]{Images/logo-sed.png} Álgebra $8^{\circ}$}
\ExamName{Quiz-Factorización}
\ExamHead{\includegraphics[height=16pt]{Images/logo-colegio.png} IEDAB}
\newcommand{\LineaNombre}{%
\par
\vspace{\baselineskip}
Nombre:\hrulefill \; Curso: \underline{\hspace*{48pt}} \; Fecha: \underline{\hspace*{2.5cm}} \relax
\par}
\let\ds\displaystyle

\begin{document}
\ExamInstrBox{
Los puntos 1 y 3 de esta prueba valen una unidad y el punto 2, dos unidades en la calificación. El hecho de presentar la evaluación vale una unidad en la calificación.}
\LineaNombre
\begin{enumerate}
 \item Clasifique los siguientes números en primos o compuestos. Los números que resulten ser compuestos, factorícelos en sus factores primos:
 \begin{enumerate}
 \item 133\noanswer
 \item 107\noanswer
 \item 153\noanswer
 \item 139\noanswer
 \end{enumerate}
 \item Factorice completamente:
 \begin{enumerate}
 \item $6x+2y$\noanswer
 \item $20xy-15x$\noanswer
 \newpage
 \item $8x^{4}+12x^{3}-24x^{2}$\noanswer
 \item $15x^{3}y^{2}+20x^{2}y+35x^{4}y^{3}$\noanswer
 \item $3x(2a+b)-2y(2a+b)$\noanswer
 \end{enumerate}
 \item Factorice usando agrupación de términos:
 \begin{enumerate}
 \item $ax-2bx+ay+2by$\noanswer
 \item $ax^{2}-x^{2}+2a-2$\noanswer
 \end{enumerate}
 \end{enumerate}
\end{document}
