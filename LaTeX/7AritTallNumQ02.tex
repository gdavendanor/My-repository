\documentclass[10pt,twoside]{article}
\usepackage[utf8]{inputenc}
\usepackage{amsmath}
\usepackage{amsfonts}
\usepackage{amssymb}
\usepackage[spanish,es-noshorthands]{babel}
\usepackage[T1]{fontenc}
\usepackage{lmodern}
\usepackage{graphicx,hyperref}
\usepackage{tikz,pgf}
\usepackage{marvosym}
\usepackage{multicol}
\usepackage[papersize={6.5in,8.5in},width=5.5in,height=7in]{geometry}
\usepackage{fancyhdr}
\pagestyle{fancy}
\fancyhead[LE]{Colegio Arborizadora Baja}
\fancyhead[RE]{PEI:``Hacia una cultura para el desarrollo sostenible''}
\fancyfoot[RO]{\Email iedabgerman@autistici.org}
\fancyhead[LO]{\url{www.autistici.org/mathgerman}}
\fancyfoot[RE]{\Email cedarborizadoraba19@redp.edu.co}
\fancyfoot[LE]{Calle 59I \#44A - 02 \Telefon 7313994 - 7313995}
\fancyhead[RO]{Nit 830024976-8, Código DANE 11100103084-8}

\author{Germ\'an Avenda\~no Ram\'irez~\thanks{Lic. Mat. U.D., M.Sc. U.N.}}
\title{\begin{minipage}{.2\textwidth}
\includegraphics[height=1.75cm]{Images/logo-colegio.png}\end{minipage}
\begin{minipage}{.55\textwidth}
\begin{center}
Taller 2 Racionales $\mathbb{Q}$\\
Matemáticas $7^{\circ}$
\end{center}
\end{minipage}\hfill
\begin{minipage}{.2\textwidth}
\includegraphics[height=1.75cm]{Images/logo-sed.png} 
\end{minipage}}
\date{}
\begin{document}
\maketitle
\subsection*{Continuación Nivel I}
\begin{enumerate}
\item Ordena de mayor a menor los siguientes grupos de fracciones y explica como lo has hecho:
\begin{enumerate}
\begin{multicols}{4}
\item $\frac{9}{8}$, $\frac{7}{8}$, $\frac{3}{8}$, $\frac{17}{8}$, $\frac{1}{8}$
\item $\frac{6}{9}$, $\frac{6}{14}$, $\frac{6}{7}$, $\frac{6}{11}$, $\frac{6}{8}$
\item $\frac{3}{6}$, $\frac{3}{4}$, $\frac{5}{10}$
\item $\frac{8}{4}$, $\frac{4}{2}$, $\frac{6}{3}$
\end{multicols}
\end{enumerate}
\item Realiza las siguientes operaciones simplificando los resultados.
\begin{enumerate}
\begin{multicols}{3}
\item $\frac{3}{4}+\frac{5}{6}+\frac{6}{8}=$
\item $\frac{6}{10}-\frac{2}{8}=$
\item $\frac{7}{8}+\frac{3}{6}-\frac{5}{12}=$
\end{multicols}
\end{enumerate}
\item Realiza las siguientes operaciones simplificando los resultados.
\begin{enumerate}
\begin{multicols}{3}
\item $\frac{5}{8}\cdot\frac{3}{9}\cdot\frac{4}{6}=$
\item $\frac{5}{4}\cdot\frac{3}{4}\cdot\frac{7}{4}=$
\item $\frac{7}{9}\div\frac{6}{4}=$
\end{multicols} 
\end{enumerate}
\item María estudia 3/4 de hora de matemáticas, 2/3 de hora de natural, 4/6 de hora de lenguaje y 3/8 de hora de inglés. ¿Cuántas horas estudia María?
\item Pedro ha recogido 7/2 kilos de fresas. Gasta 3/4 de kilo en hacer un pastel. ¿Qué cantidad de fresas le queda todavía?
\item Hemos comprado 8 botes de 3/4 de kilo de melocotón. ¿Cuántos kilos de melocotón hemos comprado?
\item Tenemos 24 litros de vino y lo queremos embotellar en botellas de 3/4 de litro. ¿Cuántas botellas obtendremos?
\end{enumerate}

\subsection*{Nivel II}
\begin{enumerate}
\item Ordena de mayor a menor las siguientes fracciones:
\begin{enumerate}
\nopagebreak
\begin{multicols}{2}
\item $\frac{8}{7}$, $\frac{9}{8}$, $\frac{5}{4}$
\item $\frac{4}{5}$, $\frac{5}{6}$, $\frac{8}{10}$, $\frac{3}{4}$
\end{multicols}
\end{enumerate}
\item Realiza las siguientes operaciones simplificando el resultado:
\begin{enumerate}
\begin{multicols}{2}
\item $\frac{1}{5}+\frac{3}{4}-\frac{2}{8}+3=$
\item $4+\frac{1}{3}-\frac{3}{5}-\frac{8}{6}=$
\end{multicols}
\end{enumerate}
\item Realiza las siguientes operaciones simplificando el resultado:
\begin{enumerate}
\begin{multicols}{2}
\item $\frac{3}{4}\cdot\frac{5}{2}\div\frac{4}{6}=$
\item $7\div\frac{3}{5}\cdot\frac{3}{4}\cdot\frac{1}{3}=$
\end{multicols}
\end{enumerate}
\item Calcula $x$ en cada caso.
\begin{enumerate}
\begin{multicols}{3}
\item $\frac{3}{4}\cdot x=\frac{12}{20}$
\item $\frac{2}{5}\cdot x=\frac{2}{15}$
\item $4\cdot x=\frac{4}{3}$
\end{multicols}
\end{enumerate}
\item Los 2/9 del alumnado de un centro escolar participa en el curso de Educación Vial. Si participan 160 estudiantes. ¿Cuántos alumnos hay en el centro? En otro centro participan 600 alumnos en dicho curso, si estos son los 3/5 del total, ¿cuántos alumnos hay
en este otro centro?
\item En la biblioteca municipal, 27 de los 47 libros de la biblioteca juvenil son de aventuras. Cristina dice que son los 9/15 de los libros, Jorge afirma que son los 27/45, y Carmen que son los 3/5. ¿Quién tiene razón?
\item Nuria y Arturo participan en una carrera, en la primera hora habían recorrido los 3/8 del trayecto y en la segunda hora 3/10 del trayecto. ¿Qué fracción del trayecto han realizado ya? ¿Han llegado a la mitad de la carrera?
\item Marta estudia 3 asignaturas en una carrera de ingeniería. Dedica 1/4 del tiempo de estudio para preparar la primera asignatura y 2/3 para estudiar la segunda. ¿Qué fracción del tiempo de estudio dedica para preparar la tercera asignatura?
\item La cuarta parte del presupuesto de una empresa se dedica a la partida de gastos. La mitad de esta partida se dedica a la descontaminación del medio ambiente. ¿Qué parte del presupuesto se dedica al medio ambiente?
\item Angela leyó las 3/4 partes del Romancero Gitano. Hoy ha leído la mitad de lo que le quedaba. ¿Qué fracción del libro le queda todavía por leer?
\item En una representación teatral, participan 12 alumnos que son los 2/5 de la clase de primero de ESO. ¿Cuántos alumnos tiene esta clase?
\item Juan pagó como entrada 1/3 del precio de una bicicleta. Un mes más tarde pagó 2/5 del precio total. ¿Qué fracción de su coste ha pagado?
\item En una carrera atlética los deportistas han corrido 700 metros, que representan los 7/12 del trayecto total. ¿Qué longitud tiene el trayecto de esa carrera? ¿Cuántos metros les quedan por recorrer?
\end{enumerate}
\end{document}
