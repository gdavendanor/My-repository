\documentclass[10pt]{article}
\usepackage[utf8]{inputenc}
\usepackage{amsmath}
\usepackage{amsfonts}
\usepackage{amssymb}
\usepackage[spanish,es-noshorthands]{babel}
\usepackage[T1]{fontenc}
\usepackage{lmodern}
\usepackage{graphicx,hyperref}
\usepackage{tikz,pgf}
\usepackage{multicol}
\usepackage{subfig}
\usepackage[papersize={6.5in,8.5in},includeheadfoot,left=0.4in,right=0.4in,top=0.3in,bottom=0.2in]{geometry}
\usepackage{fancyhdr}
\pagestyle{fancy}
\fancyhead[LE]{\url{http://www.autistici.org/mathgerman}}
\fancyhead[RE]{}
\fancyhead[RO]{matematicas.german@gmail.com}
\fancyhead[LO]{}

\author{Germ\'an Avenda\~no Ram\'irez~\thanks{Lic. Mat. U.D., M.Sc. U.N.}}
\title{\begin{minipage}{0.15\textwidth}\includegraphics[height=2cm]{Images/logo-colegio.png}
\end{minipage}\hfill \begin{minipage}{0.7\textwidth}\begin{center}
Taller 02\\
Función lineal\\Álgebra $9^{\circ}$\end{center}
\end{minipage}\hfill
\begin{minipage}{0.15\textwidth}
\includegraphics[height=2cm]{Images/logo-sed.png} 
\end{minipage}}
\date{}

\begin{document}
\maketitle
Nombre: \hrulefill Curso: 90\underline{\hspace{12pt}}  Fecha: \underline{\hspace{2cm}}\\
\section*{Funciones}
\subsection*{Introducci\'{o}n hist\'{o}rica}
Hay muchos estudios acerca de la historia de las matemáticas, pero muy pocos se han realizado sobre el tema específico de funciones.\\

El concepto de función, al igual que los conceptos de número y medida, se encuentran en la base de gran parte de las matemáticas que el hombre ha desarrollado a través de la historia.\\

Parece que el nacimiento del concepto de función se sitúa a mediados del siglo XVII, ya que en este siglo se formulan las primeras definiciones y es cuando aparece por primera vez el término “función”, pero de una manera muy restringida.\\

Para el estudio de la historia del concepto de función Youshkevitch (1976), distingue tres grandes períodos:
\begin{enumerate}
\begin{multicols}{3}
\item[a] El mundo antiguo
\item[b] La edad media
\item[c] La edad moderna
\end{multicols}
\end{enumerate}
EL MUNDO ANTIGUO: En esta época aparecen estudios particulares de dependencia entre cantidades, pero no aparecen nociones generales, sobre variables y funciones, por esta razón difícilmente puede hablarse de funciones en general, aunque si hay muchas ideas que sirvieron como base para el concepto completo de función.\\

Algunas civilizaciones antiguas como los Babilonios, Egipcios y Griegos aportaron algo para el desarrollo de este concepto aunque se dedicaron más al estudio de la aritmética y la geometría en casos particulares de aplicación práctica.\\

Sin embargo, fue a partir de mediciones astronómicas, hechas por los indios\footnote{Natural de la India, no debe confundirse con indígena, término que se refiere a los nativos americanos} ,
que el concepto de función hace sus primeras apariciones en el mundo de la
matemática.\\

Ya en la edad media se hacen los primeros intentos sobre el tema, aparecen
más explícitas ciertas nociones de dependencia entre dos cantidades variables, pero de una manera verbal o mediante un gráfico. Se inicia con los Arabes sin poder hablar aún de un cambio substancial. En esta edad podemos mencionar las escuelas de Oxford y París como pioneros del cambio, especialmente con los tratados sobre las representaciones geométricas de Nicolás Oresma.\\

Por último la edad moderna que se inicia a fines de siglo XVI, se puede considerar, como dijimos anteriormente, el escenario de la aparición del concepto de función con aplicaciones más profundas y más generales específicamente es en siglo XVII con los trabajos de Galileo, Descartes, Fermat, Newton, Leibtnitz, Gregory, Euler y Bernoulli.\\

Es así como el término “función” aparece por primera vez en un manuscrito de
Leibnitz de 1673, y en 1718 Bernoulli da una definición más explícita del concepto de función y en 1740 Euler perfecciona la definición y da la nueva notación $f(x)$ que es la que perdura hasta nuestros días.\\

Por último, con los trabajos de Lagrange, Fourier y Dirichlet se clasifica el concepto de función y con el advenimiento de la teoría de conjuntos se le generaliza hasta adquirir la forma en que hoy en día se presenta este fundamental concepto.
\subsubsection*{Actividad 1}
Haga una s\'{i}ntesis de cada una de las etapas de desarrollo del concepto de funci\'{o}n en su cuaderno.
\subsection*{Funci\'{o}n lineal}
\subsubsection*{Actividad 2 - En parejas}
En una finca a un obrero se le paga por cada metro cuadrado que trabaje, para esto, el patrón tiene la siguiente tabla que le permite pagar sin hacer demasiadas cuentas, o sea, con solo mirar los datos de la tabla.
\begin{center}
\begin{tabular}{|c|c|c|c|c|c|c|c|c|c|c|c|c|c|c|}
\hline 
$X(m^{2})$ & 1 & 2 & 3 & 4 & 5 & 6 & 7 & 8 & 9 & 10 & 15 & 20 & 25 & 30 \\ 
\hline 
Y(\$) & 30 & 60 & 90 & 120 & 150 & 180 & 210 & 240 & 270 & 300 & 450 & 600 & 750 & 900 \\ 
\hline 
\end{tabular} 
\end{center}
En esta tabla la $x$ representa los metros cuadrados trabajados y, $y=f(x)$ los pesos pagados\\

Con base en la anterior situación realizamos los siguientes pasos:
\begin{enumerate}
\item Conformamos un conjunto de parejas donde los primeros elementos sean los metros trabajados y los segundos los pesos pagados.
\item Buscamos el operador multiplicativo que al aplicarlo a la $x$ en cada caso nos dé la $y$, diciendo si es ampliador o reductor.
\item Con base en las parejas, o en la tabla decimos si las magnitudes metros cuadrados y pesos son directamente proporcionales, y damos las razones.
\item En el plano cartesiano graficamos las parejas formadas y unimos éstas por medio de una línea continua, para ampliar el dominio a los reales.
\item De acuerdo con el factor que buscamos en el paso 2, completamos las siguientes parejas en el cuaderno, reemplazando el signo interrogación.
\[(0,?),\left(\frac{1}{2},?\right),\left(\frac{7}{2},?\right),(11,?),(18,?).\]
\item Con ayuda del operador encontrado en el paso 2 decimos si esta gráfica nos representa una función o no. Justificamos la respuesta.
\end{enumerate}
\subsubsection*{Actividad 3 - Individual}
En la situación anterior llegué junto con mi compañero, a la fórmula para el pago según los metros trabajados mediante la función $f(x) = 30x$.
\begin{enumerate}
\item[7.] Observando detenidamente la gráfica, intuyo si ésta corresponde a una función lineal o no.
\item[8.] Tomo los valores $x =2$ y $x = 4$ y hallo $f(2)$ y $f(4)$ por separado; seguidamente hallo $f(2+4)$ ó sea $f(6)$.\\

Hago lo mismo con los valores $x = 3$ y $x = 4$.
\item[9.] Ahora tomo los valores $k = 2$ y $x =4$ y realizo lo siguiente:
Calcular $f(kx)$ luego calcular $kf(x)$ para la función $f(x)=30x$\\

Hago lo mismo en los valores $k=\frac{1}{2}$ y $x=2$
\item[10.] Trato de sacar alguna conclusión de lo realizado en los apartados 8 y 9.
\end{enumerate}
\subsubsection*{Actividad 4 - En parejas}
Consideremos la función
\[f(x)=2x-1\]
\begin{enumerate}
\item[11.] Hallamos los valores $f(0)$, $f(1)$, $f(-1)$, $f(-2)$, $f(2)$, $f(-3)$, $f(3)$, $f(-4)$, $f(4)$.
\item[12.] Construimos una tabla con estos valores y hacemos la gráfica de esta función. A continuación unimos estos puntos.
\item[13.] Hallamos: $f(2)$, $f(3)$, luego hallamos $f(2+3) = f(5)$
\item[14.] Hallamos $f(1)$ y $f(5)$ luego $f(1+5)=f(6)$
\item[15.] Ahora encontramos $f(kx)$ y $kf(x)$ para $x = 2$ y $k=3$.\\
Luego para $k=2$ y $x=\frac{1}{2}$.
\item[16.] Sacamos algunas conclusiones de los apartados 13 y 14.
\item[17.] Comparamos estas conclusiones con las sacadas en el apartado 2, 3 y 4 de la actividad 2.
\end{enumerate}
\subsubsection*{Concluyamos lo aprendido}
La función lineal se representa por $f(x) = kx$ y se caracteriza por las dos propiedades siguientes:
\begin{enumerate}
\item $f(x_{1}+x_{2})=f(x_{1})+f(x_{2})$
\item $f(kx)=kf(x)$
\end{enumerate}
A la función lineal también se le llama función de proporcionalidad directa. $y = kx$ ó $f(x) = kx$ siempre es una recta que pasa por el origen.\\

Si $(x_{1},y_{1})$, $(x_{2},y_{2})$, \ldots, $(x_{n},y_{n})$, son pares de valores correspondientes de una función lineal o función de proporcionalidad directa entonces se tiene que:
\[\dfrac{y_{1}}{x_{1}}=\dfrac{y_{2}}{x_{2}}=\dfrac{y_{3}}{x_{3}}=\ldots=\dfrac{y_{n}}{x_{n}}=k\]
ó lo que es lo mismo $y_i = kx_i$.\\

Por último una función lineal puede considerarse como un operador multiplicativo que transforma una materia prima, como una magnitud o un número, en un producto determinado, que va a ser también una magnitud o un número.\\

\emph{Las matemáticas son el alfabeto con el cual Dios ha escrito el Universo.}\footnote{Galileo Galilei (1564-1642) Físico y astrónomo italiano.}
\end{document}
