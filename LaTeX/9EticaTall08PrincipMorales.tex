\documentclass[10pt,twoside]{article}
\usepackage[utf8]{inputenc}
\usepackage{amsmath}
\usepackage{amsfonts}
\usepackage{amssymb}
\usepackage[spanish,es-noshorthands]{babel}
\usepackage[T1]{fontenc}
\usepackage{lmodern}
\usepackage{graphicx,hyperref}
\usepackage{tikz,pgf}
\usepackage{multicol}
\usepackage{subfig}
\usepackage[papersize={6.5in,8.5in},width=5.5in,height=7in]{geometry}
\usepackage{fancyhdr}
\pagestyle{fancy}
\fancyhead[LE]{\includegraphics[height=12pt]{Images/logo-colegio.png} %Asig $^{\circ}$}
\fancyhead[RE]{}
\fancyhead[RO]{\textit{Germ\'an Avenda\~no Ram\'irez, Lic. U.D., M.Sc. U.N.}}
\fancyhead[LO]{}

\author{Germ\'an Avenda\~no Ram\'irez, Lic. U.D., M.Sc. U.N.}
\title{\begin{minipage}{.2\textwidth}
\includegraphics[height=1.75cm]{Images/logo-colegio.png}\end{minipage}
\begin{minipage}{.55\textwidth}
\begin{center}
Taller %,  \\
%Asig $^{\circ}$
\end{center}
\end{minipage}\hfill
\begin{minipage}{.2\textwidth}
\includegraphics[height=1.75cm]{Images/logo-sed.png} 
\end{minipage}}
\date{}
\begin{document}
\maketitle
Nombre: \hrulefill Curso: \underline{\hspace*{44pt}} Fecha: \underline{\hspace*{2.5cm}}
\subsection*{Aprendo algo nuevo}
\subsubsection*{Principios morales}
Son normas o reglas de conducta de carácter general o \emph{universal}, que orientan la acción del ser humano como ser social. Son principios en la medida en que se constituyen como base, \emph{fundamento} o cimiento de la actividad humana, para que esta garantice la supervivencia y el buen desarrollo de las comunidades y de los individuos.

En este sentido, los principios morales son el producto de la definición o la determinación de acciones y cosas que los seres humanos han señalado como inapropiados o inaceptables en determinadas circunstancias o en determinada época de la historia, creando leyes, máximas o preceptos para contrarrestarlos. De esta forma se determina como bueno o malo, correcto o incorrecto, apropiado o inapropiado, obligatorio o permitido, cualquier acción o decisión que tenga relación con la supervivencia y el bienestar de la especie, organizada en comunidades, y se crean mecanismos y medidas para evitar los efectos adversos.

En ocasiones dichas determinaciones y sus correspondientes leyes y preceptos, favorecieron a algunos individuos o comunidades, en detrimento de otras personas y grupos.

Por ejemplo, cuando la esclavitud fue aceptada moralmente por algunas sociedades o las mujeres fueron consideradas inferiores a los hombres. De forma afortunada y progresiva la humanidad se ha acercado a la comprensión y aceptación de la igualdad de todos los seres humanos, sin consideraciones de raza, sexo, credo o cualquier otra condición. En consecuencia, los principios morales se van haciendo realmente universales, sobrepasando los límites de las comunidades y de las naciones hasta llegar a relacionarse con el respeto a la vida, el amor al prójimo, la atención a los s, el cuidado del medio ambiente, la solidaridad, la integridad y la responsabilidad, entre muchos otros comportamientos de respeto y actitud práctica y razonable, con el mismo objetivo de preservar la especie humana y garantizar su desarrollo armónico.
\end{document}
