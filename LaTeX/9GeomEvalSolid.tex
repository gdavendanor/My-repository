\documentclass[10pt,letterpaper,addpoints]{exam}
\usepackage[utf8]{inputenc}
\usepackage[spanish,es-noshorthands]{babel}
\usepackage{hyperref}
\usepackage{amsmath}
\usepackage{amsfonts}
\usepackage{amssymb}
\usepackage{graphicx}
\usepackage{tikz,pgf}
\usepackage{multicol}
\usepackage[width=7in,height=9.5in]{geometry}
%\printanswers
\begin{document}
\title{\begin{minipage}{.2\textwidth}
        \includegraphics[height=1.75cm]{Images/logo-colegio.png}
       \end{minipage}
\begin{minipage}{.55\textwidth}
 \begin{center}
Prueba sólidos\\Geometría $9^{\circ}$
\end{center}
\end{minipage}
\begin{minipage}{.2\textwidth}
\includegraphics[height=1.75cm]{Images/logo-sed.png} 
\end{minipage}
}
\author{Germ\'{a}n Avendaño Ram\'{i}rez~\thanks{Lic. Matemáticas U.D., M.Sc. U.N.}}
\date{}
\maketitle
\begin{center}
\fbox{\fbox{\parbox{5.5in}{\centering
Conteste cada pregunta en el cuadro de respuestas. Debe hacer los procedimientos en su cuaderno}}}
\end{center}
\vspace{0.1in}
\makebox[\textwidth]{Nombres: \hrulefill, curso:\underline{\hspace{48pt}}, fecha:\underline{\hspace{3cm}}}
\begin{multicols}{2}
 \begin{questions}
 \question Observe la siguiente pirámide.
 \begin{center}
 \begin{tikzpicture}
 \draw (0,0,0)--(3,0,0)--(3.6,1,0)--(.6,1,0) --cycle;
 \draw (0,0,0)--(2.6,2.6,1.5)--(3,0,0);
 \draw (.6,1,0)--(2.6,2.6,1.5)--(3.6,1,0);	
 \end{tikzpicture}
 \end{center}
 ¿Con cuáles de los siguientes desarrollos planos se puede formar la pirámide?
 \begin{center}
 \includegraphics[scale=.45]{Images/Pantallazo.png} 
 \end{center}
 \begin{choices}
 \choice Con I y III solamente
 \CorrectChoice Con I, II y IV solamente
 \choice Con II y con IV solamente
 \choice Con II, con III y con IV solamente
 \end{choices}
 \question Observe la casa 
 \begin{center}
 \includegraphics[scale=.15]{Images/frente-casa.png} 
 \end{center}
 ¿Cuál es la vista de frente de esta casa?
 \begin{center}
 \includegraphics[scale=.45]{Images/Pantallazo-1.png} 
 \end{center}
 \question Un carpintero construye un mueble que tiene cajones como el que aparece en la siguiente figura:
 \begin{center}
 \includegraphics[scale=.6]{Images/Pantallazo-2.png} 
 \end{center}
 ¿Cuál es la capacidad en cm$^{3}$ de uno de los cajones del mueble?
 \begin{choices}
 \choice 60 cm$^{3}$
 \choice 500 cm$^{3}$
 \choice 4000 cm$^{3}$
 \CorrectChoice 6000 cm$^{3}$
 \end{choices}
 \question Con el molde que se presenta a continuación se va a construir un dado. A cada uno de los cuadrados en el molde, se le asignó uno de los números del 1 al 6 como se ilustra.
 \begin{center}
 \includegraphics[scale=.6]{Images/Pantallazo-3.png} 
 \end{center}
 ¿En cuál de las siguientes figuras se muestra la ubicación correcta de los números en las caras del dado?
 \begin{center}
 \includegraphics[scale=.45]{Images/Pantallazo-4.png} 
 \end{center}
 \question En un supermercado se empacan botellas de aceite del mismo tamaño en cajas rectangulares con capacidad para 6 botellas, como se muestra en la siguiente figura.
 \begin{center}
 \includegraphics[scale=.6]{Images/Pantallazo-5.png} 
 \end{center}
 Una caja rectangular del mismo ancho que el de la figura, en la que se puedan empacar 8 de estas botellas, debe tener
 \begin{choices}
 \choice 33 cm de largo
 \choice 35 cm de largo
 \CorrectChoice 40 cm de largo
 \choice 60 cm de largo
 \end{choices}
 \question Las siguientes figuras representan dos tipos de recipientes, I y II, utilizados para empacar alimentos.
 \begin{center}
 \includegraphics[scale=.6]{Images/Pantallazo-6.png} 
 \end{center}
 ¿Cuál de las siguientes afirmaciones, respecto al espacio ocupado por los recipientes tipo I y tipo II, es correcta?
 \begin{choices}
 \CorrectChoice El recipiente tipo I ocupa el doble del espacio utilizado por el recipiente tipo II.
 \choice El recipiente tipo II ocupa el doble del espacio utilizado por el recipiente tipo I.
 \choice Cuatro recipientes tipo I ocupan el mismo espacio que tres recipientes tipo II.
 \choice Cuatro recipientes tipo II ocupan el mismo espacio que tres recipientes tipo I.
 \end{choices}
 \question ¿Cuál de las figuras que se muestran a continuación, representa un sólido que tiene exactamente 6 caras?
 \begin{center}
 \includegraphics[scale=.45]{Images/Pantallazo-8.png} 
 \end{center}
 \question La siguiente figura representa un prisma triangular.
 \begin{center}
 \includegraphics[scale=.75]{Images/Pantallazo-9.png} 
 \end{center}
 ¿Cuál(es) de los siguientes desarrollos planos permite(n) armar un prisma triangular?
 \begin{center}
 \includegraphics[scale=.65]{Images/Pantallazo-10.png} 
 \end{center}
 \begin{choices}
 \choice II solamente
 \choice III solamente
 \CorrectChoice I y II solamente
 \choice I y III solamente
 \end{choices}
 \end{questions}
\end{multicols}
\end{document}
