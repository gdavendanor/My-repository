\documentclass[10pt,twoside]{article}
\usepackage[utf8]{inputenc}
\usepackage{amsmath,amsfonts,amssymb}
\usepackage[spanish,es-noshorthands]{babel}
\usepackage[T1]{fontenc}
\usepackage{lmodern}
\usepackage{graphicx,hyperref}
\usepackage{tikz,pgf}
\usepackage{marvosym}
\usepackage{multicol}
\usepackage{subfig}
\usepackage[papersize={6.5in,8.5in},left=1cm, right=1cm, top=1.5cm, bottom=1.7cm]{geometry}
\usepackage{fancyhdr}
\pagestyle{fancy}
\fancyhead[LE]{Colegio Arborizadora Baja}
\fancyhead[RE]{PEI:``Hacia una cultura para el desarrollo sostenible''}
\fancyfoot[RO]{\Email gavendanor@colarborizadorabaja.edu.co}
\fancyhead[LO]{\url{www.autistici.org/mathgerman}}
\fancyfoot[RE]{\Email cedarborizadoraba19@redp.edu.co}
\fancyfoot[LE]{Calle 59I \#44A - 02 \Telefon 7313994 - 7313995}
\fancyhead[RO]{Nit 830024976-8, Código DANE 11100103084-8}

\author{Germ\'an Avenda\~no Ram\'irez~\thanks{Lic. Mat. U.D., M.Sc. U.N.}}
\title{\begin{minipage}{.2\textwidth}
\includegraphics[height=1.75cm]{Images/logo-colegio.png}\end{minipage}
\begin{minipage}{.55\textwidth}
\begin{center}
Fracciones, decimales y razones y proporciones \\
Matemáticas $7^{\circ}$
\end{center}
\end{minipage}\hfill
\begin{minipage}{.2\textwidth}
\includegraphics[height=1.75cm]{Images/logo-sed.png} 
\end{minipage}}
\date{}
\begin{document}
\maketitle
Nombre: \hrulefill Curso: \underline{\hspace*{44pt}} Fecha: \underline{\hspace*{2.5cm}}
\begin{enumerate}
\item Convierta a fracción
\begin{enumerate}
\begin{multicols}{4}
\item $2\frac{2}{9}$
\item $3\frac{3}{5}$
\item $0.856$
\item $3.051$
\end{multicols}
\item Encuentre el numero decimal correspondiente
\begin{enumerate}
\begin{multicols}{4}
\item $\dfrac{7}{5}$
\item $\dfrac{2}{5}$
\item $\dfrac{7}{15}$
\item $\dfrac{6}{11}$
\end{multicols}
\end{enumerate}
\end{enumerate}
\item Determine si el número 43 es primo o compuesto
\item Determine si el número 2'053,752 es divisible por 4
\item Calcule
\begin{enumerate}
\begin{multicols}{2}
\item $48+12\div 4-10\times 2+6892\div 4$
\item $4.7-\{0.1[1.2(3.95-1.65)+1.5\div2.5]\}$
\end{multicols}
\end{enumerate}
\item Calcule y simplifique. (\emph{Recuerde que para sumar o restar decimales, siempre se debe poner la coma debajo de la coma})
\begin{enumerate}
\begin{multicols}{3}
\item $27.68+3.019+483.297$
\item $2\frac{1}{3}+4\frac{5}{12}$
\item $\dfrac{6}{35}+\dfrac{5}{28}$
\item $40.2-9.709$
\item $73.82-0.908$
\item $\dfrac{4}{15}-\dfrac{3}{20}$
\item $37.64\times 5.9$
\item $5.678\times100$
\item $2\frac{1}{3}\cdot 1\frac{2}{7}$
\item $2.3 \div 98.9$
\item $54\div 48.546$
\item $\dfrac{7}{11}\div \dfrac{14}{33}$
\end{multicols}
\end{enumerate}
\item Escriba el número decimal 30.074 como suma de fracciones decimales.
\item Escriba en palabras el número 120.07
\newpage
\item ¿Cual número es mayor?
\begin{enumerate}
\begin{multicols}{2}
\item 0.7; \quad 0.698
\item 0.799; \quad 0.8
\end{multicols}
\end{enumerate}
\item Descomponga en factores primos el número 144
\item Encuentre el m.c.m de 18 y 30
\item ¿Qué fracción representa la parte sombreada? \begin{tikzpicture}
 \draw[fill=gray](0,0)  rectangle  (.5,.5) rectangle (0,1) (.5,0) rectangle (1,.5) rectangle (.5,1)(1,.5) rectangle (1.5,1);
 \draw (1,0) --(2,0)--(2,1)--(1.5,1)--(1.5,0);
\draw (1.5,.5)--(2,.5);
\end{tikzpicture}
\item Simplique $\dfrac{90}{144}$
\item Calcule
\begin{enumerate}
\begin{multicols}{2}
\item $\dfrac{3}{5}\times 9.53$
\item $\dfrac{1}{3}\times 0.645-\dfrac{3}{4}\times 0.048$
\end{multicols}
\end{enumerate}
\item Escriba en notación de fracción, la razón de 0.3 a 15
\item Determine si las parejas 3, 9 y 25, 75 forman una proporción
\item ¿Cuál es la razón en metros por segundo? 660 metros, 12 segundos
\item Una porción de piña de 8 onzas puede costar \$0.99 (dólares), mientras que una porción de piña de 24.5 onzas puedes costar \$3.29. ¿En qué caso es más económica?
\item Resuelva 
\begin{enumerate}
\begin{multicols}{3}
\item $\dfrac{14}{25}=\dfrac{x}{54}$
\item $423=16\cdot t$
\item $\dfrac{2}{3}\cdot y=\dfrac{16}{27}$
\item $\dfrac{7}{16}=\dfrac{56}{x}$
\item $34.56+n=67.9$
\item $t+\dfrac{7}{25}=\dfrac{5}{7}$
\end{multicols}
\end{enumerate}
\item Una taza de pasta contiene 520 calorías. ¿Cuántas calorías habrá en $\frac{3}{4}$ de taza?
\item Una máquina puede estampar 925 camisetas en 5 minutos. Si se desean estampar 1295 camisetas, ¿cuánto tiempo gastará la máquina?
\item 46 onzas de jugo se pueden empacar en $5\frac{3}{4}$ tazas. En un recipiente se pueden echar $3\frac{1}{2}$ tazas. ¿Cuántas tazas quedan si envasar?
\item Un carpintero puede poner una puerta en $\frac{2}{3}$ de hora. ¿Cuántas puertas puede poner el carpintero en 8 horas?
\item Un carro puede recorrer 543.35 metros en 8 horas. ¿Cuánto recorre el auto en una hora?
\item Un satélite puede hacer 16 órbitas durante un día. En una misión de 8.25 días, ¿cuántas órbitas puede hacer?
\end{enumerate}

\end{document}
