\documentclass[10pt,twoside]{article}
\usepackage[utf8]{inputenc}
\usepackage{amsmath}
\usepackage{amsfonts}
\usepackage{amssymb}
\usepackage[spanish,es-noshorthands]{babel}
\usepackage[T1]{fontenc}
\usepackage{lmodern}
\usepackage{graphicx,hyperref}
\usepackage{tikz,pgf}
\usepackage{multicol}
\usepackage{subfig}
\usepackage[papersize={6.5in,8.5in},width=5.5in,height=7in]{geometry}
\usepackage{fancyhdr}
\pagestyle{fancy}
\fancyhead[LE]{\includegraphics[height=12pt]{Images/logo-colegio.png} Aritmética $6^{\circ}$}
\fancyhead[RE]{}
\fancyhead[RO]{\textit{Germ\'an Avenda\~no Ram\'irez, Lic. U.D., M.Sc. U.N.}}
\fancyhead[LO]{}

\author{Germ\'an Avenda\~no Ram\'irez, Lic. U.D., M.Sc. U.N.}
\title{\begin{minipage}{.2\textwidth}
\includegraphics[height=1.75cm]{Images/logo-colegio.png}\end{minipage}
\begin{minipage}{.55\textwidth}
\begin{center}
Taller 15, Resta y simplificación de fracciones homogéneas \\
Aritmética $6^{\circ}$
\end{center}
\end{minipage}\hfill
\begin{minipage}{.2\textwidth}
\includegraphics[height=1.75cm]{Images/logo-sed.png} 
\end{minipage}}
\date{}
\begin{document}
\maketitle
Nombre: \hrulefill Curso: \underline{\hspace*{44pt}} Fecha: \underline{\hspace*{2.5cm}}
\section*{Taller}
\subsection*{Sustracción de Fracciones homog\'{e}neas}
\subsubsection*{Ejercicios}
Realice las siguientes operaciones:
\begin{enumerate}
\begin{multicols}{3}
\item $\dfrac{6}{12}-\dfrac{5}{12}$
\item $\dfrac{25}{28}-\dfrac{6}{28}$
\item $\dfrac{63}{87}-\dfrac{34}{87}$
\item $\dfrac{17}{21}-\dfrac{6}{21}$
\item $\dfrac{29}{136}-\dfrac{4}{136}$
\item $\dfrac{7}{112}-\dfrac{5}{112}$
\end{multicols}
\end{enumerate}
\subsubsection*{Problemas}
\paragraph*{Problema 1:} Felipe comi\'{o} parte de una pizza dejando $\frac{6}{8}$ de ella. Si m\'{a}s tarde comi\'{o} $\frac{2}{8}$ de la pizza, ¿cu\'{a}nto qued\'{o} de \'{e}sta?
\paragraph*{Problema 2:}
Lucia compró una botella de aceite de $\frac{2}{3}$ de litro. Si usó $\frac{1}{3}$ de litro, ¿cuánto aceite quedó?
\paragraph*{Problema 3:} Una familia en el sur compr\'{o} $\frac{3}{4}$ de tonelada de leña. Si durante el primer mes gastaron $\frac{2}{4}$ de tonelada, ¿cuánto les queda?
\paragraph*{Problema 4:} Contrataron a una empresa para pavimentar un camino. Si después de 3 días de trabajo les falta por pavimentar $\frac{4}{7}$ del camino. Si luego pavimentan $\frac{3}{7}$ del camino. ¿Cuánto les falta por pavimentar?
\paragraph*{Problema 5:} José compró $\frac{5}{8}$ de kilo de manteca para preparar pan amasado. Si sólo ocupó $\frac{2}{8}$ de kilo, ¿qué fracción de kilo de manteca le sobró?
\end{document}
