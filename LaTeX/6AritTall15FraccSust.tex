\documentclass[10pt,twoside]{article}
\usepackage[utf8]{inputenc}
\usepackage{amsmath}
\usepackage{amsfonts}
\usepackage{amssymb}
\usepackage[spanish,es-noshorthands]{babel}
\usepackage[T1]{fontenc}
\usepackage{lmodern}
\usepackage{graphicx,hyperref}
\usepackage{tikz,pgf}
\usepackage{multicol}
\usepackage{subfig}
\usepackage[papersize={6.5in,8.5in},width=5.5in,height=7in]{geometry}
\usepackage{fancyhdr}
\pagestyle{fancy}
\fancyhead[LE]{\includegraphics[height=12pt]{Images/logo-colegio.png} Aritmética $6^{\circ}$}
\fancyhead[RE]{}
\fancyhead[RO]{\textit{Germ\'an Avenda\~no Ram\'irez, Lic. U.D., M.Sc. U.N.}}
\fancyhead[LO]{}

\author{Germ\'an Avenda\~no Ram\'irez, Lic. U.D., M.Sc. U.N.}
\title{\begin{minipage}{.2\textwidth}
\includegraphics[height=1.75cm]{Images/logo-colegio.png}\end{minipage}
\begin{minipage}{.55\textwidth}
\begin{center}
Taller 15, Operaciones con fracciones\\
Aritmética $6^{\circ}$
\end{center}
\end{minipage}\hfill
\begin{minipage}{.2\textwidth}
\includegraphics[height=1.75cm]{Images/logo-sed.png} 
\end{minipage}}
\date{}
\begin{document}
\maketitle
Nombre: \hrulefill Curso: \underline{\hspace*{44pt}} Fecha: \underline{\hspace*{2.5cm}}
\section*{Sustracción de Fracciones homog\'{e}neas}
\subsection*{Ejercicios}
Realice las siguientes operaciones:
\begin{enumerate}
\begin{multicols}{3}
\item[a.] $\dfrac{6}{12}-\dfrac{5}{12}$
\item[b.] $\dfrac{25}{28}-\dfrac{6}{28}$
\item[c.] $\dfrac{63}{87}-\dfrac{34}{87}$
\item[d.] $\dfrac{17}{21}-\dfrac{6}{21}$
\item[e.] $\dfrac{29}{136}-\dfrac{4}{136}$
\item[f.] $\dfrac{7}{112}-\dfrac{5}{112}$
\end{multicols}
\end{enumerate}
\subsection*{Problemas}
\paragraph*{Problema 1:} Felipe comi\'{o} parte de una pizza dejando $\frac{6}{8}$ de ella. Si m\'{a}s tarde comi\'{o} $\frac{2}{8}$ de la pizza, ¿cu\'{a}nto qued\'{o} de \'{e}sta?
\paragraph*{Problema 2:}
Lucia compró una botella de aceite de $\frac{2}{3}$ de litro. Si usó $\frac{1}{3}$ de litro, ¿cuánto aceite quedó?
\paragraph*{Problema 3:} Una familia en el sur compr\'{o} $\frac{3}{4}$ de tonelada de leña. Si durante el primer mes gastaron $\frac{2}{4}$ de tonelada, ¿cuánto les queda?
\paragraph*{Problema 4:} Contrataron a una empresa para pavimentar un camino. Si después de 3 días de trabajo les falta por pavimentar $\frac{4}{7}$ del camino. Si luego pavimentan $\frac{3}{7}$ del camino. ¿Cuánto les falta por pavimentar?
\paragraph*{Problema 5:} José compró $\frac{5}{8}$ de kilo de manteca para preparar pan amasado. Si sólo ocupó $\frac{2}{8}$ de kilo, ¿qué fracción de kilo de manteca le sobró?
\section*{Adici\'{o}n de fracciones heterog\'{e}neas}
\subsection*{Problema resuelto}
La señora Marta horneó 2 queques iguales. Su hijo Juan comió $\frac{1}{4}$ del primero y su hija Lucía $\frac{3}{8}$ del segundo ¿Cuánto comieron entre ambos? 
\subsubsection*{Solución:} Entre ambos comieron la suma de lo que comió cada uno.
\subsubsection*{Procedimiento:} Expresar ambas fracciones $\frac{1}{4}$ y $\frac{3}{8}$ con un denominador común y luego sumarlas.
\subsubsection*{Operaci\'{o}n y resultado:}
\[\dfrac{1}{4}=\dfrac{1\cdot 2}{4\cdot 2}=\dfrac{2}{8}\qquad \mbox {luego} \qquad \dfrac{2}{8}+\dfrac{3}{8}=\dfrac{5}{8}\]
\subsubsection*{Respuesta:} Entre los dos comieron la porción equivalente a $\frac{5}{8}$ de un queque.
\subsection*{Ejercicios}
\begin{enumerate}
\begin{multicols}{3}
\item[g.] $\dfrac{2}{3}+\dfrac{3}{9}$
\item[h.] $\dfrac{4}{12}+\dfrac{1}{48}$
\item[i.] $\dfrac{34}{62}+\dfrac{2}{31}$
\item[j.] $\dfrac{2}{25}+\dfrac{3}{5}$
\item[k.] $\dfrac{13}{360}+\dfrac{7}{18}$
\item[l.] $\dfrac{2}{5}+\dfrac{18}{30}$
\end{multicols}
\end{enumerate}
\subsection*{Problemas}
\paragraph*{Problema 6:} Marta compró un corte de género (tela) para confeccionar un juego de sábanas. En la sábana de abajo ocupó $\frac{3}{10}$ del corte, en la de arriba $\frac{2}{5}$ del corte y en las fundas $\frac{1}{10}$. ¿Qué fracción del corte de género utilizó?
\paragraph*{Problema 7:} Luisa compró $\frac{1}{5}$ de Kg. de chocolate amargo y $\frac{7}{15}$ de Kg. de chocolate dulce ¿Cuánto compró en total?
\paragraph*{Problema 8:} Cuánto tiempo gastó José en subir y bajar un cerro si tardó $\frac{3}{4}$ de hora en subirlo y $\frac{1}{2}$ de hora en bajarlo?
\paragraph*{Problema 9:} En su testamento, una mujer le dejó a su esposo $\frac{6}{13}$ de sus bienes y a sus hijos $\frac{11}{26}$. ¿Le dejó algo a otras personas?
\paragraph*{Problema 10:} Dos amigos decidieron compartir una botella de jugo. El primero tomó $\frac{1}{4}$ de la botella, el segundo $\frac{5}{8}$ de ella. ¿Qué parte de la botella de jugo bebieron?
\section*{Resta de fracciones heterog\'{e}neas}
\subsection*{Problema resuelto}
Juan llevó al colegio $\frac{5}{8}$ de una resma de 
 papel carta. En el recreo, su hermana Lucía se dio cuenta que necesitaba papel para hacer un trabajo y pidió $\frac{1}{4}$ de resma. ¿Con cuánto papel se quedó Juan? 
\subsubsection*{Soluci\'{o}n}
Para determinar la cantidad de papel con que se quedó Juan, se debe restar a la cantidad que tenía originalmente, la cantidad que le sacó Lucía.\\
Recordemos que para poder restar las fracciones, éstas deben ser homogéneas.
\subsection*{Ejercicios}
\begin{enumerate}
\begin{multicols}{3}
\item[m.] $\dfrac{4}{5}-\dfrac{27}{40}$
\item[n.] $\dfrac{5}{6}-\dfrac{57}{72}$
\item[ñ.] $\dfrac{7}{9}-\dfrac{34}{81}$
\item[o.] $\dfrac{1}{2}-\dfrac{169}{350}$
\item[p.] $\dfrac{2}{7}-\dfrac{4}{140}$
\item[q.] $\dfrac{77}{112}-\dfrac{1}{4}$
\end{multicols}
\end{enumerate}
\subsection*{Problemas}
\paragraph*{Problema 11:} Un camión de basura ha recogido suficientes desechos para copar $\frac{5}{6}$ de su capacidad. Si al descargar los materiales reciclables, el camión queda con $\frac{11}{24}$ de su capacidad. ¿Qué fracción de la capacidad del camión estaba constituida por basura reciclable?
\paragraph*{Problema 12:} De una botella con $\frac{3}{4}$ de litro de aceite, Juna llena una alcuza de $\frac{1}{8}$ de litro de capacidad, ¿cuánto aceite quedó en la botella?
\paragraph*{Problema 13:} Después de haber pavimentado $\frac{1}{3}$ de una calle, se descubre una cañería de gas rota por lo cual deben romper el pavimento de $\frac{2}{9}$ de la calle. ¿Qué fracción de la calle queda pavimentada?
\paragraph*{Problema 14:} Guillermo tenía $\frac{3}{4}$ de un cajón de tomates para hacer salsa. Si antes de hacer la salsa regaló $\frac{1}{8}$ de cajón a su hermana. ¿Qué fracción de cajón le quedó?
\paragraph*{Problema 15:} Un estanque lleno con agua hasta la mitad de su capacidad pierde por una filtración una cantidad de agua igual a $\frac{1}{8}$ de su capacidad. ¿Cuánta agua queda en el estanque?
\section*{Suma de fracciones en general}
\subsection*{Ejercicios}
\begin{enumerate}
\begin{multicols}{3}
\item[r.] $\dfrac{4}{5}+\dfrac{3}{9}$
\item[s.] $\dfrac{4}{11}+\dfrac{1}{2}$
\item[t.] $\dfrac{4}{15}+\dfrac{2}{25}$
\item[u.] $\dfrac{12}{14}+\dfrac{2}{21}$
\item[v.] $\dfrac{4}{3}+\dfrac{8}{5}$
\item[w.] $\dfrac{6}{10}+\dfrac{1}{12}$
\end{multicols}
\end{enumerate}
\subsection*{Problemas:}
\paragraph*{Problema 16:} Javier y Francisco tenían que llevar arroz al colegio para una campaña de ayuda solidaria. Javier llevó $\frac{2}{3}$ de un paquete de kilo, Francisco llevó un kilo. ¿Cuántos tercios de kilo llevaron entre los dos?
\paragraph*{Problema 17:} Juan y Ramón trabajan en turnos consecutivos en una fábrica que funciona sin parar. Juan trabajó $\frac{2}{3}$ de día; y Ramón $\frac{2}{5}$ del día. ¿Qué parte del día cubrieron entre ambos?
\paragraph*{Problema 18:} Marta quería tejerse un chaleco, para ello compró una bolsa de ovillos de lana. Cuando terminó el chaleco sólo había ocupado $\frac{1}{2}$ de bolsa. Decidió entonces tejerse un gorro, en el que ocupó $\frac{1}{6}$ de la bolsa. Como aún le sobraba, se tejió también una bufanda en la que ocupó $\frac{1}{3}$ más de la bolsa. ¿Qué
fracción de la bolsa de lana usó?
\paragraph*{Problema 19:} Paulina decidió atender a sus amigos haciendo sándwiches con dos tipos de pasta para lo cual compró dos panes de molde. La pasta de jamón sólo le alcanzó para preparar $\frac{3}{8}$ de un pan de molde, en cambio la pasta de queso le alcanzó para $\frac{1}{6}$ del otro pan. ¿Cuánto pan de molde ocupó en total?
\paragraph*{Problema 20:} Juan decidió alimentar a sus mascotas, 2 grandes perros, con un tipo nuevo de comida, para lo cual compró una ración adecuada para 1 mes. Como los perros no estaban acostumbrados a ese tipo de alimento, durante la primera semana sólo
consumieron la décima parte de la ración comprada
para el mes, en la segunda semana un quinto de la
ración y tanto en la tercera como en la cuarta
consumieron un cuarto de la ración comprada para
el mes. ¿Alcanzó la ración comprada?
\end{document}
