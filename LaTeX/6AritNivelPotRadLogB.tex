\documentclass[fleqn]{article}
\usepackage[spanish,es-noshorthands]{babel}
\usepackage[utf8]{inputenc} 
\usepackage[papersize={5.5in,8.5in},left=1cm, right=1cm, top=1.5cm, bottom=1.7cm]{geometry}
\usepackage{mathexam}
\usepackage{tikz,pgf}
\usepackage{amsmath}
\usepackage{graphicx}
\usepackage{multicol}
\ExamClass{\includegraphics[height=16pt]{Images/logo-sed.png} Aritmética $6^{\circ}$}
\ExamName{``Nivelación III''}
\ExamHead{\includegraphics[height=16pt]{Images/logo-colegio.png} IEDAB}
\newcommand{\LineaNombre}{%
\par
\vspace{\baselineskip}
Nombre:\hrulefill \; Curso: \underline{\hspace*{48pt}} \; Fecha: \underline{\hspace*{2.5cm}} \relax
\par}
\let\ds\displaystyle

\begin{document}
\ExamInstrBox{
\textit{Respuesta sin justificar mediante procedimiento no será tenida en cuenta en la calificación. Escriba sus respuestas en el espacio indicado. Tiene 45 minutos para contestar esta prueba.}}
\LineaNombre
\begin{enumerate}
  \item Encuentre las 5 primeras potencias de los números
  \begin{enumerate}
    \item $ 1 $: \hrulefill
    \item $ 2 $: \hrulefill
    \item $ 3 $: \hrulefill
    \item $ 4 $: \hrulefill
  \end{enumerate}
  \item Resuelva cada una de las siguientes operaciones e identifique la base, el exponente y la potencia
  \begin{enumerate}\begin{multicols}{2}
    \item $ 6^3= $: \underline{\hspace{3.5cm}}
    \item $ 9^3= $: \hrulefill\end{multicols}
  \end{enumerate}
  \item Encuentre el número que falta en el paréntesis en cada una de las siguientes operaciones indicadas:
  \begin{enumerate}\begin{multicols}{2}
    \item $ 3^6= $[\underline{\hspace{1cm}}]
    \item $ [\hspace{0.5cm}]^3=64 $
    \item $ 5^{[\hspace{0.2cm}]}=625 $
    \item $ \sqrt[3]{729}=[\hspace{0.5cm}] $
  \end{multicols}
  \end{enumerate}
  \item Determine las siguientes raíces, descomponiendo cada número en sus factores primos y usando la siguiente propiedad:
  \[ \sqrt[n]{a\cdot b}=\sqrt[n]{a}\cdot\sqrt[n]{b} \]
  \textbf{Ejemplo:} $ \sqrt{144}=\sqrt{4^2\cdot3^2 }= \sqrt{4^2}\cdot\sqrt{3^2}=4\cdot3=12$\\
  Ya que al descomponer 144 en sus factores primos obtenemos\\
  \[ 144=2^4\cdot3^2=2^2\cdot2^2\cdot3^2=4^2\cdot3^2 \]
  \begin{enumerate}
    \item $ \sqrt{256}= $: \hrulefill
    \item $ \sqrt[3]{3375}= $: \hrulefill
  \end{enumerate}
    \item Escriba en forma de potencia cada una de las siguientes raíces:
    \begin{enumerate}\begin{multicols}{2}
      \item $ \sqrt[3]{216}=6 $; $\rightarrow$ \underline{\hspace{2cm}}
      \item $ \sqrt{10000}=100 $ $\rightarrow$ \underline{\hspace{2cm}}
\end{multicols}
   \end{enumerate}
  \item Escriba en forma de potencia cada uno de los siguientes logaritmos:
  \begin{enumerate}\begin{multicols}{2}
    \item $ \log_3{81}=4 $: \underline{\hspace{3cm}}
    \item $ \log_5{125}=3 $: \underline{\hspace{3cm}}
  \end{multicols}
  \end{enumerate}
\item Determine el área de una cancha de baloncesto que mide 28 metros de largo por 15 metros de ancho.\noanswer
\item Si se forma un rectángulo con 30 cuadrados como éste \tikz \draw (0,0) rectangle (.4,.4);, ¿cuántos cuadrados debería agregar para formar un cuadrado? Dibuje. ¿Cuántos cuadrados quedarían por cada lado? Represente por medio de una operación.\noanswer
  \item Si se forma un cuadrado con 36 cuadrados como éste \tikz \draw (0,0) rectangle (.4,.4); ¿cuántos cuadrados caben por cada lado? Dibújelo y represente su respuesta por medio de una operación.\noanswer
\end{enumerate}
\end{document}
