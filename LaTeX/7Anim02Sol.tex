\documentclass[10pt,twoside]{article}
\usepackage[utf8]{inputenc}
\usepackage{amsmath}
\usepackage{amsfonts}
\usepackage{amssymb}
\usepackage[spanish,es-noshorthands]{babel}
\usepackage[T1]{fontenc}
\usepackage{lmodern}
\usepackage{graphicx,hyperref}
\usepackage{tikz,pgf}
\usepackage{multicol}
\usepackage{subfig}
\usepackage[papersize={6.5in,8.5in},width=5.5in,height=7in]{geometry}
\usepackage{fancyhdr}
\pagestyle{fancy}
\fancyhead[LE]{\includegraphics[height=12pt]{Images/logo-colegio.png} Álgebra $8^{\circ}$}
\fancyhead[RE]{}
\fancyhead[RO]{\textit{Germ\'an Avenda\~no Ram\'irez, Lic. U.D., M.Sc. U.N.}}
\fancyhead[LO]{}

\author{Germ\'an Avenda\~no Ram\'irez, Lic. U.D., M.Sc. U.N.}
\title{\begin{minipage}{.2\textwidth}
\includegraphics[height=1.75cm]{Images/logo-colegio.png}\end{minipage}
\begin{minipage}{.55\textwidth}
\begin{center}
Taller, Animaplano 02 \\
Álgebra $8^{\circ}$
\end{center}
\end{minipage}\hfill
\begin{minipage}{.2\textwidth}
\includegraphics[height=1.75cm]{Images/logo-sed.png} 
\end{minipage}}
\date{}
\begin{document}
\maketitle
Nombre: \hrulefill Curso: \underline{\hspace*{44pt}} Fecha: \underline{\hspace*{2.5cm}}
\section*{Cuestionario}
Responde al frente de cada pregunta y luego, ubique las repuestas en el plano dispuesto para ello.
\begin{enumerate}
 \item Reste a $10^{2}$, la suma de los 4 primeros números primos \[=10^{2}-(2+3+5+7)=100-17=83\]
 \item En unidades, 1 centena, menos 2 docenas, menos 2 unidades \[=1(100)-2(12)-2=100-24-2=74\]
 \item ¿Cuánto le falta a 27, para que de 100? \[100-27=73\]
 \item Al producto entre 20 y 4, reste el producto entre 9 y 2 \[20(4)-9(2)=80-18=62\]
 \item Multiplique la tercera parte de 63, por el número 2 \[\frac{1}{3}(63)\cdot 2=21\cdot 2=42\]
 \item ¿Cuánto le sobra a 98, para que de 50? \[98-50=48\]
 \begin{multicols}{2}
 \item Multiplique por 2, el doble de 4.5 \[2[2(4,5)]=2[9]=18\]
 \item Los años en 4 lustros \[4(5)=20\]
 \end{multicols}
 \item La equivalencia en años de 5 décadas \[5(10)=50\]
 \begin{multicols}{2}
 \item Halle $7\times \sqrt{49}$ \[7\times\sqrt{49}=7\times7=49\]
 \item Tres veces veintitrés \[3(23)=69\]
 \item Halle $(3^{2})^{2}-3^{1}=$ \[(9)^{2}-3=81-3=78\]
 \item Si $23+m=100$, luego $m=$? \[m=100-23=77\]
 \end{multicols}
 \item ¿Cuánto le restamos a 110 para que de 22? \[110-22=88\]
 \item En unidades cuadradas, al área del cuadrado de lado 9 unidades, sume $2^{3}$ unidades cuadradas. \[9^{2}+8=81+8=89\]
 \item $\sqrt{10\,000}-10^{2}+(10\times10)=$ \[100-100+100=100\]
 \item Sume al triple del n\'umero 30, la ra\'iz cuadrada de 1 \[3(30)+\sqrt{1}=90+1=91\]
 \item Exprese como n\'umero decimal el n\'umero romano $LXXXII=82$
 \item Sume al triple del n\'umero 20, el triple del n\'umero 9 \[3(20)+3(9)=60+27=87\]
 \item En cent\'imetros, un metro, menos 3 dec\'imetros, m\'as 6 cent\'imetros \[100-30+6=76\]
 \item En unidades, 3 docenas + 3 decenas \[3(12)+3(10)=36+30=66\]
 \item Sume 7, al 50\% de 100 \[7+50=57\]
 \item A la cuarta parte de 200, sume dos elevado a la tres \[\frac{1}{4}(200)+2^{3}=50+8=58\]
 \item Si $m-47=20$, entonces $m=$? \[m=20+47=67\]
 \item La suma de 15, con el producto entre 9 y 8 \[15+9(8)=15+72=87\]
 \item Sume cuatro veces 22 \[4(22)=88\]
\end{enumerate}
\section*{Plano}
\begin{center}
\begin{tikzpicture}
 \fill (1,0) node[above]{1} circle (0.2ex);
 \fill (2,0) node[above]{2} circle (0.2ex);
 \fill (3,0) node[above]{3} circle (0.2ex);
 \fill (4,0) node[above]{4} circle (0.2ex);
 \fill (5,0) node[above]{5} circle (0.2ex);
 \fill (6,0) node[above]{6} circle (0.2ex);
 \fill (7,0) node[above]{7} circle (0.2ex);
 \fill (8,0) node[above]{8} circle (0.2ex);
 \fill (9,0) node[above]{9} circle (0.2ex);
 \fill (10,0) node[above]{10} circle (0.2ex);
 \fill (1,-1) node[left]{11} circle (0.2ex);
 \fill (2,-1) circle (0.2ex);
 \fill (3,-1) circle (0.2ex);
 \fill (4,-1) circle (0.2ex);
 \fill (5,-1) circle (0.2ex);
 \fill (6,-1) circle (0.2ex);
 \fill (7,-1) circle (0.2ex);
 \fill (8,-1) circle (0.2ex);
 \fill (9,-1) circle (0.2ex);
 \fill (10,-1) circle (0.2ex);
 \fill (1,-2) node[left]{21} circle (0.2ex);
 \fill (2,-2) circle (0.2ex);
 \fill (3,-2) circle (0.2ex);
 \fill (4,-2) circle (0.2ex);
 \fill (5,-2) circle (0.2ex);
 \fill (6,-2) circle (0.2ex);
 \fill (7,-2) circle (0.2ex);
 \fill (8,-2) circle (0.2ex);
 \fill (9,-2) circle (0.2ex);
 \fill (10,-2) circle (0.2ex);
 \fill (1,-3) node[left]{31} circle (0.2ex);
 \fill (2,-3) circle (0.2ex);
 \fill (3,-3) circle (0.2ex);
 \fill (4,-3) circle (0.2ex);
 \fill (5,-3) circle (0.2ex);
 \fill (6,-3) circle (0.2ex);
 \fill (7,-3) circle (0.2ex);
 \fill (8,-3) circle (0.2ex);
 \fill (9,-3) circle (0.2ex);
 \fill (10,-3) circle (0.2ex);
 \fill (1,-4) node[left]{41} circle (0.2ex);
 \fill (2,-4) circle (0.2ex);
 \fill (3,-4) circle (0.2ex);
 \fill (4,-4) circle (0.2ex);
 \fill (5,-4) circle (0.2ex);
 \fill (6,-4) circle (0.2ex);
 \fill (7,-4) circle (0.2ex);
 \fill (8,-4) circle (0.2ex);
 \fill (9,-4) circle (0.2ex);
 \fill (10,-4) node[right]{50} circle (0.2ex);
 \fill (1,-5) node[left]{51} circle (0.2ex);
 \fill (2,-5) circle (0.2ex);
 \fill (3,-5) circle (0.2ex);
 \fill (4,-5) circle (0.2ex);
 \fill (5,-5) circle (0.2ex);
 \fill (6,-5) circle (0.2ex);
 \fill (7,-5) circle (0.2ex);
 \fill (8,-5) circle (0.2ex);
 \fill (9,-5) circle (0.2ex);
 \fill (10,-5) circle (0.2ex);
 \fill (1,-6) node[left]{61} circle (0.2ex);
 \fill (2,-6) circle (0.2ex);
 \fill (3,-6) circle (0.2ex);
 \fill (4,-6) circle (0.2ex);
 \fill (5,-6) circle (0.2ex);
 \fill (6,-6) circle (0.2ex);
 \fill (7,-6) circle (0.2ex);
 \fill (8,-6) circle (0.2ex);
 \fill (9,-6) circle (0.2ex);
 \fill (10,-6) circle (0.2ex);
 \fill (1,-7) node[left]{71} circle (0.2ex);
 \fill (2,-7) circle (0.2ex);
 \fill (3,-7) circle (0.2ex);
 \fill (4,-7) circle (0.2ex);
 \fill (5,-7) circle (0.2ex);
 \fill (6,-7) circle (0.2ex);
 \fill (7,-7) circle (0.2ex);
 \fill (8,-7) circle (0.2ex);
 \fill (9,-7) circle (0.2ex);
 \fill (10,-7) circle (0.2ex);
 \fill (1,-8) node[left]{81} circle (0.2ex);
 \fill (2,-8) circle (0.2ex);
 \fill (3,-8) circle (0.2ex);
 \fill (4,-8) circle (0.2ex);
 \fill (5,-8) circle (0.2ex);
 \fill (6,-8) circle (0.2ex);
 \fill (7,-8) circle (0.2ex);
 \fill (8,-8) circle (0.2ex);
 \fill (9,-8) circle (0.2ex);
 \fill (10,-8) circle (0.2ex);
 \fill (1,-9) node[left]{91} circle (0.2ex);
 \fill (2,-9) circle (0.2ex);
 \fill (3,-9) circle (0.2ex);
 \fill (4,-9) circle (0.2ex);
 \fill (5,-9) circle (0.2ex);
 \fill (6,-9) circle (0.2ex);
 \fill (7,-9) circle (0.2ex);
 \fill (8,-9) circle (0.2ex);
 \fill (9,-9) circle (0.2ex);
 \fill (10,-9) node[right]{100} circle (0.2ex);
 \draw (3,-8)--(4,-7)--(3,-7)--(2,-6)--(2,-4)--(8,-4)--(8,-1)--(10,-1)--(10,-4)--(9,-4)--(9,-6)--(8,-7)--(7,-7)--(8,-8)--(9,-8)--(10,-9)--(1,-9)--(2,-8)--(7,-8)--(6,-7)--(6,-6)--(7,-5)--(8,-5)--(7,-6)--(7,-8)--(8,-8);
\end{tikzpicture}
\end{center}
\end{document}
