\documentclass[letterpaper,fleqn]{article}
\usepackage[spanish,es-noshorthands]{babel}
\usepackage[utf8]{inputenc} 
\usepackage[left=1cm, right=1cm, top=1.5cm, bottom=1.7cm]{geometry}
\usepackage{mathexam}
\usepackage{amsmath}
\usepackage{amsfonts}
\usepackage{amssymb}
\usepackage{graphicx}
\usepackage{multicol}
\usepackage{tikz,pgf}
\ExamClass{\includegraphics[height=16pt]{Images/logo-sed.png} Matemáticas  11$^{\circ}$}
\ExamName{Nivelación 2016}
\ExamHead{\includegraphics[height=16pt]{Images/logo-colegio.png} IEDAB}
\newcommand{\LineaNombre}{%
\par
\vspace{\baselineskip}
Nombre:\hrulefill \; Curso: \underline{\hspace*{48pt}} \; Fecha: \underline{\hspace*{2.5cm}} \relax
\par}
\let\ds\displaystyle

\begin{document}
\ExamInstrBox{
Respuesta sin justificar mediante procedimiento no será tenida en cuenta en la calificación. Escriba sus respuestas en el espacio indicado. Tiene 50 minutos para contestar esta prueba.}
\LineaNombre
Los números Reales $\mathbb{R}$ se componen de ciertos subconjuntos, tal como se muestra a continuación.
\begin{center}
\begin{tikzpicture}[level 1/.style={sibling distance=30mm},
level 2/.style={sibling distance=15mm}
level 3/.style={sibling distance=10mm}
level 4/.style={sibling distance=1mm}]
\node {Reales $\mathbb{R}$}
child {node {Racionales $\mathbb{Q}$}
child {node {Enteros $\mathbb{Z}$}
child {node {Naturales $\mathbb{N}$}}
child {node {Cero $0$} }
child {node {Negativos}}
}
}
child {node {Irracionales $\mathbb{I}$}};
\end{tikzpicture}
\end{center}
\begin{enumerate}
\item Teniendo en cuenta el diagrama anterior, diga a que conjunto o conjuntos pertenecen los siguientes números: (\emph{Recuerde que $i$ es la unidad imaginaria y surge de $\sqrt{-1}=i$ \quad o \quad $i^{2}=-1$})
\begin{enumerate}
\begin{multicols}{2}
\item $-\sqrt{4}$
\item $3i$
\item $7,\overline{25}$
\item $\sqrt{7}$
\item $\pi$
\item $3-\frac{1}{2}$
\item $-e$
\item $16$
\item $\sqrt{-8}+3$
\end{multicols}
\end{enumerate}
 \item El duodécimo término de una progresión aritmética es 32, y el quinto término es 18. Encuentre el vigésimo término.\noanswer
\begin{minipage}{.45\textwidth}
\item Los postes de teléfono son puestos en pila, con 25 postes en el primer nivel, 24 en el segundo y así sucesivamente. Si hay 12 niveles, ¿cuántos postes de teléfono contiene la pila de postes?
\end{minipage}\hfill
\begin{minipage}{.45\textwidth}
\includegraphics[scale=.9]{Images/postes.png} 
\end{minipage}
\noanswer
\item Un cultivo de bacterias tiene inicialmente 5000 bacterias y su número aumenta 8\% cada hora. 
\begin{enumerate}
\item ¿Cuántas bacterias hay al cabo de 5 horas?\noanswer
\item Encuentre una expresi\'{o}n que indique el n\'{u}mero de bacterias que hay al cabo de $n$ horas.\noanswer
\end{enumerate}
\newpage
 \item Sea 
$
f(x)= \left\{ \begin{array}{lcl}
4 & \mbox{ si } & x\leq 2 \\
x-2 & \mbox{ si } & x>2
\end{array}
\right.
$
\begin{enumerate}
\item Evalúe 
\begin{enumerate}
\begin{multicols}{3}
\item $f(0)=$ 
\item $f(1)=$
\item $f(2)=$ 
\item $f(3)=$ 
\item $f(4)=$ 
\item $f(5)=$
\end{multicols}
\end{enumerate}
\item Haga la gráfica de $f$
\begin{center}
\begin{tikzpicture}[scale=.9]
\draw[dotted,style=help lines] (-2,-2) grid (5,5);
\draw[<->] (-2.2,0)--(5.3,0) node[right] {$x$};
\foreach \x in {1} \draw[shift={(\x,0)},color=black] (0pt,2pt) -- (0pt,-2pt) node[below] {\footnotesize $\x$};
\draw[<->](0,-2.2)--(0,5.2)node[left]{$y$};
\foreach \y in {1} \draw[shift={(0,\y)},color=black] (2pt,0)--(-2pt,0) node[left]{\footnotesize $\y$};
\end{tikzpicture}
\end{center}
\item Encuentre los siguientes límites
\begin{enumerate}
\begin{multicols}{2}
\item $\displaystyle{\lim_{x\rightarrow 2^{-}}f(x)}=$ 
\item $\displaystyle{\lim_{x\rightarrow 2^{+}}f(x)}=$ 
\item $\displaystyle{\lim_{x\rightarrow 2}f(x)}=$ 
\item $\displaystyle{\lim_{x\rightarrow 3}f(x)}=$ 
\end{multicols}
\end{enumerate}
\end{enumerate}
\item Evalúe los límites, si existen, usando el procedimiento que crea conveniente.
\begin{enumerate}
\item $\ds{\lim_{x\rightarrow 3}\dfrac{x^{2}+4x-21}{x-3}}=$\noanswer
\item $\ds{\lim_{x\rightarrow -3}\dfrac{x^{2}+4x-21}{x-3}}=$\noanswer
\item $\ds{\lim_{x\rightarrow 2}\dfrac{x^{2}+4}{x-2}}=$\noanswer
\end{enumerate} 
\item Mil personas seleccionadas de cierta enfermedad reciben un examen clínico. Como consecuencia del examen, la muestra de 1000 personas se clasifica de acuerdo con su estatura y situaci\'{o}n de su enfermedad.
\begin{center}
\begin{tabular}{lccccc}
 & \multicolumn{5}{c}{Situación de enfermedad} \\ \hline 
Estatura & Ninguno & Benigno & Moderado & Grave & Total \\ \hline
Alta & 122 & 78 & 139 & 61 & 400 \\ 
Media & 74 & 51 & 90 & 35 & 250 \\ 
Corta & 104 & 71 & 121 & 54 & 350 \\ 
\hline 
Total & 300 & 200 & 350 & 150 & 1000 \\ 
\hline 
\end{tabular} 
\end{center}
Use la información de la tabla para estimar la probabilidad de ser de estatura media o corta y tener situación de enfermedad moderada o grave.\noanswer
 \end{enumerate}

\end{document}
