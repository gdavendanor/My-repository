\documentclass[fleqn]{article}
\usepackage[spanish,es-noshorthands]{babel}
\usepackage[utf8]{inputenc} 
\usepackage[papersize={6.5in,8.5in},left=1cm, right=1cm, top=1.5cm, bottom=1.7cm]{geometry}
\usepackage{mathexam}
\usepackage{amsmath}
\usepackage{graphicx}
\usepackage{tikz,pgf}
%\usepackage{pst-solides3d}
\ExamClass{\includegraphics[height=16pt]{Images/logo-sed.png} Matemáticas $8^{\circ}$}
\ExamName{Áreas y volúmenes}
\ExamHead{\includegraphics[height=16pt]{Images/logo-colegio.png} IEDAB}
\newcommand{\LineaNombre}{%
\par
\vspace{\baselineskip}
Nombre:\hrulefill \; Curso: \underline{\hspace*{48pt}} \; Fecha: \underline{\hspace*{2.5cm}} \relax
\par}
\let\ds\displaystyle

\begin{document}
\ExamInstrBox{
Respuesta sin justificar mediante procedimiento no será tenida en cuenta en la calificación. Escriba sus respuestas en el espacio indicado. Tiene 45 minutos para contestar esta prueba.}
\LineaNombre
\begin{enumerate}
\item Halle el volumen y el área superficial de un cubo cuya arista mide 9 cm \noanswer

\begin{minipage}{0.5\textwidth}
    \item Complete el dibujo.¿Cuántos cubos como éste \tikz \draw (0,.5) -- (0,0) -- (.5,0) -- (.5,.5) -- (0,.5) -- (.2,.7) -- (.7,.7) --(.5,.5)--(.5,0)--(.7,.2)--(.7,.7); de $1u^{3}$, forman el prisma cuadrangular de la figura?
     \end{minipage}\hfill
    \begin{minipage}{0.4\textwidth}
      \begin{tikzpicture}[scale=.5]
      \draw (0,4,0)--(3,4,0)--(3,4,2)--(0,4,2)--cycle;
      \draw (0,4,2)--(0,0,2)--(3,0,2)--(3,4,2);
      \draw (3,4,0)--(3,0,0)--(3,0,2);
      \draw (0,1,2)--(1,1,2);
      \node[below] at (1.5,0,2) {3};
      \node [right] at (3,0,1) {2};
      \node [right] at (3,2,0) {4};
      \draw (1,4,2)--(1,0,2);
      \draw (2,4,2)--(2,0,2);
      \draw (0,4,1)--(3,4,1);
      \end{tikzpicture}
    \end{minipage} \noanswer  
 \item  \begin{minipage}{.2\textwidth}
 \begin{center}
 \begin{tikzpicture}[scale=2.5]
\draw (0,0) -- (1,0) -- node[right]{10} (.5,0.866) -- cycle;
\draw[dashed] (0.5,0) -- (0.5,.866);
\node at (0.5,0.15) {$h=5\sqrt{3}$};  
\end{tikzpicture}
 \end{center}
 \end{minipage}\hfill
 \begin{minipage}{.65\textwidth}
Halle el área del triángulo equilátero de la figura, cuya altura es $h=5\sqrt{3}$ 
 \end{minipage}\noanswer
 \item Calcule el área superficial de un tetraedro cuya arista mide 7 cm.\noanswer
 \item Calcula el volumen y el área superficial de un ortoedro cuyas dimensiones son 4 cm, 8 cm y 3 cm.\noanswer
 \item \begin{minipage}{.55\textwidth} 
 Calcule el área superficial de un tetraedro cuya arista mide 14 cm. Calcule
 \end{minipage}\hfill
 \begin{minipage}{.4\textwidth}
 \begin{tikzpicture}[scale=1.5]
\draw (0,0) -- (.5,-.43) --  (1,0) --node[right]{14} (.5,.866) -- cycle;
\draw (0.5,-.43)  --   (0.5,.866);
\draw[dashed](0,0)--(1,0); 
\end{tikzpicture}
 \end{minipage}
\noanswer
 
 \end{enumerate}

\end{document}
