\documentclass[letterpaper]{article}
\usepackage[utf8]{inputenc}
\usepackage{amsmath,amsfonts,amssymb,amsthm,latexsym}
\usepackage[spanish,es-noshorthands]{babel}
\usepackage[T1]{fontenc}
\usepackage{lmodern}
\usepackage{graphicx,hyperref}
\usepackage{tikz,pgf}
\usepackage{multicol}
\usepackage{subfig}
\usepackage{fancyhdr}
\usepackage{marvosym}
\usepackage[includeheadfoot,left=0.4in,right=0.3in,top=0.3in,bottom=0.3in]{geometry}
\pagestyle{fancy}
\fancyhead[LE]{\url{www.autistici.org/mathgerman}}
\fancyhead[RE]{\Email~ matematicas.german@gmail.com}
\fancyhead[RO]{\url{www.autistici.org/mathgerman}}
\fancyhead[LO]{\Email~ matematicas.german@gmail.com}

\author{Germ\'an Dar\'io Avenda\~no Ram\'irez~\thanks{Lic. Mat U.D., M.Sc. U.N.}}
\title{\begin{minipage}{0.15\textwidth}\includegraphics[height=2cm]{Images/logo-colegio.png}
\end{minipage}\hfill \begin{minipage}{0.7\textwidth}\begin{center}
Guía 5, Probabilidad condicional\\
Probabilidad $11^{\circ}$\end{center}
\end{minipage}\hfill
\begin{minipage}{0.15\textwidth}
  \includegraphics[height=2cm]{Images/logo-sed.png} 
\end{minipage}}
\date{}

\begin{document}
\maketitle
Nombre: \hrulefill Curso: 110\underline{\hspace{12pt}}  Fecha: \underline{\hspace{2cm}}\\

\begin{multicols}{2}
\section*{Para recordar}
Es muy importante para resolver problemas de probabilidad, tener claro cual es el espacio muestra $S$
\subsubsection*{Espacios muestrales}
Un conjunto que consiste en todos los resultados de un experimento aleatorio se llama un espacio muestral y cada uno de los resultados se denomina punto muestral. Con frecuencia habrá más de un espacio muestral que describe los resultados de un experimento pero hay comúnmente
sólo uno que suministra la mayoría de la información. Obsérvesé que $S$ córresponde al conjunto universal.
\subparagraph*{Ejemplo:}  si lanzamos una moneda dos veces y utilizamos 0 para representar sellos y 1 para representar caras el espacio muestral puede dibujarse por puntos como en la Fig. donde, por ejemplo, (0,1 ) representa sello en el primer lanzamiento y cara en el segundo lanzamiento, es decir SC,
\begin{center}
\begin{tikzpicture}
\draw[->](-1,0)--(2,0) ;
\draw[->](0,-.5)--(0,1.5);
\fill(1,0)node[below right]{(1,0)} circle(2pt) ;
\fill(0,0)node[below left]{(0,0)}circle(2pt) ;
\fill(0,1)node[left]{(0,1)}circle(2pt) ;
\fill(1,1)node[right]{(1,1)}circle(2pt);
\end{tikzpicture}
\end{center}
\subsection*{Sucesos:}
Un suceso es un subconjunto $A$ del espacio muestral $S$, es decir es un conjunto de resultados posibles. Si el resultado de un experimento es un elemento de $A$ decimos que el suceso $A$ ha ocurrido. Un suceso que consiste de un solo punto de $S$ frecuentemente se llama un \emph{Suceso elemental} o \emph{simple}.\\

Puesto que los sucesos son conjuntos es lógico que las
proposiciones relativas a sucesos pueden traducirse en el lenguaje de la teoría de conjuntos e inversamente. En particular tenemos un álgebra de sucesos que corresponde al álgebra de conjuntos.\\

Empleando las operaciones de conjuntos en sucesos en $S$ podemos obtener otros sucesos en $S$. Así si $A$ y $B$ son sucesos, entonces
\begin{enumerate}
\item $A\cup B$ es el suceso "$A$ ó $B$ o ambos".
\item $A\cap B$ es el suceso "tanto $A$ como $B$".
\item $A^{\complement}$ es el suceso "no $A$".
\item $A-B$ es el suceso "$A$ pero no $B$".
\end{enumerate}
\subparagraph*{Ejemplo}: Haciendo referencia al experimento de lanzar una moneda dos veces, sea $A$ el suceso "por lo menos resulte una cara" y $B$ el suceso "el segundo lanzamiento sea un sello". Entonces $A=\{CS,SC,CC\}$, $B=\{CS,SS\}$, así tenemos
\[A\cup B=\{CS,SC,CC,SS\}=S\qquad A\cap B=\{CS\}\]
\[A^{\complement}=\{SS\}, \qquad A-B=\{SC,CC\}\]
\subsection*{El concepto de probabilidad}
En cualquier experimento aleatorio siempre hay incertidumbre sobre si un suceso específico ocurrirá o no. Como medida de la \emph{oportunidad} o \emph{probabilidad} con la que podemos esperar que un suceso ocurra es conveniente asignar un número entre 0 y 1. Si estamos seguros de que el suceso ocurrirá decimos que su probabilidad es \%100 ó 1, pero si estamos seguros de que el suceso no ocurrirá decimos que su probabilidad es cero. Por ejemplo, si la probabilidad es de 1/4, diríamos que hay un 25\% de oportunidad de que ocurra y un 75\% de oportunidad de que no ocurra. Equivale a decir que la \emph{probabilidad} contra su ocurrencia es del 75\% al 25\% o de 3 a 1.

Ahora bien, tenemos dos reglas que nos permiten trabajar con problemas de probabilidad donde se involucran dos o más sucesos (dependientes o independientes) y son:
\begin{equation}
P(A\cup B)=P(A)+P(B)-P(A\cap B)\label{eq1}
\end{equation}
y
\begin{equation}
P(A\cap B)=P(A)P(B|A) \label{eq2}
\end{equation}
Obsérvese que si los eventos $A$ y $B$ son independientes, tanto $P(A\cap B)=0$ y $P(B|A)=P(B)$ (la ocurrencia de $A$ no afecta la ocurrencia de $B$) con lo cual las ecuaciones \ref{eq1} y \ref{eq2} quedarían $P(A\cup B)=P(A)+P(B)$ y $P(A\cap B)=P(A)P(B)$ respectivamente.
\subsubsection*{Ejemplo 1}
Un sistema contiene dos componentes $A$ y $B$ y se conecta de manera que éste funciona si cualesquiera de las componentes funciona. Se sabe que la probabilidad de que $A$ funciones es $P(A)=0.9$ y la de $B$ es $P(B)=0.8$ y la probabilidad de ambos $P(A\cap B)=0.72$. Determinar la probabilidad de que el sistema funcione.

La probabilidad de que el sistema funcione es igual a la probabilidad de la unión entre $A$ y $B$; de esta manera
\begin{align*}
P(A\cup B)&=P(A)+P(B)-P(A\cap B)\\
&=0.9+0.8-0.72=0.98
\end{align*}
\subsubsection*{Ejemplo 2}
A los habitantes de una gran ciudad se les hizo una encuesta con el propósito de determinar el número de lectores de \textit{Time} y \textit{Newsweek}. Los resultados de la encuesta fueron los siguientes: 20\% de los habitantes leen el \textit{Time}, el 16\% lee le \textit{Newsweek} y un 1\% lee ambos semanarios. Si se selecciona al azar a un lector de \textit{Time}, ¿cuál es la probabilidad de que también lea el \textit{Newsweek}?

Sean $A$ y $B$ los eventos que representan el número de lectores del \textit{Time} y \textit{Newsweek} respectivamente; dado que $P(A)=0.2$, $P(B)=0.16$ y $P(A\cap B)=0.01$
\[P(B|A)=\dfrac{P(A\cap B)}{P(A)}=\dfrac{0.01}{0.2}=0.05\]
Esta última ecuación se deriva de la ecuación \ref{eq2} despejando la probabilidad de que ocurra $B$ dado que ha ocurrido $A$

Por otra parte, también puede determinarse la probabilidad de que un lector del \textit{Newsweek} lea también el \textit{Time}; esto es:
\[P(A|B)=\dfrac{P(A\cap B)}{P(B)}=\dfrac{0.01}{0.16}=0.0625\]
y se verifica la relación $P(A)P(B|A)=P(B)P(A|B)$, o $(0.2)(0.05)=(0.16)(0.0625)$
\section*{Actividad}
\begin{enumerate}
\item En una cadena de producción el producto fabricado pasa por tres procesos independientes. En el primero hay un 5\% de fallos, en el segundo un 10\% y en el tercero un 15\%. Calcula la probabilidad de que un producto tenga:
\begin{enumerate}
\begin{multicols}{3}
\item 0 defectos
\item 1 defecto
\item 2 defectos
\end{multicols}
\end{enumerate}
\item De cada 8 partidos que juegan los equipos de Cabanillas y Fustiñana, Cabanillas gana 5, empata 2 y pierde 1. Si ambos juegan un torneo a tres partidos, calcula la probabilidad de que:
\begin{enumerate}
\item Cabanillas gane los tres partidos.
\item Dos partidos terminen en empate.
\end{enumerate}
\item El coche de Quique no funciona muy bien, pues el 15\% de las veces no arranca a la primera. Cuando arranca, llega tarde al trabajo con una probabilidad de 0,3, pero si no arranca, la probabilidad de que llegue tarde es de 0,8.
\begin{enumerate}
\item Calcula la probabilidad de que llegue tarde y haya arrancado el coche a la primera
\item ¿Cuál es la probabilidad de que llegue pronto al trabajo?
\end{enumerate}
\item En un país hay elecciones para la presidencia del Gobierno y se presentan dos candidatos. El 35\% de los votantes vota al candidato A, el 32\% al candidato B y el resto se abstiene. Se sabe, además, que el 12\% de los votantes de A, el 20\% de los de B y el 10\% de los que se abstienen pertenecen a la clase media. Si elegimos un votante, ¿cuál es la probabilidad de que pertenezca a la clase media?
\item Una caja de galletas contiene siete galletas de chocolate y cuatro de coco, otra caja tiene cinco de chocolate y tres de coco. Si elegimos una caja al azar y de ella se extraen dos galletas sin reemplazamiento, calcula:
\begin{enumerate}
\item ¿Cuál es la probabilidad de que las dos sean de chocolate?
\item ¿Y de que las dos sean de coco?
\item ¿Cuál es la probabilidad de que sea una de chocolate y una de coco?
\end{enumerate}
\item La probabilidad de que un alumno apruebe matemáticas es de 0,5 y la probabilidad de aprobar ciencias, habiendo aprobado matemáticas, es de 0,6. ¿Cuál es la probabilidad de aprobar matemáticas y ciencias?
\item Se sacan dos cartas sucesivamente y sin devolución de una baraja española. Hallar las siguientes probabilidades:
\begin{enumerate}
\item Que la segunda carta extraída sea espada, sabiendo que la primera carta fue espada
\item Que la segunda carta extraída sea espada, sabiendo que la primera carta fue copa
\end{enumerate}
\item De una urna que contiene nueve bolas negras y cinco rojas se extraen sucesivamente dos bolas. Halla la probabilidad de los siguientes sucesos:
\begin{enumerate}
\item Que las dos bolas sean negras
\item Que las dos bolas sean rojas
\item Que la primera sea roja y la segunda negra
\item Que una sea roja y la otra negra
\end{enumerate}
\item Se lanza un dado diez veces. ¿Cuál es la probabilidad de obtener sólo números pares?
\item Un producto está formado por tres partes: A, B y C. El proceso de fabricación es tal
que la probabilidad de un defecto en A es 0,03, de un defecto en B es 0,04 y de un defecto en C es 0,08. ¿Cuál es la probabilidad de que el producto no sea defectuoso?
\end{enumerate}
\end{multicols}
\end{document}
