\documentclass[letterpaper,fleqn]{article}
\usepackage[spanish,es-noshorthands]{babel}
\usepackage[utf8]{inputenc} 
\usepackage[papersize={6.5in,8.5in},left=1cm, right=1cm, top=1.5cm, bottom=1.7cm]{geometry}
\usepackage{mathexam}
\usepackage{amsmath}
\usepackage{amsfonts}
\usepackage{amssymb}
\usepackage{graphicx}

\ExamClass{\includegraphics[height=16pt]{Images/logo-sed.png} Aritmética $6^{\circ}$}
\ExamName{Operaciones y problemas con $\mathbb{N}$}
\ExamHead{\includegraphics[height=16pt]{Images/logo-colegio.png} IEDAB}
\newcommand{\LineaNombre}{%
\par
\vspace{\baselineskip}
Nombre:\hrulefill \; Curso: \underline{\hspace*{48pt}} \; Fecha: \underline{\hspace*{2.5cm}} \relax
\par}
\let\ds\displaystyle

\begin{document}
\ExamInstrBox{
Respuesta sin justificar mediante procedimiento no será tenida en cuenta en la calificación. Escriba sus respuestas en el espacio indicado. Tiene 45 minutos para contestar esta prueba.}
\LineaNombre
\begin{enumerate}
\item Resuelva las siguientes operaciones:
\begin{enumerate}
\item $12\cdot 3+5\cdot 7-8\cdot 3=$ \noanswer
\item $25\cdot [3\cdot 5-8]=$ \noanswer
\item $3^{2}+5^{2}+2^{3}-10=$\noanswer
\item $148\cdot 13+125\cdot 18-24\cdot 15=$  \noanswer
\end{enumerate}
 \item ¿Cuál es el número que al dividirlo entre 45, su cociente exacto es 54?\noanswer
 \newpage
 \item ¿Cuál es el número que al dividirlo entre 24 su cociente es 32 y su residuo es 8? \noanswer
 \item En un almacén se han vendido 48 lápices, si una docena y media cuestan \$6300, ¿cuánto cuesta cada lápiz? ¿Cuánto cuestan los 48 lápices?\noanswer
 \item Un camionero carga 3 neveras y dos lavadoras. Si cada nevera pesa como dos lavadoras y en total ha cargado 80 kilos, ¿cuánto pesa cada aparato?\noanswer
 \end{enumerate}

\end{document}
