\documentclass[letterpaper,10pt,twoside]{article}
\usepackage[utf8]{inputenc}
\usepackage{amsmath,amsfonts,amssymb,amsthm,latexsym,cancel}
\usepackage[spanish,es-noshorthands]{babel}
\usepackage[T1]{fontenc}
\usepackage{lmodern}
\usepackage{graphicx,hyperref}
\usepackage{tikz,pgf}
\usepackage{multicol}
\usepackage{subfig}
\usepackage{marvosym}
\usepackage[width=7.5in,height=9.5in]{geometry}
\usepackage{fancyhdr}
\usepackage{textcomp}
\pagestyle{fancy}
\fancyhead[LE]{\url{https://www.autistici.org/mathgerman}}
\fancyhead[RE]{}
\fancyhead[RO]{\Email~ iedabgerman@autistici.org}
\fancyhead[LO]{}

\author{Germ\'an Dar\'io Avenda\~no Ram\'irez~\thanks{Lic. Mat. U.D., M.Sc. U.N.}}
\title{\begin{minipage}{0.15\textwidth}\includegraphics[height=1.7cm]{Images/logo-colegio.png}
\end{minipage}\hfill \begin{minipage}{0.85\textwidth}\begin{center}
Recomendaciones período I\\Matemáticas $11^{\circ}$\end{center}
\end{minipage}}
\date{}

\begin{document}
\maketitle
\section{Cálculo}
Este taller debe ser resuelto en hoja examen y entregado en la fecha estipulada, para después ser sustentado. Al hacer este taller, se espera que el estudiante repase y aprenda los temas que le causaron dificultad en el primer período.
\subsection{Taller}
  \begin{enumerate}
  \item Complete esta tabla con sí o no
  \begin{center}
 \begin{tabular}{|l|c|c|c|c|c|c|c|}\hline
 Número & 8 & $ \sqrt{3} $ & $ -1,15 $ & $ 1,1515515551\ldots $ & $ 7/8 $ & $ \sqrt{25} $ & $\sqrt{-8}$\\
 \hline Natural & & & & & & &\\
 \hline Entero & & & & & & &\\
 \hline Racional & & & & & & &\\
 \hline Irracional & & & & & & &\\
 \hline Real & & & & & & &\\\hline
 \end{tabular}
  \end{center}
\setlength{\columnsep}{25pt}\begin{multicols}{2}
    \item Escribe los siguientes números en forma decimal y redondeando a la céntesimas: (puedes usar calculadora)
  \begin{enumerate}\begin{multicols}{2}
    \item $ 2\pi $ \item $ \sqrt{5} $  \item $ 1,2626\ldots $  
    \item $ 1,7575\ldots $ \item $ \frac{7}{9} $ \item $ 2\sqrt{3} $
  \end{multicols}
  \end{enumerate}
  \item Escribe tres números racionales comprendidos entre $ \frac{1}{15} $ y $ \frac{2}{15} $
  \item Representa en la recta real los siguientes intervalos
  \begin{enumerate}\begin{multicols}{2}
    \item $ [-2,3] $    \item $ (-5,-3) $\item $ [3,5) $
    \item $ (-8,-5) $ \item $ (-\infty,5] $ \item $ (3,\infty) $
  \end{multicols}
  \end{enumerate}
  \item Representa en la recta real los números que verifican:
  \begin{enumerate}\begin{multicols}{3}
    \item $ |x|=0 $    \item $ |x|=2 $ \item $ |x|=|-3| $
    \item $ |x|=-1 $ \item $ |x|=8 $ \item $ |x|=|-4| $
  \end{multicols}
  \end{enumerate}
  \item Representa en la recta real los intervalos que verifican:
  \begin{enumerate}\begin{multicols}{3}
    \item $ |x|\leq 3 $    \item $ |x|<3 $
    \item $ |x|\geq 3 $ \item $ |x|>3 $
  \end{multicols}
  \end{enumerate}
  \item Encuentra las fracciones generatrices de:
  \begin{enumerate}\begin{multicols}{4}
    \item 1,212 \item $ 10,\overline{3} $
    \item $ 2,\overline{34} $ \item $ 3,01\overline{72} $    
  \end{multicols}
  \end{enumerate}
  \item En la descomposición de cierta cantidad de agua por electrólisis, se obtienen 4 litros de hidrógeno y 32 litros de oxígeno ¿Cuál es la producción de hidrógeno? ¿Y de oxígeno? Expresa los resultados en tanto por ciento. ¿Qué cantidad de oxígeno se obtendrán con 54 litros de agua?
  \item Ordena de menor a mayor los siguientes números reales
  \[ \sqrt{5},\quad 175,\quad \frac{-2}{3},\quad 3\pi, \quad -0,55, \quad 2,\overline{75}, \quad 1,7\overline{8}, \quad -\frac{2}{5} \]
  \item Calcula:
  \begin{enumerate}\begin{multicols}{2}
    \item $ \sqrt{1024} $    \item $ \sqrt{441} $
    \item $ \sqrt[3]{729} $ \item $ \sqrt[4]{1296} $
    \item $ \sqrt[5]{-1} $ \item $ 28-2\sqrt{81} $
    \item $ \sqrt{4+2\cdot16} $
    \item $ 8+2\sqrt[3]{-8} $ \item $ \sqrt{400-16-60} $
    \item $ \sqrt{5+\sqrt{13\sqrt{9}}} $
    \item $ \sqrt{10+2\sqrt{7+\sqrt[3]{8}}} $
    \item $ \sqrt{4,7+1,06} $
    \item $ 3\sqrt[3]{0,001+2} $ \item $ \sqrt[4]{\frac{1}{625}} $
    \item $ \sqrt[5]{\frac{-1}{32}} $ \item $ \sqrt{\frac{1}{4}}-\sqrt{\frac{9}{25}} $ \item $ \sqrt{\frac{9}{4}\div \sqrt{\frac{121}{25}}} $
  \end{multicols}
  \end{enumerate}
  \item Calcula
  \begin{enumerate}
    \item $ \left(\sqrt{5+\sqrt{5}}\right)\left(\sqrt{5-\sqrt{2}}\right) $    \item $ (2+\sqrt{3})^2 $ \item $ (1+\sqrt{2})(1+\sqrt{2})\sqrt{2} $
  \end{enumerate}
  \item Transforma en radicales
  \begin{enumerate}\begin{multicols}{2}
    \item $ (-3)^{\frac{1}{5}} $  \item $ \left(\frac{3}{5}\right)^{3/7} $  \item $ \left(\frac{2}{3}\right)^{-3/2} $
    \item $ \left(\frac{1}{5}\right)^{-1/4} $
  \end{multicols}
  \end{enumerate}
  \item Halla usando la calculadora
  \[ \sqrt[5]{12^3}\qquad \dfrac{1}{\sqrt[7]{7^4}}\qquad \sqrt[3]{11^2} \]
  \item Encuentra todos los números de tres cifras que sean cubos de un número natural
  \item ¿Cuántas baldosas cuadradas de 20 cm de lado se necesitan para recubrir una superficie de 27,04 $ m^2 $?
  \item Halla la arista de un cubo cuyo volumen es 46,656 $m^3$.
  \item Un depósito cúbico tiene una capacidad de 157.464 litros. ¿Cuál es la superficie de cada una de las paredes del deposito?
\item Calcula y simplifica
\begin{enumerate}\begin{multicols}{2}
  \item $ \sqrt{12}-\sqrt{48}+\sqrt{27} $  
  \item $ \sqrt[3]{24}-\sqrt[3]{375}+\sqrt[3]{81} $
\end{multicols}
\end{enumerate}
\item Calcula y simplifica
\begin{enumerate}\begin{multicols}{2}
  \item $ (\sqrt{5}+\sqrt{2})^2 $  \item $ (2\sqrt{5}+\sqrt{3})^2 $ \item $ (\sqrt{3}-\sqrt{2})(2-\sqrt{2}) $ \item $ (\sqrt{5}+\sqrt{3})(\sqrt{5}-\sqrt{3}) $
\end{multicols}
\end{enumerate}
\item Calcula y simplifica
\begin{enumerate}\begin{multicols}{3}
  \item  $ \sqrt{\frac{7}{5}}\sqrt{35} $
  \item $ \sqrt{\frac{3}{2}}\sqrt{\frac{8}{3}} $
  \item $ \sqrt{\frac{10}{3}}\sqrt{7,5}$
\end{multicols}
\end{enumerate}
\item Dibuja un cuadrado de 6 cm de lado. Dibuja otro cuadrado que tenga doble área.
\item Dibuja un rectángulo cuya diagonal valga 7
\item Las dimensiones de una aula son 10 m de largo, 8 m de ancho y 3,40 m de alto. Dos moscas revoletean por el aula. ¿Cuál es la distancia máxima a que pueden encontrarse?
\item Representa:
\begin{enumerate}\begin{multicols}{2}
  \item $ [4,6]\cup (9,11)$
  \item $ [-6,5]\cap (2,5) $ 
  \item $ (2,7)\cap (5,9)\cap (6,10) $
  \item $ (-3,2)\cup (-\infty,0) $
\end{multicols}
\end{enumerate}
Conteste el punto 25 con base en la siguiente información:\\

\emph{Racionalizar} consiste en quitar los radicales que aparezcan en el denominador de una fracción, multiplicando por éste mismo radical o por su conjugado en el caso de que haya un binomio así:
\paragraph{Ejemplo 1:} Racionalizar la expresión $ \dfrac{6}{\sqrt{3}} $. En este caso multiplicamos tanto el numerador como el denominador, por el radical del denominador así:
\[\dfrac{6}{\sqrt{3}}=\dfrac{6\sqrt{3}}{\sqrt{3}\sqrt{3}}=\dfrac{6\sqrt{3}}{3}=2\sqrt{3}\]
\paragraph{Ejemplo 2:} Racionalizar la expresión $ \dfrac{5}{2-\sqrt{3}} $. En este caso buscamos el conjugado del denominador que es $ 2+\sqrt{3} $ y luego multiplicamos tanto el numerador, como el denominador por el conjugado del denominador así:
\begin{align*}
\dfrac{5}{2-\sqrt{3}}&=\dfrac{5(2+\sqrt{3})}{(2-\sqrt{3})(2+\sqrt{3})}=\dfrac{5(2+\sqrt{3})}{2^2-\sqrt{3}^2}=\dfrac{5(2+\sqrt{3})}{4-3}\\
&=\dfrac{5(2+\sqrt{3})}{1}=5(2+\sqrt{3})=10+5\sqrt{3}
\end{align*}
  \item Racionalice las siguientes expresiones:
  \begin{multicols}{2}
  \begin{enumerate}
    \item $ \dfrac{7}{\sqrt[3]{2}} $ \item $ \dfrac{2}{\sqrt[5]{3^3}} $
    \item $ \dfrac{1}{\sqrt{x+y}} $ \item $ \dfrac{1}{\sqrt{a}-\sqrt{b}} $
  \end{enumerate}
  \end{multicols}
  Recuerde que $ \sqrt[n]{a^x}=a^{\frac{x}{n}} $.
  \item Calcule, expresando el resultado en forma radical
  \begin{multicols}{2}
    \begin{enumerate}
      \item $ \dfrac{(2^{1/3}\cdot3^{1/2})}{3^3\cdot2^{2/5}} $
      \item $ \dfrac{a^{1/2}a^2a^{2/3}}{(a^{-2}a^{3/2})^2} $
      \item $ \sqrt[3]{\dfrac{a^2b^3}{8c}}\sqrt[6]{\dfrac{9c}{4ab^2}} $
    \end{enumerate}
      \end{multicols}
    \item Una célula se divide en dos partes cada 0,5 segundos, que a su vez se dividen en otras dos en el mismo tiempo. ¿Cuántas células habrá al cabo de 20 segundos? Exprese el resultado en notación científica.
    \item Encuentre dos números capicúas entre 400 y 800 que sean cuadrados de un número natural. (Un número capicúa es un número simétrico que puede leerse de igual manera de izquierda a derecha que de derecha a izquierda; por ejemplo el número 252 es un número capicúa.)
    \item Señale si son ciertos o falsos los siguientes enunciados:
    \begin{enumerate}
      \item El número 7/11 es irracional porque tiene una cantidad ilimitada de cifras decimales
      \item La longitud de cualquier circunferencia es más de 6 veces la de su radio
      \item 7,73 es una aproximación por defecto de $ \sqrt{3} $
      \item Todo número real es racional
      \item No es posible medir con exactitud la diagonal de un rectángulo cuyos lados miden 8 cm y 6 cm
    \end{enumerate}
    \item La superficie en $ m^2 $ de una esfera es igual a numéricamente a su volumen expresado en $ m^3 $. ¿Puede ser esto posible? Si su respuesta es afirmativa, ¿cuánto vale el radio de esa esfera?
  \end{multicols}
  \end{enumerate}
 \section{Probabilidad}
 \subsection*{Estad\'istica y los dulces}
De dónde vienen todos estos dulces tan coloridos?

¿Sabía usted que tienen 21 colores?

¿Sabía usted que la idea para los Dulces Sencillos de Chocolate “M\&M’s” nació
en el “telón de fondo” de la guerra civil española? Cuenta la leyenda que en un
viaje a España, Forrest Mars Sr. encontró soldados que comían bolitas de chocolate
cubiertas de una capa azucarada dura para evitar que se derritieran. Mr. Mars se
inspiró en este concepto y regresó a casa e inventó la receta para los Dulces Sencillos de Chocolate “M\&M’s”.
\begin{table}[h!]
\centering
\begin{tabular}{cc}
\multicolumn{2}{c}{Colores M\&M's por cantidad} \\ \hline
Color & Cantidad \\ \hline
Café & 91 \\ 
Amarillo & 112 \\ 
Rojo & 102 \\ 
Azul & 151 \\ 
Naranja & 137 \\ 
Verde & 99 \\ \cline{2-2}
 & 692 \\ 
\hline
\end{tabular}
\caption{Tabla 1} \label{tab1}
\end{table}

La clase de estadística había comenzado y el maestro estaba hablando de porcentajes, proporciones y probabilidad, y en qué forma son semejantes pero diferentes. De pronto una estudiante dijo que escuchó que el grupo del semestre anteriorhizo una lección usando, y comiendo, chocolates M\&M’s; ella preguntó si el grupo
de este año haría algo semejante. La conversación pronto se enfocó por entero en los chocolates M\&M’s, sus combinaciones de color y el porcentaje de cada color. A los 24 miembros del grupo se les pidió que calcularan el porcentaje de cada color que ellos pensaban estaba contenido en estas pequeñas bolsas de color café de los
Dulces Sencillos de Chocolate M\&M’s. Se les dijo que habría un premio para la persona cuyo cálculo fuera el más cercano al número real. Cada estudiante escribió los porcentajes y los entregó; a su vez, los estudiantes recibieron una pequeña bolsa café. “Ah, ¡esto es esa lección!”. “Sí” dijo el maestro, “y antes que abran esas
bolsas, debemos tener un plan”. Cada estudiante debía contar el número de chocolates M\&M's de cada color en su bolsa y anotar las seis cantidades; a continuación podrían determinarse los totales del grupo. En la tabla~\ref{tab1} aparece la distribución de cantidades resultante.

Los totales del grupo se convirtieron a porcentajes (tabla~\ref{tab2}), y a cada estudiante se le pidió determinar los seis porcentajes que observaran en su propia bolsa de chocolates M\&M's.

La discusión que siguió se centró en la variación que había de una bolsa a la
otra, con algunos estudiantes bastante sorprendidos de ver tanta variación. Varias
bolsas no tenían nada o sólo una pastilla de un color, y unas pocas bolsas tenían una
proporción más bien grande de sólo uno o dos colores. ¿Alguna vez había usted observado algunos de estos extremos cuando abría una bolsa de chocolates M\&M's?
\begin{table}[h!]
\centering
\begin{tabular}{cc}
\multicolumn{2}{c}{Colores de M\&M's en porcentajes} \\ 
\hline 
Color & Porcentaje \\ 
\hline 
Café & 13.2 \\ 
Amarillo & 16.2 \\ 
Rojo & 14.7 \\ 
Azul & 21.8 \\ 
Naranja & 19.8 \\ 
Verde & 14.3 \\ \cline{2-2} 
 & 100.0 \\ 
\hline 
\end{tabular} 
\caption{Tabla 2} \label{tab2}
\end{table}

Los porcentajes reportados en la tabla~\ref{tab2} son los de cada color hallados en esta
muestra de 692 bolsas M\&M's’s. Los porcentajes se comportan en forma muy semejante a números de probabilidad, pero la pregunta que se hace en probabilidad es diferente. En la ilustración precedente, estamos tratando la información como datos muestrales y describiendo los resultados que encontramos. Si ahora pensamos
en términos de una probabilidad, vamos a dar un giro y tratar todo el conjunto de
las 692 bolsas de M\&M's’s como si fuera la lista completa de posibilidades, y hacer
preguntas acerca de la semejanza de ciertos eventos cuando se selecciona una bolsa
de M\&M's’s de todo el conjunto de 692 bolsas.

Por ejemplo, supongamos que se vacían las 692 bolsas de M\&M's’s en un gran
tazón y mezclamos perfectamente los chocolates. Ahora considere la pregunta “Si
al azar se selecciona un chocolate del tazón, ¿cuál es la probabilidad de que sea de
color naranja?” Esperamos que el lector piense así: seleccionados al azar significa
que cada chocolate M\&M's’s tiene la misma probabilidad de ser elegido y, como hay
137 chocolates color naranja en el tazón, la probabilidad de seleccionar un chocolate de color naranja M\&M's’s es 137/692, es decir 0.198

Ya antes hemos visto este número 0.198, sólo que se expresaba como 19.8\%.
Los porcentajes y los números de probabilidad son “lo mismo, pero diferentes.” (Es
probable que ya antes y en algún lugar usted haya oído esto.) Los números tienen
el mismo valor y se comportan con las mismas propiedades; no obstante, la orientación de la situación y las preguntas hechas son diferentes, como veremos más adelante.
\subsubsection*{Ejercicios}
\begin{enumerate}
\item 
\begin{enumerate}
\item Si compró una bolsa de chocolates M\&M’s, ¿qué color de M\&M’s esperaría ver más? ¿Qué color menos? ¿Por qué? 
\item Si compró una bolsa de chocolates M\&M’s, ¿esperaría hallar los porcentajes mencionados en la tabla~\ref{tab2} Si no es así, ¿por qué y qué esperaría? 
\end{enumerate}
\item
\begin{enumerate}
\item Construya una gráfica de barras que muestre los porcentajes de la tabla 4.2 obtenidos a
partir de los 692 chocolates M\&M’s.
\item Con base en su gráfica, ¿qué color de chocolates M\&M’s hubo con más frecuencia?
¿Cómo se muestra esto en su gráfica?
\item Con base en su gráfica, ¿qué color de chocolates M\&M’s hubo con menos frecuencia?
¿Cómo se muestra esto en su gráfica?
\end{enumerate}
\item Si recibiera una pequeña bolsa de 40 chocolates M\&M’s, usando los porcentajes de la tabla~\ref{tab2} ¿cuán   tos de cada color “esperaría” encontrar?
\item ¿Tablas malas? Así como hay gráficas malas, hay tablas malas, es decir, tablas engañosas y difíciles de leer. Un grupo llamado  Madres Contra Conductores Borrachos (MADD, por sus siglas en inglés) presentó la siguiente tabla referente a 6764 muertos en accidentes de tránsito que ocurrieron en 2002.
\begin{center}
\begin{tabular}{lcc}
 &  & Total muertes \\ 
 & Total muertes & relacionadas \\ 
\hspace{15pt} Días festivos 2002 & en tránsito & con alcohol \\ 
\hline 
Víspera de años nuevo & 118 & 45 \\ 
Día de año nuevo & 165 & 94 \\ 
Días festivos de año nuevo & 575 & 301 \\ 
Domingo de super tazón & 147 & 86 \\ 
Día de San Patricio & 158 & 72 \\ 
Conmemoración de los caídos & 491 & 237 \\ 
Cuatro de julio & 683 & 330 \\ 
Fin de semana de día del trabajo & 541 & 300 \\ 
Halloween & 268 & 109 \\ 
Día de gracias & 543 & 265 \\ 
Día de gracias-año nuevo & 4019 & 1561 \\ 
Navidad & 130 & 68 \\ 
Víspera de año nuevo 2002 & 123 & 57 \\ 
\hline \footnote{Fuente: Mothers Against Drunk Driving (MADD), \url{http://www.infoplease.com/ipa/A0777960.html}}
\end{tabular}
\end{center}
\begin{enumerate}
\item Los totales de columna no están incluidos porque
serían valores que carecen de sentido. Examine la
tabla y explique por qué.
\item Seleccione los días festivos apropiados que no se
traslapan (columna 1) y verifique el número total
de 6764 muertos en accidentes de tránsito para
2002.
\item Usando los días festivos seleccionados en la parte b, encuentre el número total de muertos en accidentes de tránsito relacionados con alcohol en
días festivos en 2002.
\item Describa cómo organizaría esta tabla para hacerla que tenga sentido.
\end{enumerate}
\item Haga el experimento propuesto, o, utilice calculadora o computadora generando números aleatorios para simular lo siguiente:
\begin{enumerate}
\item Tirar 50 veces un dado; exprese sus resultados como frecuencias relativas.
\item Tirar al aire una moneda 100 veces; exprese sus resultados como frecuencia relativa.
\end{enumerate}
\item Utilice ya sea, calculadora o computadora para simular la selección aleatoria de 100 números de un solo dígito, 0 al 9.
\begin{enumerate}
\item Haga una lista de los 100 dígitos.
\item Elabore una distribución de frecuencia relativa de los 100 dígitos.
\item Elabore un histograma de frecuencia relativa de la distribución en la parte b.
\end{enumerate}
\end{enumerate}
\end{document}
