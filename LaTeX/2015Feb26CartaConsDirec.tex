\documentclass[spanish,11pt]{letter}
\usepackage[T1]{fontenc}
\usepackage[utf8]{inputenc}
\usepackage[spanish]{babel}
\usepackage{lmodern}
\usepackage{amsmath}
\usepackage{amsfonts}
\usepackage{amssymb}
\usepackage[papersize={8.in,13in},top=4cm,left=3cm,right=3cm,bottom=2.5cm]{geometry}
\address{Profesores j.m.\\Colegio Arborizadora Baja I.E.D.}
\signature{Profesores j.m. firmantes en hoja anexa\\Colegio Arborizadora Baja I.E.D.}
\date{26 de febrero de 2015}

\begin{document}

\begin{letter}{CONSEJO DIRECTIVO\\Colegio Arborizadora Baja I.E.D.}
	
\opening{Cordial saludo}
En nuestra calidad de docentes de la jornada mañana, manifestamos las siguientes inconformidades que afectan nuestra labor con los estudiantes y nuestra salud física y mental por los altos niveles de estr\'{e}s generados.
\begin{itemize}
\item Elevado n\'{u}mero de estudiantes por curso sobre todo en los grados $6^{\circ}$, $7^{\circ}$ y en algunos cursos de básica primaria que sobrepasan ampliamente los par\'{a}metros legales vigentes (hasta 47 estudiantes por curso en secundaria, cuando lo m\'{a}ximo permitido son 40 estudiantes por sal\'{o}n)
\item Falta de unidad de criterios en cuanto a asignación de salones para las dos jornadas, lo que hace que en la mañana encontremos dificultades con el mobiliario. A diario encontramos pupitres, sillas y demás mobiliario deteriorado o simplemente no se encuentra o se encuentra reducido en su número. Nos es imposible hacernos responsables por los salones y sus elementos cuando en la tarde el uso de salones atiende a otra dinámica diferente.
\item Debido a que las aulas en secundaria no están diseñadas para más de 40 estudiantes y además no tienen el suficiente número de pupitres, vemos perjudicada nuestra labor diaria. Constantemente debemos dejar salir estudiantes del aula para que busquen silla y mesa y puedan recibir su clase en ''condiciones normales``.
\end{itemize}
En vista de las anteriores consideraciones, solicitamos: 
\begin{enumerate}
\item Se equilibre el número de estudiantes por curso en las dos jornadas y se respeten los parámetros de ley.
\item Dotar las aulas con el número suficiente de pupitres y sillas de acuerdo a los parámetros legales. Continuar con el proceso de dotación de las aulas especializadas.
\item Respetar lo aprobado en Consejo Directivo en cuanto a lo que tiene que ver con el sistema de rotación y manejo de aulas de clase y la dotación de las aulas especializadas.
\end{enumerate}

\closing{Atentamente,}

%cc{}
%\ps{PS: PostScriptum}
\encl{Listado de firmas}

\end{letter}
\end{document}
