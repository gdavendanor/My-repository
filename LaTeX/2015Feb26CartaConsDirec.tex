\documentclass[spanish,letterpaper,11pt]{letter}
\usepackage[T1]{fontenc}
\usepackage[utf8]{inputenc}
\usepackage[spanish]{babel}
\usepackage{lmodern}
\usepackage{amsmath}
\usepackage{amsfonts}
\usepackage{amssymb}
\address{Profesores j.m.\\Colegio Arborizadora Baja I.E.D.}
\signature{Profesores j.m.\\Colegio Arborizadora Baja I.E.D.}
\date{27 de febrero de 2015}

\begin{document}

\begin{letter}{CONSEJO DIRECTIVO\\Colegio Arborizadora Baja I.E.D.}
	
\opening{Cordial saludo}
En nuestra calidad de docentes de la jornada mañana, manifestamos las siguientes inconformidades que afectan nuestra labor con los estudiantes y nuestra salud física y mental por los altos niveles de estr\'{e}s generados.
\begin{itemize}
\item Elevado n\'{u}mero de estudiantes por curso sobre todo en los grados $6^{\circ}$, $7^{\circ}$ y en algunos cursos de básica primaria que sobrepasan ampliamente los par\'{a}metros legales vigentes (hasta 47 estudiantes por curso en secundaria, cuando lo m\'{a}ximo permitido son 40 estudiantes por sal\'{o}n)
\item Falta de unidad de criterios en cuanto a asignación de salones para las dos jornadas, lo que hace que en la mañana encontremos dificultades con el mobiliario. A diario encontramos pupitres, sillas y demás elementos deteriorados o simplemente no se encuentran o se encuentran reducidos en su número. Nos es imposible hacernos responsables por los salones y sus elementos cuando en la tarde el uso de salones atiende a otra dinámica diferente.
\item Debido a que las aulas en secundaria no están diseñadas para más de 40 estudiantes y además no tienen el suficiente número de pupitres, vemos perjudicada nuestra labor diaria. Constantemente debemos dejar salir estudiantes del aula para que busquen silla y mesa y puedan recibir su clase en ``condiciones normales''.
\item Los video-beams instalados en las aulas de clase no funcionan con excepción del que se encuentra en el salón D301.
\item Deterioro y falta de mantenimiento de las instalaciones eléctricas de la institución. En algunos espacios la iluminación es inadecuada. Así mismo, los baños de los estudiantes y de los maestros se encuentran en condiciones inapropiadas para su uso.
\item Negligencia en la compra aprobada del tablero inteligente. El dinero asignado viene con destinación específica y no aceptamos que no se compre éste y se intente reemplazar su compra por una pantalla que reposa en almacén.
\end{itemize}
En vista de las anteriores consideraciones, solicitamos: 
\begin{enumerate}
\item Se equilibre el número de estudiantes por curso en las dos jornadas y se respeten los parámetros de ley.
\item Dotar las aulas con el número suficiente de pupitres y sillas de acuerdo a los parámetros legales.
\item Respetar lo aprobado en Consejo Directivo en cuanto a lo que tiene que ver con el sistema de rotación y manejo de aulas de clase. Solicitamos igualmente se continúe con el proceso de  dotación de las aulas especializadas.
\item Designar los recursos necesarios para atender las necesidades de mantenimiento urgente que requiere la institución para su normal funcionamiento.
\item Ejecutar las compras aprobadas por el comité de compras a la menor brevedad posible.
\item Arreglo de los baños dañados para ponerlos a disposición de los estudiantes en los momentos necesarios y oportunos.
\end{enumerate}

\closing{Atentamente,}

%cc{}
%\ps{PS: PostScriptum}
\encl{Listado de firmas}

\end{letter}
\end{document}
