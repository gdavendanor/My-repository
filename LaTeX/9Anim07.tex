\documentclass[letterpaper,11pt,twoside]{article}
\usepackage[utf8]{inputenc}
\usepackage{amsmath,amsfonts,amssymb,amsthm,latexsym}
\usepackage[spanish,es-noshorthands]{babel}
\usepackage[T1]{fontenc}
\usepackage{lmodern}
\usepackage{graphicx,hyperref}
\usepackage{tikz,pgf}
\usepackage{marvosym}
\usepackage{multicol}
\usepackage{fancyhdr}
\usepackage[height=9.5in,width=7in]{geometry}
\usepackage{fancyhdr}
\pagestyle{fancy}
\fancyhead[LE]{\Email matematicas.german@gmail.com}
\fancyhead[RE]{\url{https://www.autistici.org/mathgerman}}
\fancyhead[RO]{\url{https://www.autistici.org/mathgerman}}
\fancyhead[LO]{\Email matematicas.german@gmail.com}

\author{Germ\'an Avenda\~no Ram\'irez~\thanks{Lic. Mat. U.D., M.Sc. U.N.}}
\title{\begin{minipage}{.2\textwidth}
\includegraphics[height=1.75cm]{Images/logo-colegio.png}\end{minipage}
\begin{minipage}{.55\textwidth}
\begin{center}
Animaplano 07\\
Matemáticas $9^{\circ}$
\end{center}
\end{minipage}\hfill
\begin{minipage}{.2\textwidth}
\includegraphics[height=1.75cm]{Images/logo-sed.png} 
\end{minipage}}
\date{}
\thispagestyle{plain}
\begin{document}
\maketitle
Nombre: \hrulefill Curso: \underline{\hspace*{44pt}} Fecha: \underline{\hspace*{2.5cm}}
\begin{multicols}{2}
\emph{Resuelva el siguiente cuestionario, mostrando los procedimientos necesarios para llegar a la respuesta. Respuesta que requiera procedimiento y éste no sea mostrado, no se tendrá en cuenta.}
\section*{Cuestionario}
\begin{enumerate}
\item Sume $3\frac{1}{5}+4\frac{4}{5}$
\item El octavo número primo
\item El décimo número primo
\item El volumen de un cono está dado por la expresión $V=\pi r^{2}h$ donde $r$ es el radio y $h$ es la altura. Si aproximamos $\pi \approx 3$, el volumen de un cono de radio 4cm y altura 3cm es? \label{q01}
\item El quíntuple del quinto número primo
\item El valor de $a^{2}$ con base en el triángulo
\tikz \draw (0,0) -- node[below]{$b=6$} (1.5,0)-- node[right]{$c=10$} (0,2)--node[left]{$a$} cycle;
\item Número de dos cifras donde la diferencia entre la cifra de las decenas y las unidades es 5, su suma 11 y su producto 24.
\item La diferencia entre $\sqrt{100}$ y $\sqrt{64}$
\item Resuelva $(\sqrt{100}\times \sqrt{81})+\sqrt{81}^{0}=$?
\item El quíntuplo del número 13 sumado con la tercera parte del número 45
\item El área de un rectángulo cuyo perímetro es 34 y cuyo largo mide 3 unidades más que su ancho.
\item El 20\% de la tercera parte de 915
\item Si $a=nb^{k}$, donde $n=3$, $b=4$ y $k=2$, entonces: $ak-7b+14=$?
\item Si $3n^{2}=75$, entonces $n!-8n-9=$? \label{q02}
\item Con base en \ref{q02} el valor de $n!-8n+1$ es?
\item Las edades de Juan Luis y su nieto suman 124 años. Si Juan Luis tiene 40 años más que su nieto, ¿cuál es la edad de Juan Luis?
\item ¿Cuál es la edad del nieto de Juan Luis?
\item El noveno número primo
\item Halle el valor de $c^{2}$ con base en el triángulo \tikz \draw (0,0)--node[below]{$4$}(2,0)--node[right]{$c$}(0,1.5)--node[left]{$3$} cycle;
\item Hay 185 visitantes, si 4/5 son mujeres ¿cuántos hombres hay?
\item Piense y escriba el valor que falta: 40(25)10, \quad 38(\hspace*{15pt})16
\item El área del triángulo OPQ disminuída en 10 unidades \tikz \draw (0,0)node[left]{O}--node[below]{9} (2.25,0)node[right]{P}--(0,2)node[above]{Q} --node[left]{8}cycle;
\item En la ecuación $2x-(10-x)=30-(10-x)$. ¿el valor de $x$ es?
\item El residuo de la división $103\div 7$ es?
\item Al simplificar $\sqrt{-49}+\sqrt{-16}+\sqrt{-25}$ el resultado es \underline{\hspace*{12pt}}$i$?
\item El número de minutos que hay en 420 segundos
\item El número de divisores del número 24
\item El 6 es un \emph{número perfecto} porque la suma de sus divisores propios es el mismo número $6=1+2+3$. El 1 es divisor propio de todo número, mientras que el mismo número no (en este caso 6 no es divisor propio de sí mismo). El menor \emph{número perfecto} de dos cifras es:
\begin{enumerate}
\begin{multicols}{4}
\item 36
\item 28
\item 12
\item 15
\end{multicols}
\end{enumerate}
\item El duodécimo número primo
\item El área de un triángulo rectángulo cuya base mide 11 y cuya altura mide 8
\item En el gráfico del triángulo rectángulo OPQ, $q=37$ grados, entonces $p=$?
\begin{center}
\begin{tikzpicture}
\draw (0,0)node[left]{O}--(2,0)node[right]{Q}--(0,1.5)node[above]{P}--cycle;
\draw (0,1.2) arc [start angle=270, end angle=323, radius=3mm];
\draw (1.7,0) arc [start angle=180,end angle=143,radius=3mm] node[left]{$37^{\circ}$};
\end{tikzpicture}
\end{center}
\item El 25\% de 292 es?
\item En una tabla de datos las $f_{i}$ son: (23, 25, 24, 20), luego $F_{4}$ es?\footnote{$f_{i}$ se lee como la frecuencia absoluta, mientras que $F_{i}$ es la frecuencia absoluta acumulada}
\item Mi estatura no alcanza a ser 1 m pero tres veces mi estatura suma más de 2 m. Cinco veces mi estatura es un número de metros completos. ¿Cuál es mi estatura en centímetros?
\end{enumerate}
\end{multicols}
\section*{Animaplano}
\begin{center}
\input{Tikz/Anim_0-99}
\end{center}
\end{document}
