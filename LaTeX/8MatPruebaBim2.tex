\documentclass[10pt,letterpaper,addpoints]{exam}
\usepackage[utf8]{inputenc}
\usepackage[spanish,es-noshorthands]{babel}
\usepackage{hyperref}
\usepackage{amsmath}
\usepackage{amsfonts}
\usepackage{amssymb}
\usepackage{graphicx}
\usepackage{tikz}
\usepackage{multicol}
\usepackage[width=7in,height=9.5in]{geometry}
%\printanswers
\begin{document}
\title{\begin{minipage}{.2\textwidth}
        \includegraphics[height=1.75cm]{Images/logo-colegio.png}
       \end{minipage}
\begin{minipage}{.55\textwidth}
 \begin{center}
Prueba bimestral \\Álgebra $8^{\circ}$
\end{center}
\end{minipage}
\begin{minipage}{.2\textwidth}
\includegraphics[height=1.75cm]{Images/logo-sed.png} 
\end{minipage}
}
\author{Germ\'{a}n Avendaño Ram\'{i}rez\\Lic. Matemáticas U.D., M.Sc. U.N.}
\date{}
\maketitle
\begin{center}
\fbox{\fbox{\parbox{5.5in}{\centering
Instrucciones}}}
\end{center}
\vspace{0.1in}
\makebox[\textwidth]{Nombres: \hrulefill, curso:\underline{\hspace{48pt}}, fecha:\underline{\hspace{3cm}}}
\begin{questions}
\question
\begin{minipage}{.65\textwidth}
Una máquina corta moldes de cartón que se doblan y se pegan para construir cajas, con las medidas que se muestran en el siguiente dibujo.
\end{minipage}\hfill
\begin{minipage}{.35\textwidth}
\begin{tikzpicture}
\draw (0,0) rectangle (4,1);
\draw (1,-2) rectangle (3,2);
\draw (1,-1)--(3,-1);
\draw[|-] (-.2,1)--node[below]{10cm}(-.2,.6);
\draw[|-](-.2,0)--(-.2,.4);
\draw[|-](0,-.2)--node[right]{10cm}(.1,-.2);
\draw[-|](.9,-.2)--(1,-.2);
\draw[|-](1,-2.2)--(1.5,-2.2)node[right]{20cm};
\draw[-|](2.5,-2.2)--(3,-2.2);
\draw[|-](3.2,-.1)--(3.2,-.4)node{10cm};
\draw[-|](3.2,-.6)--(3.2,-1);
\end{tikzpicture}
\end{minipage}
\begin{center}
\includegraphics[scale=.75]{Images/cajas.png} 
\end{center}
%Pregunta
%\begin{oneparchoices}
%\choice[1] Nunca
%\end{oneparchoices}
%\answerline
\end{questions}
%cuadro de puntajes
%\begin{center}
%\gradetable[h][pages]
%\end{center}
\end{document}