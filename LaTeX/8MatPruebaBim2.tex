\documentclass[10pt,letterpaper,addpoints]{exam}
\usepackage[utf8]{inputenc}
\usepackage[spanish,es-noshorthands]{babel}
\usepackage{hyperref}
\usepackage{amsmath}
\usepackage{amsfonts}
\usepackage{amssymb}
\usepackage{graphicx}
\usepackage{tikz}
\usepackage{multicol}
\usepackage[width=7in,height=9.5in]{geometry}
%\printanswers
\begin{document}
\title{\begin{minipage}{.2\textwidth}
        \includegraphics[height=1.75cm]{Images/logo-colegio.png}
       \end{minipage}
\begin{minipage}{.55\textwidth}
 \begin{center}
Prueba bimestral \\Álgebra $8^{\circ}$
\end{center}
\end{minipage}
\begin{minipage}{.2\textwidth}
\includegraphics[height=1.75cm]{Images/logo-sed.png} 
\end{minipage}
}
\author{Germ\'{a}n Avendaño Ram\'{i}rez\\Lic. Matemáticas U.D., M.Sc. U.N.}
\date{}
\maketitle
\begin{center}
\fbox{\fbox{\parbox{5.5in}{\centering
Responda en el cuadro de respuestas, rellenando el óvalo completamente. Haga sus procedimientos en una hoja aparte.}}}
\end{center}
\vspace{0.1in}
\makebox[\textwidth]{Nombres: \hrulefill, curso:\underline{\hspace{48pt}}, fecha:\underline{\hspace{3cm}}}
\begin{questions}
\begin{minipage}{.65\textwidth}
\question
Una máquina corta moldes de cartón que se doblan y se pegan para construir cajas, con las medidas que se muestran en el siguiente dibujo.\\

¿Cuál de las siguientes cajas se arma con el molde del dibujo?
\end{minipage}\hfill
\begin{minipage}{.35\textwidth}
\begin{tikzpicture}
\draw (0,0) rectangle (4,1);
\draw (1,-2) rectangle (3,2);
\draw (1,-1)--(3,-1);
\draw[|-] (-.2,1)--node[below]{10cm}(-.2,.6);
\draw[|-](-.2,0)--(-.2,.4);
\draw[|-](0,-.2)--node[right]{10cm}(.1,-.2);
\draw[-|](.9,-.2)--(1,-.2);
\draw[|-](1,-2.2)--(1.5,-2.2)node[right]{20cm};
\draw[-|](2.5,-2.2)--(3,-2.2);
\draw[|-](3.2,-.1)--(3.2,-.4)node{10cm};
\draw[-|](3.2,-.6)--(3.2,-1);
\end{tikzpicture}
\end{minipage}
\begin{center}
\includegraphics[scale=.75]{Images/cajas.png} 
\end{center}
\begin{minipage}{.35\textwidth}
\question
En un juego Juan lanzó tres dardos a un tablero como el siguiente:\\

El puntaje del juego se obtiene sumando los puntos asignados a la posición donde cae cada dardo.

Los tres dardos que lanzó Juan quedaron ubicados en los recuadros E5, F6 y D7.\\

¿Qué puntaje obtuvo Juan?
\end{minipage}\hfill
\begin{minipage}{.65\textwidth}
\begin{center}
\includegraphics[scale=.50]{Images/juego_dados.png} 
\end{center}
\end{minipage}

\begin{oneparchoices}
\choice 15 puntos.
\choice 18 puntos.
\choice 20 puntos.
\CorrectChoice 25 puntos.
\end{oneparchoices}

\begin{minipage}{.5\textwidth}
\question 
La siguiente tabla muestra los nombres de los atletas de un equipo y sus respectivos pesos.

El equipo realiza algunos ejercicios en parejas. La diferencia de pesos entre los atletas que conforman una pareja no debe sobrepasar los 3 kilogramos.\\

¿Cuáles de los siguientes atletas del equipo \textbf{no} pueden realizar los ejercicios en pareja?
\end{minipage}\hfill
\begin{minipage}{.5\textwidth}
\begin{tabular}{|c|c|}
\hline 
\textbf{Nombre del atleta} & \textbf{Peso en kilogramos} \\ 
\hline 
Oscar & 60 \\ 
\hline 
Andrés & 62.5 \\ 
\hline 
Víctor & 58.6 \\ 
\hline 
Fernando & 61.3 \\ 
\hline 
César & 65.2 \\ 
\hline 
Héctor & 59.4 \\ 
\hline 
\end{tabular} 
\end{minipage}

\begin{oneparchoices}
\choice Oscar y Víctor.
\choice Fernando y Héctor.
\CorrectChoice César y Víctor.
\choice Andrés y Fernando.
\end{oneparchoices}
\question El piso de la sala de una casa tiene una superficie de 13,6 $m^{2}$. Para cubrir el piso de la sala, se van a comprar baldosas que solamente son vendidas en cajas que contienen baldosas suficientes para cubrir 2 $m^2$ de superficie.

¿Cuál es el número mínimo de cajas que se debe comprar?

\begin{oneparchoices}
\choice 6
\CorrectChoice 7
\choice 13
\choice 14
\end{oneparchoices}
\question En un almacén deportivo quieren empacar balones de 10 centímetros de radio en cajas cúbicas. Disponen de los siguientes moldes para armar las cajas
\begin{center}
\includegraphics[scale=.75]{Images/moldes.png} 
\end{center}
¿Cuál es el molde más adecuado para construir estas cajas?

\begin{oneparchoices}
\choice El molde 1
\choice El molde 2
\CorrectChoice El molde 3
\choice El molde 4
\end{oneparchoices}
\question Cuatro atletas: Juan, Pedro, Carlos y Jorge entrenan para una competencia de atletismo, en una pista de 100 metros. Cada uno de ellos dio tres vueltas a la pista. A continuación
se relaciona el tiempo empleado por ellos en cada una de las vueltas.
\begin{center}
\begin{tabular}{|c|c|c|c|c|}
\hline 
 & \textbf{Tiempo} & \textbf{Tiempo} & \textbf{Tiempo} & \textbf{Tiempo} \\ 
\textbf{VUELTA} & \textbf{empleado por} & \textbf{empleado por} & \textbf{empleado por} & \textbf{empleado por} \\ 
 & \textbf{Juan (en segundos)} & \textbf{Pedro (en segundos)} & \textbf{Carlos (en segundos)} & \textbf{Jorge (en segundos)} \\ 
\hline 
Primera & 30 & 22 & 16 & 25 \\ 
\hline 
Segunda & 15 & 24 & 18 & 20 \\ 
\hline 
Tercera & 15 & 26 & 20 & 18 \\ 
\hline 
\end{tabular} 
\end{center}
¿Cuál de los atletas tuvo un menor tiempo por vuelta?
 
\begin{oneparchoices}
\choice Juan
\choice Pedro
\CorrectChoice Carlos
\choice Jorge
\end{oneparchoices}
\question Pablo tiene dos dados con forma de cubo, cada cara de los dados está marcada con un número distinto.

Las caras de uno de los dados están marcadas con los números 2, 4, 6, 8 ,10, 12, respectivamente.

Y las caras del otro dado, están marcadas con los números 1, 3, 5, 7, 9, 11, respectivamente.

Pablo lanza los dados, luego suma los números marcados en la cara superior de cada uno, y registra el resultado.
\\
¿Cuál de los siguientes resultados es \textbf{IMPOSIBLE} que obtenga Pablo?

\begin{oneparchoices}
\choice 11
\choice 13
\CorrectChoice 14
\choice 15
\end{oneparchoices}

\begin{minipage}{.5\textwidth}
\question Cuenta una leyenda que un rey pagó al inventor del ajedrez, un grano de maíz por el cuadrado número 1, el doble por el segundo, el doble del segundo por el tercer cuadrado y así sucesivamente. La siguiente ilustración muestra un tablero de ajedrez en el cual se han numerado algunos de sus cuadrados.\\

De acuerdo a la leyenda, ¿cuántos granos de maíz tuvo que pagar el rey, por el cuadrado número 15?
\end{minipage}\hfill
\begin{minipage}{.5\textwidth}
\begin{center}
\includegraphics[scale=.55]{Images/ajedrez.png} 
\end{center}
\end{minipage}

\begin{oneparchoices}
\CorrectChoice $2^{14}$
\choice $2^{16}$
\choice $15^{2}$
\choice $2\times 15$
\end{oneparchoices}
\question En una sala de cine se organiza una rifa entre los asistentes a una de las funciones. Cada asistente marca la boleta de la entrada con sus datos y la introduce en una urna, al final de la función se extrae una boleta al azar. De los asistentes, $\frac{1}{6}$ son hombres adultos, $\frac{1}{5}$ son mujeres adultas, $\frac{1}{3}$ son niños y  $\frac{3}{10}$     son niñas. Es \textbf{menos} probable que la rifa la gane

\begin{oneparchoices}
\choice una niña
\choice un niño
\choice una mujer adulta
\CorrectChoice un hombre adulto
\end{oneparchoices}
\question Una cuadra mide 100 metros aproximadamente. Un anuncio en una tienda dice: “Gran oferta a tan sólo 1.200 metros de aquí \ldots ”.\\ 

¿Cuántas cuadras en total tendrá que caminar una persona desde la tienda hasta el sitio donde se encuentra la gran oferta?

\begin{oneparchoices}
\choice 10
\CorrectChoice 12
\choice 100
\choice 120
\end{oneparchoices}
\question Diego intentó solucionar la ecuación $x + 3 = 5 - x$, pero en uno de los pasos cometió un error.

Observa su solución.
\begin{align*}
 x+x&=5-3 & \mbox{Paso 1}\\
 2x&=2 & \mbox{Paso 2}\\
  x&=2-2 & \mbox{Paso 3}\\
  x&=0 & \mbox{Paso 4}
\end{align*}
¿En cuál de los pasos cometió el \textbf{error}?

\begin{oneparchoices}
\choice En el paso 1
\choice En el paso 2
\CorrectChoice En el paso 3
\choice En el paso 4
\end{oneparchoices}
\question En un laboratorio está estudiándose una población de bacterias. En la siguiente tabla se muestra la cantidad que había inicialmente y la cantidad presente transcurrido(s) 1, 2 y 3 minutos.
\begin{center}
\begin{tabular}{|p{2cm}|c|c|c|c|c|}
\hline 
\textbf{Tiempo (minutos)} & \textbf{0} & \textbf{1} & \textbf{2} & \textbf{3} & \textbf{\ldots} \\ 
\hline 
\textbf{Número de bacterias} & 1\,000 & 3\,000 & 9\,000 & 27\,000 & \ldots \\ 
\hline 
\end{tabular} 
\end{center}
Si la regularidad que se muestra en la tabla se mantiene, ¿cuántas bacterias habrá en total a los 5 minutos?

\begin{oneparchoices}
\choice 135\,000
\choice 150\,000
\CorrectChoice 243\,000
\choice 300\,000
\end{oneparchoices}
\question Una empresa produce adornos navideños. Los adornos son empacados en cajas de tres tamaños:

En una caja grande caben 40 unidades.\\
En una caja mediana caben 30 unidades.\\
En una caja pequeña caben 20 unidades.  \\

La empresa ha recibido un pedido de 300 adornos. ¿Cuál o cuáles de los siguientes grupos de cajas puede emplear la empresa para empacar el pedido?
\begin{itemize}
\item[I.] 3 cajas grandes, 1 caja mediana, 5 cajas pequeñas.
\item[II.] 4 cajas grandes, 4 cajas medianas, 1 caja pequeña.
\item[III.] 5 cajas grandes, 2 cajas medianas, 2 cajas pequeñas.
\end{itemize}

\begin{oneparchoices}
\choice I solamente
\choice II solamente
\choice I y III solamente
\CorrectChoice II y III solamente
\end{oneparchoices}
\question Al solucionar la ecuación $x+5=12$ la solución obtenida es:

\begin{oneparchoices}
\choice $x=5$
\CorrectChoice $x=7$
\choice $x=4$
\choice $x=17$
\end{oneparchoices}
\question La solución de la ecuación $3x=-21$ es:

\begin{oneparchoices}
\choice $x=7$
\CorrectChoice $x=-7$
\choice $x=6$
\choice $x=-6$
\end{oneparchoices}
\question La solución de la ecuación $3x-6=-9$ es:

\begin{oneparchoices}
\choice $x=-2$
\choice $x=1$
\choice $x=-3$
\CorrectChoice $x=-1$
\end{oneparchoices}
\question El grado absoluto del polinomio $3x^{4}-2x^{3}+3x^{2}+5x-2$ es:

\begin{choices}
\choice 10 porque se suman los exponentes
\choice 9 porque se suman los exponentes
\CorrectChoice 4 porque es el mayor exponente
\choice 3 porque es el exponente del 2$^{\circ}$ término.
\end{choices}
\begin{minipage}{.5\textwidth}
\question Para realizar un experimento, se llenan con un líquido botellas de diferentes capacidades, como las que se muestran a continuación.\\

Posteriormente, para elaborar una mezcla, se debe pasar el líquido de algunas botellas al recipiente que aparece a continuación.
\end{minipage}
\begin{minipage}{.5\textwidth}
\begin{center}
\includegraphics[scale=.7]{Images/botellas.png} 
\end{center}
\end{minipage}

\begin{minipage}{.5\textwidth}
\begin{center}
\includegraphics[scale=.75]{Images/recipiente.png} 
\end{center}
\end{minipage}\hfill
\begin{minipage}{.5\textwidth}
El recipiente se llena exactamente con el líquido de las botellas
\begin{choices}
\choice 1 y 2
\choice 2 y 3
\CorrectChoice 1 y 4
\choice 2 y 4
\end{choices}
\end{minipage}
\question La medida m (en grados) de cualquier ángulo de polígono regular de $n$ lados puede determinarse usando la expresión $m=\dfrac{180^{\circ}(n-2)}{n}$.

¿Cuál es la medida $m$ de uno de los ángulos de un polígono regular de 15 lados?

\begin{oneparchoices}
\choice 150$^{\circ}$
\CorrectChoice 156$^{\circ}$
\choice 165$^{\circ}$
\choice 170$^{\circ}$
\end{oneparchoices}
\question Observa la siguiente secuencia de números:
\begin{center}
\begin{tabular}{|p{1.5cm}|p{1.5cm}|p{1.5cm}|p{1.5cm}|p{1.5cm}|p{1.5cm}|p{1.5cm}|}
\hline 
\textbf{Primer término} & \textbf{Segundo término} & \textbf{Tercer término} & \textbf{Cuarto término} & \textbf{Quinto término} & \textbf{Sexto término} & \textbf{Séptimo término} \\ 
\hline 
2 & 3 & 5 & 8 & 12 & ? & 23 \\ 
\hline 
\end{tabular} 
\end{center}
¿Cuál de los siguientes números debe sumarse a 12 para hallar el sexto término de la secuencia?

\begin{oneparchoices}
\choice 1
\choice 3
\CorrectChoice 5
\choice 7
\end{oneparchoices}
\end{questions}
%cuadro de puntajes
%\begin{center}
%\gradetable[h][pages]
%\end{center}
\end{document}