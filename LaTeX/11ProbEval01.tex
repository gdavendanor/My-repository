\documentclass[letterpaper,fleqn]{article}
\usepackage[spanish,es-noshorthands]{babel}
\usepackage[utf8]{inputenc} 
\usepackage[left=1cm, right=1cm, top=1.5cm, bottom=1.7cm]{geometry}
\usepackage{mathexam}
\usepackage{amsmath}
\usepackage{graphicx}
\usepackage{multicol}

\ExamClass{\includegraphics[height=16pt]{Images/logo-sed.png} Probabilidad $11^{\circ}$}
\ExamName{Introducción a la probabilidad}
\ExamHead{\includegraphics[height=16pt]{Images/logo-colegio.png} IEDAB}
\newcommand{\LineaNombre}{%
\par
\vspace{\baselineskip}
Nombre:\hrulefill \; Curso: \underline{\hspace*{48pt}} \; Fecha: \underline{\hspace*{2.5cm}} \relax
\par}
\let\ds\displaystyle

\begin{document}
\ExamInstrBox{
Respuesta sin justificar mediante procedimiento no será tenida en cuenta en la calificación. Escriba sus respuestas en el espacio indicado. Tiene 45 minutos para contestar esta prueba.}
\LineaNombre
\begin{enumerate}
 \item Se lanza un dado. Halla la probabilidad:
 \begin{enumerate}
 \item de salir el 3 \noanswer
 \item de salir un número par \noanswer
 \item de salir un número mayor que 2 \noanswer
 \end{enumerate}
 \item Calcula la probabilidad de que al lanzar dos monedas:
 \begin{enumerate}
 \item salgan dos caras \noanswer
 \item salgan cara y cruz \noanswer 
 \item salgan dos cruces \noanswer
 \end{enumerate}
 \item Calcula la probabilidad de que al lanzar dos dados la suma de sus puntos sea:
 \begin{enumerate}
 \item igual a 5 \noanswer 
 \item mayor que 10 \noanswer
 \end{enumerate}
 \item En una baraja de 40 cartas, se saca una. Halla la probabilidad de que sea:
 \begin{enumerate}
 \item sea el as de oros  \noanswer
 \item sea rey \noanswer 
 \item sea oros
 \item no sea oros \noanswer
 \end{enumerate}
 \item En una familia de tres hijos. Cuál es la probabilidad de que:
 \begin{enumerate}
 \item los tres sean chicos \noanswer 
 \item sean dos chicos y una chica \noanswer 
 \item alguno sea chico \noanswer
 \end{enumerate}
 \section*{Prueba Saber}
 \begin{minipage}{.35\textwidth}
\item Se lanzan 2 dados y se considera la suma de los puntajes obtenidos. La tabla muestra las parejas posibles para algunos puntajes.  \\

Si se lanzan dos veces los 2 dados, ¿cuántas posibilidades hay de obtener 10 puntos en total, de manera que en el primer lanzamiento se obtengan 6 puntos?
\begin{enumerate}
\begin{multicols}{2}
\item 8
\item 15
\item 16
\item 24
\end{multicols}
\end{enumerate}
 \end{minipage}
 \begin{minipage}{.6\textwidth}
\begin{center}
\begin{tabular}{|c|l|c|}
\hline 
\textbf{Puntaje} & \qquad \textbf{Parejas} & \textbf{\# de posibilidades}\\ \hline
2 & (1,1) & \hspace*{1cm}1 \\ 
\hline 
3 & (1,2), (2,1) & \hspace*{1cm}2 \\ 
\hline 
4 & (1,3), (2,2), (3,1) & \hspace*{1cm}3 \\ 
\hline 
5 & (1,4), (2,3), (3,2), (4,1) & \hspace*{1cm}4 \\ 
\hline 
6 & (1,5), (2,4), (3,3), (4,2), (5,1) & \hspace*{1cm}5 \\ 
\hline 
7 & (1,6), (2,5), (3,4), (4,3), (5,2), (6,1) & \hspace*{1cm}6 \\ 
\hline 
\end{tabular} 
\end{center} 
 \end{minipage}
 \end{enumerate}

\end{document}
