\documentclass[letterpaper,fleqn]{article}
\usepackage[spanish,es-noshorthands]{babel}
\usepackage[utf8]{inputenc} 
\usepackage[left=1cm, right=1cm, top=1.5cm, bottom=1.7cm]{geometry}
\usepackage{mathexam}
\usepackage{amsmath}
\usepackage{graphicx}
\usepackage{multicol}

\ExamClass{\includegraphics[height=16pt]{Images/logo-sed.png} Matemáticas $9^{\circ}$}
\ExamName{Nivelación 2015}
\ExamHead{\includegraphics[height=16pt]{Images/logo-colegio.png} IEDAB}
\newcommand{\LineaNombre}{%
\par
\vspace{\baselineskip}
Nombre:\hrulefill \; Curso: \underline{\hspace*{48pt}} \; Fecha: \underline{\hspace*{2.5cm}} \relax
\par}
\let\ds\displaystyle

\begin{document}
\ExamInstrBox{
Respuesta sin justificar mediante procedimiento no será tenida en cuenta en la calificación. Escriba sus respuestas en el espacio indicado. Tiene 55 minutos para contestar esta prueba.}
\LineaNombre
\begin{enumerate}
\item Use la notación científica para calcular más fácilmente las siguiente operaciones:
\begin{enumerate}
\begin{multicols}{2}
\item $(120\,000)(300\,000)=$
\item $\dfrac{0.000045}{0.0003}=$
\end{multicols}
\end{enumerate}\noanswer[40pt]
 \item Evalué el valor numérico de las expresiones
 \begin{enumerate}\begin{multicols}{2}
 \item $\left(\dfrac{2}{3}\right)^{-2}=$
 \item $(4^{-2}\cdot 4^{2})^{-1}$\
 \end{multicols}
 \end{enumerate}\noanswer[40pt]
\item Realice las operaciones indicadas y exprese la respuesta en la forma estándar de un número complejo
\begin{enumerate}
\begin{multicols}{2}
\item $(4-10i)-(7-9i)=$
\item $(-4+i)-(2+3i)=$
\end{multicols}
 \end{enumerate}\noanswer[100pt]
 \item Realice las operaciones indicadas y simplifique
 \begin{enumerate}
 \begin{multicols}{2}
 \item $\sqrt{-2}\sqrt{18}=$ 
 \item $\dfrac{\sqrt{-6}}{\sqrt{2}}=$\noanswer[60pt]
 \item $(5-7i)(6+8i)=$ \noanswer
 \item $\dfrac{-1-i}{-2+5i}=$ \noanswer
 \end{multicols}
 \end{enumerate}
 \newpage
 \item Resuelva las ecuaciones cuadráticas usando la factorización
\begin{enumerate}
\begin{multicols}{2}
\item $x^{2}=6x$
\item $x^{2}-3x-40=0$
\end{multicols}
\end{enumerate} \noanswer
\item Use la fórmula cuadrática $x=\dfrac{-b\pm \sqrt{b^{2}-4ac}}{2a}$ que expresa las soluciones de la ecuación general $ax^{2}+bx+c=0$ para solucionar la ecuación
\[5x^{2}-x-3=0\]\noanswer
\item El perímetro de un rectángulo es 56 cm. El largo del rectángulo es tres veces su ancho. Encuentre las dimensiones del rectángulo\noanswer
 \end{enumerate}

\end{document}
