\documentclass[fleqn]{article}
\usepackage[spanish,es-noshorthands]{babel}
\usepackage[utf8]{inputenc} 
\usepackage[papersize={5.5in,8.5in},left=1cm, right=1cm, top=1.5cm, bottom=1.7cm]{geometry}
\usepackage{mathexam}
\usepackage{amsmath}
\usepackage{graphicx}
\usepackage{multicol}
\usepackage{textcomp}

\ExamClass{\includegraphics[height=16pt]{Images/logo-sed.png} Aritmética $7^{\circ}$}
\ExamName{Números enteros}
\ExamHead{\includegraphics[height=16pt]{Images/logo-colegio.png} IEDAB}
\newcommand{\LineaNombre}{%
\par
\vspace{\baselineskip}
Nombre:\hrulefill \; Curso: \underline{\hspace*{48pt}} \; Fecha: \underline{\hspace*{2.5cm}} \relax
\par}
\let\ds\displaystyle

\begin{document}
\ExamInstrBox{
Conteste con el procedimiento adecuado en el espacio indicado. Si es necesario anexe una hoja de operaciones}
\LineaNombre
\begin{enumerate}
\item ¿Qué temperatura está marcando un termómetro si:
  \begin{enumerate}
  \item Marcaba 18\textcelsius\, y disminuyó 12\textcelsius ? \dotfill
  \item Marcaba 12\textcelsius\, bajo cero y aumentó 5\textcelsius? \dotfill
  \item Marcaba 25\textcelsius\, y aumentó 4\textcelsius? \dotfill
  \item Marcaba 8\textcelsius \, bajo cero y disminuyó 6\textcelsius? \dotfill
    \end{enumerate}
\item Escribe una situación que pueda representar cada número
\begin{enumerate}
  \item -15 m \dotfill
  \item 12\textcelsius \, \dotfill
  \item 30 k/h \dotfill
  \item \$ -5860 \dotfill
  \end{enumerate}
\item Estima la temperatura de los elementos de la derecha y luego aparéalos con la temperatura aproximada, de la columna de la izquierda\\
\begin{center}
\begin{tabular}{p{3cm}r}
		1200\textcelsius  & Temperatura del cuerpo humano \\
		-30\textcelsius  & temperatura de la superficie del sol \\
		3\textcelsius  & temperatura de un congelador \\
		5750\textcelsius & temperatura en Siberia\\
		37\textcelsius & temperatura de una nevera\\
		-50\textcelsius & temperatura de la lava de un volcán\\
	\end{tabular}
\end{center}
\item Analiza cuáles afirmaciones son verdaderas y cuáles son falsas. Explica cada caso sobre una recta numérica:
\begin{enumerate}
  \item 6 está a la derecha de -4 \noanswer
  \item -4 está a la izquierda de 3 \noanswer
  \item -8 está a la derecha de -6 \noanswer
  \item Entre 5 y 3 hay 2 unidades de distancia.\noanswer
  \newpage
  \item La distancia entre -2 y 2 es de 2 unidades. \noanswer[.25in]
  \item De cero a -5 la distancia es de 5 unidades.\noanswer[.25in]
  \item Entre -3 y 8 la distancia es de 5 unidades.\noanswer[.25in]
\end{enumerate}
  \item ¿Qué año se encuentra después del 600 a. de C.? \dotfill
  \item ¿Qué año está antes de 1356 a. de C.?\dotfill
  \item ¿Qué puedes concluir al respecto del orden en los números negativos $Z^-$?\noanswer
\item Un gusano sube por una pared lisa. Si por cada 4 cm que avanza se desliza 2 cm, ¿al cabo de cuántos intentos logra trepar 7 cm? \noanswer
\item Buscando una dirección, Luis caminó inicialmente 8 cuadras, pero como no la encontró retrocedió 4 cuadras y luego retrocedió 6 cuadras más, ¿a cuántas cuadras quedó de donde inició su búsqueda?\noanswer
\item A las 6:00 a.m. el termómetro marca -6°C. A las 10:00 a.m. la temperatura es 15°C más alta y después de esta hora hasta las 9:00 p.m. bajó 12°C. Expresa la temperatura a las 9:00 p.m. \noanswer
\end{enumerate}
%%%%%%%%%%%%%%%%%%%%%%%%%% Para incluir gráficos %%%%%%%%%%%%%%%%%%%%%%%%%%%%%%%%%%%5
%\includegraphics[scale=0.40]{image}
%	\captionof{figure}{figure-name$}
%	\label{fig: }
%%%%%%%%%%%%%%%%%%%%%%%%%% Para insertar tablas %%%%%%%%%%%%%%%%%%%%%%%%%%%%%%%%%%%%%
%\begin{tabular}{lll}
%		11 & 12 & 13 \\
%		21 & 22 & 23 \\
%		31 & 32 & 33 \\
%	\end{tabular}
%	\captionof{table}{name-table}
%	\label{tab:}
%%%%%%%%%%%%%%%%%%%%%%%%%%%%%%%%%%%%%5
\end{document}
