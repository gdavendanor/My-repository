\documentclass[11pt,twoside]{article}
\usepackage[utf8]{inputenc}
\usepackage{amsmath}
\usepackage{amsfonts}
\usepackage{amssymb}
\usepackage[spanish,es-noshorthands]{babel}
\usepackage[T1]{fontenc}
\usepackage{lmodern}
\usepackage{graphicx,hyperref}
\usepackage{tikz,pgf}
\usepackage{multicol}
\usepackage{subfig}
\usepackage[papersize={8.5in,11in},includeheadfoot,left=0.4in,right=0.3in,top=0.3in,bottom=0.2in]{geometry}
\usepackage{fancyhdr}
\pagestyle{fancy}
\fancyhead[LE]{\url{http://gerdar.byethost9.com}}
\fancyhead[RE]{}
\fancyhead[RO]{\textit{Germ\'an Dar\'io Avenda\~no Ram\'irez, Lic - M.Sc.}}
\fancyhead[LO]{}

\author{Germ\'an Dar\'io Avenda\~no Ram\'irez, Lic. - M.Sc.}
\title{\begin{minipage}{0.15\textwidth}\includegraphics[height=1.7cm]{Images/logo-colegio.png}
\end{minipage}\hfill \begin{minipage}{0.85\textwidth}\begin{center}
Animaplano 1\\Matemáticas $11^{\circ}$\end{center}
\end{minipage}}
\date{}

\begin{document}
\maketitle
Nombre: \hrulefill Curso: \underline{\hspace{1cm}}  Fecha: \underline{\hspace{2cm}}\\

\emph{La solución a esta actividad debe ser entregada al finalizar la clase}
\section{Animaplano}
Un animaplano consta de un cuestionario cuyas respuestas son números entre 1 y 100. Las respuestas se irán ubicando en un plano con 100 puntos. Posteriormente esto puntos se unen en el mismo orden del cuestionario, lo cual generará una figura interesante.

Responda con procedimientos al frente de cada pregunta.
\subsection{Cuestionario}
 \begin{enumerate}
  \item $(2\times3!)-1!=$
  \item $\sqrt{9}-\sqrt{9}+9^{0}=$
  \item La raíz cúbica del número 8
  \item 1/4 del número 52
  \item $(-1)(-17)(-2)(-1)=$
  \item En grados, la mitad de la mitad de un ángulo llano
  \item $2(m+3)=98$, entonces $m=$
  \item La suma de dos números es 40, si el doble del menor es 28, ¿el número mayor es?
  \item En decámetros, 1 Hectómetro
  \item En años, 7 lustros + 2 años
  \item Si $m+n=51$, halle la suma de $(m+10)+(n-4)=$
  \item Si $x=4$, calculo el resultado de la expresión $x^{3}+x^{2}-\frac{x}{2}=$
  \item Si el perímetro de un cuadrado es $40\,u$ ¿su área en $u^{2}$ (unidades cuadradas) es?
  \item Halle $(3^{2})^{2}+18$
  \item Si dos ángulos internos de un triángulo isósceles suman $103^{\circ}$, ¿el tercer ángulo mide?
  \item Los años que hay en 12 lustros, más 1/2 década
  \newpage
  \item Resuelva $2^{3}\times2^{3}=$
  \item $\sqrt{10000}-3^{2}=$
  \item Si el dividendo es 159 y el cociente es 3, ¿el divisor es?
  \item En años, 1 siglo -- 9 lustros
  \item El cuádruplo del número 11
  \item Resuelva $-(-23)=$
  \item Si las diagonales de un rombo miden 6 y 4 cm, ¿el área en $cm^{2}$ del rombo es? (Recuerde que: $ A=\dfrac{D\cdot d}{2} $, donde $ D $ es la diagonal mayor y $ d $ la diagonal menor).
  \item 1/4 de centena menos 3
  \item $[(-11)+(-11)+(-11)]\times(-1)=$
  \item Resuelva $8\times8/2$
  \item Raíz cuadrada de 121
  \item El perímetro de un hexágono regular de lado $l=\sqrt{4}\,u$ es?
 \end{enumerate}
\subsection{Plano}
\begin{center}
\begin{tikzpicture}
 \fill (1,0) node[above]{1} circle (0.2ex);
 \fill (2,0) node[above]{2} circle (0.2ex);
 \fill (3,0) node[above]{3} circle (0.2ex);
 \fill (4,0) node[above]{4} circle (0.2ex);
 \fill (5,0) node[above]{5} circle (0.2ex);
 \fill (6,0) node[above]{6} circle (0.2ex);
 \fill (7,0) node[above]{7} circle (0.2ex);
 \fill (8,0) node[above]{8} circle (0.2ex);
 \fill (9,0) node[above]{9} circle (0.2ex);
 \fill (10,0) node[above]{10} circle (0.2ex);
 \fill (1,-1) node[left]{11} circle (0.2ex);
 \fill (2,-1) circle (0.2ex);
 \fill (3,-1) circle (0.2ex);
 \fill (4,-1) circle (0.2ex);
 \fill (5,-1) circle (0.2ex);
 \fill (6,-1) circle (0.2ex);
 \fill (7,-1) circle (0.2ex);
 \fill (8,-1) circle (0.2ex);
 \fill (9,-1) circle (0.2ex);
 \fill (10,-1) circle (0.2ex);
 \fill (1,-2) node[left]{21} circle (0.2ex);
 \fill (2,-2) circle (0.2ex);
 \fill (3,-2) circle (0.2ex);
 \fill (4,-2) circle (0.2ex);
 \fill (5,-2) circle (0.2ex);
 \fill (6,-2) circle (0.2ex);
 \fill (7,-2) circle (0.2ex);
 \fill (8,-2) circle (0.2ex);
 \fill (9,-2) circle (0.2ex);
 \fill (10,-2) circle (0.2ex);
 \fill (1,-3) node[left]{31} circle (0.2ex);
 \fill (2,-3) circle (0.2ex);
 \fill (3,-3) circle (0.2ex);
 \fill (4,-3) circle (0.2ex);
 \fill (5,-3) circle (0.2ex);
 \fill (6,-3) circle (0.2ex);
 \fill (7,-3) circle (0.2ex);
 \fill (8,-3) circle (0.2ex);
 \fill (9,-3) circle (0.2ex);
 \fill (10,-3) circle (0.2ex);
 \fill (1,-4) node[left]{41} circle (0.2ex);
 \fill (2,-4) circle (0.2ex);
 \fill (3,-4) circle (0.2ex);
 \fill (4,-4) circle (0.2ex);
 \fill (5,-4) circle (0.2ex);
 \fill (6,-4) circle (0.2ex);
 \fill (7,-4) circle (0.2ex);
 \fill (8,-4) circle (0.2ex);
 \fill (9,-4) circle (0.2ex);
 \fill (10,-4) node[right]{50} circle (0.2ex);
 \fill (1,-5) node[left]{51} circle (0.2ex);
 \fill (2,-5) circle (0.2ex);
 \fill (3,-5) circle (0.2ex);
 \fill (4,-5) circle (0.2ex);
 \fill (5,-5) circle (0.2ex);
 \fill (6,-5) circle (0.2ex);
 \fill (7,-5) circle (0.2ex);
 \fill (8,-5) circle (0.2ex);
 \fill (9,-5) circle (0.2ex);
 \fill (10,-5) circle (0.2ex);
 \fill (1,-6) node[left]{61} circle (0.2ex);
 \fill (2,-6) circle (0.2ex);
 \fill (3,-6) circle (0.2ex);
 \fill (4,-6) circle (0.2ex);
 \fill (5,-6) circle (0.2ex);
 \fill (6,-6) circle (0.2ex);
 \fill (7,-6) circle (0.2ex);
 \fill (8,-6) circle (0.2ex);
 \fill (9,-6) circle (0.2ex);
 \fill (10,-6) circle (0.2ex);
 \fill (1,-7) node[left]{71} circle (0.2ex);
 \fill (2,-7) circle (0.2ex);
 \fill (3,-7) circle (0.2ex);
 \fill (4,-7) circle (0.2ex);
 \fill (5,-7) circle (0.2ex);
 \fill (6,-7) circle (0.2ex);
 \fill (7,-7) circle (0.2ex);
 \fill (8,-7) circle (0.2ex);
 \fill (9,-7) circle (0.2ex);
 \fill (10,-7) circle (0.2ex);
 \fill (1,-8) node[left]{81} circle (0.2ex);
 \fill (2,-8) circle (0.2ex);
 \fill (3,-8) circle (0.2ex);
 \fill (4,-8) circle (0.2ex);
 \fill (5,-8) circle (0.2ex);
 \fill (6,-8) circle (0.2ex);
 \fill (7,-8) circle (0.2ex);
 \fill (8,-8) circle (0.2ex);
 \fill (9,-8) circle (0.2ex);
 \fill (10,-8) circle (0.2ex);
 \fill (1,-9) node[left]{91} circle (0.2ex);
 \fill (2,-9) circle (0.2ex);
 \fill (3,-9) circle (0.2ex);
 \fill (4,-9) circle (0.2ex);
 \fill (5,-9) circle (0.2ex);
 \fill (6,-9) circle (0.2ex);
 \fill (7,-9) circle (0.2ex);
 \fill (8,-9) circle (0.2ex);
 \fill (9,-9) circle (0.2ex);
 \fill (10,-9) node[right]{100} circle (0.2ex);
 \draw (1,-1)--(1,0)--(2,0)--(3,-1)--(4,-3);
\end{tikzpicture}
\end{center}
\end{document}
