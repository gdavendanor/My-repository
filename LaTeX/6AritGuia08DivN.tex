\documentclass[letterpaper,11pt,twoside]{article}
\usepackage[utf8]{inputenc}
\usepackage{amsmath,amsfonts,amssymb,amsthm,latexsym}
\usepackage[spanish,es-noshorthands]{babel}
\usepackage[T1]{fontenc}
\usepackage{lmodern}
\usepackage{graphicx,hyperref}
\usepackage{tikz,pgf}
\usepackage{multicol}
\usepackage{fancyhdr}
\usepackage[height=9.5in,width=7in]{geometry}
\usepackage{fancyhdr}
\pagestyle{fancy}
\fancyhead[LE]{\includegraphics[height=12pt]{Images/logo-colegio.png} Aritm\'{e}tica $6^{\circ}$}
\fancyhead[RE]{}
\fancyhead[RO]{\textit{Germ\'an Avenda\~no Ram\'irez, Lic. U.D., M.Sc. U.N.}}
\fancyhead[LO]{}

\author{Germ\'an Avenda\~no Ram\'irez, Lic. U.D., M.Sc. U.N.}
\title{\begin{minipage}{.2\textwidth}
\includegraphics[height=1.75cm]{Images/logo-colegio.png}\end{minipage}
\begin{minipage}{.55\textwidth}
\begin{center}
Taller 08, Divisi\'{o}n en $\mathbb{N}$\\
Aritmética $6^{\circ}$
\end{center}
\end{minipage}\hfill
\begin{minipage}{.2\textwidth}
\includegraphics[height=1.75cm]{Images/logo-sed.png} 
\end{minipage}}
\date{}
\thispagestyle{plain}
\begin{document}
\maketitle
Nombre: \hrulefill Curso: \underline{\hspace*{44pt}} Fecha: \underline{\hspace*{2.5cm}}
\begin{multicols}{2}
\section*{Lo que s\'{e}}
Realice la siguiente actividad en el cuaderno:

Manuel compró un terreno, con las dimensiones que se observan en el plano, por un precio de \$ 18'750\,000.
\begin{center}
\includegraphics[scale=.5]{Images/terreno.png} 
\end{center}
\begin{itemize}
\item ¿Cuál es el área del terreno?
\item ¿Cuál es el valor de cada metro cuadrado del terreno?
\item ¿Qué operación deben efectuar para resolver la situación propuesta?
\end{itemize}
Para resolver la situación, primero se halla el área del terreno: 300 m$^{2}$. ¿Por qué?

Luego, se puede realizar la siguiente \emph{división de números naturales:}
\begin{center}
\includegraphics[scale=.5]{Images/division.png} 
\end{center}
\begin{center}
\includegraphics[scale=.5]{Images/divison02.png} 
\end{center}
Por lo tanto, el precio de cada metro cuadrado del terreno es de \$ 62\,500.

Cada uno de los términos de la división recibe un nombre particular.

Observen.
\begin{center}
\includegraphics[scale=.75]{Images/dividendo.png} 
\end{center}
\begin{itemize}
\item Copien y completen las siguientes frases en el cuaderno.\\
El \emph{dividendo} es el número que \hrulefill\\
El \emph{divisor} es el número que \hrulefill\\
El \emph{cociente} es el \hrulefill\\
El \emph{residuo} es el \hrulefill
\end{itemize}
Manuel quiere repartir el terreno comprado entre sus siete hijos.
\begin{itemize}
\item ¿Es posible dividir el terreno en siete partes iguales, sin que sobren metros cuadrados?
\item ¿Cuántos metros cuadrados le corresponden a cada uno?
\item Copien y efectúen la siguiente división.
\[300\div7\]
\item ¿Cuál es el residuo de la división?
\item ¿Cuándo una división es exacta?
\item ¿Cuándo una división es inexacta?

Una división es \emph{exacta} cuando su residuo es cero. Y es \emph{inexacta} cuando el residuo es diferente de cero.
\item Realicen cada división e indiquen si es exacta o
inexacta.
\begin{tabbing}
\hspace{2cm} \= \hspace{2cm} \= \hspace{2cm} \= \kill
$45\div 5$ \> $83\div 9$ \> $108\div 12$ \> $96\div 15$ 
\end{tabbing}
\item Ubiquen los términos de una de las divisiones que resultaron inexactas, donde corresponda en la siguiente igualdad.
\end{itemize}
\begin{tabbing}
DIVIDENDO \= = \= (DIVISOR $\times$ COCIENTE) \= + \= RESIDUO\\
\tikz \draw (0,0) rectangle (2.2,.5); \> = \> \tikz \draw (0,0) rectangle (4.5,.5);  \> + \> \tikz \draw (0,0)rectangle(1.8,.5); 
\end{tabbing} 
¿Se cumple la igualdad?

\emph{En toda división de números naturales se cumple
la siguiente igualdad.}
\[DIVIDENDO=(DIVISOR\times COCIENTE)+RESIDUO\]
Recuerden que la multiplicación de números naturales
cumple algunas propiedades. ¿La división cumplirá propiedades similares?. Averígüenlo desarrollando las siguientes actividades.

Copien y efectúen las siguientes divisiones:
\[65\div 13 \qquad 65\div 9\]
\begin{itemize}
\item ¿Cuál es el resultado de la primera división? ¿Y de la segunda?
\item ¿La división de dos números naturales es siempre un número natural? Expliquen la respuesta.

Ahora realicen estas divisiones en el cuaderno.
\[72\div 6 \qquad 6\div 72\]
\item ¿Qué resultado obtuvieron en la primera división?
\item ¿Cuál es el resultado de la segunda división?
\item ¿Es posible intercambiar el orden del dividendo y el divisor sin que el altere el cociente? Analicen otros ejemplos.

Calculen el resultado de las siguientes operaciones. Recuerden que primero se realizan las operaciones indicadas entre paréntesis.
\[(45\div 3)\div 5\qquad 45\div (3\div 5) \]
\item ¿Obtuvieron los mismos resultados?
\item ¿Es posible asociar los términos de una división sin que se altere el cociente?
\item ¿Es lo mismo $14 \div 1$ que $1 \div 14$? Expliquen su respuesta.
\item ¿Cuáles son los resultados de las siguientes
operaciones? Completen el siguiente proceso en el
cuaderno, para determinar la respuesta.
\begin{center}
\includegraphics[scale=.55]{Images/divisiones.png} 
\end{center}
\item ¿Dividir un número entre la suma de otros dos es igual a la suma de los cocientes que se obtienen al dividir el número entre cada sumando?
\end{itemize}
\section*{Ejercito lo aprendido}
Resuelve cada situación.

Calcula la medida del lado desconocido en cada terreno de forma rectangular.
\begin{center}
\includegraphics[scale=.4]{Images/trapecios.png} 
\end{center}
\begin{itemize}
\item El terreno comprado por Manuel es de forma rectangular con medidas 20 m de largo por 15 m de ancho.
\item ¿Cuántos postes debe comprar para cercarlo, si va a colocarlos cada dos metros alrededor del terreno?
\item Manuel quiere repartir el terreno en cinco partes iguales para regalarle a sus hijos. ¿Cuántos metros cuadrados le corresponden a cada uno?
\item Para obtener ganancias, los hijos de Manuel deciden vender el terreno total por \$21'000\,000. ¿Cuánto le ganaron a cada metro cuadrado?
\end{itemize}
\end{multicols}


\end{document}
