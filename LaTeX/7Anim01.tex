\documentclass[10pt]{article}
\usepackage[utf8]{inputenc}
\usepackage{amsmath,amsfonts,amssymb,amsthm,latexsym}
\usepackage[spanish]{babel}
\usepackage[T1]{fontenc}
\usepackage{lmodern}
\usepackage{graphicx,hyperref}
\usepackage{multicol}
\usepackage{tikz,pgf}
\usepackage{marvosym}
\usepackage[papersize={5.5in,8.5in},left=.75cm,right=.75cm,top=1.5cm,bottom=1.25cm]{geometry}
\usepackage{fancyhdr}
\pagestyle{fancy}
\fancyhead[LE]{\Email iedabgerman@autistici.org}
\fancyhead[RE]{}
\fancyhead[RO]{\url{https://www.autistici.org/mathgerman}}
\fancyhead[LO]{}

\author{Prof: Silvia Malaver~\thanks{Por Germán Avendaño Ramírez, Lic. Mat. U.D., M.Sc. U.N.}}
\title{\begin{minipage}{.2\textwidth}
\includegraphics[height=1.75cm]{Images/logo-colegio.png}\end{minipage}
\begin{minipage}{.55\textwidth}
\begin{center}
Animaplano 01\\
Matemáticas $7^{\circ}$
\end{center}
\end{minipage}\hfill
\begin{minipage}{.2\textwidth}
\includegraphics[height=1.75cm]{Images/logo-sed.png} 
\end{minipage}}
\date{}
\thispagestyle{plain}
\begin{document}
\maketitle
Nombre: \hrulefill Curso: \underline{\hspace{1cm}}  Fecha: \underline{\hspace{2cm}}\\

Un animaplano consiste de un cuestionario y un plano en el que se ubican las respuestas a las preguntas en su orden.
\section*{Cuestionario}
\begin{enumerate}
 \item En grados, la mitad de un \'angulo recto\begin{multicols}{2}
\item Resuelva $2^4=$
\item Reste $\sqrt{9}$, a la mitad de 30 =
\item Halle $\sqrt{25}\times \sqrt{25}=$
\item El triple del n\'umero 15=
\item Halle $4!$, (Nota: $4!=4\cdot 3\cdot 2\cdot 1$)
\item Reste $3^2$, al cu\'adruple de 10 =
\item La mitad de 68 =
\item Represente XLV =
\item Resuelva $50-(6\times 3)=$
\item Sume $15^0$, al triple de 20 =
\item 1 hora, menos 18 minutos =
\item $11+11+11+11=$
\item Reste $2^3$, al cu\'adruple de 15 =
\item Halle $9^2-\sqrt{100}=$
\item Reste $3^2$, a $\sqrt{10000}=$
\item $200-107$
\item 3 docenas + 3 decenas = \hspace{1cm}u.
\item $100-(7\times2)=$
\item Reste 6, al doble de 50 =
\item Sume $(9\times9)+(18\div3)=$
\item 1/2 centena + 6 unidades = \hspace{1cm}u.
\item Represente LXVII =
\item Halle $10^2-\sqrt{9}=$
\item Resuelva $8^2+\sqrt{16}=$
\item Sume $40+(2^2)^2=$
\item El doble de 29 =
\item $(9\times9)+2^3=$
\item Resuelva $7^2\times12^0=$
\item 3 docenas + 2 decenas =
\item El triple de $4^2=$
\item Halle $7\times2^2=$
\item Si $2m+8=60$, entonces $m=$
\item El 50\% de 90 = \end{multicols}
\end{enumerate}
As\'i que empezamos con la pregunta n\'umero uno y su respuesta ser\'a un n\'umero entre 1 y 100. Luego respondemos la pregunta n\'umero dos y su respuesta (entre 1-100) se debe conectar por medio de un segmento de recta con la respuesta de la pregunta n\'umero uno.\\

As\'i al final lograremos un gr\'afico que puede llegar a ser curiosos; un animal, un objeto, etc.
\section*{Animaplano}
\begin{center}
\begin{tikzpicture}[scale=.8]
 \fill (1,0) node[above]{1} circle (0.2ex);
 \fill (2,0) node[above]{2} circle (0.2ex);
 \fill (3,0) node[above]{3} circle (0.2ex);
 \fill (4,0) node[above]{4} circle (0.2ex);
 \fill (5,0) node[above]{5} circle (0.2ex);
 \fill (6,0) node[above]{6} circle (0.2ex);
 \fill (7,0) node[above]{7} circle (0.2ex);
 \fill (8,0) node[above]{8} circle (0.2ex);
 \fill (9,0) node[above]{9} circle (0.2ex);
 \fill (10,0) node[above]{10} circle (0.2ex);
 \fill (1,-1) node[left]{11} circle (0.2ex);
 \fill (2,-1) circle (0.2ex);
 \fill (3,-1) circle (0.2ex);
 \fill (4,-1) circle (0.2ex);
 \fill (5,-1) circle (0.2ex);
 \fill (6,-1) circle (0.2ex);
 \fill (7,-1) circle (0.2ex);
 \fill (8,-1) circle (0.2ex);
 \fill (9,-1) circle (0.2ex);
 \fill (10,-1) circle (0.2ex);
 \fill (1,-2) node[left]{21} circle (0.2ex);
 \fill (2,-2) circle (0.2ex);
 \fill (3,-2) circle (0.2ex);
 \fill (4,-2) circle (0.2ex);
 \fill (5,-2) circle (0.2ex);
 \fill (6,-2) circle (0.2ex);
 \fill (7,-2) circle (0.2ex);
 \fill (8,-2) circle (0.2ex);
 \fill (9,-2) circle (0.2ex);
 \fill (10,-2) circle (0.2ex);
 \fill (1,-3) node[left]{31} circle (0.2ex);
 \fill (2,-3) circle (0.2ex);
 \fill (3,-3) circle (0.2ex);
 \fill (4,-3) circle (0.2ex);
 \fill (5,-3) circle (0.2ex);
 \fill (6,-3) circle (0.2ex);
 \fill (7,-3) circle (0.2ex);
 \fill (8,-3) circle (0.2ex);
 \fill (9,-3) circle (0.2ex);
 \fill (10,-3) circle (0.2ex);
 \fill (1,-4) node[left]{41} circle (0.2ex);
 \fill (2,-4) circle (0.2ex);
 \fill (3,-4) circle (0.2ex);
 \fill (4,-4) circle (0.2ex);
 \fill (5,-4) circle (0.2ex);
 \fill (6,-4) circle (0.2ex);
 \fill (7,-4) circle (0.2ex);
 \fill (8,-4) circle (0.2ex);
 \fill (9,-4) circle (0.2ex);
 \fill (10,-4) node[right]{50} circle (0.2ex);
 \fill (1,-5) node[left]{51} circle (0.2ex);
 \fill (2,-5) circle (0.2ex);
 \fill (3,-5) circle (0.2ex);
 \fill (4,-5) circle (0.2ex);
 \fill (5,-5) circle (0.2ex);
 \fill (6,-5) circle (0.2ex);
 \fill (7,-5) circle (0.2ex);
 \fill (8,-5) circle (0.2ex);
 \fill (9,-5) circle (0.2ex);
 \fill (10,-5) circle (0.2ex);
 \fill (1,-6) node[left]{61} circle (0.2ex);
 \fill (2,-6) circle (0.2ex);
 \fill (3,-6) circle (0.2ex);
 \fill (4,-6) circle (0.2ex);
 \fill (5,-6) circle (0.2ex);
 \fill (6,-6) circle (0.2ex);
 \fill (7,-6) circle (0.2ex);
 \fill (8,-6) circle (0.2ex);
 \fill (9,-6) circle (0.2ex);
 \fill (10,-6) circle (0.2ex);
 \fill (1,-7) node[left]{71} circle (0.2ex);
 \fill (2,-7) circle (0.2ex);
 \fill (3,-7) circle (0.2ex);
 \fill (4,-7) circle (0.2ex);
 \fill (5,-7) circle (0.2ex);
 \fill (6,-7) circle (0.2ex);
 \fill (7,-7) circle (0.2ex);
 \fill (8,-7) circle (0.2ex);
 \fill (9,-7) circle (0.2ex);
 \fill (10,-7) circle (0.2ex);
 \fill (1,-8) node[left]{81} circle (0.2ex);
 \fill (2,-8) circle (0.2ex);
 \fill (3,-8) circle (0.2ex);
 \fill (4,-8) circle (0.2ex);
 \fill (5,-8) circle (0.2ex);
 \fill (6,-8) circle (0.2ex);
 \fill (7,-8) circle (0.2ex);
 \fill (8,-8) circle (0.2ex);
 \fill (9,-8) circle (0.2ex);
 \fill (10,-8) circle (0.2ex);
 \fill (1,-9) node[left]{91} circle (0.2ex);
 \fill (2,-9) circle (0.2ex);
 \fill (3,-9) circle (0.2ex);
 \fill (4,-9) circle (0.2ex);
 \fill (5,-9) circle (0.2ex);
 \fill (6,-9) circle (0.2ex);
 \fill (7,-9) circle (0.2ex);
 \fill (8,-9) circle (0.2ex);
 \fill (9,-9) circle (0.2ex);
 \fill (10,-9) node[right]{100} circle (0.2ex);
% \draw (5,-4)--(6,-1)--(2,-1)--(5,-2)--(5,-4)--(4,-2);
\end{tikzpicture}
\end{center}
\end{document}
