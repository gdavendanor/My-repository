\documentclass[10pt,twoside]{article}
\usepackage[utf8]{inputenc}
\usepackage{amsmath,amsfonts,amssymb,amsthm,latexsym}
\usepackage[spanish,es-noshorthands]{babel}
\usepackage[T1]{fontenc}
\usepackage{lmodern}
\usepackage{graphicx,hyperref}
\usepackage{tikz,pgf}
\usepackage{multicol}
\usepackage{subfig}
\usepackage[papersize={6.5in,8.5in},width=5.5in,height=7.3in]{geometry}
\usepackage{fancyhdr}
\pagestyle{fancy}
\fancyhead[LE]{\url{www.autistici.org/mathgerman}}
\fancyhead[RE]{}
\fancyhead[RO]{matematicas.german@gmail.com}
\fancyhead[LO]{}

\author{Germ\'an Avenda\~no Ram\'irez~\thanks{Lic. Mat. U.D., M.Sc. U.N.}}
\title{\begin{minipage}{.2\textwidth}
\includegraphics[height=1.75cm]{Images/logo-colegio.png}\end{minipage}
\begin{minipage}{.55\textwidth}
\begin{center}
Taller 2, Números $\mathbb{R}$\\
Cálculo $11^{\circ}$
\end{center}
\end{minipage}\hfill
\begin{minipage}{.2\textwidth}
\includegraphics[height=1.75cm]{Images/logo-sed.png} 
\end{minipage}}
\date{}
\thispagestyle{plain}
\begin{document}
\maketitle
Nombre: \hrulefill Curso: \underline{\hspace*{44pt}} Fecha: \underline{\hspace*{2.5cm}}\\

~\ref{item01}--~\ref{item02} Encuentre los elementos de los conjuntos dados que son:
\begin{enumerate}
\item \{0, --10, 50, $\frac{22}{7}$, 0.538, $\sqrt{7}$, $1.2\overline{3}$, $-\frac{1}{3}$, $\sqrt[3]{2}$\}\label{item01}
\item \{1.001, 0.333, \ldots, $-\pi$, --11, 11, $\frac{13}{15}$, $\sqrt{16}$, 3.14, $\frac{15}{3}$\}\label{item02}
\begin{enumerate}
\begin{multicols}{2}
\item Números naturales
\item Enteros
\item Números racionales
\item Números irracionales
\end{multicols}
\end{enumerate}
  \item Complete esta tabla con sí o no
  \begin{center}
  \begin{tabular}{|l|c|c|c|c|c|c|c|}
\hline 
Numero & 5 & $\sqrt{2}$ & $-1.12$ & $1.1212212221\ldots$ & $\frac{7}{6}$ & $\sqrt{4}$ & $\sqrt{-3}$ \\ 
\hline 
Natural &  &  &  &  &  &  &  \\ 
\hline 
Entero &  &  &  &  &  &  &  \\ 
\hline 
Racional &  &  &  &  &  &  &  \\ 
\hline 
Irracional &  &  &  &  &  &  &  \\ 
\hline 
Real &  &  &  &  &  &  &  \\ 
\hline 
\end{tabular} 
  \end{center}
    \item Escribe los siguientes números en forma decimal y redondeando a la céntesimas: (puedes usar calculadora)
  \begin{enumerate}\begin{multicols}{5}
    \item $ \pi $ \item $ \sqrt{3} $  \item $ 1,1616\ldots $  
    \item $ 1,6565\ldots $ \item $ \frac{5}{9} $
  \end{multicols}
  \end{enumerate}
~\ref{item03}--\ref{item04} Determine la propiedad de los números reales usada:
\begin{multicols}{2}
\item $7+10=10+7$\label{item03}
\item $2(3+5)=(3+5)2$
\item $(x+2y)+3z=x+(2y+3z)$
\item $2(A+B)=2A+2B$
\item $(5x+1)3=15x+3$
\item $(x+a)(x+b)=(x+a)x+(x+a)b$
\item $2x(3+y)=(3+y)2x$
\item $7(a+b+c)=7(a+b)+7c$\label{item04}
\end{multicols}
~\ref{item05}--\ref{item06} Reescriba la expresi\'{o}n usando la propiedad dada de los n\'{u}meros reales.
\item Propiedad conmutativa de la adición,\; $x+3=$\underline{\hspace*{55pt}}\label{item05}
\item P. Asociativa de la multiplicación, \; $7(3x)=\underline{\hspace*{55pt}}$
\item P. Distributiva, \; $4(A+B)=$\underline{\hspace*{55pt}}
\item Propiedad Recolectiva, \; $5x+5y=$\underline{\hspace*{55pt}}\label{item06}

\ref{item07}--\ref{item08} Use las propiedades de los n\'{u}meros reales para escribir la expresi\'{o}n sin par\'{e}ntesis.
\begin{multicols}{4}
\item $3(x+y)$\label{item07}
\item $(a-b)8$
\item $4(2m)$
\item $\frac{4}{3}(-6y)$
\item $-\frac{5}{2}(2x-4y)$
\item $(3a)(b+c-2d)$\label{item08}
\end{multicols}
\ref{item09}--\ref{item10} Realice las operaciones indicadas:
\item \label{item09}
\begin{enumerate}
\begin{multicols}{2}
\item $\frac{3}{10}+\frac{4}{15}$
\item $\frac{1}{4}+\frac{1}{5}$
\end{multicols}
\end{enumerate}
\item
\begin{enumerate}
\begin{multicols}{2}
\item $\frac{2}{3}-\frac{3}{5}$
\item $1+\frac{5}{8}-\frac{1}{6}$
\end{multicols}
\end{enumerate}
\item
\begin{enumerate}
\begin{multicols}{2}
\item $\frac{2}{3}(6-\frac{3}{2})$
\item $0.25(\frac{8}{9}+\frac{1}{2})$
\end{multicols}
\end{enumerate}
\item
\begin{enumerate}
\begin{multicols}{2}
\item $(3+\frac{1}{4})(1-\frac{4}{5})$
\item $(\frac{1}{2}-\frac{1}{3})(\frac{1}{2}+\frac{1}{3})$
\end{multicols}
\end{enumerate}
\item
\begin{enumerate}
\begin{multicols}{2}
\item $\dfrac{2}{\frac{2}{3}}-\dfrac{\frac{2}{3}}{2}$
\item $\dfrac{\frac{1}{12}}{\frac{1}{8}-\frac{1}{9}}$
\end{multicols}
\end{enumerate}
\item \label{item10}
\begin{enumerate}
\begin{multicols}{2}
\item $\dfrac{2-\frac{3}{4}}{\frac{1}{2}-\frac{1}{3}}$
\item $\dfrac{\frac{2}{5}+\frac{1}{2}}{\frac{1}{10}+\frac{3}{15}}$
\end{multicols}
\end{enumerate}
\ref{item11}--\ref{item12} Ubique el s\'{i}mbolo correcto (<, >, o =) en el espacio:
\item \label{item11}
\begin{enumerate}
\begin{multicols}{3}
\item $3$ \tikz \draw (0,0) rectangle (.5,.5); $\frac{7}{2}$
\item $-3$ \tikz \draw (0,0) rectangle (.5,.5); $-\frac{7}{2}$
\item 3.5 \tikz \draw (0,0) rectangle (.5,.5); $\frac{7}{2}$
\end{multicols}
\end{enumerate}
\newpage
\item \label{item12}
\begin{enumerate}
\begin{multicols}{4}
\item $\frac{2}{3}$ \tikz \draw (0,0) rectangle (.5,.5); 0.67
\item $\frac{2}{3}$ \tikz \draw (0,0) rectangle (.5,.5); $-0.67$
\item $|0.67|$ \tikz \draw (0,0) rectangle (.5,.5); $|-0.67|$
\end{multicols}
\end{enumerate}
\ref{item13}--\ref{item14} Determine si cada inecuación es verdadera o falsa:
\item \label{item13}
\begin{enumerate}
\begin{multicols}{2}
\item $-6<-10$
\item $\sqrt{2}>1.41$
\end{multicols}
\end{enumerate}
\item
\begin{enumerate}
\begin{multicols}{2}
\item $\dfrac{10}{11}<\dfrac{12}{13}$
\item $-\dfrac{1}{2}<-1$
\end{multicols}
\end{enumerate}
\item
\begin{enumerate}
\begin{multicols}{2}
\item $-\pi>-3$
\item $8\leq9$
\end{multicols}
\end{enumerate}
\item \label{item14}
\begin{enumerate}
\begin{multicols}{2}
\item $1.1>1.\overline{1}$
\item $8\leq8$
\end{multicols}
\end{enumerate}
\ref{item15}--\ref{item16} Escriba cada afirmación en términos de desigualdades:
\item \label{item15}
\begin{enumerate}
\begin{multicols}{2}
\item $x$ es positivo
\item $t$ es menor que 4
\item $a$ es mayor o igual que $\pi$
\item $x$ es menor que $\frac{1}{3}$ y mayor que $-5$
\item La distancia de $p$ a 3 es por mucho 5
\end{multicols}
\end{enumerate}
\item \label{item16}
\begin{enumerate}
\begin{multicols}{2}
\item $y$ es negativo
\item $z$ es mayor que 1
\item $b$ es por mucho 8
\item $w$ es positivo y es menor o igual que 17
\item $y$ está por lo menos a 2 unidades de $\pi$ 
\end{multicols}
\end{enumerate}
\ref{item17}--\ref{item18} Encuentre la operación indicada dados los conjuntos:
A = \{1, 2, 3, 4, 5, 6, 7\}, \quad B = \{2, 4, 6, 8\} \quad y \quad C = \{7, 8, 9, 10\}
\item \label{item17}
\begin{enumerate}
\begin{multicols}{2}
\item $A\cup B$
\item $A\cap B$
\end{multicols}
\end{enumerate}
\item 
\begin{enumerate}
\begin{multicols}{2}
\item $B\cup C$
\item $B\cap C$
\end{multicols}
\end{enumerate}
\item
\begin{enumerate}
\begin{multicols}{2}
\item $A\cup C$
\item $A\cap C$
\end{multicols}
\end{enumerate}
\item \label{item18}
\begin{enumerate}
\begin{multicols}{2}
\item $A\cup B\cup C$
\item $A\cap B\cap C$
\end{multicols}
\end{enumerate}
\ref{item19}--\ref{item20} Encuentre el conjunto indicado si \quad $A=\{x/x\geq-2\}$ \quad $B=\{x|x<4\}$ \\ $C=\{x|-1<x\leq 5\}$
 \newpage
 \item \label{item19}
 \begin{enumerate}
 \begin{multicols}{2}
 \item $B\cup C$
 \item $B\cap C$
 \end{multicols}
 \end{enumerate}
 \item \label{item20}
 \begin{enumerate}
 \begin{multicols}{2}
 \item $A\cap C$
 \item $A \cap B$
 \end{multicols}
 \end{enumerate}
\ref{item21}--\ref{item22} Exprese la desigualdad en notación de intervalos, y luego grafique el intervalo en la recta numérica:
\begin{multicols}{3}
 \item \label{item21} $x\leq 1$
 \item $1\leq x\leq 2$
 \item $-2<x\leq 1$
 \item $x\geq -5$
 \item $x>-1$
 \item $-5<x<2$ \label{item22}
\end{multicols}
\ref{item23}--\ref{item24} Exprese cada conjunto en notaci\'{o}n de intervalos.
\item \label{item23}
\begin{enumerate}
\begin{multicols}{2}
\item \begin{tikzpicture}[scale=.5]
\draw[<->] (-3.5,0) -- (5.5,0);
\node[below] (-3) at (-3,0) {$-3$};
\node[below] (5) at (5,0) {$5$};
\fill (-3,0)circle (1ex);
\fill (5,0) circle (1ex);
\draw[very thick] (-2.9,0.05) -- (4.9,0.05);
\end{tikzpicture}
\item \begin{tikzpicture}[scale=.5]
\draw[<-] (-3.5,0) -- (-3.1,0);
\draw[->] (-2.9,0) -- (5.5,0);
\node[below] (-3) at (-3,0) {$-3$};
\node[below] (5) at (5,0) {$5$};
\draw (-3,0)circle (1ex);
\fill (5,0) circle (1ex);
\draw (0,0.2) -- node[below]{0}(0,-.2);
\draw[thick] (-2.9,0.05) -- (4.9,0.05);
\end{tikzpicture}
\end{multicols}
\end{enumerate}
\item \label{item24}
\begin{enumerate}
\begin{multicols}{2}
\item \begin{tikzpicture}[scale=.75]
\draw[<-] (-1,0) -- (1.9,0);
\draw[->] (2.1,0) -- (2.5,0);
\node[below] (2) at (2,0) {$2$};
\draw (2,0)circle (.7ex);
\fill (0,0)node[below]{0} circle (.7ex);
\draw (0.1,.025)--(1.9,0.025);
\end{tikzpicture}
\item \begin{tikzpicture}[scale=.75]
\draw[->] (-1.9,0) -- (1,0);
\draw[<-] (-2.5,0) -- (-2.1,0);
\node[below] (-2) at (-2,0) {$-2$};
\draw (-2,0)circle (.7ex);
\fill (0,0) node[below]{0} circle (.7ex);
\draw (-1.9,0.025)--(-0.1,.025);
\end{tikzpicture}
\end{multicols}
\end{enumerate}
  \item Escribe tres números racionales comprendidos entre $ \frac{1}{15} $ y $ \frac{2}{15} $
  \item Representa en la recta real los siguientes intervalos
  \begin{enumerate}\begin{multicols}{5}
    \item $ [2,3] $    \item $ (-1,-3) $\item $ [1,3) $
    \item $ (-2,-5) $ \item $ (-\infty,7] $
  \end{multicols}
  \end{enumerate}
  \item Representa en la recta real los números que verifican:
  \begin{enumerate}\begin{multicols}{4}
    \item $ |x|=0 $    \item $ |x|=2 $ \item $ |x|=|-3| $
    \item $ |x|=-1 $
  \end{multicols}
  \end{enumerate}
  \item Representa en la recta real los intervalos que verifican:
  \begin{enumerate}\begin{multicols}{4}
    \item $ |x|\leq 2 $    \item $ |x|<2 $
    \item $ |x|\geq 2 $ \item $ |x|>2 $
  \end{multicols}
  \end{enumerate}
  \item Encuentra las fracciones generatrices de:
  \begin{enumerate}\begin{multicols}{4}
    \item 1,121 \item $ 10,\overline{1} $
    \item $ 2,\overline{13} $ \item $ 3,01\overline{27} $    
  \end{multicols}
  \end{enumerate}
  \item Ordena de menor a mayor los siguientes números reales
  \[ \sqrt{3},\quad 173,\quad \frac{-1}{3},\quad \pi, \quad -0,33, \quad 2,\overline{73}, \quad 1,7\overline{3}, \quad -\frac{1}{5} \]
  \end{enumerate}
\end{document}
