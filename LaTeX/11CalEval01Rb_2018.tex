\documentclass[fleqn]{article}
\usepackage[spanish,es-noshorthands]{babel}
\usepackage[utf8]{inputenc} 
\usepackage[papersize={6.5in,8.5in},total={5.5in,7.5in},centering]{geometry}
\usepackage{mathexam}
\usepackage{amsmath,amsthm,amsfonts,amssymb}
\usepackage{graphicx}
\usepackage{tikz,pgf}
\usepackage{multicol}

\ExamClass{\includegraphics[height=16pt]{Images/logo-sed.png} Cálculo $11^{\circ}$}
\ExamName{``Evaluación 01, Números $\mathbb{R}$''}
\ExamHead{\includegraphics[height=16pt]{Images/logo-colegio.png} IEDAB}
\newcommand{\LineaNombre}{%
\par
\vspace{\baselineskip}
Nombre:\hrulefill \; Curso: \underline{\hspace*{48pt}} \; Fecha: \underline{\hspace*{2.5cm}} \relax
\par}
\let\ds\displaystyle

\begin{document}
\ExamInstrBox{
Respuesta sin justificar mediante procedimiento no será tenida en cuenta en la calificación. Escriba sus respuestas en el espacio indicado. Tiene 45 minutos para contestar esta prueba.}
\LineaNombre
\begin{enumerate}
  \item Realice las operaciones indicadas en una hoja anexa, simplificando siempre que sea posible:
  \begin{enumerate}
  \begin{multicols}{2}
    \item $ \dfrac{5}{6}+\dfrac{1}{9}= $ 
    \item $ 3+\dfrac{3}{8}-\dfrac{1}{6}= $ 
    \item $ 0.75\left(\dfrac{7}{9}+\dfrac{2}{3}\right)= $ 
    \item $ \left(\dfrac{2}{3}-\dfrac{2}{7}\right)\left(\dfrac{1}{5}-\dfrac{1}{4}\right)= $ 
    \item $ \dfrac{3-\frac{2}{3}}{\frac{1}{3}-\frac{1}{5}}= $   
  \end{multicols}
  \end{enumerate}
   \item Sobre la línea, determine la propiedad de los números reales que se ha usado:
  \begin{enumerate}
    \item $ (a+b)(a-b)=(a-b)(a+b) $ \underline{\hspace{6cm}}
    \item $ 4(a+b)=4a+4b $ \underline{\hspace{6cm}}
    \item $ (A+1)(x+y)=(A+1)x+(A+1)y $ \underline{\hspace{6cm}}
    \item $ 3x+2y=2y+3x $ \underline{\hspace{6cm}}
  \end{enumerate}
  \item Exprese cada intervalo como una desigualdad y luego grafíquela en la recta dispuesta para ello.
  
  \begin{enumerate}
    \item $ [-3,5) $ \; \underline{\hspace{3cm}} \; \includegraphics[scale=1]{Images/Recta01.pdf} 
    \item $ (-\infty,5]$ \; \underline{\hspace{3cm}} \; \includegraphics[scale=1]{Images/Recta01.pdf}
  \end{enumerate}
  \item Exprese en notación de intervalos y luego grafique el correspondiente intervalo:
  \begin{enumerate}
    \item $ x\geq 6 $  \; \underline{\hspace{3cm}} \; \includegraphics[scale=1]{Images/Recta01.pdf}
    \item $ -2< x\leq 4 $ \; \underline{\hspace{3cm}} \; \includegraphics[scale=1]{Images/Recta01.pdf}
  \end{enumerate}
  \item Exprese como una desigualdad las siguientes expresiones:
  \begin{enumerate}
  \begin{multicols}{2}
    \item $ b $ es negativo 
    \item $ a $ es menor que 4 
    \item $ q $ es menor que 8 y mayor o igual que $ -5 $
  \end{multicols}
  \end{enumerate}
  \newpage
   \section*{Preparándonos para la Prueba Saber}
   \item ¿Cuál de las siguientes desigualdades corresponde al intervalo $[-2,1)$
  \begin{enumerate}
  \begin{multicols}{4}
  \item $-2\leq x\leq 1$
  \item $-2<x<1$
  \item $-2<x\leq 1$
  \item $-2\leq x<1$	
  \end{multicols}
  \end{enumerate}
  \item ¿Cuál es el valor de $(-2)^{4}$?
  \begin{enumerate}
  \begin{multicols}{4}
	\item 16
	\item 8
	\item $-8$  
	\item $-16$
  \end{multicols}
  \end{enumerate}
  \item ¿Cuál es la base de la expresión $-7^{2}$?
  \begin{enumerate}
  \begin{multicols}{4}
  \item $-7$
  \item $7$
  \item $-2$
  \item 2
  \end{multicols}
  \end{enumerate}
  \item ¿cuál de las siguientes es la forma simplificada de $\dfrac{x^{6}}{x^{2}}$, $x\neq 0$?
  \begin{enumerate}
  \begin{multicols}{4}
  \item $x^{-4}$
  \item $x^{2}$
  \item $x^{3}$
  \item $x^{4}$
  \end{multicols}
  \end{enumerate}
Conteste los puntos siguientes, con base en: En la recta numérica, se han señalado algunos puntos con sus respectivas coordenadas
  
  \begin{tikzpicture}
  \draw[<->] (-0.25,0) -- (10.25,0);
\draw [thick] (0,-.1)node[below]{$-1$} -- (0,0.1)node[above]{A};
\draw [thick](6,-.1)[thick]node[below]{$-\frac{1}{4}$} -- (6,0.1)node[above]{B};
\draw [thick] (7,-.1)node[below]{$-\frac{1}{8}$} -- (7,0.1)node[above]{C};
\draw [thick] (8,-.1)node[below]{$0$} -- (8,0.1)node[above]{D};
\draw [thick] (9,-.1)node[below]{$\frac{1}{8}$} -- (9,0.1)node[above]{E};
\draw [thick] (10,-.1)node[below]{$\frac{1}{4}$} -- (10,0.1)node[above]{F};
  \end{tikzpicture}
  \item Si $\overline{DE}$ se divide en $n$ segmentos congruentes, la longitud de cada uno de los $n$ segmentos es:
  \begin{enumerate}\begin{multicols}{4}
  \item $\dfrac{1}{8n}$
  \item $\dfrac{8}{n}$  
  \item $\dfrac{1}{n}$
  \item $\dfrac{4}{n}$
  \end{multicols}
  \end{enumerate}
  \item Si $M$ y $N$ son los puntos medios de $\overline{AB}$ y $\overline{CD}$ respectivamente, la longitud $MN$ es,
  \begin{enumerate}
  \begin{multicols}{4}
  \item $\dfrac{9}{15}$
  \item $\dfrac{11}{16}$
  \item $\dfrac{1}{2}$
  \item $\dfrac{5}{8}$
  \end{multicols}
  \end{enumerate}
  \item De la expresión $\left[\dfrac{1-\sqrt{3}}{2}\right]^{2}$ se puede afirmar que corresponde a un números
  \begin{enumerate}
  \begin{multicols}{2}
  \item irracional y se ubica en $\overline{CD}$
  \item irracional y se ubica en $\overline{DE}$
   \item racional y se ubica en $\overline{AB}$
  \item racional y se ubica en $\overline{BD}$  
  \end{multicols}
  \end{enumerate}
\end{enumerate}
\begin{center}
\begin{tabular}{cccccccccccccc}
6 & 7 & 8 & 9 & 10 & 11 & 12 \\ 
\textcircled{a} & \textcircled{a} & \textcircled{a} & \textcircled{a} & \textcircled{a} & \textcircled{a} & \textcircled{a}\\ 
\textcircled{b} & \textcircled{b} & \textcircled{b} & \textcircled{b} & \textcircled{b} & \textcircled{b} & \textcircled{b} \\ 
\textcircled{c} & \textcircled{c} & \textcircled{c} & \textcircled{c} & \textcircled{c} & \textcircled{c} & \textcircled{c}\\ 
\textcircled{d} & \textcircled{d} & \textcircled{d} & \textcircled{d} & \textcircled{d} & \textcircled{d} & \textcircled{d}\\ 
\end{tabular} 
\end{center}
\end{document}
