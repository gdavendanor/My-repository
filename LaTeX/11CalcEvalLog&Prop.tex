\documentclass[letterpaper,fleqn]{article}
\usepackage[spanish,es-noshorthands]{babel}
\usepackage[utf8]{inputenc} 
\usepackage[papersize={6.5in,8.5in},left=1cm, right=1cm, top=1.5cm, bottom=1.7cm]{geometry}
\usepackage{mathexam}
\usepackage{amsmath}
\usepackage{amssymb}
\usepackage{graphicx}

\ExamClass{\includegraphics[height=16pt]{Images/logo-sed.png} Cálculo $11^{\circ}$}
\ExamName{Lógica y proposiciones}
\ExamHead{\includegraphics[height=16pt]{Images/logo-colegio.png} IEDAB}
\newcommand{\LineaNombre}{%
\par
\vspace{\baselineskip}
Nombre:\hrulefill \; Curso: \underline{\hspace*{48pt}} \; Fecha: \underline{\hspace*{2.5cm}} \relax
\par}
\let\ds\displaystyle

\begin{document}
\ExamInstrBox{
Respuesta sin justificar mediante procedimiento no será tenida en cuenta en la calificación. Escriba sus respuestas en el espacio indicado. Tiene 30 minutos para contestar esta prueba.}
\LineaNombre
\begin{enumerate}
 \item Identifique cuales son y cuales no son proposiciones. A las que sean proposiciones, as\'{i}gneles el valor de verdad correspondiente
 \begin{enumerate}
 \item La tierra gira alrededor del sol
 \item ¿La tierra es un planeta?
 \item ¡A terminar la tarea!
 \item El sol gira alrededor de la tierra.
 \end{enumerate}
 \item Dadas las proposiciones\\
 P: Todas las aves vuelan\\
 Q: Los mamíferos son vertebrados\\
Escriba las siguientes proposiciones compuestas y asígneles el valor de verdad.
\begin{enumerate}
\item $P\vee Q$:
\item $P\wedge Q$:
\item $P \veebar Q$:
\item $\neg P$:
\item $P\Rightarrow Q$:
\item $Q\Leftrightarrow P$:
\item $\neg (P\wedge Q)$
\end{enumerate}
\item Dadas las proposiciones $p$, $q$ y $r$, complete la siguiente tabla de verdad:
\begin{center}
\begin{tabular}{|c|c|c|c|c|c|c|c|}
\hline 
p & q & r & $[p\Rightarrow q$ $\wedge$ $q\Rightarrow r]$ & $\Leftrightarrow$ $[p\Rightarrow r]$ \\ 
\hline 
 &  &  &   &  \\ 
\hline 
 & & & &  \\ \hline
  & & & &  \\ \hline
\end{tabular} 
\end{center}
 \end{enumerate}

\end{document}
