\documentclass[twoside]{article}
\usepackage[utf8]{inputenc}
\usepackage{amsmath,amsfonts,amssymb,amsthm,latexsym}
\usepackage[spanish,es-noshorthands]{babel}
\usepackage[T1]{fontenc}
\usepackage{lmodern}
\usepackage{graphicx,hyperref}
\usepackage{tikz,pgf}
\usepackage{marvosym}
\usepackage{multicol}
\usepackage{fancyhdr}
\usepackage[papersize={5.5in,8.5in},left=.75cm,right=.75cm,top=1.5cm,bottom=1.3cm]{geometry}
\usepackage{fancyhdr}
\pagestyle{fancy}
\fancyhead[LE]{\Email ~iedabgerman@autistici.org}
\fancyhead[RE]{}
\fancyhead[RO]{\url{https://www.autistici.org/mathgerman}}
\fancyhead[LO]{}

\author{Germ\'an Avenda\~no Ram\'irez~\thanks{Lic. Mat. U.D., M.Sc. U.N.}}
\title{\begin{minipage}{.2\textwidth}
\includegraphics[height=1.75cm]{Images/logo-colegio.png}\end{minipage}
\begin{minipage}{.55\textwidth}
\begin{center}
Introducción\\
Probabilidad $11^{\circ}$
\end{center}
\end{minipage}\hfill
\begin{minipage}{.2\textwidth}
\includegraphics[height=1.75cm]{Images/logo-sed.png} 
\end{minipage}}
\date{}
\thispagestyle{plain}
\begin{document}
\maketitle
Nombre: \hrulefill Curso: \underline{\hspace*{44pt}} Fecha: \underline{\hspace*{2.5cm}}
\section{Nivel I}
\begin{enumerate}
  \item Sea el experimento aleatorio “lanzar un dado”. Escribe el es pacio muestral e indica dos sucesos aleatorios que consten
de tres sucesos elementales cada uno.
\item Se saca una carta de una baraja española de 40 cartas. Escribe los sucesos contrarios de los siguientes:
\begin{enumerate}
  \item A = “sacar un as”
  \item B = “obtener un número primo”
  \item C = “obtener puntuación impar”
  \item D = “obtener puntuación positiva”
\end{enumerate}
\item Se lanza un dado. Escribe los siguientes sucesos y halla sus probabilidades:
\begin{enumerate}
  \item A = “obtener un número mayor que 3”
  \item B = “obtener un número primo”
  \item C = “obtener puntuación impar”
  \item D = “obtener puntuación positiva”
\end{enumerate}
\item Con los datos del problema anterior, indica qué sucesos son
los siguientes y halla la probabilidad de cada uno.
\begin{multicols}{3}
  \begin{enumerate}
    \item $ \overline{A} $
    \item $ \overline{B} $
    \item $ A\cup B $
    \item $ A\cap B $
    \item $ B\cap \overline{B} $
    \item $ \overline{A\cap B} $
    \item $ \overline{A}\cap \overline{B} $
    \item $ \overline{A\cup B} $
    \item $ \overline{A}\cup \overline{B} $
    \item $ (A\cap B)\cap C $
    \item $ \overline{(A\cap B)\cap C} $
    \item $ (\overline{A}\cap\overline{B})\cup\overline{C} $
  \end{enumerate}
\end{multicols}
\item El espacio muestral de un experimento aleatorio es  \{1,2,3,4,5,6,7,8,9\}. Sean los sucesos:
\[ A=\{3,5,6,8\}\qquad B=\{1,2,3,4,8,9\}\qquad C=\{1,4,5,7,9\} \]
Calcule la probabilidad de los sucesos:
\begin{multicols}{3}
  \begin{enumerate}
    \item $ \overline{C} $
    \item $ A\cup C $
    \item $ A\cup\overline{C}\cup B $
    \item $ A\cap\overline{B} $
    \item $ A\cup\overline{B} $
    \item $ \overline{A}\cap B $
  \end{enumerate}
\end{multicols}
\item Con los datos anteriores, halla los siguientes sucesos y sus probabilidades.
\begin{multicols}{2}
  \begin{enumerate}
    \item $ (\overline{A}\cap\overline{B})\cap\overline{C} $
    \item $ \overline{(A\cap B)\cap C} $
    \item $ (\overline{A}\cap\overline{B})\cup\overline{C} $
    \item $ B\cup(A\cap C) $
    \item $ (B\cup A)\cap(B\cup C) $
    \item $ B\cap (A\cup C) $
  \end{enumerate}
\end{multicols}
\item Se considera el experimento aleatorio “lanzar tres monedas”. Construye el espacio muestral.
\item Sea el experimento del problema anterior. Se  consideran los sucesos:
\begin{multicols}{2}
  A = “sacar solo una cara”\\
  B = “sacar al menos una cruz”\\
  C = “sacar tres caras o tres cruces”\\
\end{multicols}
Halla las probabilidades de:
\begin{multicols}{3}
  \begin{enumerate}
    \item $ A\cap B $
    \item $ A\cup C $
    \item $ C\cap \overline{B} $
    \item $ \overline{A\cup \overline{B}} $
    \item $ \overline{A}\cup B $
    \item $ (\overline{A}\cap\overline{B})\cap \overline{C} $
  \end{enumerate}
\end{multicols}
\item En un determinado experimento aleatorio el espacio muestral consta de sólo tres sucesos elementales siendo la probabilidad de los dos primeros son 0,2 y 0,5. ¿Cuál es la probabilidad del tercero?
\item En un experimento aleatorio el espacio muestral es E = \{a, b, c, d\} sabiendo que:
\[ P(\{a\})=P(\{b\})=\dfrac{1}{8}\qquad \mbox{y}\qquad P(\{c\})=\dfrac{1}{5} \qquad \mbox{Halle }P(\{d\}) \]
\item Sea el experimento aleatorio ''lanzar un dado``. Halla la probabilidad de los sucesos:
\begin{enumerate}
\item $A_{1}=$"Sacar un número"
\item $A_{2}=$"sacar un número primo"
\item $A_{3}=$"sacar un número menor que 3"
\item $A_{4=}$"sacar un número par mayor que 4"
\item $A_{5}=$"sacar un número par o mayor que 4"
\end{enumerate}
\item Halla la probabilidad de que al lanzar dos dados aparezca:
\begin{enumerate}
  \item en el primero un número impar y en el segundo un múltiplo de 3
\item en el primero par y en el segundo mayor que 2
\end{enumerate}
\item Calcula la probabilidad de que al lanzar dos dados la suma de sus puntos sea:
\begin{enumerate}
\begin{multicols}{3}
\item 5
\item mayor o igual que 10
\item múltiplo de 3
\end{multicols}
\end{enumerate}
\item Durante el curso 1986/87 el número de estudiantes de los antiguos BUP y COU, en Aragón, fue:

\begin{tabular}{|c|c|c|c|}
\hline 
 & Huesca & Teruel & Zaragoza \\ 
\hline 
Centro público & 5091 & 2277 & 17805 \\ 
\hline 
Centro privado & 1284 & 896 & 12775 \\ 
\hline 
\end{tabular} 

Si hubiese elegido una de esas personas al azar, calcula la probabilidad de que estudiase en:
\begin{enumerate}
\begin{multicols}{2}
\item Zaragoza
\item Un centro privado de Teruel
\item Un centro público
\end{multicols}
\end{enumerate}
\item Calcula la probabilidad de que al levantar una ficha del dominó:
\begin{enumerate}
\item Sea una ficha doble
\item la suma de sus puntos sea 6
\end{enumerate}
\item Tengo en la mano seis cartas con los números 1, 2, 3, 5, 6, 7. Mi amigo toma una al azar:
\begin{enumerate}
\item ¿Cuál es la probabilidad de que obtenga un número menor que 4?
\item ¿Cuál es la probabilidad de que el número que obtenga sea divisible por 2?
\end{enumerate}
\item Si extraes una carta de una baraja española, calcula la probabilidad de:
\begin{enumerate}
\begin{multicols}{2}
\item Que sea un caballo
\item Que sea una copa
\item Que sea el caballo de copas
\item Que sea un caballo o una copa
\end{multicols}
\end{enumerate}
\item Una urna contiene 8 bolas rojas, 5 amarillas y 7 verdes. Se extrae una al azar. Determina la probabilidad de que:
\begin{enumerate}
\begin{multicols}{2}
\item Sea roja o verde
\item No sea roja
\item Sea roja o amarilla
\end{multicols}
\end{enumerate}
\item Una bolsa contiene 100 papeletas de una rifa numeradas del 1 al 100. Se extrae una papeleta al azar. ¿Cuál el la probabilidad de que:
\begin{enumerate}
\item el número extraído tenga una sola cifra;
\item el número extraído tenga dos cifras;
\item el número extraído tenga tres cifras;
\item el número extraído tenga cuatro cifras?
\end{enumerate}
\item Tres amigos de edades 14, 15 y 16 años están esquiando juntos. Teniendo en cuenta que llegan al arrastre de uno en uno, ¿cuál es la probabilidad de que lleguen por orden de edades?
\item Se lanza una moneda cuatro veces. ¿Cuál el la probabilidad de que todas sean caras? ¿Y de que todas sean cruces? ¿Y de que todas sean o bien caras o bien cruces?
\item En un instituto hay 1.000 alumnos repartidos por cursos de esta forma:

\begin{tabular}{|c|c|c|c|c|}
\hline 
 & Primero & Segundo & Tercero & Cuarto \\ 
\hline 
Chicos & 120 & 100 & 95 & 85 \\ 
\hline 
Chicas & 200 & 150 & 130 & 120 \\ 
\hline 
\end{tabular}

Elegido un alumno al azar, calcula las siguientes probabilidades:
\begin{enumerate}
\begin{multicols}{2}
\item Ser chico
\item Ser chica
\item Ser alumno de primero
\item Ser alumno de segundo
\item Ser alumno de tercero
\item Ser alumno de cuarto
\item Ser chica y alumno de cuarto
\item Ser chico y alumno de segundo
\end{multicols}
\end{enumerate}
\item En una urna hay 50 bolas entre blancas, rojas y negras. ¿Cuántas bolas hay de cada color en los siguientes casos?
\begin{enumerate}
\item Si $P(B)=\frac{2}{5}$ \quad y \quad $P(N)=\frac{1}{10}$
\item Si $P(B)=\frac{2}{5}$ \quad y \quad $P(N)=P(R)$
\end{enumerate}
\item Se lanzan dos dados a la vez. ¿Cuál es la probabilidad de obtener los siguientes sucesos?
\begin{enumerate}
\begin{multicols}{2}
\item Un 4 y un 5
\item Primero 4 y después 5
\item Suma 9
\item Ni 4 ni 5
\end{multicols}
\end{enumerate}
\end{enumerate}


\end{document}
