\documentclass[10pt,letterpaper]{book}
\usepackage[utf8]{inputenc}
\usepackage[spanish,es-noshorthands]{babel}
\usepackage[T1]{fontenc}
\usepackage{graphicx}
\usepackage{lmodern}
\author{Manuel Garc\'{i}a Morente}
\begin{document}
\chapter*{Lecci\'{o}n I:\\EL CONJUNTO DE LA FILOSOF\'{I}A}
[La filosofía y su vivencia. Definiciones filosóficas y vivencias filosóficas. Sentido de la voz filosofía. La filosofía antigua. La filosofía en la Edad Media. La filosofía en la Edad Moderna. Las disciplinas filosóficas. Las ciencias y la filosofía. Las partes de la filosofía.]
\section*{La filosofía y su vivencia}
Vamos a iniciar el curso de introducción a la filosofía planteando e intentando resolver algunas de las cuestiones principales de esta disciplina. Ustedes vienen a estas aulas y yo a ellas también, para hacer juntos algo. ¿Oué es lo que vamos a hacer juntos? Lo dice el tema: vamos a hacer filosofía.

La filosofía es, por de pronto, algo que el hombre hace, que el hombre ha hecho. Lo primero que debemos intentar, pues, es definir ese "hacer" que llamamos filosofía. Deberemos por lo menos dar un concepto general de la filosofía, y quizá fuese la incumbencia de esta lección primera la de explicar y exponer qué es la filosofía. Pero esto es imposible. Es absolutamente imposible decir de antemano qué es filosofía. No se
puede definir la filosofía antes de hacerla; como no se puede definir en general ninguna ciencia, ni ninguna disciplina, antes de entrar directamente en el trabajo de hacerla.

Una ciencia, una disciplina, un "hacer" humano cualquiera, recibe su concepto claro, su noción precisa, cuando ya el hombre ha dominado ese hacer. Sólo sabrán ustedes qué es filosofía cuando sean realmente filósofos. Por consiguiente, no puedo decirles qué es filosofía. Filosofía es lo que vamos a hacer ahora juntos, durante este curso en la Universidad de Tucumán.

¿Qué quiere esto decir? Esto quiere decir que la filosofía, más que ninguna otra disciplina, necesita ser vivida. Necesitamos tener de ella una \guillemoleft vivencia \guillemoright. La palabra vivencia ha sido introducida en el vocabulario español por los escritores de la Revista de Occidente, como traducción de la palabra alemana «Erlebnis». Vivencia significa lo que tenemos real­mente en nuestro ser psíquico; lo que real y verdaderamente estamos sintiendo, teniendo, en la plenitud de la palabra «tener».
\end{document}