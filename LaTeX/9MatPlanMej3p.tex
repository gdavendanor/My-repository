\documentclass[letterpaper]{article}
\usepackage[utf8]{inputenc}
\usepackage{amsmath,amsfonts,amssymb,amsthm,latexsym}
\usepackage[spanish,es-noshorthands]{babel}
\usepackage[T1]{fontenc}
\usepackage{lmodern}
\usepackage{graphicx,hyperref}
\usepackage{tikz,pgf}
\usepackage{marvosym}
\usepackage{multicol}
\usepackage{fancyhdr}
\usepackage[height=9.5in,width=7in]{geometry}
\usepackage{fancyhdr}
\pagestyle{fancy}
\fancyhead[LE]{\Email matematicas.german@gmail.com}
\fancyhead[RE]{\url{https://www.autistici.org/mathgerman}}
\fancyhead[RO]{\url{https://www.autistici.org/mathgerman}}
\fancyhead[LO]{\Email matematicas.german@gmail.com}

\author{Germ\'an Avenda\~no Ram\'irez~\thanks{Lic. Mat. U.D., M.Sc. U.N.}}
\title{\begin{minipage}{.2\textwidth}
\includegraphics[height=1.75cm]{Images/logo-colegio.png}\end{minipage}
\begin{minipage}{.55\textwidth}
\begin{center}
Plan de mejoramiento 3 período\\
Matemáticas $9^{\circ}$
\end{center}
\end{minipage}\hfill
\begin{minipage}{.2\textwidth}
\includegraphics[height=1.75cm]{Images/logo-sed.png} 
\end{minipage}}
\date{}
\thispagestyle{plain}
\begin{document}
\maketitle
Nombre: \hrulefill Curso: \underline{\hspace*{44pt}} Fecha: \underline{\hspace*{2.5cm}}
\begin{multicols}{2}
Para los ejercicios \ref{q01}--\ref{q02} resuelva las ecuaciones cuadráticas usando la factorización como método.
\begin{enumerate}
\begin{multicols}{2}
\item \label{q01} $x^{2}+8x=0$
\item $x^{2}=6x$
\item $x^{2}-3x-28=0$
\item \label{q02} $2x^{2}+x-3=0$
\end{multicols}
Para los ejercicios \ref{q03}--\ref{q04}, use la propiedad 1\footnote{$x^{2}=a$ si y solamente sí $x=\sqrt{a}$ o, $x=-\sqrt{a}$, que se puede simplificar así $x=\pm\sqrt{a}$} para resolver la función cuadrática
\begin{multicols}{2}
\item \label{q03} $2x^{2}=90$
\item $(y-3)^{2}=-18$
\item $(2x+3)^{2}=24$
\item \label{q04} $a^{2}-27=0$
\end{multicols}
Para los problemas \ref{q05}--\ref{q06}, use la fórmula cuadrática $x=\dfrac{-b\pm\sqrt{b^{2}-4ac}}{2a}$, para resolver la ecuación:
\begin{multicols}{2}
\item \label{q05} $x^{2}+6x+4=0$
\item $x^{2}+4x+6=0$
\item $3x^{2}-2x+4=0$
\item \label{q06} $5x^{2}-x-3=0$
\end{multicols}
Para los problemas \ref{q07}--\ref{q08}, encuentre el discriminante de cada ecuación ($b^{2}-4ac$) y determine cuando la ecuación tiene (1) dos soluciones no reales (complejas), (2) una solución real y una compleja  o (3) dos soluciones reales. No es necesario solucionarlas
\begin{multicols}{2}
\item \label{q07} $4x^{2}-20x+25=0$
\item $5x^{2}-7x+31=0$
\item $7x^{2}-2x-14=0$
\item \label{q08} $5x^{2}-2x=4$
\end{multicols}
Soluciones las ecuaciones de los ejercicios \ref{09}--\ref{10}
\begin{multicols}{2}
\item \label{09} $x^{2}-17=0$
\item $(2x-1)^{2}=-64$
\item $x^{2}+2x-9=0$
\item $4\sqrt{x}=x-5$
\item $n^{2}-10n=200$
\item $x^{2}-x+3=0$
\item $2a^{2}+4a-5=0$
\item $x^{2}+4x+9=0$
\item $\frac{3}{x}+\frac{2}{x+3}=1$
\item \label{10} $\frac{3}{n-2}=\frac{n+5}{4}$
\end{multicols}
Para los problemas \ref{11}--\ref{12}, platee una ecuación y resuélvalo.
\item \label{11} El ala de un avión tiene la forma de un triángulo con ángulos de 30$^{\circ}$ y 60$^{\circ}$ al lado derecho. Si el lado opuesto al ángulo de 30$^{\circ}$ mide 20 pies y el otro lado mide 40 pies, encuentre la medida del lado más largo del ala. Aproxime la respuesta a la centésima más cercana.
\item Encuentre dos números cuya suma es 6 y cuyo producto es 2
\item El área de un cuadrado es numéricamente igual a dos veces su perímetro. Encuentre la longitud del lado del cuadrado
\item Encuentre dos números enteros pares consecutivos cuya suma de sus cuadrado es 164
\item \label{12} El perímetro de un rectángulo es 38 centímetros y su área es 84 cm$^{2}$. Encuentre el ancho y largo del rectángulo.
\end{enumerate}
\end{multicols}


\end{document}
