\documentclass[fleqn,twocolum]{article}
\usepackage[spanish,es-noshorthands]{babel}
\usepackage[utf8]{inputenc} 
\usepackage[papersize={5.5in,8.5in},left=1cm, right=1cm, top=1.5cm, bottom=1.7cm]{geometry}
\usepackage{mathexam}
\usepackage{amsmath}
\usepackage{graphicx}
\ExamClass{\includegraphics[height=16pt]{Images/logo-sed.png} Matemáticas $11^{\circ}$}
\ExamName{``Recomendaciones I, Sustentación''}
\ExamHead{\includegraphics[height=16pt]{Images/logo-colegio.png} IEDAB}
\newcommand{\LineaNombre}{%
\par
\vspace{\baselineskip}
Nombre:\hrulefill \; Curso: \underline{\hspace*{36pt}} \; Fecha: \underline{\hspace*{2.5cm}} \relax
\par}
\let\ds\displaystyle

\begin{document}
\ExamInstrBox{
Respuestas sin justificación procedimental no tendrán puntaje. Escriba sus respuestas en el espacio indicado. Usted tiene 50 minutos.}
\LineaNombre
\begin{enumerate}
   \item Complete la siguiente tabla escribiendo $\in$ o $\not\in$ según el caso:
\begin{center}
   \begin{tabular}{|c|c|c|c|c|c|}
\hline 
Número & 3 & $-5\pi$ & $-5.4$ & $-3.\overline{2}$ & $\sqrt{64}$ \\ 
\hline 
Natural & $\in$ &  & $\not\in$ &  &  \\ 
\hline 
Entero &  &  &  &  &  \\ 
\hline 
Racional &  &  &  &  &  \\ 
\hline 
Irracional &  &  &  &  &  \\ 
\hline 
Real &  &  &  &  &  \\ 
\hline 
\end{tabular} 
\end{center}
 \item Encuentre las fracciones generatrices de los siguientes números
\begin{enumerate}
\item $0,85=$\noanswer
\item $3,5=$\noanswer
\item $2,6\overline{9}=$\noanswer
\end{enumerate}
 \item Efectúe las operaciones siguientes simplificando la respuesta al máximo:
 \begin{enumerate}
 \item $\dfrac{3}{4}-\dfrac{4}{5}=$\noanswer
 \item $\dfrac{\frac{3}{4}-\frac{1}{3}}{\frac{1}{2}-\frac{1}{4}}=$\noanswer 
 \end{enumerate}
 \newpage
\item Señale si son ciertos o falsos los siguientes enunciados:
\begin{enumerate}
\item El número $\dfrac{8}{11}$ es irracional porque tiene una cantidad ilimitada de cifras decimales \underline{\hspace*{20pt}}
\item Todo número real es irracional \underline{\hspace*{20pt}}
\item Todo número natural es irracional \underline{\hspace*{20pt}}
\item $\sqrt[3]{216}$ es un número racional \underline{\hspace*{20pt}}
\item $\sqrt{48}$ es un número irracional \underline{\hspace*{20pt}}
\end{enumerate}
\item Calcule y/o simplifique:
\begin{enumerate}
\item $\sqrt{676}=$ \noanswer
\item $\sqrt[3]{5832}$\noanswer
\item $28-5\sqrt{144}=$\noanswer
\item $\dfrac{3^{12}}{9^{2}}=$\noanswer
\end{enumerate}
\item Escriba como intervalos las siguientes desigualdades y ubíquelos en la recta numérica
\begin{enumerate}
\item $-5\leq x<10$\noanswer
 \newpage
\item $-2<x \leq7$\noanswer
\item $x>4$\noanswer
\end{enumerate}
\item Escriba como desigualdad los siguientes intervalos y ubíquelos en la recta numérica
\begin{enumerate}
\item (--4,6]= \noanswer
\item $(-\infty,2)=$ \noanswer
\end{enumerate}
\item ¿Cuántas baldosas cuadradas de 30 cm de lado, se necesitan para cubrir una superficie de 8,64 $m^{2}$?\noanswer[2in]
\subsection*{Probabilidad}
\item ¿Tablas malas? Así como hay gráficas malas, hay tablas malas, es decir, tablas engañosas y difíciles de leer. Un grupo llamado  Madres Contra Conductores Borrachos (MADD, por sus siglas en inglés) presentó la siguiente tabla referente a 6764 muertos en accidentes de tránsito que ocurrieron en 2002.
\newpage
\begin{center}
\begin{tabular}{lcc}
 &  & Total muertes \\ 
 & Total muertes & relacionadas \\ 
\hspace{15pt} Días festivos 2002 & en tránsito & con alcohol \\ 
\hline 
Víspera de años nuevo & 118 & 45 \\ 
Día de año nuevo & 165 & 94 \\ 
Días festivos de año nuevo & 575 & 301 \\ 
Domingo de super tazón & 147 & 86 \\ 
Día de San Patricio & 158 & 72 \\ 
Conmemoración de los caídos & 491 & 237 \\ 
Cuatro de julio & 683 & 330 \\ 
Fin de semana de día del trabajo & 541 & 300 \\ 
Halloween & 268 & 109 \\ 
Día de gracias & 543 & 265 \\ 
Día de gracias-año nuevo & 4019 & 1561 \\ 
Navidad & 130 & 68 \\ 
Víspera de año nuevo 2002 & 123 & 57 \\ 
\hline 
\end{tabular}
\end{center}
\begin{enumerate}
\item Los totales de columna no están incluidos porque
serían valores que carecen de sentido. Examine la
tabla y explique por qué.\noanswer
\item Seleccione los días festivos apropiados que no se
traslapan (columna 1) y verifique el número total
de 6764 muertos en accidentes de tránsito para
2002.\noanswer
\item Usando los días festivos seleccionados en la parte b, encuentre el número total de muertos en accidentes de tránsito relacionados con alcohol en
días festivos en 2002.\noanswer
\item Describa cómo organizaría esta tabla para hacerla que tenga sentido.\noanswer
\end{enumerate}
\end{enumerate}
\end{document}
