\documentclass[letterpaper,fleqn]{article}
\usepackage[spanish,es-noshorthands]{babel}
\usepackage[utf8]{inputenc} 
\usepackage[left=1cm, right=1cm, top=1.5cm, bottom=1.7cm]{geometry}
\usepackage{mathexam}
\usepackage{amsmath}
\usepackage{graphicx}
\usepackage{tikz,pgf}
\usepackage{multicol}

\ExamClass{\includegraphics[height=16pt]{Images/logo-sed.png} Matemáticas $11^{\circ}$}
\ExamName{Nivelación 2015}
\ExamHead{\includegraphics[height=16pt]{Images/logo-colegio.png} IEDAB}
\newcommand{\LineaNombre}{%
\par
\vspace{\baselineskip}
Nombre:\hrulefill \; Curso: \underline{\hspace*{48pt}} \; Fecha: \underline{\hspace*{2.5cm}} \relax
\par}
\let\ds\displaystyle

\begin{document}
\ExamInstrBox{
Respuesta sin justificar mediante procedimiento no será tenida en cuenta en la calificación. Escriba sus respuestas en el espacio indicado. Tiene 55 minutos para contestar esta prueba.}
\LineaNombre
\begin{enumerate}
 \item La luz tarda aproximadamente 8 minutos en recorrer la distancia del sol a la tierra. Si la luz recorre 300\,000 kilómetros por segundo, encuentre la distancia aproximada que hay entre el sol y la tierra. Exprese su respuesta en notación científica \noanswer
 \item Sea 
$
f(x)= \left\{ \begin{array}{lcl}
4 & \mbox{ si } & x\leq 2 \\
x-3 & \mbox{ si } & x>2
\end{array}
\right.
$
\begin{enumerate}
\item Evalúe 
\begin{enumerate}
\begin{multicols}{2}
\item $f(0)=$ \noanswer
\item $f(1)=$ \noanswer
\item $f(2)=$ \noanswer
\item $f(3)=$ \noanswer
\item $f(4)=$ \noanswer
\end{multicols}
\end{enumerate}
\item Haga la gráfica de $f$
\begin{center}
\begin{tikzpicture}[scale=.9]
\draw[dotted,style=help lines] (-2,-2) grid (5,5);
\draw[<->] (-2.2,0)--(5.3,0) node[right] {$x$};
\foreach \x in {1} \draw[shift={(\x,0)},color=black] (0pt,2pt) -- (0pt,-2pt) node[below] {\footnotesize $\x$};
\draw[<->](0,-2.2)--(0,5.2)node[left]{$y$};
\foreach \y in {1} \draw[shift={(0,\y)},color=black] (2pt,0)--(-2pt,0) node[left]{\footnotesize $\y$};
\end{tikzpicture}
\end{center}
\item Encuentre los siguientes límites
\begin{enumerate}
\begin{multicols}{2}
\item $\displaystyle{\lim_{x\rightarrow 2^{-}}f(x)}=$ \noanswer
\item $\displaystyle{\lim_{x\rightarrow 2^{+}}f(x)}=$ \noanswer
\item $\displaystyle{\lim_{x\rightarrow 2}f(x)}=$ \noanswer
\item $\displaystyle{\lim_{x\rightarrow 3}f(x)}=$ \noanswer
\end{multicols}
\end{enumerate}
\end{enumerate}
\item Sea la función cuadrática $f(x)=x^{2}-4x+5$
\begin{enumerate}
\item Exprese $f$ en la forma estándar $f(x)-k=(x-h)^{2}$\noanswer
\newpage
\item Encuentre el mínimo o máximo valor de $f$\noanswer[20pt]
\item Haga su gráfica
\begin{center}
\begin{tikzpicture}[scale=.9]
\draw[dotted,style=help lines] (-1,-1) grid (5,5);
\draw[<->] (-1.2,0)--(5.3,0) node[right] {$x$};
\foreach \x in {1} \draw[shift={(\x,0)},color=black] (0pt,2pt) -- (0pt,-2pt) node[below] {\footnotesize $\x$};
\draw[<->](0,-1.2)--(0,5.2)node[left]{$y$};
\foreach \y in {1} \draw[shift={(0,\y)},color=black] (2pt,0)--(-2pt,0) node[left]{\footnotesize $\y$};
\end{tikzpicture}
\end{center}
\item Encuentre el intervalo en el cual $f$ es creciente y el intervalo en el cual $f$ es decreciente \noanswer[30pt]
\end{enumerate}
\item Para cada sucesión, encuentre los 5 primeros términos y el término n-ésimo $a_{n}$
\begin{enumerate}
\item La progresión aritmética cuyo primer término es $a_{n}=\frac{1}{2}$ y su diferencia común $d=3$ \noanswer[50pt]
\item La progresión geométrica cuyo primer término es $a_{n}=12$ y cuyo razón geométrica es $\frac{5}{6}$ \noanswer[50pt]
\end{enumerate}
\item Dada la función $g(x)=2x^{2}-3x$ 
\begin{enumerate}
\item Encuentre la pendiente de la recta tangente al punto (2,2)\noanswer
\item Encuentre la ecuación de la recta tangente al punto (2,2)\noanswer
\end{enumerate}
\item De una caja que contiene 10 canicas rojas, 30 blancas, 20 azules y 15 anaranjadas, se extrae una canica. Hallar la probabilidad de que la canica extraída sea:
\begin{enumerate}
\begin{multicols}{2}
\item anaranjada o roja \noanswer
\item no azul \noanswer
\end{multicols}
\end{enumerate}
 \end{enumerate}
\end{document}
