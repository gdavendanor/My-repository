\documentclass[10pt,a4paper]{article}
\usepackage[utf8]{inputenc}
\usepackage[spanish]{babel}
\usepackage{amsmath}
\usepackage{amsfonts}
\usepackage{amssymb}
\usepackage{graphicx}
\author{Germán Avendaño Ramírez}
\title{Ejercicio}
\begin{document}
\section*{3.4.1 Ejercicios}
\begin{enumerate}
\item[f] $2\sqrt[3]{x}-\sqrt[3]{x^2}+8=0$
\end{enumerate}
\subsection*{Soluci\'on}
Expreso lo mismo usando exponentes:
\[2x^{1/3}-x^{2/3}+8=0\]
Ahora ordeno de mayor a menor exponente, para asimilarla a una cuadrática:
\begin{align*}
-x^{2/3}+2x^{1/3}+8&=0 &\\
x^{2/3}-2x^{1/3}-8&=0 & \mbox{Multiplicando por} -1 \mbox{la ecuación}
\end{align*}
\end{document}