\documentclass[letterpaper,fleqn]{article}
\usepackage[spanish,es-noshorthands]{babel}
\usepackage[utf8]{inputenc} 
\usepackage[left=1cm, right=1cm, top=1.5cm, bottom=1.7cm]{geometry}
\usepackage{mathexam}
\usepackage{amsmath}
\usepackage{graphicx}

\ExamClass{\includegraphics[height=16pt]{Images/logo-sed.png} Aritmética $6^{\circ}$}
\ExamName{Sustentación Recomendaciones III}
\ExamHead{\includegraphics[height=16pt]{Images/logo-colegio.png} IEDAB}
\newcommand{\LineaNombre}{%
\par
\vspace{\baselineskip}
Nombre:\hrulefill \; Curso: \underline{\hspace*{48pt}} \; Fecha: \underline{\hspace*{2.5cm}} \relax
\par}
\let\ds\displaystyle

\begin{document}
\ExamInstrBox{
Respuesta sin justificar mediante procedimiento no será tenida en cuenta en la calificación. Escriba sus respuestas en el espacio indicado. Tiene 45 minutos para contestar esta prueba.}
\LineaNombre
\begin{enumerate}
 \item Dados los números 96, 112 y 104:
 \begin{enumerate}
 \item Escríbalos como producto de factores primos\noanswer
 \item ¿Cuáles son sus factores comunes?\noanswer
 \item ¿Cuál es el máximo común divisor y el mínimo común múltiplo de ellos.\noanswer
 \end{enumerate}
 \item En un campamento hay 83 niños y niñas. ¿Qué problema tienen para hacer equipos con el mismo número de integrantes?\noanswer
 \item Calcule por el método que prefiera el máximo común divisor y el mínimo común múltiplo de los siguientes grupos de números:
 \begin{enumerate}
 \item 108, 198 y 136\noanswer
 \item 100, 84 y 91\noanswer
 \item 90, 216 y 198\noanswer
 \end{enumerate}
 \item 
 \end{enumerate}

\end{document}
