\documentclass[10pt,twoside]{article}
\usepackage[utf8]{inputenc}
\usepackage{amsmath}
\usepackage{amsfonts}
\usepackage{amssymb}
\usepackage[spanish,es-noshorthands]{babel}
\usepackage[T1]{fontenc}
\usepackage{lmodern}
\usepackage{graphicx,hyperref}
\usepackage{tikz,pgf}
\usepackage{multicol}
\usepackage{subfig}
\usepackage[papersize={6.5in,8.5in},width=5.5in,height=7in]{geometry}
\usepackage{fancyhdr}
\pagestyle{fancy}
\fancyhead[LE]{\includegraphics[height=12pt]{Images/logo-colegio.png} \'Algebra $8^{\circ}$}
\fancyhead[RE]{}
\fancyhead[RO]{\textit{Germ\'an Avenda\~no Ram\'irez, Lic. U.D., M.Sc. U.N.}}
\fancyhead[LO]{}

\author{Germ\'an Avenda\~no Ram\'irez, Lic. U.D., M.Sc. U.N.}
\title{\begin{minipage}{.2\textwidth}
\includegraphics[height=1.75cm]{Images/logo-colegio.png}\end{minipage}
\begin{minipage}{.55\textwidth}
\begin{center}
Taller 03, Introducción a las expresiones algebraicas\\
Álgebra $8^{\circ}$
\end{center}
\end{minipage}\hfill
\begin{minipage}{.2\textwidth}
\includegraphics[height=1.75cm]{Images/logo-sed.png} 
\end{minipage}}
\date{}
\thispagestyle{plain}
\begin{document}
\maketitle
Nombre: \hrulefill Curso: \underline{\hspace*{44pt}} Fecha: \underline{\hspace*{2.5cm}}
 \section*{Nivel I}
 \begin{enumerate}
 \item Expresa, indicando las operaciones que debes hacer y calculando el resultado: 
\begin{center}
\begin{tabular}{|l|c|c|}
\hline 
\hspace*{.5cm}Expresión & Resultado & Operación indicada \\ 
\hline 
El doble de 5 es \ldots & 10 & $2\cdot 5$ \\ 
\hline 
La mitad de 8 es \ldots &  &  \\ 
\hline 
El triple de 9 es \ldots &  &  \\ 
\hline 
El cuadrado de 7 es \ldots &  & \\ 
\hline 
La raíz cuadrada de 25 es \ldots &  &  \\ 
\hline 
La suma de 8 y 5 es \ldots &  &  \\ 
\hline 
La diferencia entre 10 y 7 es \ldots &  &  \\ 
\hline 
El producto de 4 y 10 es \ldots &  &  \\ 
\hline 
El cociente entre 24 y 8 es \ldots &  &  \\ 
\hline 
\end{tabular} 
\end{center}
\item Expresa en lenguaje algebraico estas expresiones:
\begin{enumerate}
\item El doble de un número $n$ es \ldots Respuesta: $2\cdot n$
\item El doble de un número cualquiera es \ldots
\item La mitad de un número $p$ es \ldots
\item La mitad de un número cualquiera es \ldots
\item La suma de dos números $a$ y $b$ es \ldots
\item La suma de dos números cualesquiera es \ldots
\item La diferencia entre dos números $m$ y $h$ es \ldots
\item La diferencia entre dos números cualesquiera es \ldots
\item El producto de dos números $c$ y $d$ es \ldots
\item El producto de dos números cualesquiera es \ldots
\item El cociente entre los números $x$ y $m$ es \ldots
\item El cociente entre dos números cualesquiera es \ldots
\item El cuadrado de un número $p$ es \ldots
\item El cuadrado de un número cualquiera es \ldots
\item La raíz cuadrada de un número $h$ es \ldots
\item La raíz cuadrada de un número cualquiera es \ldots
\end{enumerate}
\item Expresa por medio de lenguaje algebraico estas expresiones:
\begin{enumerate}
\item El cuadrado de un número disminuido en 25
\item El siguiente número entero del número entero $p$
\item El número anterior al número entero $p$
\item El cuadrado de un número más el cuadrado de otro número
\item La mitad de un número menos el tripe de otro número
\item La diferencia entre el doble de un número y la mitad de otro número

\end{enumerate}
\end{enumerate}



\end{document}
