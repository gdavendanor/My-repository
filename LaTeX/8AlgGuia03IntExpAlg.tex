\documentclass[10pt,twoside]{article}
\usepackage[utf8]{inputenc}
\usepackage{amsmath}
\usepackage{amsfonts}
\usepackage{amssymb}
\usepackage[spanish,es-noshorthands]{babel}
\usepackage[T1]{fontenc}
\usepackage{lmodern}
\usepackage{graphicx,hyperref}
\usepackage{tikz,pgf}
\usepackage{multicol}
\usepackage{subfig}
\usepackage[papersize={6.5in,8.5in},width=5.5in,height=7in]{geometry}
\usepackage{fancyhdr}
\pagestyle{fancy}
\fancyhead[LE]{\includegraphics[height=12pt]{Images/logo-colegio.png} \'Algebra $8^{\circ}$}
\fancyhead[RE]{}
\fancyhead[RO]{\textit{Germ\'an Avenda\~no Ram\'irez, Lic. U.D., M.Sc. U.N.}}
\fancyhead[LO]{}

\author{Germ\'an Avenda\~no Ram\'irez, Lic. U.D., M.Sc. U.N.}
\title{\begin{minipage}{.2\textwidth}
\includegraphics[height=1.75cm]{Images/logo-colegio.png}\end{minipage}
\begin{minipage}{.55\textwidth}
\begin{center}
Taller 03, Introducción a las expresiones algebraicas\\
Álgebra $8^{\circ}$
\end{center}
\end{minipage}\hfill
\begin{minipage}{.2\textwidth}
\includegraphics[height=1.75cm]{Images/logo-sed.png} 
\end{minipage}}
\date{}
\thispagestyle{plain}
\begin{document}
\maketitle
Nombre: \hrulefill Curso: \underline{\hspace*{44pt}} Fecha: \underline{\hspace*{2.5cm}}
 \section*{Nivel I}
 \begin{enumerate}
 \item Expresa, indicando las operaciones que debes hacer y calculando el resultado: 
\begin{center}
\begin{tabular}{|l|c|c|}
\hline 
\hspace*{.5cm}Expresión & Resultado & Operación indicada \\ 
\hline 
El doble de 5 es \ldots & 10 & $2\cdot 5$ \\ 
\hline 
La mitad de 8 es \ldots &  &  \\ 
\hline 
El triple de 9 es \ldots &  &  \\ 
\hline 
El cuadrado de 7 es \ldots &  & \\ 
\hline 
La raíz cuadrada de 25 es \ldots &  &  \\ 
\hline 
La suma de 8 y 5 es \ldots &  &  \\ 
\hline 
La diferencia entre 10 y 7 es \ldots &  &  \\ 
\hline 
El producto de 4 y 10 es \ldots &  &  \\ 
\hline 
El cociente entre 24 y 8 es \ldots &  &  \\ 
\hline 
\end{tabular} 
\end{center}
\item Expresa en lenguaje algebraico estas expresiones:
\begin{enumerate}
\item El doble de un número $n$ es \ldots \fbox{$2n$}
\item El doble de un número cualquiera es \ldots
\item La mitad de un número $p$ es \ldots
\item La mitad de un número cualquiera es \ldots
\item La suma de dos números $a$ y $b$ es \ldots
\item La suma de dos números cualesquiera es \ldots
\item La diferencia entre dos números $m$ y $h$ es \ldots
\item La diferencia entre dos números cualesquiera es \ldots
\item El producto de dos números $c$ y $d$ es \ldots
\item El producto de dos números cualesquiera es \ldots
\item El cociente entre los números $x$ y $m$ es \ldots
\item El cociente entre dos números cualesquiera es \ldots
\item El cuadrado de un número $p$ es \ldots
\item El cuadrado de un número cualquiera es \ldots
\item La raíz cuadrada de un número $h$ es \ldots
\item La raíz cuadrada de un número cualquiera es \ldots
\end{enumerate}
\item Expresa por medio de lenguaje algebraico estas expresiones:
\begin{enumerate}
\item El cuadrado de un número disminuido en 25: \fbox{$n^{2}-25$}
\item El siguiente número del número entero $p$
\item El número anterior al número entero $p$
\item El cuadrado de un número más el cuadrado de otro número
\item La mitad de un número menos el tripe de otro número
\item La diferencia entre el doble de un número y la mitad de otro número
\end{enumerate}
\item Traduce a lenguaje ordinario estas expresiones algebraicas:
\begin{enumerate}
\item \underline{Ejemplo:} $2a$: \; El doble o duplo de un número $a$
\begin{multicols}{4}
 \item $\frac{b}{2}$ \item $n^{2}$ \item $a+b$ \item $m-p$ \item $a\cdot b\cdot c$
\item $2x+3y$ \item $a^{2}+b^{2}$ \item $5c+2$
\end{multicols}
\end{enumerate}
\item Copia y completa la tabla:
\begin{center}
\begin{tabular}{|c|c|c|c|c|}
\hline 
$a$ & $b$ & $c$ & Expresi\'on algebraica & Valor numérico \\ 
\hline 
$2$ & $3$ & $4$ & $a+b-c$ & $2+3-4=5-4=1$ \\ 
\hline 
$-1$ & 5 & $-2$ & $2\cdot a+3\cdot b +4\cdot c$ &  \\ 
\hline 
6 & $-2$ & 3 & $\frac{a}{2}+5\cdot b-c$ &  \\ 
\hline 
$-5$ & 4 & $-6$ & $-a-3b+\frac{c}{3}$ &  \\ 
\hline 
1 & $-3$ & 7 & $2(a+b+c)$ &  \\ 
\hline 
\end{tabular} 
\end{center}
\item Contesta a estas preguntas:
\begin{enumerate}
\item ¿A qué se llama expresión algebraica?
\item ¿Qué es un monomio?
\item ¿Cuáles son las partes de todo monomio?
\item ¿A qué se llama Coeficiente?
\item ¿Qué se entiende como Parte Literal?
\item ¿Qué es el Grado?
\item ¿Cuándo dos monomios son semejantes?
\end{enumerate}
\item Copia y completa la tabla:
\begin{center}
\begin{tabular}{|c|c|c|c|}
\hline 
Monomio & Coeficiente & Parte literal & Grado \\ 
\hline 
$2x^{3}$ & 2 & x & 3 \\ 
\hline 
$-5y^{6}$ &  &  &  \\ 
\hline 
$7b^{8}$ &  &  &  \\ 
\hline 
$-8m^{5}$ &  &  &  \\ 
\hline 
$x^{3}$ &  &  &  \\ 
\hline 
\end{tabular} 
\end{center}
\item Agrupa y reduce los monomios semejantes:
\begin{enumerate}
\begin{multicols}{2}
\item $2x + 5x - 7x + 8x$
\item $6m^{2}-9m^{2}+7m^{2}-m^{2}$
\item $4y-7y^{2}+8y-5y^{2}+6y$
\item $5b^{2}-6b+b^{2}-b+7b-3b^{2}$
\end{multicols}
\end{enumerate}
\item Realiza estas operaciones con monomios:
\begin{enumerate}
\begin{multicols}{2}
\item $3x^{2}\cdot 5x^{3}$
\item $-4m^{5}\cdot 5m^{3}$
\item $-2m^{2}\cdot (-3m^{5})$
\item $b^{2}\cdot b^{5}$
\item $(-c)^{2}\cdot (-c)^{4}$
\item $4\cdot (2x^{3})$
\item $-5\cdot (-3x^{4})$
\end{multicols}
\end{enumerate}
\item Copia y completa la tabla:
\begin{center}
\begin{tabular}{|l|c|c|}
\hline 
\hspace*{.5cm} Polinomio & Términos del polinomio (monomios) & Grado \\ 
\hline 
$3m^{2}-5m+7$ & $3m^{2}$; \quad $-5m$; \quad $7$ & 2 \\ 
\hline 
$-2x^{3}+6x^{2}-5x+3$ &  &  \\ 
\hline 
$h^{2}-7+5h^{6}$ &  &  \\ 
\hline 
$b^{5}-2+5b^{4}$ &  &  \\ 
\hline 
\end{tabular} 
\end{center}
\item Copia y completa la tabla, haciendo las operaciones fuera de ella:
\begin{center}
\begin{tabular}{|l|l|c|c|c|c|}
\hline 
\hspace*{.5cm} A & \hspace*{.5cm} B & $A+B$ & $A-B$ & $2\cdot A$ & $-3\cdot B$ \\ 
\hline 
$x+5$ & $x+3$ & $2x+8$ &  &  &  \\ 
\hline 
$3x^{2}+2x+5$ & $2x^{2}-6x-1$ &  &  &  &  \\ 
\hline 
$-4m+5m^{2}+6$ & $-4+6m-m^{2}$ &  &  &  &  \\ 
\hline 
$2b^{3}-3b+5b^{2}-4$ & $-2b+5b^{2}-b^{3}+3$ &  &  &  &  \\ 
\hline 
\end{tabular} 
\end{center}
En la tabla resulta $2x+8$, ya que en la primera fila, $A=x+5$ y $B=x+3$, por lo tanto \[A+B=(x+5)+(x+3)=(x+x)+(5+3)=2x+8\]
\item Calcula el valor numérico de estas expresiones algebraicas, dando un valor positivo y otro negativo a las letras que aparecen en ellas:
\begin{enumerate}
\begin{multicols}{2}
\item $n+n=2n$ \item $b\cdot b=b^{2}$
\end{multicols}
\end{enumerate}
¿Cómo son los resultados que se obtienen?\\
¿Serán estas expresiones algebraicas unas identidades?. Justifica tu respuesta. (Consulta que es una identidad en matemáticas)
\item Copia y completa la tabla:
\begin{center}
\begin{tabular}{|l|c|c|c|c|}
\hline 
\hspace*{0.5cm} Ecuación & Primer miembro & Segundo miembro & Términos & Incógnita \\ 
\hline 
$2x+3=7$ & $2x+3$ & 7 & $2x$; \quad $3$; \quad $7$& $x$  \\ 
\hline 
$4m-5=9m$ &  &  &  &  \\ 
\hline 
$3b-2=3-5b$ &  &  &  &  \\ 
\hline 
$2h+5-4=3h-8$ &  &  &  &  \\ 
\hline 
\end{tabular} 
\end{center}
\item Resuelve estas ecuaciones:
\begin{enumerate}
\item 
\begin{enumerate}
\begin{multicols}{4}
\item $x+5=8$
\item $x-3=7$
\item $6=4+x$
\item $-6=x-2$
\end{multicols}
\end{enumerate}
\item
\begin{enumerate}
\begin{multicols}{4}
\item $3b=12$
\item $2b=-6$
\item $20=5b$
\item $-8=-2b$
\end{multicols}
\end{enumerate}
\item 
\begin{enumerate}
\begin{multicols}{4}
\item $\dfrac{m}{2}=1$
\item $\dfrac{3m}{4}=3$
\item $\dfrac{m}{-5}=8$
\item $\dfrac{-4m}{3}=-8$
\end{multicols}
\end{enumerate}
\item 
\begin{enumerate}
\begin{multicols}{4}
\item $2p+3=7$
\item $6p-5=7$
\item $4=2p-2$
\item $10=3p+1$
\end{multicols}
\end{enumerate}
\end{enumerate}
\item Resuelve estas ecuaciones:
\begin{enumerate}
\begin{multicols}{2}
\item $4x+5=2x+9$
\item $3x-1=x+5$
\item $3x-2+4x=6x-5$
\item $2\cdot (x+5)=8$
\item $-2\cdot (x-4)=3\cdot (x-6)$
\item $6\cdot (3x-2)=12x+3\cdot (x-10)$
\item $3m-2\cdot (m+1)=3\cdot (m-1)-1$
\item $5\cdot (m+2)-3\cdot (m-1)=5\cdot (m+3)$
\item $\dfrac{b}{3}-\dfrac{b}{12}+\dfrac{1}{4}=1$
\item $\dfrac{1}{2}-\dfrac{3b}{10}-\dfrac{b}{5}+30=0$
\item $\dfrac{b}{4}+\dfrac{5}{2}-\dfrac{b}{6}=5$
\end{multicols}
\end{enumerate}
\end{enumerate}



\end{document}
