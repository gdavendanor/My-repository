\documentclass[10pt,twoside]{article}
\usepackage[utf8]{inputenc}
\usepackage{amsmath,amsfonts,amssymb,latexsym}
\usepackage[spanish,es-noshorthands]{babel}
\usepackage[T1]{fontenc}
\usepackage{lmodern}
\usepackage{graphicx,hyperref}
\usepackage{pgf,tikz}
\usepackage{multicol}
\usepackage{subfig}
\usepackage[papersize={6.5in,8.5in},width=5.5in,height=7in]{geometry}
\usepackage{fancyhdr}
\pagestyle{fancy}
\fancyhead[LE]{\url{http://germandario.byethost4.com}}
\fancyhead[RE]{}
\fancyhead[RO]{\textit{Germ\'an Dar\'io Avenda\~no Ram\'irez, Lic - M.Sc.}}
\fancyhead[LO]{}
\usetikzlibrary{arrows}

\author{Germ\'an Dar\'io Avenda\~no Ram\'irez, Lic. - M.Sc.}
\title{\begin{minipage}{0.15\textwidth}\includegraphics[height=1.7cm]{Images/logo-colegio.png}
\end{minipage}\hfill \begin{minipage}{0.85\textwidth}\begin{center}
Tema\\Guía materia $0^{\circ}$\end{center}
\end{minipage}}
\date{}

\begin{document}
\maketitle
Nombre: \hrulefill Curso: \underline{\hspace{1cm}}  Fecha: \underline{\hspace{2cm}}\\

%%Rectas en el plano
\begin{tikzpicture}
\draw[help lines](0,0) grid (3,3);
\draw (0,0) --(1,2) -- (2,3) -- (1,0);
\draw (3,0) -- (1.5,0.5);
\end{tikzpicture}

%%Recta numérica
\begin{tikzpicture}[>=stealth]
\draw[|->] (0,0) node[below]{0}-- (10,0) node[below]{10};
\draw [thick] (1,-.1) node[below]{1} -- (1,0.1);
\draw [thick] (2,-.1) node[below]{2} -- (2,0.1);
\draw [thick] (3,-.1) node[below]{3} -- (3,0.1);
\draw [thick] (4,-.1) node[below]{4} -- (4,0.1);
\draw [thick] (5,-.1) node[below]{5} -- (5,0.1);
\draw [thick] (6,-.1) node[below]{6} -- (6,0.1);
\draw [thick] (7,-.1) node[below]{7} -- (7,0.1);
\end{tikzpicture}

\begin{tikzpicture}[>=stealth]
\draw [->] (-1.5,0) -- (1.5,0);
\draw [->] (0,-1.5) -- (0,1.5);
\shadedraw (0.5,0.5) circle (0.5cm);

%Relleno
\filldraw[fill=red,even odd rule]
(-1,-1) rectangle (0,0)
(-0.5,-0.5) circle (0.4cm);
\draw[->] (-0.9,-0.2) -- +(0,1)
[above] node{Relleno};
\end{tikzpicture}

%% Cubo%%
\begin{tikzpicture}
\draw (0,0,0)--(1,0,0)--(1,1,0)--(0,1,0)--cycle;
\draw (0,0,1)--(1,0,1)--(1,1,1)--(0,1,1)--cycle;
\draw (0,0,0) -- (0,0,1);
\draw (1,0,0) -- (1,0,1);
\draw (1,1,0) -- (1,1,1);
\draw (0,1,0) -- (0,1,1);
\end{tikzpicture}

%%ploting%%%
\begin{tikzpicture}
\draw [<->](0,-1)--(0,3);
\draw [<->](-1,0)--(4,0);
\draw [help lines] (-1,-1) grid (4,3);
\draw [red] plot [domain=0.5:4] (\x, 1/\x);
\fill[color=gray!20] (1,0)--(1,1)-- plot [domain=1:2] (\x,1/\x)--(2,0)--cycle;
\end{tikzpicture}

%%%animaplano%%
\begin{center}
\begin{tikzpicture}[scale=1.2]
\fill (0,0)circle (.2ex);
\fill (1,0)circle (.2ex);
\fill (2,0)circle (.2ex);
\fill (3,0)circle (.2ex);
\fill (4,0)circle (.2ex);
\fill (5,0)circle (.2ex);
\fill (6,0)circle (.2ex);
\fill (7,0)circle (.2ex);
\fill (8,0)circle (.2ex);
\fill (9,0)circle (.2ex);
\fill (0,1)circle (.2ex);
\fill (1,1)circle (.2ex);
\fill (1,2)circle (.2ex);
\fill (1,3)circle (.2ex);
\fill (1,4)circle (.2ex);
\fill (1,5)circle (.2ex);
\fill (1,6)circle (.2ex);
\fill (1,7)circle (.2ex);
\fill (1,8)circle (.2ex);
\fill (1,9)circle (.2ex);
\fill (2,1)circle (.2ex);
\fill (2,2)circle (.2ex);
\fill (2,3)circle (.2ex);
\fill (2,4)circle (.2ex);
\fill (2,5)circle (.2ex);
\fill (2,6)circle (.2ex);
\fill (2,7)circle (.2ex);
\fill (2,8)circle (.2ex);
\fill (2,9)circle (.2ex);
\fill (0,2)circle (.2ex);
\fill (0,3)circle (.2ex);
\fill (0,4)circle (.2ex);
\fill (0,5)circle (.2ex);
\fill (0,6)circle (.2ex);
\fill (0,7)circle (.2ex);
\fill (0,8)circle (.2ex);
\fill (0,9)circle (.2ex);
\fill (3,1)circle (.2ex);
\fill (3,2)circle (.2ex);
\fill (3,3)circle (.2ex);
\fill (3,4)circle (.2ex);
\fill (3,5)circle (.2ex);
\fill (3,6)circle (.2ex);
\fill (3,7)circle (.2ex);
\fill (3,8)circle (.2ex);
\fill (3,9)circle (.2ex);
\fill (4,1)circle (.2ex);
\fill (4,2)circle (.2ex);
\fill (4,3)circle (.2ex);
\fill (4,4)circle (.2ex);
\fill (4,5)circle (.2ex);
\fill (4,6)circle (.2ex);
\fill (4,7)circle (.2ex);
\fill (4,8)circle (.2ex);
\fill (4,9)circle (.2ex);
\fill (5,1)circle (.2ex);
\fill (5,2)circle (.2ex);
\fill (5,3)circle (.2ex);
\fill (5,4)circle (.2ex);
\fill (5,5)circle (.2ex);
\fill (5,6)circle (.2ex);
\fill (5,7)circle (.2ex);
\fill (5,8)circle (.2ex);
\fill (5,9)circle (.2ex);
\fill (6,1)circle (.2ex);
\fill (6,2)circle (.2ex);
\fill (6,3)circle (.2ex);
\fill (6,4)circle (.2ex);
\fill (6,5)circle (.2ex);
\fill (6,6)circle (.2ex);
\fill (6,7)circle (.2ex);
\fill (6,8)circle (.2ex);
\fill (6,9)circle (.2ex);
\fill (7,1)circle (.2ex);
\fill (7,2)circle (.2ex);
\fill (7,3)circle (.2ex);
\fill (7,4)circle (.2ex);
\fill (7,5)circle (.2ex);
\fill (7,6)circle (.2ex);
\fill (7,7)circle (.2ex);
\fill (7,8)circle (.2ex);
\fill (7,9)circle (.2ex);
\fill (8,1)circle (.2ex);
\fill (8,2)circle (.2ex);
\fill (8,3)circle (.2ex);
\fill (8,4)circle (.2ex);
\fill (8,5)circle (.2ex);
\fill (8,6)circle (.2ex);
\fill (8,7)circle (.2ex);
\fill (8,8)circle (.2ex);
\fill (8,9)circle (.2ex);
\fill (9,1)circle (.2ex);
\fill (9,2)circle (.2ex);
\fill (9,3)circle (.2ex);
\fill (9,4)circle (.2ex);
\fill (9,5)circle (.2ex);
\fill (9,6)circle (.2ex);
\fill (9,7)circle (.2ex);
\fill (9,8)circle (.2ex);
\fill (9,9)circle (.2ex);
\node[left] at (0,8) {11};
\node[left] at (0,7) {21};
\node[left] at (0,6){31};
\node[left] at (0,5){41};
\node[left] at (0,4){51};
\node[left] at (0,3){61};
\node[left] at (0,2){71};
\node[left] at (0,1){81};
\node[left] at (0,0){91};
\node[above] at (0,9){1};
\node[above] at (1,9){2};
\node[above] at (2,9){3};
\node[above] at (3,9){4};
\node[above] at (4,9){5};
\node[above] at (5,9){6};
\node[above] at (6,9){7};
\node[above] at (7,9){8};
\node[above] at (8,9){9};
\node[above] at (9,9){10};
\end{tikzpicture}
\end{center}

\begin{tikzpicture}[line cap=round,line join=round,>=triangle 45,x=1.0cm,y=1.0cm]
\draw[<->,color=black] (-4.3,0) -- (7.4,0)node[right]{$ x $};
\foreach \x in {-4,-3,-2,-1,1,2,3,4,5,6,7}
\draw[shift={(\x,0)},color=black] (0pt,2pt) -- (0pt,-2pt) node[below] {\footnotesize $\x$};
\draw[<->,color=black] (0,-3.2) -- (0,6.3);
\foreach \y in {-3,-2,-1,1,2,3,4,5,6}
\draw[shift={(0,\y)},color=black] (2pt,0pt) -- (-2pt,0pt) node[left] {\footnotesize $\y$};
\draw[color=black] (0pt,-10pt) node[right] {\footnotesize $0$};
\clip(-4.3,-3.2) rectangle (7.4,6.3);
\draw[smooth,samples=100,domain=-4.3:7.4] plot(\x,{(\x)*sin(((\x))*180/pi)});
\end{tikzpicture}%Uncomment next line if XeTeX is used
%\def\pgfsysdriver{pgfsys-xetex.def}

\begin{center}
\usetikzlibrary{arrows}
\baselineskip=10pt
\hsize=6.3truein
\vsize=8.7truein
\tikzpicture[scale=.5,line cap=round,line join=round,>=triangle 45,x=1.0cm,y=1.0cm]
\draw[->,color=black] (-4.22,0) -- (3.82,0);
\foreach \x in {-4,-2,2}
\draw[shift={(\x,0)},color=black] (0pt,2pt) -- (0pt,-2pt) node[below] {$\x$};
\draw[->,color=black] (0,-2.8) -- (0,1.66);
\foreach \y in {-2}
\draw[shift={(0,\y)},color=black] (2pt,0pt) -- (-2pt,0pt) node[left] {$\y$};
\draw[color=black] (0pt,-10pt) node[right] {$0$};
\clip(-4.22,-2.8) rectangle (3.82,1.66);
\draw[smooth,samples=100,domain=-4.220000000000001:3.8200000000000007] plot(\x,{sin(((\x))*180/pi)});
\endtikzpicture
\end{center}

%Arbol
\tikz [font=\footnotesize,
grow=right, level 1/.style={sibling distance=6em},
level 2/.style={sibling distance=1em}, level distance=5cm]
\node {Computational Complexity} % root
child { node {Computational Problems}
child { node {Problem Measures} }
child { node {Problem Aspects} }
child { node {Problem Domains} }
child { node {Key Problems} }
}
child { node {Computational Models}
child { node {Turing Machines} }
child { node {Random-Access Machines} }
child { node {Circuits} }
child { node {Binary Decision Diagrams} }
child { node {Oracle Machines} }
child { node {Programming in Logic} }
};

\end{document}
