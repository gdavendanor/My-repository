\documentclass[fleqn,10pt]{article}
\usepackage[spanish,es-noshorthands]{babel}
\usepackage[utf8]{inputenc} 
\usepackage[papersize={6.5in,8.5in},left=1cm, right=1cm, top=1.7cm, bottom=1.5cm]{geometry}
\usepackage{mathexam}
\usepackage{amsmath}
\usepackage{pgf,tikz}
\usepackage{multicol}

\ExamClass{\includegraphics[height=16pt]{Images/logo-sed.png} Cálculo $11^{\circ}$}
\ExamName{Límites Introducción}
\ExamHead{\includegraphics[height=16pt]{Images/logo-colegio.png} IEDAB}
\newcommand{\LineaNombre}{%
\par
\vspace{\baselineskip}
Nombre:\hrulefill \; Curso: \underline{\hspace*{48pt}} \; Fecha: \underline{\hspace*{2.5cm}} \relax
\par}

\let\ds\displaystyle

\begin{document}
\ExamInstrBox{
Debe mostrar los procedimientos! Respuestas sin el procedimiento requerido, no tendrán puntuación. Puede usar calculadora, pero no se aceptan préstamos de éstas durante el examen.}
\ExamNameLine
\begin{enumerate}
   \item Calcule los siguientes límites, usando una tabla de valores. Recuerde que debe escoger valores cercanos donde no se sugieren.
      \begin{enumerate}
	 \item $\ds{\lim_{x\rightarrow5}\frac{x^2-5x}{x-5}}=$
\begin{center}
   \begin{tabular}{|c|p{1.5cm}|p{1.5cm}|p{1.5cm}||p{1.5cm}|p{1.5cm}|p{1.5cm}|}
\hline 
$x$ & 4,9 & 4,99 & 4,999 & 5,001 & 5,01 & 5,1  \\ 
\hline 
$f(x)$ &  &  &  &  & & \\ 
\hline 
\end{tabular} 
\end{center}	 
	 \item $\ds{\lim_{x\rightarrow5}\frac{\sqrt{x}-\sqrt{5}}{x-5}}=$
\begin{center}
   \begin{tabular}{|c|p{1.5cm}|p{1.5cm}|p{1.5cm}||p{1.5cm}|p{1.5cm}|p{1.5cm}|}
\hline 
$x$ &  &  &  &  &  &  \\ 
\hline 
$f(x)$ &  &  &  &  & & \\ 
\hline 
\end{tabular} 
\end{center}	
	\item $\ds{\lim_{x\rightarrow3}\dfrac{x^2-x-6}{x-3}}=$
	\begin{center}
   \begin{tabular}{|c|p{1.5cm}|p{1.5cm}|p{1.5cm}||p{1.5cm}|p{1.5cm}|p{1.5cm}|}
\hline 
$x$ &  &  &  &  &  &  \\ 
\hline 
$f(x)$ &  &  &  &  & & \\ 
\hline 
\end{tabular} 
\end{center}	
	\item $\ds{\lim_{x\rightarrow0}\dfrac{1-\cos(x)}{x}}=$
	\begin{center}
   \begin{tabular}{|c|p{1.5cm}|p{1.5cm}|p{1.5cm}||p{1.5cm}|p{1.5cm}|p{1.5cm}|}
\hline 
$x$ &  &  &  &  &  &  \\ 
\hline 
$f(x)$ &  &  &  &  & & \\ 
\hline 
\end{tabular} 
\end{center}	
      Recuerde que la calculadora debe estar en radianes para trabajar con las funciones trigonométricas
      \end{enumerate}
      \newpage
   \item Con base en la siguiente gr\'afica, determine los l\'imites pedidos y valores pedidos:
\begin{center}
\usetikzlibrary{arrows}
\baselineskip=10pt
\hsize=6.3truein
\vsize=8.7truein
\definecolor{qqwwtt}{rgb}{0.2,0.2,0.2}
\definecolor{ttqqff}{rgb}{0.4,0.4,0.4}
\definecolor{ccqqtt}{rgb}{0.33,0.33,0.33}
\definecolor{cqcqcq}{rgb}{0.75,0.75,0.75}
\tikzpicture[scale=.9,line cap=round,line join=round,>=triangle 45,x=1.0cm,y=1.0cm]
\draw [color=cqcqcq,dash pattern=on 2pt off 2pt, xstep=1.0cm,ystep=1.0cm] (-3.6,-2.19) grid (3.93,6.25);
\draw[->,color=black] (-3.6,0) -- (3.93,0);
\foreach \x in {-3,-2,-1,1,2,3}
\draw[shift={(\x,0)},color=black] (0pt,2pt) -- (0pt,-2pt) node[below] {$\x$};
\draw[->,color=black] (0,-2.19) -- (0,6.25);
\foreach \y in {-2,-1,1,2,3,4,5,6}
\draw[shift={(0,\y)},color=black] (2pt,0pt) -- (-2pt,0pt) node[left] {$\y$};
\draw[color=black] (0pt,-10pt) node[right] {$0$};
\clip(-3.6,-2.19) rectangle (3.93,6.25);
\draw[line width=1.6pt,color=ccqqtt, smooth,samples=100,domain=-3.6024908318923226:-1.0] plot(\x,{(\x)+1});
\draw[line width=1.6pt,color=ttqqff, smooth,samples=100,domain=-1.0:2.0] plot(\x,{2*(\x)^2-2});
\draw[line width=1.6pt,color=qqwwtt, smooth,samples=100,domain=2.05:3.9261156369459806] plot(\x,{0-2*(\x)+5});
\fill [color=black] (2,6) circle (2.0pt);
\draw [color=black] (2,1) circle (2.5pt);
\fill [color=black] (-1,0) circle (2.0pt);
\endtikzpicture
\end{center}
\begin{enumerate}
\item $\ds{\lim_{x\rightarrow -3}f(x)}=$
\item $\ds{\lim_{x\rightarrow-1}f(x)=}$
\item $\ds{\lim_{x\rightarrow0}f(x)=}$
\item $\ds{\lim_{x\rightarrow2^{-}}f(x)=}$ \hfill Por la izquierda de 2
\item $\ds{\lim_{x\rightarrow2^{+}}f(x)=}$ \hfill Por la derecha de 2
\item $\ds{\lim_{x\rightarrow2}f(x)=}$
\item $\ds{\lim_{x\rightarrow 3}f(x)}=$
\item $f(2)=$
\end{enumerate}
\end{enumerate}
\end{document}