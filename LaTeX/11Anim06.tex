\documentclass[letterpaper,twoside]{article}
\usepackage[utf8]{inputenc}
\usepackage{amsmath,amsfonts,amssymb,amsthm,latexsym}
\usepackage[spanish,es-noshorthands]{babel}
\usepackage[T1]{fontenc}
\usepackage{lmodern}
\usepackage{graphicx,hyperref}
\usepackage{tikz,pgf}
\usepackage{marvosym}
\usepackage{multicol}
\usepackage{fancyhdr}
\usepackage[height=9.5in,width=7in]{geometry}
\usepackage{fancyhdr}
\pagestyle{fancy}
\fancyhead[LE]{\Email matematicas.german@gmail.com}
\fancyhead[RE]{\url{https://www.autistici.org/mathgerman}}
\fancyhead[RO]{\url{https://www.autistici.org/mathgerman}}
\fancyhead[LO]{\Email matematicas.german@gmail.com}

\author{Germ\'an Avenda\~no Ram\'irez~\thanks{Lic. Mat. U.D., M.Sc. U.N.}}
\title{\begin{minipage}{.2\textwidth}
\includegraphics[height=1.75cm]{Images/logo-colegio.png}\end{minipage}
\begin{minipage}{.55\textwidth}
\begin{center}
Animaplano 06\\
Matemáticas $11^{\circ}$
\end{center}
\end{minipage}\hfill
\begin{minipage}{.2\textwidth}
\includegraphics[height=1.75cm]{Images/logo-sed.png} 
\end{minipage}}
\date{}
\thispagestyle{plain}
\begin{document}
\maketitle
Nombre: \hrulefill Curso: \underline{\hspace*{44pt}} Fecha: \underline{\hspace*{2.5cm}}
\begin{multicols}{2}
\section*{Cuestionario}
\emph{Se deben mostrar los procedimientos para llegar a la respuesta. Respuesta sin procedimientos no será tenida en cuenta}

Resuelva 1--3 teniendo en cuenta la siguiente información
\begin{center}
\includegraphics[scale=.75]{Images/Captura041115-05:04:01.png} 
\end{center}
El gráfico representa tres tanques para almacenar agua con la misma medida de radio y con la misma altura para el cono y el cilindro; además la altura equivale al doble del radio (\emph{Recuerde que el volumen de un cilindro es $V_{c}=\pi\cdot r^{2}h$, el volumen del cono es $V_{cono}=\frac{1}{3}\pi\cdot r^{2}h$, y, el volumen de la esfera es $V_{e}=\frac{4}{3}\pi r^{3}$})
\begin{enumerate}
\item El tanque con mayor capacidad es
\begin{enumerate}
\item El cilindro (\textbf{75})
\item El cono (\textbf{81})
\item la esfera (\textbf{68})
\end{enumerate}
\item Si el valor del radio es 40m, el volumen del cono equivale a
\begin{enumerate}
\item 1600$\pi/3$ (\textbf{75})
\item 3200$\pi/3$ (\textbf{45})
\item 128000$\pi/3$ (\textbf{96})
\end{enumerate}
\item Si se triplica la altura del cono, éste tendría una capacidad equivalente a
\begin{enumerate}
\item Un cilindro que mantiene la altura $2r$ (\textbf{85})
\item Un cilindro que triplica su altura $2r$ (\textbf{61})
\item Una esfera que duplica su radio (\textbf{74})
\end{enumerate}
Analiza las siguentes secuencias y en cada una escribe el número de puntos de la figura solicitada
\item \includegraphics[scale=.75]{Images/Captura-041115-05:16:53.png} en F19
\item \includegraphics[scale=.75]{Images/Captura-041115-05:18:22.png} en F6
\item \includegraphics[scale=.75]{Images/Captura-041115-05:19:23.png} en F17
\item \includegraphics[scale=.75]{Images/Captura-041115-05:20:28.png} en F31

Resuelva cada una de las siguientes situaciones
\item En una miscelánea por comprar 30 cuadernos obsequian uno. Si cada cuaderno cuesta \$500, ¿cuántos cuadernos se adquieren con \$45\,000?

Para una composición química se requieren 2 unidades del químico A2 y una cantidad del químico A6
\item Si se han empleado 41 unidades de A6, ¿cuántas unidades se han empleado de A2?
\item Si se han empleado 182 unidades de A2, ¿cuántas unidades se han empleado de A6?

Resuelva las siguientes situaciones
\item Número de dos cifras cuya suma de sus cifras es 9 y su diferencia es 5.
\item El triple del séptimo número primo
\item El decimo tercer número primo

Las edades de los hermanos Juan y Luz Marina suman 34 años. Si Juan es 10 años mayor que Luz Marina, 
\item La edad de Juan es?
\item La edad de Luz Marina es?
\item El valor de $\displaystyle{\lim_{x\rightarrow 1}\dfrac{x^{2}-x}{x-1}}=$
\item $\displaystyle{\lim_{x\rightarrow 5}\dfrac{x^{2}+3x-40}{x-5}}=$
\item El valor de $\displaystyle{\lim_{x\rightarrow 1}\left[\dfrac{x^{2}-x}{x-1}+\dfrac{x^{2}+3x-40}{x-5}\right]}=$

Responda las preguntas 19--22 teniendo en cuenta:

Sebastián y David juegan a sacar balotas de una bolsa oscura que tiene 6 balotas verdes, 5 blancas y 8 rojas.
\item La probabilidad de sacar al azar una balota verde es
\begin{enumerate}
\item 6\% (\textbf{24})
\item 6/19 (\textbf{35})
\item 1/6 (\textbf{36})
\end{enumerate}
\item Si se sacan 2 balotas al tiempo, la cantidad de posibles combinaciones es
\begin{enumerate}
\item 171 (\textbf{45})
\item 342 (\textbf{34})
\item 524 (\textbf{46})
\end{enumerate}
\item La probabilidad de sacar una balota de tal forma que esta NO sea verde es
\begin{enumerate}
\item 0.13 (\textbf{25})
\item 60\% (\textbf{35})
\item 13/19 (\textbf{26})
\end{enumerate}
\item Si se agregan 5 balotas amarillas, la probabilidad de sacar una balota roja es
\begin{enumerate}
\item 0.3 (\textbf{18})
\item 1/3 (\textbf{17})
\item 30\% (\textbf{9})
\end{enumerate}
\item El quinto término de la progresión aritmética cuyo primer término $a_{1}=-2$ y su diferencia $d$ entre término y término es 5
\item Un número perfecto es aquel cuya suma de sus divisores propios es igual al mismo número. El 1 cuenta como divisor propio, mientras que el mismo número no. El primer número perfecto es 6. ¿El segundo es?
\item La suma de las edades de Juan y Santiago es 90 años. Si Santiago es 10 años mayor que Juan, la edad de Juan es?
\item La edad de Santiago es?
\item El duodécimo número primo
\item Un rectángulo tiene un perímetro de 238 y un área de 3540. El largo del rectángulo es?
\item El ancho del rectángulo es?
\item El décimo quinto número primo
\item El quíntuple del sexto número primo
\item Un rectángulo tiene un aŕea de 2376 y su ancho es 10 unidades menos que su largo. El largo del rectángulo es?
\item El ancho del rectángulo es?
\item El décimo sexto número primo
\item El triple del cuadrado del tercer número primo

El perímetro de un rectángulo es 262 y su largo es una unidad más que su ancho
\item ¿El ancho es?
\item ¿Su largo es?

Las edades de Pedro y Domitila suman 134 años. Si hace 6 años, sus edades sumaban 122 años, siendo Pedro mayor que Domitila,
\item La edad de Pedro es:
\item La edad de Domitila es:
\end{enumerate}
\begin{tikzpicture}[scale=.75]
 \fill (1,0) node[above]{1} circle (0.2ex);
 \fill (2,0) node[above]{2} circle (0.2ex);
 \fill (3,0) node[above]{3} circle (0.2ex);
 \fill (4,0) node[above]{4} circle (0.2ex);
 \fill (5,0) node[above]{5} circle (0.2ex);
 \fill (6,0) node[above]{6} circle (0.2ex);
 \fill (7,0) node[above]{7} circle (0.2ex);
 \fill (8,0) node[above]{8} circle (0.2ex);
 \fill (9,0) node[above]{9} circle (0.2ex);
 \fill (10,0) node[above]{10} circle (0.2ex);
 \fill (1,-1) node[left]{11} circle (0.2ex);
 \fill (2,-1) circle (0.2ex);
 \fill (3,-1) circle (0.2ex);
 \fill (4,-1) circle (0.2ex);
 \fill (5,-1) circle (0.2ex);
 \fill (6,-1) circle (0.2ex);
 \fill (7,-1) circle (0.2ex);
 \fill (8,-1) circle (0.2ex);
 \fill (9,-1) circle (0.2ex);
 \fill (10,-1) circle (0.2ex);
 \fill (1,-2) node[left]{21} circle (0.2ex);
 \fill (2,-2) circle (0.2ex);
 \fill (3,-2) circle (0.2ex);
 \fill (4,-2) circle (0.2ex);
 \fill (5,-2) circle (0.2ex);
 \fill (6,-2) circle (0.2ex);
 \fill (7,-2) circle (0.2ex);
 \fill (8,-2) circle (0.2ex);
 \fill (9,-2) circle (0.2ex);
 \fill (10,-2) circle (0.2ex);
 \fill (1,-3) node[left]{31} circle (0.2ex);
 \fill (2,-3) circle (0.2ex);
 \fill (3,-3) circle (0.2ex);
 \fill (4,-3) circle (0.2ex);
 \fill (5,-3) circle (0.2ex);
 \fill (6,-3) circle (0.2ex);
 \fill (7,-3) circle (0.2ex);
 \fill (8,-3) circle (0.2ex);
 \fill (9,-3) circle (0.2ex);
 \fill (10,-3) circle (0.2ex);
 \fill (1,-4) node[left]{41} circle (0.2ex);
 \fill (2,-4) circle (0.2ex);
 \fill (3,-4) circle (0.2ex);
 \fill (4,-4) circle (0.2ex);
 \fill (5,-4) circle (0.2ex);
 \fill (6,-4) circle (0.2ex);
 \fill (7,-4) circle (0.2ex);
 \fill (8,-4) circle (0.2ex);
 \fill (9,-4) circle (0.2ex);
 \fill (10,-4) node[right]{50} circle (0.2ex);
 \fill (1,-5) node[left]{51} circle (0.2ex);
 \fill (2,-5) circle (0.2ex);
 \fill (3,-5) circle (0.2ex);
 \fill (4,-5) circle (0.2ex);
 \fill (5,-5) circle (0.2ex);
 \fill (6,-5) circle (0.2ex);
 \fill (7,-5) circle (0.2ex);
 \fill (8,-5) circle (0.2ex);
 \fill (9,-5) circle (0.2ex);
 \fill (10,-5) circle (0.2ex);
 \fill (1,-6) node[left]{61} circle (0.2ex);
 \fill (2,-6) circle (0.2ex);
 \fill (3,-6) circle (0.2ex);
 \fill (4,-6) circle (0.2ex);
 \fill (5,-6) circle (0.2ex);
 \fill (6,-6) circle (0.2ex);
 \fill (7,-6) circle (0.2ex);
 \fill (8,-6) circle (0.2ex);
 \fill (9,-6) circle (0.2ex);
 \fill (10,-6) circle (0.2ex);
 \fill (1,-7) node[left]{71} circle (0.2ex);
 \fill (2,-7) circle (0.2ex);
 \fill (3,-7) circle (0.2ex);
 \fill (4,-7) circle (0.2ex);
 \fill (5,-7) circle (0.2ex);
 \fill (6,-7) circle (0.2ex);
 \fill (7,-7) circle (0.2ex);
 \fill (8,-7) circle (0.2ex);
 \fill (9,-7) circle (0.2ex);
 \fill (10,-7) circle (0.2ex);
 \fill (1,-8) node[left]{81} circle (0.2ex);
 \fill (2,-8) circle (0.2ex);
 \fill (3,-8) circle (0.2ex);
 \fill (4,-8) circle (0.2ex);
 \fill (5,-8) circle (0.2ex);
 \fill (6,-8) circle (0.2ex);
 \fill (7,-8) circle (0.2ex);
 \fill (8,-8) circle (0.2ex);
 \fill (9,-8) circle (0.2ex);
 \fill (10,-8) circle (0.2ex);
 \fill (1,-9) node[left]{91} circle (0.2ex);
 \fill (2,-9) circle (0.2ex);
 \fill (3,-9) circle (0.2ex);
 \fill (4,-9) circle (0.2ex);
 \fill (5,-9) circle (0.2ex);
 \fill (6,-9) circle (0.2ex);
 \fill (7,-9) circle (0.2ex);
 \fill (8,-9) circle (0.2ex);
 \fill (9,-9) circle (0.2ex);
 \fill (10,-9) node[right]{100} circle (0.2ex);
 \draw (5,-7)--(6,-9)--(5,-8)--(5,-9)--(4,-6)--(2,-5)--(2,-6)--(3,-9)--(2,-8)--(1,-9)--(2,-7)--(1,-5)--(1,-4)--(2,-2)--(2,-1)--(1,0)--(3,-1)--(4,-1)--(5,-3)--(5,-4)--(6,-2)--(7,-1)--(8,-1)--(8,-2)--(10,-3)--(10,-4)--(7,-3)--(10,-5)--(9,-5)--(7,-4)--(5,-6)--(4,-5)--(4,-4)--(3,-5)--(5,-7)--(5,-6)--(6,-6)--(8,-7)--(6,-5);
\end{tikzpicture}
\end{multicols}


\end{document}
