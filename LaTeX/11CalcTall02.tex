\documentclass[10pt,twoside]{article}
\usepackage[utf8]{inputenc}
\usepackage{amsmath}
\usepackage{amsfonts}
\usepackage{amssymb}
\usepackage[spanish,es-noshorthands]{babel}
\usepackage[T1]{fontenc}
\usepackage{lmodern}
\usepackage{graphicx,hyperref}
\usepackage{tikz,pgf}
\usepackage{multicol}
\usepackage{subfig}
\usepackage[papersize={6.5in,8.5in},width=5.5in,height=7in]{geometry}
\usepackage{fancyhdr}
\pagestyle{fancy}
\fancyhead[LE]{Cálculo $11^{\circ}$}
\fancyhead[RE]{}
\fancyhead[RO]{Números reales y desigualdades}
\fancyhead[LO]{}
\fancyfoot[LE,RO]{\textit{Germ\'{a}n Avenda\~{n}o Ram\'{i}rez\\Lic en Mat. U.D., M.Sc. U.N.}}
\fancyfoot[CE,CO]{\thepage}
\author{Germ\'an Avenda\~no Ram\'irez, Lic. U.D., M.Sc. U.N.}
\title{\begin{minipage}{.2\textwidth}
\includegraphics[height=1.75cm]{Images/logo-colegio.png}\end{minipage}
\begin{minipage}{.55\textwidth}
\begin{center}
Taller 2\\
Números reales\\
Cálculo $11^{\circ}$
\end{center}
\end{minipage}\hfill
\begin{minipage}{.2\textwidth}
\includegraphics[height=1.75cm]{Images/logo-sed.png} 
\end{minipage}}
\date{}
\thispagestyle{plain}
\begin{document}
\maketitle
Nombre: \hrulefill Curso: \underline{\hspace*{44pt}} Fecha: \underline{\hspace*{2.5cm}}\\

\begin{enumerate}
  \item En la descomposición de cierta cantidad de agua por electrólisis, se obtienen 2 litros de hidrógeno y 16 litros de oxígeno ¿Cuál es la producción de hidrógeno? ¿Y de oxígeno? Expresa los resultados en tanto por ciento. ¿Qué cantidad de oxígeno se obtendrán con 54 litros de agua?
  \item ¿Cuántas baldosas cuadradas de 20 cm de lado se necesitan para recubrir una superficie de 27,04 $ m^2 $?
  \item Halla la arista de un cubo cuyo volumen es 46,656 $m^3$.
  \item Un depósito cúbico tiene una capacidad de 157.464 litros. ¿Cuál es la superficie de cada una de las paredes del deposito?
\item Calcula y simplifica
\begin{enumerate}\begin{multicols}{2}
  \item $ \sqrt{12}-\sqrt{48}+\sqrt{27} $  
  \item $ \sqrt[3]{24}-\sqrt[3]{375}+\sqrt[3]{81} $
\end{multicols}
\end{enumerate}
\item Calcula y simplifica
\begin{enumerate}\begin{multicols}{3}
  \item $ (\sqrt{5}+\sqrt{2})^2 $  \item $ (2\sqrt{5}+\sqrt{3})^2 $ \item $ (\sqrt{3}-\sqrt{2})(2-\sqrt{2}) $
\end{multicols}
\end{enumerate}
\item Calcula y simplifica
\begin{enumerate}\begin{multicols}{3}
  \item  $ \sqrt{\frac{7}{5}}\sqrt{35} $
  \item $ \sqrt{\frac{3}{2}}\sqrt{\frac{8}{3}} $
  \item $ \sqrt{\frac{10}{3}}\sqrt{7,5}$
\end{multicols}
\end{enumerate}
\item Dibuja un cuadrado de 5 cm de lado. Dibuja otro cuadrado que tenga doble área.
\item Dibuja un rectángulo cuya diagonal valga 5
\item Las dimensiones de una aula son 12 m de largo, 7 m de ancho y 3,40 m de alto. Dos moscas revoletean por el aula. ¿Cuál es la distancia máxima a que pueden encontrarse?
\item Representa:\\

  a) $ [4,6]\cup (9,11)$ \qquad b) $ [-6,5]\cap (2,5) $  \qquad
  c) $ (2,7)\cap (5,9)\cap (6,10) $
  \item Calcula:
  \begin{enumerate}\begin{multicols}{3}
    \item $ \sqrt{1024} $    \item $ \sqrt{441} $
    \item $ \sqrt[3]{729} $ \item $ \sqrt[4]{1296} $
    \item $ \sqrt[5]{-1} $ \item $ 28-2\sqrt{81} $
    \item $ \sqrt{4+2\cdot16} $
    \item $ 8+2\sqrt[3]{-8} $ \item $ \sqrt{400-16-60} $
    \item $ \sqrt{5+\sqrt{13\sqrt{9}}} $
    \item $ \sqrt{10+2\sqrt{7+\sqrt[3]{8}}} $
    \item $ \sqrt{4,7+1,06} $
    \item $ 3\sqrt[3]{0,001+2} $ \item $ \sqrt[4]{\frac{1}{625}} $
    \item $ \sqrt[5]{\frac{-1}{32}} $ \item $ \sqrt{\frac{1}{4}}-\sqrt{\frac{9}{25}} $ \item $ \sqrt{\frac{9}{4}\div \sqrt{\frac{121}{25}}} $
  \end{multicols}
  \end{enumerate}
  \item Calcula
  \begin{enumerate}\begin{multicols}{2}
    \item $ \left(\sqrt{5+\sqrt{5}}\right)\left(\sqrt{5-\sqrt{2}}\right) $    \item $ (2+\sqrt{3})^2 $ \item $ (1+\sqrt{2})(1+\sqrt{2})\sqrt{2} $
  \end{multicols}
  \end{enumerate}
  \item Transforma en radicales
  \begin{enumerate}\begin{multicols}{4}
    \item $ (-3)^{\frac{1}{5}} $  \item $ \left(\frac{3}{5}\right)^{3/7} $  \item $ \left(\frac{2}{3}\right)^{-3/2} $
    \item $ \left(\frac{1}{5}\right)^{-1/4} $
  \end{multicols}
  \end{enumerate}
  \item Halla usando la calculadora
  \[ \sqrt[5]{12^3}\qquad \dfrac{1}{\sqrt[7]{7^4}}\qquad \sqrt[3]{11^2} \]
  \item Encuentra todos los números de tres cifras que sean cubos de un número natural
\end{enumerate}
\end{document}
