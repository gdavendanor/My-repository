\documentclass[letterpaper,fleqn]{article}
\usepackage[spanish,es-noshorthands]{babel}
\usepackage[utf8]{inputenc} 
\usepackage[left=1cm, right=1cm, top=1.5cm, bottom=1.7cm]{geometry}
\usepackage{mathexam}
\usepackage{amsmath}
\usepackage{graphicx}

\ExamClass{\includegraphics[height=16pt]{Images/logo-sed.png} Cálculo $11^{\circ}$}
\ExamName{Sustentación recomendaciones II}
\ExamHead{\includegraphics[height=16pt]{Images/logo-colegio.png} IEDAB}
\newcommand{\LineaNombre}{%
\par
\vspace{\baselineskip}
Nombre:\hrulefill \; Curso: \underline{\hspace*{48pt}} \; Fecha: \underline{\hspace*{2.5cm}} \relax
\par}
\let\ds\displaystyle

\begin{document}
\ExamInstrBox{
Respuesta sin justificar mediante procedimiento no será tenida en cuenta en la calificación. Escriba sus respuestas en el espacio indicado.}
\LineaNombre
\begin{enumerate}
 \item Compruebe si los pares de valores que figuran en la siguiente tabla corresponden a la función
 \[y=3-\dfrac{1}{x-2}\]
 y complete los que faltan:\\
 
 \begin{tabular}{|c|c|c|c|c|c|c|}
 \hline 
 $x$ & 2.01 & 2.5 & 1.9 & 102 &  &  \\ 
 \hline 
$y$ & $-97$ &  & 13 &  & 3.5 & 103 \\ 
 \hline 
 \end{tabular}
 \item Represente la función a trozos
\[f(x)=\left\{ \begin{array}{lcl}
x+3 & \mbox{si} & x\leq 2\\
&
& \\
2x+1 & \mbox{si} & x> 2\\
\end{array}
\right.\]
y diga si es discontinua en algún punto.
 \end{enumerate}

\end{document}
