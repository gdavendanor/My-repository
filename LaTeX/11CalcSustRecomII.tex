\documentclass[letterpaper,fleqn]{article}
\usepackage[spanish,es-noshorthands]{babel}
\usepackage[utf8]{inputenc} 
\usepackage[left=1cm, right=1cm, top=1.5cm, bottom=1.7cm]{geometry}
\usepackage{mathexam}
\usepackage{amsmath}
\usepackage{graphicx}
\usepackage{pgf,tikz}

\ExamClass{\includegraphics[height=16pt]{Images/logo-sed.png} Cálculo $11^{\circ}$}
\ExamName{Sustentación recomendaciones II}
\ExamHead{\includegraphics[height=16pt]{Images/logo-colegio.png} IEDAB}
\newcommand{\LineaNombre}{%
\par
\vspace{\baselineskip}
Nombre:\hrulefill \; Curso: \underline{\hspace*{48pt}} \; Fecha: \underline{\hspace*{2.5cm}} \relax
\par}
\let\ds\displaystyle

\begin{document}
\ExamInstrBox{
Respuesta sin justificar mediante procedimiento no será tenida en cuenta en la calificación. Escriba sus respuestas en el espacio indicado.}
\LineaNombre
\begin{enumerate}
 \item Compruebe si los pares de valores que figuran en la siguiente tabla corresponden a la función
 \[y=3-\dfrac{1}{x-2}\]
 y complete los que faltan:\\
 
 \begin{tabular}{|c|c|c|c|c|c|c|}
 \hline 
 $x$ & 2.01 & 2.5 & 1.9 & 102 &  &  \\ 
 \hline 
$y$ & $-97$ &  & 13 &  & 3.5 & 103 \\ 
 \hline 
 \end{tabular}
 
 \begin{minipage}{.45\textwidth}
 \item Represente la función a trozos
\[f(x)=\left\{ \begin{array}{lcl}
x+3 & \mbox{si} & x\leq 2\\
&
& \\
2x+1 & \mbox{si} & x> 2\\
\end{array}
\right.\]
y diga si es discontinua en algún punto. 
 \end{minipage}
 \begin{minipage}{.5\textwidth}
%Uncomment next line if XeTeX is used
%\def\pgfsysdriver{pgfsys-xetex.def}
\usetikzlibrary{arrows}
\baselineskip=10pt
\hsize=6.3truein
\vsize=8.7truein
\definecolor{cqcqcq}{rgb}{0.75,0.75,0.75}
\tikzpicture[scale=.85,line cap=round,line join=round,>=triangle 45,x=1.0cm,y=1.0cm]
\draw [color=cqcqcq,dash pattern=on 2pt off 2pt, xstep=1.0cm,ystep=1.0cm] (-4.91,-4.31) grid (4.93,4.48);
\draw[->,color=black] (-4.91,0) -- (4.93,0);
\foreach \x in {-4,-3,-2,-1,1,2,3,4}
\draw[shift={(\x,0)},color=black] (0pt,2pt) -- (0pt,-2pt) node[below] {$\x$};
\draw[->,color=black] (0,-4.31) -- (0,4.48);
\foreach \y in {-4,-3,-2,-1,1,2,3,4}
\draw[shift={(0,\y)},color=black] (2pt,0pt) -- (-2pt,0pt) node[left] {$\y$};
\draw[color=black] (0pt,-10pt) node[right] {$0$};
\clip(-4.91,-4.31) rectangle (4.93,4.48);
\endtikzpicture 
 \end{minipage}

\item Halle el dominio y rango de las siguientes funciones:
\begin{minipage}{.45\textwidth}
%Uncomment next line if XeTeX is used
%\def\pgfsysdriver{pgfsys-xetex.def}
\usetikzlibrary{arrows}
\baselineskip=10pt
\hsize=6.3truein
\vsize=8.7truein
\definecolor{cqcqcq}{rgb}{0.75,0.75,0.75}
\tikzpicture[scale=.85,line cap=round,line join=round,x=1.0cm,y=1.0cm]
\draw [color=cqcqcq,dash pattern=on 1pt off 1pt, xstep=1.0cm,ystep=1.0cm] (-3.19,-1.32) grid (5.5,3.61);
\draw[->,color=black] (-3.19,0) -- (5.5,0);
\foreach \x in {-3,-2,-1,1,2,3,4,5}
\draw[shift={(\x,0)},color=black] (0pt,2pt) -- (0pt,-2pt) node[below] {$\x$};
\draw[->,color=black] (0,-1.32) -- (0,3.61);
\foreach \y in {-1,1,2,3}
\draw[shift={(0,\y)},color=black] (2pt,0pt) -- (-2pt,0pt) node[left] {$\y$};
\draw[color=black] (0pt,-10pt) node[right] {$0$};
\clip(-3.19,-1.32) rectangle (5.5,3.61);
\endtikzpicture
\end{minipage}
\begin{minipage}{.5\textwidth}
\begin{enumerate}
\item $y=x-3$\noanswer
\item $y=\dfrac{1}{x-1}$\noanswer
\item $y=+\sqrt{x-2}$\noanswer
\item $y=\sqrt{x+3}$\noanswer
\end{enumerate}
\end{minipage}
Haga la gráfica de la última función:\\
 \end{enumerate}

\end{document}
