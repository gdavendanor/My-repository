\documentclass[10pt,twoside]{article}
\usepackage[utf8]{inputenc}
\usepackage{amsmath}
\usepackage{amsfonts}
\usepackage{amssymb}
\usepackage[spanish,es-noshorthands]{babel}
\usepackage[T1]{fontenc}
\usepackage{lmodern}
\usepackage{graphicx,hyperref}
\usepackage{tikz,pgf}
\usepackage{multicol}
\usepackage{subfig}
\usepackage[papersize={6.5in,8.5in},width=5.5in,height=7in]{geometry}

\author{Comité sindical}
\title{\begin{minipage}{.2\textwidth}
\includegraphics[height=1.75cm]{Images/logo-colegio.png}\end{minipage}
\begin{minipage}{.75\textwidth}
\begin{center}
Colegio Arborizadora Baja I.E.D. (J.M.)\\
Guía Educativa para Estudiantes\\
\end{center}
\end{minipage}\hfill
}
\date{}
\begin{document}
\maketitle
\subsection*{Logros}
\begin{itemize}
\item Que los estudiantes identifiquen algunos de los problemas sociales, económicos y políticos  que han afectado al Sector Educativo hasta hoy.
\item Construir en los estudiantes un pensamiento crítico frente a las diferentes problemáticas educativas y lograr que se involucren en la Defensa de la Educación Pública.
\end{itemize}
\end{document}
