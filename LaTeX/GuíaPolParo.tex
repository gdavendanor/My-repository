\documentclass[10pt,twoside]{article}
\usepackage[utf8]{inputenc}
\usepackage{amsmath}
\usepackage{amsfonts}
\usepackage{amssymb}
\usepackage[spanish,es-noshorthands]{babel}
\usepackage[T1]{fontenc}
\usepackage{lmodern}
\usepackage{graphicx,hyperref}
\usepackage{tikz,pgf}
\usepackage{multicol}
\usepackage{subfig}
\usepackage[papersize={6.5in,8.5in},width=5.75in,height=7.25in]{geometry}

\author{Comité sindical}
%\title{
%}
\date{}
\begin{document}
\begin{minipage}{.2\textwidth}
\includegraphics[height=1.55cm]{Images/logo-colegio.png}\end{minipage}
\begin{minipage}{.75\textwidth}
\begin{center}
\textbf{\large Colegio Arborizadora Baja I.E.D. (J.M.)}\\
\textbf{\large Guía Educativa para Estudiantes}\\
\end{center}
\end{minipage}\hfill
\begin{center}
\large Comité sindical
\end{center}
%\maketitle
\subsection*{Logros}
\begin{itemize}
\item Que los estudiantes identifiquen algunos de los problemas sociales, económicos y políticos  que han afectado al Sector Educativo hasta hoy.
\item Construir en los estudiantes un pensamiento crítico frente a las diferentes problemáticas educativas y lograr que se involucren en la Defensa de la Educación Pública.
\end{itemize}
\paragraph*{TIEMPO DE REALIZACIÓN DE LA GUÍA:}3 y 4 Hora del día lunes 20 de Abril del 2015 en acompañamiento del Docente que tengan clase.
\section*{¿POR QUÉ DEBE HABER UN PARO NACIONAL INDEFINIDO PARA EL SECTOR EDUCATIVO EN EL AÑO 2015?}
En el Contexto de la profunda crisis económica y social que atraviesa el mundo originada por la imposición de las políticas Neoliberales y que se profundiza en Colombia  bajo el Gobierno de Juan Manuel Santos, con la destrucción de la agricultura, la ganadería y la Industria Nacional, por la implementación de los Tratados de Libre Comercio (T.L.C), por la Reforma Tributaria pasada que aumentó los impuestos a la población de clase media y baja, mientras exoneró a las grandes empresas y al capital financiero y la que se plantea ahora en el Plan Nacional de Desarrollo (PND) 2014-2018, junto a la Reforma a la Salud que favorecerá cada vez más a las EPS, la anunciada Reforma Pensional que pretende aumentar el tiempo de cotización y aumentar la edad de pensión de las mujeres, desconociendo el desgaste que tienen nuestras mujeres en la procreación y crianza de los niños y niñas, favoreciendo consecuentemente a los fondos privados de pensiones y perjudicando enormemente al pueblo trabajador que merece una pensión  que le permita vivir los últimos años de  la vida en forma digna, la propuesta de salario mínimo diferencial por regiones con el objetivo de bajarlo en las regiones apartadas de la capital, porque según nuestros dirigentes y organismos internacionales como la OCDE, en nuestro país el salario mínimo es “muy alto”, las secuelas de pobreza y miseria para el pueblo colombiano y la violación a los Derechos Humanos y la criminalización de cualquier intento de la población por  defender sus derechos.

La Educación Pública Colombiana se ve cada día más amenazada por la política de privatización, recorte de los recursos en los planteles educativos, recorte de los presupuestos, pérdida de los derechos de los Estudiantes y Docentes, des-financiación   de la Educación Superior y en general una adecuación por parte del gobierno a los intereses de los Monopolios, de los negocios financieros internacionales  convirtiéndola en una Educación del más pobre nivel no permitiendo que realmente se universalice la Educación (que sea para todos y todas) e interesado solamente en formar mano de obra barata (ofreciendo educación técnica ampliando la oferta del SENA, pero reduciendo la calidad y tiempo de sus programas).

Como se plantea en la \textbf{EDITORIAL Revista, Educación y Cultura (107), Bogotá CEID-FECODE (p.5)}; en la propuesta para estos próximos 4 años “Hacia la excelencia Docente” hecha por la Fundación Compartir y asumida por el Gobierno ya que la incluye en el documento “Bases del Plan de Desarrollo”, sin ningún tipo de Discusión con los verdaderos protagonistas de la Educación y si con 132 recomendaciones dadas por la OCDE (La Organización para la Cooperación y el Desarrollo Económicos). Dentro de esta política qué decir para la Educación de la Primera Infancia, la Básica y la Media, totalmente desarticuladas, sin metas claras, con fines y funciones muy diferentes a los propuestos como propósitos y  objetivos de la Ley 115 del 94, fortaleciendo una educación en Competencias que agudiza la desigualdad y la injusticia curricular generando una educación para ricos privilegiada que forma líderes  y una educación para los pobres que forma emprendedores dóciles que se adaptan fácilmente a las condiciones sociales, políticas y culturales con la obsesión permanente por mejorar los escasos ingresos para sobrevivir.

Otro elemento importante que es tratado en las bases del Plan Nacional de Desarrollo es la innovadora idea de la ampliación de la jornada (Art.85/Ley 115-94) contempla una sola jornada pero no hay una discusión de fondo con los verdaderos actores; de cómo implementarla, en qué instalaciones, con qué programas que mejoren la calidad de la educación, que tipo de alimentos se les va a ofrecer y mucho menos con cuales recursos, esto simplemente se  aplica como una colcha de retazos que no llevará a una Educación de Calidad, pues el magisterio y la comunidad educativa consideran que la jornada única  debería plantear importantes modificaciones al currículo, a las plantas de personal, a las instalaciones,  a los laboratorios,  al plan de estudios,  al uso del tiempo libre y en general  a ofrecer una educación pública, universal y de calidad.

Con eventos como el “Día E” se pueden ver claramente problemáticas como lo expone el documento propuesto por María Antonieta Cano en el programa Tribuna Magisterial (\url{facebook.com/tribunamagisterialbogota}), la ministra pretende imponer el ISCE (Índice Sintético para la Calidad de la Educación), que apunta sustancialmente a certificar, descentralizar y finalmente, “plantelizar” la educación para medir la calidad como si tratara de cualquier producto de una empresa. Las razones del magisterio para oponerse al ISCE se pueden resumir en los siguientes puntos:
\begin{enumerate}
\item El Estado sanciona pero no atiende: sirve a la hoja de ruta del neoliberalismo, dirigida hacia la desatención del Estado de sus obligaciones para transferirlas a las esferas locales de cada institución. El deber estatal se reduce a chequear si las normas del ISCE se cumplen o no, de lo contrario, proceder en la escala de sanciones, que pueden llegar hasta el cierre de los planteles. Este índice de calidad obliga a mostrar resultados en las pruebas estándares, nacionales e internacionales, como uno de los indicativos de la “calidad del servicio/producto”, para finalmente determinar cuál o cuáles instituciones educativas permanecen o desaparecen del “mercado” y cuál profesor merece o no bonificaciones salariales, porque de aumento, bajo este modelo, ni se habla.
\item Educación como empresa: este índice de calidad le imprime a la evaluación institucional criterios empresariales, con normas de calidad propias de la productividad industrial. Trata a los estudiantes como objetos, a las familias como clientes y a los profesores como operarios. De ahí que la relación costo-beneficio que evalúa el ISCE se mida desde la relación de número de alumnos por profesor, impeliendo a los colegios a aumentar el hacinamiento.
\item La familia y la sociedad “responsables” de la educación: El ISCE mide la promoción de estrategias financieras que alivien las carencias presupuestales estatales. Estrategias que van desde el pago de la matrícula por parte de los padres de familia, hasta el importe de tarifas especiales para que los estudiantes accedan a cursos de extensión en las áreas que han ido desapareciendo del currículo, como las artes y la educación física. La imposición de la mal llamada jornada única va en esta dirección.
\item Pérdida de la autonomía institucional y la libertad de cátedra: al promover un “mayor nivel de participación ciudadana” en el control y vigilancia de la actividad laboral de los docentes, se da poderes a padres y acudientes, que, como una suerte de “patrones”, atentan contra la autonomía escolar, pasando por encima de sus derechos, su dignidad y su quehacer docente, y establece entre unos y otros una relación en planos de confrontación, lejos de la armonía que debe coexistir entre ellos. Los hogares, para tener dicha injerencia, deben participar en la financiación de colegios.
\item Plantelización: Según el ISCE, el gobierno debe permitir que paulatinamente cada escuela administre sus propios asuntos y decida sobre la ejecución de los recursos para ir aumentando sus responsabilidades, adecuando así el principio constitucional que incluye a la familia, a la sociedad y al Estado como “responsables” de la educación.
\end{enumerate}
\subsection*{Actividad}
Lea los siguientes artículos del proyecto de Plan Nacional de Desarrollo (PND) y conteste las preguntas propuestas:
\paragraph*{Artículo 51: Obligatoriedad de la educación media:} La educación media será obligatoria, para lo cual el Estado adelantará las acciones tendientes a asegurar la cobertura hasta el grado 11 en todos los establecimientos educativos. El ministerio de Educación Nacional definirá los mecanismos para hacer exigible la atención hasta el grado (11), de manera progresiva, en todos los establecimientos educativos.
\paragraph*{Artículo 53: Jornadas en los establecimientos educativos:} Modifíquese el artículo 85 de la ley 115 de 1994, el cual quedará así:

“Artículo 85: Jornadas en los establecimientos educativos. El servicio público educativo se prestará en las instituciones educativas en jornada única, la cual se define para todos los efectos, como la jornada única escolar en la cual los estudiantes desarrollan actividades que forma parte del plan de estudios del establecimiento educativo durante al menos 7 horas al día. Tratándose de preescolar el tiempo dedicado al plan de estudios será al menos de seis (6) horas.”
\paragraph*{Artículo 56: Programa para el Estímulo a la Calidad Educativa y la implementación de la jornada única:} Creáse el programa para la implementación de la jornada única y el mejoramiento de la calidad de la educación básica y media, el cual se constituirá como un fondo cuenta de la Nación, adscrito al Ministerio de Educación Nacional. El MEN reglamentará la implementación del programa, en coordinación con el Departamento Nacional de Planeación y el Ministerio de Hacienda y Crédito Público.
\subparagraph*{Parágrafo:} El otorgamiento de estímulos a la calidad educativa sólo se podrá hacer con fundamento en las mejoras que registren los establecimientos educativos, medidas de acuerdo con el índice de calidad que defina el Gobierno Nacional a través del MEN.
\begin{enumerate}
\item 1. ¿Observa en la lectura de los artículos anteriores alguna vez mencionada la palabra gratuidad?
\item ¿Cómo garantizará el gobierno la universalidad de la educación cuando por ninguna parte habla de aumentar el presupuesto para educación?
\item Según estudios, para poder implementar la jornada única, deberían construirse 3000 nuevos colegios. Si el gobierno planea construir solamente unos pocos, ¿cómo cree Ud que podrá implementar la jornada única?
\item Si no se construyen más colegios y el número de estudiantes se mantiene constante o incluso podría aumentar, ¿que pasará con las jornadas tarde?
\item Nuestra Federación Colombiana de Trabajadores de la Educación (FECODE) propone en su pliego de peticiones estar de acuerdo con la jornada única si se cumplen los siguientes parámetros:
\begin{enumerate}
\item Infraestructura
\item Relación técnica de estudiante/docente según la norma NTC 4595 de 1999 del ICONTEC que a su vez establece:
\begin{center}
\begin{tabular}{|c|c|c|}
\hline 
Ambiente & \# máximo de estudiantes/maestro & Área ($m^{2}$/estudiante \\ 
\hline 
Prejardín (3-4 años) & 15 & 2,00 \\ 
Jardín (4-5 años) & 20 & 2,00 \\ 
Transición (5-6 años) & 30 & 2,00 \\ 
Básica y media (6-16 años) & 40 & 1,65 a 1,80 \\ 
Especial & 12 & 1,85 \\ \hline
\end{tabular} 
\end{center}
\end{enumerate}
\item Si el gobierno se comprometió a aumentar el porcentaje del PIB al 7.5\% para inversión en educación y éste no aparece en el PND, ¿cree Ud que se cumplirá lo prometido?
\begin{center}
\begin{tabular}{|c|c|}
\hline 
País & \% del PIB~\footnote{Datos promedio según la Comisión Económica para América Latina y el Caribe (CEPAL) en el período 2010-2013} \\ 
\hline 
Bolivia & 5.69 \\ 
\hline 
Colombia & 3.16 \\ 
\hline 
Cuba & 16.53 \\ 
\hline 
Ecuador & 4.66 \\ 
\hline 
Venezuela & 5.64 \\ 
\hline 
\end{tabular}
\end{center}
\item Según los datos de la tabla anterior, ¿cuál será el país que tiene un mejor sistema educativo? ¿cuál el peor? ¿Cree Ud que la inversión que hace un país está directamente relacionada con el nivel de su sistema educativo?
\item 
\end{enumerate}
\end{document}
