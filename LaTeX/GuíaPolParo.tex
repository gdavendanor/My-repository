\documentclass[10pt,twoside]{article}
\usepackage[utf8]{inputenc}
\usepackage{amsmath}
\usepackage{amsfonts}
\usepackage{amssymb}
\usepackage[spanish,es-noshorthands]{babel}
\usepackage[T1]{fontenc}
\usepackage{lmodern}
\usepackage{graphicx,hyperref}
\usepackage{tikz,pgf}
\usepackage{multicol}
\usepackage{subfig}
\usepackage[papersize={6.5in,8.5in},width=5.6in,height=7in]{geometry}

\author{Comité sindical}
\title{\begin{minipage}{.2\textwidth}
\includegraphics[height=1.75cm]{Images/logo-colegio.png}\end{minipage}
\begin{minipage}{.75\textwidth}
\begin{center}
Colegio Arborizadora Baja I.E.D. (J.M.)\\
Guía Educativa para Estudiantes\\
\end{center}
\end{minipage}\hfill
}
\date{}
\begin{document}
\maketitle
\subsection*{Logros}
\begin{itemize}
\item Que los estudiantes identifiquen algunos de los problemas sociales, económicos y políticos  que han afectado al Sector Educativo hasta hoy.
\item Construir en los estudiantes un pensamiento crítico frente a las diferentes problemáticas educativas y lograr que se involucren en la Defensa de la Educación Pública.
\end{itemize}
\paragraph*{TIEMPO DE REALIZACIÓN DE LA GUÍA:}3 y 4 Hora del día lunes 20 de Abril del 2015 en acompañamiento del Docente que tengan clase.
\section*{¿POR QUÉ DEBE HABER UN PARO NACIONAL INDEFINIDO PARA EL SECTOR EDUCATIVO EN EL AÑO 2015?}
En el Contexto de la profunda crisis económica y social que atraviesa el mundo originada por la imposición de las políticas Neoliberales y que se profundiza en Colombia  bajo el Gobierno de Juan Manuel Santos, con la destrucción de la agricultura, la ganadería y la Industria Nacional, por la implementación de los Tratados de Libre Comercio (T.L.C), por la Reforma Tributaria pasada que aumentó los impuestos a la población de clase media y baja, mientras exoneró a las grandes empresas y al capital financiero y la que se plantea ahora en el Plan Nacional de Desarrollo (PND) 2014-2018, junto a la Reforma a la Salud que favorecerá cada vez más a las EPS, la anunciada Reforma Pensional que pretende aumentar el tiempo de cotización y aumentar la edad de pensión de las mujeres, desconociendo el desgaste que tienen nuestras mujeres en la procreación y crianza de los niños y niñas, favoreciendo consecuentemente a los fondos privados de pensiones y perjudicando enormemente al pueblo trabajador que merece una pensión  que le permita vivir los últimos años de  la vida en forma digna, la propuesta de salario mínimo diferencial por regiones con el objetivo de bajarlo en las regiones apartadas de la capital, porque según nuestros dirigentes y organismos internacionales como la OCDE, en nuestro país el salario mínimo es “muy alto”, las secuelas de pobreza y miseria para el pueblo colombiano y la violación a los Derechos Humanos y la criminalización de cualquier intento de la población por  defender sus derechos.

La Educación Pública Colombiana se ve cada día más amenazada por la política de privatización, recorte de los recursos en los planteles educativos, recorte de los presupuestos, pérdida de los derechos de los Estudiantes y Docentes, des-financiación   de la Educación Superior y en general una adecuación por parte del gobierno a los intereses de los Monopolios, de los negocios financieros internacionales  convirtiéndola en una Educación del más pobre nivel no permitiendo que realmente se universalice la Educación (que sea para todos y todas) e interesado solamente en formar mano de obra barata (ofreciendo educación técnica ampliando la oferta del SENA, pero reduciendo la calidad y tiempo de sus programas).

Como se plantea en la \textbf{EDITORIAL Revista, Educación y Cultura (107), Bogotá CEID-FECODE (p.5)}; en la propuesta para estos próximos 4 años “Hacia la excelencia Docente” hecha por la Fundación Compartir y asumida por el Gobierno ya que la incluye en el documento “Bases del Plan de Desarrollo”, sin ningún tipo de Discusión con los verdaderos protagonistas de la Educación y si con 132 recomendaciones dadas por la OCDE (La Organización para la Cooperación y el Desarrollo Económicos). Dentro de esta política qué decir para la Educación de la Primera Infancia, la Básica y la Media, totalmente desarticuladas, sin metas claras, con fines y funciones muy diferentes a los propuestos como propósitos y  objetivos de la Ley 115 del 94, fortaleciendo una educación en Competencias que agudiza la desigualdad y la injusticia curricular generando una educación para ricos privilegiada que forma líderes  y una educación para los pobres que forma emprendedores dóciles que se adaptan fácilmente a las condiciones sociales, políticas y culturales con la obsesión permanente por mejorar los escasos ingresos para sobrevivir.
\end{document}
