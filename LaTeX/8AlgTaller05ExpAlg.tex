\documentclass[10pt,twoside]{article}
\usepackage[utf8]{inputenc}
\usepackage{amsmath}
\usepackage{amsfonts}
\usepackage{amssymb}
\usepackage[spanish,es-noshorthands]{babel}
\usepackage[T1]{fontenc}
\usepackage{lmodern}
\usepackage{graphicx,hyperref}
\usepackage{tikz,pgf}
\usepackage{multicol}
\usepackage{subfig}
\usepackage[papersize={6.5in,8.5in},width=5.5in,height=7in]{geometry}
\usepackage{fancyhdr}
\pagestyle{fancy}
\fancyhead[LE]{\includegraphics[height=12pt]{Images/logo-colegio.png} Álgebra $8^{\circ}$}
\fancyhead[RE]{}
\fancyhead[RO]{\textit{Germ\'an Avenda\~no Ram\'irez, Lic. U.D., M.Sc. U.N.}}
\fancyhead[LO]{}

\author{Germ\'an Avenda\~no Ram\'irez, Lic. U.D., M.Sc. U.N.}
\title{\begin{minipage}{.2\textwidth}
\includegraphics[height=1.75cm]{Images/logo-colegio.png}\end{minipage}
\begin{minipage}{.55\textwidth}
\begin{center}
Taller 05, Expresiones algebraicas Nivel II \\
Álgebra $8^{\circ}$
\end{center}
\end{minipage}\hfill
\begin{minipage}{.2\textwidth}
\includegraphics[height=1.75cm]{Images/logo-sed.png} 
\end{minipage}}
\date{}
\begin{document}
\maketitle
Nombre: \hrulefill Curso: \underline{\hspace*{44pt}} Fecha: \underline{\hspace*{2.5cm}}
\begin{enumerate}
 \item Copie y complete la tabla:
{%
\newcommand{\mc}[3]{\multicolumn{#1}{#2}{#3}}
\begin{center}
\begin{tabular}{|l|l|l|l|}\hline
\mc{1}{|c|}{\textbf{M}} & \mc{1}{c|}{\textbf{Coeficiente}} & \mc{1}{c|}{\textbf{Parte literal}} & \mc{1}{c|}{\textbf{Grado}}\\ \hline
$-a^{5}$ &  &  & \\ \hline
+8 &  &  & \\ \hline
$-m$ &  &  & \\ \hline
$-\frac{4}{3}x^{3}$ &  &  & \\ \hline
\end{tabular}
\end{center}
}%
\item Copia y completa la tabla, señalando con una X las clases a que pertenece el polinomio:
{%
\newcommand{\mc}[3]{\multicolumn{#1}{#2}{#3}}
\begin{table}[h!]
\begin{center}
\begin{tabular}{|l|c|c|c|c|c|c|}\hline
\mc{1}{|c|}{Polinomio} & Ordenado & Desordenado & Completo & Incompleto & Grado & Faltan grados\\ \hline
$2x^{2}-5x+1$ & X &  & X &  & 2 & \\ \hline
$-m+6m^{3}+2$ &  &  &  &  &  & \\ \hline
$h^{5}-2h$ &  &  &  &  &  & \\ \hline 
$b^{4}-b^{2}$ &  &  &  &  &  & \\ \hline
\end{tabular}
\end{center}\caption{Tabla 2}
            \end{table} 
}%
\item Dados los polinomios $P=(2x^{2}-5x+1)$; $R=(-x^{2}-2+6x)$; $T=(-4+6x^{2}-5x)$, realiza con ellos las siguientes operaciones:
\begin{tabbing}
 \hspace{3.5cm}\=\hspace{3.5cm}\=\hspace{3.5cm}\=\hspace{3.5cm}\=\kill
 $P-R$; \> $3R-2P$; \> $3T-2R$; \> $2\cdot (P-T-R)$; \\
 $-4\cdot (T+R-P)$; \> $P^{2}+R^{2}$; \> $R^{2}-T^{2}$;\>
\end{tabbing}
\item Escribe polinomios que tengan estas características:
\begin{enumerate}
 \item $P(x)$, que sea de grado 3, ordenado y completo
 \item $M(a)$, que sea de grado 4, desordenado e incompleto
 \item $T(h)$, que sea de grado 6 y que no tengo ni grado 2 ni término independiente
 \item $M(b)$, que sea de grado 5 y que sólo tenga dos términos
\end{enumerate}
\item Resuelve estas ecuaciones:
\begin{enumerate}
 \item $10-4(m-1)+10=$
\end{enumerate}

\end{enumerate}


\end{document}
