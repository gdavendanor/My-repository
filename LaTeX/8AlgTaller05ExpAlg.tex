\documentclass[10pt,twoside]{article}
\usepackage[utf8]{inputenc}
\usepackage{amsmath}
\usepackage{amsfonts}
\usepackage{amssymb}
\usepackage[spanish,es-noshorthands]{babel}
\usepackage[T1]{fontenc}
\usepackage{lmodern}
\usepackage{graphicx,hyperref}
\usepackage{tikz,pgf}
\usepackage{multicol}
\usepackage{subfig}
\usepackage[papersize={6.5in,8.5in},width=5.5in,height=7in]{geometry}
\usepackage{fancyhdr}
\pagestyle{fancy}
\fancyhead[LE]{\includegraphics[height=12pt]{Images/logo-colegio.png} Álgebra $8^{\circ}$}
\fancyhead[RE]{}
\fancyhead[RO]{\textit{Germ\'an Avenda\~no Ram\'irez, Lic. U.D., M.Sc. U.N.}}
\fancyhead[LO]{}

\author{Germ\'an Avenda\~no Ram\'irez, Lic. U.D., M.Sc. U.N.}
\title{\begin{minipage}{.2\textwidth}
\includegraphics[height=1.75cm]{Images/logo-colegio.png}\end{minipage}
\begin{minipage}{.55\textwidth}
\begin{center}
Taller 05, Expresiones algebraicas Nivel II \\
Álgebra $8^{\circ}$
\end{center}
\end{minipage}\hfill
\begin{minipage}{.2\textwidth}
\includegraphics[height=1.75cm]{Images/logo-sed.png} 
\end{minipage}}
\date{}
\begin{document}
\maketitle
Nombre: \hrulefill Curso: \underline{\hspace*{44pt}} Fecha: \underline{\hspace*{2.5cm}}
\begin{enumerate}
 \item Copie y complete la tabla:
{%
\newcommand{\mc}[3]{\multicolumn{#1}{#2}{#3}}
\begin{center}
\begin{tabular}{|l|l|l|l|}\hline
\mc{1}{|c|}{\textbf{M}} & \mc{1}{c|}{\textbf{Coeficiente}} & \mc{1}{c|}{\textbf{Parte literal}} & \mc{1}{c|}{\textbf{Grado}}\\ \hline
$-a^{5}$ &  &  & \\ \hline
+8 &  &  & \\ \hline
$-m$ &  &  & \\ \hline
$-\frac{4}{3}x^{3}$ &  &  & \\ \hline
\end{tabular}
\end{center}
}%
\item Copia y completa la tabla, señalando con una X las clases a que pertenece el polinomio:
{%
\newcommand{\mc}[3]{\multicolumn{#1}{#2}{#3}}
\begin{table}[h!]
\begin{center}
\begin{tabular}{|l|c|c|c|c|c|c|}\hline
\mc{1}{|c|}{Polinomio} & Ordenado & Desordenado & Completo & Incompleto & Grado & Faltan grados\\ \hline
$2x^{2}-5x+1$ & X &  & X &  & 2 & \\ \hline
$-m+6m^{3}+2$ &  &  &  &  &  & \\ \hline
$h^{5}-2h$ &  &  &  &  &  & \\ \hline 
$b^{4}-b^{2}$ &  &  &  &  &  & \\ \hline
\end{tabular}
\end{center}
            \end{table} 
}%
\item Dados los polinomios $P=(2x^{2}-5x+1)$; $R=(-x^{2}-2+6x)$; $T=(-4+6x^{2}-5x)$, realiza con ellos las siguientes operaciones:
\begin{tabbing}
 \hspace{3.5cm}\=\hspace{3.5cm}\=\hspace{3.5cm}\=\hspace{3.5cm}\=\kill
 $P-R$; \> $3R-2P$; \> $3T-2R$; \> $2\cdot (P-T-R)$; \\
 $-4\cdot (T+R-P)$; \> $P^{2}+R^{2}$; \> $R^{2}-T^{2}$;\>
\end{tabbing}
\item Escribe polinomios que tengan estas características:
\begin{enumerate}
 \item $P(x)$, que sea de grado 3, ordenado y completo
 \item $M(a)$, que sea de grado 4, desordenado e incompleto
 \item $T(h)$, que sea de grado 6 y que no tengo ni grado 2 ni término independiente
 \item $M(b)$, que sea de grado 5 y que sólo tenga dos términos
\end{enumerate}
\item Resuelve estas ecuaciones:
\begin{enumerate}
 \item $10-4(m-1)+10=3(m-1)-2(m-5)$
 \item $\dfrac{5x-2}{9}+\dfrac{x+10}{3}=-4$
 \item $\dfrac{6+b}{2}-\dfrac{2b-12}{3}=2-b$
 \item $\dfrac{3a-3}{6}-\dfrac{2a+4}{7}=\dfrac{a-3}{2}+\dfrac{2a-14}{4}$
 \item $\dfrac{x-2}{3}+\dfrac{x-20}{24}-\dfrac{2x-3}{6}=0$
 \item $\dfrac{y+9}{2}-\dfrac{1-2y}{7}=\dfrac{11-y}{14}-\dfrac{3y+5}{4}$
\end{enumerate}
\item Halla un número cuya mitad, tercera y cuarta parte sumen 39.
\item Calcula dos números impares consecutivos que sumen 24.
\item Un obrero y su mujer ganan entre los dos 10.000 pts diarias. Sabiendo que la mujer gana los 2/3 de lo que gana el marido, calcula lo que gana cada uno.
\item En una granja hay conejos y gallinas, contándose en total 39 cabezas y 126 patas. ¿Cuántos animales hay de cada clase?
\item La madre de Luis tiene triple edad que él y dentro de 14 años sólo tendrá el doble de la que entonces tenga Luis. Calcula la edad actual de cada uno.
\item Mezclamos 15 litros de agua a 80 $^{\circ}$C con 25 litros a 60 $^{\circ}$C. ¿A qué temperatura quedará la mezcla?
\item El perímetro de un triángulo isósceles es 15 cm y el lado desigual es la mitad de uno de los lados iguales. Calcula la longitud de cada uno de los lados.
\end{enumerate}
\section*{Nivel III}
\begin{enumerate}
 \item Copia y completa la tabla, realizando las operaciones fuera de ella.
 \begin{center}
\begin{tabular}{|c|c|c|c|c|}\hline
$a$ & $b$ & $c$ & Expresión algebraica & Valor numérico \\ \hline
$\frac{3}{2}$ & $\frac{3}{5}$ & $\frac{-1}{4}$ & $3a-2b+c$ & \rule[-0.3cm]{0cm}{0.8cm}\\ \hline
$+6$ & $-8$ & $-2$ & $a^{2}-\dfrac{b}{4}+2c$ & \rule[-0.3cm]{0cm}{0.8cm} \\ \hline
$\frac{-1}{2}$ & $\frac{4}{6}$ & $\frac{3}{4}$ & $\dfrac{3a+c^{2}}{2}-4b$ & \rule[-0.3cm]{0cm}{0.8cm} \\ \hline
 \end{tabular}
 \end{center}

\end{enumerate}

\end{document}
