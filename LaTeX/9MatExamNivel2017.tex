\documentclass[fleqn]{article}
\usepackage[spanish,es-noshorthands]{babel}
\usepackage[utf8]{inputenc} 
\usepackage[papersize={5.5in,8.5in},left=1cm, right=1cm, top=1.5cm, bottom=1.7cm]{geometry}
\usepackage{mathexam}
\usepackage{amsmath}
\usepackage{graphicx}
\usepackage{tikz,pgf}

\ExamClass{\includegraphics[height=16pt]{Images/logo-sed.png} Matemáticas $9^{\circ}$}
\ExamName{Nivelación 2017}
\ExamHead{\includegraphics[height=16pt]{Images/logo-colegio.png} IEDAB}
\newcommand{\LineaNombre}{%
\par
\vspace{\baselineskip}
Nombre:\hrulefill \; Curso: \underline{\hspace*{48pt}} \; Fecha: \underline{\hspace*{2.5cm}} \relax
\par}
\let\ds\displaystyle

\begin{document}
\ExamInstrBox{
Respuesta sin justificar mediante procedimiento no será tenida en cuenta en la calificación. Escriba sus respuestas en el espacio indicado. Tiene 45 minutos para contestar esta prueba.}
\LineaNombre
\begin{enumerate}
 \item Evalúe la expresión numérica:
 \begin{enumerate}
 \item \label{ej01} $4^{-3}=$\noanswer[12pt]
 \item \label{ej02} $\left(\dfrac{3^{-1}}{3^{2}}\right)^{-1}=$\noanswer[12pt]
 \end{enumerate}
 \item Use la propiedad distributiva para simplificar la expresión \[3\sqrt{45}-2\sqrt{20}-\sqrt{80}=\]\noanswer[24pt]
 \item Si la luz del sol tarda aproximadamente 8 minutos y 19 segundos en viajar del sol a la tierra, y la velocidad de la luz $c$ en el vacío es de aproximadamente $c=300\,000$ km/s, calcule la distancia en km del sol a la tierra. Exprese el resultado en notación científica.\noanswer
 \newpage
 \item \label{ej63} Una empresa de mudanzas de apartamentos cobra de acuerdo a la ecuación $c=75h+150$, donde $c$ representa el dinero en dólares y $h$ representa el número de horas para hacer el trasteo. 
\begin{enumerate}
\item Complete la tabla

\begin{center}
 \begin{tabular}{c|cccc}
\hline 
$h$ & 1 & 2 & 3 & 4 \\ 
\hline 
$c$ &  &  &  &  \\ 
\hline 
\end{tabular}
 \end{center} 
\item Haciendo que el eje horizontal sea $h$ y el eje vertical $c$, grafique la ecuación $c=75h+150$ para valores no negativos de $h$
\begin{center}
\usetikzlibrary{arrows}
\definecolor{cqcqcq}{rgb}{0.75,0.75,0.75}
\begin{tikzpicture}[line cap=round,line join=round,>=triangle 45]
\draw [dotted,xstep=1.0cm,ystep=1.0cm] (0,0) grid (4,6);
\draw[->,color=black] (-0.25,0) -- (4.25,0) node[right]{$h$};
\foreach \x in {,1,2,3,4}
\draw[shift={(\x,0)},color=black] (0pt,2pt) -- (0pt,-2pt) node[below] {\footnotesize $\x$};
\draw[->,color=black] (0,-.5) -- (0,6.25)node[left]{$c$};
\foreach \y in {1,2,3,4,5,6}
\draw[shift={(0,\y)},color=black] (2pt,0pt) -- (-2pt,0pt);
\node[left] at (0,1){75};
\draw[color=black] (0pt,-10pt) node[right] {\footnotesize $0$};
\clip(-0.47,-.5) rectangle (4.25,6.25);
\end{tikzpicture}
\end{center}
\item Use la gráfica para aproximar los valores de $c$ cuando $h=1.5$ y 3.5\noanswer
\end{enumerate}
\begin{minipage}{.5\textwidth}
\item Encuentre el área de la figura 
\end{minipage}
\begin{minipage}{.45\textwidth}
\begin{tikzpicture}
\draw (0,0) --(3,0)--(1.5,2)--cycle;
\draw[<-] (0,-.1)--(1.1,-.1);
\node at (1.5,-.1){6cm};
\draw[->]  (1.9,-.1) --(3,-.1);
\draw[dotted] (1.5,0)--node{4cm}(1.5,2);
\end{tikzpicture}
\end{minipage}
\noanswer
 \end{enumerate}

\end{document}
