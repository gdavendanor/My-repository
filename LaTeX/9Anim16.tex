\documentclass[letterpaper,11pt,twoside]{article}
\usepackage[utf8]{inputenc}
\usepackage{amsmath,amsfonts,amssymb,amsthm,latexsym}
\usepackage[spanish,es-noshorthands]{babel}
\usepackage[T1]{fontenc}
\usepackage{lmodern}
\usepackage{graphicx,hyperref}
\usepackage{tikz,pgf}
\usepackage{multicol}
\usepackage{subfig}
\usepackage{fancyhdr}
\usepackage[includeheadfoot,left=0.4in,right=0.3in,top=0.4in,bottom=0.4in]{geometry}
\pagestyle{fancy}
\fancyhead[LE]{Álgebra $9^{\circ}$}
\fancyhead[RE]{\includegraphics[height=12pt]{Images/logo-colegio.png}}
\fancyhead[LO]{Animaplano 16}
\rfoot{\textit{Germ\'an Dar\'io Avenda\~no Ram\'irez\\Lic. Mat. U.D., M.Sc. U.N.}}
\author{Germ\'an Dar\'io Avenda\~no Ram\'irez, Lic. - M.Sc.}
\date{}

\begin{document}
\thispagestyle{empty}
\begin{minipage}{.10\textwidth}\includegraphics[height=1.6cm]{Images/logo-colegio.png}
\end{minipage}\hfill
\begin{minipage}{.38\textwidth}
\Large Animaplano 16\\
Matemáticas $11^{\circ}$\\
\large Prof. Germán Avendaño Ramírez
\end{minipage}
\hfill
\begin{minipage}{.09\textwidth}
\includegraphics[height=1.5cm]{Images/logo-sed.png} 
\end{minipage}\hfill
\begin{minipage}{.39\textwidth}
\large Nombre: \hrulefill\\
Curso: 110\underline{\hspace*{12pt}}\\
Fecha: \underline{\hspace{120pt}}
\end{minipage}
\vspace*{12pt}
\section*{Animaplano 16}
\begin{multicols}{2}
\begin{enumerate}
\item El coeficiente del monomio $24x^{6}y^{4}$ es:
\item El quíntuple, de la sexta parte de 30
\item Si $a=3$, halle $a^{3}+a^{2}+5a+7$
\item Si $m=22$, $n=13$ y $p=12$. ¿El perímetro del trapecio es?
\begin{center}
\usetikzlibrary{arrows}
\baselineskip=10pt
\hsize=6.3truein
\vsize=8.7truein
\definecolor{zzttqq}{rgb}{0.27,0.27,0.27}
\tikzpicture[line cap=round,line join=round,>=triangle 45,x=1.0cm,y=1.0cm]
\clip(-1.04,1.32) rectangle (4.12,4.54);
\fill[color=zzttqq,fill=zzttqq,fill opacity=0.1] (0,4) -- (3,4) -- (4,2) -- (-1,2) -- cycle;
\draw [color=zzttqq] (0,4)-- (3,4);
\draw [color=zzttqq] (3,4)-- (4,2);
\draw [color=zzttqq] (4,2)-- (-1,2);
\draw [color=zzttqq] (-1,2)-- (0,4);
\draw [dash pattern=on 1pt off 1pt] (0,4)-- (0,2);
\draw (1.26,4.36) node[anchor=north west] {n};
\draw (1.32,1.82) node[anchor=north west] {m};
\draw (3.68,3.2) node[anchor=north west] {p};
\endtikzpicture
\end{center}
\item Si $p(m)=-(-m+m-m)$ y $(m\div2)=35$, entonces $p(m)=$
\item Si el lado de un octágono regula mide $\sqrt{100}$m. ¿El perímetro total del mismo es?
\item La suma de dos números es 155, si 1/3 del menor es 22. ¿El mayor es?
\item El cuádruple de 1/4 de 88
\item Si $p(a)=5a-8a+a-6a+19a$ \; y \; $2a=14$, el valor numérico $p(a)$ es:
\item Sume 7, a la mitad de la mitad de 240
\item Si el perímetro del trapecio isósceles de la figura es 266 cm, encuentre la longitud del lado $p$, si $m=100$cm y $n=50$cm
\item Si $a=b=c=23$, entonces $b+b+b=$
\item Identifique y sume los números primos: 31, 21, 27, 41, 33, 7, 35
\item En la figura si $m=2n$, \; $p=9$ \; y \; $m=40$; ¿el perímetro total es?
\item Si $x=4$, el valor de $x^{3}+4$ es:
\item El triple, de la mitad de 44
\item El triple de la tercera parte de 75
\item 2 docenas + 1/2 centena
\item El producto entre $2ab$ por $-2ab$ corresponde a:
\begin{enumerate}
\item $4ab+ab$ \; R=45
\item $-4a^{2}b$  \; R=63
\item $-4a^{2}b^{2}$ \; R=52
\end{enumerate}
Anote en el plano la R correcta.
\item Si el minuendo es 67 y la diferencia es 36, el sustraendo es
\item Suprimir y reducir $a+b-(-2a+3)$ corresponde a:
\begin{enumerate}
\item $3a+b-3$ \; R=53
\item $a-3b+3$ \; R=41
\item $3a+b+3$ \; R=18
\end{enumerate}
\item Halle $4!+4!+3!+1!=$
\item Si $p(x)=3x^{3}-2x^{3}-3x-6$ y sea $x=4$, calcule $p(x)$
\item El menor de tres números consecutivos cuya syma es 198
\item El coeficiente del producto entre $9x^{3}y^{2}z$ y $8x^{3}y^{2}z$ es
\item De $a+b$ restar $a-b$, la resta correcta es:
\begin{enumerate}
\item $(a-b)-(a+b)$ \; R=55
\item $2b$ \; R=73
\item $a-2b$ \; R=82
\end{enumerate}
\item $-k$ por $4kp$ es:
\begin{enumerate}
\item $-4k+-k^{2}+-kp$ \; R=66
\item $-4k^{2}p$ \; R=51
\end{enumerate}
\item Sume las áreas de dos cuadrados de lados 5 y 4 cm cada uno
\item El mayor de tres números consecutivos cuya suma es 216
\item Resuelva: Si $x+z=113$ y $x+45=84$, entonces $z=$?
\item El producto entre $-2m^{3}n^{2}$ y $-3m^{2}n^{3}$ corresponde a:
\begin{enumerate}
\item $-9m^{9}n^{4}$ \; R=69
\item $6m^{6}n^{6}$ \; R=79
\item $6m^{5}n^{5}$ \; R=83
\end{enumerate}
\item Multiplique 20.5 por el 40\% de 10
\item De $2m-3n$ restar $-m+2n$, la resta correcta es:
\begin{enumerate}
\item $3m+5n$ \; R=59
\item $5m-3n$ \; R=68
\item $3m-5n$ \; R=71
\end{enumerate}
\item Identifique y sume los números primos 31, 45, 15, 2, 29, 27, 39, 9
\item Si $x-n=36$ \; y sea \; $x-53=46$, entonces $n=$
\item Sume 6u al perímetro de un cuadrado cuya área es 144u$^{2}$
\item Sume 3u$^{2}$ al área del cuadrado si su diagonal mide 8u.
\item El lado menor de un rectángulo mide $1+x$, \; y el lado mayor es el doble del lado menor, calcule el perímetro del rectángulo en dm si $x=5$ dm
\end{enumerate}
\end{multicols}
\subsection*{Plano}
\begin{center}
\begin{tikzpicture}[scale=1.2]
 \fill (1,0) node[above]{1} circle (0.2ex);
 \fill (2,0) node[above]{2} circle (0.2ex);
 \fill (3,0) node[above]{3} circle (0.2ex);
 \fill (4,0) node[above]{4} circle (0.2ex);
 \fill (5,0) node[above]{5} circle (0.2ex);
 \fill (6,0) node[above]{6} circle (0.2ex);
 \fill (7,0) node[above]{7} circle (0.2ex);
 \fill (8,0) node[above]{8} circle (0.2ex);
 \fill (9,0) node[above]{9} circle (0.2ex);
 \fill (10,0) node[above]{10} circle (0.2ex);
 \fill (1,-1) node[left]{11} circle (0.2ex);
 \fill (2,-1) circle (0.2ex);
 \fill (3,-1) circle (0.2ex);
 \fill (4,-1) circle (0.2ex);
 \fill (5,-1) circle (0.2ex);
 \fill (6,-1) circle (0.2ex);
 \fill (7,-1) circle (0.2ex);
 \fill (8,-1) circle (0.2ex);
 \fill (9,-1) circle (0.2ex);
 \fill (10,-1) circle (0.2ex);
 \fill (1,-2) node[left]{21} circle (0.2ex);
 \fill (2,-2) circle (0.2ex);
 \fill (3,-2) circle (0.2ex);
 \fill (4,-2) circle (0.2ex);
 \fill (5,-2) circle (0.2ex);
 \fill (6,-2) circle (0.2ex);
 \fill (7,-2) circle (0.2ex);
 \fill (8,-2) circle (0.2ex);
 \fill (9,-2) circle (0.2ex);
 \fill (10,-2) circle (0.2ex);
 \fill (1,-3) node[left]{31} circle (0.2ex);
 \fill (2,-3) circle (0.2ex);
 \fill (3,-3) circle (0.2ex);
 \fill (4,-3) circle (0.2ex);
 \fill (5,-3) circle (0.2ex);
 \fill (6,-3) circle (0.2ex);
 \fill (7,-3) circle (0.2ex);
 \fill (8,-3) circle (0.2ex);
 \fill (9,-3) circle (0.2ex);
 \fill (10,-3) circle (0.2ex);
 \fill (1,-4) node[left]{41} circle (0.2ex);
 \fill (2,-4) circle (0.2ex);
 \fill (3,-4) circle (0.2ex);
 \fill (4,-4) circle (0.2ex);
 \fill (5,-4) circle (0.2ex);
 \fill (6,-4) circle (0.2ex);
 \fill (7,-4) circle (0.2ex);
 \fill (8,-4) circle (0.2ex);
 \fill (9,-4) circle (0.2ex);
 \fill (10,-4) node[right]{50} circle (0.2ex);
 \fill (1,-5) node[left]{51} circle (0.2ex);
 \fill (2,-5) circle (0.2ex);
 \fill (3,-5) circle (0.2ex);
 \fill (4,-5) circle (0.2ex);
 \fill (5,-5) circle (0.2ex);
 \fill (6,-5) circle (0.2ex);
 \fill (7,-5) circle (0.2ex);
 \fill (8,-5) circle (0.2ex);
 \fill (9,-5) circle (0.2ex);
 \fill (10,-5) circle (0.2ex);
 \fill (1,-6) node[left]{61} circle (0.2ex);
 \fill (2,-6) circle (0.2ex);
 \fill (3,-6) circle (0.2ex);
 \fill (4,-6) circle (0.2ex);
 \fill (5,-6) circle (0.2ex);
 \fill (6,-6) circle (0.2ex);
 \fill (7,-6) circle (0.2ex);
 \fill (8,-6) circle (0.2ex);
 \fill (9,-6) circle (0.2ex);
 \fill (10,-6) circle (0.2ex);
 \fill (1,-7) node[left]{71} circle (0.2ex);
 \fill (2,-7) circle (0.2ex);
 \fill (3,-7) circle (0.2ex);
 \fill (4,-7) circle (0.2ex);
 \fill (5,-7) circle (0.2ex);
 \fill (6,-7) circle (0.2ex);
 \fill (7,-7) circle (0.2ex);
 \fill (8,-7) circle (0.2ex);
 \fill (9,-7) circle (0.2ex);
 \fill (10,-7) circle (0.2ex);
 \fill (1,-8) node[left]{81} circle (0.2ex);
 \fill (2,-8) circle (0.2ex);
 \fill (3,-8) circle (0.2ex);
 \fill (4,-8) circle (0.2ex);
 \fill (5,-8) circle (0.2ex);
 \fill (6,-8) circle (0.2ex);
 \fill (7,-8) circle (0.2ex);
 \fill (8,-8) circle (0.2ex);
 \fill (9,-8) circle (0.2ex);
 \fill (10,-8) circle (0.2ex);
 \fill (1,-9) node[left]{91} circle (0.2ex);
 \fill (2,-9) circle (0.2ex);
 \fill (3,-9) circle (0.2ex);
 \fill (4,-9) circle (0.2ex);
 \fill (5,-9) circle (0.2ex);
 \fill (6,-9) circle (0.2ex);
 \fill (7,-9) circle (0.2ex);
 \fill (8,-9) circle (0.2ex);
 \fill (9,-9) circle (0.2ex);
 \fill (10,-9) node[right]{100} circle (0.2ex);
\end{tikzpicture}
\end{center}
\end{document}
