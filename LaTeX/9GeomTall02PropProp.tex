\documentclass[10pt,twoside]{article}
\usepackage[utf8]{inputenc}
\usepackage{amsmath}
\usepackage{amsfonts}
\usepackage{amssymb}
\usepackage[spanish,es-noshorthands]{babel}
\usepackage[T1]{fontenc}
\usepackage{lmodern}
\usepackage{graphicx,hyperref}
\usepackage{tikz,pgf}
\usepackage{multicol}
\usepackage{subfig}
\usepackage[papersize={6.5in,8.5in},width=5.5in,height=7in]{geometry}
\usepackage{fancyhdr}
\pagestyle{fancy}
\fancyhead[LE]{\url{www.autistici.org/mathgerman}}
\fancyhead[RE]{}
\fancyhead[RO]{matematicas.german@gmail.com}
\fancyhead[LO]{}

\author{Germ\'an Avenda\~no Ram\'irez~\thanks{Lic. Mat. U.D., M.Sc. U.N.}}
\title{\begin{minipage}{.2\textwidth}
\includegraphics[height=1.75cm]{Images/logo-colegio.png}\end{minipage}
\begin{minipage}{.55\textwidth}
\begin{center}
Taller 2\\
Propiedades de las proporciones\\
Geometría $9^{\circ}$
\end{center}
\end{minipage}\hfill
\begin{minipage}{.2\textwidth}
\includegraphics[height=1.75cm]{Images/logo-sed.png} 
\end{minipage}}
\date{}
\begin{document}
\maketitle
Nombre: \hrulefill Curso: \underline{\hspace*{44pt}} Fecha: \underline{\hspace*{2.5cm}}
\section*{Continuando con las propiedades de las proporciones}
\subsection*{Segunda propiedad}
Al cambiar los extremos o los medios de una proporci\'{o}n se obtiene otra proporci\'{o}n
\[\text{Si} \qquad \dfrac{a}{b}=\dfrac{c}{d}, \qquad \text{entonces} \qquad \dfrac{d}{b}=\dfrac{c}{a} \qquad \text{\'{o}} \qquad \dfrac{a}{c}=\dfrac{b}{d}  \]
\subsubsection*{Aplicaci\'{o}n de la propiedad dos}
\begin{enumerate}
\item Verificar la segunda propiedad y al finalizar verificar si forman una proporci\'{o}n \quad $\dfrac{2}{3}=\dfrac{4}{6}$
\item Hallar el valor de $x$ en la proporción, \qquad $\dfrac{1}{6}=\dfrac{5}{x}$
\end{enumerate}
\subsection*{Tercera propiedad}
Al invertir los términos de cada razón de una proporción se obtiene otra proporción
\[\text{Si}\qquad \dfrac{a}{b}=\dfrac{c}{d}, \qquad \text{entonces} \qquad \dfrac{b}{a}=\dfrac{d}{c} \]
\subsubsection*{Aplicaci\'{o}n de la tercera propiedad}
\begin{enumerate}
\item[3.] Verificar la tercera propiedad y al finalizar verificar si forman un proporción \quad $\dfrac{8}{5}=\dfrac{24}{15}$
\item[4.] Hallar el valor de $x$ en la proporción, \qquad $\dfrac{9}{x}=\dfrac{27}{6}$
\end{enumerate}
\subsection*{Cuarta propiedad}
La adición o sustracción del antecedente con el consecuente de la primera razón, es a su consecuente como la adición o sustracción del antecedente con el consecuente de la segunda razón, es a su consecuente.
\[\text{Sí} \qquad \dfrac{a}{b}=\dfrac{c}{d}, \qquad \text{entonces} \qquad \dfrac{a\pm b}{b}=\dfrac{c\pm d}{d} \]
\subsubsection*{Aplicaci\'{o}n de la cuarta propiedad}
\begin{enumerate}
\item[5.] Verificar la cuarta propiedad y al finalizar verificar si forman una proporción, \quad $\dfrac{4}{7}=\dfrac{12}{21}$
\item[6.] Hallar el valor de $x$ en la proporción, \qquad $\dfrac{x-4}{5}=\dfrac{5}{20}$
\end{enumerate}
\subsection*{Quinta propiedad}
La adición o sustracción de los antecedentes, es a la adición o sustracción de los consecuentes, como el antecedente es a su
consecuente de una de las razones.
\[\text{Sí} \qquad \dfrac{a}{b}=\dfrac{c}{d}, \qquad \text{entonces} \qquad \dfrac{a\pm c}{b\pm d}=\dfrac{a}{b} \qquad \text{ó} \qquad \dfrac{a\pm c}{b\pm d}=\dfrac{c}{d}\]
\subsubsection*{Aplicaci\'{o}n de la quinta propiedad}
\begin{enumerate}
\item[7.] Verificar la quinta propiedad y al finalizar verificar si forman una proporci\'{o}n, \quad $\dfrac{2}{3}=\dfrac{10}{15}$
\item[8.] Hallar el valor de $x$ en la proporci\'{o}n, \qquad $\dfrac{x+7}{3}=\dfrac{-16}{6}$
\end{enumerate}
\subsection*{Sexta propiedad}
De la proporción $a$ es a $n$ como $n$ es a $b$, diremos que $n$ es media proporcional entre $a$ y $b$
\[\text{Sí} \qquad \dfrac{a}{n}=\dfrac{n}{b} \qquad \text{es una proporción, entonces} \qquad\ n\cdot n=a\cdot b\]
\subsubsection*{Aplicaci\'{o}n de la sexta propiedad}
\begin{enumerate}
\item[9.] Verificar la sexta propiedad y al finalizar verificar si forman una proporción, \quad $\dfrac{16}{8}=\dfrac{8}{4}$
\item[10.] Hallar el valor de $x$ en la proporción \qquad $\dfrac{2}{x}=\dfrac{x}{8}$
\end{enumerate}
\subsection*{Resumiendo}
\begin{itemize}
\item[$P_{1}$:] $\dfrac{a}{b}=\dfrac{c}{d}\Leftrightarrow ad=bc$
\item[$P_{2}$:] $\dfrac{a}{b}=\dfrac{c}{d}\Rightarrow \dfrac{a}{c}=\dfrac{b}{d}$
\item[$P_{3}$:] $\dfrac{a}{b}=\dfrac{c}{d}\Rightarrow \dfrac{b}{a}=\dfrac{d}{c}$
\item[$P_{4}$:] $\dfrac{a}{b}=\dfrac{c}{d}\Rightarrow \dfrac{a\pm c}{b\pm d}=\dfrac{a}{b}$
\item[$P_{5}$:] $\dfrac{a}{b}=\dfrac{c}{d}\Rightarrow \dfrac{a\pm b}{b}=\dfrac{c\pm d}{d}$
\item[$P_{6}$:] $\dfrac{a}{b}=\dfrac{b}{d}\Rightarrow b\cdot b =a\cdot d$ \qquad $b$, es medio proporcional entre $a$ y $d$
\end{itemize}
\section*{Teorema de Thales}
\begin{minipage}{.5\textwidth}
Los segmentos determinados por rectas paralelas a dos rectas
secantes son proporcionales. En el gráfico se puede por ejemplo establecer la siguiente proporción
\[\dfrac{\overline{AB}}{\overline{CD}}=\dfrac{\overline{A'B'}}{\overline{C'D'}}\]
\end{minipage}
\begin{minipage}{.45\textwidth}
\usetikzlibrary{arrows}
\baselineskip=10pt
\hsize=6.3truein
\vsize=8.7truein
\definecolor{xdxdff}{rgb}{0.49,0.49,1}
\definecolor{uququq}{rgb}{0.25,0.25,0.25}
\definecolor{qqqqff}{rgb}{0,0,1}
\tikzpicture[scale=.75,line cap=round,line join=round,>=triangle 45,x=1.0cm,y=1.0cm]
\clip(-2.62,-0.32) rectangle (3.94,5.68);
\draw [domain=-2.62:3.94] plot(\x,{(-11-5*\x)/-1});
\draw [domain=-2.62:3.94] plot(\x,{(--18-6*\x)/1});
\draw [domain=-2.62:3.94] plot(\x,{(--35-0*\x)/7});
\draw [domain=-2.62:3.94] plot(\x,{(--7-0*\x)/7});
\draw [domain=-2.62:3.94] plot(\x,{(--26.44-0*\x)/7});
\draw [domain=-2.62:3.94] plot(\x,{(--17.34-0*\x)/7});
\fill [color=qqqqff] (-2,1) circle (1.5pt);
\draw[color=qqqqff] (-2.18,1.24) node {$D$};
\fill [color=uququq] (-1.2,5) circle (1.5pt);
\draw[color=uququq] (-1.44,5.22) node {$A$};
\fill [color=xdxdff] (-1.44,3.78) circle (1.5pt);
\draw[color=xdxdff] (-1.68,4.04) node {$B$};
\fill [color=xdxdff] (-1.7,2.48) circle (1.5pt);
\draw[color=xdxdff] (-1.96,2.72) node {$C$};
\fill [color=uququq] (2.17,5) circle (1.5pt);
\draw[color=uququq] (2.34,5.26) node {$A'$};
\fill [color=uququq] (2.37,3.78) circle (1.5pt);
\draw[color=uququq] (2.56,4.04) node {$B'$};
\fill [color=uququq] (2.59,2.48) circle (1.5pt);
\draw[color=uququq] (2.76,2.74) node {$C'$};
\fill [color=uququq] (2.83,1) circle (1.5pt);
\draw[color=uququq] (3.04,1.26) node {$D'$};
\endtikzpicture
\end{minipage}
\subsection*{Consecuencias del teorema}
\begin{enumerate}
\begin{minipage}{.45\textwidth}
\item Toda recta paralela a uno de los lados de un triángulo y que corta a los otros dos, divide a éstos en segmentos proporcionales. Esto es:
\[\dfrac{\overline{AC}}{\overline{DC}}=\dfrac{\overline{BC}}{\overline{EC}}\]
\end{minipage}
\begin{minipage}{.45\textwidth}
\definecolor{uququq}{rgb}{0.25,0.25,0.25}
\definecolor{xdxdff}{rgb}{0.49,0.49,1}
\definecolor{zzttqq}{rgb}{0.6,0.2,0}
\definecolor{qqqqff}{rgb}{0,0,1}
\tikzpicture[scale=.75,line cap=round,line join=round,x=1.0cm,y=1.0cm]
\clip(-3.74,0.12) rectangle (5.76,5.84);
\fill[color=zzttqq,fill=zzttqq,fill opacity=0.1] (-1,5) -- (-3,1) -- (4,3) -- cycle;
\draw [color=zzttqq] (-1,5)-- (-3,1);
\draw [color=zzttqq] (-3,1)-- (4,3);
\draw [color=zzttqq] (4,3)-- (-1,5);
\draw [domain=-3.74:5.76] plot(\x,{(-10.81-4*\x)/-2});
\draw [domain=-3.74:5.76] plot(\x,{(--10-4*\x)/-2});
\fill [color=qqqqff] (-1,5) circle (1.5pt);
\draw[color=qqqqff] (-1.24,5.08) node {$A$};
\fill [color=qqqqff] (-3,1) circle (1.5pt);
\draw[color=qqqqff] (-3.28,1.08) node {$B$};
\fill [color=qqqqff] (4,3) circle (1.5pt);
\draw[color=qqqqff] (4.22,2.96) node {$C$};
\fill [color=xdxdff] (-0.33,4.73) circle (1.5pt);
\draw[color=xdxdff] (-0.08,4.86) node {$D$};
\draw[color=black] (-0.1,5.56) node {$d$};
\draw[color=black] (4.98,4.58) node {$e$};
\fill [color=uququq] (-2.07,1.27) circle (1.5pt);
\draw[color=uququq] (-1.84,1.18) node {$E$};
\endtikzpicture
\end{minipage}

\begin{minipage}{.45\textwidth}
\item La bisectriz de un ángulo interior de un triángulo, divide al lado opuesto en segmentos proporcionales a los dos lados adyacentes
\end{minipage}
\
\end{enumerate}
\end{document}
