\documentclass[fleqn]{article}
\usepackage[spanish,es-noshorthands]{babel}
\usepackage[utf8]{inputenc} 
\usepackage[papersize={6.5in,8.5in},left=1cm, right=1cm, top=1.5cm, bottom=1.7cm]{geometry}
\usepackage{mathexam}
\usepackage{tikz,pgf}
\usepackage{amsmath}
\usepackage{graphicx}
\usepackage{multicol}
\ExamClass{\includegraphics[height=16pt]{Images/logo-sed.png} Aritmética $6^{\circ}$}
\ExamName{``Potenciación, radicación y logaritmación''}
\ExamHead{\includegraphics[height=16pt]{Images/logo-colegio.png} IEDAB}
\newcommand{\LineaNombre}{%
\par
\vspace{\baselineskip}
Nombre:\hrulefill \; Curso: \underline{\hspace*{48pt}} \; Fecha: \underline{\hspace*{2.5cm}} \relax
\par}
\let\ds\displaystyle

\begin{document}
\ExamInstrBox{
\textit{Respuesta sin justificar mediante procedimiento no será tenida en cuenta en la calificación. Escriba sus respuestas en el espacio indicado. Tiene 45 minutos para contestar esta prueba.}}
\LineaNombre
\begin{enumerate}
\item Si se forma un cuadrado con 49 cuadrados como éste \tikz \draw (0,0) rectangle (.4,.4); ¿cuántos cuadrados caben por cada lado? Dibújelo y represente su respuesta por medio de una operación.\noanswer
\item Si se forma un rectángulo con 20 cuadrados como éste \tikz \draw (0,0) rectangle (.4,.4);, ¿cuántos cuadrados debería agregar para formar un cuadrado? Dibuje. ¿Cuántos cuadrados quedarían por cada lado? Represente por medio de una operación.\noanswer
\item Escriba el número que corresponde a cada rectángulo. Justifique su respuesta
\begin{enumerate}\begin{multicols}{2}
  \item $ \tikz \draw (0,0) rectangle (.6,.4);^2=36 $, \; Just: \underline{\hspace*{2cm}}
  \item $ \tikz \draw (0,0) rectangle (.6,.4);^3=343 $, \; Just: \underline{\hspace*{2cm}}
  \item $ \tikz \draw (0,0) rectangle (.6,.4);^2=64 $, \; Just: \underline{\hspace*{2cm}}
  \item $ \tikz \draw (0,0) rectangle (.6,.4);^5=32 $, \; Just: \underline{\hspace*{2cm}}
\end{multicols}
\end{enumerate}
\item Resuelvo los siguientes ejercicios y justifico la respuesta:
\begin{enumerate}\begin{multicols}{2}
  \item $ \tikz \draw (0,0) rectangle (1,.4);^2=121 $, \; Just: \underline{\hspace*{2cm}}
  \item $ \tikz \draw (0,0) rectangle (1,.4);^5=243 $, \; Just: \underline{\hspace*{2cm}}
  \item $ \tikz \draw (0,0) rectangle (1,.4);^2=144 $, \; Just: \underline{\hspace*{2cm}}
  \item $ \tikz \draw (0,0) rectangle (1,.4);^2=225 $, \; Just: \underline{\hspace*{2cm}}
  \item $ \sqrt[4]{256}= $ \tikz \draw (0,0) rectangle (1,.4);, \; Just: \underline{\hspace*{2cm}}
  \item $ \sqrt[3]{512}= $ \tikz \draw (0,0) rectangle (1,.4);, \; Just: \underline{\hspace*{2cm}}
\end{multicols}
\end{enumerate}
\newpage
\item Escriba en forma de potencia cada uno de los siguientes logaritmos:
  \begin{enumerate}\begin{multicols}{2}
    \item $ \log_3{81}=4 $: \underline{\hspace{3cm}}
    \item $ \log_5{125}=3 $: \underline{\hspace{3cm}}
  \end{multicols}
  \end{enumerate}
  \item Determine el área de una cancha de baloncesto que mide 28 metros de largo por 15 metros de ancho.\noanswer
  \item Si quisiéramos poner baldosas a un salón de reuniones que mide 12 metros de largo por 8 metros de ancho, ¿cuántas baldosas de 25 cm de lado se necesitarían?\noanswer
  \item Halle el volumen de una piscina de 6 metros de largo, 4 de ancho y 2 metros de profundidad.\noanswer
   \item Complete el dibujo y resuelva la operación
 
 \begin{minipage}{0.45\textwidth}
\begin{tikzpicture}[scale=.75]
\draw (0,0,2)--(4,0,2)--(4,3,2)--(0,3,2)--cycle;
\draw (0,3,0)--(4,3,0)--(4,0,0);
\draw (0,3,2)--(0,3,0);
\draw (4,3,2)--(4,3,0);
\draw (4,0,2)--(4,0,0);
\node [below] at (2,0,2) {4};
\node [right] at (4,1.5,0) {3};
\node [right] at (4,0,1) {2};
\draw (1,0,2)--(1,3,2)--(1,3,0);
\end{tikzpicture}
\end{minipage}\hfill
\begin{minipage}{0.5\textwidth}
  $ 4\times2\times3= $\underline{\hspace{2cm}}
\end{minipage}
\item Haga la operación y escriba el número que corresponda en cada cuadrado
\begin{enumerate}\begin{multicols}{3}
  \item $ 3^2= $  \underline{\hspace*{1cm}} $=$ \tikz \draw (0,0) rectangle (.6,.4);
  \item $ 2^3= $ \underline{\hspace*{1cm}} $=$ \tikz \draw (0,0) rectangle (.6,.4);
  \item $ 4^2= $ \underline{\hspace*{1cm}} $=$ \tikz \draw (0,0) rectangle (.6,.4);
  \item $ 2^4= $ \underline{\hspace*{1cm}} $=$ \tikz \draw (0,0) rectangle (.6,.4);
  \item $ 13^2= $ \underline{\hspace*{1cm}} $=$ \tikz \draw (0,0) rectangle (.6,.4);
  \item $ 4^5= $ \underline{\hspace*{1cm}} $=$ \tikz \draw (0,0) rectangle (.6,.4);
  \end{multicols}
\end{enumerate}
 \end{enumerate}
\end{document}
