\documentclass[10pt,letterpaper,addpoints]{exam}
\usepackage[utf8]{inputenc}
\usepackage[spanish]{babel}
\usepackage{hyperref}
\usepackage{amsmath}
\usepackage{amsfonts}
\usepackage{amssymb}
\usepackage{graphicx}
\usepackage{multicol}
\usepackage{pgf,tikz}
\usepackage[left=2cm,right=2cm,top=2cm,bottom=2cm]{geometry}
%\printanswers
\begin{document}
\title{\begin{minipage}{.2\textwidth}
        \includegraphics[height=1.75cm]{Images/logo-colegio.png}
       \end{minipage}
\begin{minipage}{.55\textwidth}
 \begin{center}
Prueba Bimestral I \\Matemáticas 9$^{\circ}$\\Formulario \textbf{A}
\end{center}
\end{minipage}
\begin{minipage}{.2\textwidth}
\includegraphics[height=1.75cm]{Images/logo-sed.png} 
\end{minipage}
}
\author{Germ\'{a}n Avendaño Ram\'{i}rez~\thanks{Lic. Mat. U.D. y M.Sc. U.N.}}
\date{}
\maketitle
\begin{center}
\fbox{\fbox{\parbox{5.5in}{\centering
Esta prueba consta de preguntas tipo I (selección múltiple con única respuesta). Marque la respuesta que considere correcta en el cuadro de respuestas dado. Las operaciones debe hacerlas en otra hoja. \textit{\textbf{No} marque ni dañe este formulario}}}}
\end{center}
\vspace*{.2in}
\begin{multicols}{2}
\begin{questions}
\question
Una cinta de longitud $7\frac{1}{5}$ dec\'imetros se divide en 6 partes de igual longitud. Cada parte mide:
\begin{choices}
\CorrectChoice $\frac{6}{5}$ dec\'imetros
\choice $\frac{5}{6}$ dec\'imetros
\choice $\frac{7}{5}$ dec\'imetros
\choice 3 dec\'imetros
\end{choices}
\question
Un obrero empieza a trabajar el $1^\circ$ de enero de 2005 con un sueldo de \$800.000 mensuales. Cada 6 meses recibe un aumento de 5\% sobre el sueldo anterior. En febrero de 2006, el obrero recibirá:

\begin{choices}
\choice \$840000
\CorrectChoice \$882000
\choice \$880000
\choice \$892000
\end{choices}
\question Seis amigos acordaron comprar su afiche favorito en partes iguales pero uno de ellos se arrepintió y a última hora cada uno de los restantes tuvo que pagar \$60 más. ¿Cuánto costaba el afiche?
\begin{choices}
\choice \$ 1200
\choice \$ 1400
\choice \$ 1500
\CorrectChoice \$ 1800
\end{choices}
\question Encuentre el número cuyo duplo más 8 es igual a 46

\begin{oneparchoices}
\choice 15
\CorrectChoice 19
\choice 18
\choice 20
\end{oneparchoices}
\question Si Ángela le da a Luisa \$1, ambas tienen lo mismo. Y si Luisa le da a Ángela \$1, Ángela tendrá el triple de lo que le queda a Luisa. ¿Cuánto tiene Ángela?

\begin{oneparchoices}
\choice \$8
\choice \$6
\CorrectChoice \$5
\choice \$7
\end{oneparchoices}
\question La menor de las fracciones es:

\begin{oneparchoices}
\choice $\dfrac{2}{3}$
\choice $\dfrac{5}{8}$
\CorrectChoice $\dfrac{3}{5}$
\choice $\dfrac{51}{80}$
\end{oneparchoices}
\question Entre las operaciones propuestas, la que da como resultado una fracción irreducible es:
\begin{choices}
\CorrectChoice $4+2^{-1}$
\choice $4-(\frac{2}{3})^{0}$
\choice $\frac{2}{3}+(\frac{8}{3}-\frac{1}{3})$
\choice $\sqrt{(\frac{1}{4})}+\frac{1}{2}$
\end{choices}
\question Si $a$, $b$, $c$, $d$ son números negativos, la fracción $\dfrac{abc}{d}$ será un número:

\begin{choices}
\CorrectChoice Positivo
\choice Primo
\choice Impar
\choice Negativo
\end{choices}
\question El precio de un radio se rebaja 20\%. Para volverlo al precio original, el nuevo precio debe aumentarse en:

\begin{oneparchoices}
\choice 18\%
\CorrectChoice 25\%
\choice 21\%
\choice 20\%
\end{oneparchoices}
\question Se debe empapelar una pared con papel de colgadura cuadrado (área=160 $m^{2}$). Después de hacerlo se encuentra que es necesario cortar 2 metros desde uno de los extremos para fijarlo correctamente. ¿Cuál es el área de la pared en $m^{2}$?

\begin{oneparchoices}
\choice 117
\choice 121
\CorrectChoice 143
\choice 165
\end{oneparchoices}
\question En una encuesta se encontraron los siguientes datos:\\ \{1, 2, 2, 2, 3, 3, 2, 1, 2, 1, 2, 1, 3, 2, 4, 2, 1, 1, 2, 1\}. La moda es:

\begin{oneparchoices}
\choice 3
\choice 4
\choice 1
\CorrectChoice 2
\end{oneparchoices}
\question La media de los siguientes datos\\ \{1, 2, 3, 2, 3, 2, 3, 3, 1, 1, 2, 2, 3, 1, 1\} es:

\begin{oneparchoices}
\CorrectChoice 2
\choice 1
\choice 3
\choice 15
\end{oneparchoices}
\question La mediana de los siguientes datos\\ \{1, 3, 2, 3, 2, 3, 2, 3, 1, 1, 2, 1, 2, 3, 1\} es:

\begin{oneparchoices}
\choice 1
\CorrectChoice 2
\choice 3
\choice 15
\end{oneparchoices}
\question La frecuencia absoluta de la nota 2 en la siguiente tabla es:

\begin{center}
\begin{tabular}{|c|c|}
\hline 
Nota & No alumnos \\ 
\hline 
2 & 2 \\ 
\hline 
3 & 1 \\ 
\hline 
4 & 2 \\ 
\hline 
\end{tabular} 
\end{center}

\begin{oneparchoices}
\CorrectChoice 2
\choice $\frac{2}{5}$
\choice $\frac{1}{2}$
\choice 3
\end{oneparchoices}
\question En el trapecio ABCD que muestra la figura, $BC=4$ cm y AD mide el triple de BC. Si AB y CD miden cada uno 5 cm, el área del trapecio es:
\begin{center}
%Uncomment next line if XeTeX is used
%\def\pgfsysdriver{pgfsys-xetex.def}
\usetikzlibrary{arrows}
\baselineskip=10pt
\hsize=6.3truein
\vsize=8.7truein
\definecolor{zzttqq}{rgb}{0.27,0.27,0.27}
\definecolor{qqqqff}{rgb}{0.33,0.33,0.33}
\tikzpicture[scale=.5,line cap=round,line join=round,x=1.0cm,y=1.0cm]
\clip(-4.55,-0.4) rectangle (8.86,3.77);
\fill[color=zzttqq,fill=zzttqq,fill opacity=0.1] (-4,0) -- (0,3) -- (4,3) -- (8,0) -- cycle;
\draw [color=zzttqq] (-4,0)-- (0,3);
\draw [color=zzttqq] (0,3)-- (4,3);
\draw [color=zzttqq] (4,3)-- (8,0);
\draw [color=zzttqq] (8,0)-- (-4,0);
\fill [color=qqqqff] (-4,0) circle (1.5pt);
\draw[color=qqqqff] (-3.77,0.38) node {$A$};
\fill [color=qqqqff] (0,3) circle (1.5pt);
\draw[color=qqqqff] (0.25,3.38) node {$B$};
\fill [color=qqqqff] (4,3) circle (1.5pt);
\draw[color=qqqqff] (4.24,3.38) node {$C$};
\fill [color=qqqqff] (8,0) circle (1.5pt);
\draw[color=qqqqff] (8.23,0.38) node {$D$};
\endtikzpicture
\end{center}
\begin{oneparchoices}
\choice 20 cm$^{2}$
\CorrectChoice 24 cm$^{2}$
\choice 36 cm$^{2}$
\choice 40 cm$^{2}$
\end{oneparchoices}
\question En la figura, el perímetro del cuadrado, si el radio del círculo inscrito mide 4 cm, es:
\begin{center}
%Uncomment next line if XeTeX is used
%\def\pgfsysdriver{pgfsys-xetex.def}
\usetikzlibrary{arrows}
\baselineskip=10pt
\hsize=6.3truein
\vsize=8.7truein
\definecolor{zzttqq}{rgb}{0.27,0.27,0.27}
\tikzpicture[scale=.5,line cap=round,line join=round,x=1.0cm,y=1.0cm]
\clip(-7.46,-6.07) rectangle (10.06,7.28);
\fill[color=zzttqq,fill=zzttqq,fill opacity=0.1] (-4,-4) -- (4,-4) -- (4,4) -- (-4,4) -- cycle;
\draw [color=zzttqq] (-4,-4)-- (4,-4);
\draw [color=zzttqq] (4,-4)-- (4,4);
\draw [color=zzttqq] (4,4)-- (-4,4);
\draw [color=zzttqq] (-4,4)-- (-4,-4);
\draw(0,0) circle (4cm);
\draw (0,0)-- (-3.2,-2.4);
\node[rotate=40]at(-1.6,-1.5) {4cm};
\endtikzpicture
\end{center}
\begin{oneparchoices}
\CorrectChoice 16
\choice 12
\choice 32
\choice 20
\end{oneparchoices}
\question El valor de $h$ en la figura siguiente es:
\begin{center}
\begin{tikzpicture}
\draw (0,0) -- (0,7) -- (5,0) -- cycle;
\draw (3.5,0) -- (3.5,2.1);
\node[below] at (2.5,0) {6};
\node[left] at (0,3.5) {h};
\node[below] at (4.25,0) {2};
\node[left] at (3.5,1.05) {7};
\end{tikzpicture}
\end{center}
\begin{oneparchoices}
\choice 28
\choice 30
\choice 25
\CorrectChoice 21
\end{oneparchoices}
\question Calcula el valor del lado de un cuadrado cuya \'area es 60 cm$^{2}$

\begin{oneparchoices}
\CorrectChoice $2\sqrt{15}$ cm
\choice 30 cm
\choice 6 cm
\choice 4 cm
\end{oneparchoices}

\uplevel{Conteste las preguntas \ref{quest:10}-\ref{quest:11} teniendo presente la figura (tri\'angulo)}
\begin{center}
\begin{tikzpicture}
\draw (0,0) -- (4,0) -- (0,3) -- cycle;
\node[left] at (0,1.5) {t};
\node[below] at (2,0) {t+1};
\node[above] at (2,1.6) {t+2};
\end{tikzpicture}
\end{center}
\question \label{quest:10} El perímetro de este triángulo es:

\begin{choices}
\choice $4t+3$
\choice $3t-3$
\CorrectChoice $3t+3$
\choice $4t-3$
\end{choices}
\question \label{quest:11}
El \'area del tri\'angulo de la figura mostrada es igual a:
\begin{choices}
\choice $t^{2}/2$
\CorrectChoice $t^{2}/2+t/2$
\choice $t^{2}+t/2$
\choice $t^{2}+t$
\end{choices}
\end{questions}
\end{multicols}
\end{document}