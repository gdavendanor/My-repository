% $Header: /home/vedranm/bitbucket/beamer/solutions/short-talks/speaker_introduction-ornate-2min.en.tex,v 90e850259b8b 2007/01/28 20:48:30 tantau $

\documentclass{beamer}

% This file is a solution template for:

% - Introducing another speaker.
% - Talk length is about 2min.
% - Style is ornate.



% Copyright 2004 by Till Tantau <tantau@users.sourceforge.net>.
%
% In principle, this file can be redistributed and/or modified under
% the terms of the GNU Public License, version 2.
%
% However, this file is supposed to be a template to be modified
% for your own needs. For this reason, if you use this file as a
% template and not specifically distribute it as part of a another
% package/program, the author grants the extra permission to freely
% copy and  modify this file as you see fit and even to delete this
% copyright notice. 


\mode<presentation>
{
  \usetheme{Warsaw}
  % or ...

  \setbeamercovered{transparent}
  % or whatever (possibly just delete it)
}  
\usepackage[spanish,es-noshorthands]{babel}
% or whatever

\usepackage[utf8]{inputenc}
% or whatever

\usepackage{lmodern}
\usepackage[T1]{fontenc}
% Or whatever. Note that the encoding and the font should match. If T1
% does not look nice, try deleting the line with the fontenc.



\begin{document}

\begin{frame}{Alejandro Álvarez Gallego}{Acerca de nuestro invitado}

  \begin{itemize}
  \item
    Filiación

    % Examples:
    \begin{itemize}
    \item
      Profesor, Universidad Pedagógica Nacional.
    \item
      Grupo Historia de la Práctica Pedagógica
    \end{itemize}
  \item
    Formación Académica
    % Optional. Use this if it is appropriate to slightly flatter the
    % speaker, for example if the speaker has been invited. 
    % Using subitems, list things that make the speaker look
    % interesting and competent.

    % Examples:
    \begin{itemize}
    \item
      Pregrado/Universitario Universidad Pedagógica Nacional - U.P.N.\\
Licenciatura En Ciencias Sociales - 1976 - 1981
    \item
	Maestria/Magister Pontificia Universidad Javeriana - Puj - Sede Bogotá\\
Estudios Políticos --1982 - 1984\\
El estado liberal en América Latina
    \item
      Doctorado Universidad Nacional de Educación a Distancia\\
Historia de La Educación y Educación Comparada\\
Enero de 1996 - Febrero de 2008\\
Las Ciencias Sociales y el Currículo Escolar: Colombia 1930-1960.\\
    \end{itemize}
    \end{itemize}
        \end{frame}
            \begin{frame}{La mirada empresarial de la educación}{A propósito del informe "Compartir"}
El grupo PUYA José Antequera, los invita a la conferencia sobre el estudio:\\                
\begin{center}
      LA MIRADA EMPRESARIAL DE LA EDUCACIÓN\\
A PROPÓSITO DEL INFORME “COMPARTIR”
\end{center}
    \end{itemize}
  \end{itemize}
  Lugar: Auditorio FECODE Cra 13A \# 34-34\\
  Hora: 2:00 p.m.\\
  Fecha: sábado 24 de mayo de 2014.\\
  Invita: Grupo PUYA
\end{frame}
\end{document}


