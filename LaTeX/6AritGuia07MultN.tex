\documentclass[10pt,twoside]{article}
\usepackage[utf8]{inputenc}
\usepackage{amsmath}
\usepackage{amsfonts}
\usepackage{amssymb}
\usepackage[spanish,es-noshorthands]{babel}
\usepackage[T1]{fontenc}
\usepackage{lmodern}
\usepackage{graphicx,hyperref}
\usepackage{tikz,pgf}
\usepackage{multicol}
\usepackage{subfig}
\usepackage[papersize={6.5in,8.5in},width=5.5in,height=7in]{geometry}
\usepackage{fancyhdr}
\pagestyle{fancy}
\fancyhead[LE]{\includegraphics[height=12pt]{Images/logo-colegio.png} Aritm\'etica $6^{\circ}$}
\fancyhead[RE]{}
\fancyhead[RO]{\textit{Germ\'an Avenda\~no Ram\'irez, Lic. U.D., M.Sc. U.N.}}
\fancyhead[LO]{}

\author{Germ\'an Avenda\~no Ram\'irez, Lic. U.D., M.Sc. U.N.}
\title{\begin{minipage}{.2\textwidth}
\includegraphics[height=1.75cm]{Images/logo-colegio.png}\end{minipage}
\begin{minipage}{.55\textwidth}
\begin{center}
Taller 07, Multiplicación en $\mathbb{N}$ \\
Aritm\'etica $6^{\circ}$
\end{center}
\end{minipage}\hfill
\begin{minipage}{.2\textwidth}
\includegraphics[height=1.75cm]{Images/logo-sed.png} 
\end{minipage}}
\date{}
\begin{document}
\maketitle
Nombre: \hrulefill Curso: \underline{\hspace*{44pt}} Fecha: \underline{\hspace*{2.5cm}}
\section*{Multiplicaci\'on en $\mathbb{N}$}
\subsection*{Acitividad 1}
Realice en forma individual en su cuaderno lo siguiente:
\begin{itemize}
 \item Tome una hoja de papel. D\'oblela de manera que queden, bien 4 filas y 8 columnas o bien, 8 filas y 4 columnas as\'i:
\begin{center}
\begin{tikzpicture}
\draw[help lines](0,0)grid(8,4);
\node[above] at (0.5,4){Columna 1};
\node[left]at(0,3.5){Fila 1};
\end{tikzpicture}
\end{center}
\item Responda las siguientes preguntas:
\begin{itemize}
 \item ?`En cu\'antas partes queda dividido el papel?
\item ?`Cu\'antos cuadrados tiene cada columna?
\item ?`Cu\'antos cuadrados tiene cada fila?
\item ?`Cu\'anto es 8 veces 4?, es decir, $4+4+4+4+4+4+4+4$
\item ?`Cu\'anto es 4 veces 8?, es decir, $8+8+8+8$.
\item ?`C\'omo se escribe abreviadamente 4 veces 8?, ?`8 veces 4?
\item ?`Qu\'e resultado se obtiene?
\end{itemize}
\item Recuerda:
\end{itemize}
La operación, que es una suma abreviada de sumandos iguales, se llama MULTIPLICACIÓN.
La multiplicación entre dos números naturales $a$ y $b$, se simboliza así:
\[a\cdot b \qquad \text{ó} \qquad a\times b, \qquad \mbox{8 veces 4}=8\times 4\]

\begin{enumerate}
\item 
\end{enumerate}

\end{document}
