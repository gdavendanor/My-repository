\documentclass[letterpaper,11pt,twoside]{article}
\usepackage[utf8]{inputenc}
\usepackage{amsmath,amsfonts,amssymb,amsthm,latexsym}
\usepackage[spanish,es-noshorthands]{babel}
\usepackage[T1]{fontenc}
\usepackage{lmodern}
\usepackage{graphicx,hyperref}
\usepackage{tikz,pgf}
\usepackage{marvosym}
\usepackage{multicol}
\usepackage{fancyhdr}
\usepackage[height=9.5in,width=7in]{geometry}
\usepackage{fancyhdr}
\pagestyle{fancy}
\fancyhead[LE]{matematicas.german@gmail.com}
\fancyhead[RE]{\url{https://www.autistici.org/mathgerman}}
\fancyhead[RO]{\url{https://www.autistici.org/mathgerman}}
\fancyhead[LO]{\Email matematicas.german@gmail.com}

\author{Germ\'an Avenda\~no Ram\'irez~\thanks{Lic. Mat. U.D., M.Sc. U.N.}}
\title{\begin{minipage}{.2\textwidth}
\includegraphics[height=1.75cm]{Images/logo-colegio.png}\end{minipage}
\begin{minipage}{.55\textwidth}
\begin{center}
Animaplano 06\\
Matemáticas $9^{\circ}$
\end{center}
\end{minipage}\hfill
\begin{minipage}{.2\textwidth}
\includegraphics[height=1.75cm]{Images/logo-sed.png} 
\end{minipage}}
\date{}
\thispagestyle{plain}
\begin{document}
\maketitle
Nombre: \hrulefill Curso: \underline{\hspace*{44pt}} Fecha: \underline{\hspace*{2.5cm}}
\begin{multicols}{2}
\section*{Cuestionario}
\begin{enumerate}
\item El cuádruple del cuarto número primo
\item Si $2x=10$, entonces $x^{2}+3x+18=$?
\item Si $m+n=51$, entonces $(m+10)+(n-4)=$?
\item La moda entre: (69, 67, 55, 67, 32, 67) ¿es?
\item Resuelva $(13i^{2})(5i^{2})=$
\item En años, reste 5 lustros a 1 siglo
\item Número primo de dos cifras que al sumar sus cifras da 13
\item El 30\% de 290
\item La quinta parte de 480
\item Si $2t^{2}=18$, entonces $30t=$?
\item La suma de dos números es 111 y el mayor excede al menor en 9 unidades. ¿El número mayor es?
\item Del anterior punto, ¿el número menor es?
\item El décimo sexto número primo
\item Si $\pi \approx 3$, sume 1 cm$^{3}$, al volumen de una esfera de radio 2 cm.\footnote{El volumen de una esfera se calcula mediante la ecuación $V_{E}=\frac{4}{3}\pi r^{3}$}
\item Las edades de Luis y Juan suman 78 años y Luis es 10 años menor que Juan. ¿La edad de Luis es?
\item ¿La edad de Juan es?
\item 19(31)50, \quad 33($n$)88. El número $n$ en la secuencia es:
\item Un triángulo rectángulo de catetos 3 y 4. ¿El valor de la hipotenusa al cuadrado?
\item Determine $2^{2}\cdot 2^{2}\cdot 2^{0}=$
\item El doble del sexto número primo
\item $1\frac{1}{2}+17\frac{1}{2}+2\frac{1}{2}+15\frac{1}{2}=$
\item El área de un triángulo de base 9 y altura 6
\item Si $y=16x+5$, entonces $m=$?
\item El ángulo complementario\footnote{Los ángulos complementarios son aquellos ángulos cuyas medidas suman 90$^{\circ}$} de 62$^{\circ}$
\item El octavo número primo
\item El quinto número primo
\item El m.c.d entre 8 y 15
\item La hipotenusa de un triángulo cuyos catetos miden 12 y 5
\item Las edades de Carlos y Juliana suman 112 años. Si Carlos es 10 años mayor que Juliana, ¿cuántos años tiene Juliana?
\item ¿Cuál es la edad de Carlos?
\item El décimo quinto número primo.
\end{enumerate}
\end{multicols}
\newpage
\section*{Animaplano}
\begin{tikzpicture}
%%Animaplano (0,0)--(9,9)
 \fill (0,0) node[above]{0} circle (0.2ex);
 \fill (1,0) node[above]{1} circle (0.2ex);
 \fill (2,0) node[above]{2} circle (0.2ex);
 \fill (3,0) node[above]{3} circle (0.2ex);
 \fill (4,0) node[above]{4} circle (0.2ex);
 \fill (5,0) node[above]{5} circle (0.2ex);
 \fill (6,0) node[above]{6} circle (0.2ex);
 \fill (7,0) node[above]{7} circle (0.2ex);
 \fill (8,0) node[above]{8} circle (0.2ex);
 \fill (9,0) node[above]{9} circle (0.2ex);
 \fill (0,-1) node[left]{10} circle (0.2ex);
 \fill (1,-1) circle (0.2ex);
 \fill (2,-1) circle (0.2ex);
 \fill (3,-1) circle (0.2ex);
 \fill (4,-1) circle (0.2ex);
 \fill (5,-1) circle (0.2ex);
 \fill (6,-1) circle (0.2ex);
 \fill (7,-1) circle (0.2ex);
 \fill (8,-1) circle (0.2ex);
 \fill (9,-1) circle (0.2ex);
 \fill (0,-2) node[left]{20} circle (0.2ex);
 \fill (1,-2) circle (0.2ex);
 \fill (2,-2) circle (0.2ex);
 \fill (3,-2) circle (0.2ex);
 \fill (4,-2) circle (0.2ex);
 \fill (5,-2) circle (0.2ex);
 \fill (6,-2) circle (0.2ex);
 \fill (7,-2) circle (0.2ex);
 \fill (8,-2) circle (0.2ex);
 \fill (9,-2) circle (0.2ex);
 \fill (0,-3) node[left]{30} circle (0.2ex);
 \fill (1,-3) circle (0.2ex);
 \fill (2,-3) circle (0.2ex);
 \fill (3,-3) circle (0.2ex);
 \fill (4,-3) circle (0.2ex);
 \fill (5,-3) circle (0.2ex);
 \fill (6,-3) circle (0.2ex);
 \fill (7,-3) circle (0.2ex);
 \fill (8,-3) circle (0.2ex);
 \fill (9,-3) circle (0.2ex);
 \fill (0,-4) node[left]{40} circle (0.2ex);
 \fill (1,-4) circle (0.2ex);
 \fill (2,-4) circle (0.2ex);
 \fill (3,-4) circle (0.2ex);
 \fill (4,-4) circle (0.2ex);
 \fill (5,-4) circle (0.2ex);
 \fill (6,-4) circle (0.2ex);
 \fill (7,-4) circle (0.2ex);
 \fill (8,-4) circle (0.2ex);
 \fill (9,-4) node[right]{49} circle (0.2ex);
 \fill (0,-5) node[left]{50} circle (0.2ex);
 \fill (1,-5) circle (0.2ex);
 \fill (2,-5) circle (0.2ex);
 \fill (3,-5) circle (0.2ex);
 \fill (4,-5) circle (0.2ex);
 \fill (5,-5) circle (0.2ex);
 \fill (6,-5) circle (0.2ex);
 \fill (7,-5) circle (0.2ex);
 \fill (8,-5) circle (0.2ex);
 \fill (9,-5) circle (0.2ex);
 \fill (0,-6) node[left]{60} circle (0.2ex);
 \fill (1,-6) circle (0.2ex);
 \fill (2,-6) circle (0.2ex);
 \fill (3,-6) circle (0.2ex);
 \fill (4,-6) circle (0.2ex);
 \fill (5,-6) circle (0.2ex);
 \fill (6,-6) circle (0.2ex);
 \fill (7,-6) circle (0.2ex);
 \fill (8,-6) circle (0.2ex);
 \fill (9,-6) circle (0.2ex);
 \fill (0,-7) node[left]{70} circle (0.2ex);
 \fill (1,-7) circle (0.2ex);
 \fill (2,-7) circle (0.2ex);
 \fill (3,-7) circle (0.2ex);
 \fill (4,-7) circle (0.2ex);
 \fill (5,-7) circle (0.2ex);
 \fill (6,-7) circle (0.2ex);
 \fill (7,-7) circle (0.2ex);
 \fill (8,-7) circle (0.2ex);
 \fill (9,-7) circle (0.2ex);
 \fill (0,-8) node[left]{80} circle (0.2ex);
 \fill (1,-8) circle (0.2ex);
 \fill (2,-8) circle (0.2ex);
 \fill (3,-8) circle (0.2ex);
 \fill (4,-8) circle (0.2ex);
 \fill (5,-8) circle (0.2ex);
 \fill (6,-8) circle (0.2ex);
 \fill (7,-8) circle (0.2ex);
 \fill (8,-8) circle (0.2ex);
 \fill (9,-8) circle (0.2ex);
 \fill (0,-9) node[left]{90} circle (0.2ex);
 \fill (1,-9) circle (0.2ex);
 \fill (2,-9) circle (0.2ex);
 \fill (3,-9) circle (0.2ex);
 \fill (4,-9) circle (0.2ex);
 \fill (5,-9) circle (0.2ex);
 \fill (6,-9) circle (0.2ex);
 \fill (7,-9) circle (0.2ex);
 \fill (8,-9) circle (0.2ex);
 \fill (9,-9) node[right]{99} circle (0.2ex);
 \draw (8,-2)--(8,-5)--(7,-5)--(7,-6)--(5,-6)--(5,-7)--(7,-6)--(7,-8)--(6,-9)--(0,-9)--(0,-6)--(1,-5)--(3,-5)--(3,-3)--(4,-3)--(4,-4)--(5,-5)--(5,-2)--(6,-1)--(6,-2)--(7,-3)--(7,-2)--(6,-1)--(8,-2)--(9,-1)--(1,-1)--(1,0)--(3,-1)--(1,-5)--(1,-6)--(4,-9);
\end{tikzpicture}
\end{document}
