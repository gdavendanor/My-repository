\documentclass[10pt,letterpaper]{article}
\usepackage[utf8]{inputenc}
\usepackage[spanish]{babel}
\usepackage{lmodern}
\author{Germán Avendaño Ramírez \thanks{Representante de los docentes jornada mañana al Consejo Directivo}}

\title{Reunión de consejo directivo}
\begin{document}
\maketitle
\section*{Orden del día}
\begin{enumerate}
\item Verificación del quorum
\item Lectura del acta enterior
\item Informe del rector
\item Lista de útiles
\item Varios
\end{enumerate}
\section*{Desarrollo de la agenda}
\subsection*{3. Informe financiero}
Se dejaron de percibir \$14'227\,086 por transferencias y otros. Quedó un excedente de \$1'705\,439, el cual se invertirá así: \$250\,000 gastos de computador, \$655.000 mantenimiento de equipos y \$800\,000 mantenimiento de la entidad.

Al colegio le han girado hasta ahora, \$9'140\,650 según Resolución 003 del 15 de enero de 2018, que se supone corresponden al 30\% de lo que se debe recibir este año por parte de la SED. Por tanto, se presume una reducción presupuestal, teniendo en cuenta que se esperaban \$38 millones
\subsection*{Elecciones de los representantes del gobierno escolar}
El próximo viernes 2 de febrero, se haría la elección de los representantes al consejo de padres. El 12 de febrero se reuniría el consejo de padres para elegir sus representantes a los diferentes comités y consejo directivo.
\subsection*{Lista de útiles escolares}
Se recibe al representante de una editorial, para apreciar la oferta que se tiene de los textos que vamos a sugerir para básica primaria y las áreas en secundaria.
\subsection*{Varios}
Se aprueba diseñar la encuesta para aplicar a los padres de familia, en torno al cambio de nombre de la institución y hacerle un homenaje a José Benigno Segura. El martes se entregará el diseño del formato para la encuesta a padres de familia.
\end{document}