\documentclass[10pt,addpoints]{exam}
\usepackage[utf8]{inputenc}
\usepackage[spanish,es-noshorthands]{babel}
\usepackage{hyperref}
\usepackage{amsmath}
\usepackage{amsfonts}
\usepackage{amssymb}
\usepackage{graphicx}
\usepackage{tikz}
\usepackage{multicol}
\usepackage[papersize={5.5in,8.5in},total={4.5in,7.25in},centering]{geometry}
%\printanswers
\begin{document}
\title{\begin{minipage}{.2\textwidth}
        \includegraphics[height=1.75cm]{Images/logo-colegio.png}
       \end{minipage}
\begin{minipage}{.55\textwidth}
 \begin{center}
Quiz \\Ética $11^{\circ}$
\end{center}
\end{minipage}
\begin{minipage}{.2\textwidth}
\includegraphics[height=1.75cm]{Images/logo-sed.png} 
\end{minipage}
}
\author{Germ\'{a}n Avendaño Ram\'{i}rez~\thanks{Lic. Mat. U.D., M.Sc. U.N.}}
\date{}
\maketitle
  \vspace*{-.5in}
\begin{center}
\fbox{\fbox{\parbox{4.5in}{\centering
Responda cada pregunta en el espacio asignado para ello o en una hoja anexa si así lo necesita}}}
\end{center}
\vspace{0.1in}
\makebox[\textwidth]{Nombres: \hrulefill, curso:\underline{\hspace{48pt}}, fecha:\underline{\hspace{3cm}}}
\begin{questions}
  \question
  Cuando el autor menciona: \emph{Se puede vivir sin saber astrofísica, ni física, ni fútbol, incluso sin saber leer ni escribir: se vive peor, si quieres, pero se vive.}, ¿a qué hace referencia?. ¿Sin saber qué, si no se puede vivir bien?\vspace{1.25in}
  \question
  Siempre se nos ha dicho que mentir es \emph{malo}, sin embargo se podría decir que en ciertas ocasiones es bueno. Mencione algunas
  \newpage
  \question ¿Por qué el autor habla de la \emph{libertad} en el capítulo 1?.\vspace{1.25in}
  \question Explique a qué se refiere el autor cuando dice: \emph{A ese saber vivir, o ``arte de vivir'' si prefieres, es a lo que llaman ética:}.\vspace{1.25in}
  \question
  Haga una síntesis de:
  \begin{parts}
    \part Capítulo 2: Órdenes, costrumbres y caprichos \vspace{2in}
    \part Capítulo 3: Haz lo que quieras\vspace{1in}
  \end{parts}
%\begin{oneparchoices}
%\choice[1] Nunca
%\end{oneparchoices}
%\answerline
\end{questions}
%cuadro de puntajes
%\begin{center}
%\gradetable[h][pages]
%\end{center}
\end{document}