\documentclass[letterpaper,spanish]{letter}
\usepackage[T1]{fontenc}
\usepackage[utf8]{inputenc}
\usepackage[spanish]{babel}
\usepackage{lmodern}
\usepackage{marvosym}
\address{Profesores y profesoras\\Colegio Arborizadora Baja I.E.D. j.m.}
\signature{Profesores y profesoras firmantes en hoja anexa}
\date{6 de noviembre de 2015}
\begin{document}

\begin{letter}{PATRICIA ESPINOSA\\Almacenista\\Colegio Arborizadora Baja I.E.D.}
	
\opening{Cordial saludo}
Los maestros y maestras firmantes de la jornada mañana, no podemos firmar recibido de inventario ya que no ha habido unidad de criterios en la asignación de salones y material en las dos jornadas.

Es importante aclarar que la jornada mañana funciona en la asignación de salones de acuerdo a lo aprobado en Consejo Directivo, que consiste en aulas fijas para la mayoría de maestros y maestras de cada área y son los estudiantes quienes rotan, mientras que la jornada tarde maneja otra dinámica diferente, siendo los maestros quienes rotan y los estudiantes de cada curso permanecen en un salón fijo durante la jornada, no habiendo responsable de los elementos de dotación de las aulas.

\emph{Nota:} Este año fue ratificada la decisión de Consejo Directivo de continuar con las aulas fijas para los maestros de cada área.
\closing{Atentamente,}

cc{:\\CONSEJO DIRECTIVO}
%\ps{PS: PostScriptum}
%\encl{Listado de adjuntos}

\end{letter}
\end{document}
