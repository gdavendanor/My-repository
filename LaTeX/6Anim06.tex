\documentclass[10pt,twoside]{article}
\usepackage[utf8]{inputenc}
\usepackage{amsmath}
\usepackage{amsfonts}
\usepackage{amssymb}
\usepackage[spanish,es-noshorthands]{babel}
\usepackage[T1]{fontenc}
\usepackage{lmodern}
\usepackage{graphicx,hyperref}
\usepackage{tikz,pgf}
\usepackage{marvosym}
\usepackage{multicol}
\usepackage{subfig}
\usepackage[papersize={5.5in,8.5in},width=4.5in,height=7in]{geometry}
\usepackage{fancyhdr}
\pagestyle{fancy}
\fancyhead[LE]{\url{www.autistici.org/mathgerman}}
\fancyhead[RE]{}
\fancyhead[RO]{\Email~ matematicas.german@gmail.com}
\fancyhead[LO]{}

\author{Germ\'an Avenda\~no Ram\'irez~\thanks{Lic. Mat. U.D., M.Sc. U.N.}}
\title{\begin{minipage}{.2\textwidth}
\includegraphics[height=1.75cm]{Images/logo-colegio.png}\end{minipage}
\begin{minipage}{.55\textwidth}
\begin{center}
Animaplano  \\
Aritmética $6^{\circ}$
\end{center}
\end{minipage}\hfill
\begin{minipage}{.2\textwidth}
\includegraphics[height=1.75cm]{Images/logo-sed.png} 
\end{minipage}}
\date{}
\begin{document}
\maketitle
Nombre: \hrulefill Curso: \underline{\hspace*{44pt}} Fecha: \underline{\hspace*{2.5cm}}
\section*{Cuestionario}
Responda cada una de los siguientes ejercicios con el procedimiento.
\begin{enumerate}
\item $4^{2}\div 4=$
\item $2\times 7=$
\item $(7\times 5)-(5\times 2)=$
\item $5\times 8-5=$
\item $3\times 5-2=$
\item $(5\times 4)-(2^{3)=}$
\item $2^{2}\times \sqrt{64}=$
\item $\log_{(2)}32\times 8-\sqrt{49}=$
\item $4\times 7-\sqrt[3]{125}=$
\item $\log_{(2)}8 \times \sqrt[3]{343}=$
\item $\sqrt{36}\times 7- \frac{8}{8}=$
\item $(4\times 8)+(\sqrt[3]{64}\times 5)=$
\item $\sqrt{81}\times \sqrt[3]{216}=$
\item $\sqrt[3]{729}\times \sqrt{49}-1=$
\item $\log_{(2)}16 \times \sqrt{64}+50=$
\item $9\times 7+ \frac{60}{2}=$
\item $\sqrt{25}\times \sqrt{81}+\frac{120}{4}=$
\item $4\times 20+\frac{30}{6}=$
\item $8\times \frac{28}{4}+\frac{90}{3}=$
\item $6^{2}+2^{3}\times 5=$
\item $2^{3}\times \sqrt{36}+(7^{2}+1)=$
\item $7^{2}+40$
\item $9\times 7+6=$
\item $\sqrt{81}\times 6+\log_{(4)}64=$
\item $6\times 9+5=$
\item $7^{2}+1=$
\item $5\times 7-5=$
\item $\log_{(3)}81\times 7=$
\item $6^{2}+2=$
\item $5\times 8 -1=$
\item $5^{2}-6=$
\item $\log_{(4)}64\times 6=$
\item $4\times \sqrt[3]{729}=$
\item $4\times 7-\log_{(5)}25=$
\item $4^{2}+1=$
\item $\log_{(5)}125\times 9-20=$
\end{enumerate}
Luego de resuelto el cuestionario, ubique las respuestas en el plano con puntos numerados de 1--100, para formar una figura interesante.
%\begin{center}
%\begin{tikzpicture}[scale=.9]
 %\fill (1,0) node[above]{1} circle (0.2ex);
 %%\fill (2,0) node[above]{2} circle (0.2ex);
 %\fill (3,0) node[above]{3} circle (0.2ex);
 %\fill (4,0) node[above]{4} circle (0.2ex);
 %\fill (5,0) node[above]{5} circle (0.2ex);
 %\fill (6,0) node[above]{6} circle (0.2ex);
 %\fill (7,0) node[above]{7} circle (0.2ex);
 %\fill (8,0) node[above]{8} circle (0.2ex);
 %\fill (9,0) node[above]{9} circle (0.2ex);
 %\fill (10,0) node[above]{10} circle (0.2ex);
 %\fill (1,-1) node[left]{11} circle (0.2ex);
 %\fill (2,-1) circle (0.2ex);
 %\fill (3,-1) circle (0.2ex);
 %\fill (4,-1) circle (0.2ex);
 %\fill (5,-1) circle (0.2ex);
 %\fill (6,-1) circle (0.2ex);
 %\fill (7,-1) circle (0.2ex);
 %\fill (8,-1) circle (0.2ex);
 %\fill (9,-1) circle (0.2ex);
 %\fill (10,-1) circle (0.2ex);
 %\fill (1,-2) node[left]{21} circle (0.2ex);
 %\fill (2,-2) circle (0.2ex);
 %\fill (3,-2) circle (0.2ex);
 %\fill (4,-2) circle (0.2ex);
 %\fill (5,-2) circle (0.2ex);
 %\fill (6,-2) circle (0.2ex);
 %\fill (7,-2) circle (0.2ex);
 %\fill (8,-2) circle (0.2ex);
 %\fill (9,-2) circle (0.2ex);
 %\fill (10,-2) circle (0.2ex);
 %\fill (1,-3) node[left]{31} circle (0.2ex);
 %\fill (2,-3) circle (0.2ex);
 %\fill (3,-3) circle (0.2ex);
 %\fill (4,-3) circle (0.2ex);
 %\fill (5,-3) circle (0.2ex);
 %\fill (6,-3) circle (0.2ex);
 %\fill (7,-3) circle (0.2ex);
 %\fill (8,-3) circle (0.2ex);
 %\fill (9,-3) circle (0.2ex);
 %\fill (10,-3) circle (0.2ex);
 %\fill (1,-4) node[left]{41} circle (0.2ex);
 %\fill (2,-4) circle (0.2ex);
 %\fill (3,-4) circle (0.2ex);
 %\fill (4,-4) circle (0.2ex);
 %\fill (5,-4) circle (0.2ex);
 %\fill (6,-4) circle (0.2ex);
 %\fill (7,-4) circle (0.2ex);
 %\fill (8,-4) circle (0.2ex);
 %\fill (9,-4) circle (0.2ex);
 %\fill (10,-4) node[right]{50} circle (0.2ex);
 %\fill (1,-5) node[left]{51} circle (0.2ex);
 %\fill (2,-5) circle (0.2ex);
 %\fill (3,-5) circle (0.2ex);
 %\fill (4,-5) circle (0.2ex);
 %\fill (5,-5) circle (0.2ex);
 %\fill (6,-5) circle (0.2ex);
 %\fill (7,-5) circle (0.2ex);
 %\fill (8,-5) circle (0.2ex);
 %\fill (9,-5) circle (0.2ex);
 %\fill (10,-5) circle (0.2ex);
 %\fill (1,-6) node[left]{61} circle (0.2ex);
 %\fill (2,-6) circle (0.2ex);
 %\fill (3,-6) circle (0.2ex);
 %\fill (4,-6) circle (0.2ex);
 %\fill (5,-6) circle (0.2ex);
 %\fill (6,-6) circle (0.2ex);
 %\fill (7,-6) circle (0.2ex);
 %\fill (8,-6) circle (0.2ex);
 %\fill (9,-6) circle (0.2ex);
 %\fill (10,-6) circle (0.2ex);
 %\fill (1,-7) node[left]{71} circle (0.2ex);
 %\fill (2,-7) circle (0.2ex);
 %\fill (3,-7) circle (0.2ex);
 %\fill (4,-7) circle (0.2ex);
 %\fill (5,-7) circle (0.2ex);
 %\fill (6,-7) circle (0.2ex);
 %\fill (7,-7) circle (0.2ex);
 %\fill (8,-7) circle (0.2ex);
 %\fill (9,-7) circle (0.2ex);
 %\fill (10,-7) circle (0.2ex);
 %\fill (1,-8) node[left]{81} circle (0.2ex);
 %\fill (2,-8) circle (0.2ex);
 %\fill (3,-8) circle (0.2ex);
 %\fill (4,-8) circle (0.2ex);
 %\fill (5,-8) circle (0.2ex);
 %\fill (6,-8) circle (0.2ex);
 %\fill (7,-8) circle (0.2ex);
 %\fill (8,-8) circle (0.2ex);
 %\fill (9,-8) circle (0.2ex);
 %\fill (10,-8) circle (0.2ex);
 %\fill (1,-9) node[left]{91} circle (0.2ex);
 %\fill (2,-9) circle (0.2ex);
 %\fill (3,-9) circle (0.2ex);
 %\fill (4,-9) circle (0.2ex);
 %\fill (5,-9) circle (0.2ex);
 %\fill (6,-9) circle (0.2ex);
 %\fill (7,-9) circle (0.2ex);
 %\fill (8,-9) circle (0.2ex);
 %\fill (9,-9) circle (0.2ex);
 %\fill (10,-9) node[right]{100} circle (0.2ex);
% \draw (4,0)--(4,-1)--(5,-2)--(5,-3)--(3,-1)--(2,-1)--(2,-3)--(3,-3)--(3,-2)--(1,-2)--(1,-4)--(2,-5)--(4,-5)--(2,-6)--(2,-8)--(3,-9)--(5,-7)--(5,-8)--(6,-8)--(6,-7)--(8,-9)--(9,-8)--(9,-6)--(7,-5)--(9,-5)--(10,-4)--(10,-2)--(8,-2)--(8,-3)--(9,-3)--(9,-1)--(8,-1)--(6,-3)--(6,-2)--(7,-1)--(7,-0);
%\end{tikzpicture}
%\end{center}
\end{document}
