\documentclass[10pt,twoside]{article}
\usepackage[utf8]{inputenc}
\usepackage{amsmath}
\usepackage{amsfonts}
\usepackage{amssymb}
\usepackage[spanish,es-noshorthands]{babel}
\usepackage[T1]{fontenc}
\usepackage{lmodern}
\usepackage{graphicx,hyperref}
\usepackage{tikz,pgf}
\usepackage{multicol}
\usepackage{subfig}
\usepackage[papersize={6.5in,8.5in},width=5.5in,height=7in]{geometry}
\usepackage{fancyhdr}
\pagestyle{fancy}
\fancyhead[LE]{\includegraphics[height=12pt]{Images/logo-colegio.png} Aritmética $6^{\circ}$}
\fancyhead[RE]{}
\fancyhead[RO]{\textit{Germ\'an Avenda\~no Ram\'irez, Lic. U.D., M.Sc. U.N.}}
\fancyhead[LO]{}

\author{Germ\'an Avenda\~no Ram\'irez, Lic. U.D., M.Sc. U.N.}
\title{\begin{minipage}{.2\textwidth}
\includegraphics[height=1.75cm]{Images/logo-colegio.png}\end{minipage}
\begin{minipage}{.55\textwidth}
\begin{center}
Taller 13, Fracciones   \\
Aritmética $6^{\circ}$
\end{center}
\end{minipage}\hfill
\begin{minipage}{.2\textwidth}
\includegraphics[height=1.75cm]{Images/logo-sed.png} 
\end{minipage}}
\date{}
\begin{document}
\maketitle
Nombre: \hrulefill Curso: \underline{\hspace*{44pt}} Fecha: \underline{\hspace*{2.5cm}}
\section*{Nivel I}
\begin{enumerate}
 \item Enumera los términos de una fracción y di qué indica cada uno de ellos. Pon varios ejemplos.
\item ¿Qué fracción de hora son 20 minutos? Y ¿35 minutos? Y ¿55 minutos?
\item Para elaborar un tarro de frutas se han necesitado 400 gramos de plátanos, 350 gramos de fresas, 250 gramos de azúcar y 50 gramos de manzanas. ¿Qué fracción del total representa cada uno de estos productos?
\item Calcula
\begin{enumerate}\begin{multicols}{4}
 \item $\dfrac{5}{10}$ de 90
 \item $\dfrac{7}{9}$ de 72
 \item $\dfrac{4}{3}$ de 42
 \item $\dfrac{5}{9}$ de 540
\end{multicols}
 \end{enumerate}
\item  En una clase de 24 alumnos 5/8 son chicos. ¿Cuántos chicos y chicas hay en clase?
\item El depósito de un coche tiene una capacidad de 63 litros de gasolina, si gasta los 5/9 en una excursión, ¿cuántos litros le quedan al volver de viaje?
\item En la puerta de un cine hay 12 mujeres por cada 8 hombres y 16 niños. ¿Cuál es la relación entre hombres y mujeres?
¿Entre hombres y niños? Y ¿Entre mujeres y niños?

\end{enumerate}

\end{document}
