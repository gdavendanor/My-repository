\documentclass[10pt,twoside]{article}
\usepackage[utf8]{inputenc}
\usepackage{amsmath}
\usepackage{amsfonts}
\usepackage{amssymb}
\usepackage[spanish,es-noshorthands]{babel}
\usepackage[T1]{fontenc}
\usepackage{lmodern}
\usepackage{graphicx,hyperref}
\usepackage{tikz,pgf}
\usepackage{multicol}
\usepackage{subfig}
\usepackage[papersize={6.5in,8.5in},width=5.5in,height=7in]{geometry}
\usepackage{fancyhdr}
\pagestyle{fancy}
\fancyhead[LE]{\includegraphics[height=12pt]{Images/logo-colegio.png} Aritmética $6^{\circ}$}
\fancyhead[RE]{}
\fancyhead[RO]{\textit{Germ\'an Avenda\~no Ram\'irez, Lic. U.D., M.Sc. U.N.}}
\fancyhead[LO]{}

\author{Germ\'an Avenda\~no Ram\'irez, Lic. U.D., M.Sc. U.N.}
\title{\begin{minipage}{.2\textwidth}
\includegraphics[height=1.75cm]{Images/logo-colegio.png}\end{minipage}
\begin{minipage}{.55\textwidth}
\begin{center}
Taller 13, Fracciones   \\
Aritmética $6^{\circ}$
\end{center}
\end{minipage}\hfill
\begin{minipage}{.2\textwidth}
\includegraphics[height=1.75cm]{Images/logo-sed.png} 
\end{minipage}}
\date{}
\begin{document}
\maketitle
Nombre: \hrulefill Curso: \underline{\hspace*{44pt}} Fecha: \underline{\hspace*{2.5cm}}
\section*{Problema resuelto}
La señora Marta preparó un pastel de choclo para el almuerzo. Si lo repartió en partes iguales entre ella, su esposo y sus tres hijos, ¿qué fracción del pastel comieron en total sus hijos?
\subsection*{Solución}
La fracción de pastel que comieron los hijos corresponde al número de porciones que
comieron sus hijos, del número total de porciones.

Esto puede resumirse en el siguiente esquema:
\subsubsection*{Procedimiento:}
El número de porciones que comieron los hijos es 3 y el número total de porciones es 5, por lo tanto la fracción buscada es la correspondiente a 3 porciones de un total de 5.
\subsubsection*{Operación y resultado:}
3 de 5 es igual a $\dfrac{3}{5}$
\subsubsection*{Respuesta}
Entre los hijos comieron $\dfrac{3}{5}$ del pastel

Realiza las siguientes operaciones
\begin{enumerate}
 \item  ¿Qué fracción representa 4 de un total de 5?
 \item ¿Qué fracción representa 1 de un total de 7? 
 \item ¿Qué fracción representa 8 de un total de 17?
 \item ¿Qué fracción representa 6 de un total de 9?
 \item ¿Qué fracción representa 9 de un total de 10?¿Qué fracción representa 12 de un total de 12?

\end{enumerate}

Resuelve los siguientes problemas indicando en cada caso:
\begin{enumerate}
 \item[a)] El procedimiento
 \item[b)] La operación con su resultado
 \item[c)] La respuesta del problema.
\end{enumerate}
\paragraph*{Problema 1:}
Andrea compró una docena de huevos en un almacén. Al llegar a su casa se cayó y sólo quedaron 5 huevos enteros. ¿Qué fracción de los huevos no se quebró?
\paragraph*{Problema 2:}
Un ciclista da diariamente 30 vueltas a una pista. Ayer, mientras hacía su rutina, comenzó una gran lluvia y sólo alcanzó a pedalear 13 vueltas. ¿Qué fracción de lo que normalmente recorre alcanzó a hacer?
\paragraph*{Problema 3:}
Una micro realiza el mismo recorrido 7 veces al día. Debido a la congestión vehicular hoy sólo recorrió 5 veces su ruta. ¿Qué fracción de su recorrido habitual logró hacer?
\paragraph*{Problema 4:}
En una competencia Juan ganó 15 bolitas. Si regaló 3 de ellas a su hermano menor, ¿qué fracción de las bolitas que había regalado ganó?
\paragraph*{Problema 5:}
En un almacén tenían 100 agendas para vender. Si vendieron sólo 78 agendas, ¿qué fracción del total vendieron?

\end{document}
