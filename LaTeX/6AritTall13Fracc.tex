\documentclass[10pt,twoside]{article}
\usepackage[utf8]{inputenc}
\usepackage{amsmath}
\usepackage{amsfonts}
\usepackage{amssymb}
\usepackage[spanish,es-noshorthands]{babel}
\usepackage[T1]{fontenc}
\usepackage{lmodern}
\usepackage{graphicx,hyperref}
\usepackage{tikz,pgf}
\usepackage{multicol}
\usepackage{subfig}
\usepackage[papersize={6.5in,8.5in},width=5.5in,height=7in]{geometry}
\usepackage{fancyhdr}
\pagestyle{fancy}
\fancyhead[LE]{\includegraphics[height=12pt]{Images/logo-colegio.png} Aritmética $6^{\circ}$}
\fancyhead[RE]{}
\fancyhead[RO]{\textit{Germ\'an Avenda\~no Ram\'irez, Lic. U.D., M.Sc. U.N.}}
\fancyhead[LO]{}

\author{Germ\'an Avenda\~no Ram\'irez, Lic. U.D., M.Sc. U.N.}
\title{\begin{minipage}{.2\textwidth}
\includegraphics[height=1.75cm]{Images/logo-colegio.png}\end{minipage}
\begin{minipage}{.55\textwidth}
\begin{center}
Taller 14, Continuando con las fracciones \\
Aritmética $6^{\circ}$
\end{center}
\end{minipage}\hfill
\begin{minipage}{.2\textwidth}
\includegraphics[height=1.75cm]{Images/logo-sed.png} 
\end{minipage}}
\date{}
\begin{document}
\maketitle
Nombre: \hrulefill Curso: \underline{\hspace*{44pt}} Fecha: \underline{\hspace*{2.5cm}}
\section*{Fracciones}
Lea detenidamente como se resuelve el siguiente problema para que pueda desarrollar el taller.
\subsection*{Problema resuelto}
La señora Martha prepar\'{o} un pastel de choclo para el almuerzo. Si lo reparti\'{o} en partes iguales entre ella, su esposo y sus tres hijos, ¿qu\'{e} fracci\'{o}n del pastel comieron en total sus hijos?
\subsubsection*{Soluci\'{o}n}
La fracci\'{o}n del pastel que comieron los hijos corresponde al n\'{u}mero de porciones que comieron sus hijos, del n\'{u}mero total de porciones.\\
Esto puede resumirse en el siguiente esquema:
\paragraph*{Procedimiento:}
El n\'{u}mero de porciones que comieron los hijos es 3 y el n\'{u}mero total de porciones es 5, por lo tanto la fracci\'{o}n buscada es la correspondiente a 3 porciones de un total de 5.
\paragraph*{Operaci\'{o}n y resultado}
3 de 5 es igual a \hspace*{.5cm} $\dfrac{3}{5}$
\paragraph*{Respuesta:}
Entre los hijos se comieron $\dfrac{3}{5}$ del pastel
Responde las siguientes preguntas
\begin{enumerate}
\item ¿Qu\'{e} fracci\'{o}n representa 4 de un total de 5?
\item ¿Qu\'{e} fracci\'{o}n representa 1 de un total de 7?
\item ¿Qu\'{e} fracci\'{o}n representa 8 de un total de 17?
\item ¿Qu\'{e} fracci\'{o}n representa 6 de un total de 9?
\item ¿Qu\'{e} fracci\'{o}n representa 12 de un total de 12?
\end{enumerate}
Resuelva los siguientes problemas indicando en cada caso:
\begin{enumerate}
\begin{multicols}{2}
\item[a)] El procedimiento
\item[b)] La operación y sus resultado\end{multicols}
\item[c)] La respuesta del problema
\end{enumerate}
\paragraph*{Problema 1:} Andrea compró una docena de huevos en un almacén. Al llegar a su casa se cayó y sólo quedaron 5 huevos enteros. ¿Qué fracción de los huevos no se quebró?
\paragraph*{Problema 2:} Un ciclista da diariamente 30 vueltas a una pista. Ayer, mientras hacía su rutina, comenzó una gran lluvia y sólo alcanzó a pedalear 13 vueltas. ¿Qué fracción de lo que normalmente recorre alcanzó a hacer?
\paragraph*{Problema 3:} 
Una micro realiza el mismo recorrido 7 veces al día. Debido a la congestión vehicular hoy sólo recorrió 5 veces su ruta. ¿Qué fracción de su recorrido habitual logró hacer?
\paragraph*{Problema 4:} Problema 4:
En una competencia Juan ganó 15 bolitas. Si regaló 3 de ellas a su hermano menor, ¿qué fracción de las bolitas que había regalado ganó? 
\paragraph*{Problema 5:} En un almacén tenían 100 agendas para vender. Si vendieron sólo 78 agendas, ¿qué fracción
del total vendieron?
\section*{Equivalencia de fracciones}
\subsection*{Problema resuelto}
La señora Marta horneó 2 pasteles iguales, uno lo partió en 6 y el otro en 15 partes. Su hijo Juan comió 2 trozo de los grandes y su hija Juana comió 5 de los chicos. La señora Marta afirma que ambos comieron lo mismo, ¿es eso verdad?
\subsubsection*{Soluci\'{o}n}
Juan y Juana comieron lo mismo, si la fracción de pastel que comió Juan es equivalente a la fracción de pastel que comió Juana.\\
Esto puede resumirse en el siguiente esquema:
\paragraph*{Procedimiento:} Debemos considerar la fracción correspondiente a dos porciones de un total de 6; la porción correspondiente a 5 porciones de un total de 15, y luego comparar estas cantidades.
\paragraph*{Operación y resultado}
La fracción correspondiente a 2 entre 6 es $\frac{2}{6}$ y la correspondiente a 5 de un total de 15 es $\frac{5}{15}$. Para compararlas, observamos que si subdividimos cada trozo
del pastel que comió Juan en 15 partes iguales se obtendría en total $6\cdot 15$ pedazos, y los dos trozos que él comió
equivaldrían a $2\cdot 15$ de estos pedacitos. De la misma
manera, si dividimos cada trozo del pastel que comió Juana en 6 partes iguales se obtendría $15\cdot 6$ trocitos y los 5 trozos que ella comió equivaldrían a 5·6 de estos
trocitos. Como ambos pasteles quedarían partidos en el
mismo número de pedazos, ambos comerán lo mismo si $2\cdot 15 = 5\cdot 6$, entonces \[2\cdot 15=30= 5\cdot6\]
\paragraph*{Respuesta}
Ambos comieron igual cantidad\\
Indica sin son equivalentes las siguientes pares de fracciones entre sí. Recuerde que para compararlas, deben tener el mismo denominador.
\begin{itemize}
\begin{multicols}{3}
\item $\dfrac{1}{3}$ y $\dfrac{3}{9}$
\item $\dfrac{2}{7}$ y $\dfrac{6}{21}$
\item $\dfrac{2}{3}$ y $\dfrac{3}{6}$
\item $\dfrac{1}{5}$ y $\dfrac{5}{25}$
\item $\dfrac{4}{18}$ y $\dfrac{2}{9}$
\item $\dfrac{1}{3}$ y $\dfrac{3}{6}$
\end{multicols}
\end{itemize}
Resuelve los siguientes problemas, indicando en cada caso:
\begin{enumerate}
\item[a.] El procedimiento
\item[b.] La operación con su resultado
\item[c.] Las respuesta del problema
\end{enumerate}
\paragraph*{Problema 6:}
Francisca tomó una bebida de medio litro y María tomó dos bebidas de un cuarto de litro cada una. ¿Tomaron ambas la misma cantidad de líquido?
\paragraph*{Problema 7:}
Dos ciclistas deben recorrer un circuito. Si el primero ha recorrido dos tercios de éste y el segundo cuatro sextos del mismo, ¿han recorrido hasta ahora la misma distancia?
\paragraph*{Problema 8:} 
En la especialidad de alimentación se preparan tortas para una recepción, Susana preparó 2 tortas de igual tamaño, una de piña y otra de manjar. La de piña la dividió en 24 trozos
iguales y la otra en 12 trozos iguales. y don Juan comió 3 pedazos de torta de piña y dos de manjar, ¿comió lo mismo de ambas?
\paragraph*{Problema 9:}
Marcos y Luis deben llevar papas fritas para una convivencia. Marcos lleva $\frac{3}{4}$ de kilo y Luis lleva $\frac{4}{5}$, ¿llevan ambos la misma cantidad?
\paragraph*{Problema 10:}
Una porción de alimento alcanza para alimentar a 2 tigres y una porción igual es suficiente para 6 zorros. ¿Comen lo mismo un tigre que dos zorros?
\section*{Amplificar}
\subsection*{Problema resuelto}
Dominga preparó un queque y lo dividió en 5 trozos iguales Si cada uno de estos trozos lo divide a su vez en tres trocitos iguales, ¿a qué fracción del queque corresponde la cantidad de trocitos obtenida de 2 trozos?
\subsubsection*{Soluci\'{o}n}
La fracción de queque que corresponde a 2 trozos, equivale a la cantidad de trocitos obtenida de estos 2 pedazos, de la cantidad total de trocitos
\paragraph*{Respuesta:} 
La fracción que representa dos pedazos de 5, al dividir cada pedazo en 3 es $\dfrac{6}{15}$\\
Esto puede resumirse en el siguiente esquema:
\subparagraph*{Procedimiento}
Para determinar el número total de trocitos debemos multiplicar 3 por 5.\\
Para determinar el número de trocitos que corresponde a 2 trozos debemos multiplicar 2 por 3.\\
Luego debemos formar la fracción que corresponde a 2 de 5
\subparagraph*{Operaciones}
\[\dfrac{2}{5}=\dfrac{2\cdot 3}{5\cdot 3}=\dfrac{6}{15}\]
\subparagraph*{Respuesta:}
La fracción que representa dos pedazos de 5, al dividir cada pedazo en 5 es $\dfrac{6}{15}$
\end{document}
