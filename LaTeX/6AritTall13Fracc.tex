\documentclass[10pt,twoside]{article}
\usepackage[utf8]{inputenc}
\usepackage{amsmath}
\usepackage{amsfonts}
\usepackage{amssymb}
\usepackage[spanish,es-noshorthands]{babel}
\usepackage[T1]{fontenc}
\usepackage{lmodern}
\usepackage{graphicx,hyperref}
\usepackage{tikz,pgf}
\usepackage{multicol}
\usepackage{subfig}
\usepackage[papersize={6.5in,8.5in},width=5.5in,height=7in]{geometry}
\usepackage{fancyhdr}
\pagestyle{fancy}
\fancyhead[LE]{\includegraphics[height=12pt]{Images/logo-colegio.png} Aritmética $6^{\circ}$}
\fancyhead[RE]{}
\fancyhead[RO]{\textit{Germ\'an Avenda\~no Ram\'irez, Lic. U.D., M.Sc. U.N.}}
\fancyhead[LO]{}

\author{Germ\'an Avenda\~no Ram\'irez, Lic. U.D., M.Sc. U.N.}
\title{\begin{minipage}{.2\textwidth}
\includegraphics[height=1.75cm]{Images/logo-colegio.png}\end{minipage}
\begin{minipage}{.55\textwidth}
\begin{center}
Taller 14, Continuando con las fracciones \\
Aritmética $6^{\circ}$
\end{center}
\end{minipage}\hfill
\begin{minipage}{.2\textwidth}
\includegraphics[height=1.75cm]{Images/logo-sed.png} 
\end{minipage}}
\date{}
\begin{document}
\maketitle
Nombre: \hrulefill Curso: \underline{\hspace*{44pt}} Fecha: \underline{\hspace*{2.5cm}}
\section*{Fracciones}
Lea detenidamente como se resuelve el siguiente problema para que pueda desarrollar el taller.
\subsection*{Problema resuelto}
La señora Martha prepar\'{o} un pastel de choclo para el almuerzo. Si lo reparti\'{o} en partes iguales entre ella, su esposo y sus tres hijos, ¿qu\'{e} fracci\'{o}n del pastel comieron en total sus hijos?
\subsection*{Soluci\'{o}n}
La fracci\'{o}n del pastel que comieron los hijos corresponde al n\'{u}mero de porciones que comieron sus hijos, del n\'{u}mero total de porciones.\\
Esto puede resumirse en el siguiente esquema:
\subsubsection*{Procedimiento:}
El n\'{u}mero de porciones que comieron los hijos es 3 y el n\'{u}mero total de porciones es 5, por lo tanto la fracci\'{o}n buscada es la correspondiente a 3 porciones de un total de 5.
\subsubsection*{Operaci\'{o}n y resultado}
3 de 5 es igual a \hspace*{.5cm} $\dfrac{3}{5}$
\subsubsection*{Respuesta:}
Entre los hijos se comieron $\dfrac{3}{5}$ del pastel
Responde las siguientes preguntas
\begin{enumerate}
\item ¿Qu\'{e} fracci\'{o}n representa 4 de un total de 5?
\item ¿Qu\'{e} fracci\'{o}n representa 1 de un total de 7?
\item ¿Qu\'{e} fracci\'{o}n representa 8 de un total de 17?
\item ¿Qu\'{e} fracci\'{o}n representa 6 de un total de 9?
\item ¿Qu\'{e} fracci\'{o}n representa 12 de un total de 12?
\end{enumerate}
Resuelva los siguientes problemas indicando en cada caso:
\begin{enumerate}
\item[a)] El procedimiento
\item[b)] La operación y sus resultado
\item[c)] La respuesta del problema
\end{enumerate}
\paragraph*{Problema 1:} Andrea compró una docena de huevos en un almacén. Al llegar a su casa se cayó y sólo quedaron 5 huevos enteros. ¿Qué fracción de los huevos no se quebró?
\paragraph*{Problema 2:} Un ciclista da diariamente 30 vueltas a una pista. Ayer, mientras hacía su rutina, comenzó una gran lluvia y sólo alcanzó a pedalear 13 vueltas. ¿Qué fracción de lo que normalmente recorre alcanzó a hacer?
\paragraph*{Problema 3:} 
Una micro realiza el mismo recorrido 7 veces al día. Debido a la congestión vehicular hoy sólo recorrió 5 veces su ruta. ¿Qué fracción de su recorrido habitual logró hacer?
\paragraph*{Problema 4:} Problema 4:
En una competencia Juan ganó 15 bolitas. Si regaló 3 de ellas a su hermano menor, ¿qué fracción de las bolitas que había regalado ganó? 
\paragraph*{Problema 5:} En un almacén tenían 100 agendas para vender. Si vendieron sólo 78 agendas, ¿qué fracción
del total vendieron?
\section*{Equivalencia de fracciones}

\end{document}
