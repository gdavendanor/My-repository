\documentclass[twoside,letterpaper]{article}
\usepackage[utf8]{inputenc}
\usepackage{amsmath,amsfonts,amssymb,amsthm,latexsym}
\usepackage[spanish,es-noshorthands]{babel}
\usepackage[T1]{fontenc}
\usepackage{lmodern}
\usepackage{graphicx,hyperref}
\usepackage{tikz,pgf}
\usepackage{marvosym}
\usepackage{multicol}
\usepackage{fancyhdr}
\usepackage[left=.75cm,right=.75cm,top=1.5cm,bottom=1.25cm]{geometry}
\usepackage{fancyhdr}
\pagestyle{fancy}
\fancyhead[LE]{Colegio Arborizadora Baja}
\fancyhead[RE]{PEI:``Hacia una cultura para el desarrollo sostenible''}
\fancyfoot[RO]{\Email iedabgerman@autistici.org}
\fancyhead[LO]{\url{www.autistici.org/mathgerman}}
\fancyfoot[RE]{\Email cedarborizadoraba19@redp.edu.co}
\fancyfoot[LE]{Calle 59I \#44A - 02 \Telefon 7313994 - 7313995}
\fancyhead[RO]{Nit 830024976-8, Código DANE 11100103084-8}

\author{Germ\'an Avenda\~no Ram\'irez~\thanks{Lic. Mat. U.D., M.Sc. U.N.}}
\title{\begin{minipage}{.2\textwidth}
\includegraphics[height=1.75cm]{Images/logo-colegio.png}\end{minipage}
\begin{minipage}{.55\textwidth}
\begin{center}
Animaplano 10\\
Matemáticas $7^{\circ}$
\end{center}
\end{minipage}\hfill
\begin{minipage}{.2\textwidth}
\includegraphics[height=1.75cm]{Images/logo-sed.png} 
\end{minipage}}
\date{}
\thispagestyle{plain}
\begin{document}
\maketitle
Nombre: \hrulefill Curso: \underline{\hspace*{44pt}} Fecha: \underline{\hspace*{2.5cm}}\\

Resuelva el animplano, haciendo el procedimiento correspondiente.
\begin{enumerate}
\begin{multicols}{2}
\item Resuelva $2^{2}\cdot 2^{2}\cdot 2^{2}=$
\item Halle $\sqrt{10.000}-3^{2}=$
\end{multicols}
\item Sea $U=\{4,7,9,6\}$, si $A=\{6,4\}$ El complemento\footnote{El complemento de un conjunto A, $A^{c}$, es el conjunto de los elementos de $A$ que hacen falta para ser el conjunto universal $U$} de A, $A^{c}=\{\hspace{1cm}\}$. Determine el producto de los elementos de $A^{c}$
\begin{multicols}{2}
\item Si $x-27=27$, entonces $x=$
\item 1 decena + 4 docenas
\item 1/2 centena $-$ 1 docena
\item 2 décadas + 4 lustros
\item En segundos, 1/2 minuto
\item Halle $5^{2}+2^{2}=$
\item Resuelva $(-7)\cdot (-7)=$
\item Resuelva $(4!\cdot 2)-1!=$
\item $(34\div 2)$ mas el 50\% de 100
\item 28 es la mitad del número
\item Reste 11 al doble de 20
\item Halle $5!-(160\div 2)=$
\item $(9\cdot 3)+(200\div 5)=$
\item Resuelva $2^{3}\cdot 2^{3}=$
\item $2\cdot 2\cdot 2\cdot 2\cdot 2+2$
\item Halle $7(\frac{4}{2}-\frac{8}{4})+7$
\item Resuelva $(-12\div 4)\cdot(-6)$
\item El perímetro de un octágono regular de lado 2 cm
\item Área de un cuadrado si su perímetro mide 24 m
\item 1 decena + 2 docenas
\item Raíz cuadrada de 25 por la diferencia entre 15 y 6
\item Nueve veces dos
\item Simplifique $7\frac{2}{3}+\frac{1}{3}$
\item Halle $\frac{9}{3}+\frac{9}{3}+\frac{3}{3}=$
\item El triple de nueve
\item Descubra el número desconocido\\
\tikz \draw (0,0) rectangle node{4} (.5,.5);\tikz \draw (0,0) rectangle node {16} (.5,.5); \; \tikz \draw (0,0) rectangle node {5} (.5,.5);\tikz \draw (0,0) rectangle  (.5,.5); \; \tikz \draw (0,0) rectangle node {8} (.5,.5);\tikz \draw (0,0) rectangle node {64} (.5,.5);
\item 1/2 siglo + 3 lustros =
\item Escriba el número faltante \tikz \draw (0,0) rectangle node {4} (.5,.5) (.5,0) rectangle node{16}(1,.5) (.25,.5) rectangle node{8}(.75,1); \;
\tikz \draw (0,0) rectangle node {5} (.5,.5) (.5,0) rectangle node{20}(1,.5) (.25,.5) rectangle node{10} (.75,1); \; \tikz \draw (0,0) rectangle node {37} (.5,.5) (.5,0) rectangle node{148}(1,.5) (.25,.5) rectangle node{}(.75,1);
\item Halle $5!-29=$

\begin{tikzpicture}[scale=.9]
 \fill (1,0) node[above]{1} circle (0.2ex);
 \fill (2,0) node[above]{2} circle (0.2ex);
 \fill (3,0) node[above]{3} circle (0.2ex);
 \fill (4,0) node[above]{4} circle (0.2ex);
 \fill (5,0) node[above]{5} circle (0.2ex);
 \fill (6,0) node[above]{6} circle (0.2ex);
 \fill (7,0) node[above]{7} circle (0.2ex);
 \fill (8,0) node[above]{8} circle (0.2ex);
 \fill (9,0) node[above]{9} circle (0.2ex);
 \fill (10,0) node[above]{10} circle (0.2ex);
 \fill (1,-1) node[left]{11} circle (0.2ex);
 \fill (2,-1) circle (0.2ex);
 \fill (3,-1) circle (0.2ex);
 \fill (4,-1) circle (0.2ex);
 \fill (5,-1) circle (0.2ex);
 \fill (6,-1) circle (0.2ex);
 \fill (7,-1) circle (0.2ex);
 \fill (8,-1) circle (0.2ex);
 \fill (9,-1) circle (0.2ex);
 \fill (10,-1) circle (0.2ex);
 \fill (1,-2) node[left]{21} circle (0.2ex);
 \fill (2,-2) circle (0.2ex);
 \fill (3,-2) circle (0.2ex);
 \fill (4,-2) circle (0.2ex);
 \fill (5,-2) circle (0.2ex);
 \fill (6,-2) circle (0.2ex);
 \fill (7,-2) circle (0.2ex);
 \fill (8,-2) circle (0.2ex);
 \fill (9,-2) circle (0.2ex);
 \fill (10,-2) circle (0.2ex);
 \fill (1,-3) node[left]{31} circle (0.2ex);
 \fill (2,-3) circle (0.2ex);
 \fill (3,-3) circle (0.2ex);
 \fill (4,-3) circle (0.2ex);
 \fill (5,-3) circle (0.2ex);
 \fill (6,-3) circle (0.2ex);
 \fill (7,-3) circle (0.2ex);
 \fill (8,-3) circle (0.2ex);
 \fill (9,-3) circle (0.2ex);
 \fill (10,-3) circle (0.2ex);
 \fill (1,-4) node[left]{41} circle (0.2ex);
 \fill (2,-4) circle (0.2ex);
 \fill (3,-4) circle (0.2ex);
 \fill (4,-4) circle (0.2ex);
 \fill (5,-4) circle (0.2ex);
 \fill (6,-4) circle (0.2ex);
 \fill (7,-4) circle (0.2ex);
 \fill (8,-4) circle (0.2ex);
 \fill (9,-4) circle (0.2ex);
 \fill (10,-4) node[right]{50} circle (0.2ex);
 \fill (1,-5) node[left]{51} circle (0.2ex);
 \fill (2,-5) circle (0.2ex);
 \fill (3,-5) circle (0.2ex);
 \fill (4,-5) circle (0.2ex);
 \fill (5,-5) circle (0.2ex);
 \fill (6,-5) circle (0.2ex);
 \fill (7,-5) circle (0.2ex);
 \fill (8,-5) circle (0.2ex);
 \fill (9,-5) circle (0.2ex);
 \fill (10,-5) circle (0.2ex);
 \fill (1,-6) node[left]{61} circle (0.2ex);
 \fill (2,-6) circle (0.2ex);
 \fill (3,-6) circle (0.2ex);
 \fill (4,-6) circle (0.2ex);
 \fill (5,-6) circle (0.2ex);
 \fill (6,-6) circle (0.2ex);
 \fill (7,-6) circle (0.2ex);
 \fill (8,-6) circle (0.2ex);
 \fill (9,-6) circle (0.2ex);
 \fill (10,-6) circle (0.2ex);
 \fill (1,-7) node[left]{71} circle (0.2ex);
 \fill (2,-7) circle (0.2ex);
 \fill (3,-7) circle (0.2ex);
 \fill (4,-7) circle (0.2ex);
 \fill (5,-7) circle (0.2ex);
 \fill (6,-7) circle (0.2ex);
 \fill (7,-7) circle (0.2ex);
 \fill (8,-7) circle (0.2ex);
 \fill (9,-7) circle (0.2ex);
 \fill (10,-7) circle (0.2ex);
 \fill (1,-8) node[left]{81} circle (0.2ex);
 \fill (2,-8) circle (0.2ex);
 \fill (3,-8) circle (0.2ex);
 \fill (4,-8) circle (0.2ex);
 \fill (5,-8) circle (0.2ex);
 \fill (6,-8) circle (0.2ex);
 \fill (7,-8) circle (0.2ex);
 \fill (8,-8) circle (0.2ex);
 \fill (9,-8) circle (0.2ex);
 \fill (10,-8) circle (0.2ex);
 \fill (1,-9) node[left]{91} circle (0.2ex);
 \fill (2,-9) circle (0.2ex);
 \fill (3,-9) circle (0.2ex);
 \fill (4,-9) circle (0.2ex);
 \fill (5,-9) circle (0.2ex);
 \fill (6,-9) circle (0.2ex);
 \fill (7,-9) circle (0.2ex);
 \fill (8,-9) circle (0.2ex);
 \fill (9,-9) circle (0.2ex);
 \fill (10,-9) node[right]{100} circle (0.2ex);
\end{tikzpicture}
\end{multicols}
\end{enumerate}

\end{document}
