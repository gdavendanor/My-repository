\documentclass[letterpaper,11pt,twoside]{article}
\usepackage[utf8]{inputenc}
\usepackage{amsmath,amsfonts,amssymb,latexsym}
\usepackage[spanish,es-noshorthands]{babel}
\usepackage[T1]{fontenc}
\usepackage{lmodern}
\usepackage{graphicx,hyperref}
\usepackage{tikz,pgf}
\usepackage{multicol}
\usepackage{fancyhdr}
\usepackage[height=9.5in,width=7in]{geometry}
\usepackage{fancyhdr}
\pagestyle{fancy}
\fancyhead[LE]{\includegraphics[height=12pt]{Images/logo-colegio.png} Álgebra $8^{\circ}$}
\fancyhead[RE]{}
\fancyhead[RO]{\textit{Germ\'an Avenda\~no Ram\'irez, Lic. U.D., M.Sc. U.N.}}
\fancyhead[LO]{}

\author{Germ\'an Avenda\~no Ram\'irez, Lic. U.D., M.Sc. U.N.}
\title{\begin{minipage}{.2\textwidth}
\includegraphics[height=1.75cm]{Images/logo-colegio.png}\end{minipage}
\begin{minipage}{.55\textwidth}
\begin{center}
Animaplano 4, grado $7^{\circ}$\\
Álgebra $8^{\circ}$
\end{center}
\end{minipage}\hfill
\begin{minipage}{.2\textwidth}
\includegraphics[height=1.75cm]{Images/logo-sed.png} 
\end{minipage}}
\date{}
\thispagestyle{plain}
\begin{document}
\maketitle
Nombre: \hrulefill Curso: \underline{801} Fecha: \underline{Julio 31 de 2014}
\section*{Cuestionario}
Resuelva al frente de cada pregunta con los procedimientos necesarios.
\begin{enumerate}
\item Halle $ \frac{1}{3} $ de 105
  \item Al triple del número 20, reste tres al cuadrado
  \item Reste a diez al cuadrado, el décimo número primo
  \item $ (2\times4\times4)+(100\div2)= $
  \item Sume 1, al resultado anterior
  \item $ (9\times3)+(80\div2)= $
  \item 1/2 centena - 1 unidad =
  \item La tercera parte de un ángulo recto
  \item Los segundos que hay en la tercera parte de un minuto
  \item El mínimo común múltiplo de los números 3 y 9
  \item La suma del número de lados y de vértices de un rectángulo
  \item El 35\% de 100
  \item Resuelva $ 3^3 $, más 1/3 de 90
  \item Represente en centímetros, 3 decímetros
  \item Sume $ 4!+\sqrt{25}= $ \qquad (Recuerde que $ n!=n\cdot(n-1)\cdot(n-2)\cdot\ldots\cdot3\cdot2\cdot1 $)
  \item Multiplique por 6, las veces en que está 6 en 18
  \item Multiplique por 12, las veces en que está 3 en 9
  \item Si $ 63-m=16 $, entonces $ m= $
  \item En la operación $ 38\div2=19 $, el dividendo es:
  \item Si $ m-21=16 $, entonces $ m= $
  \item En la operación $ 54\div2=27 $, el cociente es:
  \item En la operación $ 56\div28=2 $, el divisor es:
  \item El dividendo es 76, el cociente es 2, el divisor es:
  \item Sume $ 2^2+3^2+4^2= $
  \item $ 25+5+(9\times3)= $
  \item $ \sqrt{10000}-3^3= $
  \item Halle $ (3\times3)\times(4\times2)= $
  \item En unidades, 1/2 centena, más 1 decena, más 1 unidad:
\end{enumerate}
\section*{Animaplano}
\begin{center}
\begin{tikzpicture}
  \fill (1,0) node[above]{1} circle (0.2ex);
 \fill (2,0) node[above]{2} circle (0.2ex);
 \fill (3,0) node[above]{3} circle (0.2ex);
 \fill (4,0) node[above]{4} circle (0.2ex);
 \fill (5,0) node[above]{5} circle (0.2ex);
 \fill (6,0) node[above]{6} circle (0.2ex);
 \fill (7,0) node[above]{7} circle (0.2ex);
 \fill (8,0) node[above]{8} circle (0.2ex);
 \fill (9,0) node[above]{9} circle (0.2ex);
 \fill (10,0) node[above]{10} circle (0.2ex);
 \fill (1,-1) node[left]{11} circle (0.2ex);
 \fill (2,-1) circle (0.2ex);
 \fill (3,-1) circle (0.2ex);
 \fill (4,-1) circle (0.2ex);
 \fill (5,-1) circle (0.2ex);
 \fill (6,-1) circle (0.2ex);
 \fill (7,-1) circle (0.2ex);
 \fill (8,-1) circle (0.2ex);
 \fill (9,-1) circle (0.2ex);
 \fill (10,-1) circle (0.2ex);
 \fill (1,-2) node[left]{21} circle (0.2ex);
 \fill (2,-2) circle (0.2ex);
 \fill (3,-2) circle (0.2ex);
 \fill (4,-2) circle (0.2ex);
 \fill (5,-2) circle (0.2ex);
 \fill (6,-2) circle (0.2ex);
 \fill (7,-2) circle (0.2ex);
 \fill (8,-2) circle (0.2ex);
 \fill (9,-2) circle (0.2ex);
 \fill (10,-2) circle (0.2ex);
 \fill (1,-3) node[left]{31} circle (0.2ex);
 \fill (2,-3) circle (0.2ex);
 \fill (3,-3) circle (0.2ex);
 \fill (4,-3) circle (0.2ex);
 \fill (5,-3) circle (0.2ex);
 \fill (6,-3) circle (0.2ex);
 \fill (7,-3) circle (0.2ex);
 \fill (8,-3) circle (0.2ex);
 \fill (9,-3) circle (0.2ex);
 \fill (10,-3) circle (0.2ex);
 \fill (1,-4) node[left]{41} circle (0.2ex);
 \fill (2,-4) circle (0.2ex);
 \fill (3,-4) circle (0.2ex);
 \fill (4,-4) circle (0.2ex);
 \fill (5,-4) circle (0.2ex);
 \fill (6,-4) circle (0.2ex);
 \fill (7,-4) circle (0.2ex);
 \fill (8,-4) circle (0.2ex);
 \fill (9,-4) circle (0.2ex);
 \fill (10,-4) node[right]{50} circle (0.2ex);
 \fill (1,-5) node[left]{51} circle (0.2ex);
 \fill (2,-5) circle (0.2ex);
 \fill (3,-5) circle (0.2ex);
 \fill (4,-5) circle (0.2ex);
 \fill (5,-5) circle (0.2ex);
 \fill (6,-5) circle (0.2ex);
 \fill (7,-5) circle (0.2ex);
 \fill (8,-5) circle (0.2ex);
 \fill (9,-5) circle (0.2ex);
 \fill (10,-5) circle (0.2ex);
 \fill (1,-6) node[left]{61} circle (0.2ex);
 \fill (2,-6) circle (0.2ex);
 \fill (3,-6) circle (0.2ex);
 \fill (4,-6) circle (0.2ex);
 \fill (5,-6) circle (0.2ex);
 \fill (6,-6) circle (0.2ex);
 \fill (7,-6) circle (0.2ex);
 \fill (8,-6) circle (0.2ex);
 \fill (9,-6) circle (0.2ex);
 \fill (10,-6) circle (0.2ex);
 \fill (1,-7) node[left]{71} circle (0.2ex);
 \fill (2,-7) circle (0.2ex);
 \fill (3,-7) circle (0.2ex);
 \fill (4,-7) circle (0.2ex);
 \fill (5,-7) circle (0.2ex);
 \fill (6,-7) circle (0.2ex);
 \fill (7,-7) circle (0.2ex);
 \fill (8,-7) circle (0.2ex);
 \fill (9,-7) circle (0.2ex);
 \fill (10,-7) circle (0.2ex);
 \fill (1,-8) node[left]{81} circle (0.2ex);
 \fill (2,-8) circle (0.2ex);
 \fill (3,-8) circle (0.2ex);
 \fill (4,-8) circle (0.2ex);
 \fill (5,-8) circle (0.2ex);
 \fill (6,-8) circle (0.2ex);
 \fill (7,-8) circle (0.2ex);
 \fill (8,-8) circle (0.2ex);
 \fill (9,-8) circle (0.2ex);
 \fill (10,-8) circle (0.2ex);
 \fill (1,-9) node[left]{91} circle (0.2ex);
 \fill (2,-9) circle (0.2ex);
 \fill (3,-9) circle (0.2ex);
 \fill (4,-9) circle (0.2ex);
 \fill (5,-9) circle (0.2ex);
 \fill (6,-9) circle (0.2ex);
 \fill (7,-9) circle (0.2ex);
 \fill (8,-9) circle (0.2ex);
 \fill (9,-9) circle (0.2ex);
 \fill (10,-9) node[right]{100} circle (0.2ex);
\end{tikzpicture}
\end{center}
\end{document}
