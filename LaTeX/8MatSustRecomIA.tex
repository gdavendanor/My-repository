\documentclass[fleqn]{article}
\usepackage[spanish,es-noshorthands]{babel}
\usepackage[utf8]{inputenc}
\usepackage{amsmath}
\usepackage{mathexam}
\usepackage[papersize={6.5in,8.5in},left=1cm, right=1cm, top=1.5cm, bottom=1.7cm]{geometry}
\usepackage{graphicx}
\usepackage{tikz}

\ExamClass{\includegraphics[height=16pt]{Images/logo-sed.png} Matemáticas $8^{\circ}$}
\ExamName{Recomendaciones I, Sustentación}
\ExamHead{\includegraphics[height=16pt]{Images/logo-colegio.png} IEDAB}
\newcommand{\LineaNombre}{%
\par
\vspace{\baselineskip}
Nombre:\hrulefill \; Curso: \underline{603} \; Fecha: \underline{\hspace*{2.5cm}} \relax
\par}
\let\ds\displaystyle

\begin{document}
\ExamInstrBox{
Respuesta sin justificar mediante procedimiento no será tenida en cuenta en la calificación. Escriba sus respuestas en el espacio indicado. Tiene 45 minutos para contestar esta prueba.}
\LineaNombre
\begin{enumerate}
 \item Realice las siguientes operaciones, simplificando la respuesta al máximo
\begin{enumerate}
 \item $\dfrac{3}{5}+\left(\dfrac{-3}{7}\right)=$\noanswer
\item $\dfrac{3}{5}-\dfrac{4}{9}=$\noanswer
\item $\dfrac{2}{4}\cdot\left(\dfrac{-1}{5}\right)=$\noanswer
\item $\dfrac{3}{5}\div\left(\dfrac{-5}{3}\right)=$\noanswer
\item $\dfrac{5\left(\frac{3}{5}\right)+3\left(\frac{-2}{8}\right)}{\frac{7}{5}\left(\frac{-1}{5}\right)-\frac{3}{4}\left(\frac{-4}{5}\right)}=$\noanswer
\end{enumerate}
\item Solucione la ecuación \qquad $x+\dfrac{3}{5}=6$\noanswer
\item Convierta a número decimal las siguientes fracciones
 \newpage
\begin{enumerate}
 \item $\dfrac{5}{100}$\noanswer
\item $\dfrac{3}{5}$\noanswer
\item $\dfrac{4}{9}$\noanswer
\end{enumerate}
\item Convierta a número fraccionario los siguientes decimales
\begin{enumerate}
\item 3,95\noanswer
\item $0,\overline{79}$ \noanswer
\end{enumerate}
\item Ubique en la recta numérica el número $\dfrac{4}{5}$
\begin{center}
 \begin{tikzpicture}[>=stealth]
\draw[|->] (0,0) node[below]{0}-- (11,0);
\draw [thick] (1,-.1)  -- (1,0.1);
\draw [thick] (2,-.1)  -- (2,0.1);
\draw [thick] (3,-.1) -- (3,0.1);
\draw [thick] (4,-.1) -- (4,0.1);fleqn
\draw [thick] (5,-.1) -- (5,0.1);
\draw [thick] (6,-.1) -- (6,0.1);
\draw [thick] (7,-.1) -- (7,0.1);
\draw [thick] (8,-.1) -- (8,0.1);
\draw [thick] (9,-.1)--(9,.1);
\draw [thick](10,-.1)node[below]{1}--(10,.1);
\end{tikzpicture}
\end{center}
\end{enumerate}
\end{document}