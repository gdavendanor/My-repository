\documentclass[fleqn]{article}
\usepackage[spanish,es-noshorthands]{babel}
\usepackage[utf8]{inputenc} 
\usepackage[papersize={6.5in,8.5in},left=1cm, right=1cm, top=1.5cm, bottom=1.7cm]{geometry}
\usepackage{mathexam}
\usepackage{amsmath}
\usepackage{graphicx}

\ExamClass{\includegraphics[height=16pt]{Images/logo-sed.png} Matemáticas $11^{\circ}$}
\ExamName{Recomendaciones I, Sustentación}
\ExamHead{\includegraphics[height=16pt]{Images/logo-colegio.png} IEDAB}
\newcommand{\LineaNombre}{%
\par
\vspace{\baselineskip}
Nombre:\hrulefill \; Curso: \underline{\hspace*{36pt}} \; Fecha: \underline{\hspace*{2.5cm}} \relax
\par}
\let\ds\displaystyle

\begin{document}
\ExamInstrBox{
Respuestas sin justificación procedimental no tendrán puntaje. Escriba sus respuestas en el espacio indicado. Usted tiene 50 minutos.}
\LineaNombre
\begin{enumerate}
   \item Complete la siguiente tabla escribiendo $\in$ o $\not\in$ según el caso:
\begin{center}
   \begin{tabular}{|c|c|c|c|c|c|}
\hline 
Número & 2 & $-3\pi$ & $-4.5$ & $-2.\overline{3}$ & $\sqrt{49}$ \\ 
\hline 
Natural & $\in$ &  & $\not\in$ &  &  \\ 
\hline 
Entero &  &  &  &  &  \\ 
\hline 
Racional &  &  &  &  &  \\ 
\hline 
Irracional &  &  &  &  &  \\ 
\hline 
Real &  &  &  &  &  \\ 
\hline 
\end{tabular} 
\end{center}
 \item Encuentre las fracciones generatrices de los siguientes números
\begin{enumerate}
\item $0,65$\noanswer
\item $2,5=$\noanswer
\item $3,45=$\noanswer
\item $0,\overline{7}=$\noanswer
\item $2,7\overline{9}=$\noanswer
\end{enumerate}
\item Señale si son ciertos o falsos los siguientes enunciados:
\begin{enumerate}
\item El número $\dfrac{6}{11}$ es irracional porque tiene una cantidad ilimitada de cifras decimales \underline{\hspace*{20pt}}
\item Todo número real es racional \underline{\hspace*{20pt}}
\newpage
\item Todo número natural es racional \underline{\hspace*{20pt}}
\item $\sqrt[3]{216}$ es un número irracional \underline{\hspace*{20pt}}
\item $\sqrt{48}$ es un número racional \underline{\hspace*{20pt}}
\end{enumerate}
\item Calcule y/o simplifique:
\begin{enumerate}
\item $\sqrt{784}=$ \noanswer
\item $\sqrt[3]{3375}$\noanswer
\item $25-3\sqrt{144}=$\noanswer
\item $\dfrac{3^{10}}{9^{2}}=$\noanswer
\end{enumerate}
\item ¿Cuántas baldosas cuadradas de 30 cm de lado, se necesitan para cubrir una superficie de 8,82 $m^{2}$?\noanswer[2in]
\end{enumerate}
\end{document}
