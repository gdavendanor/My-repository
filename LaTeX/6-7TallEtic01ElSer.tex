\documentclass[twoside]{article}
\usepackage[utf8]{inputenc}
\usepackage{amsmath,amsfonts,amssymb,amsthm,latexsym}
\usepackage[spanish,es-noshorthands]{babel}
\usepackage[T1]{fontenc}
\usepackage{lmodern}
\usepackage{graphicx,hyperref}
\usepackage{tikz,pgf}
\usepackage{marvosym}
\usepackage{multicol}
\usepackage{fancyhdr}
\usepackage[papersize={5.5in,8.5in},left=.75cm,right=.75cm,top=1.35cm,bottom=1.25cm]{geometry}
\usepackage{fancyhdr}
\pagestyle{fancy}
\fancyhead[LE]{\Email matematicas.german@gmail.com}
\fancyhead[RE]{}
\fancyhead[RO]{\url{https://www.autistici.org/mathgerman}}
\fancyhead[LO]{}

\author{Germ\'an Avenda\~no Ram\'irez~\thanks{Lic. Mat. U.D., M.Sc. U.N.}}
\title{\begin{minipage}{.2\textwidth}
\includegraphics[height=1.75cm]{Images/logo-colegio.png}\end{minipage}
\begin{minipage}{.55\textwidth}
\begin{center}
Taller 01, ``El ser''\\
Ética $6^{\circ}$
\end{center}
\end{minipage}\hfill
\begin{minipage}{.2\textwidth}
\includegraphics[height=1.75cm]{Images/logo-sed.png} 
\end{minipage}}
\date{}
\thispagestyle{plain}
\begin{document}
\maketitle
Nombre: \hrulefill Curso: \underline{\hspace*{44pt}} Fecha: \underline{\hspace*{2.5cm}}

\section*{Lo que sé}
Observa tu reflejo en un espejo y responde:
\begin{itemize}
\item ¿Quién eres? ¿Cómo te ves?
\item ¿Qué elementos conforman tu ser?
\item ¿Te gusta tu aspecto físico? ¿Por qué?
\item ¿Te gusta tu manera de ser? ¿Por qué?
\item ¿Cuáles son tus principales cualidades?
\item ¿Cuáles son tus principales defectos?
\item ¿Cómo crees que los demás te ven a ti? ¿Por qué?
\end{itemize}
\section*{Aprendo algo nuevo}
No es fácil definir qué es el ser. Si se emplea esta palabra como un verbo y se acompaña de un sujeto, es probable que se esté refiriendo a una característica o atributo de alguien o algo, por ejemplo:
\begin{itemize}
\item Horacio \emph{es} enfermero.
\item Yo \emph{soy} muy generoso.
\item El universo \emph{es} infinito.
\end{itemize}
Sin embargo, cuando se emplea esta palabra como algo
absoluto o como un sujeto en sí, es decir \emph{el ser}, es probable que el nivel de reflexión vaya más allá y se refiera a varias cosas: Dios, esencia, espíritu, elemento\ldots Saber qué es el ser, ha sido una de las grandes preocupaciones de la humanidad.

En la antigüedad, algunos filósofos se dedicaron a pensar en la naturaleza y sus procesos, y en consecuencia, en la existencia de un elemento que constituyera todo lo demás.
Es así como el fuego, el aire, el agua y la tierra se consideraron como los elementos \emph{físicos} principales que conformaban todo cuanto existía.

Luego, esta idea fue revaluada y se prefirió pensar en la
existencia de algo \emph{indeterminado} como elemento constitutivo del mundo físico. Pero ninguna de estas ideas terminaron por convencer a \emph{Parménides}, uno de los tantos filósofos griegos cuya principal preocupación fue definir cómo se llegaba al conocimiento. Parménides concluyó que existían dos formas de hacerlo: los sentidos y la razón. Sin embargo para él, las percepciones que se experimentaban mediante los sentidos podían ser tan cambiantes o engañosas que no era posible tener certeza acerca del conocimiento que proporcionaban. Fue así como este filósofo decidió pensar que solo a través de la razón se podía llegar al conocimiento. El concluyó que todos los objetos, independientemente de sus elementos físicos constitutivos, tienen algo en común: \emph{son}, es decir, cualquier elemento de la naturaleza existe porque es. De esta manera Parménides llegó al concepto del ser como un elemento fundamental que puede \emph{pensarse}, \emph{es} o existe.

En este sentido también se puede llegar a explicar qué es
el ser humano.
\section*{El ser humano}
El ser humano es aquel que tiene la capacidad de pensarse
a sí mismo y de reconocerse como un ser \emph{especial}, \emph{único} e \emph{irrepetible}, diferente a los demás seres y elementos de la naturaleza.

Los seres humanos son seres vivos porque tienen vida
o realizan actividades que les permiten vivir o adaptarse al medio y por ello están constituidos de materia o cuerpo. También se caracterizan por sentir y experimentar emociones y, por último, por tener la capacidad de aprender o de reconstruir sus experiencias.

Es así como un ser humano posee tres componentes básicos: el ser \emph{corporal}, el ser \emph{emocional} y el ser \emph{trascendente}.
\subsection*{El ser corporal}
Se refiere al cuerpo, es decir a los elementos físicos o biológicos que conforman a los seres humanos.

El cuerpo de cada ser humano es el producto de su herencia genética; por medio de éste es posible respirar, sentir, percibir las sensaciones del medio y experimentar placer. Por eso, el cuidado del cuerpo no es solo una cuestión de salud; también es una obligación moral.

Se suele afirmar que una persona expresa qué tanto se
quiere a sí misma por la manera como cuida su cuerpo. Esto no significa que todos los cuerpos deban ser perfectos dentro de los modelos estéticos que comúnmente se presentan. Lo que sugiere, es que el aspecto de las personas demuestra en buena parte cuál es su nivel de \textbf{autoestima}. Por ello los seres humanos tienen responsabilidades específicas con su ser corporal, dentro de las cuales están:
\begin{itemize}
\item Alimentarse sanamente.
\item Descansar el tiempo necesario para recuperar energías. 
\item Hacer ejercicio de manera regular.
\item No caer en excesos que hagan daño a tu organismo.
\end{itemize}
\section*{Ejercito lo aprendido}
Analice y responda
\begin{enumerate}
\item ¿Estas de acuerdo con Parménides en relación
con la información que brindan los sentidos? ¿Por qué?
\item Elabora tu propio concepto acerca del ser y
compártelo con tus compañeros.
\item ¿De qué forma demuestras que eres responsable de tu ser corporal?
\end{enumerate}
\subsection*{Ser emotivo}
Corresponde al ser que comprende las \textbf{emociones} y los \textbf{sentimientos}. Se relaciona con la capacidad que poseen los seres humanos para expresar lo que sienten frente a diferentes estímulos. El ser emotivo reúne las habilidades que tienen las personas para relacionarse con otros y consigo mismas.

Todos los seres humanos experimentan emociones, es decir, alteraciones del ánimo que manifiestan físicamente a través del llanto, la risa, la ansiedad o cualquier otra forma de expresión. Los sentimientos por su parte, son el resultado de las emociones ya que después de experimentar un estímulo dado por ciertos comportamientos, acciones y/o actitudes, las personas son susceptibles de sentir amor, odio, esperanza, temor\ldots

No existen emociones ni sentimientos buenos o malos,
lo importante es manejarlos adecuadamente y no permitir el dominio de aquellos que entorpecen las relaciones interpersonales o intrapersonales.
\subsubsection*{La inteligencia emocional}
Se define como la capacidad que tienen los seres humanos
para responder frente a una situación en la que entran en
juego sus sentimientos y emociones.
\paragraph*{La inteligencia emocional se evidencia con:}
\begin{itemize}
\item El conocimiento de las propias emociones y sus
respectivas causas.
\item La habilidad para controlar las emociones y las
reacciones que éstas generan cuando se desbordan.
\item La capacidad para manejar los conflictos que surgen
entre la razón y los sentimientos.
\item La creatividad para sentirse constantemente motivado frente a diferentes situaciones.
\item La pericia para mantener el equilibrio frente a las
tensiones externas.
\end{itemize}
\includegraphics[scale=.55]{Images/Captura_260116-18:56:25.png} 
\section*{Evaluación}
Interprete y opine teniendo en cuenta el gráfico anterior
\begin{enumerate}
\item Lee estos casos y describe cuál sería tu reacción en cada uno de ellos.
\item Describe los sentimientos que despierta en ti cada situación.
\item ¿Qué estrategias emplearías para controlar tus emociones en cada caso?
\end{enumerate}
\section*{Actividad final}
Ejercita tu autoafirmación:
\begin{itemize}
\item Escribe una lista de todas tus habilidades, competencias y destrezas.
\item Evalúa en cuál de ellas te desempeñas mejor.
\item Elabora una lista de tus limitaciones y al frente de cada una de ellas escribe cómo podrías superarlas.
\end{itemize}
\end{document}
