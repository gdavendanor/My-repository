\documentclass[fleqn]{article}
\usepackage[spanish,es-noshorthands]{babel}
\usepackage[utf8]{inputenc} 
\usepackage[papersize={6.5in,8.5in},total={5.5in,7.25in},centering]{geometry}
\usepackage{mathexam}
\usepackage{amsmath}
\usepackage{graphicx}
\usepackage{textcomp}
\usepackage{multicol}
\usepackage{tikz,pgf}
\ExamClass{\includegraphics[height=16pt]{Images/logo-sed.png} Matemáticas $11^{\circ}$}
\ExamName{''Niv. Funciones``}
\ExamHead{\includegraphics[height=16pt]{Images/logo-colegio.png} IEDAB}
\newcommand{\LineaNombre}{%
\par
\vspace{\baselineskip}
Nombre:\hrulefill \; Curso: \underline{\hspace*{48pt}} \; Fecha: \underline{\hspace*{2.5cm}} \relax
\par}
\let\ds\displaystyle

\begin{document}
\ExamInstrBox{
Respuesta sin justificar mediante procedimiento no será tenida en cuenta en la calificación. Escriba sus respuestas en el espacio indicado. Tiene 45 minutos para contestar esta prueba.}
\LineaNombre
\begin{enumerate}
 \item Exprese la regla en notación de funciones. Por ejemplo, la regla:\\
  ``El cuadrado y luego reste 5'', es expresada en notación funcional como $f(x)=x^{2}-5$
 \begin{enumerate}
  \item El cuadrado de la diferencia entre $x$ y 5: \dotfill
  \item El cociente entre la diferencia de $x$ y 3 y la diferencia de $x$ y 2: \dotfill
  \item La raíz cuadrada del cubo de la diferencia entre $x$ y 4: \dotfill
 \end{enumerate}
\item Exprese la función (o regla) en palabras
\begin{enumerate}
 \item $f(x)=3\sqrt{x-3}$: \dotfill
 \item $\dfrac{x-4}{x+5}$: \dotfill
 \item $3x^{3}-4x^{2}+5x-1$: \dotfill
\end{enumerate}
\item Complete la tabla para la función dada por $f(x)=-x^{2}+3$ y luego grafíquela en el plano

\begin{minipage}{0.5\textwidth}
\begin{tabular}{|c|c|}\hline
$x$ & $f(x)$\\ \hline
--3 & \\ \hline
--2 & \\ \hline
--1 & \\ \hline
0 & \\ \hline
1 & \\ \hline
2 & \\ \hline
3 & \\ \hline
\end{tabular}
\end{minipage}\hfill
\begin{minipage}{0.45\textwidth}
 \begin{tikzpicture}[scale=0.5]
  \draw [dashed,help lines] (-3.5,-6.5) grid (3.5,3.5);
  \draw [thick,<->](-3.5,0)-- (3.55,0)node[right]{$x$};
  \draw [thick,<->](0,-6.5)--(0,3.5)node[right]{$y$};
 \end{tikzpicture}
\end{minipage}
\item Para la función definida a trozos:
\[f(x)=\left\{ \begin{array}{lcl}
              x^{2}+2x & \mbox{si} & x\leq-1\\
              2x & \mbox{si} & x>-1\\
             \end{array}
\right. \]
Halle:
\begin{enumerate}
\begin{multicols}{3}
 \item $f(-2)=$
 \item $f(-1)=$
 \item $f(0)=$
 \item $f(1)=$
 \item $f(2)=$
 \item $f(\frac{1}{2})=$
\end{multicols}
\end{enumerate}
%\item Para la función $f(x)=3x^{2}-2x+4$ halle:
%\begin{enumerate}
% \item $f(a)=$\vspace{10pt}
% \item $f(2x)=$\vspace{10pt}
% \item $2f(x)=$\vspace{10pt}
% \item $f(x+1)=$\vspace{10pt}
% \item $f(x)+f(1)=$\vspace{10pt}
% \item $f(a+h)=$
%\end{enumerate}
\item Dada las funciones del gráfico (página siguiente), encuentre:

\begin{minipage}{.35\textwidth}
\begin{tikzpicture}[scale=.65,domain=-2:2.1,smooth]
\draw[dashed,help lines] (-2,-2)grid(2,4.25);
\draw[<->](-2.2,0)--(2.2,0)node[below]{$x$};
\draw[<->](0,-2.2)--(0,4.25)node[right]{$y$};
\draw plot (\x,4*\x*\x-\x*\x*\x*\x-1)node[right]at(1.25,3.25){$f$};
\draw plot (\x,-\x*\x+4) node[right] at (0,3.15){$g$};
\end{tikzpicture}
\end{minipage}
\begin{minipage}{.6\textwidth}
\begin{enumerate}
\begin{multicols}{2}
\item $f(0)=$\noanswer
\item $g(0)=$\noanswer
\item $f(-2)=$\noanswer
\item $g(-1)=$\noanswer
\item $f(-1)=$\noanswer
\item $f(1)=$\noanswer
\item $g(1)=$\noanswer
\end{multicols}
\end{enumerate}
\end{minipage}
La siguiente gráfica muestra la relación entre la velocidad de un molino y el tiempo de funcionamiento en un día.}
\begin{center}
\begin{tikzpicture}[scale=.65]
\draw[<->] (0,0)--(6.5,0);
\node[below] at (3.25,-0.5){Tiempo (horas)};
\foreach \x in {0.5,1,1.5,2,2.5,3,3.5,4,4.5,5,5.5,6,6.5} \draw[shift={(\x,0)},color=black] (0pt,0pt) -- (0pt,-4pt) node[below] {\footnotesize $\x$};
\draw[<->](0,0)--(0,5);
\foreach \y in {0.5,1,1.5,2,2.5,3,3.5,4,4.5,5} \draw[shift={(0,\y)},color=black] (0pt,0)--(-4pt,0) node[left]{\footnotesize $\y$};
\node[rotate=90] at (-1,2.5) {Velocidad (en miles de rpm)};
\draw(0,0)--(2,1)--(3,2.5)--(3.5,1)--(4.5,4.5)--(6,0);
\fill[black](6,0) circle [radius=2pt] ;
\fill[black](4.5,4.5) circle [radius=2pt];
\fill[black] (3.5,1) circle [radius=2pt];
\fill[black] (3,2.5) circle [radius=2pt];
\fill[black] (0,0) circle [radius=2pt];
\fill[black] (2,1) circle [radius=2pt];
\end{tikzpicture}
\end{center}
\item \label{q1} El molino aumentó más rápidamente su velocidad entre
\begin{enumerate}
	\begin{multicols}{2}
\item la hora 3 y la hora 5
\item la hora 3,5 y la hora 4,5
\item la hora 4,5 y la hora 6
\item la hora 2 y la hora 3
	\end{multicols}
\end{enumerate}
\item En una empresa el costo de producir un computador es $c$. Si se venden $y$ computadores con un precio de $v$ cada uno, entonces la expresión correcta para la ganancia $g$ es:
\begin{enumerate}
	\begin{multicols}{4}
\item $g=vy-c$
\item $g=c-vy$
\item $g=y(v-c)$
\item $g=y(v+c)$
	\end{multicols}
Justifique sus respuestas
\end{enumerate}


%\section*{Probabilidad}
%\item Si se lanzan un par de dados y $P(6)$ indica la probabilidad de que la suma de los números superiores del dado sea 6, entonces halle:
%\begin{enumerate}
%\begin{multicols}{2}
%\item $P(5)=$\noanswer
%\item $P(4)=$ \noanswer
%\item $P(3)=$\noanswer
%\item $P(2)=$\noanswer
%\end{multicols}
%\end{enumerate}
%\item Se lanzan tres monedas simultáneamente, encuentre:
%\begin{enumerate}
%\item El espacio muestral $S$ \noanswer%$S=\{ccc,ccs,csc,scc,css,scs,ssc,sss\}$
%\item La probabilidad de obtener dos caras:\noanswer
%\item Obtener solamente una cara:\noanswer
%\item La probabilidad de obtener tres caras:\noanswer
%\end{enumerate}
 \end{enumerate}
\end{document}
