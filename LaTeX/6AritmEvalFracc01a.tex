\documentclass[fleqn]{article}
\usepackage[spanish,es-noshorthands]{babel}
\usepackage[utf8]{inputenc} 
\usepackage[papersize={6.5in,8.5in},left=1cm, right=1cm, top=1.5cm, bottom=1.7cm]{geometry}
\usepackage{mathexam}
\usepackage{amsmath}
\usepackage{graphicx}
\usepackage{tikz,pgf}
\usepackage{mathrsfs}
\usetikzlibrary{arrows}
\usepackage{multicol}

\ExamClass{\includegraphics[height=16pt]{Images/logo-sed.png} Aritmética $6^{\circ}$}
\ExamName{Fracciones}
\ExamHead{\includegraphics[height=16pt]{Images/logo-colegio.png} IEDAB}
\newcommand{\LineaNombre}{%
\par
\vspace{\baselineskip}
Nombre:\hrulefill \; Curso: \underline{\hspace*{48pt}} \; Fecha: \underline{\hspace*{2.5cm}} \relax
\par}
\let\ds\displaystyle

\begin{document}
\ExamInstrBox{
Respuesta sin justificar mediante procedimiento no será tenida en cuenta en la calificación. Escriba sus respuestas en el espacio indicado. Tiene 45 minutos para contestar esta prueba.}
\LineaNombre
\begin{enumerate}
 \item Identifique en las siguientes gráficas la fracción correspondiente a la región sombreada, escríbala y si se puede simplifíquela.
 \begin{enumerate}
 \begin{multicols}{2}
 \item \definecolor{zzttqq}{rgb}{0.6,0.2,0.}
\begin{tikzpicture}[line cap=round,line join=round,>=triangle 45,x=1.0cm,y=1.0cm,scale=.4]
\clip(-4.94,-3.24) rectangle (5.6,4.);
\draw (0.32142135623730983,2.9758787847867993)-- (-2.3958787847868,-2.61857864376269);
\draw (-3.8344574285494897,1.5373001410241098)-- (1.76,-1.18);
\draw(-1.037228714274745,0.17865007051205492) circle (3.1097296496746387cm);
\draw (-2.0544574285494894,3.117300141024109)-- (-0.02,-2.76);
\draw (-3.9758787847867993,-0.8385786437626899)-- (1.9014213562373095,1.1958787847867993);
\draw [shift={(-1.037228714274745,0.17865007051205492)},color=zzttqq,fill=zzttqq,fill opacity=0.6]  (0,0) --  plot[domain=1.1186435857875154:5.045634402774757,variable=\t]({1.*3.1097296496746387*cos(\t r)+0.*3.1097296496746387*sin(\t r)},{0.*3.1097296496746387*cos(\t r)+1.*3.1097296496746387*sin(\t r)}) -- cycle ;
\end{tikzpicture}
\item \definecolor{zzttqq}{rgb}{0.6,0.2,0.}
\begin{tikzpicture}[line cap=round,line join=round,>=triangle 45,x=1.0cm,y=1.0cm]
\clip(-4.92,0.34) rectangle (-2.1,3.14);
\draw (-4.740281662206479,2.1827687752661227)-- (-2.36,1.34);
\draw (-2.59014083110324,2.5813843876330616)-- (-4.51014083110324,0.9413843876330616);
\draw(-3.5501408311032407,1.761384387633062) circle (1.26253712816693cm);
\draw (-3.78028166220648,3.0027687752661225)-- (-3.32,0.52);
\draw [shift={(-3.5501408311032407,1.761384387633062)},color=zzttqq,fill=zzttqq,fill opacity=0.6]  (0,0) --  plot[domain=-1.3874870416330785:2.801303163153312,variable=\t]({1.*1.2625371281669315*cos(\t r)+0.*1.2625371281669315*sin(\t r)},{0.*1.2625371281669315*cos(\t r)+1.*1.2625371281669315*sin(\t r)}) -- cycle ;
\end{tikzpicture}
  \end{multicols}
 \end{enumerate}
 \item Un atleta diariamente da 24 vueltas a una pista. Hoy, cuando corría, sufrió una lesión y solamente había hecho 18 vueltas. ¿Qué  fracción de lo que normalmente corre alcanzó a hacer? \noanswer
 \item Encuentre 10 fracciones equivalentes a cada una de las siguientes fracciones:
 \begin{enumerate}
 \item $\dfrac{2}{3}=\dfrac{4}{6}=$ \dots
 \item $\dfrac{5}{8}=$
 \item $\dfrac{1}{3}=$
 \item $\dfrac{5}{6}=$
 \item $\dfrac{7}{8}=$
 \end{enumerate}
 \item Con base en lo resuelto en el anterior punto, realice las siguientes operaciones:
 \begin{enumerate}
 \item $\dfrac{1}{3}+\dfrac{5}{8}=$
 \item $\dfrac{5}{8}-\dfrac{2}{3}=$
 \item $\dfrac{5}{6}+\dfrac{2}{3}=$
 \item $\dfrac{7}{8}-\dfrac{1}{3}=$
 \end{enumerate}
 \item Simplique al máximo las siguientes fracciones:
\begin{enumerate}
\begin{multicols}{2}
\item $\dfrac{4}{10}=$
\item $\dfrac{6}{15}=$
\end{multicols}
\end{enumerate}
\item Julian quiere comprar $\frac{4}{6}$ de kilo de Jamón pero en el supermercado solo encuentra  paquetes de $\frac{1}{3}$ de kilo. ¿Cuántos paquetes debe comprar Julian?\noanswer
\item Valentina y Reinel comen torta. Si Valentina come $\frac{4}{8}$ de torta y Reinel $\frac{2}{4}$ de torta, ¿comen ambos la misma cantidad de torta?\noanswer
\item Joseph el pastelero, necesita $\frac{4}{12}$ de kilo de levadura. Si en la cocina hay medidas de $\frac{1}{2}$, $\frac{1}{3}$, $\frac{1}{4}$ y $\frac{1}{6}$, ¿cu\'{a}l es la medida m\'{a}s grande que debe usar para que no le sobre ni le falte levadura?\noanswer
 \item Enriqueta compró una papaya para compartirla con su familia. Si al hijo mayor le dió $\frac{2}{12} $ de papaya, a su hijo menor $\frac{1}{12}$ de papaya, a su marido $\frac{3}{12}$ de papaya y ella se comió $\frac{2}{12}$ de papaya. ¿Se comieron toda la papaya? ¿Cuánta papaya les sobró?\noanswer
 \end{enumerate}

\end{document}
