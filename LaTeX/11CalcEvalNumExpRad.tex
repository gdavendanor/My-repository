\documentclass[fleqn]{article}
\usepackage[spanish,es-noshorthands]{babel}
\usepackage[utf8]{inputenc} 
\usepackage[papersize={6.5in,8.5in},left=1cm, right=1cm, top=1.5cm, bottom=1.7cm]{geometry}
\usepackage{mathexam}
\usepackage{amsmath}
\usepackage{graphicx}

\ExamClass{\includegraphics[height=16pt]{Images/logo-sed.png} Cálculo $11^{\circ}$}
\ExamName{Exponentes y radicales}
\ExamHead{\includegraphics[height=16pt]{Images/logo-colegio.png} IEDAB}
\newcommand{\LineaNombre}{%
\par
\vspace{\baselineskip}
Nombre:\hrulefill \; Curso: 110\underline{\hspace*{12pt}} \; Fecha: \underline{\hspace*{2.5cm}} \relax
\par}
\let\ds\displaystyle

\begin{document}
\ExamInstrBox{
Respuestas sin procedimiento no tendrán puntaje. Escriba sus procedimientos y respuestas en el espacio dado. Usted tiene 45 minutos.}
\LineaNombre
\begin{enumerate}
   \item Evalúe cada expresión
   \begin{enumerate}
   \item $-7^{2}$\noanswer
   \item $(-7)^{2}$\noanswer
   \item $\dfrac{11^{5}}{11^{3}}=$ \noanswer
   \item $\dfrac{4^{3}}{4^{0}}=$ \noanswer
   \item $(\frac{3}{4})^{-2}=$ \noanswer
   \item $\sqrt[3]{-216}=$\noanswer
   \item $\sqrt{48}\sqrt{3}=$\noanswer
   \item $\dfrac{\sqrt{28}}{\sqrt{7}}=$\noanswer
   \item $\sqrt[3]{81}\sqrt[3]{9}=$\noanswer
   \item $81^{-0.25}=$\noanswer
   \end{enumerate}
   \newpage
   \item Calcule:
   \begin{enumerate}
   \item $\sqrt{2304}=$\noanswer
   \item $\sqrt[3]{-1728}=$\noanswer
   \item $\sqrt{41+2\sqrt{13+\sqrt[3]{27}}}=$\noanswer
   \item $\sqrt{\frac{16}{9}\div\sqrt{\frac{144}{36}}}=$\noanswer
   \item $\sqrt[4]{\frac{16}{1296}}=$\noanswer
   \end{enumerate}
   \end{enumerate}
\end{document}
