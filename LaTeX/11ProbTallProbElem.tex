\documentclass[10pt,twoside]{article}
\usepackage[utf8]{inputenc}
\usepackage{amsmath}
\usepackage{amsfonts}
\usepackage{amssymb}
\usepackage[spanish,es-noshorthands]{babel}
\usepackage[T1]{fontenc}
\usepackage{lmodern}
\usepackage{graphicx,hyperref}
\usepackage{tikz,pgf}
\usepackage{multicol}
\usepackage{subfig}
\usepackage[papersize={6.5in,8.5in},includeheadfoot,width=5.5in,height=7.5in]{geometry}
\usepackage{fancyhdr}
\pagestyle{fancy}
\fancyhead[LE]{\url{http://www.autistici.org/mathgerman}}
\fancyhead[RE]{}
\fancyhead[RO]{matematicas.german@gmail.com}
\fancyhead[LO]{}

\author{Germ\'an Dar\'io Avenda\~no Ram\'irez~\thanks{Lic. Mat. U.D., M.Sc. U.N.}}
\title{\begin{minipage}{0.15\textwidth}\includegraphics[height=1.7cm]{Images/logo-colegio.png}
\end{minipage}\hfill \begin{minipage}{0.85\textwidth}\begin{center}
Probabilidad elemental\\Probabilidad $11^{\circ}$\end{center}
\end{minipage}}
\date{}

\begin{document}
\maketitle
Nombre: \hrulefill Curso: \underline{\hspace{1cm}}  Fecha: \underline{\hspace{2cm}}\\
\section{Taller}
\subsection{Nivel I}
\begin{enumerate}
  \item Sea el experimento aleatorio “lanzar un dado”. Escribe el es pacio muestral e indica dos sucesos aleatorios que consten
de tres sucesos elementales cada uno.
\item Se saca una carta de una baraja española de 40 cartas. Escribe los sucesos contrarios de los siguientes:
\begin{enumerate}
  \item A = “sacar un as”
  \item B = “obtener un número primo”
  \item C = “obtener puntuación impar”
  \item D = “obtener puntuación positiva”
\end{enumerate}
\item Se lanza un dado. Escribe los siguientes sucesos y halla sus probabilidades:
\begin{enumerate}
  \item A = “obtener un número mayor que 3”
  \item B = “obtener un número primo”
  \item C = “obtener puntuación impar”
  \item D = “obtener puntuación positiva”
\end{enumerate}
\item Con los datos del problema anterior, indica qué sucesos son
los siguientes y halla la probabilidad de cada uno.
\begin{multicols}{3}
  \begin{enumerate}
    \item $ \overline{A} $
    \item $ \overline{B} $
    \item $ A\cup B $
    \item $ A\cap B $
    \item $ B\cap \overline{B} $
    \item $ \overline{A\cap B} $
    \item $ \overline{A}\cap \overline{B} $
    \item $ \overline{A\cup B} $
    \item $ \overline{A}\cup \overline{B} $
    \item $ (A\cap B)\cap C $
    \item $ \overline{(A\cap B)\cap C} $
    \item $ (\overline{A}\cap\overline{B})\cup\overline{C} $
  \end{enumerate}
\end{multicols}
\item El espacio muestral de un experimento aleatorio es  \{1,2,3,4,5,6,7,8,9\}. Sean los sucesos:
\[ A=\{3,5,6,8\}\qquad B=\{1,2,3,4,8,9\}\qquad C=\{1,4,5,7,9\} \]
Calcule la probabilidad de los sucesos:
\begin{multicols}{3}
  \begin{enumerate}
    \item $ \overline{C} $
    \item $ A\cup C $
    \item $ A\cup\overline{C}\cup B $
    \item $ A\cap\overline{B} $
    \item $ A\cup\overline{B} $
    \item $ \overline{A}\cap B $
  \end{enumerate}
\end{multicols}
\item Con los datos anteriores, halla los siguientes sucesos y sus probabilidades.
\begin{multicols}{2}
  \begin{enumerate}
    \item $ (\overline{A}\cap\overline{B})\cap\overline{C} $
    \item $ \overline{(A\cap B)\cap C} $
    \item $ (\overline{A}\cap\overline{B})\cup\overline{C} $
    \item $ B\cup(A\cap C) $
    \item $ (B\cup A)\cap(B\cup C) $
    \item $ B\cap (A\cup C) $
  \end{enumerate}
\end{multicols}
\item Se considera el experimento aleatorio “lanzar tres monedas”. Construye el espacio muestral.
\item Sea el experimento del problema anterior. Se  consideran los sucesos:
\begin{multicols}{2}
  A = “sacar solo una cara”\\
  B = “sacar al menos una cruz”\\
  C = “sacar tres caras o tres cruces”\\
\end{multicols}
Halla las probabilidades de:
\begin{multicols}{3}
  \begin{enumerate}
    \item $ A\cap B $
    \item $ A\cup C $
    \item $ C\cap \overline{B} $
    \item $ \overline{A\cup \overline{B}} $
    \item $ \overline{A}\cup B $
    \item $ (\overline{A}\cap\overline{B})\cap \overline{C} $
  \end{enumerate}
\end{multicols}
\item En un determinado experimento aleatorio el espacio muestral consta de sólo tres sucesos elementales siendo la probabilidad de los dos primeros son 0,2 y 0,5. ¿Cuál es la
probabilidad del tercero?
\end{enumerate}
\end{document}
