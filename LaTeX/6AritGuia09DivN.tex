\documentclass[letterpaper,11pt,twoside]{article}
\usepackage[utf8]{inputenc}
\usepackage{amsmath,amsfonts,amssymb,amsthm,latexsym}
\usepackage[spanish,es-noshorthands]{babel}
\usepackage[T1]{fontenc}
\usepackage{lmodern}
\usepackage{graphicx,hyperref}
\usepackage{tikz,pgf}
\usepackage{multicol}
\usepackage{fancyhdr}
\usepackage[height=9.5in,width=7in]{geometry}
\usepackage{fancyhdr}
\pagestyle{fancy}
\fancyhead[LE]{\includegraphics[height=12pt]{Images/logo-colegio.png} Aritmética $6^{\circ}$}
\fancyhead[RE]{}
\fancyhead[RO]{\textit{Germ\'an Avenda\~no Ram\'irez, Lic. U.D., M.Sc. U.N.}}
\fancyhead[LO]{}

\author{Germ\'an Avenda\~no Ram\'irez, Lic. U.D., M.Sc. U.N.}
\title{\begin{minipage}{.2\textwidth}
\includegraphics[height=1.75cm]{Images/logo-colegio.png}\end{minipage}
\begin{minipage}{.55\textwidth}
\begin{center}
Taller 09, Divisi\'{o}n en $\mathbb{N}$\\
Aritm\'{e}tica $6^{\circ}$
\end{center}
\end{minipage}\hfill
\begin{minipage}{.2\textwidth}
\includegraphics[height=1.75cm]{Images/logo-sed.png} 
\end{minipage}}
\date{}
\thispagestyle{plain}
\begin{document}
\maketitle
Nombre: \hrulefill Curso: \underline{\hspace*{44pt}} Fecha: \underline{\hspace*{2.5cm}}
\begin{multicols}{2}
 \section*{Aplico lo aprendido}
 \begin{itemize}
 \item En la tienda de doña Rosario se encuentran los
siguientes productos.
\begin{center}
\includegraphics[scale=.5]{Images/productos.png} 
\end{center}
Mario debe llevar: cinco kilos de arroz, tres docenas de
huevos, siete libras de tomate, cuatro libras de café, dos
paquetes de pasta, tres frascos de aceite, cuatro paquetes
de harina y ocho kilos de papa.

¿Mario podrá pagar sus compras con \$ 70\,000? ¿Por qué?
Mario decide pagar el valor total de sus compras en cuotas.
Si cada una es de \$ 17\,800, ¿cuántas cuotas debe pagar?
\item Maritza compró ocho paquetes de almojábanas, pero
cuando empieza a desempacarlas se da cuenta que en vez
de tener las 72 unidades que esperaba sólo tiene 64.

¿Cuántas almojábanas pensaba Maritza que debía recibir
por paquete? ¿Cuántas almojábanas por paquete recibió
realmente Maritza?
\item Cinco toros con masas iguales pesan entre todos 2\,840
kg. ¿Cuál es el peso de cada uno?
\item A las fiestas patronales de Los Pinos asistieron 1\,253 hombres, 1\,786 mujeres y 175 menores de edad. Si la
plaza de toros del pueblo cobra \$ 15\,700 por la entrada
y se espera que asistan todos los visitantes, ¿cuánto se
recaudaría por entradas a la corrida de toros?
\item Para empacar 1\,500 huevos se dispone de bandejas en
cada una de las cuales caben doce unidades. ¿Cuántas
bandejas se necesitan?
\item Pilar hizo una llamada telefónica de doce minutos de
duración. Si le cobraron \$ 1\,500, ¿cuál es el costo de
cada minuto?
 \end{itemize}
 
\end{multicols}


\end{document}
