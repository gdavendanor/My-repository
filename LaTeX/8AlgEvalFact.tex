\documentclass[letterpaper,fleqn]{article}
\usepackage[spanish,es-noshorthands]{babel}
\usepackage[utf8]{inputenc} 
\usepackage[papersize={6.5in,8.5in},left=1cm, right=1cm, top=1.5cm, bottom=1.7cm]{geometry}
\usepackage{mathexam}
\usepackage{amsmath}
\usepackage{graphicx}

\ExamClass{\includegraphics[height=16pt]{Images/logo-sed.png} Álgebra $8^{\circ}$}
\ExamName{Quiz-Factorización}
\ExamHead{\includegraphics[height=16pt]{Images/logo-colegio.png} IEDAB}
\newcommand{\LineaNombre}{%
\par
\vspace{\baselineskip}
Nombre:\hrulefill \; Curso: \underline{\hspace*{48pt}} \; Fecha: \underline{\hspace*{2.5cm}} \relax
\par}
\let\ds\displaystyle

\begin{document}
\ExamInstrBox{
Respuesta sin justificar mediante procedimiento no será tenida en cuenta en la calificación. Escriba sus respuestas en el espacio indicado. Tiene 45 minutos para contestar esta prueba.}
\LineaNombre
\begin{enumerate}
 \item Clasifique los siguientes números en primos o compuestos. Los números que resulten ser compuestos, factorícelos en sus factores primos:
 \begin{enumerate}
 \begin{multicols}{2}
 \item 95
 \item 97
 \item 135
 \item 149
 \end{multicols}
 \end{enumerate}
 \item Factorice completamente:
 \begin{enumerate}
 \item $6x+3y$
 \item $28y^{2}-4y$
 \item $12x^{3}-39x^{4}y^{3}$
 \item $15x^{2}y^{3}+20xy^{2}+35x^{3}y^{4}$
 \item $x(y+2)+3(y+2)$
 \end{enumerate}
 \item Factorice usando agrupación de términos:
 \begin{enumerate}
 \item $3ax-3bx-ay+by$
 \item $2ax+2x+ay+y$
 \end{enumerate}
 \end{enumerate}

\end{document}
