\documentclass[fleqn]{article}
\usepackage[spanish,es-noshorthands]{babel}
\usepackage[utf8]{inputenc} 
\usepackage[papersize={6.5in,8.5in},left=1cm, right=1cm, top=1.5cm, bottom=1.7cm]{geometry}
\usepackage{mathexam}
\usepackage{amsmath}
\usepackage{graphicx}

\ExamClass{\includegraphics[height=16pt]{Images/logo-sed.png} Matemáticas $6^{\circ}$}
\ExamName{Prueba Bimestral III}
\ExamHead{\includegraphics[height=16pt]{Images/logo-colegio.png} IEDAB}
\newcommand{\LineaNombre}{%
\par
\vspace{\baselineskip}
Nombre:\hrulefill \; Curso: \underline{\hspace*{48pt}} \; Fecha: \underline{\hspace*{2.5cm}} \relax
\par}
\let\ds\displaystyle

\begin{document}
\ExamInstrBox{
Respuesta sin justificar mediante procedimiento no será tenida en cuenta en la calificación. Escriba sus respuestas en el espacio indicado. Dispone de 70 minutos para contestar esta prueba.}
\LineaNombre
\begin{enumerate}
 \item Encuentre el conjunto de los primeros 10 múltiplos del número 14
 \noanswer
 \[\text{Primero 10 múltiplos de }14=\{\]
 \item Encuentre el conjunto de divisores del número 54
 \noanswer
 \[\text{Divisores de }54=\{\]
 \item Encuentre el m.c.m (mínimo común múltiplo) de los números 12 y 15
 \answer*{$m.c.m(12,15)=$}
 \item Encuentre el M.C.D. (Máximo Común Divisor) de los números 36 y 48
 \answer*{$M.C.D.(36,48)=$}
 \newpage
 \item Los siguientes datos corresponden al número de minutos que dedican los estudiantes de un curso a hacer tareas
 
 0, 15, 10, 20, 0, 20, 30, 40, 50, 0, 20, 30, 15, 15, 20, 15, 20, 15, 15, 12, 20, 0, 25, 30, 24, 20, 20, 30, 40, 50
 \begin{enumerate}
  \item Haga la tabla de frecuencias correspondiente
  \begin{center}
\begin{tabular}{c|c}\hline
Dato & Frecuencia\\
$x_{i}$ & $f_{i}$\\\hline
 & \\
 & \\
 & \\
 & \\
 & \\
 & \\
 & \\
 & \\
 & 
  \end{tabular}
  \end{center}
  \item Determine la moda\answer*[0pt]{Moda:}
  \item Determine la mediana \answer*[12pt]{Mediana:}
  \item Determine la media (promedio) \answer*{Media: $\overline{x}=$}
  \item Haga el diagrama de barras\noanswer
 \end{enumerate}
 \end{enumerate}
\end{document}
