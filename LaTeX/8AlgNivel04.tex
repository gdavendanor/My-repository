\documentclass[letterpaper,fleqn]{article}
\usepackage[spanish,es-noshorthands]{babel}
\usepackage[utf8]{inputenc} 
\usepackage[papersize={6.5in,8.5in},left=1cm, right=1cm, top=1.5cm, bottom=1.7cm]{geometry}
\usepackage{mathexam}
\usepackage{amsmath}
\usepackage{graphicx}

\ExamClass{\includegraphics[height=16pt]{Images/logo-sed.png} Álgebra $8^{\circ}$}
\ExamName{Nivelación 04 período}
\ExamHead{\includegraphics[height=16pt]{Images/logo-colegio.png} IEDAB}
\newcommand{\LineaNombre}{%
\par
\vspace{\baselineskip}
Nombre:\hrulefill \; Curso: \underline{\hspace*{48pt}} \; Fecha: \underline{\hspace*{2.5cm}} \relax
\par}
\let\ds\displaystyle

\begin{document}
\ExamInstrBox{
Respuesta sin justificar mediante procedimiento no será tenida en cuenta en la calificación. Escriba sus respuestas en el espacio indicado. Tiene 50 minutos para contestar esta prueba.}
\LineaNombre
\begin{enumerate}
 \item Decida si los siguientes números son primos o compuestos. Los que sean compuestos, factorícelos como el producto de sus factores primos.
 \begin{enumerate}
 \item 225\noanswer
 \item 109 \noanswer
 \end{enumerate}
 \item Factorice completamente usando factor común
 \begin{enumerate}
 \item $18a^{2}b+27ab^{2}$\noanswer
 \item $x(x+2)+5(x+2)$\noanswer
 \end{enumerate}
 \item Factorice usando factor común por agrupación de términos
 \begin{enumerate}
 \item $ax+4x+ay+4y$\noanswer
 \item $3ax-3bx-ay+by$\noanswer
 \end{enumerate}
 \item Factorice completamente usando el caso diferencia de cuadrados
 \[a^{2}-b^{2}=(a-b)(a+b)\]
\begin{enumerate}
\item $9x^{2}-25y^{2}$\noanswer
\item $25x^{2}y^{2}-36$\noanswer
\end{enumerate}
\item Encuentre el lado de un cuadrado cuya área sea el doble de su perímetro.
 \end{enumerate}

\end{document}
