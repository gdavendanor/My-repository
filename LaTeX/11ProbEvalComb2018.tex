\documentclass[10pt,addpoints]{exam}
\usepackage[utf8]{inputenc}
\usepackage[spanish,es-noshorthands]{babel}
\usepackage{hyperref}
\usepackage{amsmath}
\usepackage{amsfonts}
\usepackage{amssymb}
\usepackage{graphicx}
\usepackage{tikz}
\usepackage{multicol}
\usepackage[papersize={6.5in,8.5in},total={5.5in,7.25in},centering]{geometry}
%\printanswers
\begin{document}
\title{\begin{minipage}{.2\textwidth}
        \includegraphics[height=1.75cm]{Images/logo-colegio.png}
       \end{minipage}
\begin{minipage}{.55\textwidth}
 \begin{center}
Eval. Combinatoria \\Probabilidad $11^{\circ}$
\end{center}
\end{minipage}
\begin{minipage}{.2\textwidth}
\includegraphics[height=1.75cm]{Images/logo-sed.png} 
\end{minipage}
}
\author{Germ\'{a}n Avendaño Ram\'{i}rez~\thanks{Lic. Mat. U.D., M.Sc. U.N.}}
\date{}
\maketitle
\vspace*{-.35in}
\begin{center}
\fbox{\fbox{\parbox{5.5in}{\centering
Los procedimientos se deben hacer en una hoja anexa, así como las respuestas a las preguntas abiertas. Conteste las preguntas de selección múltiple en el cuadro respectivo de respuestas. }}}
\end{center}
\vspace{0.05in}
\makebox[\textwidth]{Nombres: \hrulefill, curso:\underline{\hspace{48pt}}, fecha:\underline{\hspace{3cm}}}
\begin{questions}
\question Calcule:
\begin{parts}
\begin{multicols}{3}
\part $6!$
\part $9P3$
\part $10C4$
\end{multicols}
\end{parts}
\vspace*{.5in}
\question Hay cuatro candidatas para ser reina y tres para ser rey. ¿Cuántos pares de rey-reina son posibles?\vspace*{.5in}
\question En un país, las placas de las motocicletas constan de 3 letras (sin contar la ñ) y dos dígitos. ¿Cuántas posibles placas habrán?\vspace*{.5in}
\question ¿Cuántos posibles arreglos distintos se pueden hacer con las letras de la palabra ``amargar''?\vspace*{.5in}
\question Los 12 miembros del club ``Los Lagartos'' están eligiendo al presidente, al vicepresidente y al secretario de entre sus doce miembros. ¿De cuántas maneras distintas puede hacerse esto?\vspace*{.5in}
\question El profesor Jirafales le da a su grupo 12 preguntas para estudiar; de ellas va a seleccionar 5 para el examen final. ¿De cuántas formas puede seleccionar las preguntas?\vspace*{.5in}
\question La comida del Gritsy Palace consiste en un plato fuerte, una guarnición con dos tipos de vegetales y un postre. Si hay cuatro platos fuertes, seis tipos de vegetales y seis postres para elegir, ¿cuántas posibles comidas hay?
%Pregunta

\begin{oneparchoices}
\choice 16
\choice 25
\choice 144
\CorrectChoice 360
\choice 720
\end{oneparchoices}
\question De cuántas formas distintas los jueces pueden elegir del 5o al 1er lugar de diez finalistas de Miss Estados Unidos?

\begin{oneparchoices}
\choice 50
\choice 120
\choice 252
\CorrectChoice 30\;240
\choice 3'628\;800
\end{oneparchoices}
\question Suponga que $r$ y $n$ son enteros positivos y además que $r< n$. ¿Cuál de los siguientes números no es igual a 1?

\begin{oneparchoices}
\choice $(n-n)!$
\CorrectChoice $nPn$
\choice $nCn$
\choice $\left(\begin{array}{c}
n\\
n
\end{array}
\right)$
\choice $\left(\begin{array}{c}
n\\
r
\end{array}\right)\div \left(\begin{array}{c}
n\\
n-r
\end{array}\right) $
\end{oneparchoices}
\question Una organización va a elegir mediante votación a tres nuevos miembros de su consejo de administración. A los miembros se les dan papeletas con los nombres de los cinco candidatos y se les pide que marquen los nombres de los candidatos que elijan (que podría ser ninguno, o incluso los cinco). Los tres candidatos con el mayor número de votos
son los elegidos. ¿De cuántas maneras distintas puede un miembro llenar su papeleta?

\begin{oneparchoices}
\choice 10
\choice 20
\CorrectChoice 32
\choice 125
\choice 243
\end{oneparchoices}
%\answerline
\end{questions}
\begin{center}
\section*{Cuadro de respuestas}
\begin{tabular}{cccc}
7 & 8 & 9 & 10 \\ 
\textcircled{a} & \textcircled{a} & \textcircled{a} & \textcircled{a}\\ 
\textcircled{b} & \textcircled{b} & \textcircled{b} & \textcircled{b} \\ 
\textcircled{c} & \textcircled{c} & \textcircled{c} & \textcircled{c}\\ 
\textcircled{d} & \textcircled{d} & \textcircled{d} & \textcircled{d}\\ 
\textcircled{e} & \textcircled{e} & \textcircled{e} & \textcircled{e}\\
\end{tabular}
\end{center}
%cuadro de puntajes
%\begin{center}
%\gradetable[h][pages]
%\end{center}
\end{document}