\documentclass[letterpaper,fleqn]{article}
\usepackage[spanish,es-noshorthands]{babel}
\usepackage[utf8]{inputenc} 
\usepackage[papersize={6.5in,8.5in},left=1cm, right=1cm, top=1.5cm, bottom=1.7cm]{geometry}
\usepackage{mathexam}
\usepackage{amsmath}
\usepackage{graphicx}
\usepackage{multicol}
\usepackage{tikz,pgf}

\ExamClass{\includegraphics[height=16pt]{Images/logo-sed.png} Aritmética $6^{\circ}$}
\ExamName{Fracciones}
\ExamHead{\includegraphics[height=16pt]{Images/logo-colegio.png} IEDAB}
\newcommand{\LineaNombre}{%
\par
\vspace{\baselineskip}
Nombre:\hrulefill \; Curso: \underline{\hspace*{48pt}} \; Fecha: \underline{\hspace*{2.5cm}} \relax
\par}
\let\ds\displaystyle

\begin{document}
\ExamInstrBox{
Respuesta sin justificar mediante procedimiento no será tenida en cuenta en la calificación. Escriba sus respuestas en el espacio indicado. Tiene 45 minutos para contestar esta prueba.}
\LineaNombre
\begin{enumerate}
 \item La señora de Rojas hizo una torta que dividió en 8 partes iguales, de las cuales ella comió una porción, su esposo dos porciones, su hijo mayor dos porciones y su hijo menor una porción. La fracción de la torta que comieron la señor de Rojas y su familia es: \noanswer
 \item Un atleta diariamente da 24 vueltas a una pista. Hoy, cuando corría, sufrió una lesión y solamente había hecho 18 vueltas. ¿Qué  fracción de lo que normalmente corre alcanzó a hacer? \noanswer
\item Juan y Pedro deben llevar cemento para hacer una obra. si Juan lleva $\frac{3}{4}$ de bulto y Pedro $\frac{3}{5}$ de bulto, ¿llevan ambos la misma cantidad?\noanswer
\item Valentina y Reinel comen torta. Si Valentina come $\frac{4}{8}$ de torta y Reinel $\frac{2}{4}$ de torta, ¿comen ambos la misma cantidad de torta?\noanswer
\item Julian quiere comprar $\frac{4}{6}$ de kilo de Jamón pero en el supermercado solo encuentra  paquetes de $\frac{1}{3}$ de kilo. ¿Cuántos paquetes debe comprar Julian?\noanswer
\item Simplique las siguientes fracciones
\begin{enumerate}
\begin{multicols}{2}
\item $\dfrac{4}{10}=$\noanswer
\item $\dfrac{6}{15}=$\noanswer
\end{multicols}
\end{enumerate}
\item Joseph el pastelero, necesita $\frac{4}{12}$ de kilo de levadura. Si en la cocina hay medidas de $\frac{1}{2}$, $\frac{1}{3}$, $\frac{1}{4}$ y $\frac{1}{6}$, ¿cu\'{a}l es la medida m\'{a}s grande que debe usar para que no le sobre ni le falte levadura?\noanswer
\item Complete con los signos $<$ (menor que), $>$ (mayor que) o $=$ (igual que) según corresponda:
\begin{enumerate}
\begin{multicols}{3}
\item $\frac{3}{4}$ \tikz \draw rectangle (.55,.55); $\frac{6}{8}$
\item $\frac{2}{3}$ \tikz \draw rectangle (.55,.55); $\frac{3}{2}$
\item $\frac{4}{5}$ \tikz \draw rectangle (.55,.55);$\frac{3}{4}$
\end{multicols}
\end{enumerate}
 \end{enumerate}

\end{document}
