\documentclass[letterpaper,fleqn]{article}
\usepackage[spanish,es-noshorthands]{babel}
\usepackage[utf8]{inputenc} 
\usepackage[left=1cm, right=1cm, top=1.5cm, bottom=1.7cm]{geometry}
\usepackage{mathexam}
\usepackage{amsmath}
\usepackage{graphicx}
\usepackage{tikz,pgf}
\usepackage{multicol}

\ExamClass{\includegraphics[height=16pt]{Images/logo-sed.png} Matemáticas $11^{\circ}$}
\ExamName{Prueba Bimestral III}
\ExamHead{\includegraphics[height=16pt]{Images/logo-colegio.png} IEDAB}
\newcommand{\LineaNombre}{%
\par
\vspace{\baselineskip}
Nombre:\hrulefill \; Curso: \underline{\hspace*{48pt}} \; Form: \underline{A} \; Fecha: \underline{\hspace*{2.5cm}} \relax
\par}
\let\ds\displaystyle

\begin{document}
\ExamInstrBox{
Respuesta sin justificar mediante procedimiento no será tenida en cuenta en la calificación. Escriba sus respuestas en el espacio indicado. Tiene 70 minutos para contestar esta prueba.}
\LineaNombre
\begin{enumerate}
 \item Encuentre los siguientes 5 términos de las progresiones y encuentre el término n-ésimo
 \begin{enumerate}
  \item -3,1,5,\ldots \addanswer*[0pt]{siguientes 5 términos:}\answer*[0pt]{$a_{n}=$}
  \item 18, 6, 2, $\dfrac{2}{3}$, \ldots \addanswer*[0pt]{5 términos siguientes:} \answer*[12pt]{$b_{n}=$}
 \end{enumerate}
\item Encuentre los 10 primeros términos de las progresiones cuyo término general se da:
\begin{enumerate}
 \item $c_{n}=3n+5$ \answer*{Primeros diez términos}
 \item $d_{n}=2(3)^{n-1}$ \answer*{Primeros diez términos}
\end{enumerate}
\item Grafique en el plano cartesiano las siguientes funciones. Puede hacer las dos en  el mismo plano. Se sugiere hacer tabla de valores para cada función para hacer su gráfica.
\begin{enumerate}
\begin{multicols}{2}
 \item $f(x)=-2x+1$ \noanswer
\item $g(x)=3-x^{2}$ \noanswer 
\end{multicols}
\end{enumerate}
\begin{center}
\usetikzlibrary{arrows}
\baselineskip=10pt
\hsize=6.3truein
\vsize=8.7truein
\definecolor{cqcqcq}{rgb}{0.75,0.75,0.75}
\tikzpicture[scale=.75,line cap=round,line join=round,>=triangle 45,x=1.0cm,y=1.0cm]
\draw [color=cqcqcq,dash pattern=on 2pt off 2pt, xstep=1.0cm,ystep=1.0cm] (-3.82,-6.64) grid (3.68,4.1);
\draw[->,color=black] (-3.82,0) -- (3.68,0);
\foreach \x in {-3,-2,-1,1,2,3}
\draw[shift={(\x,0)},color=black] (0pt,2pt) -- (0pt,-2pt) node[below] {$\x$};
\draw[->,color=black] (0,-6.64) -- (0,4.1);
\foreach \y in {-6,-5,-4,-3,-2,-1,1,2,3,4}
\draw[shift={(0,\y)},color=black] (2pt,0pt) -- (-2pt,0pt) node[left] {$\y$};
\draw[color=black] (0pt,-10pt) node[right] {$0$};
\clip(-3.82,-6.64) rectangle (3.68,4.1);
\endtikzpicture
\end{center}
\newpage
\item Se lanzan al aires tres monedas y se registra el número de caras observadas. Encuentre la probabilidad para cada uno de los posibles resultados:
\begin{enumerate}
 \item ``0 caras observadas'': \answer*[0pt]{$P(0C)=$}
 \item ``1 cara observada'': \answer*[0pt]{$P(1C)=$}
 \item ``2 caras observadas'': \answer*[0pt]{$P(2C)=$}
 \item ``3 caras observadas'': \answer*[0pt]{$P(3C)=$}
\end{enumerate}
\item Una computadora genera (de manera aleatoria) pares de enteros. El primer entero es entre 1 y 5, inclusive, y el segundo es entre 1 y 4, inclusive.
\begin{enumerate}
 \item Represente el espacio muestral $S$ como un diagrama de árbol \noanswer
 \item Haga una lista de sus resultados como pares ordenados, con $x$ como el primer entero y $y$ como el segundo entero.\noanswer
\end{enumerate}

\end{enumerate}

\end{document}
