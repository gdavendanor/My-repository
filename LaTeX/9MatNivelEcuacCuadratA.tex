\documentclass[fleqn]{article}
\usepackage[spanish,es-noshorthands]{babel}
\usepackage[utf8]{inputenc} 
\usepackage[papersize={6.5in,8.5in},left=1cm, right=1cm, top=1.5cm, bottom=1.7cm]{geometry}
\usepackage{mathexam}
\usepackage{amsmath}
\usepackage{graphicx}
\usepackage{printsudoku}

\ExamClass{\includegraphics[height=16pt]{Images/logo-sed.png} 
Àlgebra $9^{\circ}$}
\ExamName{Ecuación cuadrática}
\ExamHead{\includegraphics[height=16pt]{Images/logo-colegio.png} IEDAB}
\newcommand{\LineaNombre}{%
\par
\vspace{\baselineskip}
Nombre:\hrulefill \; Curso: \underline{\hspace*{48pt}} \; Fecha: \underline{\hspace*{2.5cm}} \relax
\par}
\let\ds\displaystyle

\begin{document}
\ExamInstrBox{
Respuesta sin justificar mediante procedimiento no será tenida en cuenta en la calificación. Escriba sus respuestas en el espacio indicado. Tiene 45 minutos para contestar esta prueba.}
\LineaNombre
\begin{enumerate}
 \item Verifique si los números $x=-3$ \; y \; $x=5$ son soluciones de la ecuación cuadrática \[2x^{2}-7x-15=0\] \noanswer

\item[II.] Resuelva las siguientes ecuaciones cuadráticas
 \item $3x^{2}-12=0$\noanswer
 \item $4x^{2}+12x=0$\noanswer
 \newpage
 \item $x^{2}-x-6=0$\noanswer
 \item $9x^{2}+13x+8=0$\noanswer
 \item[III.] Resuelva el siguiente problema
 \item La tercera parte del cuadrado de un número es 48. ¿Cuál es el número?\noanswer
 \end{enumerate}
 \begin{minipage}{.4\textwidth}
\paragraph*{Punto-extra}: Resuelva el siguiente sudoku 
 \end{minipage}
 \cluefont{\Large}
\cellsize{1.5\baselineskip}
\begin{minipage}{.55\textwidth}
\begin{minipage}{0.45\linewidth}\begin{center}
TG5 (gentle) \\
\sudoku{tg5.sud}
\end{center}\end{minipage}
\end{minipage}
\end{document}
