\documentclass[10pt,twoside]{article}
\usepackage[utf8]{inputenc}
\usepackage{amsmath}
\usepackage{amsfonts}
\usepackage{amssymb}
\usepackage[spanish,es-noshorthands]{babel}
\usepackage[T1]{fontenc}
\usepackage{lmodern}
\usepackage{graphicx,hyperref}
\usepackage{tikz,pgf}
\usepackage{multicol}
\usepackage{subfig}
\usepackage[papersize={6.5in,8.5in},width=5.5in,height=7in]{geometry}
\usepackage{fancyhdr}
\pagestyle{fancy}
\fancyhead[LE]{\includegraphics[height=12pt]{Images/logo-colegio.png} Cálculo $11^{\circ}$}
\fancyhead[RE]{}
\fancyhead[RO]{\textit{Germ\'an Avenda\~no Ram\'irez, Lic. U.D., M.Sc. U.N.}}
\fancyhead[LO]{}

\author{Germ\'an Avenda\~no Ram\'irez, Lic. U.D., M.Sc. U.N.}
\title{\begin{minipage}{.2\textwidth}
\includegraphics[height=1.75cm]{Images/logo-colegio.png}\end{minipage}
\begin{minipage}{.55\textwidth}
\begin{center}
Animaplano 1, $10^{\circ}$  \\
Cálculo $11^{\circ}$
\end{center}
\end{minipage}\hfill
\begin{minipage}{.2\textwidth}
\includegraphics[height=1.75cm]{Images/logo-sed.png} 
\end{minipage}}
\date{}
\begin{document}
\maketitle
Nombre: \hrulefill Curso: \underline{\hspace*{44pt}} Fecha: \underline{\hspace*{2.5cm}}
\section*{Cuestionario}
Responda haciendo los procedimientos al frente de cada pregunta
\begin{enumerate}
 \item $8^{2}=4^{3}=2^{6}=$
 \item Si $2^{x}=64$, entonces $x+61=$?
 \item $10^{2}-6^{2}-4^{2}+3^{2}=$
 \item La media de los datos \{63, 64, 65, 66, 67\} es:
 \item $22xy-3xy+35xy=\underline{\hspace{24pt}}xy$
 \item Calcule $\log_{2}256^{8}=$
 \item El 25\% de 292 es:
 \item $9^{4}/9^{2}-10^{0}$ menos raíz cuadrada de 81 =
 \item (raíz cuadrada de 64 $\times$ raíz cuadrada de 64) menos 4
 \item la pendiente de la recta $y=45+81x$ es?
 \item El vigésimo término de la progresión aritmética 26, 29, 32, 35, \dots es:
 \item $10^{2}-10^{2}/4=$
 \item Si 3/2=n/14, entonces $(n\times4)+9=$
 \item Si $sen(n-17)=\frac{\sqrt{3}}{2}$, entonces $n=$?
 \item Si 15 es el primer término de una progresión aritmética cuya diferencia entre dos términos consecutivos es 8, entonces el 10 término es:
\end{enumerate}

\end{document}
