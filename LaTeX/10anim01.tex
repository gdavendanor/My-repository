\documentclass[11pt,twoside,letterpaper]{article}
\usepackage[utf8]{inputenc}
\usepackage{amsmath}
\usepackage{amsfonts}
\usepackage{amssymb}
\usepackage[spanish,es-noshorthands]{babel}
\usepackage[T1]{fontenc}
\usepackage{lmodern}
\usepackage{graphicx,hyperref}
\usepackage{tikz,pgf}
\usepackage{multicol}
\usepackage{subfig}
\usepackage[width=7.5in,height=9.5in]{geometry}
\usepackage{fancyhdr}
\pagestyle{fancy}
\fancyhead[LE]{\url{www.autistici.org/mathgerman}}
\fancyhead[RE]{}
\fancyhead[RO]{matematicas.german@gmail.com}
\fancyhead[LO]{}

\author{Germ\'an Avenda\~no Ram\'irez~\thanks{Lic. Mat. U.D., M.Sc. U.N.}}
\title{\begin{minipage}{.2\textwidth}
\includegraphics[height=1.75cm]{Images/logo-colegio.png}\end{minipage}
\begin{minipage}{.55\textwidth}
\begin{center}
Animaplano 1, $10^{\circ}$  \\
Cálculo $11^{\circ}$
\end{center}
\end{minipage}\hfill
\begin{minipage}{.2\textwidth}
\includegraphics[height=1.75cm]{Images/logo-sed.png} 
\end{minipage}}
\date{}
\begin{document}
\maketitle
Nombre: \hrulefill Curso: \underline{\hspace*{44pt}} Fecha: \underline{\hspace*{2.5cm}}
\section*{Cuestionario}
\textit{Responda haciendo los procedimientos al frente de cada pregunta. Al finalizar entregue al estudiante encargado (1101 Sergio Cruz, 1102 Valentina Ávila) para que él o ella a su vez entregue a la coordinadora Yaneth Reyes.}
\begin{enumerate}
 \item $8^{2}=4^{3}=2^{6}=$
 \item Si $2^{x}=64$, entonces $x+61=$?
 \item $10^{2}-6^{2}-4^{2}+3^{2}=$
 \item La media de los datos \{63, 64, 65, 66, 67\} es:
 \item $22xy-3xy+35xy=\underline{\hspace{24pt}}xy$
 \item Calcule $\log_{2}256^{8}=$
 \item El 25\% de 292 es:
 \item $9^{4}/9^{2}-10^{0}$ menos raíz cuadrada de 81 =
 \item (raíz cuadrada de 64 $\times$ raíz cuadrada de 64) menos 4
 \item la pendiente de la recta $y=45+81x$ es?
 \item El vigésimo término de la progresión aritmética 26, 29, 32, 35, \dots es:
 \item $10^{2}-10^{2}/4=$
 \item Si 3/2=n/14, entonces $(n\times4)+9=$
 \item Si $sen(n-17)=\frac{\sqrt{3}}{2}$, entonces $n=$?
 \item Si 15 es el primer término de una progresión aritmética cuya diferencia entre dos términos consecutivos es 8, entonces el 10 término es:
 \item La imagen de 4 en la función $f(x)=x^{3}+2x+6$ es:
 \item El tercer término de una progresión geométrica cuya razón es 2 y cuyo primer término es 17 es:
 \item $5!-6^{2}-5$
 \item Si $2x=10$, entonces $x^{2}+3x+18=$?
 \item $(14-18+2)^{2})\cdot(4+12-4)$
 \item $\log_{8}8+\log_{2}64^{6}=$
 \item $4^{2x}=2^{40}$, \textbf{entonces} $x+25=$?
 \item $3^{3x}=27$, entonces $x+44=$?
 \item Si $4x=24$, entonces $x^{2}=$?
 \item Halle $4!+2=$?
 \item El doceavo término de una progresión aritmética cuyos primeros términos son -14, --11, -8, -5, \ldots es:
 \item $(\sqrt[3]{64})^{2}=$
 \item El cuarto término de una progresión geométrica cuyo primer término es $\frac{1}{3}$ y cuya razón es 3.
 \item El tercer número primo
 \item Número cuyas cifras suman 5 y cuyo producto es 4 y es menor que 20
 \item Si el perímetro del rectángulo \tikz \draw (0,0)--node[below]{$x$}(1.5,0)--node[right]{5}(1.5,.5)--(0,.5)--cycle; mide 40, entonces $x+10=$?
 \item El quíntuple de 7
 \item $7\times8-3^{1}=$
 \item Número cuya dos cifras suman 9 y cuyo producto es 20 y está entre 50 y 99.
\end{enumerate}
\section*{Plano}
\begin{center}
\begin{tikzpicture}
 \fill (0,0) node[above]{0} circle (0.2ex);
 \fill (1,0) node[above]{1} circle (0.2ex);
 \fill (2,0) node[above]{2} circle (0.2ex);
 \fill (3,0) node[above]{3} circle (0.2ex);
 \fill (4,0) node[above]{4} circle (0.2ex);
 \fill (5,0) node[above]{5} circle (0.2ex);
 \fill (6,0) node[above]{6} circle (0.2ex);
 \fill (7,0) node[above]{7} circle (0.2ex);
 \fill (8,0) node[above]{8} circle (0.2ex);
 \fill (9,0) node[above]{9} circle (0.2ex);
 \fill (0,-1) node[left]{10} circle (0.2ex);
 \fill (1,-1) circle (0.2ex);
 \fill (2,-1) circle (0.2ex);
 \fill (3,-1) circle (0.2ex);
 \fill (4,-1) circle (0.2ex);
 \fill (5,-1) circle (0.2ex);
 \fill (6,-1) circle (0.2ex);
 \fill (7,-1) circle (0.2ex);
 \fill (8,-1) circle (0.2ex);
 \fill (9,-1) circle (0.2ex);
 \fill (0,-2) node[left]{20} circle (0.2ex);
 \fill (1,-2) circle (0.2ex);
 \fill (2,-2) circle (0.2ex);
 \fill (3,-2) circle (0.2ex);
 \fill (4,-2) circle (0.2ex);
 \fill (5,-2) circle (0.2ex);
 \fill (6,-2) circle (0.2ex);
 \fill (7,-2) circle (0.2ex);
 \fill (8,-2) circle (0.2ex);
 \fill (9,-2) circle (0.2ex);
 \fill (0,-3) node[left]{30} circle (0.2ex);
 \fill (1,-3) circle (0.2ex);
 \fill (2,-3) circle (0.2ex);
 \fill (3,-3) circle (0.2ex);
 \fill (4,-3) circle (0.2ex);
 \fill (5,-3) circle (0.2ex);
 \fill (6,-3) circle (0.2ex);
 \fill (7,-3) circle (0.2ex);
 \fill (8,-3) circle (0.2ex);
 \fill (9,-3) circle (0.2ex);
 \fill (0,-4) node[left]{40} circle (0.2ex);
 \fill (1,-4) circle (0.2ex);
 \fill (2,-4) circle (0.2ex);
 \fill (3,-4) circle (0.2ex);
 \fill (4,-4) circle (0.2ex);
 \fill (5,-4) circle (0.2ex);
 \fill (6,-4) circle (0.2ex);
 \fill (7,-4) circle (0.2ex);
 \fill (8,-4) circle (0.2ex);
 \fill (9,-4) node[right]{49} circle (0.2ex);
 \fill (0,-5) node[left]{50} circle (0.2ex);
 \fill (1,-5) circle (0.2ex);
 \fill (2,-5) circle (0.2ex);
 \fill (3,-5) circle (0.2ex);
 \fill (4,-5) circle (0.2ex);
 \fill (5,-5) circle (0.2ex);
 \fill (6,-5) circle (0.2ex);
 \fill (7,-5) circle (0.2ex);
 \fill (8,-5) circle (0.2ex);
 \fill (9,-5) circle (0.2ex);
 \fill (0,-6) node[left]{60} circle (0.2ex);
 \fill (1,-6) circle (0.2ex);
 \fill (2,-6) circle (0.2ex);
 \fill (3,-6) circle (0.2ex);
 \fill (4,-6) circle (0.2ex);
 \fill (5,-6) circle (0.2ex);
 \fill (6,-6) circle (0.2ex);
 \fill (7,-6) circle (0.2ex);
 \fill (8,-6) circle (0.2ex);
 \fill (9,-6) circle (0.2ex);
 \fill (0,-7) node[left]{70} circle (0.2ex);
 \fill (1,-7) circle (0.2ex);
 \fill (2,-7) circle (0.2ex);
 \fill (3,-7) circle (0.2ex);
 \fill (4,-7) circle (0.2ex);
 \fill (5,-7) circle (0.2ex);
 \fill (6,-7) circle (0.2ex);
 \fill (7,-7) circle (0.2ex);
 \fill (8,-7) circle (0.2ex);
 \fill (9,-7) circle (0.2ex);
 \fill (0,-8) node[left]{80} circle (0.2ex);
 \fill (1,-8) circle (0.2ex);
 \fill (2,-8) circle (0.2ex);
 \fill (3,-8) circle (0.2ex);
 \fill (4,-8) circle (0.2ex);
 \fill (5,-8) circle (0.2ex);
 \fill (6,-8) circle (0.2ex);
 \fill (7,-8) circle (0.2ex);
 \fill (8,-8) circle (0.2ex);
 \fill (9,-8) circle (0.2ex);
 \fill (0,-9) node[left]{90} circle (0.2ex);
 \fill (1,-9) circle (0.2ex);
 \fill (2,-9) circle (0.2ex);
 \fill (3,-9) circle (0.2ex);
 \fill (4,-9) circle (0.2ex);
 \fill (5,-9) circle (0.2ex);
 \fill (6,-9) circle (0.2ex);
 \fill (7,-9) circle (0.2ex);
 \fill (8,-9) circle (0.2ex);
 \fill (9,-9) node[right]{99} circle (0.2ex);
 \draw (4,-6)--(7,-6)--(7,-5)--(5,-6)--(4,-5)--(4,-6)--(3,-7)--(3,-7)--(1,-7)--(0,-6)--(1,-8)--(3,-8)--(5,-7)--(3,-9)--(7,-7)--(7,-8)--(8,-7)--(8,-6)--(9,-7)--(8,-5)--(8,-4)--(7,-3)--(5,-3)--(5,-4)--(6,-3)--(6,-2)--(9,-1)--(6,-1)--(9,0)--(5,0)--(4,-1)--(5,-2)--(5,-3)--(3,-5)--(4,-5);
\end{tikzpicture}
\end{center}
\end{document}
