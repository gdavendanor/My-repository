\documentclass[letterpaper,fleqn]{article}
\usepackage[spanish,es-noshorthands]{babel}
\usepackage[utf8]{inputenc} 
\usepackage[papersize={5.5in,8.5in},left=1cm, right=1cm, top=1.5cm, bottom=1.7cm]{geometry}
\usepackage{mathexam}
\usepackage{amsmath}
\usepackage{graphicx}

\ExamClass{\includegraphics[height=16pt]{Images/logo-sed.png} Aritmética $6^{\circ}$}
\ExamName{Sustentación Recomendaciones 1}
\ExamHead{\includegraphics[height=16pt]{Images/logo-colegio.png} IEDAB}
\newcommand{\LineaNombre}{%
\par
\vspace{\baselineskip}
Nombre:\hrulefill \; Curso: \underline{\hspace*{48pt}} \; Fecha: \underline{\hspace*{2.5cm}} \relax
\par}
\let\ds\displaystyle

\begin{document}
\ExamInstrBox{
Respuesta sin justificar mediante procedimiento no será tenida en cuenta en la calificación. Escriba sus respuestas en el espacio indicado. Tiene 45 minutos para contestar esta prueba.}
\LineaNombre
\begin{enumerate}
\item Efectúe las siguientes operaciones
\begin{enumerate}
\item $24+3\cdot (2+3)=$\noanswer
\item $4096\div 32=$ \noanswer
\item $15+2\cdot 13+18\div 3-3\cdot 4=$ \noanswer
\item $4000-2909=$\noanswer
\item $6+3(5+4\times 2)=$\noanswer
\item $[12\div (3+1)]\times 5-2=$\noanswer
\end{enumerate}
 \item ¿Cuál es el valor del dígito en cada caso?
 \begin{enumerate}
 \item 3 en el número $2'703,480$ \noanswer
 \item 5 en el número $3'658,749$ \noanswer
 \end{enumerate}
 \newpage
 \item Un número tiene un 1 en las centenas y decenas, un 3 en las unidades de mil, un 2 en las unidades de millón, un 4 en las unidades, un 6 en las centenas de mil y un 8 en las decenas de mil. ¿Cuál es el número?\noanswer
 \item ¿Cuál es la diferencia entre \$4 millones y \$250,000? \noanswer
 \item Un dólar americano valía ayer aproximadamente \$2,945 pesos colombianos. ¿Cuántos dólares americanos se necesitan para cambiar \$60,000 pesos colombianos?\noanswer
 \item Halle el cociente y residuo al hacer la siguiente división
 \[1273\div 26=\]\noanswer
 \end{enumerate}

\end{document}
