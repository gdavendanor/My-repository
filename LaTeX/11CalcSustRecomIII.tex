\documentclass[letterpaper,fleqn]{article}
\usepackage[spanish,es-noshorthands]{babel}
\usepackage[utf8]{inputenc} 
\usepackage[papersize={6.5in,8.5in},left=1cm, right=1cm, top=1.5cm, bottom=1.7cm]{geometry}
\usepackage{mathexam}
\usepackage{amsmath}
\usepackage{graphicx}

\ExamClass{\includegraphics[height=16pt]{Images/logo-sed.png} Cálculo $11^{\circ}$}
\ExamName{Sustentación Recomendaciones III}
\ExamHead{\includegraphics[height=16pt]{Images/logo-colegio.png} IEDAB}
\newcommand{\LineaNombre}{%
\par
\vspace{\baselineskip}
Nombre:\hrulefill \; Curso: \underline{\hspace*{48pt}} \; Fecha: \underline{\hspace*{2.5cm}} \relax
\par}
\let\ds\displaystyle

\begin{document}
\ExamInstrBox{
Respuesta sin justificar mediante procedimiento no será tenida en cuenta en la calificación. Escriba sus respuestas en el espacio indicado.}
\LineaNombre
\begin{enumerate}
\item Un ciclista sale de excursión a un lugar que dista 20 km. de su casa. A los quince minutos de salida, cuando se encuentra a seis km hace una parada de 10 minutos. Reanuda la marcha y llega a su destino una hora después de haber salido.
\begin{enumerate}
\item Representa la gráfica tiempo-distancia a su casa.

\usetikzlibrary{arrows}
\baselineskip=10pt
\hsize=6.3truein
\vsize=8.7truein
\definecolor{cqcqcq}{rgb}{0.75,0.75,0.75}
\tikzpicture[line cap=round,line join=round,>=triangle 45,x=1.0cm,y=1.0cm]
\draw [color=cqcqcq,dash pattern=on 5pt off 5pt, xstep=5.0cm,ystep=5.0cm] (-2,-0.76) grid (61,21);
\draw[->,color=black] (-2,0) -- (61,0);
\foreach \x in {,5,10,15,20,25,30,35,40,45,50,55,60}
\draw[shift={(\x,0)},color=black] (0pt,2pt) -- (0pt,-2pt) node[below] {$\x$};
\draw[->,color=black] (0,-0.76) -- (0,21);
\foreach \y in {,2,4,6,8,10,12,14,16,18,20}
\draw[shift={(0,\y)},color=black] (2pt,0pt) -- (-2pt,0pt) node[left] {$\y$};
\draw[color=black] (0pt,-10pt) node[right] {$0$};
\clip(-2,-0.76) rectangle (61,21);
\endtikzpicture
\item ¿Lleva la misma velocidad antes y después de la parada? Suponemos que en cada etapa la velocidad es constante?\noanswer
\item Busca la expresión analítica de esta función.\noanswer
\end{enumerate}
\item En un aparcamiento nos cobran por la primera hora 200 ptas y cada una de las horas siguientes a 150 ptas. Fíjate en que es una función escalonada:
\begin{enumerate}
\item Haz una tabla de valores para las 6 primeras horas.\noanswer
\item Represéntela gráficamente\noanswer
\item ¿En qué puntos es discontinua la función?\noanswer
\end{enumerate}
\item Una sucesión aritmética inicia con 2, 5, 8, 11, 14, \ldots
\begin{enumerate}
\item Encuentre la diferencia común $d$ para esta sucesión.\noanswer
\item Determine una fórmula para el n-ésimo término $a_{n}$ de la sucesión.\noanswer
\item Halle el trigésimoquinto término de la sucesión.\noanswer
\end{enumerate}
\item Una sucesión geométrica inicia con 12, 3, 3/4, 3/16, 3/64, \ldots
\begin{enumerate}
\item Determine la razón común $r$ de esta sucesión \noanswer
\item Encuentre una fórmula para el n-ésimo término $a_{n}$ de la sucesión.\noanswer
\item Calcule el décimo término de la sucesión\noanswer
\end{enumerate}
 \item Un cachorro pesa 0.85 lb al nacer, y cada semana gana 24\% de peso. Sea $a_{n}$ su peso en libras al final de la n-ésima semana de vida.
\begin{enumerate}
\item Encuentre una fórmula para $a_{n}$\noanswer
\item ¿Cuánto pesa el cachorro cuando tiene seis semanas de vida?\noanswer
\item ¿Es la sucesión $a_{1},a_{2},a_{3},\ldots$ aritmética, o geométrica o de ninguno de los dos tipos?\noanswer
\end{enumerate}
 \end{enumerate}

\end{document}
