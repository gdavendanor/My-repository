\documentclass[letterpaper,fleqn]{article}
\usepackage[spanish,es-noshorthands]{babel}
\usepackage[utf8]{inputenc} 
\usepackage[papersize={6.5in,8.5in},left=1cm, right=1cm, top=1.5cm, bottom=1.7cm]{geometry}
\usepackage{mathexam}
\usepackage{amsmath}
\usepackage{graphicx}

\ExamClass{\includegraphics[height=16pt]{Images/logo-sed.png} Cálculo $11^{\circ}$}
\ExamName{Sustentación Recomendaciones III}
\ExamHead{\includegraphics[height=16pt]{Images/logo-colegio.png} IEDAB}
\newcommand{\LineaNombre}{%
\par
\vspace{\baselineskip}
Nombre:\hrulefill \; Curso: \underline{\hspace*{48pt}} \; Fecha: \underline{\hspace*{2.5cm}} \relax
\par}
\let\ds\displaystyle

\begin{document}
\ExamInstrBox{
Respuesta sin justificar mediante procedimiento no será tenida en cuenta en la calificación. Escriba sus respuestas en el espacio indicado.}
\LineaNombre
\begin{enumerate}
 \item Un cachorro pesa 0.85 lb al nacer, y cada semana gana 24\% de peso. Sea $a_{n}$ su peso en libras al final de la n-ésima semana de vida.
\begin{enumerate}
\item Encuentre una fórmula para $a_{n}$\noanswer
\item ¿Cuánto pesa el cachorro cuando tiene seis semanas de vida?\noanswer
\item ¿Es la sucesión $a_{1},a_{2},a_{3},\ldots$ aritmética, o geométrica o de ninguno de los dos tipos?\noanswer
\end{enumerate}
 \end{enumerate}

\end{document}
