\documentclass[letterpaper,11pt,twoside]{article}
\usepackage[utf8]{inputenc}
\usepackage{amsmath,amsfonts,amssymb,amsthm,latexsym}
\usepackage[spanish,es-noshorthands]{babel}
\usepackage[T1]{fontenc}
\usepackage{lmodern}
\usepackage{graphicx,hyperref}
\usepackage{tikz,pgf}
\usepackage{multicol}
\usepackage{fancyhdr}
\usepackage[height=9.5in,width=7in]{geometry}
\usepackage{fancyhdr}
\pagestyle{fancy}
\fancyhead[LE]{\includegraphics[height=12pt]{Images/logo-colegio.png} Geometría $6^{\circ}$}
\fancyhead[RE]{}
\fancyhead[RO]{\textit{Germ\'an Avenda\~no Ram\'irez, Lic. U.D., M.Sc. U.N.}}
\fancyhead[LO]{}

\author{Germ\'an Avenda\~no Ram\'irez, Lic. U.D., M.Sc. U.N.}
\title{\begin{minipage}{.2\textwidth}
\includegraphics[height=1.75cm]{Images/logo-colegio.png}\end{minipage}
\begin{minipage}{.55\textwidth}
\begin{center}
Taller 06, La medida es cosa seria\\
Geometría $6^{\circ}$
\end{center}
\end{minipage}\hfill
\begin{minipage}{.2\textwidth}
\includegraphics[height=1.75cm]{Images/logo-sed.png} 
\end{minipage}}
\date{}
\thispagestyle{plain}
\begin{document}
\maketitle
Nombre: \hrulefill Curso: \underline{\hspace*{44pt}} Fecha: \underline{\hspace*{2.5cm}}
\begin{multicols}{2}
 \section*{Explora tus conocimientos}
 Para hacer unas reformas de carpintería en su casa, Rosario compró los siguientes materiales:
\begin{itemize}
\item 10 listones de 2 m
\item 21 listones de 75 cm
\item 50 listones de 15 Dm
\item 100 pequeños listones de 1 dm
\item 1 rollo de alambre de 10 Dm
\end{itemize}
Observe que no es lo mismo dm que Dm, en el primer caso se representan decímetros y en el segundo Decámetros.
\begin{itemize}
\item[a.] ¿Cuántos centímetros de listón compró en total?
\item[b.] Si devolvió la mitad de los listones pequeños, y la tercera parte de los grandes, ¿cuántos metros utilizó en las reformas?
\item[c.] ¿Cuántos metros de alambre empleó?
\end{itemize}
\section*{¿Qué est\'{a} cerca y qu\'{e} est\'{a} lejos}
Intuitivamente conocemos lo que es longitud o largo. En la práctica, lo que realmente medimos es la distancia o separación entre dos puntos, y dependiendo de la unidad de medida elegida, podemos decidir si uno de los puntos está cerca o lejos del otro.

En esta guía estudiarás las unidades pactadas universalmente para medir, no solamente longitudes; también las que se usan para medir superficies y el tiempo.
\section*{Lo que s\'{e}}
En grupos de 3 personas desarrollar en el cuaderno de cada uno las siguientes actividades:
\begin{itemize}
\item Fijen dos puntos diferentes y alejados uno del otro, en un
espacio abierto del colegio, coloquen un objeto en cada punto
elegido. Luego, cada uno de los integrantes del grupo debe
contar la cantidad pasos que separan esos dos puntos.
\item Anoten los resultados en sus cuadernos, en una tabla como la siguiente.
\end{itemize}
\begin{center}
\begin{tabular}{|l|c|c|c|}
\hline 
Estudiante &  &  &  \\ 
\hline 
Número de pasos &  &  &  \\ 
\hline 
\end{tabular} 
\end{center}
\end{multicols}


\end{document}
