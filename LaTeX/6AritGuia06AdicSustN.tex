\documentclass[10pt,twoside]{article}
\usepackage[utf8]{inputenc}
\usepackage{amsmath}
\usepackage{amsfonts}
\usepackage{amssymb}
\usepackage[spanish,es-noshorthands]{babel}
\usepackage[T1]{fontenc}
\usepackage{lmodern}
\usepackage{graphicx,hyperref}
\usepackage{tikz,pgf}
\usepackage{multicol}
\usepackage{subfig}
\usepackage[papersize={6.5in,8.5in},width=5.5in,height=7in]{geometry}
\usepackage{fancyhdr}
\pagestyle{fancy}
\fancyhead[LE]{\includegraphics[height=12pt]{Images/logo-colegio.png} Aritmética $6^{\circ}$}
\fancyhead[RE]{}
\fancyhead[RO]{\textit{Germ\'an Avenda\~no Ram\'irez, Lic. U.D., M.Sc. U.N.}}
\fancyhead[LO]{}

\author{Germ\'an Avenda\~no Ram\'irez, Lic. U.D., M.Sc. U.N.}
\title{\begin{minipage}{.2\textwidth}
\includegraphics[height=1.75cm]{Images/logo-colegio.png}\end{minipage}
\begin{minipage}{.55\textwidth}
\begin{center}
Taller 06\\Adición y sustracción en $\mathbb{N}$  \\
Aritmética $6^{\circ}$
\end{center}
\end{minipage}\hfill
\begin{minipage}{.2\textwidth}
\includegraphics[height=1.75cm]{Images/logo-sed.png} 
\end{minipage}}
\date{}
\begin{document}
\maketitle
Nombre: \hrulefill Curso: \underline{\hspace*{44pt}} Fecha: \underline{\hspace*{2.5cm}}\\
\section*{Continuaci\'on gu\'ia 5}
\subsection*{Actividad 1}
En tu cuaderno:
\begin{enumerate}
\item ¿Para qué valor de $x$ se cumple que:
\begin{enumerate}
\begin{multicols}{2}
\item $x+12=17$?
\item $8+x=20$?
\item $7+x=7$?
\item $x+11=11$?
\item $(5+x)+3=10$?
\item $13+17=x$?
\end{multicols}
\end{enumerate}
\item Un agricultor recogió la cosecha de papa en una semana así: el lunes 23 bultos, el martes 36 bultos, el miércoles 17 bultos, el jueves 19 bultos, el viernes 18 bultos y el sábado 21 bultos. ¿Cuántos bultos de papa recogió en total?

\begin{minipage}{0.45\textwidth}
\item Completa el siguiente cuadro en tu cuaderno con los números naturales
correspondientes.
\end{minipage}\hfill
\begin{minipage}{.45\textwidth}
\begin{tabular}{|c|c|c|c|}
\hline 
\multicolumn{3}{|c|}{SUMANDOS} & TOTAL \\ 
\hline 
8 & 2 & 0 &  \\ 
\hline 
9 &  & 1 & 20 \\ 
\hline 
0 &  & 5 & 12 \\ 
\hline 
 & 7 & 6 & 18 \\ 
\hline 
\end{tabular} 
\end{minipage}

\begin{minipage}{.45\textwidth}
\item Completa el siguiente cuadro en tu cuaderno de tal forma que en la diagonal
aparezcan las adiciones correspondientes.
\end{minipage}\hfill
\begin{minipage}{.45\textwidth}
\begin{tabular}{ccccc}
 & $a$ & $b$ & $c$ & $a+b+c$ \\ 
\cline{2-5} 
a & \vline \hfill 5 \hfill \vline & 8 \hfill \vline & 3 \hfill \vline & \hfill \vline   \\ \cline{2-5}
b & \vline \hfill 8 \hfill \vline & 16 \hfill \vline & \hfill \vline & \\ \cline{2-4}
c & \vline \hfill 3 \hfill \vline & \hfill \vline & 6 \hfill \vline & \\ \cline{2-4}
$a+b+c$ &  \vline \hfill \hspace*{10pt} \hfill \vline & & \\ \cline{2-2}
\end{tabular} 
\end{minipage}
\end{enumerate}
\section*{Sustracci\'on de n\'umeros naturales}
\subsection*{Actividad 2}
Supón que en la finca de tu vecino se recogieron ayer 9 bultos de naranja y se
llevaron a la ciudad 7 de ellos para venderlos. ¿Cuántos bultos de naranja le
quedaron al vecino?

Responde:
\begin{itemize}
\item ¿Cuántos bultos de naranja tenía inicialmente?
\item ¿Cuántos bultos de naranja vendió?
\item ¿Cuántos bultos le quedan en la finca?
\item Si sumas el número de bultos que vendió con el número de bultos que le
    quedan en la finca, ¿cuántos bultos obtiene en total?
\item ¿Cuanto le falta a 7 para ser igual a 9?
\item ¿Cuánto le falta a 2 para ser igual a 7?

Lo anterior se puede expresar así: 
\begin{align*}
7 + 2 &= 9\\
2+7&=9\\
\mbox{Si } 2+7=9, \mbox{ entonces }\qquad 9-2&=7\\
9-7&=2 
\end{align*}
\item ¿Qué clase de número es el 7?
\end{itemize}
Analiza la siguiente conclusión:

La operación inversa de la adición de números naturales es la SUSTRACCIÓN,
luego si $a + b = c$, entonces $c - a = b$. Al número natural $c$ se llama MINUENDO,
al natural $a$ SUSTRAENDO y al natural $b$ DIFERENCIA.

En el caso anterior:
\[
\begin{array}{ccccc}
9 & - & 2 & = & 7 \\
Minuendo &  & Sustraendo &  & Diferencia
\end{array}
\] 
\end{document}
