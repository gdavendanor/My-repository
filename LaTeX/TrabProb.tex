\documentclass[10pt,letterpaper]{article}
\usepackage[utf8]{inputenc}
\usepackage[spanish]{babel}
\usepackage{amsmath}
\usepackage{amsfonts}
\usepackage{amssymb}
\usepackage{lmodern}
\author{Germán Avendaño Ramírez}
\title{Ejercicios}
\begin{document}
\maketitle
\begin{itemize}
\item[8.] La función de probabilidad asociada para un avi\'{o}n de 4 motores es:
\[P(x)=\displaystyle{4 \choose x}(0.4)^{x}(0.6)^{4-x}\]
La probabilidad de que un vuelo sea seguro es la probabilidad de que no falle más de 1 motor de 4.
$x=0$, 1, 
\end{itemize}
\begin{align*}
P(0)+P(1)&=\displaystyle{4 \choose 0}(0.4)^{0}(0.6)^{4}+\displaystyle{4 \choose 1}(0.4)^{1}(0.6)^{3}\\
&=1(0.1296)+4(0.4)(0.216)\\
&=0.1296+0.3456\\
&=0.4752
\end{align*}
Ahora bien en el caso de un avión de dos motores, su función de probabilidad asociada es:
\[P(x)=\displaystyle{2\choose x}(0.4)^{x}(0.6)^{2-x}\]
Para que un vuelo sea exitoso, deben funcionar los dos motores. La probabilidad de que un vuelo sea exitoso se da para $x=0$ y se tiene
\begin{align*}
P(0)&=\displaystyle{2\choose 0}(0.4)^{0}(0.6)^{2}
&=0.36
\end{align*}
Por tanto el avión de 4 motores tiene una probabilidad más alta de tener vuelo exitoso que el avión de 2 motores.
\end{document}