\documentclass[letterpaper,fleqn]{article}
\usepackage[spanish,es-noshorthands]{babel}
\usepackage[utf8]{inputenc} 
\usepackage[left=1cm, right=1cm, top=1.5cm, bottom=1.7cm]{geometry}
\usepackage{mathexam}
\usepackage{amsmath}
\usepackage{graphicx}
\usepackage{pgf,tikz}

\ExamClass{\includegraphics[height=16pt]{Images/logo-sed.png} Cálculo $11^{\circ}$}
\ExamName{Sustentación recomendaciones II}
\ExamHead{\includegraphics[height=16pt]{Images/logo-colegio.png} IEDAB}
\newcommand{\LineaNombre}{%
\par
\vspace{\baselineskip}
Nombre:\hrulefill \; Curso: \underline{\hspace*{48pt}} \; Fecha: \underline{\hspace*{2.5cm}} \relax
\par}
\let\ds\displaystyle

\begin{document}
\ExamInstrBox{
Respuesta sin justificar mediante procedimiento no será tenida en cuenta en la calificación. Escriba sus respuestas en el espacio indicado.}
\LineaNombre
\begin{enumerate}
\begin{minipage}{.5\textwidth}
 \item Compruebe si los pares de valores que figuran en la siguiente tabla corresponden a la función
 \[y=3-\dfrac{1}{x-2}\]
 y complete los que faltan:
\end{minipage}
\begin{minipage}{.5\textwidth}
 \begin{tabular}{|c|c|c|c|c|c|c|}
 \hline 
 $x$ & 2.01 & 2.5 & 1.9 & 102 &  &  \\ 
 \hline 
$y$ & $-97$ &  & 13 &  & 3.5 & 103 \\ 
 \hline 
 \end{tabular}
\end{minipage}
 \noanswer
 \begin{minipage}{.45\textwidth}
 \item Represente la función a trozos
\[f(x)=\left\{ \begin{array}{lcl}
x+3 & \mbox{si} & x\leq 2\\
&
& \\
2x+1 & \mbox{si} & x> 2\\
\end{array}
\right.\]
y diga si es discontinua en algún punto.
 \end{minipage}
 \begin{minipage}{.5\textwidth}
%Uncomment next line if XeTeX is used
%\def\pgfsysdriver{pgfsys-xetex.def}
\usetikzlibrary{arrows}
\baselineskip=10pt
\hsize=6.3truein
\vsize=8.7truein
\definecolor{cqcqcq}{rgb}{0.75,0.75,0.75}
\tikzpicture[scale=.75,line cap=round,line join=round,>=triangle 45,x=1.0cm,y=1.0cm]
\draw [color=cqcqcq,dash pattern=on 2pt off 2pt, xstep=1.0cm,ystep=1.0cm] (-4.91,-4.31) grid (4.93,4.48);
\draw[->,color=black] (-4.91,0) -- (4.93,0);
\foreach \x in {-4,-3,-2,-1,1,2,3,4}
\draw[shift={(\x,0)},color=black] (0pt,2pt) -- (0pt,-2pt) node[below] {$\x$};
\draw[->,color=black] (0,-4.31) -- (0,4.48);
\foreach \y in {-4,-3,-2,-1,1,2,3,4}
\draw[shift={(0,\y)},color=black] (2pt,0pt) -- (-2pt,0pt) node[left] {$\y$};
\draw[color=black] (0pt,-10pt) node[right] {$0$};
\clip(-4.91,-4.31) rectangle (4.93,4.48);
\endtikzpicture 
 \end{minipage}
 \noanswer
\item Halle el dominio y rango de las siguientes funciones (recuerde que el dominio es el conjunto de valores que toma la variable independiente $x$ y el rango es el conjunto de valores que toma la variable dependiente $y$):
\begin{enumerate}
\item $y=x-3$\noanswer
\item $y=\dfrac{1}{x-1}$\noanswer
\item $y=+\sqrt{x-2}$\noanswer
\item $y=+\sqrt{x+3}$\noanswer
\end{enumerate}
Haga la gráfica de la última función ítem d)
\begin{center}
%Uncomment next line if XeTeX is used
%\def\pgfsysdriver{pgfsys-xetex.def}
\usetikzlibrary{arrows}
\baselineskip=10pt
\hsize=6.3truein
\vsize=8.7truein
\definecolor{cqcqcq}{rgb}{0.75,0.75,0.75}
\tikzpicture[scale=.85,line cap=round,line join=round,x=1.0cm,y=1.0cm]
\draw [color=cqcqcq,dash pattern=on 1pt off 1pt, xstep=1.0cm,ystep=1.0cm] (-3.19,-1.32) grid (5.5,3.61);
\draw[->,color=black] (-3.19,0) -- (5.5,0);
\foreach \x in {-3,-2,-1,1,2,3,4,5}
\draw[shift={(\x,0)},color=black] (0pt,2pt) -- (0pt,-2pt) node[below] {$\x$};
\draw[->,color=black] (0,-1.32) -- (0,3.61);
\foreach \y in {-1,1,2,3}
\draw[shift={(0,\y)},color=black] (2pt,0pt) -- (-2pt,0pt) node[left] {$\y$};
\draw[color=black] (0pt,-10pt) node[right] {$0$};
\clip(-3.19,-1.32) rectangle (5.5,3.61);
\endtikzpicture
\end{center}
\newpage
\begin{minipage}{.5\textwidth}
\item Una científica del Centro Superior de Investigaciones Científicas estudia el crecimiento de una población de bacterias. Con los datos que tiene ha elaborado la siguiente tabla:
\end{minipage}
\begin{minipage}{.45\textwidth}
\begin{center}
\begin{tabular}{|c|c|c|c|c|c|c|}
\hline 
Tiempo (min) & 0 & 1 & 2 & 3 & 4 & 5 \\ 
\hline 
\# de bacterias & 2 & 4 & 8 & 16 & 32 & 64 \\ 
\hline 
\end{tabular} 
\end{center}
\end{minipage}

\begin{enumerate}
\item ¿Son variables directamente proporcionales?\noanswer
\item ¿Cuál es la expresión matemática que representa el crecimiento de esta población de bacterias?\noanswer

\begin{minipage}[b]{.5\textwidth}
\item Representa la función gráficamente
\end{minipage}
\begin{minipage}{.5\textwidth}
\begin{center}
 \begin{tikzpicture}[scale=.85,xscale=1,yscale=.15]
 \draw[help lines,dotted,xstep=1,ystep=2](-.025,-.25)grid(5.25,64.25);
\foreach \x in {1,2,...,5}
\draw (\x,-2pt) -- (\x,1pt) node[anchor=north]{\x};
 \draw[->] (-.125,0) -- coordinate (x axis mid) (5.125,0)node[right]{$x$};
\foreach \y in {2,16,64}
\draw (-.25pt,\y) -- (.25pt,\y)node[left]{\y};
\draw[->] (0,-.25) --coordinate (y axis mid)  (0,64.25)node[above]{$y$};
 \end{tikzpicture}
\end{center}
\end{minipage}
\end{enumerate}
\begin{minipage}{.5\textwidth}
\begin{center}
 \begin{tikzpicture}[scale=.6]
 \draw[help lines,dashed](-5,-1)grid(5,5);
 \draw[->] (-5,0) -- coordinate (x axis mid) (5,0)node[right]{$x$};
\draw[->] (0,-1) --coordinate (y axis mid)  (0,5)node[left]{$y$};
\draw[domain=-5:-2,thick] plot(\x,4);
\draw[domain=-2:0,smooth,thick,samples=100] plot(\x,\x*\x);
\draw[domain=0:5,samples=100,thick] plot(\x,\x);
 \end{tikzpicture}
\end{center}
\end{minipage}
\begin{minipage}[t]{.5\textwidth}
 \item Escriba la ecuación que corresponde a la gráfica:
\end{minipage}
\noanswer
\newpage 
\item Los ángulos de un triángulo rectángulo están en progresión aritmética. Hállalos.\noanswer
\item Una persona se quiere hacer un plan de pensiones aportando al principio de cada año \$500\,000. El banco le da un interés compuesto del 6\% anual. ¿Con qué capital contará al cabo de 20 años? \noanswer
 \item Sea el experimento aleatorio “lanzar un dado”. Halla la probabilidad de los sucesos:
 \begin{enumerate}
 \item $A_{1}=$ "sacar un número par"
 \item $A_{2}=$ "sacar un número primo"
 \item $A_{3}=$ "sacar un número menor que 3"
 \item $A_{4}=$ "sacar un número par mayor que 4"
 \item $A_{5}=$ "sacar un número par o mayor que 4" 
 \end{enumerate}
 \item Calcula la probabilidad de que al lanzar dos dados la suma de sus puntos sea:
 \begin{enumerate}
 \item 5
 \item mayor o igual que 10
 \item múltiplo de 3
 \end{enumerate}
 \item En un instituto hay 1.000 alumnos repartidos por cursos de esta forma:
\begin{center}
\begin{tabular}{|c|c|c|c|c|}
\hline 
 & Primero & Segundo & Tercero & Cuarto \\ 
\hline 
Chicos & 120 & 100 & 95 & 85 \\ 
\hline 
Chicas & 200 & 150 & 130 & 120 \\ 
\hline 
\end{tabular} 
\end{center}
Elegido un alumno al azar, calcula las siguientes probabilidades:
\begin{enumerate}
\item Ser chico
\item Ser chica
\item Ser alumno de primero
\item Ser alumno de segundo
\item Ser alumno de tercero
\item Ser alumno de cuarto
\item Ser chica y alumno de cuarto
\item Ser chico y alumno de segundo
\end{enumerate}
 \end{enumerate}

\end{document}
