\documentclass[letterpaper,fleqn]{article}
\usepackage[spanish,es-noshorthands]{babel}
\usepackage[utf8]{inputenc} 
\usepackage[left=1cm, right=1cm, top=1.5cm, bottom=1.7cm]{geometry}
\usepackage{mathexam}
\usepackage{amsmath}
\usepackage{graphicx}

\ExamClass{\includegraphics[height=16pt]{Images/logo-sed.png} Álgebra $8^{\circ}$}
\ExamName{Sustentación Recomendaciones III}
\ExamHead{\includegraphics[height=16pt]{Images/logo-colegio.png} IEDAB}
\newcommand{\LineaNombre}{%
\par
\vspace{\baselineskip}
Nombre:\hrulefill \; Curso: \underline{\hspace*{48pt}} \; Fecha: \underline{\hspace*{2.5cm}} \relax
\par}
\let\ds\displaystyle

\begin{document}
\ExamInstrBox{
Respuesta sin justificar mediante procedimiento no será tenida en cuenta en la calificación. Escriba sus respuestas en el espacio indicado.}
\LineaNombre
\begin{enumerate}
 \item Encuentre el polinomio que representa el área superficial del solido rectangular de la figura (Recuerde que \'{e}ste es un paralelep\'{i}pedo de 6 caras rectangulares)
\begin{center}
\includegraphics[scale=1]{Images/solido01.png} 
\end{center}
Ahora use el polinomio obtenido para determinar la superficie de los sólidos cuyas dimensiones se especifican:
\begin{enumerate}
\item 3 por 5 por 4
\item 3 por 5 por 13
\end{enumerate}
 \end{enumerate}

\end{document}
