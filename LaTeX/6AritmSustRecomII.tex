\documentclass[letterpaper,fleqn]{article}
\usepackage[spanish,es-noshorthands]{babel}
\usepackage[utf8]{inputenc} 
\usepackage[left=1cm, right=1cm, top=1.5cm, bottom=1.7cm]{geometry}
\usepackage{mathexam}
\usepackage{amsmath}
\usepackage{graphicx}

\ExamClass{\includegraphics[height=16pt]{Images/logo-sed.png} Aritmética $6^{\circ}$}
\ExamName{Sustentación Recomendaciones II}
\ExamHead{\includegraphics[height=16pt]{Images/logo-colegio.png} IEDAB}
\newcommand{\LineaNombre}{%
\par
\vspace{\baselineskip}
Nombre:\hrulefill \; Curso: \underline{\hspace*{48pt}} \; Fecha: \underline{\hspace*{2.5cm}} \relax
\par}
\let\ds\displaystyle

\begin{document}
\ExamInstrBox{
Respuesta sin justificar mediante procedimiento no será tenida en cuenta en la calificación. Escriba sus respuestas en el espacio indicado. Tiene 45 minutos para contestar esta prueba.}
\LineaNombre
\begin{enumerate}
 \item Escriba los 10 primeros múltiplos de los números: 25, 40, 37, 102. ¿Puede encontrar más múltiplos de estos números? ¿Cuántos?\noanswer
 \item Halle los divisores de los números: 12, 30, 100, 36, 25 y 17. ¿Cuántos divisores encontró de cada número? ¿Puede encontrar más?\noanswer
 \item Realice la tabla de los números primos comprendidos entres los 100 primeros números naturales.\noanswer
 \newpage
 \item Halle los divisores comunes de cada grupo de números y luego escriba el mayor de ellos:
 \begin{enumerate}
 \item 75 y 36 \noanswer
 \item 42, 14 y 56 \noanswer
 \item 63, 27 y 45\noanswer
 \end{enumerate}
 \item Halle los múltiplos comunes de cada grupo de números y luego escriba el menor:
 \begin{enumerate}
 \item 25 y 100\noanswer
 \item 12 y 36\noanswer
 \item 6, 9 y 15\noanswer
 \end{enumerate}
 \end{enumerate}

\end{document}
