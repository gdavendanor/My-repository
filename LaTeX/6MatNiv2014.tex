\documentclass[letterpaper,fleqn]{article}
\usepackage[spanish,es-noshorthands]{babel}
\usepackage[utf8]{inputenc} 
\usepackage[left=1cm, right=1cm, top=1.5cm, bottom=1.7cm]{geometry}
\usepackage{mathexam}
\usepackage{amsmath}
\usepackage{graphicx}

\ExamClass{\includegraphics[height=16pt]{Images/logo-sed.png} Matemáticas $6^{\circ}$}
\ExamName{Nivelación 2014}
\ExamHead{\includegraphics[height=16pt]{Images/logo-colegio.png} IEDAB}
\newcommand{\LineaNombre}{%
\par
\vspace{\baselineskip}
Nombre:\hrulefill \; Curso: \underline{\hspace*{48pt}} \; Fecha: \underline{\hspace*{2.5cm}} \relax
\par}
\let\ds\displaystyle

\begin{document}
\ExamInstrBox{
Respuesta sin justificar mediante procedimiento no será tenida en cuenta en la calificación. Escriba sus respuestas en el espacio indicado. Tiene 60 minutos para contestar esta prueba.}
\LineaNombre
\begin{enumerate}
 \item Realice las siguientes operaciones:
 \begin{enumerate}
 \item $3+6\cdot 5-3\cdot 4-2=$\noanswer
 \item $7\cdot 3+[6+2\cdot (8\div 4+3\cdot 2)-7\cdot 2]+9\div 3=$\noanswer
 \end{enumerate}
 \item Resuelva los siguientes problemas:
 \begin{enumerate}
 \item Don Tomás quiere repartir unos libros entre sus hijos. Puede hacerlo dándoles 1 al mayor, 2 al segundo, 3 al tercero \ldots Otro modo de repartirlos sería dar 7 a cada uno. ¿Cuántos hijos y cuántos libros tiene don Tomás?\noanswer
 \item Maité quiere comprar sellos. Tiene menos de 100 pesetas, si los compra todos de 5 pesetas, le sobra una peseta. Si los compra de 8 pesetas le sobran 6 pesetas. Le falta una peseta para comprar un n\'{u}mero exacto de sellos de 29 pesetas. ¿Cu\'{a}nto dinero tiene Mait\'{e}?\noanswer
 \end{enumerate}
 \item ¿Cuál es el menor número que tiene por divisores 3, 4 y 12?\noanswer
 \item Compruebe que para saber si un número menor que 100 es primo, es suficiente con dividir por 2, 3, 5 y 7. ¿Por cuántos números como máximo tendrá que dividir para saber si es primo el número 497?\noanswer
 \item En una granja se ha recogido un número de huevos entre setecientos y ochocientos. Forman un número exacto de docenas. También se podrían colocar exactamente en cartones de 15 huevos. ¿Cuántos huevos se han recogido en la granja?\noanswer
 \end{enumerate}

\end{document}
