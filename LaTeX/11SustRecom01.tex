\documentclass[letterpaper,fleqn]{article}
\usepackage[spanish,es-noshorthands]{babel}
\usepackage[utf8]{inputenc} 
\usepackage[papersize={6.5in,8.5in},left=1cm, right=1cm, top=1.5cm, bottom=1.7cm]{geometry}
\usepackage{mathexam}
\usepackage{amsmath}
\usepackage{graphicx}

\ExamClass{\includegraphics[height=16pt]{Images/logo-sed.png} Matemáticas $11^{\circ}$}
\ExamName{Sustentación Recomendaciones I}
\ExamHead{\includegraphics[height=16pt]{Images/logo-colegio.png} IEDAB}
\newcommand{\LineaNombre}{%
\par
\vspace{\baselineskip}
Nombre:\hrulefill \; Curso: \underline{\hspace*{48pt}} \; Fecha: \underline{\hspace*{2.5cm}} \relax
\par}
\let\ds\displaystyle

\begin{document}
\ExamInstrBox{
Respuesta sin justificar mediante procedimiento no será tenida en cuenta en la calificación. Escriba sus respuestas en el espacio indicado. Tiene 60 minutos para contestar esta prueba.}
\LineaNombre
\begin{enumerate}
 \item Efectúe las operaciones siguientes simplificando la respuesta al máximo:
 \begin{enumerate}
 \item $\dfrac{3}{4}-\dfrac{4}{5}=$\noanswer
 \item $\dfrac{\frac{3}{4}-\frac{1}{3}}{\frac{1}{2}-\frac{1}{4}}=$\noanswer
 \end{enumerate}
\item Dados los intervalos $A=(-2,5)$ y $B=[-6,\infty)$, ubíquelos en la recta numérica y halle:
\begin{enumerate}
\item $A\cup B=$ \noanswer
\item $A \cap B=$ \noanswer
\item $A^{c}=$ \noanswer
\end{enumerate}
\item Escriba como intervalos las siguientes desigualdades y ubíquelos en la recta numérica
\begin{enumerate}
\item $-5\leq x<10$\noanswer
\item $-2<x \leq7$\noanswer
\item $x>4$\noanswer
\end{enumerate}
\newpage
\item Escriba como desigualdad los siguientes intervalos y ubíquelos en la recta numérica
\begin{enumerate}
\item (--4,6]= \noanswer
\item $(-\infty,2)=$ \noanswer
\end{enumerate}
\item Haga la tabla de verdad de la proposición compuesta:
\[[\neg(p\wedge q)]\Rightarrow [\neg p \vee \neg q]\]
\begin{tabular}{|c|c|c|c|c|c|c|c|}
\hline 
p & q & $\neg p$ & $\neg q$ & $p\wedge q$ & $\neg (p\wedge q)$ & $[\neg p \vee \neg q] $& $[\neg(p\wedge q)]\Rightarrow [\neg p \vee \neg q]$  \\ 
\hline 
 &  &  &  &  &  &  &  \\ 
\hline 
 &  &  &  &  &  &  &  \\ 
\hline 
 &  &  &  &  &  &  &  \\ 
\hline 
 &  &  &  &  &  &  &  \\ 
\hline 
\end{tabular} 
\item ¿Cuál es la probabilidad de que al lanzar dos dados, su suma sea 7?\noanswer
 \end{enumerate}

\end{document}
