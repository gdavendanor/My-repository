\documentclass[10pt,twoside]{article}
\usepackage[utf8]{inputenc}
\usepackage{amsmath}
\usepackage{amsfonts}
\usepackage{amssymb}
\usepackage[spanish,es-noshorthands]{babel}
\usepackage[T1]{fontenc}
\usepackage{lmodern}
\usepackage{graphicx,hyperref}
\usepackage{tikz,pgf}
\usepackage{multicol}
\usepackage{subfig}
\usepackage[papersize={6.5in,8.5in},width=5.5in,height=7in]{geometry}
\usepackage{fancyhdr}
\pagestyle{fancy}
\fancyhead[LE]{\includegraphics[height=12pt]{Images/logo-colegio.png} Cálculo $11^{\circ}$}
\fancyhead[RE]{}
\fancyhead[RO]{\textit{Germ\'an Avenda\~no Ram\'irez, Lic. U.D., M.Sc. U.N.}}
\fancyhead[LO]{}

\author{Germ\'an Avenda\~no Ram\'irez, Lic. U.D., M.Sc. U.N.}
\title{\begin{minipage}{.2\textwidth}
\includegraphics[height=1.75cm]{Images/logo-colegio.png}\end{minipage}
\begin{minipage}{.55\textwidth}
\begin{center}
Taller, Límites de funciones en $\mathbb{R}$\\
Cálculo $11^{\circ}$
\end{center}
\end{minipage}\hfill
\begin{minipage}{.2\textwidth}
\includegraphics[height=1.75cm]{Images/logo-sed.png} 
\end{minipage}}
\date{}
\begin{document}
\maketitle
Nombre: \hrulefill Curso: \underline{\hspace*{44pt}} Fecha: \underline{\hspace*{2.5cm}}
\section*{Introducci\'{o}n}
\subparagraph*{Materiales:}
Regla, escuadra, calculadora, esferos o lápices de diferentes colores.
\begin{enumerate}
\item Grafica cada una de las siguientes funciones definidas en el conjunto de los números Reales:
\begin{enumerate}
\begin{multicols}{3}
\item $y=f(x)=2x+1$
\item $y=g(x)=x^{2}-4$
\item $y=h(x)=x^{3}-2x$
\end{multicols}
\end{enumerate}
\item En la siguiente recta numérica, escoge un par de unidades consecutivas y cada una divídelas en 10 partes
iguales. Coloca el número correspondiente a cada división. ¿Cuáles serían los números si cada unidad es dividida en 100 partes iguales?
\begin{center}
\begin{tikzpicture}[>=stealth,scale=1.2]
\draw[<->] (-2.5,0) -- (3.5,0);
\draw [thick] (-2,-.1) node[below]{$-2$} -- (-2,0.1);
\draw [thick] (-1,-.1) node[below]{$-1$} -- (-1,0.1);
\draw [thick] (0,-.1) node[below]{0} -- (0,0.1);
\draw [thick] (1,-.1) node[below]{1} -- (1,0.1);
\draw [thick] (2,-.1) node[below]{2} -- (2,0.1);
\draw [thick] (3,-.1) node[below]{3} -- (3,0.1);
\end{tikzpicture}
\end{center}
\item A continuación encontrarás dibujadas dos rectas. Traza perpendiculares por los puntos dibujados
\begin{center}
\begin{tikzpicture}[>=stealth,scale=.9]
\draw[<->] (-2.5,0) -- (3.5,0);
\filldraw (-2,0)circle (1.5pt);
\filldraw (2,0)circle(1.5pt);
\filldraw (1,0)circle(1.5pt);
\draw[<->] (4.5,0)--(10.5,0);
\filldraw (5,1)circle(1.5pt);
\filldraw (7,3)circle(1.5pt);
\filldraw(8,2)circle(1.5pt);
\end{tikzpicture}
\end{center}
\item Consideremos la función definida mediante la expresión $y=j(x)=4-x^{2}$. Observemos los valores del recorrido ($y$) cuando los del dominio ($x$) están cerca de 1. Para ello:
\begin{enumerate}
\item Elaboramos una tabla de valores donde se observen los valores de “$y$” cuando los de “$x$” se están
acercando a 1:
\begin{center}
\begin{tabular}{c|c|c|c|c|c|c|c|}
 & \multicolumn{3}{c|}{Por la izquierda de 1} &  &\multicolumn{3}{c}{Por la derecha de 1} \\
   & \multicolumn{3}{c|}{$\longrightarrow$} &  &\multicolumn{3}{c}{$\longleftarrow$} \\
\hline 
$x$ & 0.97 & 0.98 & 0.99 & \textbf{1} & 1.01 & 1.02 & 1.03 \\ 
\hline 
$y$ &  &  &  &  &  &  &  \\ 
\hline 
\end{tabular} 
\end{center}
\item Construimos su gráfica conectando mediante segmentos de rectas, los elementos del Dominio próximos a 1, con su correspondiente elemento del recorrido:
\begin{center}
\begin{tikzpicture}[xscale=1.5]
\draw [<->](0,-3.2)--(0,4.3);
\foreach \y in {-3,-2,-1,1,2,3}
\draw[shift={(0,\y)},color=black] (1pt,0pt) -- (-1pt,0pt) node[left] {$\y$};
\draw [<->](-3,0)--(3,0);
\foreach \x in {-3,-2,-1,1,2,3}
\draw[shift={(\x,0)},color=black] (0pt,2pt) -- (0pt,-2pt) node[below] {$\x$};
\draw plot [domain=-2.7:2.7] (\x, 4-\x*\x);
\draw[ultra thin](0.9,0)--(0.9,3.19)--(0,3.19);
\draw[ultra thin](0.95,0)--(.95,3.0975)--(0,3.0975);
\draw[very thick] (1,0)--(1,3)--(0,3);
\draw[ultra thin] (1.1,0)--(1.1,2.79)--(0,2.79);
\draw[ultra thin] (1.05,0)--(1.05,2.8975)--(0,2.8975);
\end{tikzpicture}
\end{center}
\item Hacia que valores se aproximan los de “$y$”, cuando los de “$x$” se acercan a 1?
\item Observemos que ocurre gráficamente. Para ello haz cuatro gráficas de la función. En cada una de ellas:
\begin{itemize}
\item Dibuja en el eje “$y$”, una de las siguientes vecindades del 3: $V_{1}(3)$, $V_{\frac{1}{2}}(3)$, $V_{\frac{1}{4}}(3)$ y $V_{\frac{1}{10}}(3)$. (Vecindad $V_{\frac{1}{2}}(3)$ significa que cerca de tres se construye una vecindad de radio $\frac{1}{2}$, es decir de radio 0.5; tenemos entonces el invervalo abierto (--2.5,3.5))
\item Escoge varios puntos de la vecindad (pueden ser dos, por encima y por debajo de 3). Levanta en
cada uno de ellos una perpendicular que llegue hasta la gráfica. A continuación, traza desde aquí, otra perpendicular que llegue hasta el eje “$x$”.
\item ¿Dentro de qué vecindad quedan los puntos de los extremos de los segmentos que llegan hasta el
eje “$x$”?
\item ¿Qué pasa cuando la vecindad es más pequeña?
\end{itemize}
\end{enumerate}
\item De lo anterior, podemos darnos cuenta que no importa la vecindad de 3 que escojamos, que siempre tendremos una vecindad (del número 1) en el eje “$x$” dentro de la cual se encuentran los valores del dominio próximo a él, pero que cuanta más pequeña sea la vecindad escogida en el eje “$y$”, más cercanos al número 1 estarán los valores de "$x$". Ver gráficos:
\begin{multicols}{2}
\begin{center}
\begin{tikzpicture}
\draw [<->](0,-1.2)--(0,4.3);
\foreach \y in {-1,1,2,3}
\draw[shift={(0,\y)},color=black] (1pt,0pt) -- (-1pt,0pt) node[left] {$\y$};
\draw [<->](-3,0)--(3,0);
\foreach \x in {-3,-2,-1,1,2,3}
\draw[shift={(\x,0)},color=black] (0pt,2pt) -- (0pt,-2pt) node[below] {$\x$};
\draw plot [domain=-2.1:2.1] (\x, 4-\x*\x);
\draw[ultra thin](0.707106781,0)--(0.707106781,3.5)--(0,3.5);
\draw[very thick] (1,0)--(1,3)--(0,3);
\draw[ultra thin] (1.224744871,0)--(1.224744871,2.5)--(0,2.5);
\end{tikzpicture}
\end{center}
\begin{center}
\begin{tikzpicture}
\draw [<->](0,-1.2)--(0,4.3);
\foreach \y in {-1,1,2,3}
\draw[shift={(0,\y)},color=black] (1pt,0pt) -- (-1pt,0pt) node[left] {$\y$};
\draw [<->](-3,0)--(3,0);
\foreach \x in {-3,-2,-1,1,2,3}
\draw[shift={(\x,0)},color=black] (0pt,2pt) -- (0pt,-2pt) node[below] {$\x$};
\draw plot [domain=-2.1:2.1] (\x, 4-\x*\x);
\draw[ultra thin](0.948683298,0)--(0.948683298,3.1)--(0,3.1);
\draw[semithick] (1,0)--(1,3)--(0,3);
\draw[ultra thin] (1.048808848,0)--(1.048808848,2.9)--(0,2.9);
\end{tikzpicture}
\end{center}
\end{multicols}
La situación anterior es descrita en matemáticas diciendo que el Límite de la función $j(x)=4-x^{2}$, cuando $x$ esta próxima (o tiende) a 1, es igual a 3. También suele decirse que “$j(x)$ tiende a 3, cuando $x$ tiende a 1” y se escribe:
\[\displaystyle{\lim_{x\rightarrow1}}j(x)=3\]
En ocasiones se escribe $j(x) \rightarrow 3$ cuando $x\rightarrow 1$
\item A continuación te presentamos varias gráficas de funciones definidas en los números reales para que determines el valor hacia donde se acercan los de “$y = f(x)$” cuando “$x$” se aproxima al valor indicado, escribiendo el resultado con notación de límites:
\begin{minipage}{.45\textwidth}
\begin{center}
\begin{tikzpicture}
\draw [<->](0,-0.2)--(0,4.1);
\foreach \y in {1,2,3,4}
\draw[shift={(0,\y)},color=black] (1pt,0pt) -- (-1pt,0pt) node[left] {$\y$};
\draw [<->](-2.3,0)--(2.3,0);
\foreach \x in {-2,-1,1,2}
\draw[shift={(\x,0)},color=black] (0pt,2pt) -- (0pt,-2pt) node[below] {$\x$};
\draw plot [domain=-1.7:1.7] (\x, \x*\x+1);
\node at (0,-0.3)[below] {"$x$ esté próximo a $-1$"};
\end{tikzpicture}
\end{center}
\end{minipage}
\begin{minipage}{.45\textwidth}
\begin{center}
\begin{tikzpicture}
\draw [<->](0,-2)--(0,4.1);
\foreach \y in {1,2,3,4}
\draw[shift={(0,\y)},color=black] (1pt,0pt) -- (-1pt,0pt) node[left] {$\y$};
\draw [<->](-2.3,0)--(2.3,0);
\foreach \x in {-2,-1,1,2}
\draw[shift={(\x,0)},color=black] (0pt,2pt) -- (0pt,-2pt) node[below] {$\x$};
\draw plot [domain=-1:2.1] (\x, {(\x)^2*((\x)-1)});
\node at (0,-0.5)[below] {"$x$ esté próximo a $2$"};
\end{tikzpicture}
\end{center}
\end{minipage}


\begin{minipage}{.45\textwidth}
\begin{center}
%Uncomment next line if XeTeX is used
%\def\pgfsysdriver{pgfsys-xetex.def}
\tikzpicture[line cap=round,scale=.9,line join=round,x=1.0cm,y=1.0cm]
\draw[->,color=black] (-1.04,0) -- (3.71,0);
\foreach \x in {-1,1,2,3}
\draw[shift={(\x,0)},color=black] (0pt,2pt) -- (0pt,-2pt) node[below] {$\x$};
\draw[->,color=black] (0,-0.76) -- (0,6.65);
\foreach \y in {,1,2,3,4,5,6}
\draw[shift={(0,\y)},color=black] (2pt,0pt) -- (-2pt,0pt) node[left] {$\y$};
\draw[color=black] (0pt,-10pt) node[right] {$0$};
\clip(-1.04,-0.76) rectangle (3.71,6.65);
\draw[smooth,samples=100,domain=0.0:2.0] plot(\x,{(\x)^2});
\draw[smooth,samples=100,domain=2.0:3.7054278612737157] plot(\x,{(\x)+3});
\fill (2,4) circle (1.5pt);
\draw (2,5) circle (1.5pt);
\node at (2.2,.6)[below]{''x cerca de 2''};
\endtikzpicture
\end{center}
\end{minipage}\hfill
\begin{minipage}{.45\textwidth}
\begin{center}
\tikzpicture[line cap=round,scale=.9,line join=round,x=1.0cm,y=1.0cm]
\draw[->,color=black] (-3.45,0) -- (2.84,0);
\foreach \x in {-3,-2,-1,1,2}
\draw[shift={(\x,0)},color=black] (0pt,2pt) -- (0pt,-2pt) node[below] {$\x$};
\draw[->,color=black] (0,-0.83) -- (0,4.67);
\foreach \y in {,1,2,3,4}
\draw[shift={(0,\y)},color=black] (2pt,0pt) -- (-2pt,0pt) node[left] {$\y$};
\draw[color=black] (0pt,-10pt) node[right] {$0$};
\clip(-3.45,-0.83) rectangle (2.84,4.67);
\draw[smooth,samples=100,domain=1.01:2.835754159855163] plot(\x,{(\x)+2});
\draw[smooth,samples=100,domain=-3.0:1.0] plot(\x,{sqrt((\x)+3)});
\fill (1,2) circle (1.5pt);
\draw (1,3) circle (1.5pt);
\node at (1.2,1)[below]{"x cerca de 1"};
\endtikzpicture
\end{center}
\end{minipage}
\item Determine los siguientes límites haciendo la tabla de valores cercanos al número indicado por el límite. (Ejemplo, si $\displaystyle{\lim_{x\rightarrow 2}}f(x)$, entonces se deberá hacer una tabla de valores cercanos a 2 por la izquierda y derecha, los cuales podrían ser 1.9, 1.99, 1.999 por izquierda y 2.001, 2.01, 2.1 por derecha).
\begin{enumerate}
\begin{multicols}{3}
\item $\displaystyle{\lim_{x\rightarrow1}}3x+1=$
\item $\displaystyle{\lim_{x\rightarrow3}}x^{2}-4=$
\item $\displaystyle{\lim_{x\rightarrow2}}\dfrac{x^{2}-x-2}{x-2}$
\item $\displaystyle{\lim_{x\rightarrow2}}\dfrac{x-2}{x^{2}-4}$
\item $\displaystyle{\lim_{x\rightarrow6}}\dfrac{\sqrt{x+3}-\sqrt{3}}{x}$
\end{multicols}
\end{enumerate}
\end{enumerate}
\end{document}
