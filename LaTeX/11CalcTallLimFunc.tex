\documentclass[10pt,twoside]{article}
\usepackage[utf8]{inputenc}
\usepackage{amsmath}
\usepackage{amsfonts}
\usepackage{amssymb}
\usepackage[spanish,es-noshorthands]{babel}
\usepackage[T1]{fontenc}
\usepackage{lmodern}
\usepackage{graphicx,hyperref}
\usepackage{tikz,pgf}
\usepackage{multicol}
\usepackage{subfig}
\usepackage[papersize={6.5in,8.5in},width=5.5in,height=7in]{geometry}
\usepackage{fancyhdr}
\pagestyle{fancy}
\fancyhead[LE]{\includegraphics[height=12pt]{Images/logo-colegio.png} Cálculo $11^{\circ}$}
\fancyhead[RE]{}
\fancyhead[RO]{\textit{Germ\'an Avenda\~no Ram\'irez, Lic. U.D., M.Sc. U.N.}}
\fancyhead[LO]{}

\author{Germ\'an Avenda\~no Ram\'irez, Lic. U.D., M.Sc. U.N.}
\title{\begin{minipage}{.2\textwidth}
\includegraphics[height=1.75cm]{Images/logo-colegio.png}\end{minipage}
\begin{minipage}{.55\textwidth}
\begin{center}
Taller, Límites de funciones en $\mathbb{R}$\\
Cálculo $11^{\circ}$
\end{center}
\end{minipage}\hfill
\begin{minipage}{.2\textwidth}
\includegraphics[height=1.75cm]{Images/logo-sed.png} 
\end{minipage}}
\date{}
\begin{document}
\maketitle
Nombre: \hrulefill Curso: \underline{\hspace*{44pt}} Fecha: \underline{\hspace*{2.5cm}}
\section*{Introducci\'{o}n}
\subparagraph*{Materiales:}
Regla, escuadra, calculadora, esferos o lápices de diferentes colores.
\begin{enumerate}
\item Grafica cada una de las siguientes funciones definidas en el conjunto de los números Reales:
\begin{enumerate}
\begin{multicols}{3}
\item $y=f(x)=2x+1$
\item $y=g(x)=x^{2}-4$
\item $y=h(x)=x^{3}-2x$
\end{multicols}
\end{enumerate}
\item En la siguiente recta numérica, escoge un par de unidades consecutivas y cada una divídelas en 10 partes
iguales. Coloca el número correspondiente a cada división. ¿Cuáles serían los números si cada unidad es dividida en 100 partes iguales?
\begin{center}
\begin{tikzpicture}[>=stealth,scale=1.2]
\draw[<->] (-2.5,0) -- (3.5,0);
\draw [thick] (-2,-.1) node[below]{$-2$} -- (-2,0.1);
\draw [thick] (-1,-.1) node[below]{$-1$} -- (-1,0.1);
\draw [thick] (0,-.1) node[below]{0} -- (0,0.1);
\draw [thick] (1,-.1) node[below]{1} -- (1,0.1);
\draw [thick] (2,-.1) node[below]{2} -- (2,0.1);
\draw [thick] (3,-.1) node[below]{3} -- (3,0.1);
\end{tikzpicture}
\end{center}
\item A continuación encontrarás dibujadas dos rectas. Traza perpendiculares por los puntos dibujados
\begin{center}
\begin{tikzpicture}[>=stealth,scale=.9]
\draw[<->] (-2.5,0) -- (3.5,0);
\filldraw (-2,0)circle (2pt);
\filldraw (2,0)circle(2pt);
\filldraw (1,0)circle(2pt);
\draw[<->] (4.5,0)--(10.5,0);
\filldraw (5,3)circle(2pt);
\filldraw (7,5)circle(2pt);
\filldraw(8,4)circle(2pt);
\end{tikzpicture}
\end{center}
\item Consideremos la función definida mediante la expresión $y=j(x)=4-x^{2}$. Observemos los valores del recorrido ($y$) cuando los del dominio ($x$) están cerca de 1. Para ello:
\begin{enumerate}
\item Elaboramos una tabla de valores donde se observen los valores de “$y$” cuando los de “$x$” se están
acercando a 1:
\begin{center}
\begin{tabular}{c|c|c|c|c|c|c|c|}
 & \multicolumn{3}{c|}{Por la izquierda de 1} &  &\multicolumn{3}{c}{Por la derecha de 1} \\
   & \multicolumn{3}{c|}{$\longrightarrow$} &  &\multicolumn{3}{c}{$\longleftarrow$} \\
\hline 
$x$ & 0.97 & 0.98 & 0.99 & \textbf{1} & 1.01 & 1.02 & 1.03 \\ 
\hline 
$y$ &  &  &  &  &  &  &  \\ 
\hline 
\end{tabular} 
\end{center}
\item Construimos su gráfica conectando mediante segmentos de rectas, los elementos del Dominio próximos a 1, con su correspondiente elemento del recorrido:
\begin{center}
\begin{tikzpicture}[xscale=1.5]
\draw [<->](0,-3.2)--(0,4.3);
\foreach \y in {-3,-2,-1,1,2,3}
\draw[shift={(0,\y)},color=black] (2pt,0pt) -- (-2pt,0pt) node[left] {$\y$};
\draw [<->](-3,0)--(3,0);
\foreach \x in {-3,-2,-1,1,2,3}
\draw[shift={(\x,0)},color=black] (0pt,2pt) -- (0pt,-2pt) node[below] {$\x$};
\draw plot [domain=-2.7:2.7] (\x, 4-\x*\x);
\draw[thick](0.97,0)--(0.97,3.0591);
\draw[thick](0.98,0)--(.98,3.0396);
\end{tikzpicture}
\end{center}
\end{enumerate}
\end{enumerate}
\end{document}
