\documentclass[twoside]{article}
\usepackage[utf8]{inputenc}
\usepackage[spanish,es-noshorthands]{babel}
\usepackage[T1]{fontenc}
\usepackage{lmodern}
\usepackage{graphicx,hyperref}
\usepackage{tikz,pgf}
\usepackage{marvosym}
\usepackage{multicol}
\usepackage{fancyhdr}
\usepackage{framed}
\usepackage{color}
\usepackage{wrapfig}\definecolor{shadecolor}{RGB}{224,238,238}
\usepackage[papersize={8.5in,13in},total={7.5in,11.75in},centering]{geometry}
\usepackage{fancyhdr}
\pagestyle{fancy}
\fancyhead[LE]{IEDAB}
\fancyhead[RE]{PEI:``Hacia una cultura para el desarrollo sostenible''}
\fancyfoot[RO]{\Email iedabgerman@autistici.org}
\fancyhead[LO]{\url{www.autistici.org/mathgerman}}
\fancyfoot[RE]{\Email iedabgerman@autistici.org}
\fancyfoot[LE]{\url{www.autistici.org/mathgerman}}
\fancyhead[RO]{IEDAB}

%\author{Germ\'an Avenda\~no Ram\'irez~\thanks{Lic. Mat. U.D., M.Sc. U.N.}}
\title{\begin{minipage}{.2\textwidth}
\includegraphics[height=1.75cm]{Images/logo-colegio.png}\end{minipage}
\begin{minipage}{.55\textwidth}
\begin{center}
GUÍA: PEDAGOGÍA SOBRE CONSULTA POPULAR DEL 26 DE AGOSTO DE 2018\\
\end{center}
\end{minipage}\hfill
\begin{minipage}{.2\textwidth}
\includegraphics[height=1.75cm]{Images/logo-sed.png} 
\end{minipage}}
\date{}
\thispagestyle{plain}
\begin{document}
\maketitle
\vspace{-1cm}
\section*{Logros}
\begin{itemize}
\item Conocer e identificar  los mecanismos de participación ciudadana, en este caso la consulta popular anti-corrupción.
\item Conocer los siete puntos  que hacen parte de la consulta anti-corrupción. 
\end{itemize}
\fbox{
  \begin{minipage}{0.9\linewidth}
    \emph{Se realizará el día miércoles 22 de agosto de 2018 en las primeras 2 horas de clase  en primaria con acompañamiento del grado 1003 y en bachillerato las 2 últimas horas de clase con acompañamiento del grado 1103.}
    \end{minipage}
  }\\
  
Según la Constitución Política de Colombia de 1991 en el artículo 1 y 2 Colombia es un “ESTADO SOCIAL DE DERECHO, ES DEMÓCRATICA, PARTICIPATIVA Y PLURALISTA”. El estado debe garantizar a través de la educación el cumplimiento de estos fines, para que la población colombiana tenga participación real en la toma de decisiones que afectan al país en lo económico, administrativo, político y cultural.De acuerdo al artículo 103 de la Constitución política de Colombia, Ley 134 del /94, son mecanismos de participación del pueblo en ejercicio de su soberanía: el voto, el plebiscito, referendo, cabildo abierto, iniciativa legislativa, revocatoria del mandato y consulta popular.

La consulta popular es un mecanismo de participación mediante el cual unas preguntas de carácter general sobre temas de trascendencia nacional, departamental, distrital, municipal y local son sometidas por el presidente de la república, el gobernador, el alcalde a consideración del pueblo para que este se pronuncie sobre ellas; en nuestro caso el 26 de agosto del presente año se realizara la consulta popular anticorrupción en Colombia con las siguientes preguntas:
\begin{enumerate}
\item REDUCIR EL SALARIO DE LOS CONGRESISTAS Y ALTOS FUNCIONARIOS DEL ESTADO: ¿Aprueba usted reducir el salario de los congresistas de 40 a 25 salarios mínimos legales mensuales vigentes, fijando un tope de 25 (SMLV) como máxima remuneración mensual de los congresistas y altos funcionarios del estado, señalados en el artículo 197 de la constitución política?
    \begin{wrapfigure}[0]{r}{0.4\textwidth}
    \vspace{-15pt}
\begin{shaded}
  Se calcula que entre salarios, primas sobresueldos, etc. De congresistas y altos funcionarios, el país se ahorraría cada año más de 200 mil millones de pesos por esta razón apoyamos la consulta. Vamos a votar sí
  \end{shaded}
\end{wrapfigure}
\end{enumerate}
\end{document}
