\documentclass[fleqn]{article}
\usepackage[spanish,es-noshorthands]{babel}
\usepackage[utf8]{inputenc} 
\usepackage[papersize={5.5in,8.5in},left=1cm, right=1cm, top=1.5cm, bottom=1.7cm]{geometry}
\usepackage{mathexam}
\usepackage{amsmath}
\usepackage{graphicx}

\ExamClass{\includegraphics[height=16pt]{Images/logo-sed.png} Matemáticas $9^{\circ}$}
\ExamName{Plan de Mejoramiento 3}
\ExamHead{\includegraphics[height=16pt]{Images/logo-colegio.png} IEDAB}
\newcommand{\LineaNombre}{%
\par
\vspace{\baselineskip}
Nombre:\hrulefill \; Curso: \underline{\hspace*{48pt}} \; Fecha: \underline{\hspace*{2.5cm}} \relax
\par}
\let\ds\displaystyle

\begin{document}
\ExamInstrBox{
No se permite el uso de calculadora ni de cualquier dispositivo electrónico. Quien haga caso omiso de esta advertencia corre el riesgo de que se le anule su examen. Respuesta sin justificar mediante procedimiento no será tenida en cuenta en la calificación. Escriba sus respuestas en el espacio indicado. Tiene 45 minutos para contestar esta prueba.}
\LineaNombre
\begin{enumerate}
 \item Solucione las ecuaciones cuadráticas siguientes usando la factorización como método de solución:
 \begin{enumerate}
 \item $x^{2}+9x=0$\noanswer
\item $x^{2}=10x$ \noanswer
\item $x^{2}-2x-24=0$\noanswer
 \end{enumerate}
 \item Solucione las siguientes ecuaciones cuadráticas usando la propiedad 1\footnote{$x^{2}=a$ si y solamente sí $x=\sqrt{a}$ o, $x=-\sqrt{a}$, que se puede simplificar así $x=\pm\sqrt{a}$}
 \begin{enumerate}
 \item $3x^{2}=45$\noanswer
\item $x^{2}-18=0$\noanswer
 \end{enumerate}
 \newpage
 \item Resuelva las ecuaciones cuadráticas siguientes usando la fórmula cuadrática $x=\dfrac{-b\pm\sqrt{b^{2}-4ac}}{2a}$
 \begin{enumerate}
 \item $x^{2}+5x+8=0$\noanswer
\item $2x^{2}-3x+5=0$\noanswer
 \end{enumerate}
 \item El perímetro de un rectángulo es 16 centímetros y su área es 15 cm$^{2}$. Encuentre el ancho y largo del rectángulo.\noanswer
 \end{enumerate}

\end{document}
