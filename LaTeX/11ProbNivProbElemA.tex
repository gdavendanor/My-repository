\documentclass[fleqn]{article}
\usepackage[spanish,es-noshorthands]{babel}
\usepackage[utf8]{inputenc} 
\usepackage[papersize={5.5in,8.5in},total={4.5in,7.25in},centering]{geometry}
\usepackage{mathexam}
\usepackage{amsmath}
\usepackage{graphicx}
\usepackage{multicol}

\ExamClass{\includegraphics[height=16pt]{Images/logo-sed.png} Probabilidad $11^{\circ}$}
\ExamName{Niv. Prob. Elemental}
\ExamHead{\includegraphics[height=16pt]{Images/logo-colegio.png} IEDAB}
\newcommand{\LineaNombre}{%
\par
\vspace{\baselineskip}
Nombre:\hrulefill \; Curso: \underline{\hspace*{48pt}} \; Fecha: \underline{\hspace*{2.5cm}} \relax
\par}
\let\ds\displaystyle

\begin{document}
\ExamInstrBox{
Respuesta sin justificar mediante procedimiento no será tenida en cuenta en la calificación. Escriba sus respuestas en el espacio indicado. Tiene 45 minutos para contestar esta prueba.}
\LineaNombre
\begin{enumerate}
 \item Se lanzan 3 monedas. Construye el espacio muestral y cada uno de los siguientes sucesos y halla la probabilidad correspondiente
 \begin{enumerate}
 \item $A=$ ``Sacar 2 sellos y una cara'' \noanswer
 \item $B=$ ``Sacar 2 caras y un sello'' \noanswer
 \item $\overline{B}=$ \noanswer
 \item $A\cup B=$ \noanswer
 \item $A\cap B=$ \noanswer
 \end{enumerate}
 \item Sea el experimento aleatorio ''lanzar un dado``. Halle los sucesos descrito y su correspondiente probabilidad de los sucesos:
\begin{enumerate}
\item $A_{1}=$"Sacar un número"; $A_{1}=\{$ \noanswer
\item $A_{2}=$"sacar un número múltiplo de 3 " \noanswer
\item $A_{3}=$"sacar un número menor que 3" \noanswer
\item $A_{4=}$"sacar un número par mayor que 3" \noanswer
\item $A_{5}=$"sacar un número impar o mayor que 4"\noanswer
\end{enumerate}
  \newpage
\item Halla la probabilidad de que al lanzar dos dados aparezca:
\begin{enumerate}
 \item en el primero un número par y en el segundo un múltiplo de 3 \noanswer
 \item en el primero par y en el segundo menor que 5 \noanswer
\end{enumerate}
\item Calcula la probabilidad de que al lanzar dos dados la suma de sus puntos sea:
\begin{enumerate}
\item 4 \noanswer
\item mayor o igual que 9 \noanswer
\item múltiplo de 4 \noanswer
\item 6 \noanswer
\end{enumerate}
\item Durante el curso 1986/87 el número de estudiantes de los antiguos BUP y COU, en Aragón, fue:

\begin{tabular}{|c|c|c|c|}
\hline 
 & Huesca & Teruel & Zaragoza \\ 
\hline 
Centro público & 5091 & 2277 & 17805 \\ 
\hline 
Centro privado & 1284 & 896 & 12775 \\ 
\hline 
\end{tabular} 

Si hubiese elegido una de esas personas al azar, calcula la probabilidad de que estudiase en:
\begin{enumerate}
\item Huesca \noanswer
\item Un centro público de Zaragoza \noanswer
\item Un centro privado \noanswer
\end{enumerate}
 \end{enumerate}

\end{document}
