\documentclass[10pt,letterpaper,addpoints]{exam}
\usepackage[utf8]{inputenc}
\usepackage[spanish,es-noshorthands]{babel}
\usepackage{hyperref}
\usepackage{amsmath}
\usepackage{amsfonts}
\usepackage{amssymb}
\usepackage{graphicx}
\usepackage{tikz}
\usepackage{multicol}
\usepackage[centering,total={7.5in,9.75in}]{geometry}
\printanswers
\begin{document}
\title{\begin{minipage}{.2\textwidth}
        \includegraphics[height=1.75cm]{Images/logo-colegio.png}
       \end{minipage}
\begin{minipage}{.55\textwidth}
 \begin{center}
Preguntas tipo Icfes \\Matemáticas $11^{\circ}$
\end{center}
\end{minipage}
\begin{minipage}{.2\textwidth}
\includegraphics[height=1.75cm]{Images/logo-sed.png} 
\end{minipage}
}
\author{Germ\'{a}n Avendaño Ram\'{i}rez\\Lic. Matemáticas U.D., M.Sc. U.N.}
\date{}
\maketitle
%\begin{center}
%\fbox{\fbox{\parbox{5.5in}{\centering
%Instrucciones}}}
%\end{center}
\vspace{0.1in}
%\makebox[\textwidth]{Nombres: \hrulefill, curso:\underline{\hspace{48pt}}, fecha:\underline{\hspace{3cm}}}
\begin{questions}
\question
El caudal ($Q$) se define como el volumen de algún líquido que pasa por un conducto en un determinado tiempo
\[Q=\dfrac{V}{t}\]
Donde $V$ es el volumen del líquido y $t$ es el tiempo que tarda en pasar.
De acuerdo con esto, una unidad de medida del caudal de líquido puede ser

\begin{oneparchoices}
\choice $\dfrac{m^{3}}{litro}$
\choice $\dfrac{km}{hora}$
\choice $\dfrac{litro}{dm}$
\CorrectChoice $\dfrac{cm^{3}}{seg}$
\end{oneparchoices}
\question En la figura se representa el plano del primer piso de un edificio, conformado por cuatro apartamentos de igual forma y medida que comparten un espacio común de forma cuadrada donde se encuentra una escalera.
\begin{center}
\begin{tikzpicture}[scale=1.9]
\shadedraw (0,0) rectangle (6,2);
\draw (3,0)--(3,0.6);
\draw (3,1.4)--(3,2.0);
\draw (3.4,1)--(6,1);
\draw (0,1)--(2.6,1);
\filldraw[fill=white] (2.6,0.6) rectangle node {Escalera} (3.4,1.4);
\node (Ap3) at (1.5,.5) {Apartamento 3};
\node (Ap1) at (1.5,1.5) {Apartamento 1};
\node (Ap2) at (4.5,1.5) {Apartamento 2};
\node (Ap4) at (4.5,.5) {Apartamento 4};
\node (x) at (6.2,1.5) {$x$};
\draw[|-|] (6.1,1)--(6.1,2);
\draw[|-|] (3,2.1)--(6,2.1);
\node (y) at (4.5,2.2) {$y$};
\draw[|-|] (2.6,.5)--(3.4,.5);
\node (x-2) at (3,0.4) {$x-2$};
\end{tikzpicture}
\end{center}
¿Cuál de las siguientes expresiones representa el área total de los 4 apartamentos (área sombreada)?

\begin{oneparchoices}
\choice $4xy-x+2$
\CorrectChoice $4xy-(x-2)^{2}$
\choice $2xy-(x-2)^{2}$
\choice $2xy-x+2$
\end{oneparchoices}
\question
Si $a$, $b$ y $c$ son números primos diferentes y $n=\dfrac{a^{-1}b^{-3}}{a^{-2}b^{-4}c^{-2}}$, es correcto afirmar que
\begin{choices}
\begin{multicols}{2}
 \CorrectChoice $n$ es entero
 \choice $n$ es primo
 \choice $n$ es un racional negativo
 \choice $n$ es irracional
\end{multicols}
\end{choices}
\question Un almacén distribuye computadores de dos marcas (1 y 2). Durante el mes de diciembre uno de sus vendedores vendió 60 computadores. Por cada tres computadores de la marca 1 vendió dos de la marca 2. Si recibió una comisión de \$10\,000 por cada computador de la marca 1 y una comisión de \$20\,000 por cada computador de la marca 2, la comisión total que recibió en el mes de diciembre fue

\begin{oneparchoices}
  \choice \$60\,000
  \choice \$120\,000
  \CorrectChoice \$840\,000
  \choice \$720\,000
\end{oneparchoices}
%Pregunta
%\begin{oneparchoices}
%\choice[1] Nunca
%\end{oneparchoices}
%\answerline
\end{questions}
%cuadro de puntajes
%\begin{center}
%\gradetable[h][pages]
%\end{center}
\end{document}