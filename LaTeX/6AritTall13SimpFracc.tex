\documentclass[10pt,twoside]{article}
\usepackage[utf8]{inputenc}
\usepackage{amsmath}
\usepackage{amsfonts}
\usepackage{amssymb}
\usepackage[spanish,es-noshorthands]{babel}
\usepackage[T1]{fontenc}
\usepackage{lmodern}
\usepackage{graphicx,hyperref}
\usepackage{tikz,pgf}
\usepackage{multicol}
\usepackage{subfig}
\usepackage[papersize={6.5in,8.5in},width=5.5in,height=7in]{geometry}
\usepackage{fancyhdr}
\pagestyle{fancy}
\fancyhead[LE]{\includegraphics[height=12pt]{Images/logo-colegio.png} Aritmética $6^{\circ}$}
\fancyhead[RE]{}
\fancyhead[RO]{\textit{Germ\'an Avenda\~no Ram\'irez, Lic. U.D., M.Sc. U.N.}}
\fancyhead[LO]{}

\author{Germ\'an Avenda\~no Ram\'irez, Lic. U.D., M.Sc. U.N.}
\title{\begin{minipage}{.2\textwidth}
\includegraphics[height=1.75cm]{Images/logo-colegio.png}\end{minipage}
\begin{minipage}{.55\textwidth}
\begin{center}
Taller 13, Simplificación de fracciones  \\
Aritmética $6^{\circ}$
\end{center}
\end{minipage}\hfill
\begin{minipage}{.2\textwidth}
\includegraphics[height=1.75cm]{Images/logo-sed.png} 
\end{minipage}}
\date{}
\begin{document}
\maketitle
Nombre: \hrulefill Curso: \underline{\hspace*{44pt}} Fecha: \underline{\hspace*{2.5cm}}

\section*{Simplificaci\'{o}n de fracciones}
\subsection*{Ejercicios}
Simplifique las siguientes fracciones:
\begin{enumerate}
\begin{multicols}{3}
\item[a.] $\dfrac{2}{4}$
\item[b.] $\dfrac{6}{9}$
\item[c.] $\dfrac{2}{6}$
\item[d.] $\dfrac{4}{16}$
\item[e.] $\dfrac{5}{10}$
\item[f.] $\dfrac{17}{51}$
\end{multicols}
\end{enumerate}
\subsection*{Problemas}
\paragraph*{Problema 1}
Para preparar una torta Juanita necesita 2 tazas de harina. Si cada taza equivale a $\frac{1}{4}$ de kilo y en su casa sólo hay paquetes de $\frac{1}{2}$ kilo de harina, ¿cuántos de éstos ocupará?
\paragraph*{Problema 2}
Pedro, el pastelero, está preparando 6 tortas simultáneamente. Si necesita $\frac{6}{8}$ de kilo de mantequilla y en el local solo hay mantequilla en paquetes de $\frac{1}{4}$ de kilo, ¿cuántos de éstos ocupará? 
\paragraph*{Problema 3}
Pedro el pastelero, necesita $\frac{4}{16}$ de kilo de levadura. Si en la cocina hay medidas de $\frac{1}{2}$, $\frac{1}{3}$, $\frac{1}{4}$ y $\frac{1}{16}$ kilo, ¿cuál es la medida más grande que debe usar para que no le sobre levadura?
\paragraph*{Problema 4}
Para preparar tortas, Ana necesitó $\frac{1}{2}$ litro de mayonesa, ¿a cuántos envases de $\frac{1}{4}$ de litro equivale lo que ocup\'{o}?
\paragraph*{5}
Inés ocupó $\frac{4}{8}$ de kilo de cacao al preparar ponqués para su cumpleaños y el cacao viene en bolsas de $\frac{1}{4}$ kilo, ¿cuántas de éstas ocupó?
\section*{Orden entre fracciones}
\subsection*{Problema resuelto}
Juan y Juana compraron 1 bolsa de dulces cada uno. Después de 2 horas a Juan lo queda $\frac{2}{5}$ de la bolsa y a Juana $\frac{4}{9}$, ¿a quién le queda más?
\subsubsection*{Soluci\'{o}n}
Le quedar\'{a} m\'{a}s a aquel tal que la fracci\'{o}n correspondiente a lo que le queda en la bolsa sea mayor.

Esto puede resumirse en el siguiente esquema.
\paragraph*{Procedimiento}
Para comparar las fracciones $\dfrac{2}{5}$ y $\dfrac{4}{9}$, la amplificamos por 9 y por 5 respectivamente, como los denominadores que se obtienen son iguales bastar\'{i}a comparar los numeradores, es decir comparar $2\cdot 9$ con $4\cdot 5$
\paragraph*{Operaci\'{o}n y resultado}
$2\cdot 9=18$ y $4\cdot 5=20$, como $18<20$ entonces $\dfrac{2}{5}<\dfrac{4}{9}$
\paragraph*{Respuesta}
A Juana le quedan m\'{a}s dulces que a Juan.
\subsection*{Ejercicios}
Complete con los signos < (menor que), > (mayor que) o = (igual que) según corresponda
\begin{itemize}
\begin{multicols}{3}
\item $\dfrac{1}{4}$ \tikz \draw rectangle (.55,.55); $\dfrac{1}{6}$
\item $\dfrac{1}{5}$ \tikz \draw rectangle (.55,.55); $\dfrac{1}{3}$
\item $\dfrac{4}{9}$ \tikz \draw rectangle (.55,.55); $\dfrac{3}{7}$
\item $\dfrac{7}{8}$ \tikz \draw rectangle (.55,.55); $\dfrac{6}{7}$
\item $\dfrac{7}{9}$ \tikz \draw rectangle (.55,.55); $\dfrac{8}{11}$ 
\item $\dfrac{4}{10}$ \tikz \draw rectangle (.55,.55); $\dfrac{3}{7}$
\end{multicols}
\end{itemize}
\subsection*{Problemas}
\paragraph*{Problema 6}
Un curso debe resolver una gu\'{i}a de ejercicios durante la clase de matem\'{a}tica. El grupo de Ana alcanza a resolver $\frac{1}{3}$ de la gu\'{i}a, mientras que el grupo de Martha resuelve $\frac{1}{2}$ de \'{e}sta. ¿Qu\'{e} grupo resolvi\'{o} m\'{a}s ejercicios?
\paragraph*{Problema 7}
Miguel y Roberto deben leer un libro para castellano. Miguel ha le\'{i}do $\frac{5}{8}$ del texto y Roberto $\frac{1}{2}$ ¿A qui\'{e}n le faltan menos p\'{a}ginas por leer?
\paragraph*{Problema 8}
El profesor de deportes debe medir la resistencia de cada estudiante. La prueba consiste en trotar 15 minutos sin detenerse. El estudiante que pare antes de tiempo debe retirarse y obtendr\'{a} una nota de acuerdo al tiempo que corri\'{o}

Si Patricio corri\'{o} $\frac{7}{9}$ del tiempo y Javier $\frac{5}{6}$ ¿qui\'{e}n tiene mejor resistencia?
\paragraph*{Problema 9}
Un d\'{i}a de verano, Sof\'{i}a y Gabriela llegaron a su casa con mucho calor. Cada una prepar\'{o} un litro de jugo de su sabor preferido, manzana y piña respectivamente. Sof\'{i}a bebi\'{o} $\frac{4}{7}$ de su jarro y Gabriela $\frac{2}{3}$ del suyo. ¿De qu\'{e} jugo sobr\'{o} m\'{a}s?
\paragraph*{Problema 10}
Mar\'{i}a y Elena comparten un paquete de galletas durante el recreo. Si Mar\'{i}a come $\frac{3}{8}$ del paquete y Elena $\frac{1}{4}$, ¿qui\'{e}n come m\'{a}s?
\section*{Suma y simplificaci\'{o}n de fracciones homog\'{e}neas}
\subsection*{Problema resuelto}
La señora Marta horneó dos tortas del mismo tamaño. Su hijo Juan comió $\frac{1}{8}$ de la primera y su hija Lucía comió $\frac{3}{8}$ de la segunda. ¿Cuánto comieron entre ambos?
\subsubsection*{Solución}
Entre ambos comieron lo que comió Juan más lo que comió Lucía. Esto se puede resumir en el siguiente esquema
\paragraph*{Procedimiento}
A $\frac{1}{8}$ de la torta que comi\'{o} Juan debemos sumar los $\frac{3}{8}$ que comi\'{o} Luc\'{i}a.
\paragraph*{Operaciones}
$\dfrac{1}{8}+\dfrac{3}{8}=\dfrac{1+3}{8}=\dfrac{4}{8}=\dfrac{1\cdot 4}{2\cdot 4}=\dfrac{1}{2}$
\paragraph*{Respuesta}
Entre ambos comieron $\dfrac{1}{2}$ de torta.
\subsection*{Ejercicios}
\begin{enumerate}
\begin{multicols}{3}
\item $\dfrac{2}{8}+\dfrac{5}{8}$
\item $\dfrac{6}{18}+\dfrac{4}{18}$
\item $\dfrac{34}{62}+\dfrac{23}{62}$
\item $\dfrac{2}{25}+\dfrac{18}{25}$
\item $\dfrac{13}{36}+\dfrac{8}{36}$
\item $\dfrac{38}{95}+\dfrac{18}{95}$
\end{multicols}
\end{enumerate}
\subsection*{Problemas}
\paragraph*{Problema 11}
Doña Carmen necesitaba rellenar dos cojines por lo que compró espuma.

Para rellenar el primero, usó $\frac{2}{5}$ de la espuma y para rellenar el segundo cojín, utilizó $\frac{3}{5}$ de la espuma. ¿Qué fracción del total de espuma usó doña Carmen en rellenar los dos cojines?
\paragraph*{Problema 12}
En una carrera de relevos cuatro amigos compitieron por su colegio. Mario corrió $\frac{1}{8}$ del recorrido total, Ricardo $\frac{1}{8}$, Roberto $\frac{3}{8}$, y, Gonzalo $\frac{1}{8}$. ¿Llegó a la meta este equipo de cuatros atletas?
\paragraph*{Problema 13}
Verónica compró una bandeja de 12 huevos. Usó $\frac{1}{12}$ del total en preparar mayonesa, $\frac{4}{12}$ en hacer una tortilla y $\frac{5}{12}$ para hornear un ponqué. ¿Qué cantidad de huevos ocupó Verónica?
\paragraph*{Problema 14}
Carolina compró un melón para la hora del almuerzo y lo repartió de la siguiente manera: le dió $\frac{2}{5}$ a su hija Daniela, $\frac{2}{5}$ a su hijo Vicente y ella comió $\frac{1}{5}$ ¿Se comieron todo el melón Carolina y sus dos hijos?
\paragraph*{Problema 15}
Para reparar una carretera se arrendaron dos máquinas asfaltadoras, la primera pavimentó $\frac{2}{6}$ del camino y la segund $\frac{3}{6}$ del camino. ¿Qué parte de la carretera se asfaltó?
\end{document}
