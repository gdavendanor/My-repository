\documentclass[10pt,twoside]{article}
\usepackage[utf8]{inputenc}
\usepackage{amsmath}
\usepackage{amsfonts}
\usepackage{amssymb}
\usepackage[spanish,es-noshorthands]{babel}
\usepackage[T1]{fontenc}
\usepackage{lmodern}
\usepackage{graphicx,hyperref}
\usepackage{tikz,pgf}
\usepackage{multicol}
\usepackage{subfig}
\usepackage[papersize={6.5in,8.5in},width=5.5in,height=7in]{geometry}
\usepackage{fancyhdr}
\pagestyle{fancy}
\fancyhead[LE]{\includegraphics[height=12pt]{Images/logo-colegio.png} Aritmética $6^{\circ}$}
\fancyhead[RE]{}
\fancyhead[RO]{\textit{Germ\'an Avenda\~no Ram\'irez, Lic. U.D., M.Sc. U.N.}}
\fancyhead[LO]{}

\author{Germ\'an Avenda\~no Ram\'irez, Lic. U.D., M.Sc. U.N.}
\title{\begin{minipage}{.2\textwidth}
\includegraphics[height=1.75cm]{Images/logo-colegio.png}\end{minipage}
\begin{minipage}{.55\textwidth}
\begin{center}
Taller 13, Simplificación de fracciones  \\
Aritmética $6^{\circ}$
\end{center}
\end{minipage}\hfill
\begin{minipage}{.2\textwidth}
\includegraphics[height=1.75cm]{Images/logo-sed.png} 
\end{minipage}}
\date{}
\begin{document}
\maketitle
Nombre: \hrulefill Curso: \underline{\hspace*{44pt}} Fecha: \underline{\hspace*{2.5cm}}

\section*{Simplificaci\'{o}n de fracciones}
\subsection*{Ejercicios}
Simplifique las siguientes fracciones:
\begin{enumerate}
\begin{multicols}{3}
\item[a.] $\dfrac{2}{4}$
\item[b.] $\dfrac{6}{9}$
\item[c.] $\dfrac{2}{6}$
\item[d.] $\dfrac{4}{16}$
\item[e.] $\dfrac{5}{10}$
\item[f.] $\dfrac{17}{51}$
\end{multicols}
\end{enumerate}
\subsection*{Problemas}
\paragraph*{Problema 1}
Para preparar una torta Juanita necesita 2 tazas de harina. Si cada taza equivale a $\frac{1}{4}$ de kilo y en su casa sólo hay paquetes de $\frac{1}{2}$ kilo de harina, ¿cuántos de éstos ocupará?
\paragraph*{Problema 2}
Pedro, el pastelero, está preparando 6 tortas simultáneamente. Si necesita $\frac{6}{8}$ de kilo de mantequilla y en el local solo hay mantequilla en paquetes de $\frac{1}{4}$ de kilo, ¿cuántos de éstos ocupará? 
\paragraph*{Problema 3}
Pedro el pastelero, necesita $\frac{4}{16}$ de kilo de levadura. Si en la cocina hay medidas de $\frac{1}{2}$, $\frac{1}{3}$, $\frac{1}{4}$ y $\frac{1}{16}$ kilo, ¿cuál es la medida más grande que debe usar para que no le sobre levadura?
\paragraph*{Problema 4}
Para preparar tortas, Ana necesitó $\frac{1}{2}$ litro de mayonesa, ¿a cuántos envases de $\frac{1}{4}$ de litro equivale lo que ocup\'{o}?
\paragraph*{5}
Inés ocupó $\frac{4}{8}$ de kilo de cacao al preparar ponqués para su cumpleaños y el cacao viene en bolsas de $\frac{1}{4}$ kilo, ¿cuántas de éstas ocupó?
\section*{Orden entre fracciones}
\subsection*{Ejercicios}
Complete con los signos < (menor que), > (mayor que) o = (igual) según corresponda
\begin{itemize}
\begin{multicols}{3}
\item $\dfrac{1}{4}$ \tikz \draw (0,-.2)rectangle (.7,.7); $\dfrac{1}{6}$
\item $\dfrac{1}{5}$ \tikz \draw (0,0) rectangle (.7,.7); $\dfrac{1}{3}$
\item $\dfrac{4}{9}\underline{\quad}\dfrac{3}{7}$
\end{multicols}
\end{itemize}
\end{document}
