\documentclass[10pt,twoside]{article}
\usepackage[utf8]{inputenc}
\usepackage{amsmath}
\usepackage{amsfonts}
\usepackage{amssymb}
\usepackage[spanish,es-noshorthands]{babel}
\usepackage[T1]{fontenc}
\usepackage{lmodern}
\usepackage{graphicx,hyperref}
\usepackage{tikz,pgf}
\usepackage{multicol}
\usepackage{subfig}
\usepackage[papersize={6.5in,8.5in},width=5.5in,height=7in]{geometry}
\usepackage{fancyhdr}
\pagestyle{fancy}
\fancyhead[LE]{\includegraphics[height=12pt]{Images/logo-colegio.png} Matemáticas $6^{\circ}$}
\fancyhead[RE]{}
\fancyhead[RO]{\textit{Germ\'an Avenda\~no Ram\'irez, Lic. U.D., M.Sc. U.N.}}
\fancyhead[LO]{}

\author{Germ\'an Avenda\~no Ram\'irez, Lic. U.D., M.Sc. U.N.}
\title{\begin{minipage}{.2\textwidth}
\includegraphics[height=1.75cm]{Images/logo-colegio.png}\end{minipage}
\begin{minipage}{.55\textwidth}
\begin{center}
Taller Nivelación 2014 \\
Matemáticas $6^{\circ}$
\end{center}
\end{minipage}\hfill
\begin{minipage}{.2\textwidth}
\includegraphics[height=1.75cm]{Images/logo-sed.png} 
\end{minipage}}
\date{}
\begin{document}
\maketitle
Nombre: \hrulefill Curso: \underline{603} Fecha: \underline{\hspace*{2.5cm}}
\section*{N\'{u}meros naturales}
\begin{enumerate}
\item Explique la propiedad distributiva de la multiplicaci\'{o}n con respecto a la suma y resuelva de dos maneras los siguientes productos
\begin{enumerate}
\begin{multicols}{3}
\item $17\cdot 38+17\cdot 12$
\item $96\cdot 59+4\cdot 59$
\item $149\cdot 19+52\cdot 19$
\end{multicols}
\end{enumerate}
\item Saca el factor común en las siguientes expresiones:
\begin{enumerate}
\begin{multicols}{3}
\item $120+130+170$
\item $25+35+50$
\item $48-16+72$
\end{multicols}
\end{enumerate}
\item Resuelva y compruebe:
\begin{enumerate}
\begin{multicols}{3}
\item $(3^{4})^{4}$
\item $(8^{2})^{3}$
\item $(9^{3})^{2}$
\end{multicols}
\end{enumerate}
\item Realice las siguientes operaciones:
\begin{enumerate}
\begin{multicols}{2}
\item $3+6\cdot 5-3\cdot 4-2=$
\item $3+(6+4)\cdot 5-4\cdot 6-3+(2\cdot 8)\div 4=$
\item $7\cdot 3+[6+2\cdot (8\div 4+3\cdot 2)-7\cdot 2]+9\div 3=$
\end{multicols}
\end{enumerate}
\end{enumerate}
\end{document}
