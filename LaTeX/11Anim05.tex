\documentclass[letterpaper,11pt,twoside]{article}
\usepackage[utf8]{inputenc}
\usepackage{amsmath,amsfonts,amssymb,amsthm,latexsym}
\usepackage[spanish,es-noshorthands]{babel}
\usepackage[T1]{fontenc}
\usepackage{lmodern}
\usepackage{graphicx,hyperref}
\usepackage{tikz,pgf}
\usepackage{marvosym}
\usepackage{multicol}
\usepackage{fancyhdr}
\usepackage[height=9.5in,width=7in]{geometry}
\usepackage{fancyhdr}
\pagestyle{fancy}
\fancyhead[LE]{matematicas.german@gmail.com}
\fancyhead[RE]{\url{https://www.autistici.org/mathgerman}}
\fancyhead[RO]{\url{https://www.autistici.org/mathgerman}}
\fancyhead[LO]{\Email matematicas.german@gmail.com}

\author{Germ\'an Avenda\~no Ram\'irez~\thanks{Lic. Mat. U.D., M.Sc. U.N.}}
\title{\begin{minipage}{.2\textwidth}
\includegraphics[height=1.75cm]{Images/logo-colegio.png}\end{minipage}
\begin{minipage}{.55\textwidth}
\begin{center}
Animaplano 05\\
Matemáticas $11^{\circ}$
\end{center}
\end{minipage}\hfill
\begin{minipage}{.2\textwidth}
\includegraphics[height=1.75cm]{Images/logo-sed.png} 
\end{minipage}}
\date{}
\thispagestyle{plain}
\begin{document}
\maketitle
Nombre: \hrulefill Curso: \underline{\hspace*{44pt}} Fecha: \underline{\hspace*{2.5cm}}
\begin{multicols}{2}
\textit{Debe resolver el cuestionario siguiente, mostrando los procedimientos hechos para llegar a la respuesta. Respuesta que requiera procedimiento y éste no se explicite, no se tendrá en cuenta.}
\section*{Cuestionario}
Resuelva las preguntas ~\ref{q01}--\ref{q02} de acuerdo a la siguiente información.

Para la construcción de un nuevo parque un arquitecto realizó el siguiente diseño en el que el área de los semicírculos $A_{1}$, $A_{2}$ y $A_{3}$ incluyen atracciones para niños, adultos y adultos mayores respectivamente.
\begin{tikzpicture}
\draw (0,0)--node[above]{$b$} (2,0)--node[left]{$a$}(0,1.5)--node[right]{$c$} cycle;
\draw[dashed] (2,0) arc [start angle=0,end angle=-180,radius=1cm];
\draw[dashed] (0,1.5) arc [start angle=90,end angle=270,radius=0.75cm];
\draw[dashed] (2,0) arc (-36.87:143.13:1.25);
\node at (-0.325,.75){$A_{1}$};
\node at (1,-.5){$A_{3}$};
\node at (1.25,1){$A_{2}$};
\end{tikzpicture}
\begin{enumerate}
\item \label{q01} Si las medidas de los lados $a$ y $b$ son 17 Dm y 15 Dm respectivamente, el área de juegos para niños equivale a
\begin{enumerate}
\item $\left[\pi \left(\dfrac{17-15}{2}\right)^{2}\right]\div 2$ \qquad (\textbf{49})
\item $\left[\pi \sqrt{\left(\dfrac{17^{2}-15^{2}}{2}\right)}\right]\div 2$ \qquad (\textbf{63})
\item $\left[\pi \dfrac{\sqrt{17^{2}-15^{2}}}{2}\right]\div 2$ \qquad (\textbf{58})
\end{enumerate}
\item \label{q02} El área de adultos es equivalente a la suma de las áreas para niños y adultos mayores porque
\begin{enumerate}
\item $A_{1}$ es el área más amplia y puede contener a $A_{2}$ y $A_{3}$ \quad \textbf{(72)}
\item el área $A_{2}$ es $\frac{3}{4}$ de $A_{1}$ y, $A_{3}$ es $\frac{1}{4}$ de $A_{1}$ \quad \textbf{(47)}
\item el cuadrado de $a$ es equivalente a la suma de los cuadrados de $b$ y $c$ \quad \textbf{(69)}
\end{enumerate}
\item El área de juegos para adultos equivale a
\begin{enumerate}
\item $36,125\pi\, Dm^{2}$ \quad \textbf{(89)}
\item $8,5\pi \, Dm^{2}$ \quad \textbf{(100)}
\item $72,25\pi \, Dm^{2}$ \quad \textbf{(93)}
\end{enumerate}
\item El doble del cuadrado del cuarto número primo
\item El cuádruple del noveno número primo
\item El vigésimo número primo
\item Décimo sexto número primo
\item Las edades de Pedro y Juan suman 59 años y Pedro es 7 años mayor que Juan. La edad de Pedro es?
\item La edad de Juan es?
\item El valor de $\displaystyle{\lim_{x\rightarrow 9}\dfrac{x^{2}-x-72}{x-9}}=$
\item La altura de un triángulo rectángulo cuya base mide 24 y su hipotenusa mide 30.
\item El valor de $\displaystyle{\lim_{x\rightarrow 5}\dfrac{x^{2}-3x-10}{x-5}}$
\item La base de un triángulo rectángulo cuya hipotenusa mide 5 y cuya altura mide 3
\item El ancho de un rectángulo cuyo perímetro es 108 y su ancho es 10 unidades menos que el largo
\item El largo del rectángulo del item anterior.

Si $a\circledast b=a+b\cdot (ab)$. Observe el punto \ref{q03}
\item $(5\circledast 4)+(-42)=\textbf{43}$ \label{q03}
\item $(2\circledast 3)+6=$?
\item $(12\circledast 2)+(-41)=$?
\item $(3\circledast 2)+(7\times 2)=$?
\item $(3\circledast 4)-13=$?
\item $(5\circledast 5)-95=$?
\item $(4\circledast 2)+5^{2}+10=$?
\item $(9\circledast 1)+2(23)=$?
\item $(2\circledast 5)+62/2$?
\item $(2\circledast 7)-\sqrt{36}=$?
\item $(7\circledast 3)+3(-2)$?
\item $(5\circledast 2)+(7\times 6)=$?
\item $(3\circledast 3)+8^{2}+\sqrt{9}=$?
\item $(3\circledast 4)+3^{2}=$?
\item $(6\circledast 2)+74/2=$?
\item $(9\circledast 3)-2(16)=$?
\item $(3\circledast 6)-(7\times 9)=$?
\item $(\sqrt{36}\circledast 2)+\sqrt{49}=$?
\end{enumerate}
\end{multicols}
\section*{Animaplano}
\begin{center}
\input{Tikz/Anim_1-100}
\end{center}
\end{document}
