\documentclass[10pt,a4paper]{article}
\usepackage[utf8]{inputenc}
\usepackage[spanish]{babel}
\usepackage{amsmath}
\usepackage{amsfonts}
\usepackage{amssymb}
\usepackage{lmodern}
\author{Darío Yimy Benavides García}
\title{Examen Final de Estadística}
\begin{document}
\begin{enumerate}
\item En una bolsa de tela hemos puesto 10 balotas rojas, 4 balotas negras y 6 balotas verdes, se le ha pedido a alguien que saque tres balotas de la bolsa, una vez que esta persona saca una balota, esta no regresa a la balota nuevamente, de acuerdo a esta información Calcule:
\begin{enumerate}
\item La probabilidad de que la primer balota sea negra
\[p(1N)=\dfrac{4}{10+4+6}=\dfrac{4}{20}=\dfrac{1}{5}=0.2\]
\item Si la primer balota en salir no fue verde cual es la probabilidad de que salga una balota verde en el segundo intento
\[P(2V)=\dfrac{6}{19}\]
\item La probabilidad de que la primer balota en salir sea roja
\[P(1R)=\dfrac{10}{20}=\dfrac{1}{2}=0.5\]
\end{enumerate}
\item Un estudio demostró que en promedio un estudiante de la facultad de educaciòn puede correr 400 mts en 90 segundos, con una desviación estándar de 5 segundos, de acuerdo con estos datos determine.
\end{enumerate}
\end{document}