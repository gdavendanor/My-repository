\documentclass[10pt,a4paper]{article}
\usepackage[utf8]{inputenc}
\usepackage[spanish]{babel}
\usepackage{amsmath}
\usepackage{amsfonts}
\usepackage{amssymb}
\usepackage{lmodern}
\author{Karen}
\title{Examen Final de Estadística II Inferencial\\
Cuarto Período}
\begin{document}
\maketitle
\begin{enumerate}
\item En una bolsa de tela hemos puesto 10 balotas rojas, 4 balotas negras y 6 balotas verdes, se le ha pedido a alguien que saque tres balotas de la bolsa, una vez que esta persona saca una balota, esta no regresa a la balota nuevamente, de acuerdo a esta información Calcule:
\begin{enumerate}
\item La probabilidad de que la primer balota sea negra
\[p(1N)=\dfrac{4}{10+4+6}=\dfrac{4}{20}=\dfrac{1}{5}=0.2\]
\item Si la primer balota en salir no fue verde cual es la probabilidad de que salga una balota verde en el segundo intento
\[P(2V)=\dfrac{6}{19}\]
\item La probabilidad de que la primer balota en salir sea roja
\[P(1R)=\dfrac{10}{20}=\dfrac{1}{2}=0.5\]
\end{enumerate}
\item Un estudio demostró que en promedio un estudiante de la facultad de educación puede correr 400 mts en 90 segundos, con una desviación estándar de 5 segundos, de acuerdo con estos datos determine.
\begin{enumerate}
\item Que proporción de los estudiantes de psicología tardan más de 98 segundos en correr 400 mts. Como 5s es la desviación estándar, luego 8s corresponde a 1.6 desviaciones estándar y según la regla de Chebychev, se tiene:
\[1-\dfrac{1}{k^{2}}=1-\dfrac{1}{(8/5)^{2}}=1-\dfrac{1}{1.6^{2}}=0.609375\approx 61\%\]
Luego, el 61\% corre los 400 mts entre $90-8=82$ segundos y $90+8=98$ segundos; lo cual quiere decir que el 39\% recorre los 400 m gastando menos de 82 segundos o más de 98 segundos. Si la distribución es normal y simétrica, se puede asumir que el 19.5\% de los estudiantes de sicología, recorre los 400m gastando más de 98 segundos
\item Que proporción de los estudiantes tarda menos de 77 segundos en correr 400 mts.
\[1-\dfrac{1}{(13/5)^{2}}=1-\dfrac{1}{2.6^{2}}=0.852071005\approx 0.85 \]
Es decir el 85\% de los estudiantes recorre los 400m entre 77 y 103 segundos. Por tanto el $100\%-85\%=\%15$ recorre los 400m gastando menos de 77 segundos y más de 103 segundos. Así que aproximadamente el 7.5\% de los estudiantes recorre los 400 m en menos de 77 segundos.
\end{enumerate}
\end{enumerate}
\end{document}