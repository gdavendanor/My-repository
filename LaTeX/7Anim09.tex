\documentclass[10pt,twoside]{article}
\usepackage[utf8]{inputenc}
\usepackage{amsmath,amsfonts,amssymb}
\usepackage[spanish,es-noshorthands]{babel}
\usepackage[T1]{fontenc}
\usepackage{lmodern}
\usepackage{graphicx,hyperref}
\usepackage{tikz,pgf}
\usepackage{marvosym}
\usepackage{printsudoku}
\usepackage{multicol}
\usepackage[papersize={6.5in,8.5in},left=1cm, right=1cm, top=1.5cm, bottom=1.7cm]{geometry}
\usepackage{fancyhdr}
\pagestyle{fancy}
\fancyhead[LE]{Colegio Arborizadora Baja}
\fancyhead[RE]{PEI:``Hacia una cultura para el desarrollo sostenible''}
\fancyfoot[RO]{\Email gavendanor@colarborizadorabaja.edu.co}
\fancyhead[LO]{\url{www.autistici.org/mathgerman}}
\fancyfoot[RE]{\Email cedarborizadoraba19@redp.edu.co}
\fancyfoot[LE]{Calle 59I \#44A - 02 \Telefon 7313994 - 7313995}
\fancyhead[RO]{Nit 830024976-8, Código DANE 11100103084-8}

\author{Germ\'an Avenda\~no Ram\'irez~\thanks{Lic. Mat. U.D., M.Sc. U.N.}}
\title{\begin{minipage}{.2\textwidth}
\includegraphics[height=1.75cm]{Images/logo-colegio.png}\end{minipage}
\begin{minipage}{.55\textwidth}
\begin{center}
Animaplano 09 \\
Matemáticas $7^{\circ}$
\end{center}
\end{minipage}\hfill
\begin{minipage}{.2\textwidth}
\includegraphics[height=1.75cm]{Images/logo-sed.png} 
\end{minipage}}
\date{}
\usetikzlibrary{arrows}
\newcommand{\degre}{\ensuremath{^\circ}}
\definecolor{qqwuqq}{rgb}{0,0.39,0}
\begin{document}
\maketitle
Nombre: \hrulefill Curso: \underline{\hspace{1cm}}  Fecha: \underline{\hspace{2cm}}
\subsection*{Cuestionario}
Conteste las preguntas del cuestionario, haciendo los prodecimientos necesarios. Las respuestas se ubican en el plano dispuesto para ello. También como reto se ha puesto un sudoku fácil.
\begin{enumerate}
  \item Sea $ A=\{2,17,9\} $ $ B=\{2,5,8,17\} $. El producto de los elementos de $ A\cap B= $\\ \textbf{Nota:} $A\cap B$~\footnote{\emph{Recuerde que $A\cap B$ está formado por los elementos que pertenecen a $A$ y $B$ simultáneamente}}
\begin{multicols}{2}
 \item El triple del número 11:
  \item Halle $ (9^5\div9^3)+2^1= $
      \end{multicols} 
  \item Halle $ (10^6\div10^4)-4^1= $
  \item Halle $ 10^2 $, menos $ (8\times2!)$ \textit{Recuerde como se define $n!$}~\footnote{Recuerde que el factorial de un número natural $ n! $ es $ n!=n\times(n-1)\times(n-2)\times\ldots\times3\times2\times1 $, es decir el número por todos sus antecesores.} 
  \item $ 1000-947= $
  \item Si $A+B=37$, entonces $(A+10)+(B+27)=$
\begin{multicols}{2}
  \item El 43\% de 200
  \item 1 centena menos 1 docena 
\end{multicols}
  \item Sume al triple del número 9, el cuádruple del número 10:
  \begin{multicols}{2}
  \item[11] $ (1000\div10)-(7\times3)= $
  \item[12] $ (-7\times-7)+(5\times2)= $
  \item[13] $ (-5\times2)+48= $
  \item[14] $ (-1\times-2\times-5)+36= $
  \item[15] El triple del número 13
  \item[16] La tercera parte de 180
  \item[17] Si $p+14=43$, entonces $p=$
  \item[18] Sume 4 veces, el número 6
  \end{multicols}
  \item[19] Si el perímetro~\footnote{El perímetro de una figura es la suma de sus lados} de un cuadrado mide 16. ¿Cuál es su área?
  \item[20] Halle $ 4!-\sqrt{25}= $
  \item[21] Sea $ A=\{15,5,9\} $ $ B=\{7,15,9,5\} $. Halle la raíz cuadrada de la suma de los elementos de $ A\cup B= $
  \item[22] $A-B$ en grados
  \usetikzlibrary{arrows}
\definecolor{qqwuqq}{rgb}{0,0.39,0}
\begin{tikzpicture}[line cap=round,line join=round,>=triangle 45,x=1.0cm,y=1.0cm,scale=1]
\draw[->,color=black] (-1.02,0) -- (1.46,0);
\filldraw[color=qqwuqq,fill=qqwuqq,fill opacity=0.1] (-.35,0) rectangle (0,.35);
\foreach \x in {-1,1}
\draw[shift={(\x,0)},color=black] (0pt,-2pt);
\draw[->,color=black] (0,-0.44) -- (0,1.54);
\foreach \y in {,1}
\draw[shift={(0,\y)},color=black] (2pt,0pt) -- (-2pt,0pt);
\clip(-1.02,-0.44) rectangle (1.46,1.54);
\draw [shift={(0,0)},color=qqwuqq,fill=qqwuqq,fill opacity=0.1] (0,0) -- (0:0.8) arc (0:51:0.8) -- cycle;
\draw [shift={(0,0)},color=qqwuqq,fill=qqwuqq,fill opacity=0.1] (0,0) -- (51:0.6) arc (51:90:0.6) -- cycle;
\draw [domain=-1.02:1.46] plot(\x,{(-0--1.23*\x)/1});
\draw (-0.74,0.64) node[anchor=north west] {$A$};
\draw (0.06,0.94) node[anchor=north west] {$B$};
\begin{scriptsize}
\draw (1.02,0.44) node {$51\textrm{\degre}$};
\end{scriptsize}
\end{tikzpicture}
    \begin{multicols}{2}
  \item[23] 1 siglo menos 18 años
  \item[24] 1/2 siglo + 2 años
  \item[25] La tercera parte de 99
  \item[26] 1 decena, más 1 docena
    \end{multicols}
  \item[27] El número de vértices de un polígno de 12 lados
  \item[28] El número de lados de un polígono con 13 vértices
\begin{multicols}{2}
  \item[29] Halle $ 96\div4 $
  \item[30] En años, 6 lustros más 4 años
\end{multicols}
  \item[31] Si $ m+28=95 $, entonces $ m= $
\end{enumerate}
  \begin{minipage}{.5\textwidth}
\begin{center}
\begin{tikzpicture}[scale=.7]
 \fill (1,0) node[above]{1} circle (0.2ex);
 \fill (2,0) node[above]{2} circle (0.2ex);
 \fill (3,0) node[above]{3} circle (0.2ex);
 \fill (4,0) node[above]{4} circle (0.2ex);
 \fill (5,0) node[above]{5} circle (0.2ex);
 \fill (6,0) node[above]{6} circle (0.2ex);
 \fill (7,0) node[above]{7} circle (0.2ex);
 \fill (8,0) node[above]{8} circle (0.2ex);
 \fill (9,0) node[above]{9} circle (0.2ex);
 \fill (10,0) node[above]{10} circle (0.2ex);
 \fill (1,-1) node[left]{11} circle (0.2ex);
 \fill (2,-1) circle (0.2ex);
 \fill (3,-1) circle (0.2ex);
 \fill (4,-1) circle (0.2ex);
 \fill (5,-1) circle (0.2ex);
 \fill (6,-1) circle (0.2ex);
 \fill (7,-1) circle (0.2ex);
 \fill (8,-1) circle (0.2ex);
 \fill (9,-1) circle (0.2ex);
 \fill (10,-1) circle (0.2ex);
 \fill (1,-2) node[left]{21} circle (0.2ex);
 \fill (2,-2) circle (0.2ex);
 \fill (3,-2) circle (0.2ex);
 \fill (4,-2) circle (0.2ex);
 \fill (5,-2) circle (0.2ex);
 \fill (6,-2) circle (0.2ex);
 \fill (7,-2) circle (0.2ex);
 \fill (8,-2) circle (0.2ex);
 \fill (9,-2) circle (0.2ex);
 \fill (10,-2) circle (0.2ex);
 \fill (1,-3) node[left]{31} circle (0.2ex);
 \fill (2,-3) circle (0.2ex);
 \fill (3,-3) circle (0.2ex);
 \fill (4,-3) circle (0.2ex);
 \fill (5,-3) circle (0.2ex);
 \fill (6,-3) circle (0.2ex);
 \fill (7,-3) circle (0.2ex);
 \fill (8,-3) circle (0.2ex);
 \fill (9,-3) circle (0.2ex);
 \fill (10,-3) circle (0.2ex);
 \fill (1,-4) node[left]{41} circle (0.2ex);
 \fill (2,-4) circle (0.2ex);
 \fill (3,-4) circle (0.2ex);
 \fill (4,-4) circle (0.2ex);
 \fill (5,-4) circle (0.2ex);
 \fill (6,-4) circle (0.2ex);
 \fill (7,-4) circle (0.2ex);
 \fill (8,-4) circle (0.2ex);
 \fill (9,-4) circle (0.2ex);
 \fill (10,-4) node[right]{50} circle (0.2ex);
 \fill (1,-5) node[left]{51} circle (0.2ex);
 \fill (2,-5) circle (0.2ex);
 \fill (3,-5) circle (0.2ex);
 \fill (4,-5) circle (0.2ex);
 \fill (5,-5) circle (0.2ex);
 \fill (6,-5) circle (0.2ex);
 \fill (7,-5) circle (0.2ex);
 \fill (8,-5) circle (0.2ex);
 \fill (9,-5) circle (0.2ex);
 \fill (10,-5) circle (0.2ex);
 \fill (1,-6) node[left]{61} circle (0.2ex);
 \fill (2,-6) circle (0.2ex);
 \fill (3,-6) circle (0.2ex);
 \fill (4,-6) circle (0.2ex);
 \fill (5,-6) circle (0.2ex);
 \fill (6,-6) circle (0.2ex);
 \fill (7,-6) circle (0.2ex);
 \fill (8,-6) circle (0.2ex);
 \fill (9,-6) circle (0.2ex);
 \fill (10,-6) circle (0.2ex);
 \fill (1,-7) node[left]{71} circle (0.2ex);
 \fill (2,-7) circle (0.2ex);
 \fill (3,-7) circle (0.2ex);
 \fill (4,-7) circle (0.2ex);
 \fill (5,-7) circle (0.2ex);
 \fill (6,-7) circle (0.2ex);
 \fill (7,-7) circle (0.2ex);
 \fill (8,-7) circle (0.2ex);
 \fill (9,-7) circle (0.2ex);
 \fill (10,-7) circle (0.2ex);
 \fill (1,-8) node[left]{81} circle (0.2ex);
 \fill (2,-8) circle (0.2ex);
 \fill (3,-8) circle (0.2ex);
 \fill (4,-8) circle (0.2ex);
 \fill (5,-8) circle (0.2ex);
 \fill (6,-8) circle (0.2ex);
 \fill (7,-8) circle (0.2ex);
 \fill (8,-8) circle (0.2ex);
 \fill (9,-8) circle (0.2ex);
 \fill (10,-8) circle (0.2ex);
 \fill (1,-9) node[left]{91} circle (0.2ex);
 \fill (2,-9) circle (0.2ex);
 \fill (3,-9) circle (0.2ex);
 \fill (4,-9) circle (0.2ex);
 \fill (5,-9) circle (0.2ex);
 \fill (6,-9) circle (0.2ex);
 \fill (7,-9) circle (0.2ex);
 \fill (8,-9) circle (0.2ex);
 \fill (9,-9) circle (0.2ex);
 \fill (10,-9) node[right]{100} circle (0.2ex);
\end{tikzpicture}
\end{center}
\end{minipage}\hfill
\cluefont{\Large}
\cellsize{1.75\baselineskip}
\begin{minipage}{0.45\linewidth}\begin{center}
SE5 (easy) \\
\sudoku{se5.sud}
\end{center}\end{minipage}
\end{document}
