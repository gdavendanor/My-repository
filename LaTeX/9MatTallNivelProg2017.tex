\documentclass[twoside,letterpaper]{article}
\usepackage[utf8]{inputenc}
\usepackage{amsmath,amsfonts,amssymb,amsthm,latexsym}
\usepackage[spanish,es-noshorthands]{babel}
\usepackage[T1]{fontenc}
\usepackage{lmodern}
\usepackage{graphicx,hyperref}
\usepackage{tikz,pgf}
\usepackage{marvosym}
\usepackage{multicol}
\usepackage{fancyhdr}
\usepackage[left=.75cm,right=.75cm,top=1.5cm,bottom=1.25cm]{geometry}
\usepackage{fancyhdr}
\pagestyle{fancy}
\fancyhead[LE]{Colegio Arborizadora Baja}
\fancyhead[RE]{PEI:``Hacia una cultura para el desarrollo sostenible''}
\fancyfoot[RO]{\Email iedabgerman@autistici.org}
\fancyhead[LO]{\url{www.autistici.org/mathgerman}}
\fancyfoot[RE]{\Email cedarborizadoraba19@redp.edu.co}
\fancyfoot[LE]{Calle 59I \#44A - 02 \Telefon 7313994 - 7313995}
\fancyhead[RO]{Nit 830024976-8, Código DANE 11100103084-8}

\author{Germ\'an Avenda\~no Ram\'irez~\thanks{Lic. Mat. U.D., M.Sc. U.N.}}
\title{\begin{minipage}{.2\textwidth}
\includegraphics[height=1.75cm]{Images/logo-colegio.png}\end{minipage}
\begin{minipage}{.55\textwidth}
\begin{center}
Taller Nivelación - Progresiones\\
Matemáticas $9^{\circ}$
\end{center}
\end{minipage}\hfill
\begin{minipage}{.2\textwidth}
\includegraphics[height=1.75cm]{Images/logo-sed.png} 
\end{minipage}}
\date{}
\thispagestyle{plain}
\begin{document}
\maketitle
Nombre: \hrulefill Curso: \underline{\hspace*{44pt}} Fecha: \underline{\hspace*{2.5cm}}
\begin{multicols}{2}
\section*{Progresión aritmética}
\begin{enumerate}
\item Los tres primeros términos de una progresión aritmética son 8, 13, 18. Encuentra el décimo término de la progresión.
\item Para cubrir el valor de un aparato eléctrico, Aurora efectuará 12 pagos. El primero es de \$275\,000 y el segundo de \$255\,000. Si los pagos forman una progresión aritmética. ¿Cuál es el monto del último pago?
\item Para promover la asistencia a un espectáculo se anuncia que de los 500 boletos disponibles, cada uno con un costo de \$2000 se hará un descuento de: \$100 al comprador número 50, \$150 al comprador 100, \$200 al comprador 150 y así sucesivamente. ¿Si todos los boletos fueron vendidos, entonces cuánto dejó de recibir el empresario por efecto de los descuentos? ¿Cuánto pagó por su boleto el último comprador?
\section*{Progresión geométrica}
\item Los números $\frac{1}{3}$ y 9 son los dos primeros términos de una progresión geométrica. Encuentra el valor del quinto término de la progresión.
\item Las amibas son protozoarios unicelulares. Son capaces de vivir como parásitos y como organismos de vida libre. En México se detectaron por primera vez en 1611. Su nombre proviene del griego amoibe que significa cambio. Son sumamente resistentes, sobreviven a temperaturas de congelación y a soluciones con cloro. Su tamaño medio es de 0.025 milímetros y se reproducen por fisión.

Cierto tipo de amibas se reproduce por fisión cada 20 minutos. Por cada amiba encontrada en el intestino en determinado momento, ¿cuántas habrá 2 horas después? Elabore una tabla.
\end{enumerate}
\end{multicols}
\end{document}
