\documentclass[twoside]{article}
\usepackage[utf8]{inputenc}
\usepackage{amsmath,amsfonts,amssymb,amsthm,latexsym}
\usepackage[spanish,es-noshorthands]{babel}
\usepackage[T1]{fontenc}
\usepackage{lmodern}
\usepackage{graphicx,hyperref}
\usepackage{tikz,pgf}
\usepackage{marvosym}
\usepackage{multicol}
\usepackage{printsudoku}
\newcommand*{\Lpack}[1]{\textsf{#1}}
\usepackage{fancyhdr}
\usepackage[papersize={5.5in,8.5in},left=.75cm,right=.75cm,top=1.5cm,bottom=1.25cm]{geometry}
\usepackage{fancyhdr}
\pagestyle{fancy}
\fancyhead[LE]{\Email iedabgerman@autistici.org}
\fancyhead[RE]{}
\fancyhead[RO]{\url{https://www.autistici.org/mathgerman}}
\fancyhead[LO]{}

\author{Germ\'an Avenda\~no Ram\'irez~\thanks{Lic. Mat. U.D., M.Sc. U.N.}}
\title{\begin{minipage}{.2\textwidth}
\includegraphics[height=1.75cm]{Images/logo-colegio.png}\end{minipage}
\begin{minipage}{.55\textwidth}
\begin{center}
Animaplano 3\\
Matemáticas $6^{\circ}$
\end{center}
\end{minipage}\hfill
\begin{minipage}{.2\textwidth}
\includegraphics[height=1.75cm]{Images/logo-sed.png} 
\end{minipage}}
\date{}
\thispagestyle{plain}
\begin{document}
\maketitle
Nombre: \hrulefill Curso: \underline{\hspace*{44pt}} Fecha: \underline{\hspace*{2.5cm}}\\
\section*{Cuestionario}
Resuelva al frente de cada ítem en su orden cada operación y luego con los resultados unir los puntos para formar una figura.
\begin{enumerate}
\item $7\times 9+3=$
\item $9\times 9+5=$
\item $9\times 8+10=$
\item $8\times 8-2=$
\item $(20+3)\times 3=$
\item $(30+3)\times 3=$
\item $9\times 9+10=$
\item $(6\times 9)-3=$
\item $8\times 8-4=$
\item $3\times 9+1=$
\item $2\times 9+9=$
\item $2\times 9-1=$
\item $2\times 8=$
\item $3\times 8+2=$
\item $3\times 9-4=$
\item $5\times 9+6=$
\end{enumerate}
\section*{Animaplano}
\begin{tikzpicture}[scale=.9]
 \fill (1,0) node[above]{1} circle (0.2ex);
 \fill (2,0) node[above]{2} circle (0.2ex);
 \fill (3,0) node[above]{3} circle (0.2ex);
 \fill (4,0) node[above]{4} circle (0.2ex);
 \fill (5,0) node[above]{5} circle (0.2ex);
 \fill (6,0) node[above]{6} circle (0.2ex);
 \fill (7,0) node[above]{7} circle (0.2ex);
 \fill (8,0) node[above]{8} circle (0.2ex);
 \fill (9,0) node[above]{9} circle (0.2ex);
 \fill (10,0) node[above]{10} circle (0.2ex);
 \fill (1,-1) node[left]{11} circle (0.2ex);
 \fill (2,-1) circle (0.2ex);
 \fill (3,-1) circle (0.2ex);
 \fill (4,-1) circle (0.2ex);
 \fill (5,-1) circle (0.2ex);
 \fill (6,-1) circle (0.2ex);
 \fill (7,-1) circle (0.2ex);
 \fill (8,-1) circle (0.2ex);
 \fill (9,-1) circle (0.2ex);
 \fill (10,-1) circle (0.2ex);
 \fill (1,-2) node[left]{21} circle (0.2ex);
 \fill (2,-2) circle (0.2ex);
 \fill (3,-2) circle (0.2ex);
 \fill (4,-2) circle (0.2ex);
 \fill (5,-2) circle (0.2ex);
 \fill (6,-2) circle (0.2ex);
 \fill (7,-2) circle (0.2ex);
 \fill (8,-2) circle (0.2ex);
 \fill (9,-2) circle (0.2ex);
 \fill (10,-2) circle (0.2ex);
 \fill (1,-3) node[left]{31} circle (0.2ex);
 \fill (2,-3) circle (0.2ex);
 \fill (3,-3) circle (0.2ex);
 \fill (4,-3) circle (0.2ex);
 \fill (5,-3) circle (0.2ex);
 \fill (6,-3) circle (0.2ex);
 \fill (7,-3) circle (0.2ex);
 \fill (8,-3) circle (0.2ex);
 \fill (9,-3) circle (0.2ex);
 \fill (10,-3) circle (0.2ex);
 \fill (1,-4) node[left]{41} circle (0.2ex);
 \fill (2,-4) circle (0.2ex);
 \fill (3,-4) circle (0.2ex);
 \fill (4,-4) circle (0.2ex);
 \fill (5,-4) circle (0.2ex);
 \fill (6,-4) circle (0.2ex);
 \fill (7,-4) circle (0.2ex);
 \fill (8,-4) circle (0.2ex);
 \fill (9,-4) circle (0.2ex);
 \fill (10,-4) node[right]{50} circle (0.2ex);
 \fill (1,-5) node[left]{51} circle (0.2ex);
 \fill (2,-5) circle (0.2ex);
 \fill (3,-5) circle (0.2ex);
 \fill (4,-5) circle (0.2ex);
 \fill (5,-5) circle (0.2ex);
 \fill (6,-5) circle (0.2ex);
 \fill (7,-5) circle (0.2ex);
 \fill (8,-5) circle (0.2ex);
 \fill (9,-5) circle (0.2ex);
 \fill (10,-5) circle (0.2ex);
 \fill (1,-6) node[left]{61} circle (0.2ex);
 \fill (2,-6) circle (0.2ex);
 \fill (3,-6) circle (0.2ex);
 \fill (4,-6) circle (0.2ex);
 \fill (5,-6) circle (0.2ex);
 \fill (6,-6) circle (0.2ex);
 \fill (7,-6) circle (0.2ex);
 \fill (8,-6) circle (0.2ex);
 \fill (9,-6) circle (0.2ex);
 \fill (10,-6) circle (0.2ex);
 \fill (1,-7) node[left]{71} circle (0.2ex);
 \fill (2,-7) circle (0.2ex);
 \fill (3,-7) circle (0.2ex);
 \fill (4,-7) circle (0.2ex);
 \fill (5,-7) circle (0.2ex);
 \fill (6,-7) circle (0.2ex);
 \fill (7,-7) circle (0.2ex);
 \fill (8,-7) circle (0.2ex);
 \fill (9,-7) circle (0.2ex);
 \fill (10,-7) circle (0.2ex);
 \fill (1,-8) node[left]{81} circle (0.2ex);
 \fill (2,-8) circle (0.2ex);
 \fill (3,-8) circle (0.2ex);
 \fill (4,-8) circle (0.2ex);
 \fill (5,-8) circle (0.2ex);
 \fill (6,-8) circle (0.2ex);
 \fill (7,-8) circle (0.2ex);
 \fill (8,-8) circle (0.2ex);
 \fill (9,-8) circle (0.2ex);
 \fill (10,-8) circle (0.2ex);
 \fill (1,-9) node[left]{91} circle (0.2ex);
 \fill (2,-9) circle (0.2ex);
 \fill (3,-9) circle (0.2ex);
 \fill (4,-9) circle (0.2ex);
 \fill (5,-9) circle (0.2ex);
 \fill (6,-9) circle (0.2ex);
 \fill (7,-9) circle (0.2ex);
 \fill (8,-9) circle (0.2ex);
 \fill (9,-9) circle (0.2ex);
 \fill (10,-9) node[right]{100} circle (0.2ex);
%\draw (6,-6) -- (6,-8)--(2,-8)--(2,-6)--(9,-6)--(9,-9)--(1,-9)--(1,-5)--(10,-5)--(8,-2)--(7,-2)--(7,-1)--(6,-1)--(6,-2)--(3,-2)--(1,-5);
\end{tikzpicture}
\section*{Sudoku}
Resuelva el siguiente sudoku
\begin{center}
\renewcommand*{\puzzlefile}{se5.sud}
\writepuzzle%
{.2...5863}{56.2.3.9.}{.3...7251}%
{..975....}{..6..47.9}{.7..286..}%
{6.58...7.}{8....1..6}{3.7.6..4.}%
[SSBL easy 5]
\vfill
\cluefont{\Large}
\cellsize{1.5\baselineskip}
\begin{minipage}{0.4\linewidth}\begin{center}
SE5 (easy) \\
\sudoku{se5.sud}
\end{center}\end{minipage}
\end{center}
\end{document}
