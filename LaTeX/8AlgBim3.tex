\documentclass[letterpaper,fleqn]{article}
\usepackage[spanish,es-noshorthands]{babel}
\usepackage[utf8]{inputenc} 
\usepackage[height=9.5in,width=7in]{geometry}
\usepackage{mathexam}
\usepackage{amsmath}
\usepackage{graphicx}

\ExamClass{\includegraphics[height=16pt]{Images/logo-sed.png} Álgebra $8^{\circ}$}
\ExamName{Prueba Bimestral III}
\ExamHead{\includegraphics[height=16pt]{Images/logo-colegio.png} IEDAB}
\newcommand{\LineaNombre}{%
\par
\vspace{\baselineskip}
Nombre:\hrulefill \; Curso: \underline{\hspace*{48pt}} \; Form: \underline{A} \; Fecha: \underline{\hspace*{2.5cm}} \relax
\par}
\let\ds\displaystyle

\begin{document}
\ExamInstrBox{
Respuesta sin justificar mediante procedimiento no será tenida en cuenta en la calificación. Escriba sus respuestas en el espacio indicado. Tiene 70 minutos para contestar esta prueba.}
\LineaNombre
\begin{enumerate}
 \item Exprese usando lenguaje algebraico las siguientes expresiones:
 \begin{enumerate}
  \item El cuadrado de la diferencia de dos números \answer*[0pt]{Resp:}
  \item La suma del triple de un número y el cuádruple de otro \answer*[0pt]{Resp:}
  \item La raíz cuadrada del producto de dos números\answer*[0pt]{Resp:}
  \item La mitad de la suma de dos números \answer*[0pt]{Resp:}
  \item El cubo de la suma del doble de un número y el triple de otro \answer*[0pt]{Resp}
 \end{enumerate}
\item Exprese en lenguaje verbal las siguientes expresiones algebraicas
\begin{enumerate}
 \item $(a-b)^{2}$:\answer*[0pt]{Resp:}
 \item $3(x+y)$: \answer*[0pt]{Resp:}
 \item $\dfrac{x-y}{3}$:\answer*[0pt]{Resp:}
 \item $\sqrt{2x+5y}$:\answer*[0pt]{Resp:}
\end{enumerate}
\item Resuelva los siguientes problemas, planteando la ecuación correspondiente y solucionándola:
\begin{enumerate}
 \item  Entre Luis y Antonio reúnen 840 euros. Sabiendo que Antonio tiene 125 euros más que Luis, calcular los euros que tiene cada uno. \answer*{Resp:}
  \item Repartir 300 euros entre tres personas de modo que la segunda reciba 16 euros más que la primera y la tercera 28 euros más que la segunda. \answer*{Resp:}
\end{enumerate}
\item Dados los polinomios $P=3x^{2}-4x+5$ y $Q=2x^{2}+6x-8$ halle:
\begin{enumerate}
 \item $P+Q=$\answer*{Resp:}
 \item $P-Q=$\answer*{Resp:}
 \item $4P$\answer*{Resp}
 \item $P\cdot Q=$\answer*{Resp}
\end{enumerate}
\item Desarrolle usando la propiedad distributiva de la multiplicación respecto a la suma o cuando pueda use los productos notables
\begin{enumerate}
 \item $8a^{3}b^{4}(3ab-2ab^{2}+4a^{2}b^{2})$
 \item $(6x+7)(3x-10)$
 \item $(3t+7)^{2}$
 \item $(4x-2)^{3}$
 \item $(5x-2a)(5x+2a)$
\end{enumerate}
 \end{enumerate}

\end{document}
