\documentclass[letterpaper,fleqn]{article}
\usepackage[spanish,es-noshorthands]{babel}
\usepackage[utf8]{inputenc} 
\usepackage[height=9.5in,width=7in]{geometry}
\usepackage{mathexam}
\usepackage{amsmath}
\usepackage{graphicx}

\ExamClass{\includegraphics[height=16pt]{Images/logo-sed.png} Álgebra $8^{\circ}$}
\ExamName{Prueba Bimestral III}
\ExamHead{\includegraphics[height=16pt]{Images/logo-colegio.png} IEDAB}
\newcommand{\LineaNombre}{%
\par
\vspace{\baselineskip}
Nombre:\hrulefill \; Curso: \underline{\hspace*{48pt}} \; Fecha: \underline{\hspace*{2.5cm}} \relax
\par}
\let\ds\displaystyle

\begin{document}
\ExamInstrBox{
Respuesta sin justificar mediante procedimiento no será tenida en cuenta en la calificación. Escriba sus respuestas en el espacio indicado. Tiene 70 minutos para contestar esta prueba.}
\LineaNombre
\begin{enumerate}
 \item Exprese usando lenguaje algebraico las siguientes expresiones:
 \begin{enumerate}
  \item El cuadrado de la diferencia de dos números \answer*[0pt]
  \item La suma del triple de un número y el cuádruple de otro \answer*[0pt]
  \item La raíz cuadrada del producto de dos números\answer*[0pt]
  \item La mitad de la suma de dos números \answer*[0pt]
  \item El cubo de la suma del doble de un número y el triple de otro \answer*[0pt]
 \end{enumerate}
\item Exprese en lenguaje verbal las siguientes expresiones algebraicas
\begin{enumerate}
 \item $(a-b)^{2}$:
 \item $3(x+y)$: 
 \item $\dfrac{x-y}{3}$:
 \item $\sqrt{2x+5y}$:
\end{enumerate}

 \end{enumerate}

\end{document}
