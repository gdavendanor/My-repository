\documentclass[letterpaper,11pt,twoside]{article}
\usepackage[utf8]{inputenc}
\usepackage{amsmath,amsfonts,amssymb,amsthm,latexsym}
\usepackage[spanish,es-noshorthands]{babel}
\usepackage[T1]{fontenc}
\usepackage{lmodern}
\usepackage{graphicx,hyperref}
\usepackage{tikz,pgf}
\usepackage{multicol}
\usepackage{fancyhdr}
\usepackage{marvosym}
\usepackage[height=9.5in,width=7in]{geometry}
\usepackage{fancyhdr}
\pagestyle{fancy}
\fancyhead[LE]{Colegio Arborizadora Baja}
\fancyhead[RE]{PEI:``Hacia una cultura para el desarrollo sostenible''}
\fancyfoot[RO]{\Email iedabgerman@autistici.org}
\fancyhead[LO]{\url{www.autistici.org/mathgerman}}
\fancyfoot[RE]{\Email cedarborizadoraba19@redp.edu.co}
\fancyfoot[LE]{Calle 59I \#44A - 02 \Telefon 7313994 - 7313995}
\fancyhead[RO]{Nit 830024976-8, Código DANE 11100103084-8}

\author{Germ\'an Avenda\~no Ram\'irez~\thanks{Lic. Mat. U.D., M.Sc. U.N.}}
\title{\begin{minipage}{.2\textwidth}
\includegraphics[height=1.75cm]{Images/logo-colegio.png}\end{minipage}
\begin{minipage}{.55\textwidth}
\begin{center}
Recomendaciones 4 período\\
Matemáticas $11^{\circ}$
\end{center}
\end{minipage}\hfill
\begin{minipage}{.2\textwidth}
\includegraphics[height=1.75cm]{Images/logo-sed.png} 
\end{minipage}}
\date{}
\thispagestyle{plain}
\begin{document}
\maketitle
\section*{Temas básicos de estudio}
Las nivelaciones finales se harán sobre los temas fundamentales desarrollados durante el presente año lectivo, los cuales se enuncian a continuación.
\subsection*{Cálculo}
\begin{itemize}
\item Números reales con sus operaciones y propiedades de estructura y orden.
\item Relaciones, funciones y sus gráficas
\item Sucesiones, progresiones y series
\begin{itemize}
\item Progresión aritmética
\item Progresión geométrica
\end{itemize}
\item Límites de sucesiones infinitas
\item Limites de funciones
\end{itemize}
\subsection*{Probabilidad}
\begin{itemize}
\item Principios fundamentales del conteo
\item Permutaciones y combinaciones
\item Probabilida elemental
\item Probabilidad condicional
\end{itemize}
\section*{Fundamentación teórica}
En la página de Khan Academy \url{https://es.khanacademy.org/} se encuentran bastantes recursos para reforzar los temas vistos durante este año lectivo. Por ejemplo, para estudiar permutaciones y combinaciones se puede usar el siguiente enlace:
\url{https://es.khanacademy.org/math/probability/probability-and-combinatorics-topic}

Las guías y talleres desarrollados durante el presente año lectivo se constituyen en importantes herramientas de estudio.
\section*{Actividades a desarrollar}
Para tener éxito en la nivelación final, deberá desarrollar las siguientes actividades:
\begin{itemize}
\item Si tiene talleres propuestos durante este año lectivo sin hacer o talleres incompletos, deberá desarrollarlos o completarlos.
\item Corregir todas las evaluaciones hechas durante el presente año lectivo
\item Se sugiere desarrollar las siguientes actividades:
\begin{itemize}
\item Algunos ejercicios de repaso del libro "Precálculo: Matemáticas para el cálculo"~\cite{stewart} de los capítulos 1, 2, 3, 11 y 12. Ésta libro se encuentra en la red de bibliotecas públicas de Bogotá.
\item Algunos ejercicios de repaso del libro "Estadística elemental: lo esencial" del capítulo 5, página 258~\cite{esencial}
\end{itemize}
\end{itemize}
\begin{thebibliography}{99}
\bibitem{stewart} Precálculo: Matemáticas para el cálculo, \emph{Stewart James}, 5a edición, CENGAGE Learning.
\bibitem{esencial} Estadística elemental: lo esencial, \emph{Johnson Robert, \and Kuby Patricia}, décima edición, CENGAGE Learning
\end{thebibliography}
\section*{Estrategia de evaluación}
Se hará una evaluación escrita de suficiencia, valorando logros cognitivos y procedimentales. Se hará énfasis en los temas fundamentales vistos en el año y listados en ésta guía.
\section*{Fecha de presentación}
Según lo acordado en el cronograma general del colegio
\end{document}
