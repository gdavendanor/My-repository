\documentclass[fleqn]{article}
\usepackage[spanish,es-noshorthands]{babel}
\usepackage[utf8]{inputenc} 
\usepackage[papersize={6.5in,8.5in},left=1cm, right=1cm, top=1.5cm, bottom=1.7cm]{geometry}
\usepackage{mathexam}
\usepackage{amsmath}
\usepackage{graphicx}
\usepackage{tikz,pgf}
\usepackage{mathrsfs}
\usetikzlibrary{arrows}

\ExamClass{\includegraphics[height=16pt]{Images/logo-sed.png} Cálculo$11^{\circ}$}
\ExamName{Nivelación función afín}
\ExamHead{\includegraphics[height=16pt]{Images/logo-colegio.png} IEDAB}
\newcommand{\LineaNombre}{%
\par
\vspace{\baselineskip}
Nombre:\hrulefill \; Curso: \underline{\hspace*{48pt}} \; Fecha: \underline{\hspace*{2.5cm}} \relax
\par}
\let\ds\displaystyle

\begin{document}
\ExamInstrBox{
Respuesta sin justificar mediante procedimiento no será tenida en cuenta en la calificación. Escriba sus respuestas en el espacio indicado. Tiene 45 minutos para contestar esta prueba.}
\LineaNombre
\begin{enumerate}
 \item Dados los puntos $(x_{1},y_{1})=(-2,3)$ \quad y \quad $(x_{2},y_{2})=(2,-2)$
 \begin{enumerate}
 \item Ubique los puntos en el plano y trace la recta
 \begin{center}
\definecolor{cqcqcq}{rgb}{0.7529411764705882,0.7529411764705882,0.7529411764705882}
\begin{tikzpicture}[line cap=round,line join=round,>=triangle 45,x=1.0cm,y=1.0cm]
\draw [color=cqcqcq,, xstep=1.0cm,ystep=1.0cm,dotted] (-4.16,-3.5) grid (4.16,4.5);
\draw[->,color=black] (-4.16,0.) -- (4.16,0.);
\foreach \x in {-4,-3,-2,-1,1,2,3,4}
\draw[shift={(\x,0)},color=black] (0pt,2pt) -- (0pt,-2pt) node[below] {\footnotesize $\x$};
\draw[->,color=black] (0.,-3.5) -- (0.,4.5);
\foreach \y in {-3,-2,-1,1,2,3,4}
\draw[shift={(0,\y)},color=black] (2pt,0pt) -- (-2pt,0pt) node[left] {\footnotesize $\y$};
\draw[color=black] (0pt,-10pt) node[right] {\footnotesize $0$};
\clip(-4.16,-3.5) rectangle (4.16,4.5);
\end{tikzpicture}
\end{center}
 \item Determine la pendiente de la recta
  \newpage
 \item Halle el intercepto con el eje $y$ \noanswer
 \item Determine la ecuación de la recta \noanswer
 \end{enumerate}
 \item Dada la ecuación $f(x)=-\frac{1}{2}x+1$, haga su gráfica usando una tabla de valores apropiada e identifique la pendiente y el intercepto con el eje $y$ en la ecuación.
 
\begin{minipage}{.65\textwidth}
 \definecolor{cqcqcq}{rgb}{0.7529411764705882,0.7529411764705882,0.7529411764705882}
\begin{tikzpicture}[line cap=round,line join=round,>=triangle 45,x=1.0cm,y=1.0cm]
\draw [color=cqcqcq,, xstep=1.0cm,ystep=1.0cm,dotted] (-3.25,-2.25) grid (4.5,3.5);
\draw[->,color=black] (-3.25,0.) -- (4.5,0.);
\foreach \x in {-3,-2,-1,1,2,3,4}
\draw[shift={(\x,0)},color=black] (0pt,2pt) -- (0pt,-2pt) node[below] {\footnotesize $\x$};
\draw[->,color=black] (0.,-2.25) -- (0.,3.5);
\foreach \y in {-1,1,2,3}
\draw[shift={(0,\y)},color=black] (2pt,0pt) -- (-2pt,0pt) node[left] {\footnotesize $\y$};
\draw[color=black] (0pt,-10pt) node[right] {\footnotesize $0$};
\clip(-3.25,-2.25) rectangle (4.5,3.5);
\end{tikzpicture}
\end{minipage}
\begin{minipage}{.3\textwidth}
\begin{tabular}{|c|c|}
\hline 
$x$ & $y$ \\ 
\hline 
 &  \\ 
\hline 
 &  \\ 
\hline 
 &  \\ 
\hline 
 &  \\ 
\hline 
 &  \\ 
\hline 
 &  \\ 
\hline 
\end{tabular} 
\end{minipage}
\noanswer
 \end{enumerate}

\end{document}
