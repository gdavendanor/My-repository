\documentclass[fleqn,10pt]{article}
\usepackage[spanish,es-noshorthands]{babel}
\usepackage[utf8]{inputenc} 
\usepackage[papersize={6.5in,8.5in},left=1cm, right=1cm, top=1.7cm, bottom=1.5cm]{geometry}
\usepackage{mathexam}
\usepackage{amsmath}
\usepackage{pgf,tikz}
\usepackage{multicol}

\ExamClass{\includegraphics[height=16pt]{Images/logo-sed.png} Cálculo $11^{\circ}$}
\ExamName{"Límites introducción"}
\ExamHead{\includegraphics[height=16pt]{Images/logo-colegio.png} IEDAB}
\newcommand{\LineaNombre}{%
\par
\vspace{\baselineskip}
Nombre:\hrulefill \; Curso: \underline{\hspace*{48pt}} \; Fecha: \underline{\hspace*{2.5cm}} \relax
\par}


\let\ds\displaystyle

\begin{document}
\ExamInstrBox{
Debe mostrar los procedimientos! Respuestas sin el procedimiento requerido, no tendrán puntuación. Puede usar calculadora, pero no se aceptan préstamos de éstas durante el examen. Cada punto de la evaluaci\'on vale 2 pts y 1 punto por contestar la evaluación y marcarla.}
\ExamNameLine
\begin{enumerate}
   \item Calcule los siguientes límites, usando una tabla de valores. Recuerde que debe escoger valores cercanos donde no se sugieren.
      \begin{enumerate}
	 \item $\ds{\lim_{x\rightarrow6}\frac{x^{2}-6x}{x-6}}=$
\begin{center}
   \begin{tabular}{|c|p{1.5cm}|p{1.5cm}|p{1.5cm}||p{1.5cm}|p{1.5cm}|p{1.5cm}|}
\hline 
$x$ & 5,9 & 5,99 & 5,999 & 6,001 & 6,01 & 6,1  \\ 
\hline 
$f(x)$ &  &  &  &  & & \\ 
\hline 
\end{tabular} 
\end{center}	 
	 \item $\ds{\lim_{x\rightarrow3}\frac{\sqrt{x}-\sqrt{3}}{x-3}}=$
\begin{center}
   \begin{tabular}{|c|p{1.5cm}|p{1.5cm}|p{1.5cm}||p{1.5cm}|p{1.5cm}|p{1.5cm}|}
\hline 
$x$ &  &  &  &  &  &  \\ 
\hline 
$f(x)$ &  &  &  &  & & \\ 
\hline 
\end{tabular} 
\end{center}	
\item $\ds{\lim_{x\rightarrow 2}\dfrac{x^2+x-6}{x-2}}=$	 
\begin{center}
   \begin{tabular}{|c|p{1.5cm}|p{1.5cm}|p{1.5cm}||p{1.5cm}|p{1.5cm}|p{1.5cm}|}
\hline 
$x$ &  &  &  &  &  &  \\ 
\hline 
$f(x)$ &  &  &  &  & & \\ 
\hline 
\end{tabular} 
\end{center}	
\item $\ds{\lim_{x\rightarrow 0}\dfrac{1-\cos(x)}{x}}=$
\begin{center}
   \begin{tabular}{|c|p{1.5cm}|p{1.5cm}|p{1.5cm}||p{1.5cm}|p{1.5cm}|p{1.5cm}|}
\hline 
$x$ &  &  &  &  &  &  \\ 
\hline 
$f(x)$ &  &  &  &  & & \\ 
\hline 
\end{tabular} 
\end{center}	
Recuerde que la calculadora debe estar en radianes para trabajar con las funciones trigonométricas
      \end{enumerate}
      \newpage
   \item Con base en la siguiente gr\'afica, determine los l\'imites pedidos y valores pedidos:
\begin{center}
\usetikzlibrary{arrows}
\baselineskip=10pt
\hsize=6.3truein
\vsize=8.7truein
\definecolor{ttqqff}{rgb}{0.4,0.4,0.4}
\definecolor{qqwwtt}{rgb}{0.2,0.2,0.2}
\definecolor{ccqqtt}{rgb}{0.33,0.33,0.33}
\definecolor{cqcqcq}{rgb}{0.75,0.75,0.75}
\tikzpicture[scale=.9,line cap=round,line join=round,>=triangle 45,x=1.0cm,y=1.0cm]
\draw [color=cqcqcq,dash pattern=on 2pt off 2pt, xstep=1.0cm,ystep=1.0cm] (-3.25,-4.72) grid (4.85,5.21);
\draw[->,color=black] (-3.25,0) -- (4.85,0);
\foreach \x in {-3,-2,-1,1,2,3,4}
\draw[shift={(\x,0)},color=black] (0pt,2pt) -- (0pt,-2pt) node[below] {$\x$};
\draw[->,color=black] (0,-4.72) -- (0,5.21);
\foreach \y in {-4,-3,-2,-1,1,2,3,4}
\draw[shift={(0,\y)},color=black] (2pt,0pt) -- (-2pt,0pt) node[left] {$\y$};
\draw[color=black] (0pt,-10pt) node[right] {$0$};
\clip(-3.25,-4.72) rectangle (4.85,5.21);
\draw[line width=1.6pt,color=ccqqtt, smooth,samples=100,domain=-3.2451528332377877:-1.0] plot(\x,{(\x)+3});
\draw[line width=1.6pt,color=qqwwtt, smooth,samples=100,domain=2.05:4.852865362569915] plot(\x,{0-2*(\x)+7});
\draw[line width=1.6pt,color=ttqqff, smooth,samples=100,domain=-1.0:2.0] plot(\x,{0-2*(\x)^2+4});
\fill [color=black] (2,-4) circle (2.0pt);
\draw [color=black] (2,3) circle (2.5pt);
\fill [color=black] (-1,2) circle (2.0pt);
\endtikzpicture
\end{center}
\begin{enumerate}
\item $\ds{\lim_{x\rightarrow -2}f(x)}=$
\item $\ds{\lim_{x\rightarrow-1}f(x)=}$
\item $\ds{\lim_{x\rightarrow0}f(x)=}$
\item $\ds{\lim_{x\rightarrow2^{-}}f(x)=}$ \hfill Por la izquierda de 2
\item $\ds{\lim_{x\rightarrow2^{+}}f(x)=}$ \hfill Por la derecha de 2
\item $\ds{\lim_{x\rightarrow2}f(x)=}$
\item $\ds{\lim_{x\rightarrow4}}=$
\item $f(2)=$
\end{enumerate}
\end{enumerate}
\end{document}