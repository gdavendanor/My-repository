\documentclass[fleqn]{article}
\usepackage[spanish,es-noshorthands]{babel}
\usepackage[utf8]{inputenc} 
\usepackage[papersize={5.5in,8.5in},left=1cm, right=1cm, top=1.5cm, bottom=1.7cm]{geometry}
\usepackage{mathexam}
\usepackage{amsmath}
\usepackage{graphicx}
\usepackage{tikz,pgf}

\ExamClass{\includegraphics[height=16pt]{Images/logo-sed.png} Matemáticas $9^{\circ}$}
\ExamName{Examen 2, Nivelación 2017}
\ExamHead{\includegraphics[height=16pt]{Images/logo-colegio.png} IEDAB}
\newcommand{\LineaNombre}{%
\par
\vspace{\baselineskip}
Nombre:\hrulefill \; Curso: \underline{\hspace*{48pt}} \; Fecha: \underline{\hspace*{2.5cm}} \relax
\par}
\let\ds\displaystyle

\begin{document}
\ExamInstrBox{
Respuesta sin justificar mediante procedimiento no será tenida en cuenta en la calificación. Escriba sus respuestas en el espacio indicado. Tiene 45 minutos para contestar esta prueba.}
\LineaNombre
\begin{enumerate}
 \item Para promover la asistencia a un espectáculo se anuncia que de los 500 boletos disponibles, cada uno con un costo de \$2000 se hará un descuento de: \$100 al comprador número 50, \$150 al comprador 100, \$200 al comprador 150 y así sucesivamente. ¿Si todos los boletos fueron vendidos, entonces cuánto dejó de recibir el empresario por efecto de los descuentos? ¿Cuánto pagó por su boleto el último comprador? \noanswer
 \end{enumerate}

\end{document}
