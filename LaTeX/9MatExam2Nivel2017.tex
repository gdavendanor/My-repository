\documentclass[fleqn]{article}
\usepackage[spanish,es-noshorthands]{babel}
\usepackage[utf8]{inputenc} 
\usepackage[papersize={5.5in,8.5in},left=1cm, right=1cm, top=1.5cm, bottom=1.7cm]{geometry}
\usepackage{mathexam}
\usepackage{amsmath}
\usepackage{graphicx}
\usepackage{tikz,pgf}

\ExamClass{\includegraphics[height=16pt]{Images/logo-sed.png} Matemáticas $9^{\circ}$}
\ExamName{Examen 2, Nivelación 2017}
\ExamHead{\includegraphics[height=16pt]{Images/logo-colegio.png} IEDAB}
\newcommand{\LineaNombre}{%
\par
\vspace{\baselineskip}
Nombre:\hrulefill \; Curso: \underline{\hspace*{48pt}} \; Fecha: \underline{\hspace*{2.5cm}} \relax
\par}
\let\ds\displaystyle

\begin{document}
\ExamInstrBox{
Respuesta sin justificar mediante procedimiento no será tenida en cuenta en la calificación. Escriba sus respuestas en el espacio indicado. Tiene 45 minutos para contestar esta prueba.}
\LineaNombre
\begin{enumerate}
\item Determine si los pares ordenados (0,--2), (--3,0), (1,2) son soluciones de la ecuación $2x+3y=-6$\noanswer
\item \label{ej56} La medida del ángulo más grande de un triángulo es dos veces la medida del ángulo más pequeño. La suma de las medidas del ángulo más grande y el más pequeño es dos veces la medida del ángulo restante. Encuentre las medidas de los ángulos del triángulo.\noanswer
\item Solucione la ecuación cuadrática $x^{2}-3x-28=0$\noanswer
\newpage
\item Encuentre dos números cuya suma es 6 y cuyo producto es 2.\noanswer
 \item Para promover la asistencia a un espectáculo se anuncia que de los 500 boletos disponibles, cada uno con un costo de \$2000 se hará un descuento de: \$100 al comprador número 50, \$150 al comprador 100, \$200 al comprador 150 y así sucesivamente. ¿Si todos los boletos fueron vendidos, entonces cuánto dejó de recibir el empresario por efecto de los descuentos? ¿Cuánto pagó por su boleto el último comprador? \noanswer
 \item Las amibas son protozoarios unicelulares. Son capaces de vivir como parásitos y como organismos de vida libre. En México se detectaron por primera vez en 1611. Su nombre proviene del griego amoibe que significa cambio. Son sumamente resistentes, sobreviven a temperaturas de congelación y a soluciones con cloro. Su tamaño medio es de 0.025 milímetros y se reproducen por fisión.

Cierto tipo de amibas se reproduce por fisión cada 20 minutos. Por cada amiba encontrada en el intestino en determinado momento, ¿cuántas habrá 2 horas después? Elabore una tabla.
\begin{center}
\begin{tabular}{|c|c|}
\hline 
Período de tiempo en minutos & \# de amibas \\ 
\hline 
0 &  \\ 
\hline 
20 &  \\ 
\hline 
 &  \\ 
\hline 
 &  \\ 
\hline 
 &  \\ 
\hline 
 &  \\ 
\hline 
 &  \\ 
\hline 
\end{tabular} 

\end{center}
 \end{enumerate}

\end{document}
