\documentclass[letterpaper,spanish]{letter}
\usepackage[T1]{fontenc}
\usepackage[utf8]{inputenc}
\usepackage[spanish]{babel}
\usepackage{lmodern}
\usepackage{marvosym}

\address{Colegio Arborizadora Baja I.E.D j.m.}
\signature{Docentes firmantes}

\begin{document}

\begin{letter}{CONSEJO DIRECTIVO\\CONSEJO ACADÉMICO\\Colegio Arborizadora Baja I.E.D.}
	
\opening{Respetuoso saludo}
Los/as docentes que participamos en el cese de actividades académicas entre el 15 de mayo y el 16 de junio del presente año 2017, sentimos que nunca fue consultada nuestra opinión respecto al trabajo a realizar en la semana de desarrollo institucional de octubre, que según la resolución 1177 del MEN, debe ser usada para reponer actividades académicas con los estudiantes que no tuvieron clase durante el paro, y no para desarrollar actividades que nos competen a todos/as los/as maestros/as y a todos/as los/as estudiantes de la institución. Las tareas que propone el ``equipo directivo'' tienen que ver con actividades que desarrollaremos las diferentes áreas y en éstas se requiere la presencia de todos y todas. Además éstas actividades ya estaban programadas en el calendario y como tal deben desarrollarse.

Por otro lado, consideramos que la directiva ministerial 37 del MEN se dió en el marco del paro como una amenaza del gobierno al magisterio, por su decidida lucha por asegurar recursos necesarios para la educación. Ésta y otras circulares emitidas por el MEN y la SED, buscaban amedrentar a los/as maestros/as que dimos una lucha incansable en las calles, mostrando el gobierno, la administración distrital y algunos directivos docentes una actitud revanchista. Además éstas resoluciones y/o circulares son ilegítimas y muestran la doble moral del gobierno, que constantemente repite interesarle el bienestar de los niños y niñas de Colombia, y por otro lado les vulneran a los estudiantes el derecho a reponer todas las actividades académicas, como queda en evidencia en la resolución 1304 de la SED, que en el parágrafo del artículo octavo dice: 
\begin{quote}
\textit{Si durante el cese de actividades los estudiantes no asistieron al establecimiento educativo, y el docente asistió, éste dejará algunas actividades de formación que serán desarrolladas por el estudiante}
\end{quote}
Es decir, que si los estudiantes fueron solidarios con los maestros, entonces se les castiga, siendo la reposición de tiempo no laborado un compromiso del magisterio con estudiantes y padres de familia, que estamos dispuestos a asumir, aunque gran parte de la responsabilidad de la duración del paro fue del mismo gobierno que no quiso solucionar la crisis antes.

A nivel institucional, discrepamos de la resolución rectorial 06, que dice adoptar en su artículo primero la resolución 1304 de la SED, sin embargo desconoce que ésta en su artículo tercero dice: \begin{quotation}
``\textit{\textbf{Actividades de desarrollo institucional}: Las dos semanas de desarrollo institucional por realizar, previstas en la resolución 1974 de 2016, se llevarán a cabo los días: 20 de julio, 12 de agosto, 26 de agosto, 21 de octubre y 11 de noviembre}''.\end{quotation} Es decir que éstas actividades de desarrollo institucional se desarrollarán en esta 5 fechas y no como dice la resolución rectorial 06 en 10 fechas.

Concluyendo, solicitamos respetuosamente que las decisiones importantes que nos competen a todos y todas se den en los espacios y órganos del gobierno escolar (Consejo académico, Consejo directivo, consejo estudiantil, consejo de padres y los diferentes comités) tal como está consagrado en la ley (ley 115 de 1994 y decreto 1860). No puede ser que las decisiones importantes recaigan exclusivamente en el ``equipo directivo'', órgano que no está establecido en la ley.

Por último, solicitamos que nuestro comité sindical tenga un espacio el 21 de agosto, para socializar con las madres y padres de familia de la institución, los puntos firmados el 16 de julio entre nuestra Federación Fecode y el MEN, y, hacer el balance respectivo.
 
\closing{Atentamente,}

%cc{CONSEJO ACADÉMICO}
%\ps{PS: PostScriptum}
\encl{Listado de firmas}

\end{letter}
\end{document}
