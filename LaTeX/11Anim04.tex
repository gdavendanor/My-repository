\documentclass[letterpaper,11pt,twoside]{article}
\usepackage[utf8]{inputenc}
\usepackage{amsmath,amsfonts,amssymb,amsthm,latexsym}
\usepackage[spanish,es-noshorthands]{babel}
\usepackage[T1]{fontenc}
\usepackage{lmodern}
\usepackage{graphicx,hyperref}
\usepackage{tikz,pgf}
\usepackage{marvosym}
\usepackage{multicol}
\usepackage{fancyhdr}
\usepackage[top=1cm,right=.75cm,left=.75cm,bottom=1cm]{geometry}
\usepackage{fancyhdr}
\pagestyle{fancy}
\fancyhead[LE]{matematicas.german@gmail.com}
\fancyhead[RE]{\url{https://www.autistici.org/mathgerman}}
\fancyhead[RO]{\url{https://www.autistici.org/mathgerman}}
\fancyhead[LO]{\Email matematicas.german@gmail.com}

\author{Germ\'an Avenda\~no Ram\'irez~\thanks{Lic. Mat. U.D., M.Sc. U.N.}}
\title{\begin{minipage}{.2\textwidth}
\includegraphics[height=1.75cm]{Images/logo-colegio.png}\end{minipage}
\begin{minipage}{.55\textwidth}
\begin{center}
Animaplano 4\\
Cálculo $11^{\circ}$
\end{center}
\end{minipage}\hfill
\begin{minipage}{.2\textwidth}
\includegraphics[height=1.75cm]{Images/logo-sed.png} 
\end{minipage}}
\date{}
\thispagestyle{plain}
\begin{document}
\maketitle
Nombre: \hrulefill Curso: \underline{\hspace*{44pt}} Fecha: \underline{\hspace*{2.5cm}}
\begin{multicols}{2}
\section{Cuestionario}
Seleccione la respuesta correcta a cada una de las situaciones de los numerales 1--4.

Un algodonero recoge 30 kg cada hora, y demora media hora preparándose todos los días cuando inicia la jornada. La función que representa esta situación es $y=30x-5$, donde $y$ representa los kg de algodón recogido y $x$ el tiempo transcurrido en horas.
\begin{enumerate}
\item Es correcto afirmar que
\begin{enumerate}
\begin{tabular}{|p{5cm}|c|}
\hline 
\item A medida que avanza el tiempo disminuye la cantidad de algodón recolectada & (53) \\ 
\hline 
\item La cantidad de tiempo es equivalente a la cantidad de algodón recogido & (61) \\ 
\hline 
\item La cantidad de algodón recogida aumenta a medida que avanza el tiempo & (74) \\ 
\hline 
\end{tabular} 
\end{enumerate}
\item La cantidad de algodón recogida luego de 7 horas es
\begin{center}
\begin{tabular}{|c|c|c|}
\hline 
a) 165 kg & b) 195 kg & c) 215 kg\\ 
\hline 
(81) & (92) & (87) \\ 
\hline 
\end{tabular} 
\end{center}
Un fabricante de lácteos produce 1800 kg de queso tipo crema cada 6 días y mantiene su producción constante durante todo el año.
\item La función de $x$ (en días) que modela esta situación es
\begin{center}
\begin{tabular}{|c|c|c|}
\hline 
a) $f(x)=1800x$ & b) $f(x)=6x$ & c) $f(x)=300x$ \\ 
\hline 
(63) & (78) & (99) \\ 
\hline 
\end{tabular} 
\end{center}
\item Si trabaja de lunes a viernes y el sábado medio día ¿cuál es el total de la producción semanal?
\begin{center}
\begin{tabular}{|c|c|c|}
\hline 
a) 1650 kg & b) 1500 kg & 1800 kg \\ 
\hline 
(76) & (69) & (58) \\ 
\hline 
\end{tabular} 
\end{center}
En cada una de las secuencias de los numerales 5 al 19 encuentre los numerales que faltan
\begin{center}
\begin{tabular}{c|c|c|c|c|c|c|c}
\cline{2-7} 
5. & \hspace*{20pt} & 90 & 77 & 82 & 69 & \hspace*{20pt} & 6. \\ 
\cline{2-7} 
7. &  & 81 & 71 & 75 & 65 &  & 8. \\ 
\cline{2-7} 
9. &  & 56 & 35 & 42 & 21 &  & 10. \\ 
\cline{2-7} 
11. &  & 20 & 15 & 18 & 13 &  & 12. \\ 
\cline{2-7} 
13. &  & 10 & 6 & 12 & 8 &  & 14. \\ 
\cline{2-7} 
15. &  & 14 & 7 & 15 & 8 &  & 16. \\ 
\cline{2-7} 
17. &  & 20 & 14 & 21 & 15 &  & 18. \\ 
\cline{2-7} 
19. &  & 27 & 26 & 30 & 29 &  & 20. \\ 
\cline{2-7} 
\end{tabular} 
\end{center}
En el arreglo la suma de los valores numéricos de las letras de cada fila y columna se encuentran indicados en los extremos de las mismas. Determine el valor de cada letra.

\begin{minipage}{1.5cm}
\item[21.] A = \underline{\quad}
\item[22.] B = \underline{\quad}
\item[23.] C = \underline{\quad}
\item[24.] D = \underline{\quad}
\item[25.] E = \underline{\quad}
\end{minipage}
\begin{minipage}{.5\textwidth}
\begin{tikzpicture}
\draw (-2,2)--(0,2)--(0,1)--(-2,1)--cycle;
\draw (-2,1)--(-2.25,1)--(-2.5,1.5)--(-2.25,2)--(-2,2);
\draw (-1,2) rectangle (0,0);
\draw (0,1)--(0,-2)--(1,-2)--(1,1)--cycle;
\draw (0,0)--(2,0)--(2,1)--(1,1);
\draw (0,-1)--(2,-1)--(2,0);
\draw (-1,2)--(-1,2.25)--(-0.5,2.5)--(0,2.25)--(0,2);
\draw (2,1)--(2.25,1)--(2.5,.5)--(2.25,0)--(2,0)--(2.25,0)--(2.5,-.5)--(2.25,-1)--(2,-1)--(2,-1.25)--(1.5,-1.5)--(1,-1.25);
\draw (0,-2)--(0,-2.25)--(0.5,-2.5)--(1,-2.25)--(1,-2);
\node at (-.5,2.25){65};
\node[rotate=90] at (-2.25,1.5){126};
\node[rotate=-90] at (2.25,.5){110};
\node[rotate=-90] at (2.25,-.5){73};
\node at (1.5,-1.25){45};
\node at (0.5,-2.25){178};
\node at (-1.5,1.5){C};
\node at (-.5,1.5){A};
\node at (-.5,0.5){13};
\node at (.5,.5){D};
\node at (1.5,.5){F};
\node at (.5,-.5){B};
\node at (1.5,-.5){11};
\node at (.5,-1.5){E};
\end{tikzpicture}
\end{minipage}

\begin{minipage}{5cm}
\begin{tikzpicture}
\draw (-1,1)rectangle (1,-1);
\draw (0,0)rectangle (2,2);
\draw (1,1)--(1,2)--(1,2.25)--(1.5,2.5)--(2,2.25)--(2,2)--(2.25,2)--(2.5,1.5)--(2.25,1)--(1,1)--(2.25,1)--(2.5,.5)--(2.25,0)--(2,0);
\draw (0,0)--(-1.25,0)--(-1.5,-.5)--(-1.25,-1)--(-1,-1)--(-1,-1.25)--(-.5,-1.5)--(0,-1.25)--(0,0)--(0,-1.25)--(0.5,-1.5)--(1,-1.25)--(1,-1);
\node at (1.5,2.25){142};
\node[rotate=-90] at (2.25,1.5){81};
\node[rotate=-90] at (2.25,.5){126};
\node at (.5,-1.25){78};
\node at (-.5,-1.25){56};
\node[rotate=90] at (-1.25,-.5){69};
\node at (.5,1.5){15};
\node at (1.5,1.5){J};
\node at (1.5,.5){K};
\node at (.5,.5){H};
\node at (-.5,.5){G};
\node at (-.5,-.5){32};
\node at (.5,-.5){I};
\end{tikzpicture}
\end{minipage}
\begin{minipage}{1.5cm}
\item[26.] F = \underline{\quad}
\item[27.] G = \underline{\quad}
\item[28.] H = \underline{\quad}
\item[29.] I = \underline{\quad}
\item[30.] J = \underline{\quad}
\item[31.] K = \underline{\quad}
\end{minipage}
\end{enumerate}
\end{multicols}


\end{document}
