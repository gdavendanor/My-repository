\documentclass[letterpaper,twoside]{article}
\usepackage[utf8]{inputenc}
\usepackage{amsmath,amsfonts,amssymb,amsthm,latexsym}
\usepackage[spanish,es-noshorthands]{babel}
\usepackage[T1]{fontenc}
\usepackage{lmodern}
\usepackage{graphicx,hyperref}
\usepackage{tikz,pgf}
\usepackage{marvosym}
\usepackage{multicol}
\usepackage{fancyhdr}
\usepackage[height=9.5in,width=7.25in]{geometry}
\usepackage{fancyhdr}
\pagestyle{fancy}
\fancyhead[LE]{\Email matematicas.german@gmail.com}
\fancyhead[RE]{\url{https://www.autistici.org/mathgerman}}
\fancyhead[RO]{\url{https://www.autistici.org/mathgerman}}
\fancyhead[LO]{\Email matematicas.german@gmail.com}

\author{Germ\'an Avenda\~no Ram\'irez~\thanks{Lic. Mat. U.D., M.Sc. U.N.}}
\title{\begin{minipage}{.2\textwidth}
\includegraphics[height=1.75cm]{Images/logo-colegio.png}\end{minipage}
\begin{minipage}{.55\textwidth}
\begin{center}
Nivelación 2015\\
Matemáticas $9^{\circ}$
\end{center}
\end{minipage}\hfill
\begin{minipage}{.2\textwidth}
\includegraphics[height=1.75cm]{Images/logo-sed.png} 
\end{minipage}}
\date{}
\thispagestyle{plain}
\begin{document}
\maketitle
Nombre: \hrulefill Curso: \underline{\hspace*{44pt}} Fecha: \underline{\hspace*{2.5cm}}
\begin{multicols}{2}
\section*{Exponentes y radicales}
Para los ejercicios \ref{ej01}--\ref{ej02}, evalúe la expresión numérica 
\begin{enumerate}
\begin{multicols}{2}
\item \label{ej01} $4^{-3}$ 
\item $(3^{2}\cdot 3^{-3})^{-1}$
\item \label{ej02} $\left(\dfrac{3^{-1}}{3^{2}}\right)^{-1}$
\end{multicols}
Para los ejercicios \ref{ej03}--\ref{ej04}, simplifique y exprese el resultado final usando exponentes positivos solamente 
\begin{multicols}{2}
\item \label{ej03} $(x^{-3}y^{4})^{-2}$
\item $\left(\dfrac{4a^{-2}}{3b^{-2}}\right)^{-2}$
\item $\left(\dfrac{6x^{-2}}{2x^{4}}\right)^{-2}$
\item $(-5x^{-3})(2x^{6})$
\item $\dfrac{a^{-1}b^{-2}}{a^{4}b^{-5}}$
\item \label{ej04} $\dfrac{-12x^{3}}{6x^{5}}$
\end{multicols}
Para los ejercicios \ref{ej05}--\ref{ej06}, exprese como una fracción simple usando solamente exponentes positivos
\begin{multicols}{2}
\item \label{ej05} $x^{-2}+y^{-1}$
\item \label{ej06} $2x^{-1}+3y^{-2}$
\end{multicols}
Para los ejercicios \ref{ej07}--\ref{ej08}, exprese el radical en su forma más simple. Suponga que las variables representan números reales positivos.
\begin{multicols}{2}
\item \label{ej07} $\sqrt{54}$
\item $\sqrt[3]{56}$
\item $\frac{3}{4}\sqrt{150}$
\item $\dfrac{4\sqrt{3}}{\sqrt{6}}$
\item $\dfrac{\sqrt[3]{2}}{\sqrt[3]{9}}$
\item \label{ej08} $\sqrt{\dfrac{3x^{3}}{7}}$
\end{multicols}
Para los ejercicios \ref{ej09}--\ref{ej10}, use la propiedad distributiva para simplificar cada expresión
\item \label{ej09} $3\sqrt{45}-2\sqrt{20}-\sqrt{80}$
\item $4\sqrt[3]{24}+3\sqrt[3]{3}-2\sqrt[3]{81}$
\item $3\sqrt{24}-\dfrac{2\sqrt{54}}{5}+\dfrac{\sqrt{96}}{4}$
\item \label{ej10} $-2\sqrt{12x}+3\sqrt{27x}-5\sqrt{48x}$

Para los ejercicios \ref{ej11}--\ref{ej12}, multiplique y simplifique. Suponga que las variables representan número s reales no negativos
\item \label{ej11} $(3\sqrt{48})(4\sqrt{5})$
\item $(\sqrt{6xy})(\sqrt{10x})$
\item $3\sqrt{2}(4\sqrt{6}-2\sqrt{7})$
\item $(2\sqrt{5}-\sqrt{3})(2\sqrt{5}+\sqrt{3})$
\item \label{ej12} $(2\sqrt{a}+\sqrt{b})(3\sqrt{a}-4\sqrt{b})$

Para los ejercicios \ref{ej13}--\ref{ej14}, racionalice el denominador y simplifique
\begin{multicols}{2}
\item \label{ej13} $\dfrac{4}{\sqrt{7}-1}$
\item \label{ej14} $\dfrac{3}{2\sqrt{3}+3\sqrt{5}}$
\end{multicols}

Para los ejercicios \ref{ej15}--\ref{ej16}, resuelva la ecuación
\begin{multicols}{2}
\item \label{ej15} $\sqrt{7x-3}=4$
\item $\sqrt{2x}=x-4$
\item $\sqrt[3]{2x-1}=3$
\item \label{ej16} $\sqrt{x^{2}+3x-6}=x$
\end{multicols}
\item La ecuación $T=2\pi\sqrt{\dfrac{L}{32}}$ es usada para describir el movimiento de un péndulo, donde $T$ representa el período del péndulo en segundos y $L$ representa la longitud del péndulo en pies. Encuentre la longitud de un péndulo, aproximando a la décima más cercana de pie, si el período es de 2.4 segundos.

Para los ejercicios \ref{ej17}--\ref{ej18}, simplifique
\begin{multicols}{2}
\item \label{ej17} $4^{\frac{5}{2}}$
\item $\left(\dfrac{8}{27}\right)^{\frac{2}{3}}$
\item $(27)^{-\frac{2}{3}}$
\item \label{ej18} $9^{\frac{3}{2}}$
\end{multicols}
Para los ejercicios \ref{ej19}--\ref{ej20}, escriba la expresión usando exponente racionales positivos
\begin{multicols}{2}
\item \label{ej19} $\sqrt[5]{x^{3}y}$
\item \label{ej20} $6\sqrt[4]{y^{2}}$
\end{multicols}
Para los ejercicios \ref{ej21}--\ref{ej22}, exprese el resultado final usando exponentes positivos
\begin{multicols}{2}
\item \label{ej21} $(4x^{\frac{1}{2}})(5x^{\frac{1}{5}}$
\item $\left(\dfrac{x^{3}}{y^{4}}\right)^{-\frac{1}{3}}$
\item \label{ej22} $(x^{\frac{4}{5}})^-{\frac{1}{2}}$
\end{multicols}
Para los ejercicios \ref{ej23}--\ref{ej24}, realice la operación indicada y exprese la respuesta en su forma radical más simple
\begin{multicols}{2}
\item \label{ej23} $\sqrt[4]{3}\sqrt{3}$
\item \label{ej24} $\dfrac{\sqrt[3]{5}}{\sqrt[4]{5}}$
\end{multicols}
Para los ejercicios \ref{ej25}--\ref{ej26}, escriba el número en notación científica
\item \label{ej25} $540\,000\,000$
\item \label{ej26} $0.000000032$

Para los ejercicios \ref{ej27}--\ref{ej28}, escriba el número en notación decimal ordinaria
\begin{multicols}{2} 
\item \label{ej27} $(1.4)(10^{-6})$
\item \label{ej28} $(4.12)(10^{7})$
\end{multicols}
Para los problemas \ref{ej29}--\ref{ej30}, use la notación científica y la propiedad de los exponentes para ayudar en el cálculo
\item \label{ej29} $(0.00002)(0.0003)$
\item $(0.000015)(400\,000)$
\item $\dfrac{(0.00042)(0.0004)}{0.006}$
\item $\sqrt[3]{0.000000008}$ \label{ej30}
\section*{Ecuación de primer grado}
Determine si los pares ordenados son soluciones de las ecuaciones dadas en \ref{ej57}--\ref{ej58}
\item \label{ej57} $4x+y=6$; \quad (1,2), (6,0), (--1,10)
\item $3x+2y=12$; \quad (2,3), (--2,9), (3,2)
\item \label{ej58} $2x+3y=-6$; \quad (0,--2), (--3,0), (1,2)

Para \ref{ej59}--\ref{ej60}, complete la tabla de valores para la ecuación y haga la gráfica
\item \label{ej59} $y=2x-5$ 

\begin{tabular}{c|cccc}
\hline 
$x$ & --1 & 0 & 1 & 4 \\ 
\hline 
$y$ &  &  &  &  \\ 
\hline 
\end{tabular} 
\item \label{ej60} $y=\dfrac{3x-4}{2}$ 
\begin{center}
\begin{tabular}{c|cccc}
\hline 
$x$ & --2 & 0 & 2 & 4 \\ 
\hline 
$y$ &  &  &  &  \\ 
\hline 
\end{tabular} 
\end{center}

En los ejercicios \ref{ej61}--\ref{ej62}, grafique cada ecuación encontrando los intercepto en el eje $x$ y $y$
\item \label{ej61} $2x-y)6$
\item \label{ej62} $x-2y=4$

Resuelva el problema \ref{ej63}
\item \label{ej63} Una empresa de mudanzas de apartamentos cobra de acuerdo a la ecuación $c=75h+150$, donde $c$ representa el dinero en dólares y $h$ representa el número de horas para hacer el trasteo. 
\begin{enumerate}
\item Complete la tabla

\begin{center}
 \begin{tabular}{c|cccc}
\hline 
$h$ & 1 & 2 & 3 & 4 \\ 
\hline 
$c$ &  &  &  &  \\ 
\hline 
\end{tabular}
 \end{center} 
\item Haciendo que el eje horizontal sea $h$ y el eje vertical $c$, grafique la ecuación $c=75h+150$ para valores no negativos de $h$
\item Use la gráfica para aproximar los valores de $c$ cuando $h=1.5$ y 3.5
\end{enumerate}


\section*{Ecuación cuadrática}
Para los problemas \ref{ej31}--\ref{ej32}, realice las operaciones indicadas y exprese las respuestas en la forma standard de un número complejo
\item \label{ej31} $(-7+3i)+(9-5i)$
\item \label{ej32} $(6-3i)-(-2+5i)$

Para los problemas \ref{ej33}--\ref{ej34}, escriba la expresión en término de $i$ y simplifique
\begin{multicols}{2}
\item \label{ej33} $\sqrt{-8}$
\item \label{ej34} $3\sqrt{-16}$
\end{multicols}
Para los ejercicios \ref{ej35}--\ref{ej36}, realice la operación indicada y simplifique
\item \label{ej35} $\sqrt{-2}\sqrt{-6}$
\item $\dfrac{\sqrt{-42}}{\sqrt{-6}}$
\item $5i(3-6i)$
\item $(-2-3i)(4-8i)$
\item \label{ej36} $\dfrac{4+3i}{6-2i}$
\item Efectúe $\dfrac{3+4i}{2i}$

Para los problemas \ref{ej37}--\ref{ej38}, resuelva la ecuación cuadrática factorizando
\begin{multicols}{2}
\item \label{ej37} $x^{2}+8x=0$
\item \label{ej38} $x^{2}-3x-28=0$
\end{multicols} 
Para los problemas \ref{ej39}--\ref{ej40}, resuelva la ecuación cuadrática
\begin{multicols}{2}
\item \label{ej39} $2x^{2}=90$
\item \label{ej40} $(2x+3)^{2}=24$
\end{multicols}
Para los problemas \ref{ej41}--\ref{ej42}, use el método de "completar el cuadrado" para solucionar la ecuación cuadrática
\begin{multicols}{2}
\item \label{ej41} $y^{2}+18y-10=0$
\item \label{ej42} $x^{2}-10x+1=0$
\end{multicols}
Para los ejercicios \ref{ej43}--\ref{ej44}, use la fórmula cuadrática para solucionar la ecuación.
\begin{multicols}{2}
\item \label{ej43} $x^{2}+6x+4=0$
\item $3x^{2}-2x+4=0$ \label{ej44}
\end{multicols}
Para los ejercicios \ref{ej45}--\ref{ej46}, solucione la ecuación
\begin{multicols}{2}
\item \label{ej45} $x^{2}-17x=0$
\item $(2x-1)^{2}=-64$
\item $x^{2}+2x-9=0$
\item $4\sqrt{x}=x-5$
\item $n^{2}-10n=200$
\item $x^{2}-x+3=0$
\item $2a^{2}+4a-5=0$
\item $x^{2}+4x+9=0$
\item $\dfrac{3}{x}+\dfrac{2}{x+3}=1$
\item \label{ej46} $\dfrac{3}{n-2}=\dfrac{n+5}{4}$
\end{multicols}
Para los problemas \ref{ej47}--\ref{ej48}, platee una ecuación para resolverlos
\item \label{ej47} Encuentre dos números cuya suma es 6 y cuyo producto es 2
\item Naidú viajó 270 millas en una hora más de lo que le tomó a Liseth viajar 260 millas. Liseth condujo a 7 millas por hora más rápido que Naidú. ¿Qué tan rápido viajaron cada una?
\item \label{ej48} Encuentre dos números pares consecutivos cuya suma de sus cuadrados es 164.
\section*{Sistemas de ecuaciones de primer grado}
Solucione los ejercicios \ref{ej49}--\ref{ej50}, usando el método de sustitución.
\begin{multicols}{2}
\item $\displaystyle \left\{ {3x-y=16 \atop 5x+7y=-34} \right.$ \label{ej49}
\item \label{ej50} $\displaystyle{\left\{{2x-3y=12 \atop 3x+5y=-20}\right.}$
\end{multicols}
Solucione los ejercicios \ref{ej51}--\ref{ej52}, usando el método de eliminación por adición
\begin{multicols}{2}
\item \label{ej51} $\displaystyle{\left\{{4x-3y=34 \atop 3x+2y=0}\right.}$
\item \label{ej52} $\left\{ \begin{array}{rcl}
2x-y+3z=&-19\\
3x+2y-4z=&21\\
5x-4y-z=&-8
\end{array}
\right.$
\end{multicols}
Solucione \ref{ej53}--\ref{ej54}, usando el método que prefiera
\begin{multicols}{2}
\item \label{ej53} $\displaystyle{\left\{
{4x+7y=-15
\atop
3x-2y=25
}
\right.}$
\item \label{ej54} $\displaystyle{\left\{
{x+4y=3
\atop
3x-2y
}
\right.}$
\end{multicols}
Platee un sistema de ecuaciones y solucione \ref{ej55}--\ref{ej56}
\item \label{ej55} Antonio tiene un total de \$4200 de deuda en dos tarjetas de crédito. Una tarjeta tiene un interés mensual del 1\% y la otra del 1.5\%. Encuentre la deuda de cada tarjeta si paga \$57 en interés por mes
\item ¿Cuántas tazas de leche al 1\% de concentración deben ser mezcladas con leche al 4\% para obtener 10 tazas de mezcla de leche al 2\%?
\item \label{ej56} La medida del ángulo más grande de un triángulo es dos veces la medida del ángulo más pequeño. La suma de las medidas del ángulo más grande y el más pequeño es dos veces la medida del ángulo restante. Encuentre las medidas de los ángulos del triángulo.
\end{enumerate}
\end{multicols}


\end{document}
