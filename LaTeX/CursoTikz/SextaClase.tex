\documentclass{article}
\usepackage[utf8]{inputenc}
\usepackage[spanish]{babel}
\usepackage{graphicx}
\usepackage{amsmath,amsfonts,amssymb}

\usepackage{tikz}
\usetikzlibrary{graphdrawing,graphs,babel,arrows,calc}
\usegdlibrary{layered,circular,force,trees}

\begin{document}
\begin{figure}[ht]
	\centering
		\begin{tikzpicture}
			\graph[layered layout,nodes={circle,draw}]{
				1 -> {2,4},
				2 -> {4,5},
				3 -> {1,6},
				4 -> {3,5,6,7},
				5 -> 7,
				7 -> 6
};
\end{tikzpicture}
\caption{grafo simple}
\end{figure}

\begin{figure}[ht]
	\centering
		\begin{tikzpicture}
			\graph[simple necklace layout,nodes={circle,draw},node distance=2.5cm]{
				1->{2,4},
				2->{4,5},
				3->{1,6},
				4->{3,5,6,7},
				5->7,
				7->6
				};
		\end{tikzpicture}
	\caption{grafo con la biblioteca circular}
\end{figure}

\begin{figure}[ht]
	\centering
		\begin{tikzpicture}[>=stealth,thick,rotate=90,green!75!black]
			\graph[spring layout,nodes={circle,draw},node distance=2.5cm]{
				1->[line width=2pt,red]{2,4},
				2->{4,5},
				3->{1,6},
				4->{3,5,6,7},
				5->7,
				7->6
			};
			\begin{scope}[gray!75]
				\node[below] at ($(1)!.5!(2)$){2};
				\node[left] at ($(2)!.5!(5)$){10};
			\end{scope}
		\end{tikzpicture}
	\caption{grafo con la biblioteca force}
\end{figure}

\begin{figure}[ht]
	\centering
	\begin{tikzpicture}[rotate=0]
		\graph[tree layout,sibling sep=1.5cm,level distance=2cm]{
			A -- {B,C};
			B -- {D,E};
			C -- {F,G}	;
			D -- {H,I};
			E -- {J,K},
			G -- {L,M};
			I -- {N,O};
			L -- {P,Q};
			M -- {R}
		};
	\end{tikzpicture}					
	\caption{Grafo \'{a}rbol}
\end{figure}

\begin{figure}[ht]
\centering
\begin{tikzpicture}
	\graph[tree layout]{
	A--{B--{D--{H,I--{N,O}},E--{J,K}},C--{F,G--{L--{P,Q},M--{R}}}};
	};
\end{tikzpicture}
\caption{Otra definici\'on del grafo \'arbol}
\end{figure}
\end{document}
