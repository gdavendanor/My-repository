\documentclass{standalone}
\usepackage[utf8]{inputenc}
\usepackage[spanish,es-noshorthands]{babel}

\usepackage{tikz}
\usetikzlibrary{babel,through,intersections,backgrounds,calc}
\begin{document}
\begin{tikzpicture}
\coordinate[label=left:{\textcolor{blue}{$A$}}] (A) at (0,0);
\coordinate[label=right:{\textcolor{blue}{$B$}}](B) at (1.15,.75);

\draw[blue] (A)--(B); 

\node (D) [name path=D,draw,circle through=(B),label=left:{$D$}] at (A) {};
\node (E) [name path=E,draw,circle through=(A),label=right:$E$] at (B){};

\path[name intersections={of=D and E}];
\coordinate[label=above:{\textcolor{red}{$C$}}](C) at (intersection-1);
\draw[red](A)--(C)--(B);
\foreach \point in {A,B,C}
\fill[black,opacity=.5](\point) circle[radius=1.5pt];
\begin{pgfonlayer}{background}
\fill[orange] (A)--(B)--(C) --cycle;
\end{pgfonlayer}

\node[below,text width=10cm,align=justify] at (.5,-1.5){
\small\textbf{Proposición 1:}\par
\emph{Construir un \textcolor{orange}{triángulo equilátero} sobre un \textcolor{blue}{segmento de recta} dado.}
\par\vskip2em
Sea \textcolor{blue}{AB} un segmento de recta, construyendo...
};
\end{tikzpicture}
\end{document} 