\documentclass[11pt,letterpaper]{article}
\usepackage[utf8]{inputenc}
\usepackage[spanish]{babel}
\usepackage{amsmath}
\usepackage{amsfonts}
\usepackage{amssymb}
\usepackage{graphicx}
\usepackage{lmodern}
\usepackage{hyperref}
\author{Germán Avendaño Ramírez~\thanks{Lic. Mat. U.D.; M.Sc. U.N.; Docente del colegio Arborizadora Baja I.E.D; Delegado Asamblea ADE}}
\title{Software libre en la escuela}
\begin{document}
\maketitle
\section{Resumen}
En esta breve ponencia se tratará de definir que es el software libre y por qué debe usarse solamente éste en la escuela
\section{Introducción}
En la escuela se deben formar estudiantes autónomos, críticos y capaces de transformar la realidad. Para asegurar que nuestros estudiantes sean autónomos de verdad, no se les debe enseñar en la escuela a depender de software privativo para desarrollar sus tareas académicas, habiendo tanto software alternativo y libre. Si se enseña software privativo estaremos creando seres dependientes de las grandes multinacionales que manejan el software privativo como Microsoft, Apple, etc.
\section{¿Qué es el software libre?}
El software libre fue definido por Richard Mathew Stallman en la década de 1980. Éste debe respetar las 4 libertades esenciales \cite{gnu}
\paragraph{Libertad 0:} 
La libertad de ejecutar el programa como se desea, con cualquier propósito.
\paragraph{Libertad 1:} La libertad de estudiar cómo funciona el programa, y cambiarlo para que haga lo que usted quiera. El acceso al código fuente es una condición necesaria para ello.
\paragraph{Libertad 2:} La libertad de redistribuir copias para ayudar a su prójimo
\paragraph{Libertad 3:} La libertad de distribuir copias de sus versiones modificadas a terceros (libertad 3). Esto le permite ofrecer a toda la comunidad la oportunidad de beneficiarse de las modificaciones. El acceso al código fuente es una condición necesaria para ello.

Para tener la información de primera mano, se presentará el vídeo de Richard Stallman sobre "Software libre y educación" (6 minutos de duración)\cite{rms}
\subsection{¿Cuál es la relación entre el software libre y la educación?}
La educación debe formar en y para la libertad. Además de la libertad, en la escuela se deben formar individuos que sirvan a su comunidad. El software privativo es conocimiento secreto y restringido, por lo tanto se opone a las misiones de las instituciones educativas. El software libre por el contrario educa en la libertad y la cooperación.

El software libre no es simplemente un asunto meramente técnico sino que adquiere connotaciones éticas sociales y políticas. Es una cuestión de derechos humanos que los usuarios deben tener.\cite{rms}
\subsection{Por qué las escuelas solamente deben enseñar software libre}
Hay muchas razones por las cuales en las escuelas solamente debemos enseñar software libre. Entre algunas tenemos:\cite{esc}
\begin{itemize}
\item El software libre le brinda al usuario la libertad de saber que hacen realmente los programas instalados en sus dispositivos.Se sabe que muchos programas del software privativo tienen las denominadas puertas traseras (backdoors) que permiten a los gobiernos y/o multinacionales espiar a sus usuarios\footnote{Es famoso el caso de Amazon con su dispositivo Kindle. A miles de usuarios que habían obtenido el libro 1984 de H.G. Wells gratuitamente, les fue borrado remotamente porque su contenido no les pareció "apto"}. Además permite estudiar y mejorar el software. Brinda la oportunidad de aprender.
\item Supone un ahorro económico porque no se deben pagar onerosas licencias. El software libre generalmente es muy económico y en la mayoría de los casos es gratuito.
\item Permite saber a los y las estudiantes como funciona el software y así desarrollar habilidades para que no seamos solamente un país consumidor de tecnología, sino que seamos capaces de crear nuestros propios programas.
\item La razón más profunda tiene que ver con la educación moral. Se espera que en la escuela se enseñen a nuestros estudiantes a ser buenos ciudadanos lo cual implica enseñarles el hábito de ayudar a los demás.
\item Enseñar software privativo es crear dependencia de las multinacionales, que se reservan el derecho de esconder el código fuente para que no sepamos realmente que hacen los programas, éstos pueden hacer cosas adicionales que no queremos que hagan, como por ejemplo espiarnos. Las grandes empresas del software quieren crear seres dependientes de su software, por eso a veces hasta regalan copias de de sus programas para crear dependencia como hacen los expendedores de drogas, que regalan la primera dosis a sus futuros compradores. Luego de que egresen nuestros estudiantes no sabrán operar sino software privativo y luego allí si tendrán que pagar las licencias. 
\end{itemize}
\section{Software libre útil en la escuela}
Para cada una de las áreas de la educación y cada una de nuestras necesidades tenemos a nuestra disposición suficientes aplicaciones que podemos usar con la tranquilidad de que no estamos haciendo algo ilegal, ya que éstas hacen uso de las licencias del software libre (GPL, GPL2, etc). A continuación una tabla con las aplicaciones alternativas al software privativo más usual.
\begin{center}
\textbf{Alternativas al software privativo}
\begin{tabular}{|l|l|}
\hline 
\textbf{Aplicación libre} & \textbf{Aplicación privativa }\\ 
\hline 
Libreoffice & Microsoft office \\ 
\hline 
Gimp & Adobe PhotoShop \\ 
\hline 
Inkscape & Adobe illustrator \\  
\hline 
mplayer, smplayer, vlc & Reproductor de windows multimedia \\ 
\hline 
Audacious, amarok, clementine & Winamp \\ 
\hline 
Firefox, midori, chromium \ldots & Chrome, Internet Explorer \ldots\\ 
\hline 
\end{tabular} 
\end{center}
Para la educación tenemos un abanico de posibilidades, algunas son:

\begin{tabbing}
\hspace{4cm}\=\kill
Matemáticas \> Geogebra, kig, drgeo, kbruch, kalgebra, etc \\ 
Geografía \> kgeography, marble, etc \\ 
Idiomas \> Goldendict, kletters, khangman, etc \\ 
Química \> chemtool, easychem, gperiodic, kalzium\\
Música \> Denemo, solfege,etc \\ 
\ldots \> \ldots
\end{tabbing} 
\section{Conclusiones}
Es urgente dar un viraje en cuanto al uso de las TICS en nuestras escuelas, porque el movimiento que impulsa éstas, no se ha detenido a mirar que podemos estar cayendo en el uso de aplicaciones que no permitirán a los futuros ciudadanos ser libres, sino que generaremos dependencia. Es necesario analizar las implicaciones éticas, sociales y políticas de continuar usando en nuestras aulas software privativo.
\begin{thebibliography}{99}
\bibitem{gnu} ¿Que es el software libre?, \emph{Stallman Richard, Proyecto GNU} \url{https://www.gnu.org/philosophy/free-sw.es.html}
\bibitem{rms} ¿Cuál es la relación entre software libre y la educación?, \emph{Stallman Richard} \url{https://www.gnu.org/education/education.es.html}
\bibitem{esc} Por qué las escuelas deben enseñar únicamente software libre, \emph{Stallman Ricard} \url{https://www.gnu.org/education/edu-schools.es.html}
\end{thebibliography}

\end{document}