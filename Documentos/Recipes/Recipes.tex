\documentclass[letterpaper]{recipe}
\usepackage[T1]{fontenc}
\usepackage[spanish,es-noshorthands]{babel}
\usepackage[utf8]{inputenc}
\usepackage{bookman}

\newcommand{\bsi}[2]{%
  \fontencoding{T1}\fontfamily{pbs}\fontseries{xl}\fontshape{n}%
  \fontsize{#1}{#2}\selectfont}

\renewcommand{\inghead}{\textbf{Ingredientes (para 2 raciones)}:\ }
\renewcommand{\rechead}{\centering\bsi{24pt}{30pt}}

\makeatletter
\renewcommand*\l@subsubsection{\@dottedtocline{3}{3em}{0em}}
\makeatother

\setlength\parindent{0pt}
\setlength\parskip{2ex plus 0.5ex}

\begin{document}
\chapter{Sopas}
\recipe{Crema de champiñones}
\ingred{2 cebollas cortadas finas; 1 coliflor pequeña cortada a trozos; 2 tazas de champiñones cortados finos; aceite; sal marina; laurel; 1 y 1/2 cucharada de miso blanco; 2 cucharadas de aceite de oliva}
\begin{enumerate}
\item Sofreír las cebollas con un poco de aceite, una pizca de sal marina y laurel durante unos 12 minutos
\item Añadir coliflor y dos tazas de agua. Tapar y cocer a fuego medio durante 20 minutos
\item Retirar el laurel y hacer un puré
\item Añadir los champiñones cortados finamente y cocer 5 minutos
\item Añadir más líquido, según la consistencia deseada y una cucharada de miso blanco. Servir la crema caliente o fría.
\end{enumerate}
\chapter{Legumbres}
\renewcommand{\inghead}{\textbf{Ingredientes (para 3 raciones)}:\ }
\recipe{Cocido de lentejas}
\ingred{1 tira de alga kombu; 1 taza de lentejas (remojadas de 2 a 3 horas; 1/4 de calabaza pequeña cortada a dados; 3 zanahorias cortadas a grandes trozos; 2 cebollas cortadas a dados; 1 diente de ajo picado; 1 cucharadita de mugi miso; aceite de oliva; tomillo; sal marina}

Saltear las cebollas y el ajo con un poco de aceite de oliva y una pizca de sal marina.

Añadir las lentejas (sin el agua del remojo), el alga kombu, la calabaza, el tomillo y agua que cubra el volumen de los ingredientes. Llevar a ebullición, tapar y cocer a fuego lento durante unos 45 minutos o hasta que las lentejas estés blandas.

Diluir el miso con un poco del jugo de cocción y añadirlo a las lentejas. Cocer durante dos minutos más y apagar el fuego. Servir con la guarnición
\paragraph*{Guarnición} Cebollinos cortados finos
\tableofcontents
\end{document}