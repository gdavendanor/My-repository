\documentclass[a4paper]{recipe}
\usepackage[T1]{fontenc}
\usepackage[spanis,es-noshorthands]{babel}
\usepackage{bookman}

\newcommand{\bsi}[2]{%
  \fontencoding{T1}\fontfamily{pbs}\fontseries{xl}\fontshape{n}%
  \fontsize{#1}{#2}\selectfont}

\renewcommand{\inghead}{\textbf{Ingredientes (para 4 personas)}:\ }
\renewcommand{\rechead}{\centering\bsi{24pt}{30pt}}

\makeatletter
\renewcommand*\l@subsubsection{\@dottedtocline{3}{3em}{0em}}
\makeatother

\setlength\parindent{0pt}
\setlength\parskip{2ex plus 0.5ex}

\begin{document}
\chapter{Primi piatti}
\recipe{Spaghetti alla carbonara}
\ingred{spaghetti, 400 gr.;
  pancetta coppata, 300 gr.;
  4 rossi d'uovo;
  panna fresca, 200 gr.;
  pepe, burro, parmigiano, pecorino.}

In una zuppiera mescolate bene i 4 rossi d'uovo alla
panna, al parmigiano ed al pecorino (grosso modo in
uguale quantit\`a, aggiustando le proporzioni a seconda
del gusto) e ad abbondante pepe.

Fate rosolare la pancetta coppata tagliata a pezzetti
in poco burro; e, quando si sar\`a freddata, unitela
al resto degli ingredienti.

Cuocete la pasta al dente, scolatela, mettetela nella
zuppiera e mescolate bene.

\recipe{Farfalle al gorgonzola}
\ingred{farfalle, 400 gr.;
  gorgonzola, 150 gr.;
  parmigiano grattugiato, 100 gr.;
  panna fresca, 200 gr.;
  pepe, burro, prezzemolo.}

In una padella fate sciogliere il burro ed unite la
panna; quando sar\`a calda aggiungete il gorgonzola
ed il parmigiano, che farete sciogliere a fuoco basso
mescolando continuamente.
Aggiungete infine il pepe ed il prezzemolo tritato;
se il sugo fosse troppo denso allungatelo con un poco
di latte.

Cuocete la pasta al dente, scolatela, mettetela nella
padella con il condimento e mescolate bene.
\tableofcontents
\end{document}