\documentclass[letterpaper,spanish]{letter}
\usepackage[T1]{fontenc}
\usepackage[utf8]{inputenc}
\usepackage[spanish]{babel}
%opening
\address{Germ\'{a}n Avenda\~{n}o Ram\'{i}rez\\ \Email ~iedabgerman@autistici.org}
\signature{Germ\'{a}n Avenda\~{n}o Ram\'{i}rez\\Lic. U.D. M.Sc. U.N.\\Representante docentes Consejo Directivo}
\date{9 de noviembre de 2018}

\begin{document}
	\begin{letter}{Consejo Directivo\\Colegio Arborizadora Baja I.E.D.}
		\opening{Cordial saludo}
Si bien es cierto que los proyectos que la SED pretende que adoptemos en modalidad "jornada extendida" fueron socializados por el rector en reunión del 28 de septiembre, es importante que tomemos una postura crítica frente a su adopción, teniendo en cuenta el poco tiempo de discusión sobre estos y también la peśima experiencia de otras instituciones escolares que han pasado por situaciones conflictivas, debido al desorden generado por la multitud de proyectos sin los espacios físicos y los recursos económicos necesarios para su desarrollo.

La estrategia de la SED para impulsar la jornada única ha sido el
engaño, usando un remedo de jornada única (que denomina extendida) sin cumplir con las 4 condiciones que se establecen en el decreto 2105:
\begin{quotation}
\textbf{Artículo 2.3.3.6.1.4. Condiciones para el reconocimiento de la jornada única}:\footnote{Decreto 2105 del 14 de diciembre de 2017} Para el reconocimiento de la implementación de la Jornada Única por parte de las entidades territoriales certificadas, de tal manera que la instauración paulatina del servicio educativo garantice que pueda ser prestado de manera continua, oportuna y adecuada, se deben cumplir las siguientes condiciones previas: 
\begin{enumerate}
\item Infraestructura disponible y en buen estado
\item  Un plan de alimentación escolar en modalidad "almuerzo" en el marco de la ejecución del PAE adoptado por las entidades territoriales certificadas, para los estudiantes que se encuentren desarrollando la "Jornada Única", a fin de disminuir el ausentismo y la deserción y fomentar estilos de vida saludables de niños, niñas, adolescentes y jóvenes.
\item El recurso humano necesario para la ampliación de la jornada escolar
\item  El funcionamiento regular y suficiente de los servicios públicos.
\end{enumerate}
\end{quotation}
Evidentemente en nuestro colegio no se satisfacen las condiciones arriba mencionadas para ampliar la jornada (como en el caso de la jornada extendida) que la SED incluye en el mismo documento (Manual de instrucciones) como si se tratara de la jornada única.

Por otro lado, el Consejo Directivo no puede modificar el PEI, ni el programa de estudios en un momento, en una sesión de dos o tres horas, sino que para eso se requiere de un procedimiento contemplado en el decreto 1860 de 1994, el cual en el capitulo III, artículo 15 dice\footnote{Decreto 1860 de 1994}:
\begin{quotation}
	\begin{enumerate}
	\item[3] Las modificaciones. Las modificaciones al proyecto educativo institucional podrán ser solicitadas al rector por 
cualquiera  de  los  estamentos  de  la  comunidad educativa.  Este  procederá  a  someterlas  a  discusión  de  los  
demás estamentos y concluida esta etapa, el Consejo Directivo procederá a decidir sobre las propuestas, previa 
consulta con el Consejo Académico.

Si  se  trata  de  materias  relacionadas  con  los  numerales  1,  3,  5,  7  y  8  del  artículo  14  del  presente  Decreto,  las propuestas de  modificación  que  no  hayan  sido  aceptadas  por  el  Consejo  Directivo,  deberán  ser  sometidas  a  
una  segunda  votación,  dentro  de  un  plazo  que  permita  la  consulta  a  los  estamentos  representados  en  el Consejo y, en caso de ser respaldadas por la mayoría que fije su reglamento, se procederá a adoptarlas.
\item[4] La  agenda  del  proceso.  El  Consejo  Directivo  al  convocar  a  la  comunidad  señalará  las  fechas  límites  para  
cada evento del proceso, dejando suficiente tiempo para la comunicación, la deliberación y la reflexión.
	\end{enumerate}
\end{quotation}

Por último aclaramos que en su mayoría, los/as maestros/as si estamos de acuerdo que se desarrollen proyectos extraescolares. Lo que no aceptamos es el chantaje a que nos quiere someter la Secretaría de Educación Distrital SED, condicionando el desarrollo de proyectos a la denominada "jornada extendida".

\closing{Atentamente,}
\encl{Acta de reunión de maestros/as del 31 de octubre del presente}
	\end{letter}
\end{document}
